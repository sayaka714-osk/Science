%!TEX TS-program = xelatex
%!TEX encoding = UTF-8

\documentclass[12pt]{book}


%%%%%%%  章节  %%%%%%

% Options: Sonny, Lenny, Glenn, Conny, Rejne, Bjarne, Bjornstrup
\usepackage[Lenny]{fncychap}
\ChTitleVar{\Large}

% 标题格式
\usepackage{titlesec}
\titleformat{\part}{\centering\Huge\bfseries}{第\Roman{part}部分}{1em}{}
\titleformat{\chapter}{\centering\huge\bfseries}{第\arabic{chapter}章}{1em}{}
\titleformat{\section}{\LARGE\bfseries}{第\arabic{section}节}{1em}{}
\titleformat{\subsection}{\Large\bfseries}{\arabic{subsection}.}{1em}{}
\titleformat{\subsubsection}{\large\bfseries}{\arabic{subsection}.\arabic{subsubsection}.}{1em}{}
\titleformat{\paragraph}{\normalsize\bfseries}{\arabic{subsection}.\arabic{subsubsection}.\arabic{paragraph}.}{1em}{}
\titleformat{\subparagraph}{\normalsize\bfseries}{\arabic{subsection}.\arabic{subsubsection}.\arabic{paragraph}.\arabic{subparagraph}.}{1em}{}

% 索引
\usepackage[xindy]{imakeidx}
\makeindex[columns=2, program=truexindy, intoc=true, options=-M texindy -I xelatex -C utf8, title={Index}]




%%%%%%%  页面设置  %%%%%%
\usepackage{geometry}   % 页面设置
\geometry{a4paper,left=2.5cm,right=2.5cm,top=2.5cm,bottom=2.5cm}
\usepackage{pdflscape}   % 页面横置
\usepackage{ragged2e}    % 两端对齐
\usepackage{indentfirst} % 首行缩进
%\usepackage{setspace}  % 间距
\setcounter{tocdepth}{7}  % 标题深度
\setcounter{secnumdepth}{7}
%\setlength{\baselineskip}{20pt} % 行距
\setlength{\parindent}{2em} % 首行缩进

% 页眉页脚
\usepackage{fancyhdr}
\pagestyle{fancy} % 设置页眉  
\lhead{}
\chead{}
\rhead{}
\cfoot{\thepage}
\rfoot{}
\lfoot{}
\renewcommand{\headrulewidth}{0pt}  %页眉线宽,设为0可以去页眉线


%%%%%%%  字体  %%%%%%
\usepackage{polyglossia} % 多语言

\usepackage{fontspec} % 字体
\usepackage{xeCJK}
\setmainfont{Times New Roman} 
\setCJKmainfont{SimSun} 

%%%%%%%%  表格  %%%%%%%%
\usepackage{supertabular}
\usepackage{tabularx}      % 表格自动换行
\usepackage{longtable}
\usepackage{tabu}
\usepackage{booktabs}      % 表格线条
\usepackage{makecell}
\usepackage{multirow}    % 单元格合并
\usepackage{caption}

% \columnwidth  当前分栏的宽度
% \linewidth    当前行的宽度
% \textwidth    整个页面版芯的宽度
% \paperwidth   整个页面纸张的宽度

\newcommand{\tablelist}[3]{
    \parbox{#2}{
        \begin{enumerate}[#1]
            #3
        \end{enumerate}
    }
}



%%%%%%%%%  编号  %%%%%%%%%
\usepackage{enumerate}   % 条目 
\usepackage{enumitem}
%\setitemsize[1]{itemsep=0pt,partopsep=0pt,parsep=\parskip,topsep=0pt}
\usepackage{amsmath}
\usepackage{paralist}
\let\itemize\compactitem
\let\enditemize\endcompactitem
\let\enumerate\compactenum
\let\endenumerate\endcompactenum
\let\description\compactdesc
\let\enddescription\endcompactdesc





%%%%%%%% 公式
\usepackage{bm} % 数学公式加粗
\usepackage{bbm}
\usepackage{amsfonts}
\usepackage{amssymb}
\usepackage{breqn}


%%%%%%%  绘图  %%%%%%
\usepackage{graphicx}    % 图

\usepackage{tikz}
\usetikzlibrary{trees,calc,quotes}
\usetikzlibrary{angles,patterns,datavisualization}
\usetikzlibrary{arrows,intersections}
\usetikzlibrary{graphs}
\newcommand{\treegraph}[1]{
\usetikzlibrary{trees}
\tikzstyle{every node}=[draw=black,thick,anchor=west]
\begin{tikzpicture}[
        grow via three points={one child at (0.5,-0.7) and
                two children at (0.5,-0.7) and (0.5,-1.4)},
        edge from parent path={(\tikzparentnode.south) |- (\tikzchildnode.west)}]
    #1
    ;
\end{tikzpicture}
}

\tikzset{eaxis/.style={->,>=stealth}}
\tikzset{elegant/.style={smooth,thick,samples=50,cyan}}

\usepackage{tikz-cd}
\usetikzlibrary{matrix,arrows,decorations.pathmorphing}

\usepackage{pgfplots}
\usepgfplotslibrary{groupplots}

%%%%%%%%  标记  %%%%%%%%%
\usepackage[breaklinks,colorlinks,linkcolor=black,citecolor=black,urlcolor=black]{hyperref}




%%%%%%%  颜色  %%%%%%

\usepackage{color}
\definecolor{codeKeyword}{RGB}{200,0,100}
\definecolor{codeString}{RGB}{200,100,40}
\definecolor{codeComment}{RGB}{0,100,0}
\definecolor{codeNumber}{RGB}{128,128,128}
\definecolor{codeBackground}{RGB}{242,242,242}

% Matlab highlight color settings
%\definecolor{mBasic}{RGB}{248,248,242}       % default
\definecolor{mKeyword}{RGB}{0,0,255}          % bule
\definecolor{mString}{RGB}{160,32,240}        % purple
\definecolor{mComment}{RGB}{34,139,34}        % green
\definecolor{mBackground}{RGB}{245,245,245}   % lightgrey
\definecolor{mNumber}{RGB}{128,128,128}       % gray

\definecolor{Numberbg}{RGB}{237,240,241}     % lightgrey

% Python highlight color settings
%\definecolor{pBasic}{RGB}{248, 248, 242}     % default
\definecolor{pKeyword}{RGB}{228,0,128}        % magenta
\definecolor{pString}{RGB}{148,0,209}         % purple
\definecolor{pComment}{RGB}{117,113,94}       % gray
\definecolor{pIdentifier}{RGB}{166, 226, 46}  %
\definecolor{pBackground}{RGB}{245,245,245}   % lightgrey
\definecolor{pNumber}{RGB}{128,128,128}       % gray


\usepackage{xcolor}

%%%%%%%%%%  模块  %%%%%%%%%%%%


\usepackage{tcolorbox}
\newcommand{\informationBox}[1]{
    \small
    \begin{tcolorbox}[colback=gray!10!white,colframe=gray!30!white]
        #1
    \end{tcolorbox}
}

% 思维导图
\newcommand{\mindmap}[1]{
    \begin{cases}
        #1
    \end{cases}\\
}


% 引用文本
\newcommand{\refdocument}[1]{
    {\kaishu 
    \begin{quotation}
        #1
    \end{quotation}
    }
}









%%%%%%%%%%%%%%%%%%%%%%%%%%%%%%%%%


\usepackage[UTF8]{ctex}
\usepackage{autobreak}
\usepackage[utf8]{inputenc} % Required for inputting international characters

\usepackage{adjustbox}   % 调整box大小

\usepackage{tipa}
\usepackage{CJKfntef}
\usepackage{pdfpages}

\usepackage{blindtext}
\usepackage{verbatim}
\usepackage{ascmac}
\usepackage{xpinyin} % 拼音









\begin{document}
\title{科学\\Science}  %%书名
\author{Maobin Xu} %%作者
%\date{} %%如果没有这句,会生成时间
\maketitle  %%生成书名


\tableofcontents  %%生成目录
\mainmatter %%表示文章的正文部分,在生成目录后将从第一页开始

\part{体系}

科学包括:数理与数据科学(Mathematics and Data Science)、自然与工程科学(Natural and Engineering Science)和人文与社会科学(Humanities and Social Science)。




{\tiny
\begin{equation*}
    \text{数理与数据科学}\mindmap{
        \text{数学理论}\mindmap{
            \text{确定性数学}\mindmap{
                \text{数理逻辑}\\
                \text{代数(数)}\mindmap{
                    \text{算数}\\
                    \text{数论}\\
                    \text{代数:线性代数}
                }
                \text{几何(形)}\mindmap{
                    \text{解析几何}\\
                    \text{拓扑学}\\
                    \text{非欧几何}
                }
                \text{分析(极限)}\mindmap{
                    \text{函数论}\\
                    \text{微积分}\\
                    \text{泛函分析}
                }
            }
            \text{不确定数学}\mindmap{
                \text{随机变量}\\
                \text{统计推断}\mindmap{
                    \text{参数估计}\\
                    \text{假设检验}\mindmap{
                        \text{参数检验}\\
                        \text{非参数检验}
                    }
                }
            }
        }
        \text{数据分析}\mindmap{
            \text{机器学习}\mindmap{
                \text{符号学习}\\
                \text{统计学习}\mindmap{
                    \text{有监督学习}\mindmap{
                        \text{回归分析}\\
                        \text{分类分析}
                    }
                    \text{无监督学习}\mindmap{
                        \text{聚类分析}\\
                        \text{降维分析}
                    }
                }
                \text{连接学习}\mindmap{
                    \text{深度学习/神经网络}
                }
            }
            \text{强化学习}
        }
    }
\end{equation*}



\begin{equation*}
    \text{自然与工程科学}\mindmap{
        \text{自然科学}\mindmap{
            \text{物理学}\\
            \text{化学}\\
            \text{生物学}
        }
        \text{工程科学}\mindmap{
            \text{计算机科学}\mindmap{
                \text{工具}\mindmap{
                    \text{LaTeX}\\
                    \text{Python}\\
                    \text{Stata}
                }
                \text{计算机视觉}\\
                \text{语音识别}\\
                \text{自然语言处理}\\
                \text{自动程序设计}
            }
        }
    }
\end{equation*}



}


\clearpage


人文与社会科学包括:经济学(Economics)、政治学(Politics)、文化学(Culture)和历史学(History)。

构成(是什么),理论(为什么),方法(怎么做)

{\tiny
\begin{equation*}
    \text{经济学}\mindmap{
        \text{构成}\mindmap{
            \text{实体经济}\mindmap{
                \text{商品市场}\mindmap{
                    \text{产品市场}\mindmap{
                        \text{商品}
                    }
                    \text{生产要素市场}\mindmap{
                        \text{劳动力}\\
                        \text{资本}
                    }
                }
            }
            \text{虚拟经济}\mindmap{
                \text{资金市场}\mindmap{
                    \text{基础要素}\mindmap{
                        \text{货币}\\
                        \text{信用}\\
                        \text{利率}\\
                        \text{汇率}\\
                        \text{金融工具}
                    }
                    \text{运作载体}\mindmap{
                        \text{金融市场}\mindmap{
                            \text{货币市场}\\
                            \text{资本市场}\\
                            \text{衍生品市场}
                        }
                        \text{金融机构}\mindmap{
                            \text{金融中介机构}\mindmap{
                                \text{存款类金融机构}\mindmap{
                                    \text{管理性}\mindmap{
                                        \text{中央银行}
                                    }
                                    \text{营业性}\mindmap{
                                        \text{商业性银行}\\
                                        \text{政策性银行}
                                    }
                                }
                                \text{非存款类金融机构}
                            }
                            \text{金融辅助机构}
                        }
                        \text{非金融机构:公司金融}\\
                        \text{家庭:家庭金融}
                    }
                }
                \text{房地产市场}
            }
        }
        \text{理论}\mindmap{
            \text{完全竞争市场}\mindmap{
                \text{静态分析}\mindmap{
                    \text{长期}\mindmap{
                        \text{古典AD-AS模型}\\
                        \text{剩余价值理论}
                    }
                    \text{极短期:凯恩斯AD-AS模型}\mindmap{
                        \text{IS-LM-BP模型}\mindmap{
                            \text{产品市场}\mindmap{
                                \text{收入-支出模型}
                            }
                            \text{货币市场}\\
                            \text{国际市场}
                        }
                    }
                    \text{短期:新凯恩斯AD-AS模型}\mindmap{
                        \text{粘性价格理论}\\
                        \text{粘性工资理论}\\
                        \text{不完全信息理论}\\
                        \text{菲利普斯曲线}
                    }
                }
                \text{比较静态分析}\\
                \text{动态分析}\mindmap{
                    \text{经济增长}\mindmap{
                        \text{外生增长}\mindmap{
                            \text{索罗模型}\\
                            \text{无限期和世代交叠模型}
                        }
                        \text{内生增长}\mindmap{
                            \text{内生增长模型}
                        }
                    }
                    \text{经济周期}\mindmap{
                        \text{名义冲击}\mindmap{
                            \text{货币经济周期理论}\mindmap{
                                \text{理性预期假说}\\
                                \text{持续市场出清假说}\\
                                \text{自然率假说}
                            }
                            \text{新凯恩斯主义DSGE模型}
                        }
                        \text{实际冲击}\mindmap{
                            \text{真实经济周期模型(RBC)}
                        }
                    }
                }
            }
            \text{不完全竞争市场}\mindmap{
                \text{缓冲经济系统的扭曲}\mindmap{
                    \text{不确定性}\\
                    \text{不完全信息}\\
                    \text{外部性}
                }
                \text{默认经济系统的扭曲}\mindmap{
                    \text{市场结构}\mindmap{
                        \text{垄断竞争市场}\\
                        \text{寡头垄断市场}\\
                        \text{完全垄断市场}
                    }
                }
                \text{吸收经济系统的扭曲}\mindmap{
                    \text{博弈论}
                }
            }
        }
        \text{方法}\mindmap{
            \text{目标}\mindmap{
                \text{内部均衡}\mindmap{
                    \text{经济增长-产品市场}\mindmap{
                        \text{核算}\mindmap{
                            \text{资产负债表}\\
                            \text{资金流量表}\\
                            \text{收入表}
                        }
                    }
                    \text{充分就业-劳动力市场}\mindmap{
                        \text{自然失业率模型}
                    }
                    \text{物价稳定-货币市场}
                }
                \text{外部均衡}\mindmap{
                    \text{国际收支平衡-国际市场}
                }
            }
            \text{政府}\mindmap{
                \text{财政政策}\\
                \text{货币政策}\\
                \text{贸易政策}\\
                \text{监管政策}\\
                \text{行政政策}
            }
        }
    }
\end{equation*}
}


政治经济学(Political Economics):广义上,研究一定社会生产、交换、分配和消费等经济活动中的经济关系和经济规律。狭义上,在中国特指马克思主义政治经济学,研究资本主义生产方式以及和它相适应的生产关系和交换关系(不再是自然经济),生产关系是马克思主义政治经济学研究的核心,关注价值。

西方经济学:包括微观经济学和宏观经济学,关注价格。


{\tiny
\begin{equation*}
    \text{政治学}\mindmap{
        \text{政治发展}\mindmap{
            \text{马克思主义中国化}
        }
    }
\end{equation*}

\begin{equation*}
    \text{文化学}\mindmap{
        \text{哲学}\mindmap{
            \text{哲学史}\\
            \text{辩证唯物主义}\mindmap{
                \text{世界观}\mindmap{
                    \text{辩证唯物论}
                }
                \text{方法论}\mindmap{
                    \text{唯物辩证法}
                }
                \text{历史唯物主义}\mindmap{
                    \text{社会观}\\
                    \text{价值观(个人)}
                }
            }
        }
        \text{伦理学}\\
        \text{语音文字学}\\
        \text{文学}\mindmap{
            \text{纯文学}\\
            \text{新闻学}
        }
        \text{艺术}\mindmap{
            \text{音乐}\mindmap{
                \text{器乐}\mindmap{
                    \text{弹拨乐器}\mindmap{
                        \text{琵琶}
                    }
                    \text{声乐}
                }
                \text{表演}
            }
        }
    }
\end{equation*}













\begin{equation*}
    \text{历史学}\mindmap{
        \text{史前时期}\\
        \text{古代史}\\
        \text{近代史}\mindmap{
            \text{中国近代史}}
        \text{现代史}\\
        \text{形势与政策}
    }
\end{equation*}
\clearpage


}






\part{数理与数据科学}


\chapter{函数论}




函数:
设$x$和$y$是两个变量(均在实数集$R$内取值),$D$是一个给定的非空数集,如果对于每个数$x\in D$,按照某个对应法则$f$,变量$y$都有唯一确定的数值和它对应,则称变量$y$是变量$x$的函数,记作$y= f(x)$。
$D$称为函数$y= f(x)$的定义域,$x$称为自变量,$y$称为因变量,函数值$ f(x)$的全体所构成的集合称为函数$f$的值域

1、反函数:
设$y= f(x)$的定义域为$D$,值域为$W$。若$\forall y \in W$,存在唯一确定的$x \in D$,满足$y= f(x)$,则得到$x$是$y$的函数,记为$ x =\varphi(y)$,称为$y= f(x)$的反函数,习惯上记为$ y= f^{-1}(x)$

2、复合函数:
设$ y = f(u)$,$u=\varphi(x)$,若$\varphi(x)$的值域与$f(u)$的定义域有非空交集,则由$ y = f(u)$及$u=\varphi(x)$可复合而成复合函数$ y = f[\varphi(x)]$,$u$称为中间变量

3、隐函数:
设有关系式$ F(x,y)=0$,若$\forall x\in D$,存在唯一确定的$y$满足$F(x,y)= 0$与$x$相对应,由此确定的$y$与$ x $的函数关系$y = y(x)$称为由方程$  F(x,y)= 0  $所确定的隐函数






\section{函数性质}


\subsection{有界性}

定义:设$y=f(x)$在区间$I$上有定义,如果存在正数$M$,对于任意$ x \in M$,恒有$|f(z)|\leq M$,则称$y=f(x)$在区间$I$上有界;否则称无界

如果存在实数$M_1$,对于任意$x\in I$,恒有$f(x)\leq M_1$,则称$y=f(x)$在区间$I$上有上界;如果存在实数$ M_2$,对于任意$x\in  I$,恒有$ f(x)\geq M_2$,则称$y=f(x)$在区间$I$上有下界

$y=f(x)$在$I$上有界$\Leftrightarrow$ 既有上界又有下界

\subsection{连续性}

定义:如果对于$\forall x_0\in (a,b)$,$f(x)$在 $x_0$连续,称$f(x)$在$(a,b)$内连续。如果$f(x)$在$(a,b)$内连续,且在点$x=a$右连续,在$x=b$左连续,则称$f(x)$在$[a,b]$上连续,又称一致连续

设函数$y=f(x)$在点$x_0$的某一邻域内有定义,如果$\lim_{x\rightarrow x_0}{f(x)}= f(x_0)$,那么称函数$f(x)$在点$x_0$连续。若$\lim_{x\rightarrow x_0}{f(x)}= f(x_0^+)$,则称$f(x)$在$x_0$右连续;若$\lim_{x\rightarrow x_0}{f(x)}= f(x_0^-)$,则称$f(x)$在$x_0$左连续

\subsection{单调性}

定义:$\forall x_1, x_2\in I$,若$x_1<x_2$时, $f(x_1)<f(x_2)$(或$f(x_1)>f(x_2)$),则称$f(x)$在 $I$内单调递增(单调递减);若$x_1\leq x_2$时,$f(x_1)\leq f(x_2)$(或$f(x_1)\geq f(x_2)$),则称 $f(x)$在$I$内广义单调递增(或广义单调递减)

若$\forall x\in[a,b]$,$f^\prime(x)>0$(或$f'(x)<0$),则$f(x)$在$[a,b]$内单调增(或单调减)

设$ f(x)$在$x_0$的邻域内有定义,那么对$x_0$某空心邻域内的任一$x$,若$f(x)< f(x_0)$,则称$x_0$是 $f(x)$的极大值点;若$f(x)> f(x_0)$,则称$x_0$是 $f(x)$的极小值点

若$f'(x_0)=0$,则$x_0$是$f(x)$的驻点

\subsection{凹凸性}

定义:对于可导函数$f(x)$的图形,若在区间$[a,b]$中,$f(x)$都位于它每一点切线的上侧(下侧),即$ f(x+\Delta x)>f(x)+f'(x)\cdot A x$($ f(x+\Delta x)<f(x)+f'(x)\cdot A x$),则称曲线$f(x)$在$[a,b]$中是向上凹的(向上凸的)


若在$[a,b]$中$ f''(x)> 0$,则曲线$f(x)$在$[a,b]$中是凹(下凸)的;若在$[a,b]$中$f''(x)<0$则曲线$f(x)$在$[a,b]$中凸(上凹)的

设曲线$y=f(x)$连续且处处有切线,则其凹与凸的分界点称为此曲线的拐点,拐点处$f''(x)=0$或$f''(x)$不存在

\subsection{周期性}

定义:设$f(x)$的定义域为$D$,如果存在一个不为零的常数$T$,使得对于任一$ x \in D$,有$x\pm T\in D$且 $f(x+T)= f(x)$,则$f(x)$称为周期函数,$T$称为$f(x)$的周期。通常把满足上式的最小正数$T$称为$f(x)$的周期


并非每个周期函数都有最小正周期(如,常数函数、狄利克雷函数)

\subsection{奇偶性}

定义:设$f(x)$的定义域$D$关于原点对称,如果对任一$x\in D$,恒有$f(-x)=f(x)$(或$ f(-x)=- f(x)$),则称$f(x)$为偶函数(或奇函数)


偶函数的图形关于$y$轴对称,奇函数的图形关于坐标原点对称

偶+偶=偶,奇+奇=奇,偶x偶=偶,奇x奇=偶,奇x偶=奇







\chapter{参数估计}

参数估计:利用样本信息对总体数字特征做出的估计(parameter estimation)
\\

\begin{enumerate}[1.]
    \item 原则
          \begin{enumerate}[(1)]
              \item 无偏性
                    \begin{gather*}
                        \lim_{n→\infty}{E(\hat{\theta})=\theta } \\
                        E\left(\overline{\bm{X}}\right)=\bm{\mu},\ E\left(\frac{1}{n-1}\bm{S}\right)=\bm{\Sigma}
                    \end{gather*}
              \item 有效性
                    \begin{gather*}
                        D(\hat{\theta}_1)<D(\hat{\theta}_2),\hat{\theta}_1比\hat{\theta}_2有效
                    \end{gather*}
              \item 一致性
                    \begin{gather*}
                        \lim_{n\rightarrow\infty}{P\left({\left|{\hat{\theta}}_n-\theta\right|<\varepsilon}\right)}=1
                    \end{gather*}
          \end{enumerate}
    \item 内容
          \begin{enumerate}[(1)]
              \item 点估计(point estimation)
              \item 区间估计(interval estimation):$z_{\frac{\alpha}{2}}\approx 1.96$
                    \begin{enumerate}[a.]
                        \item $1-\alpha$:置信度,置信区间以$1-\alpha$的概率覆盖总体未知参数。置信度越大,置信区间越宽
                        \item 样本均值:中心位置
                        \item 总体标准差:总体波动越小,置信区间越窄
                        \item 样本容量n:样本容量越大,置信区间越窄
                        \item 样本标准差s:样本标准差越大,置信区间越宽
                    \end{enumerate}
          \end{enumerate}
\end{enumerate}


\section{最小二乘估计}





普通最小二乘法(Ordinary Least Square,OLS)

\subsection{方法}

\subsubsection{均值估计}
总体均值估计$\approx$样本均值

\paragraph{一元}

\begin{gather*}
    \overline{X}=\frac{1}{n}\sum_{i=0}^{n}X_i
\end{gather*}



\paragraph{多元}

\begin{align*}
    \hat{\bm{\mu}}
    = & \overline{\mathbf{X}}
    =\frac{1}{n}\sum_{a=1}^{n}\mathbf{X}_{(a)}
    =\frac{1}{n}\left[\left[
            \begin{matrix}
                \begin{matrix}
                    X_{11} \\
                    X_{12} \\
                \end{matrix} \\
                \begin{matrix}
                    \vdots \\
                    X_{1p} \\
                \end{matrix} \\
            \end{matrix}\right]  +\left[
            \begin{matrix}
                \begin{matrix}
                    X_{21} \\
                    X_{22} \\
                \end{matrix} \\
                \begin{matrix}
                    \vdots \\
                    X_{2p} \\
                \end{matrix} \\
            \end{matrix}\right] +\cdots+ \left[
    \begin{matrix}
                \begin{matrix}
                    X_{n1} \\
                    X_{n2} \\
                \end{matrix} \\
                \begin{matrix}
                    \vdots \\
                    X_{np} \\
                \end{matrix} \\
            \end{matrix}\right]\right] \\
    = & \frac{1}{n} \left[
        \begin{matrix}
            \begin{matrix}
                X_{11}+X_{21}+\cdots+X_{n1} \\
                X_{12}+X_{22}+\cdots+X_{n2} \\
            \end{matrix} \\
            \begin{matrix}
                \vdots                      \\
                X_{1p}+X_{2p}+\cdots+X_{np} \\
            \end{matrix} \\\end{matrix}\right]
    =\left({\overline{X}}_1,{\overline{X}}_2,\cdots,{\overline{X}}_p\right)^\prime
\end{align*}


\subsubsection{方差估计}

总体方差(离差)估计$\approx$样本方差(离差)

\paragraph{一元(离差)}

\begin{gather*}
    S^2=\frac{1}{n}\sum_{i=1}^{n}{(X_i-\overline{X})^2}, S^2=\frac{1}{n-1}\sum_{i=1}^{n}{(X_i-X)^2}
\end{gather*}


\paragraph{多元(样本离差阵)}

\begin{align*}

    \mathbf{S}_{p\times p}
     & = \sum_{a=1}^{n}{\left(\mathbf{X}_{\left(a\right)}-\overline{\mathbf{X}}\right)\left(\mathbf{X}_{\left(a\right)}-\overline{\mathbf{X}}\right)^\prime}
    = \sum_{a=1}^{n}
    \begin{bmatrix}
        \begin{bmatrix}
            X_{a1}-{\overline{X}}_1 \\
            X_{a2}-{\overline{X}}_2 \\
            \vdots             \\
            X_{ap}-{\overline{X}}_p \\
        \end{bmatrix}
         & \left(X_{a1}-{\overline{X}}_1,X_{a2}-{\overline{X}}_2,\cdots,X_{ap}-{\overline{X}}_p\right) \\
    \end{bmatrix}                                                                                                                              \\
     & = \sum_{a=1}^{n}
    \begin{bmatrix}
        \left(X_{a1}-{\overline{X}}_1\right)^2
         & \left(X_{a1}-{\overline{X}}_1\right)\left(X_{a2}-{\overline{X}}_2\right)
         & \cdots
         & \left(X_{a1}-{\overline{X}}_1\right)\left(X_{ap}-{\overline{X}}_p\right) \\

        \left(X_{a2}-{\overline{X}}_2\right)\left(X_{a1}-{\overline{X}}_1\right)
         & \left(X_{a2}-{\overline{X}}_2\right)^2
         & \cdots
         & \left(X_{a2}-{\overline{X}}_2\right)\left(X_{ap}-{\overline{X}}_p\right) \\
        \vdots
         & \vdots                                                         \\
         & \                                                              \\
         & \vdots                                                         \\
        \left(X_{ap}-{\overline{X}}_p\right)\left(X_{a1}-{\overline{X}}_1\right)
         & \left(X_{ap}-{\overline{X}}_p\right)\left(X_{a2}-{\overline{X}}_2\right)
         & \cdots
         & \left(X_{ap}-{\overline{X}}_p\right)^2                              \\
    \end{bmatrix}                                                                                                                               \\
     & = \begin{bmatrix}
        s_{11} & s_{12} & \cdots & s_{1p} \\
        s_{21} & s_{22} & \cdots & s_{2p} \\
        \vdots & \vdots & \      & \vdots \\
        s_{p1} & s_{p2} & \cdots & s_{pp} \\
    \end{bmatrix}
    = \left(s_{ij}\right)_{p\times p}
\end{align*}



\subsubsection{系数估计}

\paragraph{一元}


\begin{gather*}
    Y_i = \beta_0+\beta_{1}X_{i}+{\hat{\varepsilon}}_i\\
    \min_{{\hat{\beta}}_1}{\sum_{i=1}^{n}{\hat{\varepsilon}}_i^2} =\sum_{i=1}^{n}{\left(Y_i-{\hat{\beta}}_0-{\hat{\beta}}_1 X_i\right)^2}\\
    \frac{\partial}{\partial \beta_{1}}  \sum_{i=1}^{n}{\hat{ε}_i^2} =-2\sum_{i=1}^{n}{(Y_i-\hat{\beta}_{0}-\hat{\beta}_{1} X_i) X_{i}}=0\\
    \hat{\beta}_{1} = \frac{\sum_{i=1}^{n}\left(X_i-\overline{X}\right)\left(Y_i-\overline{Y}\right)}{\sum_{i=1}^{n}\left(X_i-\overline{X}\right)^2}=\frac{\sum_{i=1}^{n}{x_i y_i}}{\sum_{i=1}^{n}x_i^2},\hat{\beta}_{0}=\overline{Y}-\hat{\beta}_{1}\overline{X}\\
    {\hat{\beta}}_1 \sim N\left(\beta_1,\frac{\sigma_2}{\sum_{i=1}^{n}{x_i^2}}\right),\hat{\beta}_{0} \sim N\left(\beta_0, \frac{\sum_{i=1}^{n}{X_i^2}}{ n \sum_{i=1}^{n}{x_i^2}} \sigma^2\right)\\
\end{gather*}


随机干扰项$\mu_{i}$的$\hat{\sigma}^2 =\frac{\sum_{i=1}^{n}{\varepsilon_i^2}}{n-2}$

\paragraph{多元}

\begin{gather*}
    \hat{\bm{\beta}} \equiv \left(\bm{X}^\prime\bm{X}\right)^{-1}\bm{X}^\prime\bm{Y}
\end{gather*}

随机干扰项$\varepsilon_i$的方差
$\hat{\sigma}^2  = \sum_{i=1}^{n}{ 	\frac{ \hat{\varepsilon}_i^2 }{n-k-1} 	} = \frac{RSS}{n-k-1} = \frac{\bm{\hat{\varepsilon}'\hat{\varepsilon}}}{n-k-1}$

$k$:变量个数

置信区间:
\begin{gather*}
    \left( \hat{\beta}_j-S_{\hat{\beta}_j} t_{\frac{\alpha}{2}}(n-k-1), \hat{\beta}_j+S_{\hat{\beta}_j} t_{\frac{\alpha}{2}}(n-k-1) \right)
\end{gather*}

缩小置信区间方法:增大样本容量n;提高模型拟合优度(减小残差平方和)

\subsection{性质}

高斯-马尔可夫定理(Gauss-Markov Theorem):(无需随机扰动项正态分布)最小二乘法是最佳线性无偏估计(Best Linear Unbiased Estimator,BLUE)

\subsubsection{小样本性质}

(small-sample properties)


\paragraph{线性性}

一元:
\begin{gather*}
    {\hat{\beta}}_1=\frac{\sum_{i=1}^{n}{x_iy_i}}{\sum_{i=1}^{n}x_i^2}=\frac{\sum_{i=1}^{n}{x_i\left(Y_i-\overline{Y}\right)}}{\sum_{i=1}^{n}x_i^2}=\frac{\sum_{i=1}^{n}{x_iY_i}}{\sum_{i=1}^{n}x_i^2}-\frac{\overline{Y}\sum_{i=1}^{n}x_i}{\sum_{i=1}^{n}x_i^2}=\sum_{i=1}^{n}{k_iY_i}, k_i=\frac{x_i}{\sum_{i=1}^{n}{x_i^2}}
\end{gather*}

多元:
\begin{gather*}
    \hat{\bm{\beta}}\equiv\left(\mathbf{X}^\prime\mathbf{X}\right)^{-1}\mathbf{X}^\prime\mathbf{Y}=\mathbf{CY}\text{(关于Y线性)}
\end{gather*}


\paragraph{无偏性}

一元:
\begin{gather*}
    E\left({\hat{\beta}}_1\middle| X\right)=E\left[\left(\beta_1+\sum_{i=1}^{n}{k_i\varepsilon_i}\right)\middle| X\right]=\beta_1+\sum_{i=1}^{n}{k_iE(\varepsilon_i|X)}=\beta_1
\end{gather*}


多元:
\begin{gather*}
    E(\hat{\bm{\beta}}|\bm{X}) =\bm{\beta},E(\bm{s}^2|\bm{X})=\sigma^2\\
    E\left(\hat{\bm{\beta}}-\bm{\beta}\middle| X\right) =E\left[\left(\bm{X}^\prime\bm{X}\right)^{-1}\bm{X}^\prime\bm{\varepsilon}\middle|\bm{X}\right] =\left[\left(\bm{X}^\prime\bm{X}\right)^{-1}\bm{X}^\prime\right]E\left(\bm{\varepsilon}\middle|\bm{X}\right)=\bm{0}
\end{gather*}


\paragraph{有效性/最小方差性}

一元:
\begin{gather*}
    Var(\hat{\beta}_1|X)=\frac{\sigma^2}{\sum_{i=1}^{n}x_i^2} \\
    {\hat{\beta}}_1^\ast=\sum_{i=1}^{n}{c_iY_i}=\sum_{i=1}^{n}{\left(k_i+d_i\right)Y_i},d_i\text{为不全为零的常数}\\
    Var(\hat{\beta}_1^\ast)\geqslant Var(\hat{\beta}_1)
\end{gather*}


多元:
\begin{gather*}
    Var\left(\hat{\bm{\beta}}\middle|\bm{X}\right)=\sigma^2\left(\bm{X}^\prime\bm{X}\right)^{-\bm{1}}
\end{gather*}


\subsubsection{大样本渐进性质}



(large-sample asymptotic properties)
\paragraph{一致性}

\begin{gather*}
    \lim_{n\rightarrow\infty}{\hat{\bm{\beta}}} = \bm{\beta}
\end{gather*}


\paragraph{渐近无偏性}

\paragraph{渐近有效性}


















\section{矩估计法}





矩估计法(Method of Moment,MM)

\subsection{方法}

\subsubsection{均值估计}

\begin{gather*}
    \begin{cases}
        E(X) = \frac{1}{n}\sum_{i=1}^{n}{X_i}     \\
        E(X^2) = \frac{1}{n}\sum_{i=1}^{n}{X_i^2} \\
        \vdots                                    \\
        E(X^k) = \frac{1}{n}\sum_{i=1}^{n}{X_i^k}
    \end{cases}
\end{gather*}

\subsubsection{系数估计}

\begin{gather*}
    E( \mathbf{X_i} \varepsilon_{i} ) =\mathbf{0} \\
    \hat{\bm{\beta}} \equiv  (\mathbf{X}^{\prime}\mathbf{X})^{-1} \mathbf{X}^{\prime} \mathbf{Y}
\end{gather*}










\section{极大似然估计}







极大似然估计:又称最大似然法(Maximum Likelihood,ML)

\subsection{方法}
\subsubsection{均值估计}

\begin{gather*}
    L\left(x_1,x_2,\cdots,x_n;\theta_1,\theta_2,\cdots,\theta_n\right) =\prod_{i=1}^{n}f\left(x_i;\theta_1,\theta_2,\cdots,\theta_k\right)\\
    lnL=\sum_{i=1}^{n}lnf\left(x_i;\theta_1,\theta_2,\cdots,\theta_k\right)\\
    \frac{\partial lnL}{\partial\theta_i}=0,\ i=1,2,\cdots,k\\
    \begin{cases}
        \hat{\theta}_1 = \hat{\theta}_1(x_1,x_2,\cdots,x_n) \\
        \hat{\theta}_2 = \hat{\theta}_2(x_1,x_2,\cdots,x_n  \\
        \vdots                                              \\
        \hat{\theta}_k = \hat{\theta}_k(x_1,x_2,\cdots,x_n) \\
    \end{cases}	\\
    L\left(\bm{\mu},\bm{\Sigma}\right) =\prod_{i=1}^{n}{f(\bm{X}_i,\bm{\mu},\bm{\Sigma})}
\end{gather*}


\subsubsection{系数估计}

\begin{gather*}
    L\left(\bm{\beta},\sigma^2\right) 	=P\left(Y_1,Y_2,\cdots,Y_n\right) \\
    \hat{\bm{\beta}} 	\equiv \left(\bm{X}^\prime\bm{X}\right)^{-1}\bm{X}^\prime\bm{Y}
\end{gather*}


$Y$服从正态分布

随机干扰项$\sigma_i$的方差
\begin{gather*}
    {\hat{\sigma}}^2
    =\frac{
    \sum_{i=1}^{n}{
    {\hat{\bm{\varepsilon}}}_i^2}
    }{n}
    =\frac{
        {\hat{\bm{\varepsilon}}}^\prime \hat{\bm{\varepsilon}}
    }{n}
\end{gather*}



















\section{贝叶斯估计}


\chapter{假设检验}





\section{参数检验}



假设是针对总体分布中未知参数提出的,适用于定量、连续变量。

原理:小概率事件不太可能在一次事件中发生,若发生则有理由拒绝原假设
\begin{enumerate}[(1)]
    \item 第一类错误:拒真,$P\{拒绝H_0|H_0为真\}$,能够计算
    \item 第二类错误:取伪,$P\{接受H_0|H_0不真\}$
    \item 一般将样本支持的放在备择假设,样本背离的放在原假设;药物治愈、培训成果等将无效放在原假设;等号放在原假设
\end{enumerate}






\subsection{单个正态总体}





单个正态总体的参数检验。

\informationBox{
    对于匹配样本(paired-sample),样本量必须一样。通过差分变为单一样本,比较前后差异
}

\subsubsection{均值$\mu$的检验}

\paragraph{一元}

\begin{align*}
    H_0: \mu = \mu_0    & \leftrightarrow  	H_1: \mu \neq  \mu_0 \text{(双边假设检验)} \\
    H_0: \mu \leq \mu_0 & \leftrightarrow  	H_1: \mu > \mu_0 \text{(单边假设检验)}     \\
    H_0: \mu \geq \mu_0 & \leftrightarrow  	H_1: \mu < \mu_0 \text{(单边假设检验)}
\end{align*}


1、方差$\sigma^2 = \sigma^2_0$已知:$Z$检验

在$\mu = \mu_0$时,检验统计量及其分布为:
\begin{gather*}
    Z=\frac{\overline{X}-\mu_0}{\frac{\sigma}{\sqrt{n}}}\sim N(0,1)
\end{gather*}


置信区间为:
\begin{gather*}
    \left(\frac{\overline{X}-\sigma_0}{\sqrt{n}}Z_{\frac{\alpha}{2}}, \frac{\overline{X}+\sigma_0}{\sqrt{n}}Z_{\frac{\alpha}{2}} \right)
\end{gather*}


拒绝域为:
\begin{gather*}
    \left|Z\right|>z_{\frac{\alpha}{2}} \\
    Z >  z_\alpha\\
    Z < -z_\alpha
\end{gather*}


2、方差$\sigma^2$未知:$t$检验

在$\mu = \mu_0$时,检验统计量及其分布为:
\begin{gather*}
    T=\frac{\overline{X}-\mu_0}{\frac{S}{\sqrt{n}}}\sim t(n-1)
\end{gather*}


置信区间为:
\begin{gather*}
    \left( \overline{X}-\frac{S}{\sqrt{n}}t_{\frac{\alpha}{2}}(n-1), \overline{X}+\frac{S}{\sqrt{n}}t_{\frac{\alpha}{2}}(n-1) \right)
\end{gather*}


拒绝域为:
\begin{gather*}
    \left|T\right|>t_{\frac{\alpha}{2}}\left(n-1\right) \\
    T>t_\alpha\left(n-1\right) \\
    T<-t_\alpha\left(n-1\right)
\end{gather*}


\subsubsection{方差$\sigma^2$的检验}

\paragraph{一元}


\begin{gather*}
    H_0: \sigma^2=\sigma_0^2 \leftrightarrow H_1: \sigma^2\neq\sigma_0^2 \\
\end{gather*}


1、均值$\mu$已知

检验统计量及其分布为:
\begin{gather*}
    \chi^2=\frac{1}{\sigma_0^2}\sum_{i=1}^{n}{(X_i-\mu)^2}\sim \chi^2(n)
\end{gather*}


拒绝域为:
\begin{gather*}
    \chi^2>\chi_{\frac{\alpha}{2}}^{2}(n)\\
    \chi^2<\chi_{1-\frac{\alpha}{2}}^{2}(n)
\end{gather*}


2、均值$\mu$未知

检验统计量及其分布为:
\begin{gather*}
    \chi^2=\frac{1}{\sigma^2}\sum_{i=1}^{n}{X_i- \overline{X}} =\frac{(n-1)S^2}{\sigma^2} \sim \chi^2(n-1)
\end{gather*}


置信区间为:
\begin{gather*}
    \left( \frac{(n-1)S^2}{\chi_{\frac{\alpha}{2}}^{2} (n-1)}, \frac{(n-1)S^2}{\chi_{1-\frac{\alpha}{2}}^{2} (n-1)} \right)
\end{gather*}


拒绝域为:
\begin{gather*}
    \chi^2>\chi_{\frac{\alpha}{2}}^2(n-1) 或 χ^2<χ_{1-\frac{\alpha}{2}}^2(n-1)
\end{gather*}













\subsection{两个正态总体}





\subsubsection{均值$\mu$的检验}

\paragraph{一元}
$X\sim N\left(\mu_1,\sigma_1^2\right),\ Y\sim N\left(\mu_2,\sigma_2^2\right)$,两样本相互独立

\begin{align*}
    H_0:\mu_1=\mu_2    & \Leftrightarrow H_1:\mu_1\neq\mu_2 \\
    H_0:\mu_1\le\mu_2  & \Leftrightarrow H_1:\mu_1>\mu_2    \\
    H_0:\mu_1\geq\mu_2 & \Leftrightarrow H_1:\mu_1<\mu_2
\end{align*}


1、$\sigma_1^2,\sigma_2^2$已知,$\sigma_1^2=\sigma_2^2$

检验统计量及其分布:
\begin{gather*}
    U=\frac{\overline{X}-\overline{Y}}{\sqrt{\frac{\sigma_1^2}{n}+\frac{\sigma_2^2}{m}}}\sim N(0,1)
\end{gather*}


拒绝域为:
\begin{gather*}
    \left|U\right|>u_{\frac{\alpha}{2}}
\end{gather*}


2、$\sigma_1^2,\sigma_2^2$未知,$\sigma_1^2=\sigma_2^2$

检验统计量及其分布:
\begin{gather*}
    T=\frac{\overline{X}-\overline{Y}-(\mu_1-\mu_2)}{\sqrt{\frac{\left(n_1-1\right)S_1^2+\left(n_2-1\right)S_2^2}{n_1+n_2-2}}\sqrt{\frac{1}{n_1}+\frac{1}{n_2}}}\sim t(n_1+n_2-2)
\end{gather*}


拒绝域为:
\begin{gather*}
    \left|T\right|>t_{\frac{\alpha}{2}}(n_1+n_2-2)
\end{gather*}


3、$\sigma_1^2\neq\sigma_2^2$

\paragraph{多元}


\begin{gather*}
    \mathbf{X}_{a}=(X_{a1},X_{a2},\cdots,X_{ap})^\prime \sim N_p(\bm{\mu}_1,\bm{\Sigma}_1), a=1,2,\cdots,n_1\\
    \bm{Y}_a=\left(Y_{a1},Y_{a2},\cdots,Y_{ap}\right)' \sim N_p(\bm{\mu}_2,\mathbf{\Sigma}_2), a=1,2,\cdots,n_2 \\
    \mathbf{X}_{a}与\mathbf{Y}_a\text{相互独立}, n_1>p, n_2>p\\
    \overline{\bm{X}}=\frac{1}{n_1}\sum_{a=1}^{n_1}{X_a}, \overline{\bm{Y}}=\frac{1}{n_2}\sum_{a=1}^{n_2}{Y_a}
\end{gather*}



\begin{align*}
    H_0:\bm{\mu}_1=\bm{\mu}_2    & \Leftrightarrow H_1:\bm{\mu}_1\neq\bm{\mu}_2 \\
    H_0:\bm{\mu}_1\le\bm{\mu}_2  & \Leftrightarrow  H_1:\bm{\mu}_1>\bm{\mu}_2   \\
    H_0:\bm{\mu}_1\geq\bm{\mu}_2 & \Leftrightarrow H_1:\bm{\mu}_1<\bm{\mu}_2
\end{align*}


1、$\bm{\Sigma}_1,\bm{\Sigma}_2$已知,$\bm{\Sigma}_1=\bm{\Sigma}_2$

检验统计量及其分布:
\begin{gather*}
    T_0^2 =\frac{n_1\cdot n_2}{n_1+n_2}\left(\overline{\bm{X}}-\overline{\bm{Y}}\right)^\prime\bm{\Sigma}^{-1}\left(\overline{\bm{X}}-\overline{\bm{Y}}\right)=\chi^2\left(p\right)
\end{gather*}

2、$\bm{\Sigma}_1,\bm{\Sigma}_2$未知,$\bm{\Sigma}_1=\bm{\Sigma}_2$

3、$\bm{\Sigma}_1\neq\bm{\Sigma}_2$

\subsubsection{方差$\sigma^2$的检验}

\paragraph{一元}

\begin{gather*}
    H_0:\sigma^2=\sigma_0^2\Leftrightarrow H_1:\sigma^2\neq\sigma_0^2
\end{gather*}


1、均值$\mu$未知

检验统计量及其分布:
\begin{gather*}
    F=\frac{S_1^2}{S_2^2}\sim F(n_1-1,n_2-2)
\end{gather*}


拒绝域为:
\begin{gather*}
    F>F_{\frac{\alpha}{2}}(n_1-1,n_2-1)
    \text{或}
    F<F_{\frac{\alpha}{2}}(n_1-1,n_2-1)
\end{gather*}


\paragraph{多元}

\begin{gather*}
    H_0:\bm{\Sigma}_1^2=\bm{\Sigma}_2^2	\Leftrightarrow H_1:\bm{\Sigma}_1^2\neq\bm{\Sigma}_2^2
\end{gather*}



















\subsection{多个正态总体}






多个正态总体:方差分析(analysis of variance,ANOVA)

并不是两两比较均值,而是通过方差衡量均值

\subsubsection{均值$\mu$的检验}

\paragraph{一元}
$k$个正态总体$N\left(\mu_1,\sigma_1^2\right),\ \cdots,N\left(\mu_k,\sigma_k^2\right)$,$k$个样本相互独立,第$i$个样本有$n_i$个个体

${\overline{X}}_i=\frac{1}{n_i}\sum_{j=1}^{n_i}X_i^{(j)}$,
$\overline{\overline{X}}=\frac{1}{n}\sum_{i=1}^{k}\sum_{j=1}^{n_i}{X_i^j},n=n_1+n_2+\cdots+n_k$

$
    SST=\sum_{i=1}^{k}\sum_{j=1}^{n_i}{\left(X_i^j-\overline{\overline{X}}\right)^2}, SSA=\sum_{i=1}^{k}n_i\left(X_i-\overline{\overline{X}}\right) , SSE=\sum_{i=1}^{k}\sum_{j=1}^{n_i}{\left(X_i^j-\overline{\overline{X}}\right)^2}
$

\begin{gather*}
    H_0:\mu_1=\mu_2=\cdots=\mu_k\Leftrightarrow H_1:\mu_1,\mu_2,\cdots,\mu_k\text{不全相等} \\
\end{gather*}


1、$\sigma_1^2=\sigma_2^2=\cdots=\sigma_k^2$

检验统计量及其分布:
\begin{gather*}
    F =\frac{MSA}{MSE} =\frac{SSA/(k-1)}{SSE/(n-k)} =\frac{\sum_{i=1}^{k}{m(\overline{X}_i-\overline{\overline{X}})/(k-1)}}{\sum_{i=1}^{k}\sum_{j=1}^{m}{(X_{ij}-\overline{X}_i)^2/(n-k)}} \sim F(k-1,n-k)
\end{gather*}

\par F值越大越背离原假设
\par MSA:组内均方(mean square)
\par MSE:组间均方

\paragraph{多元}
各总体样品的均值向量

$\mathbf{T}=\mathbf{A}+\mathbf{E}$,$\mathbf{A}$为组间离差阵,$\bm{E}$为组内离差阵,$\bm{T}$为总离差阵

\begin{gather*}
    H_0:\bm{\mu}_1=\bm{\mu}_2=\cdots=\bm{\mu}_k	\Leftrightarrow H_1:\bm{\mu}_1,\bm{\mu}_2,\cdots,\bm{\mu}_k不全相等 \\
\end{gather*}


1、$\bm{\sigma}_1^2=\bm{\sigma}_2^2=\cdots=\bm{\sigma}_k^2$

检验统计量及其分布:
\begin{gather*}
    \bm{\Lambda}=\frac{\left|\bm{E}\right|}{\left|\bm{T}\right|}=\frac{\left|\bm{E}\right|}{\left|\bm{A}+\bm{E}\right|} \sim\bm{\Lambda}(p,n-k,k-1)
\end{gather*}


\subsubsection{方差$\sigma^2$的检验}

\paragraph{一元}

\begin{gather*}
    H_0:\sigma^2=\sigma_0^2 \Leftrightarrow	H_1:\sigma^2\neq\sigma_0^2
\end{gather*}


\paragraph{多元}

\begin{gather*}
    H_0:\bm{\Sigma}_1=\bm{\Sigma}_2=\cdots=\bm{\Sigma}_k\Leftrightarrow H_1:\left\{\bm{\Sigma}_i\right\}\text{不全相等}
\end{gather*}















\section{非参数检验}



假设是关于总体其他的统计特征(如总体分布、独立性等),适用于定类、定序数据、离散变量。






\section{回归分析}


\subsection{相关分析}



相关分析(Correlation Analysis)

\informationBox{
    \par 适用所有统计关系。因为不区分解释变量和被解释变量,因此具有对称性。
    \par 不存在线性相关并不意味着不相关,存在相关关系并不一定存在因果关系
}

\subsubsection{变量关系}

\begin{enumerate}[1.]
    \item 函数关系:规律+确定性(无扰动项)  
    \item 相关关系:规律+不确定性(有扰动项)  
    \begin{enumerate}[(1)]
        \item 原因  
        \begin{enumerate}[a.]
            \item 直接关系:直接的因果关系,X导致Y或Y导致X  
            \item 共同反应:X和Y共同为Z导致  
            \item 交叉关系: X和潜在变量Z共同导致Y  
        \end{enumerate}
        \item 表现  
        \begin{enumerate}[a.]
            \item 相关方向:正相关,负相关   
            \item 相关形式:线性相关,非线性相关  
            \item 变量数目:单相关(两个变量的相关),复相关  
        \end{enumerate}
    \end{enumerate}
\end{enumerate}




\subsubsection{衡量}


\begin{enumerate}[1.]
    \item 双变量:皮尔森单相关系数
          \begin{enumerate}[(1)]
              \item 总体相关系数:
                    \begin{gather*}
                        \rho_{XY}\equiv Corr\left(X,Y\right)\equiv\frac{Cov\left(X,Y\right)}{\sqrt{Var\left(X\right)Var\left(Y\right)}}
                        =\frac{\sigma_{XY}}{\sigma_X\sigma_Y}
                    \end{gather*}
              \item 样本相关系数:
                    \begin{gather*}
                        r_{XY} = \frac{\sum_{i=1}^{n}{ (X_i-\bar{X})(Y_i-Y)}}{\sqrt{\sum_{i=1}^{n}{(X_i-\bar{X})^2}}\cdot \sqrt{\sum_{i=1}^{n}{(Y_i-\bar{Y})^2}}  }, -1\leq r \leq 1
                    \end{gather*}
              \item 强弱:
                    \begin{align*}
                        0\le\left|r\right|<0.3:\text{无线性相关} \\
                        0.3\le\left|r\right|<0.5:\text{弱相关}   \\
                        0.5\le\left|r\right|<0.8:\text{中度相关} \\
                        0.8\le\left|r\right|\le1:\text{强相关}
                    \end{align*}
              \item 检验:
                    \begin{gather*}
                        H_0: \rho =0
                    \end{gather*}
                    \par 检验统计量:
                    \begin{gather*}
                        t = \frac{r\sqrt{n-2}}{\sqrt{(1-r)^2}} \sim t(n-2)
                    \end{gather*}
          \end{enumerate}
    \item 多变量:复相关系数,偏相关系数
\end{enumerate}


\subsection{横截面模型}







古典线性回归模型(Classical Linear Regression Model,CLRM)。 同时满足正态性假设的线性回归模型,称为经典正态线性回归模型(Classical Normal Linear Regression Model, CNLRM)。

\subsubsection{假设}

\paragraph{模型设定}


\begin{enumerate}[1.]
    \item 线性假定(linearity):总体模型为:
          \begin{gather*}
              Y_i=\beta_0+\beta_1X_{i1}+\beta_2X_{i2}+\cdots+\beta_kX_{ik}+\mu_i\ \left(i=1,\cdots,n\right)
          \end{gather*}
          若边际效应可变,可引入平方项($X_{ik}^2$)、三次方项($X_{ik}^3$)或交叉项($X_{ik}X_{im}$)
    \item 选择了正确的变量
\end{enumerate}




\paragraph{解释变量}


\begin{enumerate}[1.]
    \item 样本方差趋于非零有限常数
    \begin{gather*}
        \sum_{i=1}^{n}\frac{\left(X_i-\bar{X}\right)^2}{n}\rightarrow Q (n\rightarrow \infty )
    \end{gather*}
    \item $R\left(\bm{X}_{\bm{n}\times\left(\bm{k}+\bm{1}\right)}\right)=k+1$
    \begin{enumerate}[(1)]
        \item 无完全多重共线性:矩阵$\bm{X}$列满秩(多元特有)
        \item 样本观测值变异性:样本中的观测值不能完全相同,无自相关
    \end{enumerate}
\end{enumerate}




\paragraph{随机干扰项}

\begin{enumerate}[1.]
    \item $E\left(\bm{\varepsilon}\middle|\bm{X}\right)=\bm{0}$
          \begin{enumerate}[(1)]
              \item 条件0均值性:
                    \begin{gather*}
                        E\left(\varepsilon_i|X_1,X_2,\cdots,X_k\right)=0, i=1,2,\cdots,n
                    \end{gather*}
              \item 严格外生性(strict exogeneity) :
                    \begin{gather*}
                        Cov\left(\mathbf{X_i},\varepsilon_i\right)=E(\mathbf{X_i}\varepsilon_{i})=0, i=1,2,\cdots,n
                    \end{gather*}
                    \par 解释变量与随机项不相关,$\bm{X}$同期外生(contemporaneously exogenous),$\bm{X}$与$\varepsilon$同期不相关(contemporaneously uncorrelated)。
          \end{enumerate}
    \item 球型扰动项(spherical disturbance)
          \begin{gather*}
              Var\left(\bm{\varepsilon}\middle|\bm{X} \right)
              =E\left( \bm{\varepsilon}\bm{\varepsilon}^\prime\middle|\bm{X} 	\right)
              =E\left( \begin{matrix}
                  \varepsilon_1^2            & \cdots & \varepsilon_1\varepsilon_n \\\vdots&\ddots&\vdots\\
                  \varepsilon_n\varepsilon_1 & \cdots & \varepsilon_n\varepsilon_n \\
              \end{matrix} \middle|\bm{X}\right)
              =\left(\begin{matrix}
                      \sigma^2 & \      & 0        \\
                      \        & \ddots & \        \\
                      0        & \      & \sigma^2 \\
                  \end{matrix}\right)
              =\sigma^2\mathbf{I_n}
          \end{gather*}
          \begin{enumerate}[(1)]
              \item 条件同方差(conditional homoskedasticity):
                    \begin{gather*}
                        Var\left(\varepsilon_i\middle| X_1,X_2,\cdots,X_k\right)=\sigma^2,i=1,2,\cdots,n
                    \end{gather*}
              \item 序列不相关:
                    \begin{gather*}
                        Cov\left (\varepsilon_i,\varepsilon_j\middle| X_1,X_2,\cdots,X_k\right)=0,i\neq j  i,j=1,2,\cdots,n
                    \end{gather*}
              \item 正态性:在采用OLS进行参数估计时,不需要正态性假设。在利用参数估计量进行统计推断时,需要假设随机项的概率分布。
                    \begin{gather*}
                        \bm{\varepsilon}|\mathbf{X}\sim N(\mathbf{0},\sigma^2\mathbf{I_n})\\
                        \varepsilon_i|X_1,X_2,\cdots,X_k\sim N(0,\sigma^2)
                    \end{gather*}
          \end{enumerate}
\end{enumerate}



\subsubsection{模型设计}

\paragraph{总体回归模型}

(population regression model, PRM)

又称数据生成过程(data creating process, DGP)

\begin{align*}
Y_{i} &= E(Y|X_{i1},X_{i2},\cdots,X_{iK})+\varepsilon_{i}\\ 
&=\beta_{0} + \beta_{1}X_{i1}+\beta_{2}X_{i2}+\cdots+\beta_{k}X_{ik}+\varepsilon_{i}, i=1,2,⋯,n\\ 
\bm{Y}&=\bm{X\beta}+\bm{\varepsilon}
\end{align*}


其中
$\bm{Y}\equiv (Y_{1} Y_{2} \cdots  Y_{n})'$,
$\bm{X}\equiv (X_{1} X_{2} \cdots X_{k})'$,
$X_{i}\equiv (X_{i1} X_{i2} ⋯ X_{ik})'$,
$\bm{\beta}\equiv (\beta_{0} \beta_{1} \beta_{2} \cdots \beta_{k})'$,
$\varepsilon \equiv (\varepsilon_{1} \varepsilon_{2} \cdots \varepsilon_{n})'$。

\begin{enumerate}[1.]
    \item $Y_i$:被解释变量(explained variable)/因变量(dependent variable)/响应变量(response variable),假设为随机的
    \item 确定性部分/系统性部分:$E(Y|X_{i1},X_{i2},\cdots,X_{ik})$;系统性误差,不同水平之间的差异,所属总体的特征
    \begin{enumerate}[(1)]
        \item $X$:解释变量(explanatory variable,regressor)/自变量(independent variable)/协变量(covariate),非随机
        \item $\beta_j(j=1,2,⋯,k)$:偏回归系数(partial regression coefficient) 
        \item 控制变量:为了准确估计感兴趣参数而控制的解释变量(control variables)
    \end{enumerate}
    \item 随机部分/非系统型部分:
    \begin{enumerate}[(1)]
        \item $\varepsilon_i$ :离差(deviation)/随机误差项(stochastic error)/随机扰动项/随机干扰项(stochastic disturbance);随机误差,水平内部存在的差异,单个个体的个别特征
        \par 设置原因:
    \begin{enumerate}[a.] 
        \item 原生:代表未知的影响因素;变量的内在随机性  
        \item 衍生(可避免):代表残缺数据;代表数据观测误差;代表众多细小影响因素;代表模型设定误差
    \end{enumerate}
    \end{enumerate}
\end{enumerate}




\paragraph{样本回归模型}

\begin{align*}
Y_i & ={\hat{Y}}_i+{\hat{\varepsilon}}_i ={\hat{\beta}}_0+{\hat{\beta}}_1X_{i1}+{\hat{\beta}}_2X_{i2}+\cdots+{\hat{\beta}}_kX_{ik}+{\hat{\varepsilon}}_i\ , i=1,2,\cdots,n \\  
\mathbf{Y}&=\mathbf{X}\hat{\bm{\beta}}+\hat{\bm{\varepsilon}}
\end{align*}


\par ${\hat{Y}}_i$:样本回归函数(sample regression function,SRF)
\par ${\hat{\varepsilon}}_i$:(样本)残差项/剩余项(residual)

\subsubsection{检验与评价}

\paragraph{拟合优度检验}

$$
TSS=ESS+RSS
$$

\par 总离差平方和(Total Sum of Squares):  
$$ TSS=\sum_{i=1}^{n}y_i^2=\sum_{i=1}^{n}\left(Y_i-\bar{Y}\right)^2=SST $$  
\par 回归平方和(Explained Sum of Squares): 回归方程能解释的,  
$$ ESS=\sum_{i=1}^{n}{\hat{y}}_i^2=\sum_{i=1}^{n}\left({\hat{Y}}_i-\bar{Y}\right)^2=SSA $$
\par 残差平方和(Residual Sum of Squares): 回归方程不能解释的  
$$ RSS=\sum_{i=1}^{n}{\hat{\varepsilon}}_i^2=\sum_{i=1}^{n}\left(Y_i-{\hat{Y}}_i\right)^2=SSE $$


1、拟合优度/可决系数/判定系数:

\begin{gather*}
R^2=\frac{ESS}{TSS} 
=\frac{\sum_{i=1}^{n}\left({\hat{Y}}_i-\bar{Y}\right)^2}{\sum_{i=1}^{n}\left(Y_i-\bar{Y}\right)^2} 
=1-\frac{RSS}{TSS}
=1-\frac{\sum_{i=1}^{n}{\hat{\varepsilon}}_i^2}{\sum_{i=1}^{n}\left(Y_i-\bar{Y}\right)^2},0≤R^2≤1 \\ 
R^2=\left[Corr\left(Y_i,{\hat{Y}}_i\right)\right]^2
\end{gather*}


\par 含义:模型能解释Y变异或波动的$R^2$  
\par 如果增加解释变量,$R^2$往往增大(至少不变)

2、校正拟合优度(adjusted $R^2$):

\begin{gather*}
    {\bar{R}}^2 
=1-\frac{\sum_{i=1}^{n}{\varepsilon_i^2/(n-k-1)}}{\sum_{i=1}^{n}{(Y_i-\bar{Y})^2}/(n-1)} 
=1-(1-R^2)\frac{n-1}{n-k-1}, 0\leqslant R^2\leqslant 1
\end{gather*}


\paragraph{显著性检验}

1、$t$检验

变量回归系数的显著性检验($n-k\geq 8$时较为稳定)

$$
H_0:\beta_j=0; H_1:\beta_j\neq 0, j=1,2,\cdots,k
$$

一元:

\begin{gather*}
\beta_1:
t
=\frac{\hat{\beta}_1-\beta_1}{S_{\hat{\beta}_1}} 
=\frac{\hat{\beta}_1-\beta_1}{\sqrt{\frac{\hat{\sigma}^2}{\sum_{i=1}^{n}{x_i^2}}}}\\
\beta_0: 
t
=\frac{\hat{\beta}_0-\beta_0}{S_{\hat{\beta}_0}} 
=\frac{\hat{\beta}_1-\beta_1}{\sqrt{\frac{\hat{\sigma}^2\sum_{i=1}^{n}{X_i^2}}{n \sum_{i=1}^{n}{x_i^2}}}}
\end{gather*}


多元:
\begin{gather*}
    t=\frac{\hat{\beta}_j-\beta_j}{S_{\hat{\beta}_j}} =\frac{\hat{\beta}_j-\beta_j}{\sqrt{c_{jj}\frac{\sum_{i=1}^{n}{\varepsilon_i^2}}{n -k-1}}} =\frac{\hat{\beta}_j-\beta_j}{\sqrt{c_{ij}\frac{\bm{\hat{\varepsilon}}' \bm{ \hat{\varepsilon}}}{n -k-1}}} \sim t(n-k-1)
\end{gather*}


$c_{jj}$:$\left(\bm{X}^\prime\bm{X}\right)^{-1}$对角线上的第$j$个元素

若$\left|t\right|>t_{\frac{a}{2}}$,拒绝$H_0$

2、$F$检验

总体线性关系显著性($n\geq 30$或至少$n\geq3\left(k+1\right)$基本要求)

\begin{gather*}
    H_0:\beta_1=\beta_2=\cdots=\beta_k=0;H_1:\beta_{j}, j=1,2,\cdots,k\text{不全为零}
\end{gather*}


一元:t检验与F检验一致

多元:
\begin{gather*}
F=\frac{ESS/k}{RSS/(n-k-1)} = \frac{\sum_{i=1}^{n}{\hat{y}_i^2/k}}{\sum_{i=1}^{n}{\hat{\varepsilon}_i^2/(n-k-1)}}\sim F(k,n-k-1)
\end{gather*}


若$F>F_\alpha$,拒绝$H_0$

$$
{\bar{R}}^2=1-\frac{n-1}{n-k-1+kF}, F=\frac{R^2/k}{(1-R^2)/(n-k-1)}
$$












\subsection{时间序列模型}

\subsubsection{平稳性}







时间序列的平稳性:假定某个时间序列是由某一随机过程(stochastic process)生成的,即假定时间序列${X_t}(t=1,2,…)$的每一个数值都是从一个概率分布中随机得到,如果满足下列条件:
\begin{enumerate}[(1)]
    \item 均值$E(X_t)=\mu$是与时间$t$无关的常数  
    \item 方差$Var(X_t)=\sigma^2$是与时间$t$无关的常数  
    \item 协方差$Cov\left(X_t,X_{t+k}\right)=\gamma_k$是只与时期间隔$k$有关,与时间$t$无关的常数
\end{enumerate}


\paragraph{平稳时间序列}



\begin{enumerate}[1.]
    \item 分类
    \begin{enumerate}[(1)]
        \item 严平稳
        \begin{enumerate}[a.]
            \item $\forall \text{正整数}m,\forall t_1,t_2,⋯,t_m\in T,\forall \text{正整数}\tau$,有:
            \begin{gather*}
                F_{t_1,t_2,⋯,t_m}(x_1,x_2,⋯,x_m)=F_{t_{1+\tau },t_{2+\tau },⋯,t_{m+\tau }}(x_1,x_2,⋯,x_m)
            \end{gather*}
        \end{enumerate}
        \item 宽平稳
        \begin{enumerate}[a.]
            \item $EX_t^2<\infty,\forall t\in T$  
            \item $EX_t=\mu$,$\mu$为常数,$\forall t\in T$  
            \item $\gamma (t,s)=\gamma (k,k+s-t)$,$\forall t,s,k$且$k+s-t\in T$
        \end{enumerate}
    \end{enumerate}
    \item 好处
    \begin{enumerate}
        \item 可替代随机抽样假定
        \item 有效减少虚假回归/伪回归(spurious regression)
    \end{enumerate}
    \item 例子
\begin{enumerate}[(1)]
    \item 白噪声序列(white noise):$X_t=\mu_t, \mu_t\sim N(0,\sigma^2)$,又称纯随机序列
\begin{enumerate}[a.]
    \item 纯随机性:$\gamma(k)=0,\forall k\neq 0$,各序列值之间没有任何相关关系,即为“没有记忆”的序列,如果时间序列是白噪声,分析则无意义  
    \item 方差齐性:$DX_t=\gamma(0)=\sigma^2$。根据马尔可夫定理,只有方差齐性假定成立时,用最小二乘法得到的未知参数估计值才是准确的、有效的
\end{enumerate}
\item 检验:
    \begin{enumerate}[a.]
        \item $H_0:\rho _1=\rho _2=\cdots =\rho _m=0,\forall m\geqslant 1$ 。延迟期数小于或等于m期的序列值之间相互独立。为白噪声序列  
        \item $H_1$:至少存在某个$\rho _k\neq 0,\forall m\geqslant 1,k\leqslant m$。延迟期数小于或等于m期的序列值之间有相关性  
        \item Q统计量:$Q=n\sum_{k=1}^{m}{\hat{\rho}}_k^2\sim \chi^2\left(m\right)$  
        \item LB(Ljung-Box)统计量:$LB=n(n+2)\sum_{k=1}^{m}{(\frac{{\hat{\rho}}_k^2}{n-k})}\sim\chi^2(m)$
    \end{enumerate}
    \item d阶单整序列(integrated of d):原时间序列经过d次差分变平稳
    \begin{enumerate}[a.]
        \item 一阶自回归$AR(1)$过程:$X_t=\phi X_{t-1}+\mu_t,\mu_t\sim N(0,\sigma_2)$  
        \item $\left|\phi\right|>1$:随机过程生成的时间序列发散  
        \item $\Phi=1$:随机游走(random walk)过程,非平稳  
        \item $\Delta X_t=X_t-X_{t-1}=\mu_t$
    \end{enumerate}
\end{enumerate}
\end{enumerate}








\paragraph{检验}
单位根检验:通过检验特征根是在单位圆内还是单位圆上(外),来检验序列的平稳性(unit root test)

\subparagraph{图示法}

\begin{enumerate}[1.]
    \item 时序图检验:根据平稳时间序列均值、方差为常数的性质,平稳序列的时序图应该显示出该序列始终在一个常数值附近随机波动,而且波动的范围有界、无明显趋势及周期特征  
    \item 自相关图检验:平稳序列通常具有短期相关性。该性质用自相关系数来描述就是随着延迟期数的增加,平稳序列的自相关系数会很快地衰减向零
\end{enumerate}


\subparagraph{DF检验(Dicky-Fuller test)}

一阶序列相关

\par $X_t=\alpha+\rho X_{t-1}+\mu_t$或$\Delta X_t=\alpha +\delta X_{t-1}+\mu _{t}$  
\par $H_0$:$\delta =0$,即$\rho =1$;$H_1$:$\delta <0$,即$\rho <1$  
\par $\rho=\left|\varphi_1\right|-1, \tau=\frac{\hat{\rho}}{S(\hat{\rho})}$  
\par $H_0$:$\varphi _1≥1$,非平稳;  
\par $\left|\varphi_1\right|<1$时,$t(\varphi _1)=\frac{\hat{\varphi }_1-\varphi _1}{S(\hat{\varphi }_1)}$渐进$N(0,1)$  
\par $\left|\varphi_1\right|=1$时,$\tau =\frac{|\hat{\varphi  }_1|-1}{S(\hat{\varphi }_1)}$极限$\frac{\int_{0}^{1}W(r)dW(r)}{\int_{0}^{1}[W(r)]^2dr}$  
\par 若拒绝$H_0$($t<\tau$临界值,$t$统计量有偏误)则为平稳序列  
\begin{gather*}
x_t=\phi_{1}x_{t-1}+\varepsilon_t\\
x_t=\mu+\varphi_{1}x_{t-1}+\varepsilon_t\\
x_t=\mu+\beta t+\varphi_{1}x_{t-1}+\varepsilon_t
\end{gather*}


\subparagraph{ADF检验(Augmented Dickey-Fuller test)}

多阶序列相关,用于方差齐性的情况。

若$AR(p)$序列有单位根存在,则自回归系数之和恰好等于1  
\begin{gather*}
    \lambda^p-\varphi_1\lambda^{p-1}-\cdots-\varphi_p=0
    \Rightarrow^{\lambda=1} 1-\varphi_1-\cdots-\varphi_p=0
    \Rightarrow\varphi_1+\varphi_2+\cdots+\varphi_p=1\\
    x_t=\varphi_{1}x_{t-1}+\cdots+\varphi_{p}x_{t-p}+\varepsilon_t\\
    x_t=\mu+\varphi_{1}x_{t-1}+\cdots+\varphi_{p}x_{t-p}+\varepsilon_t \\
    x_t=\mu+\beta t+\varphi_{1}x_{t-1}+\cdots+\varphi_{p}x_{t-p}+\varepsilon_t
\end{gather*}


 
\par 模型1:$\Delta X_t=\delta X_{t-1}+\sum_{i=1}^{m}\beta _i\Delta X_{t-i}+\varepsilon _t$  
\par 模型2:$\Delta X_t=\alpha +\delta X_{t-1}+\sum_{i=1}^{m}\beta _i\Delta X_{t-i}+\varepsilon _{t}$  
\par 模型3:$\Delta X_t=\alpha +\beta _T+\delta X_{t-1}+\sum_{i=1}^{m}\beta _i\Delta X_{t-i}+\varepsilon _{t}$  
\par $T$:趋势项,时间变量,代表时间序列随时间变化的趋势  
\par $\alpha$:漂移项   
\par $m$:滞后阶数,一般用$LM$检验确定  
\par $H_0:\delta =0; H_1:\delta <0$;$\tau=\frac{\hat{\rho}}{S(\hat{\rho})}$  
\par 自下而上检验,拒绝$H_0$($t<\tau$临界值)则为平稳序列,否则继续检验

\subparagraph{PP检验}

异方差的平稳性检验

\begin{gather*}
    Z\left(\tau\right) =\tau(\hat{\sigma}^2/\hat{\sigma}_{Sl}^2 ) -\frac{1}{2}  (\hat{\sigma}_{Sl}^2-\hat{\sigma}^2) T \sqrt{\hat{\sigma}_{Sl}^2 \sum_{t=2}^{T}{(x_{t-1}-x_{T-1})^2}} \\
    {\hat{\sigma}}^2=T^{-1}\sum_{t=1}^{T}{\hat{\varepsilon}}_t^2\\
    {\hat{\sigma}}_{Sl}^2=T^{-1}\sum_{t=1}^{T}{\hat{\varepsilon}}_t^2+2T^{-1}\sum_{j=1}^{l}{w_j(l)\sum_{t=j+1}^{T}{{\hat{\varepsilon}}_t{\hat{\varepsilon}}_{t-j}}}\\
    \bar{x}_{T-1}=\frac{1}{T-1}\sum_{t=1}^{T-1}{x_t}
\end{gather*}












\subsection{面板模型}


\informationBox{
    * 面板类型:
    \begin{enumerate}[(1)]
    \item 短面板:时间维度T较少,个体维度n较多(short panel)
    \item 长面板:时间维度T较多,个体维度n较少(long panel)
    \item 平衡面板:样本中每个时期的个体完全一样(balanced panel)
    \item 非平衡面板:样本中每个时期的个体不完全一样(unbalanced panel),又称不完全面板(incomplete panel)
    \item 动态面板:解释变量包括被解释变量的滞后值(dynamic panel)
    \item 静态面板:解释变量不包括被解释变量的滞后值(static panel)  
    \end{enumerate}
}

\par 检验:豪斯曼检验(Hausman test)
\par $Cov(\alpha_i,X_{it})=0$ → 无关(随机效应模型):FE和RE都是一致的,但RE更有效
\par $Cov(\alpha_i,X_{it})\neq 0 $ → 有关(固定效应效应):FE仍然一致,但RE是有偏的



\subsubsection{固定效应模型}

假设: $Cov(\alpha_i,X_{it})\neq0$

截面个体和时点变截距模型(双向固定效应模型):
$$Y_{it} =\alpha_i+\gamma_t+\beta_1X_{1,it}+\beta_2X_{2,it}+\cdots+\beta_kX_{k,it}+\varepsilon_{it}\left(i=1,2,\cdots,n\right) $$ \\  
\par $\alpha_i$ :个体固定效应(individual fixed effects),随个体而变,不随时间而变
\par $\gamma_t$ :时间固定效应(time fixed effect),随时间而变,不随个体而变

\informationBox{
    * 估计方法
    \begin{enumerate}[1.]
        \item 虚拟变量法:
        \begin{enumerate}[(1)]
            \item 实质上就是在传统的线性回归模型中加入N-1个虚拟变量,使得每个截面都有自己的截距项
        \end{enumerate}
        \item 组内离差法:
         \begin{enumerate}[(1)]
            \item 可以通过先从Y和X中减去个体和时间平均值,然后估计被减后的Y关于被减后的X的多元回归方程的方法来估计X的系数。这种方法可以避免二元变量的出现。
            \item 从Y, X和时间指示变量中减去个体(不是时间)均值然后估计,被减后的Y对被减后的X和被减后的时间指示变量的多元回归中的k+T个系数  
         \end{enumerate}
    \end{enumerate}
}

\subsubsection{随机效应模型}
假设: $Cov(\alpha_i,X_{it})=0 $

模型:
\begin{align*} 
    Y_{it} & = \beta_1 X_{1,it}+\beta_2 X_{2,it}+\cdots +\beta_k X_{k,it}+v_{it} \\ & =\beta_1 X_{1,it}+\beta_2 X_{2,it}+\cdots +\beta_k X_{k,it}+\alpha_i+u_{it} (i=1,2,⋯,n) 
\end{align*} 
\par $v_{it}$ :随机误差项
\par $\alpha_i$ :不随时间变化的误差项
\par $u_{it}$ :随时间变化的误差项


\subsubsection{混合回归模型(polled regression model)}
每个个体都有完全一样的回归方程,实质为OLS




\subsection{放宽假设:多重共线性}



\subsubsection{截面数据}




多重共线性:某几个解释变量之间出现相关性(multi-collinearity)

$$
Cov\left(X_i,X_j\right)\neq0,\ i\neq j 
$$

\paragraph{原因}

\begin{enumerate}[1.]
    \item 经济变量相关的共同趋势  
    \item 模型设定不谨慎  
    \item 样本资料的限制
\end{enumerate}


\paragraph{后果}


\begin{enumerate}[1.]
    \item 完全共线性(perfect multicollinearity)
          \begin{enumerate}[(1)]
              \item 参数估计量不存在:相关的变量会合并,多元回归模型会退化
              \item 存在$c_1X_{i1}+c_2X_{i2}+\cdots+c_kX_{ik}=0,c_i$不全为0,即某几个解释变量可以用其他解释变量的线性组合表示
          \end{enumerate}
    \item 近似共线性(approximate multicollinearity),又称交互相关(intercorrelated)
          \begin{enumerate}[(1)]
              \item 普通最小二乘法参数估计量的方差变大。存在$c_1X_{i1}+c_2X_{i2}+\cdots+c_kX_{ik}+v_i=0$,$c_i$不全为0,$v_i$为随机干扰项
              \item 例:二元线性回归:$Var(\hat{\beta}_{1})=\frac{σ2}{\sum{i=1}^{n}{x_{i1}^{2}}}⋅\frac{1}{(1-r)^2}>\frac{σ2}{\sum_{i=1}^{n}{x_{i1}^{2}}}$
              \item 方差膨胀因子(variance inflation factor,VIF):$VIF\left({\hat{\beta}}_1\right)=\frac{1}{1-r^2}$,越高说明其越能被其他自变量所解释
          \end{enumerate}
    \item 参数估计量经济意义不合理
          \begin{enumerate}[(1)]
              \item 相关变量的参数反映共同影响,非单独影响
          \end{enumerate}
    \item 变量的显著性检验和模型的预测功能失去意义
          \begin{enumerate}[(1)]
              \item 不影响一致性,只影响有效性(方差变大)
              \item 方差变大,t值小于临界值误导做出参数为零的推断
              \item 方差变大,区间变大
          \end{enumerate}
\end{enumerate}


\paragraph{检验}


\begin{enumerate}[1.]
    \item 是否存在
          \begin{enumerate}[(1)]
            \item 两个解释变量:简单相关系数法
              \begin{enumerate}[a.]
                \item  $|r|$接近1
              \end{enumerate}
              \item 多个解释变量:综合统计检验法
              \begin{enumerate}[a.]
                \item  模型的$R^2$与$F$值较大 → 各解释变量对$Y$的联合线性作用显著
                \item  各参数估计的$t$检验值较小 → 各解释变量间存在共线性使得它们对$Y$的独立作用不能分辨
              \end{enumerate}
          \end{enumerate}
    \item 存在范围
          \begin{enumerate}[(1)]
              \item 判定系数检验法
                    \begin{enumerate}[a.]
                        \item 构造F检验:$F_j=\frac{R_{j\cdot}^{2}/(k-1)}{(1-R_{j\cdot}^{2})/(n-k)}\sim F(k-1,n-k)$
                        \item $R_{j\cdot}$是第$j$个解释变量对其他解释变量的回归方程的决定系数
                        \item 进行辅助回归(auxiliary regression),如果某一种回归$X_{ji}=\alpha_{1}X_{1i}+\alpha_{2}X_{2i}+\cdots+\alpha_{L}X_{Li}$的判定系数较大,说明$X_j$与其他$X$间存在共线性。
                    \end{enumerate}
              \item 排除变量法(Stepwise Backward Regression)
                    \begin{enumerate}[a.]
                        \item 在模型中排除某一个解释变量$X_j$,估计模型;
                        \item 如果拟合优度与包含$X_j$时十分接近,则说明$X_j$与其它解释变量之间存在共线性
                    \end{enumerate}
              \item 逐步回归法(Stepwise Forward Regression)
                    \begin{enumerate}[a.]
                        \item 以$Y$为被解释变量,逐个引入解释变量
                        \item 若拟合优度变化(增加)显著,说明新引入变量是独立解释变量
                    \end{enumerate}
          \end{enumerate}
\end{enumerate}



\paragraph{解决方法}



\begin{enumerate}[1.]
    \item 剔除
          \begin{enumerate}[(1)]
              \item 剩余系数的估值和经济意义发生变化;可能产生内生性问题
          \end{enumerate}
    \item 减小参数估计量的方差
          \begin{enumerate}[(1)]
              \item 增加样本容量
              \item 岭回归法(ridge regression)
          \end{enumerate}
\end{enumerate}






\subsection{放宽假设:序列相关性}

\subsubsection{截面数据}
空间相关(spatial autocorrelation),空间计量经济学。




\subsubsection{时序数据}







序列相关性:模型的随机干扰项之间存在相关性(serial correlation),又称自相关(autocorrelation)

$$
Cov\left(\varepsilon_i,\ \varepsilon_j\right)=E\left(\varepsilon_i\varepsilon_j\right)\neq0\\
$$

\paragraph{原因}

1、经济变量固有的惯性:如消费习惯  
2、模型设定的偏误(specification error):虚假序列相关(只有非平稳时间序列之间才能出现虚假回归),如人口和GDP总量  
3、数据生成:根据已知数据生成所需数据时,生成数据与原数据产生内在联系

\paragraph{后果}

1、参数估计量非有效:OLS估计量仍然具有线性性、无偏性、一致性、渐进正态分布  
2、变量的显著性检验失去意义:t、F统计量利用了同方差条件  
3、模型的预测失效

\paragraph{检验}

原理:分析随机误差项的近似估计量${\hat{\varepsilon}}_t\approx\widetilde{{\hat{\varepsilon}}_t}=Y_t-\left({\hat{Y}}_t\right)_{OLS}$之间的相关性


\begin{enumerate}[1.]
    \item 图示法
    \item 回归检验法:存在某一种函数形式,使得方程显著成立,则存在序列相关性
    \begin{gather*}
    {\hat{\varepsilon}}_t=\rho{\hat{\varepsilon}}_{t-1}+e_t, t=2,\cdots,T\\ 
    {\hat{\varepsilon}}_t=\rho_1{\hat{\varepsilon}}_{t-1}+\rho_2{\hat{\varepsilon}}_{t-2}+e_t,t=3,⋯,T
    \end{gather*}
    \item DW检验(Durbin-Watson test):检验一阶序列相关
    \begin{enumerate}[(1)]
        \item 假设:
        \begin{enumerate}[a.]
            \item 解释变量X非随机  
            \item 随机误差项$\varepsilon_t$为一阶自回归形式:$\varepsilon_t=\rho\varepsilon_{t-1}+e_t$  
            \item $\rho$:自协方差系数(coefficient of autocovariance),又称一阶自相关系数(first-order coefficient of autocorrelation)  
            \item 回归模型中不应含有滞后因变量作为解释变量  
            \item 回归含有截距项
        \end{enumerate}
        \item 步骤:
        \begin{gather*}
            H_0: E(\varepsilon _{t},\varepsilon _{t-1})=0\Leftrightarrow \rho =0(\varepsilon _{t}\text{不存在一阶自回归})\\ 
            D.W.=\frac{\sum_{t=2}^{n}\left({\hat{\varepsilon}}_t-{\hat{\varepsilon}}_{t-1}\right)^2}{\sum_{t=1}^{n}{\hat{\varepsilon}}_t^2}\\ 
            \text{若}0<D.W.<d_{L} \text{,存在正自相关}\\ 
            \text{若}d_{L}<D.W.<d_{U} \text{,不能确定}\\ 
            \text{若}d_{U}<D.W.<4-d_{U} \text{,无自相关}(D.W.\approx 2)\\ 
            \text{若}4-d_{U}<D.W.<4-d_{L} \text{,不能确定}\\ 
            \text{若}4-d_{L}<D.W.<4 \text{,存在负自相关}
        \end{gather*}
        \item Durbin h检验:当回归因子包含延迟因变量时,残差序列的DW统计量是一个有偏统计量。在这种场合下使用DW统计量容易产生残差序列正自相关性不显著的误判
        \begin{gather*}
            Dh=DW\frac{n}{1-n\sigma_\beta^2}
        \end{gather*} 
    \end{enumerate}
    \item 拉格朗日乘数检验(Lagrange multiplier)检验(LM test;GB test):p阶序列相关
    \begin{gather*}
        \varepsilon_t=\rho_1\varepsilon_{t-1}+\rho_2\epsilon_{t-2}+\cdots+\rho_p\varepsilon_{t-p}+e_t\\ 
        Y_t=\beta_0+\beta_1X_{t1}+\cdots+\beta_kX_{tk}+\rho_1\varepsilon_{t-1}+\rho_2\varepsilon_{t-2}+\cdots+\rho_p\varepsilon_{t-p}+e_t
    \end{gather*}
    \begin{enumerate}[(1)]
        \item $H_0$:$\rho_1=\rho_2=\cdots=\rho_p=0$
        \item $LM={(T-p)R}^2\sim\chi_\alpha^2(p)$
        \item $T$:$\varepsilon_t=\beta_0+\beta_{1}X_{t1}+\cdots+\beta_{k}X_{tk}+\rho_{1}\varepsilon_{t-1}+\rho_{2}\varepsilon_{t-2}+\cdots+\rho_{p}\varepsilon_{t-p}+e_{t}$的样本容量
        \item $R^2$:$\varepsilon_t=\beta_0+\beta_1X_{t1}+\cdots+\beta_kX_{tk}+\rho_1\varepsilon_{t-1}+\rho_{2}\varepsilon_{t-2}+\cdots+\rho_{p}\varepsilon_{t-p}+e_{t}$的可决系数
        \item 若LM超过临界值,拒$H_0$
    \end{enumerate}
\end{enumerate}


\paragraph{解决方法}

\begin{enumerate}[1.]
    \item 模型变换
          \begin{enumerate}[(1)]
              \item 广义最小二乘法
              \item 广义差分法(generalized difference method)
          \end{enumerate}
    \item 序列方差稳健估计法(serial correlation-robust method)
          \begin{enumerate}[(1)]
              \item 序列相关稳健标准误法(method of serial correction-robust standard error)
          \end{enumerate}
\end{enumerate}




















\subsection{放宽假设:异方差性}


\subsubsection{截面数据}







异方差性:对于不同的样本点,随机干扰项的方差不是常数,互不相同(heteroscedasticity)

\informationBox{
    * 类型:
    \begin{enumerate}
        \item 单调递增型:$\sigma_i^2$随X的增大而增大  
        \item 单调递减型:$\sigma_i^2$随X的增大而减小  
        \item 复杂型:$\sigma_i^2$与$X$的变化呈复杂形式  
    \end{enumerate}
}

\paragraph{后果}

\begin{enumerate}[1.]
    \item 参数估计量非有效:OLS估计量仍然具有线性性、无偏性、一致性、渐近正态
    \item 变量的显著性检验失去意义:t、F统计量利用了同方差条件
    \item 模型的预测失效
\end{enumerate}


\paragraph{检验}

原理:首先采用OLS估计,得到残差估计值,用它的平方近似随机误差项的方差


\begin{enumerate}[1.]
    \item 图示检验法
          \begin{enumerate}[(1)]
              \item $X$与$e_i^2$的散点图
          \end{enumerate}
    \item 帕克检验(Park test)
          \begin{enumerate}[(1)]
              \item $\tilde{e} _i^2 =f(X_{ji})+\varepsilon_{i} =\sigma^2X_{ji}^\alpha e^{\varepsilon_i}$
              \item 若$\alpha$在统计上是显著的,表明存在异方差性
          \end{enumerate}
    \item 戈里瑟检验(Glejser test)
          \begin{gather*}
              \left|{\widetilde{e}}_i^2\right| =f\left(X_{ji}\right)+\varepsilon_i
          \end{gather*}
    \item 布罗施-帕甘检验(Breusch-Pagan test)
          \begin{enumerate}[(1)]
              \item $\varepsilon_i^2\approx{\hat{\varepsilon}}_i^2=\delta_0+\delta_1X_{i1}+\delta_2X_{i2}+\cdots+\delta_kX_{ik}+e$ ,可决系数为$R_{{\hat{\varepsilon}}^2}^2$
              \item $H_0$:$δ_0=δ_1=δ_2=\cdots=δ_k=0$
                    \begin{gather*}
                        F =  \frac{ R_{{\hat{\varepsilon}}^2}^2/k}{ {(1-R}_{{\hat{\varepsilon}}^2}^2)/(n-k-1)}
                    \end{gather*}
              \item 拉格朗日乘数检验:$LM=n⋅R_{{\hat{\varepsilon}}^2}^2\sim \chi(k)$
              \item 若$F$或$LM$>临界值则拒绝$H_0$,表明存在异方差性
          \end{enumerate}
    \item 怀特检验(White test)
          \begin{enumerate}[(1)]
              \item 二元为例:
                    \begin{gather*}
                        \varepsilon_i^2\approx{\hat{\varepsilon}}_i^2=\delta_0+\delta_1X_{i1}+\delta_2X_{i2}+\delta_3X_{i1}^2+\delta_4X_{i2}^2+\delta_5X_{i1}X_{i2}+e_i \\
                        H_0:δ_1=δ_2=\cdots=δ_5=0\\
                        F=\frac{R_{{\hat{\varepsilon}}^2}^2/k}{{(1-R}_{{\hat{\varepsilon}}^2}^2)/(n-k-1)}\\
                        LM=n⋅R_{{\hat{\varepsilon}}^2}^2\sim \chi(k)
                    \end{gather*}
              \item 若$F$或$LM$>临界值则拒绝$H_0$,表明存在异方差性
          \end{enumerate}
    \item 斯皮尔曼等级相关检验法(Spearman Rank Relation test)
    \item 戈德弗尔德-匡特检验法(Goldfeld Quandt test)
\end{enumerate}



\paragraph{解决方法}

\begin{enumerate}[1.]
    \item 广义最小二乘法(Generalized Least Squares,GLS)
    \item 加权最小二乘法(Weighted Least Squares, WLS)
          \begin{enumerate}[(1)]
              \item GLS的特例,在采用OLS方法时:对较小的残差平方 $\hat{\varepsilon}_i^2$赋予较大的权数;对较大的残差平方$\hat{\varepsilon}_i^2$赋予较小的权数
                    \begin{gather*}
                        \sum{w_i \hat{\varepsilon}_i^2}=\sum{w_i\left[Y_i-(\hat{\beta}_0+\hat{\beta}_1 X_1 + \hat{\beta}_2 X_2 +\cdots+\hat{\beta}_k X_k)\right]^2}, w_i\text{为权数}
                    \end{gather*}
          \end{enumerate}
    \item 异方差稳健标准误法(Heteroscedasticity-Consistent Variances and Standard Errors)
          \begin{enumerate}[(1)]
              \item 仍然采用OLS,但对OLS估计量的标准差进行修正
              \item 一元回归:大样本下,$\sum{ \frac{x_i^2 \hat{\varepsilon}_i^2}{(\sum{x_i^2})^2}}$是$Var(\hat{\beta}_1) = \sum{ \frac{x_i^2 \sigma_i^2}{(\sum{x_i^2})^2}}$的一致估计
          \end{enumerate}
\end{enumerate}










\subsubsection{时序数据}









异方差性:随机误差序列的方差随时间变化而变化

\begin{gather*}
    Var(\varepsilon_t)=h(t)
\end{gather*}


\paragraph{后果}
忽视异方差的存在会导致残差的方差会被严重低估,继而参数显著性检验容易犯纳伪错误(Type II error),这使得参数的显著性检验失去意义,最终导致模型的拟合精度受影响

\paragraph{检验}

残差序列的方差实际上就是它平方的期望:
$Var(\varepsilon_t)=E(\varepsilon_t^2)$,所以考察残差序列是否方差齐性,主要是考察残差平方序列是否平稳

\begin{enumerate}[1.]
    \item 残差图
    \item 残差平方图
\end{enumerate}


\paragraph{解决方法}

\begin{enumerate}[1.]
    \item 已知异方差函数形式:方差齐性变化
          \begin{enumerate}[(1)]
              \item 序列显示出显著的异方差性,且方差与均值之间具有某种函数关系$σ_t^2=h(\mu_t)$,$h(⋅)$是某个已知函数
              \item 尝试寻找一个转换函数$g(⋅)$,使得经转换后的变量满足方差齐性$Var[g(x_t)]=\sigma^2$
              \item 转换函数$g(x_t)$在$\mu_t$附近作一阶泰勒展开 $g(x_{t})\cong g(\mu_t)+(x_{t} - \mu_{t} ) g' (\mu_t)$
              \item 求转换函数的方差  $Var[g(x_t)]\cong Var[g(\mu_t)+(x_t-\mu_t)g^{'} (\mu_t)]=[g^{'} (\mu_t)]^2 h(\mu_t)$
              \item 转换函数的确定$g^{'} (\mu_t)=1/\sqrt{h(\mu_t)}$
          \end{enumerate}
    \item ARCH模型
          \begin{enumerate}[(1)]
              \item 自回归条件异方差模型:通过构造残差平方序列的自回归模型来拟合异方差函数(Autoregressive Conditional Heteroscedasticity Model,ARCH)
          \end{enumerate}
    \item GARCH模型
          \begin{enumerate}[(1)]
              \item 广义自回归条件异方差模型(generalized autoregressive conditional heteroscedasticity model,GARCH)
          \end{enumerate}
    \item TGARCH模型
    \item EGARCH模型
\end{enumerate}










\subsection{放宽假设:内生解释变量}








内生性:同期相关(同期内生),解释变量与扰动项相关

$$
Cov(\mathbf{X_i},\varepsilon_i)\neq\mathbf{0}
$$

\subsubsection{原因}

\begin{enumerate}[1.]
    \item 双向因果关系:面板数据被解释变量领先一期
    \item 样本自选择
    \item 遗漏变量
    \item 测量误差
\end{enumerate}


\subsubsection{解决办法}




\paragraph{工具变量法(IV)}

\paragraph{工具变量(instrumental variable,IV)}


\begin{enumerate}[1.]
    \item 要求
    \begin{enumerate}[(1)]
        \item 相关性:工具变量与所替代的随机解释变量高度相关,$Cov(Z_i,X_i)$  
        \item 外生性:工具变量与随机误差项不相关,$Cov(\varepsilon_i,Z_i)=0$  
    \end{enumerate}
    \item 方法:两阶段最小二乘法(TSLS)、有限信息极大似然法(LIML)
    \item 只有过度识别(工具变量数量>内生解释变量数量)时才能检验工具变量的外生性;  
    \item 恰好识别时只能依靠经济学原理、历史背景分析等
\end{enumerate}




\paragraph{倾向得分匹配(PSM)}


\paragraph{Heckman selection model}




\paragraph{合成控制法}







\subsection{放宽假设:模型设定偏误}













模型设定偏误主要包括函数形式偏误和变量选择偏误。

\subsubsection{函数形式:非线性回归分析}


\paragraph{模型转换}


\begin{enumerate}[1.]
    \item 指数函数:$y=a{x}^b\rightarrow y^\prime=lny,\ x^\prime=lnx,\ A=lna\rightarrow y^\prime=A+bx^\prime$
    \item 对数变换:
          \begin{enumerate}[(1)]
              \item 单对数变换:$X$变化1单位,$Y$平均增长率为 $\beta$
                    \begin{gather*}
                        lnY=\beta_0+\beta_1X+\varepsilon
                    \end{gather*}
              \item 双对数变换:$X$变化1单位,$X$对$Y$的弹性变化$\beta$
                    \begin{align*}
                        lnY&=\beta_0+\beta_1lnX+\varepsilon\\
                        \ln{Y_2}-lnY_1
                        &=\left[\beta_0+\beta_0\ln{\left(X_1+1\right)}+\varepsilon\right]-\left(\beta_0+\beta_0\ln{X_1}+\varepsilon\right)\\
                        &=ln\frac{Y_2}{Y_1} =\beta_0\ln{\frac{X_1+1}{X_1}}\\
                        \approx\frac{\Delta Y}{Y} =\frac{Y_2-Y_1}{Y}
                    \end{align*}
          \end{enumerate}
    \item 级数展开法
\end{enumerate}



\paragraph{非线性最小二乘法}


\begin{enumerate}[1.]
    \item 门限自回归模型
\end{enumerate}


\subsubsection{函数形式:误设函数形式}

\paragraph{检验}

\begin{enumerate}[1.]
    \item Ramsey's RESET检验(regression equation specification error test)
\end{enumerate}

\subsubsection{变量:遗漏变量}

\paragraph{解决办法}

\begin{enumerate}[1.]
    \item 加入尽可能多的控制变量  
    \item 使用代理变量(proxy variable):多余性(代理变量仅通过影响遗漏变量而作用于被解释变量)+剩余独立性(遗漏变量中不受代理变量影响的剩余部分与所有解释变量均不相关)  
    \item 工具变量法  
    \item 使用面板数据  
    \item 随机实验和自然实验 
\end{enumerate}


\subsubsection{变量:无关变量}



\paragraph{检测}

要求增加的解释变量能够减少对应值

\begin{enumerate}[1.]
    \item 赤池信息准则(Akaike information criterion,AIC)
    \begin{gather*}
        AIC = \ln{ \frac{{\hat{\bm{\varepsilon}}}^\prime\hat{\bm{\varepsilon}}}{n} }+\frac{2(k+1)}{n} + 1 + \ln(2\pi)
    \end{gather*}
    \item 施瓦茨准则(Schwarz criterion)
    \begin{gather*}
        SC = \ln{\frac{{ \hat{ \bm{\varepsilon} }}^\prime\hat{\bm{\varepsilon}}}{n}}+\frac{k+1}{n} \ln{(n)} + 1 + \ln{(2\pi)} 
    \end{gather*}
    \item 贝叶斯信息准则(Bayesian information criterion, BIC)
    \item 汉南-昆信息准则(Hannan-Quinn information criterion, HQIC)
\end{enumerate}















\subsection{放宽假设:数据问题}






\begin{equation*}
    \text{数据问题}\mindmap{
        \text{数据质量}\mindmap{
            \text{极端数据} \\
            \text{缺失数据} \\
            \text{受限被解释变量}\mindmap{
                \text{断尾回归模型} \\
                \text{归并回归模型} \\
                \text{样本选择模型}
            }
        }
        \text{数据性质}\mindmap{
            \text{解释变量}\mindmap{
                \text{定性}\mindmap{
                    \text{虚拟变量法} \\
                    \text{双重差分法} \\
                    \text{三重差分法}
                }
            }
            \text{被解释变量}\mindmap{
                \text{离散选择模型}\mindmap{
                    \text{二值选择模型} \\
                    \text{多值选择模型} \\
                    \text{计数模型}     \\
                    \text{排序模型}
                }
            }
        }
    }
\end{equation*}

\subsubsection{数据质量}


\paragraph{极端数据}


\paragraph{缺失数据}


\paragraph{受限被解释变量}

(limited dependent variable)

\subparagraph{断尾回归模型(truncated regression model)}

$\bm{Y}=\bm{X\beta}+\bm{\mu}$,$Y_i<c$时数据缺失

\subparagraph{归并回归模型(censored regression)}

$\bm{Y}=\bm{X\beta}+\bm{\mu}$,$Y_i\geq c$时数据被归并为$c$

\subparagraph{样本选择模型(sample selection)/偶然断尾(incidental truncation)}

$\bm{Y}=\bm{X\beta}+\bm{\mu}$,$Y_i$的断尾与$z_i$(二值选择变量)有关



\subsubsection{数据性质}


\paragraph{解释变量:定性}

\subparagraph{一重差分法(虚拟变量)}


\begin{enumerate}[1.]
    \item 虚拟变量只作为解释变量,自变量为离散型,一般选0和1
    \item 方法
    \begin{enumerate}[(1)]
        \item 加法方式:斜率相同,截距不同(平行回归);系数表示1类平均比0类多β
        \item 乘法方式:斜率不同,截距相同;系数表示X每增加一单位,1类比0类多增加β
        \item 加法乘法:斜率不同,截距不同
    \end{enumerate}
    \item 设置原则:每一定性变量所需的虚拟变量个数=定性变量类别-1,否则与截距项完全共线,少的变量即为常数项。全为0的为基准组。
\end{enumerate}


\subparagraph{双重差分法(differsence in difference, DID)}

\begin{enumerate}[1.]
    \item 假设
          \begin{enumerate}[(1)]
              \item 平行趋势假定
              \item 政策变量外生性
          \end{enumerate}
    \item 模型
          \begin{enumerate}[(1)]
              \item 两期:
                    \begin{gather*}
                        y_{it}=\alpha+\gamma D_t+\beta x_{it}+u_i+\varepsilon_{it} (i=1,2,\cdots,n;t=1,2) \\ \Delta y_i = \gamma+\beta x_{i2}+\Delta \varepsilon_i \\ {\hat{\beta}}_{OLS}=\Delta{\bar{y}}_{treat}-\Delta \bar{y}_{control} = \bar{y}_{treat,2}-\bar{y}_{treat,1}- \bar{y}_{control,2}-\bar{y}_{control,1}\
                    \end{gather*}
                    \par $D_t$ :实验期虚拟变量(1=实验后;0=实验前)
                    \par $u_i$ :不可观测的个体特征
                    \par $x_{it}$ :政策虚拟变量($1=i\in $实验组且$t=2$;0=其他)
          \end{enumerate}
\end{enumerate}


\subparagraph{三重差分法}


\paragraph{被解释变量:离散选择模型(discrete choice model)}


\subparagraph{二值选择模型}

(1)Probit模型

(2)Logit模型
\begin{gather*}
    \ln{\left(\frac{p}{1-p}\right)}=\\ 
\end{gather*}

$\frac{p}{1-p}$ :几率比(odds ratio),意味着y=1的概率是y=0概率的 $\frac{p}{1-p}$ 倍

\subparagraph{多值选择模型}


\subparagraph{排序模型}


\subparagraph{计数模型}


























\part{自然与工程科学}



\part{人文与社会科学}


\chapter{商品}




\begin{equation*}
    \text{商品}\mindmap{
        \text{价值}\mindmap{
            \text{原因—劳动}\mindmap{
                \text{生产}\mindmap{
                    \text{具体劳动}\\
                    \text{抽象劳动}
                }
                \text{交换}\mindmap{
                    \text{个别劳动}\\
                    \text{社会劳动}
                }
            }
            \text{衡量—价值量}\mindmap{
                \text{决定}\mindmap{
                    \text{劳动时间}\mindmap{
                        \text{个别劳动时间}\\
                        \text{社会劳动时间}
                    }
                    \text{劳动复杂程度}\mindmap{
                        \text{简单劳动}\\
                        \text{复杂劳动}
                    }
                }
                \text{指标—劳动生产率}
            }
        }
        \text{使用价值}
    }
\end{equation*}



定义:商品是用来交换的劳动产品
\informationBox{
    * 自给的农产品:交换(×),劳动产品(√)
}


\section{价值构成}



 {\noindent
  \begin{tabu}{|c|X|X|}
      \hline
                                 & \multicolumn{1}{c|}{价值}  & \multicolumn{1}{c|}{使用价值} \\ \hline
      定义
                                 & \begin{tabu}{X}
          凝结在商品中的无差别的一般人类劳动
      \end{tabu} &
      \begin{tabu}{X}
          物品和服务能够满足人们某种需要的属性,即物品和服务的有用性 \\
          * 商品使用价值与一般物品使用价值的区别                     \\
          \tablelist{(1)}{0.5\columnwidth}{
              \item 来源:是劳动产品的使用价值
              \item 过程:必须通过交换让渡
              \item 去处:对他人、社会有用,不是对生产者
          }
      \end{tabu}                                                              \\\hline
      \multicolumn{1}{|c|}{对立} & \multicolumn{2}{l|}{
          \tablelist{(1)}{\columnwidth}{
              \item 生产者和消费者不能同时占有价值和使用价值
              \item 价值:反映人与社会的关系
              \item 使用价值:反映人与自然的物质关系
      }}                                                                                      \\ \hline
      \multicolumn{1}{|c|}{统一} & \multicolumn{2}{l|}{
          \tablelist{(1)}{\columnwidth}{
              \item 使用价值是价值的基础、物质载体
              \item 通过交换,生产者消费者互相让渡使用价值和价值
          }
      }                                                                                       \\ \hline
  \end{tabu}
 }





\subsection{交换价值}

定义:商品能够通过买卖具有和其他商品相交换的属性

\informationBox{
    \par * 表现:一种使用价值同另一种使用价值相交换的量的比例关系
    \par * 关系
    \begin{enumerate}[(1)]
        \item 使用价值是交换价值的物质承担者
        \item 价值是交换价值的基础,交换价值是价值的表现形式(价值自己无法表示自己的价值大小,但交换价值的表示可能不一致)
    \end{enumerate}
    *  商品流通
    \begin{enumerate}[(1)]
        \item 可能性:价值量相同
        \item 必要性:使用价值不同
    \end{enumerate}
}


\section{价值原因:劳动}

\subsection{商品生产}

(劳动二重性)


{\noindent
\begin{tabu}{|c|X|X|}
    \hline
                               & \multicolumn{1}{c|}{具体劳动}                   & \multicolumn{1}{c|}{抽象劳动}                                     \\ \hline
    定义                       & \multicolumn{1}{l|}{在一定具体形式下进行的劳动} & \multicolumn{1}{l|}{撇开劳动的特定具体形式的无差别的一般人类劳动} \\\hline
    \multicolumn{1}{|c|}{对立}
                               & \tablelist{(1)}{0.5\columnwidth}{
        \item 反映人与自然的关系(自然属性)
        \item 创造使用价值,不是唯一源泉
        \item 形式千差万别
    }
                               &
    \tablelist{(1)}{0.45\columnwidth}{
        \item 反映人与社会的关系(社会属性)
        \item 创造价值,是唯一源泉
        \item 无质的差别
    }                                                                                                                                                \\ \hline
    \multicolumn{1}{|c|}{统一} & \multicolumn{2}{l|}{
        \tablelist{(1)}{\columnwidth}{
            \item 是同一劳动的两个方面(不是两次或两种劳动),在时空上不可分割
            \item 具体劳动是抽象劳动的基础
            \item 只有通过商品交换,具体劳动才能还原为抽象劳动
    }}                                                                                                                                               \\ \hline
\end{tabu}
}



\subsection{商品交换}

{\noindent
    \begin{tabu}{|c|X|X|}
        \hline
                                   & \multicolumn{1}{c|}{个别劳动}                 & \multicolumn{1}{c|}{社会劳动}                                 \\\hline
        定义                       & \multicolumn{1}{l|}{商品生产者各自独立的劳动} & \multicolumn{1}{l|}{经过市场交换得到社会承认的个别劳动的汇总} \\\hline
        \multicolumn{1}{|c|}{对立} & \tablelist{(1)}{0.45\columnwidth}{
            \item 表现
            \begin{enumerate}[a.]
                \item 不同所有制的劳动
                \item 私有制条件下,个人劳动即私人劳动
                \item 同一所有制内部,不同的有独立利益的企业
            \end{enumerate}
            \item 个人生产不是直接的社会生产

        }                          & \tablelist{(1)}{0.5\columnwidth}{
            \item 劳动的社会性是劳动的本质特征,是人类社会的本质特征
            \item 在商品经济中,由于社会分工的存在,商品生产者之间是相互联系、相互依存的
        }                                                                                                                                          \\\hline
        \multicolumn{1}{|c|}{统一} & \multicolumn{2}{l|}{
            \tablelist{(1)}{\columnwidth}{
                \item 个人只能为社会和在社会中生产
                \item 通过市场交换个别劳动得到市场认可,成为社会劳动
            }
        }                                                                                                                                          \\\hline
    \end{tabu}
}



\section{价值衡量:价值量}

定义:生产商品所耗费的劳动量,即凝结在商品中的一般人类劳动量

\subsection{决定}

1、劳动时间

{\noindent
\begin{tabu}{|c|X|X|}
    \hline
                               & \multicolumn{1}{c|}{个别劳动时间}                  & \multicolumn{1}{c|}{社会劳动时间} \\\hline
    \multicolumn{1}{|c|}{定义} & \begin{tabu}{X}
        独立的生产者在个别的生产条件下生产商品所耗费的劳动时间\end{tabu}
                               & \begin{tabu}{X}
        在现有的正常的生产条件下,在社会平均的劳动熟练程度和劳动强度下制造某种使用价值所需要的劳动时间 \\\\
        \tablelist{(1)}{0.5\columnwidth}{
            \item 生产同种商品的不同生产者之间形成的,涉及的是同种商品生产上的劳动耗费
            \item 生产不同商品的生产者之间形成的社会必要劳动时间,涉及社会总劳动时间在各种商品上的分配
        }
    \end{tabu}                                                             \\\hline
    \multicolumn{1}{|c|}{对立} & \tablelist{(1)}{0.5\columnwidth}{
        \item 供给角度
        \item 单个相同商品的价值
        \item 考察价值的决定
    }                          & \tablelist{(1)}{0.5\columnwidth}{
        \item 需求角度
        \item 部门商品的总价值
        \item 考察价值的实现
        \item 反映不同商品的使用价值量被社会接受的程度
    }                                                                                                                   \\\hline
    \multicolumn{1}{|c|}{统一} & \multicolumn{2}{l|}{\tablelist{(1)}{\columnwidth}{
            \item 二者共同决定商品的价值
    }}                                                                                                                  \\\hline
\end{tabu}
\ \\
}




2、劳动的复杂程度

{\noindent
\begin{tabu}{|c|X|X|}
    \hline
                               & \multicolumn{1}{c|}{简单劳动}                      & \multicolumn{1}{c|}{复杂劳动} \\\hline
    \multicolumn{1}{|c|}{定义} & \begin{tabu}{X}
        不经过专门训练和学习就能胜任的劳动\end{tabu}
                               & \begin{tabu}{X}
        需要经过专门训练和学习,具有一定技术专长的劳动 \\\\
        \tablelist{(1)}{0.5\columnwidth}{
            \item 科技劳动:科技本身不创造价值,掌握和运用科技的劳动者的活劳动创造价值
            \item 管理劳动
        }
    \end{tabu}                                                         \\\hline
    \multicolumn{1}{|c|}{对立} & \tablelist{(1)}{0.5\columnwidth}{
        \item 收入分配少
    }                          & \tablelist{(1)}{0.5\columnwidth}{
        \item 收入分配多
    }                                                                                                               \\\hline
    \multicolumn{1}{|c|}{统一} & \multicolumn{2}{l|}{\tablelist{(1)}{\columnwidth}{
            \item 复杂劳动要转化为简单劳动来比较
            \item 简单劳动、复杂劳动是相对的,随着科技发展和文化教育水平的提高,复杂劳动正变为简单劳动
    }}                                                                                                              \\\hline
\end{tabu}
}




\subsection{指标}

\subsubsection{劳动生产率}

定义:劳动者在一段时间内生产某种使用价值的效率

\begin{gather*}
    \text{劳动生产率}=\frac{\text{产品量}}{\text{劳动时间}}
\end{gather*}

\begin{enumerate}[1.]
    \item 含义
    \begin{enumerate}[(1)]
        \item 单位时间内生产的产品数量
        \item 生产单位产品所耗费的劳动时间
    \end{enumerate}
    \item 规律
    \begin{enumerate}[(1)]
        \item 同一劳动在同样的时间内提供的价值量相同
        \item 劳动生产率同商品的使用价值量成正比,同商品的价值量成反比
    \end{enumerate}
    \item 影响因素
    \begin{enumerate}[(1)]
        \item 劳动者:劳动者的平均熟练程度
        \item 劳动工具:科技发展水平,及其在生产中的应用程度
        \item 劳动对象:劳动对象的状况
        \item 管理:生产过程的社会结合(分工协作、劳动组织、生产管理)形式(生产过程各要素的契合程度)
        \item 自然条件(农业受的影响大)
    \end{enumerate}
\end{enumerate}
    





\subsubsection{利润率}


\begin{gather*}
    \text{利润率}=\frac{\text{利润额}}{\text{预付资本}}
=\frac{m}{C}
=\frac{m}{v+c}
\end{gather*}

\par $C$:全部预付资本  
\par 成本:为生产商品和实现资本价值增殖而发生的资本耗费
\par 利润:商品价值扣除生产成本后的余额


\begin{enumerate}[1.]
    \item 影响因素
          \begin{enumerate}[(1)]
              \item 高低
                    \begin{enumerate}[a.]
                        \item 劳动:剩余劳动的活劳动占比
                        \item 资本:资本的节省,资本运动的快慢
                        \item 市场:竞争,需求形势;经营者的市场经营能力和市场拓展能力
                    \end{enumerate}
              \item 变动(等式右边分子分母同时除以v)
                    \begin{enumerate}[a.]
                        \item 剩余价值率
                        \item 资本的有机构成:不变资本和可变资本的比例(部门),由技术资本构成决定并反映资本技术构成变化的资本价值构成
                        \item 资本周转速度
                        \item 不变资本的节省状况
                    \end{enumerate}
              \item
                    \begin{enumerate}[a.]
                        \item 成本:不变资本节省,活劳动占比大
                        \item 销量:数量,速度;需求,销售能力
                    \end{enumerate}
          \end{enumerate}
\end{enumerate}




\subsubsection{剩余价值率}



1、物化劳动表示法:

\begin{gather*}
m'=\frac{\text{剩余价值}}{\text{必要劳动价值}}
=\frac{\text{剩余价值}}{\text{可变资本}}
= \frac{m}{v}
\end{gather*}


2、活劳动表示法:
\begin{gather*}
    m'=\frac{\text{剩余劳动时间}}{\text{必要劳动时间}}
\end{gather*}
\par 剩余价值:活劳动在剩余劳动时间里创造的价值  
\par 必要劳动:劳动者用以实现劳动再生产而付出的劳动  
\par 剩余劳动:一定时期内劳动者的劳动中超出必要劳动的部分









\informationBox{
    * 关系: 
    \begin{enumerate}[(1)]
        \item 利润额=剩余价值额,利润率≠剩余价值率  
        \item 利润率表示劳动占比,剩余价值率表示剥削率
    \end{enumerate} 
}





\begin{enumerate}[1.]
    \item 生产方法
          \informationBox{
              \begin{enumerate}[(1)]
                  \item 方法:延长剩余劳动时间
                  \item 本质:剥削,获取剩余价值
                  \item 绝对剩余价值是资本主义剥削的一般基础,是相对剩余价值的起点
              \end{enumerate}
          }
          \begin{enumerate}[(1)]
              \item 绝对剩余价值生产
                    \begin{enumerate}[a.]
                        \item 方法:生产技术不变
                              \begin{enumerate}[(a)]
                                  \item 延长工作日
                                  \item 提高劳动强度
                              \end{enumerate}
                        \item 结果
                              \begin{enumerate}[(a)]
                                  \item 绝对剩余价值:在必要劳动时间不变的条件下,通过延长工作日长度而生产的剩余价值
                              \end{enumerate}
                    \end{enumerate}
              \item 相对剩余价值生产
                    \begin{enumerate}[a.]
                        \item 方法:生产技术不断变革
                              \begin{enumerate}[(a)]
                                  \item 提高社会生产率
                                  \item 降低生活资料价值
                                  \item 降低劳动力的价值
                              \end{enumerate}
                        \item 结果:社会劳动生产率提高
                              \begin{enumerate}[(a)]
                                  \item 超额剩余价值:企业由于提高劳动生产率而使商品个别价值低于社会价值的差额(单个企业)
                                  \item 相对剩余价值:在工作日长度不变的条件下,通过缩短必要劳动时间来相应延长剩余劳动时间的剩余价值(行业)
                              \end{enumerate}
                    \end{enumerate}
          \end{enumerate}
\end{enumerate}










\chapter{劳动力}




\chapter{资本}



定义:不断在运动中谋求自身增殖的价值,是价值的一种特殊形式
\\

\informationBox{
    \begin{enumerate}[1.]
        \item 一般性:生产要素,市场经济的一种基本要素;
        \item 特殊性:不同社会经济制度下的资本反映着不同的社会生产关系
              \begin{enumerate}[(1)]
                  \item 增殖性:是区别于一般商品和货币的根本特征
                        \par * 货币流通公式与资本流通公式的区别
                        \begin{enumerate}[a.]
                            \item 流通形式:货币:$W1-G-W2$;资本:$G-W-G'$
                            \item 流通目的:货币:获得使用价值;资本:价值增殖
                            \item 流通限度:货币:有限的;资本:无限的($G—W—G'—W—G''……$)
                        \end{enumerate}
                  \item 运动性
                  \item 返还性:主观目的上
                  \item 风险性:要有前瞻性
              \end{enumerate}
    \end{enumerate}
}



\section{实体资本}

定义:能定期带来收入的,以实物或货币形式表现的资本
\\

\begin{enumerate}[1.]
    \item 货币资本形态
          \begin{enumerate}[(1)]
              \item 货币资本:是资本最一般的和初始的形态
          \end{enumerate}
    \item 实物资本形态
          \begin{enumerate}[(1)]
              \item 实物资本:以物质形态表现的资本,包括投入生产过程和流通过程的一切物的要素和待售的产出品,又称物质资本
              \item 分类:生产要素(生产资本)和待售产出品(商品资本)
          \end{enumerate}
    \item 无形资本形态
          \begin{enumerate}[(1)]
              \item 无形资产:以知识形态存在的特有经济资源
              \item 构成:专利权、商标权、版权、著作权、特许经营权、商誉、技术秘密等
          \end{enumerate}

\end{enumerate}



\section{虚拟资本}

定义:能定期带来收入,以有价证券形式表现的资本
\\

\begin{enumerate}[1.]
    \item 形式:
          \begin{enumerate}[(1)]
              \item 信用形式的虚拟资本
              \item 收入资本化形式的虚拟资本
                    \par * 现代分类
              \item 货币证券:银行券、银行票据(期票、本票等)
              \item 资本证券:股票、公司债券
          \end{enumerate}
    \item 特点:
          \begin{enumerate}[(1)]
              \item 经济性
                    \begin{enumerate}[a.]
                        \item 价值符号及它们的交换也是以劳动价值为基础的,没有价值及价值交换,谈不上它的经济性
                        \item 无论是纸币,还是股票等各种有价票证,它们的发行和流通基础就是价值和信誉,它们代表的是实体价值,是实体价值的代表,并且还可以为实体经济服务
                    \end{enumerate}
              \item 虚拟性
                    \begin{enumerate}[a.]
                        \item 交换物在形态上是虚拟的而非实物的,脱离了价值实体,成了实体价值的影子,是虚拟现实
                        \item 资金作为价值的表现,只有当它没有与实物商品进行交换,而只与它的同类即价值符号进行交换时。它才能被归为虚拟经济范畴
                    \end{enumerate}
          \end{enumerate}
    \item 产生条件:
          \begin{enumerate}[(1)]
              \item 资本虚拟化:指在当代发达市场经济体制之下,与实体资本相对应的虚拟资本数量不断膨胀,种类不断演化,并与个别实体资本逐渐脱离关系的过程和趋势。具体表现为各类金融市场的不断扩张,包括股票、债券、期货、期权以及其他金融衍生品市场等。
              \item 前提:货币的虚拟化
                    \begin{enumerate}[a.]
                        \item 货币摆脱贵金属束缚,不以有价值的实际资产作为货币材料
                        \item 在商业信用和银行信用不断发展的基础上,产生了代替金银及其铸币进行流通的信用流通工具,货币进一步虚拟化
                        \item 在高度发达的信用制度基础上发展的
                    \end{enumerate}
              \item 根据:借贷资本信用关系
                    \begin{enumerate}[a.]
                        \item 借贷资本是通过资本使用权的有偿转让,凭借债券来获得定期收入的
                        \item 利息是资本所有权的果实,借贷资本成为独立的收入来源
                    \end{enumerate}
              \item 基础:社会信用制度的逐步完善
                    \begin{enumerate}[a.]
                        \item 银行制度的发展,直接推动了货币的虚拟化
                        \item 货币的虚拟化也可以视为银行制度发展的重要动力,两者之间存在互动的关系
                        \item 在货币虚拟化基础上发展出来的资本虚拟化,同样离不开社会信用制度的发展和完善
                    \end{enumerate}

          \end{enumerate}
\end{enumerate}




\section{关系}



\begin{enumerate}[1.]
    \item 统一:
          \begin{enumerate}[(1)]
              \item 实体资本是虚拟资本的客观基础(虚拟资本的存在和运动必然要以它所表现的实体资本为基础)
                    \begin{enumerate}[a.]
                        \item 实体资本的运动状况决定虚拟资本的运行状况,证券发行者的生产经营状况决定着证券投资者的收益
                        \item 实体资本运用的规模影响着虚拟资本的发行规模,社会再生产规模决定着证券投资规模
                        \item 实体资本的循环周期影响着虚拟资本的周期波动
                    \end{enumerate}
              \item 虚拟资本对实体资本有制约作用(价值发现和风险化解)
                    \begin{enumerate}[a.]
                        \item 影响实体资本运用的过程和规模
                        \item 其流向影响着实体资本的分配比例和结构
                        \item 扩大了实体资本的活动范围
                    \end{enumerate}
          \end{enumerate}
    \item 对立:
          \begin{enumerate}[(1)]
              \item 虚拟资本价格不由实体资本价值决定,由预期收入和平均利率决定(不一定由经营状况决定)
              \item 虚拟资本价格变动相对独立,可能不随实体资本价值的变动而变动,而与其背离
          \end{enumerate}
\end{enumerate}


























\chapter{货币}




\begin{enumerate}[1.]
    \item 定义:固定充当一般等价物的特殊商品(money)
    \begin{enumerate}[(1)]
        \item 属性:
        \begin{enumerate}[a.]
            \item 名义价值(市场价值):该货币与其他货币的比值
            \item 实际价值(内在价值):货币材料(作为商品)的价值
        \end{enumerate}
        \item 货币与商品伴生,在商品交换过程中产生的,是商品交换发展到一定阶段的必然产物
        \item 具有一般商品的特性,又执行一般等价物的职能
    \end{enumerate}
    \item 规律:外在形式不断变化,货币本质、职能不变
    \begin{enumerate}[(1)]
        \item 静态:
        \begin{enumerate}[a.]
            \item 内在价值支撑信用
            \item 国家制度支撑主权货币信用
            \item 内在价值和主权信用支撑国际货币信用
        \end{enumerate}

        \item 动态:价值属性不断消失(价值和使用价值分离,弹性供给),支付属性不断增强(制度共识+价值共识)
        \begin{enumerate}[a.]
        \item 货币越来越摆脱具体的物质实体,成为单纯代表着按照一定单位对社会财富占有数量的数字
           \item 货币本身的价值基础越来越脱离作为价值实体的无差别人类劳动,越来越依赖机构与制度(中央银行制度)的作用来赋予货币所能代表的价值
        \end{enumerate}
    \end{enumerate}
   
\end{enumerate}





\section{货币分类}







\subsection{价值形式}

其发展出于降低交易成本、更方便的需要


\begin{enumerate}[1.]
    \item 简单的、个别的或偶然的价值形式
          \begin{enumerate}[(1)]
              \item 1只绵羊=2把斧子
              \begin{enumerate}[a.]
                  \item 1只绵羊:
                        \begin{enumerate}[(a)]
                            \item 相对价值形式(相对:相对于等价形式;劳动生产率变化,价值量也会变化)
                            \item 主动地位
                        \end{enumerate}
                  \item 2把斧子:
                        \begin{enumerate}[(a)]
                            \item 等价形式:使用价值成为价值的表现形式,具体劳动成为抽象劳动的表现形式,私人劳动成为社会劳动的表现形式
                            \item 被动地位
                        \end{enumerate}
              \end{enumerate}
              \item 交换行为、比例、对象等具有偶然性
          \end{enumerate}
    \item 总和的或扩大的价值形式
          \begin{enumerate}[(1)]
              \item 优点:等价物的数量扩大
              \item 缺点:没有形成统一等价物
              \item 价值第一次表现为无差别的劳动凝结
          \end{enumerate}
    \item 一般价值形式
          \begin{enumerate}[(1)]
              \item 优点:分离出公认的一般等价物(质的变化),产生物物交换的媒介
              \item 缺点:等价物商品太多,实物货币的不足
          \end{enumerate}
    \item 货币形式
          \begin{enumerate}[(1)]
              \item 优点:金属货币(其优点也是相对的)固定充当一般等价物,更加便利
              \item 最发达的价值形式,是商品的交换价值的最高形式
              \item 商品与商品的对立表现为商品与货币的对立
          \end{enumerate}
\end{enumerate}













\subsection{物质形式}


\begin{equation*}
    \text{物质形式}\mindmap{
        \text{实体货币}\mindmap{
            \text{商品货币}\mindmap{
                \text{实物货币} \\
            \text{金属货币}
            }
            \text{现钞}\mindmap{
                \text{纸质货币}
            }
        }
        \text{虚拟货币}\mindmap{
            \text{存款货币}\\
            \text{电子货币}\\
            \text{数字货币}
        }
    }
\end{equation*}


\subsubsection{实体货币}

定义:货币币材可触摸的有形货币(physical currency),又称为真实货币
\informationBox{
    * 物理属性
}



\begin{enumerate}[1.]
    \item 商品货币:有内在价值的货币(commodity money/currency)
          \begin{enumerate}[(1)]
              \item 实物货币
                    \begin{enumerate}[a.]
                        \item 定义:以自然界存在的某种物品(非贵金属)或人们生产的某种物品来充当货币(physical money)
                        \item 不足:不易分割、储存、运输
                    \end{enumerate}
              \item 金属货币
                    \begin{enumerate}[a.]
                        \item 定义:以金属如铜、银、金等作为材料的货币(metallic money)
                        \item 分类:称量货币,铸币(coins)
                        \item 优点:价值稳定;易于分割;易于贮藏
                    \end{enumerate}
          \end{enumerate}
    \item 现钞:包括纸币、硬币的法令货币(fiat money/currency),又称法偿货币(legal tender)
    \par 分类:可兑换的信用货币;不可兑换的信用货币
          \begin{enumerate}[(1)]
              \item 纸质货币
                    \begin{enumerate}[a.]
                        \item 定义:包括国家答应的纸质货币符号、商人发行的兑换券和银行发行的纸质信用货币等(paper money/currency),简称纸币
                              \begin{enumerate}[(a)]
                                  \item 纸币本身没有价值,仅是价值符号
                                  \item 纸币只有代表金量才有价值
                                  \item 纸币由国家通过法定途径发行,以国家信用做担保,一旦进入流通会受到流通规律的支配(脱离一定的表现材料,不再依赖生产货币的劳动,更多地依赖机构和制度)
                              \end{enumerate}
                        \item 分类:纸币、银行券(bank note)等
                        \item 原因:代替金属货币执行流通手段职能
                              \begin{enumerate}[(a)]
                                  \item 金银自然属性与充当价值符号的社会属性的矛盾
                                  \item 商品转化为作为流通手段的货币只是价值形态的变化,金属货币转手仍不方便
                                  \item 商品生产流通的扩大化要求货币量相应增加,蕴藏、开采、提炼跟不上
                                  \item 信用货币代替金属货币充当支付手段和流通手段的信用证券
                              \end{enumerate}
                    \end{enumerate}
          \end{enumerate}
\end{enumerate}



\subsubsection{虚拟货币}

定义:无实体的“货币”、在某些领域执行货币职能的价值代表物(virtual currency)

\informationBox{
    \par * 广义:央行和私人都能发行
    \par * 狭义:非官方发行
    \par * 分类(发行机构)
    \begin{enumerate}[(a)]
        \item 由游戏平台、网络社交平台发行的虚拟货币:不涉及金融机构,实际上被当作商品出售
        \item 基于银行账户相关联的记账式虚拟货币:电子货币,金融机构发行
        \item 没有发行机构的虚拟货币:比特币等
    \end{enumerate}
    \par * 分类(经济功能)
    \begin{enumerate}[(a)]
        \item 支付工具型:Q 币,法币预付充值
        \item 交易媒介型:魔兽金币,虚拟社区居民之间交易
        \item 促销工具型:折扣积分
        \item 激励合作型:论坛积分,提高网络资源共享程度
    \end{enumerate}
}






\begin{enumerate}[1.]
    \item 存款货币
\begin{enumerate}[(1)]
    \item 定义:能够发挥货币交易媒介和资产职能的银行存款(deposit money)
    \item 经济属性:后期的存款货币为电子货币
    \item 分类:可以直接进行转账支付的活期存款和企业定期存款、居民储蓄存款等
\end{enumerate}
\item 电子货币
    \begin{enumerate}[(1)]
        \item 定义:以金融数字化网络为基础,通过计算机网络系统,以传输电子信息的方式实现支付功能的电子数据(electronic money)
        \item 技术属性
        \item 分类:
        \begin{enumerate}[a.]
            \item 卡基电子货币:借记卡,贷记卡,储值卡等(card-based)
            \item 数基电子货币:网银,电子现金(soft-based)
        \end{enumerate}
        \item 电子货币是信用货币与虚拟货币的过渡阶段(市场形式的转变),是记账式虚拟货币,法币的电子化,传统金融机构负债的电子化
        \item 对经济的影响:现金先行约束(CIA)的放松 → 减少预防性的现金留存 → 改变货币结构 → 提高金融体系中的金融资源 → 促进投资,进而促进经济增长  
    \end{enumerate}
    \item 数字货币
    \begin{enumerate}[(1)]
        \item 定义:具备某些货币属性的数字化价值代表物(digital money/currency)
        \item 分类:
        \begin{enumerate}[a.]
            \item 央行发行的数字货币(中心化)
            \item 私人发行的数字货币(去中心化)
        \end{enumerate}
    \end{enumerate}
\end{enumerate}




\subsection{流动性}

\informationBox{
    \par 流动性强弱变化导致货币范围变化
    \par 国家金融制度越发达,金融产品越丰富,货币层次就越多
    \par 不同国家货币层次包含内容不同
    \par 金融产品创新和金融环境改变,需重新划分层次
    \par 层次划分只能在一定程度上反映货币流通状况
}

\subsubsection{IMF}

\begin{enumerate}[1.]
    \item 通货:中央银行或财政部发行流通于银行体系以外的现钞(currency,$M_0$)
    \item 可转让存款:存款性公司发行的活期存款、银行本票、旅行本票,邮政储蓄机构发行的可转让存款,非存款性公司发行的旅行支票
    \item 货币市场基金份额
    \item 债务凭证
    \item 存款公司发型的大额存单、商业票据等
\end{enumerate}

\subsubsection{我国}

\par $M_0$ =流通中的现金
\par $M_1$ (货币)=$M_0$+企业活期存款
\par $M_2$ = $M_1$+准货币(企业单位定期存款+城乡居民储蓄存款+证券公司的客户保证金存款+其他存款)
\par 狭义货币$M_1$反映整个社会对商品和劳务服务的直接购买力,广义货币$M_2$反映整个社会潜在的购买力


\informationBox{
其他:
\begin{enumerate}[1.]
    \item 资产负债
    \begin{enumerate}[(1)]
        \item 金属货币:持有者的资产,不是任何人的负债
        \item 纸质货币:持有者的资产,发行者的负债
        \item 可兑换的银行券:持有者的资产,发行者的负债
        \item 中央银行发行的现钞:持有者的资产,中央银行的负债
        \item 商业银行发行的存款货币:持有者的资产,商业银行的负债
        \item 虚拟货币(游戏公司或社交平台发行):持有者的资产,不是销售方的负债
        \item 虚拟货币(电子货币):持有者的资产,发行者的负债
        \item 虚拟货币(数字货币):持有者的资产,不是任何人的负债
    \end{enumerate}
    \item 货币价值
    \begin{enumerate}[(1)]
        \item 商品货币:名义价值与内在价值相一致
        \item 金属货币(银行券):名义价值远小于内在价值
        \item 信用货币(银行券):无内在价值
        \item 信用货币:无内在价值,所以必须垄断货币
    \end{enumerate}
    \item 发行数量
    \begin{enumerate}[(1)]
        \item 商品货币:发行量有限
        \item 信用货币:发行量无限(不考虑经济约束)
        \item 虚拟货币(游戏公司或社交平台发行):发行量无限
        \item 虚拟货币(数字货币):发行量固定
    \end{enumerate}
    \item 实体货币、虚拟货币
    \begin{enumerate}[(1)]
        \item 相互隔绝,不存在兑换关系:在虚拟社区或者游戏之内等封闭环境当中(单机)
        \item 实体货币向虚拟货币的单向兑换:与一般产品和服务的出售没有本质差异
        \item 双向兑换(范围狭小,局部特定人群):兑换的价格或随行就市或相对固定。游戏币等,可以视为可转手的产品或服务的预付款,与一般产品和服务的出售没有本质区别
        \item 双向兑换(较大范围被接受):
        \begin{enumerate}[a.]
            \item 兑换比价波幅大:比特币。小范围内流通;投资品而不是货币
            \item 兑换比价固定:支付宝“零钱”
        \end{enumerate}
    \end{enumerate}
\end{enumerate}
}





\section{决定理论}



\subsection{货币供给}



定义:一定时期内一国银行系统向经济中投入或抽离货币的行为过程(money supply)




\subsubsection{传导机制}





\subsubsection{理论:货币乘数}


\begin{equation*}
    m=\frac{M_s}{B}=\frac{C+D}{C+R}=\frac{\frac{C}{D}+1}{\frac{C}{D}+\frac{R}{D}}=\frac{cr+1}{cr+rr}
\end{equation*}
\par $m$:货币供应量与基础货币的倍数关系
\par $M_s=C+D$:货币供给=通货+活期存款
\par $B=C+R$:基础货币=通货+准备金
\par $C$:公众以通货形式持有的货币(现金)
\par $R$:银行以准备金形式持有的货币(存款准备金)
\par $\frac{C}{D}$:通货-存款比率:流通中的现金与商业银行全部存款的比率(currency-deposit ratio)
\par $\frac{R}{D}$:准备-存款比率:商业银行法定准备金和超额准备金的总和占全部存款的比重(reserve-deposit ratio)
  













\subsection{货币需求}


\subsection{货币均衡}







\subsection{货币不均衡}




\subsubsection{内部不均衡}

\paragraph{通货膨胀}



定义:由于货币供给过多而引起货币贬值、物价普遍持续上涨的货币现象(inflation)
\informationBox{
  \begin{enumerate}[(1)]
    \item 本质:货币现象
    \item 以货币流通为前提,通过物价上涨表现
    \item 物价总水平(平均物价),持续上涨(一段时间)
    \item 贵金属条件下和纸币条件下都可能发生,但纸币条件下更多、更容易发生,贵金属情况(①削减铸币尺寸或添加贱金属②贵金属短期迅速增加)
    \item 通货膨胀税:货币供给的增加引起通货膨胀,发行货币筹集的收入相当于一种通货膨胀税(inflation tax) 
  \end{enumerate}
}


\subparagraph{分类}	

\begin{enumerate}[1.]
  \item 通货膨胀的表现
        \begin{enumerate}[(1)]
          \item 公开型通货膨胀/显性通货膨胀(市场经济)
          \item 隐蔽型通货膨胀(计划经济)
        \end{enumerate}
  \item 价格上涨程度
        \begin{enumerate}[(1)]
          \item 爬行通货膨胀:年1\%~3\%
          \item 温和通货膨胀:年3\%~10\%
          \item 恶性通货膨胀:年10\%
        \end{enumerate}
  \item 通货膨胀预期
        \begin{enumerate}[(1)]
          \item 预期通货膨胀
          \item 非预期通货膨胀
        \end{enumerate}
  \item 通胀的原因
        \begin{enumerate}[(1)]
          \item 需求拉动型通货膨胀
          \item 成本推进型通货膨胀
          \item 供求混合型通货膨胀
          \item 结构失调型通货膨胀
          \item 体制型通货膨胀
        \end{enumerate}
\end{enumerate}


\subparagraph{指标}


\begin{enumerate}[1.]
  \item 消费者价格指数
        \begin{enumerate}[(1)]
          \item 定义:综合反映一定时期内购买并用于消费的消费品及服务价格水平的变动情况的指标(consumer price index,CPI)
          \item 拉式指数(Lasperyres index):用一篮子固定产品计算的价格指数(产品固定,价格变动)
          \item 反映的是租房的价格,不能等同于房地产价格
          \item 不足
                \begin{enumerate}[a.]
                  \item 倾向于夸大通货膨胀。
                  \item 替代偏差:未考虑消费者使用替代产品(如新产品的出现)
                  \item 质量变化无法衡量
                \end{enumerate}
        \end{enumerate}
  \item 生产者价格指数
        \begin{enumerate}[(1)]
          \item 定义:衡量工业企业产品出厂价格变动趋势和变动程度的指数,是反映某一时期生产领域价格变动情况的重要经济指标(PPI)
        \end{enumerate}
  \item 批发物价指数
        \begin{enumerate}[(1)]
          \item 定义:反应大宗生产资料和消费资料批发价格变动程度和趋势的指标(WPI)
        \end{enumerate}
  \item GDP平减指数(GDP deflator)
        \begin{gather*}
          GDP\text{平减指数}=\frac{\text{名义}GDP}{\text{实际}GDP}
        \end{gather*}
        \begin{enumerate}[(1)]
          \item 名义GDP:现期价格衡量的产品与服务的价值(nominal GDP )
          \item 实际GDP:基年不变价格衡量的产品与服务的价值(real GDP)
          \item 帕氏指数(Paasche index):用一篮子可变产品计算的指数(固定价格,数量变化)
          \item 优点:涉及全部商品和服务,更准确全面
          \item 不足:易受价格结构影响
        \end{enumerate}
\end{enumerate}














\paragraph{通货紧缩}





定义:由于货币供给不足而引起货币升值,物价普遍持续下跌的货币现象(deflation)

\informationBox{
  * 标准:
  \begin{enumerate}[(1)]
    \item 物价持续下降
    \item 信贷和货币供应量下降
    \item 伴随经济衰退
  \end{enumerate}
}

\subparagraph{指标}

\begin{enumerate}[1.]
  \item CPI
  \item WPI
  \item GDP平减指数
  \item 经济增长率
  \item 失业率
\end{enumerate}



\subparagraph{影响}

\begin{enumerate}[1.]
  \item 经济衰退,失业增加
        \begin{enumerate}[(1)]
          \item 投资:实际利率↑→实际成本↑→预期收益↓→投资↓
          \item 企业:利润↓→股价↓→市值缩水→筹资困难→工资↓+雇员↓→消费↓→加剧衰退
        \end{enumerate}
  \item 投资和消费需求不足
        \begin{enumerate}[(1)]
          \item 投资:市值缩水,筹资困难
          \item 消费:
                \begin{enumerate}[a.]
                  \item 价格效应:物价下跌→储蓄↑+消费↓
                  \item 收入效应:经济衰退→收入↓→消费↓
                \end{enumerate}
        \end{enumerate}
  \item 破坏信用关系
        \begin{enumerate}[(1)]
          \item 不良资产率↑,银行惜贷
          \item 新的信用需求↓
        \end{enumerate}
\end{enumerate}








\subsubsection{外部不均衡}




\section{作用与影响}

\informationBox{
    \begin{enumerate}[1.]
        \item 货币中性观
              \begin{enumerate}[(1)]
                  \item 认为货币对经济运行没有实质性影响
                  \item 政策意义:货币供给应与潜在经济增长相适应
              \end{enumerate}
        \item 货币非中性观
              \begin{enumerate}[(1)]
                  \item 认为货币对经济运行能够产生实质性影响
                  \item 政策意义:货币供应应逆周期动态调整
              \end{enumerate}
        \item 共识
              \begin{enumerate}[(1)]
                  \item 短期是非中性的,长期是中性的
                  \item 长短界限依赖于工资物价调整的速度(工资、物价黏性)以及信息传递的速度(信息黏性)
              \end{enumerate}
    \end{enumerate}
}


\subsection{货币职能}


\subsubsection{计价标准}

定义:用货币去计算并衡量商品或劳务的价值,从而为商品和劳务的交换标价(standard of value),又称计价单位(unit of account)

\informationBox{
    \begin{enumerate}[(1)]
        \item 价值尺度:货币表现商品的价值(质)、衡量商品价值量的大小(量)的尺度
        \item 商品流通中自发产生的
        \item 观念上的货币
    \end{enumerate}  
}



\subsubsection{交易媒介}

定义:货币在商品交易中作为交换手段、计价标准和支付手段,从而提高交易效率,降低交易成本,便利商品交换的职能(media of exchange)

\informationBox{
    \par 货币独有的、最基本的职能
    \par 交换手段和支付手段决定了货币的交易性需求,与预防性需求和投机性需求一起构成货币总需求
}


\begin{enumerate}[1.]
    \item 交换手段
          \begin{enumerate}[(1)]
              \item 定义:货币在商品交换中作为中介,通过一手交钱一手交货作为商品流通的媒介(means of exchange),又称流通手段
              \item 需要现实的货币
              \item 优点:
                    \begin{enumerate}[a.]
                        \item 打破时空界限,加速流通→形成商品生产者全面的社会联系
                        \item 分离出公认的一般等价物
                    \end{enumerate}
              \item 缺点:
                    \begin{enumerate}[a.]
                        \item  买卖脱节,蕴含流通危机
                    \end{enumerate}
              \item 商品流通:以货币为媒介的商品交换
          \end{enumerate}
    \item 支付手段
          \begin{enumerate}[(1)]
              \item 定义:货币作为延期支付的手段来结清债权债务关系(means of payment)
              \item 形式:清偿债务、支付赋税、工资、租金、利息、捐款、赔偿等
              \item 原因:赊购赊销
              \item 优点:促进商品经济发展
              \item 缺点:潜伏支付危机
              \item 货币支付手段职能是信用货币产生的基础
          \end{enumerate}
\end{enumerate}


\subsubsection{资产职能}

定义:货币可以作为人们总资产的一种存在形式,成为实现资产保值增值的一种手段,又称价值储藏手段(store of value)、贮藏手段

\begin{enumerate}[1.]
    \item 资产职能决定了货币的投机性需求
    \item 货币与其他金融工具的关系
    \begin{enumerate}[(1)]
        \item 联系:货币是一种金融工具
        \item 区别:货币收益性最低,流动性最高
    \end{enumerate}
    \item 作用:“蓄水池”
    \begin{enumerate}[(1)]
        \item 社会财富的一般代表
        \item 自发调节流通中的货币量
    \end{enumerate}
    \item 充当贮藏手段的货币必须是足值货币
    \item 现实中纸币可以以储蓄的形式贮藏,但不同于金属货币的贮藏,纸币最终还需要进入流通领域,而贮藏金属货币是退出流通领域,所以纸币没有贮藏手段的功能 
\end{enumerate}


\subsubsection{世界货币}

定义:货币在世界市场充当一般等价物的职能

\begin{enumerate}[1.]
    \item 原因:跨国贸易的需要
    \item 职能
    \begin{enumerate}[(1)]
        \item 支付手段:平衡贸易差额
        \item 购买手段:单方面向外国购买商品
        \item 贮藏:作为社会财富的代表在国家间转移
    \end{enumerate}
\end{enumerate}



\subsection{经济影响}

\begin{enumerate}[1.]
    \item 积极:
          \begin{enumerate}[(1)]
              \item 克服物物交换困难,提高交换效率,商品流通(交换手段)
              \item 便于价值衡量和交换比率确定(计价标准)
              \item 通过支付充抵部分交易金额,节约流通费用(支付手段)
              \item 流动性高,丰富了贮藏手段和投资形式(资产职能)
              \item 支付手段和存款等促进社会资金集中,便于社会化大生产(支付手段、资产职能)
          \end{enumerate}
    \item 消极:
          \begin{enumerate}[(1)]
              \item 交换过程:买卖分离,易发生商品买卖脱节和供求失衡
              \item 支付手段:债务链条复杂,易产生债务危机
              \item 跨期支付:财政超分配和信用膨胀,易造成通货膨胀
          \end{enumerate}
\end{enumerate}



\subsection{其他影响}

\begin{enumerate}[1.]
    \item 积极:
          \begin{enumerate}[(1)]
              \item 扩大了人类活动范围
              \item 激发人类想象力和创造力
              \item 激发了人们创造财富的欲望
          \end{enumerate}
    \item 消极:
          \begin{enumerate}[(1)]
              \item 货币拜物教扭曲人类的思想与行为
          \end{enumerate}
\end{enumerate}



\subsection{影响条件}

\begin{enumerate}[(1)]
    \item 携带方便
    \item 贮藏安全:身边→专门机构→云端;实体→电子(加密)
    \item 易于与其他形式资产转换,可分割性
    \item 币值稳定
    \item 货币流通量的调节机制,供给弹性
\end{enumerate}



\section{货币制度}



定义:针对货币的有关要素、货币流通的组织与管理等内容以国家法律的形式或国际协议形式加以规定所形成的制度(Monetary System)


\begin{equation*}
    \text{货币制度}\mindmap{
        \text{国家货币制度}\mindmap{
            \text{货币本位体系}\mindmap{
                \text{金属货币制度}\mindmap{
                    \text{复本位制}\mindmap{
                        \text{金银复本位制}
                    }
                    \text{单本位制}\mindmap{
                            \text{银本位制} \\
                            \text{金本位制}
                        }
                    }
                    \text{信用货币制度}\mindmap{
                        \text{不兑现的信用货币制度}
                    }
                }
                \text{货币支付体系}
            }
            \text{区域货币制度}\\
            \text{国际货币制度}\mindmap{
                \text{国际金本位制}   \\
                \text{布雷顿森林体系} \\
                \text{牙买加体系}
            }
        }
\end{equation*}








\subsection{国家货币制度}
定义:国家以法律形式确定的货币流通的结构和组织形式(National Monetary System)

\informationBox{
    * 货币主权的体现
    * 规定:
    \begin{enumerate}[(1)]
        \item 货币材料
        \item 货币单位:货币计量单位(money unit)
        \begin{enumerate}[a.]
            \item 规定货币单位名称
            \item 规定货币单位的值
        \end{enumerate}
        \item 货币种类
        \begin{enumerate}[a.]
            \item 主币:本位币,是一个国家流通中的基本通货(standard money)
            \item 辅币:本位货币单位以下的小面额货币(fractional money)
        \end{enumerate}
        \item 法定支付能力
        \begin{enumerate}[a.]
            \item 无限法偿:不论支付数额多大,不论属于何种性质的支付,对方都不能拒绝接受
            \item 有限法偿:在一次支付中若超过规定的数额,收款人有权拒绝接收,但在法定限额内不能拒收
        \end{enumerate}
        \item 铸造发行流通程序
        \begin{enumerate}[a.]
            \item 自由铸造与限制铸造
            \item 分散发行与垄断发行
        \end{enumerate}
        \item 货币发行准备制度
        \begin{enumerate}[a.]
            \item 现金准备:黄金、外汇等
            \item 证券准备:短期商业票据、财政短期国库券、政府公债券等
        \end{enumerate}
    \end{enumerate}
}


\subsubsection{货币本位体系}




\paragraph{金属货币制度}


\begin{enumerate}[1.]
    \item 复本位制:金银复本位制(Gold and Silver Bimetallism Standard )
          \begin{enumerate}[(1)]
              \item 格雷欣法则(Gresham's Law):劣币驱逐良币。两种实际价值不同而法定价格相同的货币同时流通时,市场价格偏高的货币(良币)就会被市场价格偏低的货币(劣币)所排斥,在价值规律的作用下,良币退出流通进入贮藏,而劣币充斥市场
          \end{enumerate}
    \item 单本位制:银本位制(Silver Standard)
    \item 单本位制:金本位制(Gold Standard)
          \begin{enumerate}[(1)]
              \item 黄金输送点:金本位制下,两种货币之间的汇率上下波动的界限(输出输入黄金,要支付包装费、运费、保险费、检验费等费用以及利息)
              \item 金平价:各国本位货币所含有的黄金纯量的比,也就是金本位制度下的各国货币的交换比率,又称法定平价、铸造平价(mint parity)
                    \begin{equation*}
                        CA(+) \rightarrow \text{资本流入}\Rightarrow  M_{S}\uparrow \Rightarrow 
                        \begin{Bmatrix}
                            P\uparrow \Rightarrow 
                            \begin{cases}
                                EX\downarrow \\
                                PY\uparrow \Rightarrow IM↑
                            \end{cases} \\
                            i\downarrow \Rightarrow \text{资本外流}
                        \end{Bmatrix}
                        \Rightarrow CA(balance)
                    \end{equation*}
              \item 分类
                    \begin{enumerate}[a.]
                        \item 金币本位制
                        \item 金块本位制:不铸造、不流通金币,银行券只能达到一定数量后才能兑换金块的货币制度(Gold Bullion Standard),又称生金本位制
                        \item 金汇兑本位制:国内不铸造、不使用金币,而是流通银币或银行券,但他们不能在国内兑换黄金,只能兑换本国在该国存有黄金并与其货币保持固定比价国家的外汇,然后用外汇到该国兑换黄金(Gold Exchange Standard),又称虚金本位制。多为殖民地、半殖民地国家
                    \end{enumerate}
          \end{enumerate}
\end{enumerate}

\paragraph{信用货币制度}

信用货币:通过信贷机制向实体经济投放的货币(经济属性)
\\

\begin{enumerate}[1.]
    \item 不兑现信用货币制度
          \begin{enumerate}[(1)]
              \item 特点
                    \begin{enumerate}[a.]
                        \item 现实经济中的货币都是信用货币,主要由现金和银行存款构成
                        \item 现实中的货币都是通过金融机构的业务投入到流通中去的
                        \item 国家对货币的管理调控成为经济正常发展的必要条件
                    \end{enumerate}
          \end{enumerate}
\end{enumerate}



\subsubsection{货币支付体系}

定义:在既定的法规制度框架下,由提供支付服务的组织或机构通过支付工具的应用实现债权债务清偿以及资金转移的一种综合金融安排


\begin{enumerate}[1.]
    \item 支付:付款人对收款人进行的当事人可接受的货币债权转让
    \begin{enumerate}[(1)]
        \item 过程:交易+清算(信息汇总:行内+跨行)+结算(资金划拨)
        \item 原则:安全性;便捷性;高效率;低成本
    \end{enumerate}
    \item 阶段
    \begin{enumerate}[(1)]
        \item 实物货币:实物支付
        \item 信用货币:信用支付
        \item 电子货币:电子支付
    \end{enumerate}
    \item 构成
    \begin{enumerate}[(1)]
        \item 支付服务组织
        \begin{enumerate}[a.]
            \item 中央银行(核心)
            \begin{enumerate}[(a)]
                \item 支付体系的组织者、规划者、监管者
                \item 跨行支付清算服务的提供者
                \item 举足轻重的“信心效应”
            \end{enumerate}
            \item 以商业银行为主体的金融机构(骨干力量)
            \item 清算机构(重要组成部分):专业资金清算服务:票据,证券
            \item 行业协会:在中央银行的授权或支持下,负责支付清算行业的自律性管理,并负责部分支付清算系统的运行
        \end{enumerate}
        \item 支付系统
        \begin{enumerate}[a.]
            \item 定义:支持各种支付工具应用、实现资金清算并完成资金转移的支付、结算和清算系统,是金融体系的重要基础设施
        \end{enumerate}
        \item 支付工具
        \begin{enumerate}[a.]
            \item 信用卡支付结算:指使用信用卡为载体进行的支付结算
            \item 资金汇兑:指汇款客户委托银行将其款项支付给收款人的结算方式(通常为企业间的汇款),又称银行汇款。可分为信汇(邮寄)和电汇(电报)。
            \item 支票支付结算:指以纸质支票为媒介的支付结算,本质上就是银行提供的一种特殊纸质的基于特殊格式与使用规则的支付结算工具
            \item 自动清算所支付:由成员存款机构达成的成员机构间以电子借记或贷记方式进行支付的一种安排
            \item 电子资金转账:银行在电子计算机系统中,让资金以电子信息形式在账户间转移
            \item 移动支付:不同的场景也造就了支付宝和财付通在不同场景下的市场份额
        \end{enumerate}
        \item 法规和制度
    \end{enumerate}
\end{enumerate}



\subsection{区域货币制度}

定义:由某个区域内的有关国家(地区)通过协调形成一个货币区,由联合组建的一家中央银行来发行和管理区域内的统一货币的制度

\informationBox{
    欧洲货币联盟制度、西非货币联盟制度、中非货币联盟制度、东加勒比货币联盟制度等
}


\subsubsection{最适度通货区理论}


\begin{enumerate}[1.]
    \item 传统
          \begin{enumerate}[(1)]
              \item 要素流动论(蒙代尔):相同通货区的需求转移:要素流动不同通货区的需求转移:浮动汇率解决。如果劳动力和资本在区域内能够自由流动,则组成单一货币区就可以提高微观效率 (如消除交易成本),有利于抵抗外部冲击,维护宏观经济的稳定
              \item 金融开放论(麦金农):经济开放度 TG/GDP,在外部世界价格同样稳定的前提下,那些贸易关系密切的经济开放区应组成一个共同的货币区,从而有利于实现内外部经济均衡、价格的稳定
              \item 产品多样性:低程度产品多样性的国家不适合固定汇率,适宜组成货币区,区域内固定汇率,区域外浮动汇率
              \item 国际金融一体化程度
              \item 政策一体化程度
              \item 通货膨胀相似度
          \end{enumerate}
    \item 现代:收益成本分析
          \begin{enumerate}[(1)]
              \item 收益:节省货币兑换成本及抵补外汇风险的费用、消除投机性资本流动、节省外汇储备、分散风险,节省国际收支调价成本,提高资源的配置效率、货币一体化可能加速财政一体化
              \item 成本:放弃货币政策,放弃其运用汇率工具和货币政策实现稳定国内产出和就业目标的一部分自主权,这种因固定汇率安排而产生的不稳定性被称为经济稳定性损失(economic stability loss)一体化程度越高,经济不稳定性损失越低
          \end{enumerate}
\end{enumerate}


\subsection{国际货币制度}

定义:支配各国货币关系的规则以及各国进行各种交易支付所依据的一套安排和惯例(International Monetary System)

\informationBox{
    * 主要内容:
    \begin{enumerate}[(1)]
        \item 确定国际储备资产(R)
        \item 安排汇率制度(汇率决定)
        \item 选择国际收支的调节方式(BOP+coordination agency调节机构)
    \end{enumerate}
}


\subsubsection{国际金本位体系}


\begin{enumerate}[1.]
    \item 内容
          \begin{enumerate}[(1)]
              \item 储备货币:以黄金为基础,利于黄金拥有量多的发达国家
              \item 固定汇率制:汇率自动稳定,政府不干预
          \end{enumerate}
\end{enumerate}



\subsubsection{布雷顿森林体系(Bretton Woods System)}

\begin{enumerate}[1.]
    \item 内容
          \begin{enumerate}[(1)]
              \item 国际储备货币:以黄金作为基础,以美元作为最主要的国际储备货币
              \item 固定汇率制:实行“双挂钩”,美元直接与黄金挂钩,其他货币与美元挂钩
              \item 货币调节:IMF确定法定升值(revaluation)或法定贬值(devaluation)
          \end{enumerate}
    \item 不足
          \begin{enumerate}[(1)]
              \item 内在原因
                    \begin{enumerate}[a.]
                        \item 内在缺陷:特里芬难题(Triffin dilemma):信心与清偿力的矛盾,双重身份的双重责任的矛盾。美元若要满足国际储备的需求就会造成美国国际收支逆差,必然影响美元信用,引起美元危机;若要保持美国的国际收支平衡,稳定美元,则又会断绝国际储备的来源,引起国际清偿能力的不足
                    \end{enumerate}
              \item 外在原因
                    \begin{enumerate}[a.]
                        \item 美国国际收支出现持续逆差
                        \item 西方各国经济发展不平衡加强,美国经济地位相对下降
                        \item 西方各国通货膨胀程度悬殊,固定汇率难以维持
                    \end{enumerate}
          \end{enumerate}
\end{enumerate}



\subsubsection{牙买加体系(Jamaica System)}

\begin{enumerate}[1.]
    \item 内容:
          \begin{enumerate}[(1)]
              \item 国际储备货币:多元化,SDR为主,信用货币为基础
              \item 汇率制度:汇率安排多样化,自主决定,浮动汇率为主,盯住汇率并存
              \item 国际收支调节:多种渠道:国内经济政策(影响国内供求);汇率政策(影响进出口);国际融资;加强国际协调;增减外汇储备
          \end{enumerate}
    \item 不足:
          \begin{enumerate}[(1)]
              \item 国际储备货币的发行国可以享受“铸币税”等好处,不愿承担稳定国际储备货币及其风险的责任
              \item 汇率波动,加大外汇风险,一定程度抑制国际贸易活动,刺激国际金融投机
              \item 国际收支调节机制不健全,全球性的国际收支平衡未得到根本改善
          \end{enumerate}
\end{enumerate}



\chapter{信用}







\begin{enumerate}[1.]
    \item 含义
          \begin{enumerate}[(1)]
              \item 定义:以还本付息为条件的借贷活动(credit)
              \item 赤字形成净债务,盈余形成净债权
              \item 政治经济学:特殊的价值运动形式。是与市场经济和货币流通紧密联系的经济范畴,是商品生产、货币流通、市场贸易发展到一定阶段的产物
              \item 实物借贷,货币借贷(主要)
          \end{enumerate}
    \item 产生
          \begin{enumerate}[(1)]
              \item 私有制是信用产生的基础
              \item 产权制度的建立完善为信用的良性发展奠定坚实基础
              \item 必要性:社会资源的不合理配置
                    \begin{enumerate}[a.]
                        \item 商品交易困难
                        \item 货币时空分布不均
                    \end{enumerate}
              \item 可能性:
                    \begin{enumerate}[a.]
                        \item 基础:商品经济的产生和发展
                        \item 前提:货币发挥支付手段职能
                    \end{enumerate}
          \end{enumerate}
    \item 发展
          \begin{enumerate}[(1)]
              \item 货币借贷拓展了信用范围,扩大了信用的规模
              \item 信用货币以信用为基础,是信用的产物
              \item 信用与货币结合形成金融
          \end{enumerate}
\end{enumerate}








\section{信用分类}








\subsection{传统信用}

\subsubsection{高利贷信用}

\begin{enumerate}[1.]
    \item 特点
    \begin{enumerate}[(1)]
        \item 利率极高  
        \item 利率不稳定且差异极大  
        \item 随意性大,借贷双方话语权不同
    \end{enumerate}
    \item 原因 
    \begin{enumerate}[(1)]
        \item 借贷资金的供求状况:供不应求  
        \item 贷者的垄断地位:资金供给高度分散而有限  
        \item 风险与成本的补偿:借者关乎生死存亡;贷者收回债务本息需强威慑力   
    \end{enumerate}
\end{enumerate}

\subsubsection{民间信用}

定义:游离于正规金融体系之外(比如银行借贷、证券市场融资)出于互助目标或为解决自身融资问题而进行的小规模金融活动,又称民间借贷、民间金融

\paragraph{原因}

\begin{enumerate}[1.]
    \item 供给
          \begin{enumerate}[(1)]
              \item 投资:民间信用的借款方很难从传统金融部门融得资金
                    \begin{enumerate}[a.]
                        \item 借款方缺乏抵押品或信用记录,银行等传统金融部门在控制自身风险的约束下会拒绝向其提供资金。
                        \item 交易成本:
                              \begin{enumerate}[(a)]
                                  \item 借款方融资金额较低导致单位资金融资交易成本过高,被传统金融部门信贷歧视
                                  \item 民间借贷面向亲友等小范围的借贷,合会等民间借贷模式相对成熟,因此在借贷过程中并不需要太高的交易成本
                              \end{enumerate}
                        \item 民间信用准入门槛低(不需要良好的信用记录,不需要提供抵押品),审批速度快
                    \end{enumerate}
          \end{enumerate}
    \item 需求
          \begin{enumerate}[(1)]
              \item 投资:为部分投资者提供了新的投资渠道
                    \begin{enumerate}[a.]
                        \item 需求:中国私人财富快速增长,需要寻求投资渠道;民间信用可以提高数量庞大的低财富值群体的福利
                        \item 供给:传统投资渠道收益率低(银行理财产品)或者不稳定(中国股市风险相对较高)
                    \end{enumerate}
              \item 生产:传统信贷忽略了创造就业岗位能力强的大多数小微企业,不利于就业和经济增长,还影响社会稳定。民间信用则可以缓解此问题
              \item 消费:传统信贷忽略了边际消费倾向较高的群体,从而形成消费需求不足。民间信用为该群体提供服务,可以提振消费
          \end{enumerate}
\end{enumerate}










\paragraph{形式:合会}

定义:两两之间的借贷协议,每一个会员都有义务向得会先于他的会员贷出资金,也有权利向得会后于他的会员借入资金,全称轮转储蓄与贷款协会(Rotating Savings and Credit Association,ROSCA)  
\begin{enumerate}[1.]
    \item 完全基于信用,不依赖抵押品或担保  
    \item 规则:每次举会由会员缴纳一定数量的金额(称为“会金”)轮流交一人使用。会首会得到第一笔会金,以后依不同方式决定会脚得到会金的次序。每个合会成员都有一次获得会金的机会  
    \begin{enumerate}[(1)]
        \item 会首相当于首先得到一笔贷款然后在每次举会时分期偿还  
        \item 最后一轮得会的会脚等价于参加了零存整取的储蓄  
        \item 其他的会脚则相当于先参加一个零存整取的储蓄,然后获得一笔贷款后分期偿还
    \end{enumerate}
\end{enumerate}


\paragraph{影响}

\begin{enumerate}[1.]
    \item 不足
          \begin{enumerate}[(1)]
              \item 难以形成规模效应
                    \begin{enumerate}[a.]
                        \item 民间借贷主要在熟人之间进行交易,这使得借贷的对象范围太窄,融资和投资的数量极为有限。过度本土化。
                        \item 民间信用尽管能在一定程度上解决小微企业融资难融资贵的问题,但其与现实要求相距甚远。
                    \end{enumerate}
              \item 风险分散能力较弱
                    \begin{enumerate}[a.]
                        \item 每位投资者只能在少数借款者(例如参与合会的会员)之间进行分散投资
                        \item 每位借款者也只能在少数投资者融得资金
                    \end{enumerate}
              \item 利率定价不合理
                    \begin{enumerate}[a.]
                        \item 尽管民间信用会考虑信用风险因素,但基本上仍然由经验规则确定,而且在具体借贷过程中还可能引入一些非理性因素。
                        \item 基于熟人借贷网络,市场分割,不存在统一的市场化定价
                    \end{enumerate}
              \item 内在不稳定性:可能存在风险传染。
                    \begin{enumerate}[a.]
                        \item 不同借贷网络之间的利率差异导致网络之间套利的存在,当此种套利行为比较普遍时,相对分割的各个民间信用网络就会被联系起来,进而会引发风险传染渠道。
                        \item 普遍的套利行为会催生民间信贷规模膨胀、民间金融潜在风险的累积。当潜在风险累积到一定程度并导致风险的释放
                        \item 风险的爆发→已经发放的民间借贷出现大量坏账→借贷双方信任减弱→新增的民间借贷急剧减少→民间金融市场出现信贷紧缩(credit crunch)→对实体经济产生负向的外部性。
                              \par 当民间借贷市场产生系统性的风险传染时,一些同时参与银行借贷与民间借贷的投资者会将此种风险传染至银行体系。乃至实体经济
                    \end{enumerate}
          \end{enumerate}
\end{enumerate}




\subsection{现代信用}

\subsubsection{商业信用}


\begin{enumerate}[1.]
    \item 定义:工商企业之间买卖商品时,卖方以商品形式向买方提供的信用(commercial credit)
          \begin{enumerate}[(1)]
              \item 政治经济学:以赊账方式出售商品(或提供劳务)时买卖双方之间相互提供的信用
              \item 是银行信用乃至信用体系的基础,直接信用
          \end{enumerate}
    \item 特点:
          \begin{enumerate}[(1)]
              \item 信用对象是商业资本
              \item 借贷行为和买卖行为相结合
              \item 政治经济学:债权人和债务人是职能资本家(主体)
              \item 商业信用的规模和发展程度直接依存于生产流通状况(周期性变化)
          \end{enumerate}
    \item 形式:赊销
          \begin{enumerate}[(1)]
              \item  商业票据:商业信用中被广泛使用的表明买卖双方债权债务关系的凭证,是商业信用中卖方为保证自己对买方拥有债务索取权而保有的书面凭证
                    \begin{enumerate}[a.]
                        \item 商业汇票
                        \item 商业本票
                    \end{enumerate}
              \item  背书:商业票据的债权人在转让票据时在其背面签字以承担连带责任的行为(endorsement)
          \end{enumerate}
    \item 影响:
          \begin{enumerate}[(1)]
              \item 优点
                    \begin{enumerate}[a.]
                        \item 促进生产和流通的顺畅进行
                    \end{enumerate}
              \item 不足
                    \begin{enumerate}[a.]
                        \item 规模:不能超出企业所售商品量,受单个企业资金规模的限制
                        \item 方向:卖方提供给买方;上游企业提供给下游企业
                        \item 期限:期限短,受企业生产周转时间限制,难以提供较长期信用
                        \item 流通:受信贷双方了解程度和信任程度的局限,商业票据差异大,作为支付凭证时会受限
                    \end{enumerate}
          \end{enumerate}
\end{enumerate}







\subsubsection{银行信用}


\begin{enumerate}[1.]
    \item 定义:银行或其他金融机构以货币形态提供的信用(bank credit)
          \begin{enumerate}[(1)]
              \item 我国最主要的信用形式,属于间接融资
              \item 银行信用以商业信用为基础,通过票据服务
          \end{enumerate}
    \item 特点:
          \begin{enumerate}[(1)]
              \item 间接信用,银行作为中介,吸收贷出闲置资本,资金来源于各部门闲散资金
              \item 信用对象是货币资本
              \item 存贷款数量和期限具有灵活性,可满足多样化需求
                    \par 扩大信用的规模和范围←突破个别资本数量和周转的限制,不受商品流转方向限制
          \end{enumerate}
\end{enumerate}


\subsubsection{企业信用}

定义:资本所有者与企业作为资本的使用者之间建立起的直接融资关系

\subsubsection{国家信用}


\begin{enumerate}[1.]
    \item 定义:政府作为债权人或者债务人的信用活动(state credit),又称政府信用/公共信用
          \begin{enumerate}[(1)]
              \item 国家信用是财政政策的重要工具,通常免税
          \end{enumerate}
    \item 形式:
          \begin{enumerate}[(1)]
              \item 中央政府债券:一国中央政府为弥补财政赤字或筹措建设资金而发行的债券(national debt),又称国债
                    \par 中长期国债一般有附有固定的息票(coupon),每半年付息一次
                    \begin{enumerate}[a.]
                        \item 短期国债:国库券:期限在1年或者1年以下的国债(treasury bill)
                        \item 中期国债:国库票据:期限在10年或者10年以下的国债(treasury note)
                        \item 长期国债:国库债券:期限在10年以上的国债(treasury bond)
                    \end{enumerate}
              \item 地方政府债券:(local government bonds),又称市政债券(municipal bond)
                    \begin{enumerate}[a.]
                        \item 一般义务债券:以地方政府的税收、行政收费等各项收益为偿还来源,又称一般责任债券(general obligation securities/bond)
                        \item 收益债券:以某一特定工程或某一特定业务的收入为偿还来源的债券,又称收入债券(revenue securities/bond)
                        \item 产业发展债券:为商业企业筹集资金的债券(industrial development bond)
                    \end{enumerate}
              \item 政府担保债券:政府作为担保人而由其他主体发行的债券(government guaranteed bonds)
                    \par 主体通常是政府所属的企业或者与政府相关的部门,等级仅次于中央政府债券
          \end{enumerate}
\end{enumerate}














\subsubsection{消费信用}


\begin{enumerate}[1.]
    \item 定义:工商企业、银行和其他金融机构提供给消费者用于消费支出的一种信用形式(consumer credit)
    \item 形式:
          \begin{enumerate}[(1)]
              \item 赊销:工商企业对消费者提供的短期信用(短期信用)
              \item 分期付款:消费者购买消费品或享受相关服务时,只需支付一部分货款,然后按合同条款分期支付其余货款的本金和利息(中长期信用)
              \item 消费贷款:银行及其他金融机构采用信用放贷或抵押放款方式对消费者发放的贷款(长期信用)
                    \begin{enumerate}[a.]
                        \item 买方信贷:对购买消费品的消费者直接发放的贷款
                        \item 卖方信贷:以分期付款单作抵押,对销售消费品的工商企业发放贷款,或者由银行与以信用方式出售消费品的企业签订合同将贷款直接支付给企业,再由购买商品的消费者逐步向银行还款
                    \end{enumerate}
          \end{enumerate}
    \item 影响:
          \begin{enumerate}[(1)]
              \item 积极:
                    \begin{enumerate}[a.]
                        \item 宏观经济调节(宏观)
                              \begin{enumerate}[(a)]
                                  \item 总量:通过调整消费信用的规模和投向,能够在一定程度上调节消费需求的总量和结构,有利于市场供求在总量和结构上的平衡
                                  \item 结构:调节某些领域和部门的结构,从而促进或限制某些领域或经济部门发展
                              \end{enumerate}
                        \item 生命周期内的财务安排(微观)
                              \begin{enumerate}[(a)]
                                  \item 提供将未来的预期收入用于当前消费的有效途径
                              \end{enumerate}
                    \end{enumerate}
              \item 消极:
                    \begin{enumerate}[a.]
                        \item 过度的消费信用倾向于掩盖消费品的供求矛盾,产生虚假需求,向生产者传递错误信息,导致某些消费品盲目发展,严重时可能导致产能过剩和产品的大量积压
                        \item 过量发展易导致信用膨胀,导致通货膨胀的压力
                        \item 诱导借款人误判未来预期收入,加重借款人的债务负担,降低生活水平,增加社会不稳定因素
                    \end{enumerate}
          \end{enumerate}
\end{enumerate}



\subsubsection{国际信用}


\begin{enumerate}[1.]
    \item 定义:一切跨国的借贷关系和借贷活动(international credit)
    \item 形式:
          \begin{enumerate}[(1)]
              \item 国外借贷:一国与该国之外的经济主体之间进行的借贷活动(foreign loan)
                    \begin{enumerate}[a.]
                        \item 出口信贷:(export credit)
                        \item 国际商业银行贷款:(international commercial bank loan)
                        \item 外国政府贷款:(foreign government loan)
                        \item 国际金融机构贷款:(international financial institution loan)
                        \item 国际资本市场融资:(financing at international capital markets)
                        \item 国际融资租赁:(international financial leasing)
                    \end{enumerate}
              \item 国际直接投资:一国居民、企业等直接对另一个国家的企业进行生产性投资,并由此获得对投资企业的管理与控制权(international direct investment)
          \end{enumerate}
\end{enumerate}


\section{作用与影响}

\begin{enumerate}[1.]
    \item 积极
    \begin{enumerate}[(1)]
        \item (信用拓宽了企业融资的渠道)加速资本的积聚和集中,扩大投资规模和企业规模  
        \item (企业之间的资金相互拆借、发行企业债券、购买股票)可以增加投资机会,促进资本在行业和地区之间的自由转移,优化社会资源的配置  
        \item (汇票、本票、银等信用形式使很多交易采用)非现金结算,节省流通资金,加快商品流通速度  
        \item (股票、储蓄、债券等)为居民提供多样化的投资渠道,(消费信贷等)可以改变居民的消费水平,形成合理消费结构  
        \item (国家信用,如政府债券等)可以调节和改变资金流向,促进国民经济健康发展   
    \end{enumerate}
    \item 消极
    \begin{enumerate}[(1)]
        \item (商业信用、银行信用等)过度发展有可能造成虚假繁荣,触发生产过剩的危机  
        \item (债务规模过大,不能按期偿还可能会)引发货币信用危机  
        \item 刺激投机(忽视实体支撑),加剧经济风险和经济危机
    \end{enumerate}
\end{enumerate}





\section{信用制度}


现代信用体系






\begin{enumerate}[1.]
    \item 基础性保障
          \begin{enumerate}[(1)]
              \item 道德规范仍然是信用体系构建的重要基础
              \item 高效快捷社会征信系统,是防止同一主体多次出现失信行为的利器
              \item 法律规范对失信行为的严厉制裁,是完备信用体系的终极制度保障
          \end{enumerate}
    \item 内容
          \begin{enumerate}[(1)]
              \item 信用制度:规范和约束社会信用活动和信用关系的行为规则
                    \begin{enumerate}[a.]
                        \item 信用制度是保证信用活动正常进行的基本条件
                        \item 以法律为主题的、完善的信用制度是信用活动健康发展的重要基石
                    \end{enumerate}
              \item 信用机构体系:
                    \begin{enumerate}[a.]
                        \item 信用中介机构:其主体是金融机构,包括利用互联网和金融科技发展成果的新型互联网金融平台
                        \item 信用服务机构:如信息咨询公司、投资咨询公司、征信公司、信用评估机构等专业信用服务机构,还有律师事务所、会计师事务所等非专业信用服务机构
                        \item 信用管理机构:分政府设立的监管机构和行业自律型管理机构,前者如中国人民银行、银保监会、证监会,后者如银行业协会、证券业协会和保险业协会
                    \end{enumerate}
              \item 社会征信系统:
                    \begin{enumerate}[a.]
                        \item 信用档案系统:包括个人信用档案系统和企业信用档案系统
                        \item 信用调查系统
                        \item 信用评估系统
                        \item 信息查询系统
                        \item 失信公示系统
                    \end{enumerate}
          \end{enumerate}
\end{enumerate}






























\chapter{利率}


利率:借贷期满的利息总额与贷出本金总额的比率

一、产生原因

\begin{enumerate}[(1)]
    \item 收益资本化:利息转化为收益的一般形态(capitalization of return)
          \begin{enumerate}[a.]
              \item 过程:利息本来以借贷为前提,源于产业利润的利息,逐渐被人们从借贷和生产活动中抽象出来,被赋予与借贷、生产活动无关的特性,逐渐被人们认为与资本的所有权联系,认为是资本所有权的必然产物
              \item 结果:各种有收益的事物都可以通过收益与利率的对比进行资本定价
          \end{enumerate}
    \item 货币的时间价值:同等金额的货币其现在的价值要大于其未来的价值(time value of money)
          \begin{enumerate}[a.]
              \item 机会成本:对货币的占用具有机会成本;对当前消费推迟的时间补偿
              \item 风险溢价:需要对通胀损失、投资风险进行补偿
          \end{enumerate}
\end{enumerate}


二、产生结果:利息


\begin{enumerate}[(1)]
    \item 定义:借贷关系中资金借入方支付给资金贷出方的报酬(interest)
          \begin{enumerate}[(a)]
              \item 是货币时间价值的具体体现
          \end{enumerate}
    \item 实质
          \begin{enumerate}[a.]
              \item 非货币因素
                    \begin{enumerate}[(a)]
                        \item 时差利息论:利息来源于同种和同量物品价值上的差别,这种差别由二者在时间上的差别完成;生产的费时性决定了现在物品和未来物品的差额,利息实质上来源于这种差额
                        \item 等待论:利息为纯息,利润为毛利息,利息是节欲和等待的报酬
                        \item 马克思:利息实质来源于劳动创造的价值,体现剥削或分配关系
                    \end{enumerate}
              \item 货币因素
                    \begin{enumerate}[(a)]
                        \item 流动性偏好利息论:利息是一定时期内放弃货币流动性的报酬
                    \end{enumerate}
              \item 现代
                    \begin{enumerate}[(a)]
                        \item 利息是投资者让渡资本使用权而索取的补偿或报酬。补偿包括对放弃投资于无风险资产机会成本的补偿和对风险的补偿。
                        \item 风险资产的收益率=无风险利率+风险溢价
                    \end{enumerate}
          \end{enumerate}
\end{enumerate}


三、计算
\informationBox{
\begin{enumerate}[(1)]
    \item 现金流贴现分析:计算现值的过程(discounted cash flow analysis)
          \par 贴现率:贴现(discount)时所使用的利率(discount rate),又称折现率
          \par 现金流难以预测,影响也相对短暂;折现率更为重要
    \item 收益率(yield)、回报率(returns)本质都是利率
          \par $$RET=\frac{C+P_{t+1}-P_t}{P_t}=\frac{C}{P_t}+\frac{P_{t+1}-P_t}{P_t}$$
          \par $C$ :年利息
          \par $\frac{C}{P_t}$ :当期收益率:每年的利息收入与证券购买价格的比率(股息收益率)
          \par $\frac{P_{t+1}-P_t}{P_t}$ :资本利得(损失)率:证券价格变动相对于购进价格的比率
          \par 复合收益率:考虑利息再投资的收益率
\end{enumerate}
}



\begin{enumerate}[(1)]
    \item 单利(simple interest)
          \begin{gather*}
              APR=\frac{I}{PV}·\frac{1}{n} =\frac{FV-PV}{PV}·\frac{1}{n}=\frac{r(n)}{n}
          \end{gather*}
          \par $APR$:年化百分比利率,常为投资报价利率(annual percentage rate)
          \par $I$:利息额
          \par $n$:借贷年数
          \par $PV$:现值(present value)
          \par $FV$:终值(final value)
          \par $r(n)$:n年的总利率
    \item 复利
          \begin{enumerate}[a.]
              \item 一般复利
                    \begin{gather*}
                        EAR=\left(1+\frac{APR}{m}\right)^{m\cdot n}-1=\left[1+r\left(n\right)\right]^{1/n}-1
                    \end{gather*}
                    \par $EAR$ :有效年利率,常为投资结算利率(effective annual rate)
                    \par $r$ :一般复利年利率
                    \par $m$ :每年复利次数
                    \par $n$ :借贷年数
                    \par $r(n)$ : $n$年的总利率
                    \par $\left(1+\frac{r}{m}\right)^{m·n}$ :终值复利因子
                    \par $d(m,n)=\frac{1}{\left(1+\frac{r}{m}\right)^{m·n}}$ :现值复利因子/贴现因子/折现因子(discount factor),随时间严格递减,体现了货币的时间价值
              \item 连续复利(continuous compounding)
                    \begin{gather*}
                        I=PV(e^{EAR\times n}-1)
                    \end{gather*}
                    \par $r$:连续复利年利率
                    \par 一般复利利息=连续复利利息:可求出等价的连续复利利息
                    \par 连续复利:$1+EAR=e^{APR}$
          \end{enumerate}
\end{enumerate}






\section{利率分类}



\subsection{市场利率}


\begin{enumerate}[1.]
    \item 计息时间
          \begin{enumerate}[(1)]
              \item 年利率:\%(annual interest rate)
              \item 月利率:‰(monthly interest rate)
              \item 日利率:‱(daily interest rate)
          \end{enumerate}
    \item 决定方式
          \begin{enumerate}[(1)]
              \item 市场利率:按照市场规律自发变动的利率(market interest rate)
              \item 官定利率:一国货币管理部门或者中央银行所规定的利率(official interest rate)
              \item 公定利率:由非政府部门的民间组织为维护公平竞争所确定的属于行业自律性质的利率(trade-regulated interest rate)
          \end{enumerate}
    \item 利率地位
          \begin{enumerate}[(1)]
              \item 基准利率:多种利率并存的条件下起决定作用的利率(benchmark interest rate)
                    \par 西方基准利率通常是中央银行的再贴现利率以及商业银行和金融机构之间的同业拆借利率;我国基准利率是央行对商业银行及其它金融机构的存、贷款利率,又称法定利率;货币市场的基准利率:上海银行间同业拆放利率(Shanghai interbank offered rate,SHIBOR)
              \item 一般利率:金融机构在金融市场上形成的各种利率(general interest rate)
          \end{enumerate}
    \item 信用期限
          \begin{enumerate}[(1)]
              \item 短期利率:一年期以内的信用活动适用的利率(short-term interest rate)
              \item 长期利率:一年期以上的信用活动适用的利率(long-term interest rate)
          \end{enumerate}
    \item 业务管理
          \begin{enumerate}[(1)]
              \item 普通利率
              \item 优惠利率
              \item 惩罚利率
          \end{enumerate}
    \item 市场交易
          \begin{enumerate}[(1)]
              \item 即期利率:对不同期限的金融工具以复利形式标示的利率(spot interest rate)
              \item 远期利率:给定即期汇率中隐含的未来一定期限的利率(forward interest rate)
                    \par  远期的借贷款合约可能按远期利率达成,此意义上远期利率为市场利率
          \end{enumerate}
\end{enumerate}


\subsection{政策利率}

\begin{enumerate}[1.]
    \item 币值变化
          \begin{enumerate}[(1)]
              \item 名义利率:物价水平不变即货币的实际购买力不变时的利率(real interest rate)
                    \par 名义利率是持有货币的机会成本 → 预期货币需求取决于名义利率 → 价格水平取决于现期货币量和预期的未来货币量
              \item 实际利率:包括物价变动因素的利率(nominal interest rate)
          \end{enumerate}
    \item 浮动范围
          \begin{enumerate}[(1)]
              \item 固定利率:借贷期内利息按照借贷双方事先约定的利率计算,不随市场资金供求状况而调整(fixed interest rate)
                    \par 借贷期长、市场利率波动较大的情况下不宜采用固定利率
              \item 浮动利率:借贷期内根据市场利率的变化定期调整利率(floating interest rate)
                    \par 浮动利率多用于期限较长的借贷和国际金融市场上的借贷
          \end{enumerate}
\end{enumerate}










\section{决定理论}





\subsection{马克思}

\begin{enumerate}[1.]
    \item 理论
          \begin{enumerate}[(1)]
              \item 利息量的多少取决于利润总额
              \item 利息率取决于平均利润率,介于零和平均利润率之间,取决于利润率和总利润在贷款人和借款人之间的分配比例
              \item 法律、习惯等的影响
          \end{enumerate}
    \item 利率特点
          \begin{enumerate}[(1)]
              \item 长期内平均利润率处于下降趋势,利率相同
              \item 利润率下降缓慢,利率比较稳定
              \item 利率决定有一定偶然性
          \end{enumerate}
\end{enumerate}



\subsection{可贷资金利率理论}
新剑桥学派:认为利率是借贷资金的价格,批判综合实际利率理论和流动性偏好利率(loanable-funds theory of interest)

\begin{gather*}
    F_s=F_d\\
    S+\Delta M_s=I+\Delta M_d
\end{gather*}
\par $F_s$:可贷资金的供给
\par $S$:某一时期的储蓄流量
\par $ΔM_s$:货币供给的增量
\par $I$:同期投资流量
\par $\Delta M_d$:人们希望保有的货币余额

\subsubsection{实际利率理论}

古典学派:储蓄、投资决定利率(非货币因素,实际因素)

\begin{enumerate}[1.]
    \item 投资流量(企业厂房、设备、存货等)导致的资金需求是利率的减函数
    \item 储蓄流量(家庭为主)导致的资金供给是利率的增函数
\end{enumerate}

\subsubsection{流动性偏好理论}

凯恩斯:利率与流动性偏好负相关(货币因素,短期)

\begin{enumerate}[1.]
    \item M垂直,外生变量
    \item 利率取决于货币供求数量的对比
    \item 货币当局决定货币供给
    \item 人们的流动性偏好决定货币需求
\end{enumerate}



\subsection{IS-LM模型}

\subsection{影响因素}

\subsubsection{宏观}

\begin{enumerate}[1.]
    \item 宏观经济周期(内生)
          \begin{enumerate}[(1)]
              \item 危机阶段:资金供不应求,利率走高
              \item 萧条阶段:资金需求降低,利率走低,甚至零利率
              \item 复苏阶段:资金需求增加,利率走高
              \item 繁荣阶段:利率持续提高
          \end{enumerate}
    \item 制度(外生):利率管制程度
          \begin{enumerate}[(1)]
              \item 利率管制:直接制定利率或规定上下限
              \item 利率市场化:通过市场和价值规律机制,在某一时点上由供求关系决定的利率运行机制(interest rate liberalization)
          \end{enumerate}
\end{enumerate}

\subsubsection{微观}


\par * 解释经验事实:
\par A 不同期限债券的利率随时间变化一起波动
\par B 短期利率低,收益率曲线更倾向于向上倾斜;如果短期利率高,收益率曲线可能向下倾斜
\par C 收益率曲线通常是向上倾斜的

\paragraph{预期/期望假说}

(expectation hypothesis)

\begin{enumerate}[1.]
    \item 假设:
          \begin{enumerate}[(1)]
              \item 不同债券完全可替代,投资者不偏好某种债券
              \item 所有债券都会有相同的收益率
          \end{enumerate}
    \item 理论:
          \begin{enumerate}[(1)]
              \item A 投资者在不同期限债券之间套利,使得不同期限债券价格相互影响、同升同降
                    \par $\left(1+r_n\right)^n=\left(1+f_1\right)\left(1+f_2\right)\left(1+f_3\right)..\ldots\left(1+f_n\right)$
              \item B 短期利率$r_1$,长期债券的到期收益率很大程度上取决于投资者对未来各年度远期利率$f_i$的预期。若$r_1$高位,则$f_i$下降的可能性更大
                    \par 远期利率:
                    \begin{gather*}
                        f_n = \frac{1}{\frac{P_{n-1,t}}{P_{n,t}}}-1=\frac{\left(1+r_n\right)^n}{\left(1+r_n\right)^{n-1}(n-1)}\\
                        R_F = \frac{R_2T_2-R_1T_1}{T_2-T_1}
                    \end{gather*}
                    \par $R_1$:期限为$T_1$的零息利率
                    \par $R_F$:$T_1$、$T_2$之间的远期利率
              \item C 无法解释
          \end{enumerate}
\end{enumerate}


\paragraph{市场分割理论/期限偏好理论}

(market segmentation theory)


\begin{enumerate}[1.]
    \item 假设:
          \begin{enumerate}[(1)]
              \item 不同期限的债券不是替代品,不同投资者会对不同期限的债券具有特殊偏好(一般偏好期限较短、利率风险小的证券
              \item 长期债券产生正的流动性溢价,若市场偏好长期债券,则流动性溢价为负)若短期利率超过长期利率,可能预示着经济衰退
          \end{enumerate}
    \item 理论:$f_n=E\left(r_n\right)+\text{流动性溢价}$
          \begin{enumerate}[(1)]
              \item A 无法解释
              \item B 无法解释
              \item C 需要高利率补偿长期债券的购买
          \end{enumerate}
\end{enumerate}


\paragraph{期限选择与流动性升水理论}

综合预期假说和市场分割理论(liquidity premium theory)



\begin{enumerate}[1.]
    \item 假设:
          \begin{enumerate}[(1)]
              \item 不同期限的债券是替代品,但并不完全可替代
              \item 要让投资者持有风险较大的长期债券,必须向其支付流动性升水以补偿其增加的风险
          \end{enumerate}
    \item 理论:
          \begin{enumerate}[(1)]
              \item AB同预期假说
              \item C 当考虑期限选择和流动性升水时,因为需要提供收益补偿,收益率曲线向下的概率会大大降低,向上倾斜的概率会大大增加
          \end{enumerate}
\end{enumerate}










\section{作用与影响}

\subsection{宏观}

\begin{enumerate}[(1)]
    \item 储蓄投资的规模和结构
    \item 借贷资金供求
    \item 资产价格:加息利空房地产价格和证券行市
    \item 社会总供求的调节:短期易于调节总需求,短期低利率刺激总需求(消费投资),增加总供给压力,长期倾向于增加总供给缓解供求压力,影响消费行为
    \item 资源配置效率:高利率会淘汰低弱企业,优质企业资金可得性增加,经济增长率会下降,但会导致资源消耗速度下降,资源配置效率提高
    \item 金融市场:利率是金融工具定价的基本要素
\end{enumerate}


\subsection{微观}

\begin{enumerate}[(1)]
    \item 促进企业加强核算,提高经济效益
    \item 调节个人的行为决策
\end{enumerate}



\subsection{条件}


\begin{enumerate}[(1)]
    \item 基础性条件
          \begin{enumerate}[a.]
              \item 独立的市场决策主体
                    \begin{enumerate}[(a)]
                        \item 市场主体独立决策并独立承担责任,权责利的有机结合
                        \item 微观经济主体是理性经济人
                    \end{enumerate}
              \item 市场化的利率决定机制
                    \begin{enumerate}[(a)]
                        \item 利率高低能够真实反映资金稀缺程度及其机会成本影响
                        \item 影响微观主体的投资和融资决策
                        \item 通过利率信号筛选优质项目,引导和配置资金口厂商的投资决策
                    \end{enumerate}
              \item 合理的利率弹性
          \end{enumerate}
    \item 经济制度与经济环境
          \begin{enumerate}[a.]
              \item 市场化改革:塑造独立决策、独立承担责任的市场行为主体
              \item 市场投资机会与资金的可得性:影响利率弹性和微观主体对利率的敏感性
              \item 产权制度:影响微观主体的激励和约束
          \end{enumerate}
\end{enumerate}













\chapter{汇率}







汇率:一种货币用另一种货币表示的价格(exchange rate),又称汇价

\informationBox{
    \begin{enumerate}[(1)]
        \item 对内价值:国内物价;对内价值是对外价值的基础  
        \item 对外价值:汇率  
        \begin{enumerate}[a.]
            \item 直接标价法:以一定单位的外币为标准计算应付多少本币来表示汇率(direct quotation)  
            \item 间接标价法:以一定数量的本币单位为基准,计算应收多少外币表示汇率(indirect quotation)
            \par 直接标价法与间接标价法互为倒数
        \end{enumerate}
    \end{enumerate}
}



外汇(foreign exchange):
\informationBox{
    \begin{enumerate}[(1)]
        \item 广义:以外币标示的各种金融资产。包括外币、外币有价证券(如外国政府的债券、信用级别比较高的外国公司债券和股票)、外币支付凭证等一切可用于国际结算的债券  
        \item 狭义:以外币标示的、可用于国际结算的支付手段  
        \item 条件: 
        \begin{enumerate}[a.]
            \item 可以自由输出入国境  
            \item 可以自由兑换、买卖  
            \item 在国际支付中被广泛接受
        \end{enumerate} 
    \end{enumerate}
}












\section{汇率分类}







\subsection{市场汇率}

\begin{enumerate}[1.]
    \item 外汇买卖	
    \begin{enumerate}[(1)]
        \item 买入汇率:外汇银行买进外汇(结汇)时所使用的汇率(bidding rate),又称买入价(bid price)  
        \par 现钞买入价 < 现汇买入价  
        \item 卖出汇率:银行售出外汇(售汇)时所使用的汇率(offering rate),又称卖出价(ask price)现钞卖出价、现汇卖出价  
        \par 间接标价法下买入汇率高于卖出汇率  
        \item 中间汇率:买入汇率和卖出汇率的算术平均数(intermediate exchange rate)  
    \end{enumerate}
    \item 交割期限	
    \begin{enumerate}[(1)]
        \item 即期汇率:买卖双方成交后,办理交割时所使用的汇率(spot exchange rate)  
        \item 远期汇率:买卖双方事先约定的,据以在未来一定时期(或时点)进行外汇交割时所使用的汇率(forward exchange rate)  
        \begin{enumerate}[a.]
            \item 升水:远期汇率高于即期汇率,对于外汇来说(premium)  
            \item 贴水:远期汇率低于即期汇率(discount)  
            \item 平价:远期汇率等于即期汇率(par)
        \end{enumerate}
    \end{enumerate}
    \item 兑换方式
    \begin{enumerate}[(1)]
        \item 电汇汇率:银行卖出外汇后以电信方式通知国外行或代理行付款时所使用的一种汇率(telegraphic transfer rate)  
        \par 电汇汇率是国际支付最主要的方式,是外汇汇率的基准汇率  
        \item 信汇汇率:以信函解付的方式买卖外汇时所使用的汇率(mail transfer rate)  
        \item 票汇汇率:以票据作为支付手段进行外汇买卖时所使用的汇率(note transfer rate)    
    \end{enumerate}	
    \item 营业时间
    \begin{enumerate}[(1)]
        \item 开盘汇率:外汇银行在一个营业日开始时进行首批外汇买卖时使用的汇率(open exchange rate)  
        \item 收盘汇率:外汇银行在一个营业日结束时所使用的汇率(close exchange rate)
    \end{enumerate}	 
\end{enumerate}

\subsection{政策汇率}


\begin{enumerate}[1.]
    \item 制定方法	
    \begin{enumerate}[(1)]
        \item 基准汇率:本币与对外经济交往中最常用的主要货币之间的汇率(benchmark exchange rate)  
        \par 基准汇率:一般以美元为基本外币来确定. 我国基准汇率:人民币兑美元、欧元、日元、港币和英镑  
        \item 套算汇率:根据本币基准汇率套算出本币兑非主要货币的其他外币的汇率或套算出其他外币之间的汇率(cross exchange rate ),又称交叉汇率
    \end{enumerate}
    \item 汇率管理
    \begin{enumerate}[(1)]
        \item 官方汇率:一国的外汇管理当局制定并公布实行的汇率(official exchange rate)  
        \item 市场汇率:由外汇市场供求关系决定的汇率(market exchange rate)
    \end{enumerate}	
    \item 汇率制度
    \begin{enumerate}[(1)]
        \item 固定汇率:一国货币的汇率基本固定,汇率波动幅度被限制在较小范围内(fixed exchange rate)  
        \item 浮动汇率:不规定汇率波动的上下限,汇率随外汇市场的供求关系自由波动(floating exchange rate)
    \end{enumerate} 
    \item 物价差异
    \begin{enumerate}[(1)]
        \item 名义汇率:市场上观察到的挂牌交易使用的汇率(nominal exchange rate)  
        \par 两个国家通货的相对价格  
        \item 实际汇率:在名义汇率的基础上,考虑到一种货币所在经济体与其他经济体之间物价差异因素的汇率(real exchange rate)  
        \par 两国产品的相对价格 
    \end{enumerate}
    \item 综合	
    \begin{enumerate}[(1)]
        \item 有效汇率:加权汇率指数(effective exchange rate)  
        \par $$ NEER_t=\sum_{i=1}^{n}\left(w_i\cdot\frac{E_t}{E_{base}}\right) $$  
        \par NEER:名义有效汇率    
        \par $w_i$:每个国家货币的权重,多以贸易份额  
        \par $E_t$:t 年汇率(间接标价法)    
        \par $E_{base}$:基年汇率(间接标价法)反映一国货币总体升贬值情况     
    \end{enumerate}
\end{enumerate}



















\section{决定理论}







直接标价法


\informationBox{
    \par * 人民币汇率四大决定因素:货币政策 + 经济增长前景 + 资本流动管制 + 中间价形成机制 → 名义利差 + 经济增长与长期汇率预期 + 国际收支结构 + 中间价与短期预期粘性 → 即期汇率   
    \par * Asset Pricing-Carry Trade(主流)
}



\subsection{购买力平价理论}

(purchasing power parity,PPP) 

\informationBox{
    * 适用:长期,高通胀趋势性的商品市场
}



\subsubsection{假设}

\begin{enumerate}[1.]
    \item 完全的资本流动,无交易成本  
    \item $Y$不变  
    \item 长期价格灵活性($P$可变)  
    \item 充分就业(汇率和价格调整的结果)(长期平均消费倾向不变)  
    \item $e$可变  
\end{enumerate}





\subsubsection{模型}

\paragraph{绝对购买力平价}

(absolute purchasing power parity)

\begin{gather*}
    E=\frac{P}{P^\ast}
    =\frac{E·P*}{P}·\frac{P}{P*}=e⋅\frac{P}{P*}  
\end{gather*}


货币分析法(monetary approach)

\begin{align*}
E
&=\frac{P}{P^\ast}
=\frac{\frac{M_S=M_D}{L\left(r+\pi^e,Y\right)}}{\frac{M_S^\ast=M_D^\ast}{L\left(r^\ast+\pi^{e^\ast},Y^\ast\right)}}
=\frac{\frac{M_S}{L\left(R,Y\right)}}{\frac{M_S^\ast}{L\left(R^\ast,Y^\ast\right)}}
=\frac{\frac{M_S}{kY^\alpha R^{-\beta}}}{\frac{M_S^\ast}{kY^{\ast^\alpha}R^{\ast^{-\beta}}}}
=\frac{M_S}{M_S^\ast}·(\frac{Y*}{Y})^\alpha·(\frac{R}{R^*})^\beta\\
&\approx (\dot{M_s}-\dot{M_s}^*)+\alpha(\dot{Y}^*-\dot{Y})+\beta(\dot{R}-\dot{R}^*)
\end{align*}

\par $E$:名义汇率  
\par $P$、$P^∗$:国内物价水平、外国物价水平(以本国货币衡量)  
\par $MD$:名义货币需求  
\par $L(R,Y)$:实际货币需求  
\par ${\dot{M}}_s^\ast$、${\dot{M}}_s$:外国与本国的货币供应量变动率  
\par $\dot{Y}^∗$、$\dot{Y}$:外国与本国的国民收入变动率  
\par $R$、$R^{∗}$:外国和本国的利率变动率


\paragraph{相对购买力平价}

(relative purchasing power parity):通过价格传导

\begin{gather*}
    M_S\uparrow\rightarrow Y\downarrow\rightarrow i\downarrow\rightarrow P\uparrow\rightarrow E\uparrow 
\end{gather*}

实际汇率不变:
\begin{gather*}
    \left.\begin{matrix}
        \frac{E_1}{E_0}-1 = \frac{P_1/P_0}{P_1^*/P_0^*} -1=\frac{1+\pi}{1+\pi^{*}} \\
    \text{或} lnE=ln(P/P^{*}) = lnP-lnP^{*}
      \end{matrix}\right\}
      →\frac{E_1-E_0}{E_0} = \pi-\pi^{*}
\end{gather*}


实际汇率变化:
\begin{gather*}
    \ln{E}=\ln{\left(e\frac{P}{P^\ast}\right)}=\ln{e}+\ln{P}-\ln{P^\ast}\Rightarrow\frac{E_1-E_0}{E_0}=\frac{e_1-e_0}{e_0}+\left(\pi-\pi^\ast\right)
\end{gather*}
\par $E_0$:基期汇率  
\par $P_0$、$P_0^∗$:国内、外国基期价格水平  
\par $P_1$、$P_1^∗$:国内、外国$E$时期后的价格水平   
\par $e$:实际汇率($e < 1$,本国商品竞争力弱)


\subsubsection{理论}



\subsection{利率平价理论}

(interest rate parity)  

\informationBox{
    * 适用:短期波动性的货币市场 + 外汇市场
}



\subsubsection{假设}

\begin{enumerate}[1.]
    \item 完全的资本流动,无交易成本  
    \item Y 可变短期价格刚性(P 不变)  
    \item 存在失业、资源闲置(短期边际消费倾向递减)  
    \item $\pi^e$不变 
\end{enumerate}
 


\subsubsection{模型}

\paragraph{名义利率平价}

用相同货币衡量的任意两种货币存款的预期收益率相同

\begin{gather*}
    R = R^\ast+\frac{E^e-E}{E}+\rho\left(B-A\right) \\
    R = R^\ast+\frac{f-s}{s}+\rho\left(B-A\right) 
\end{gather*}
\par $i^\ast$、$i$:国外利率、国内利率  
\par $E^e$:预期汇率(uncovered,UCIP 无抛补、未套期保值)  
\par $f$:远期汇率(远期合约 covered,CIP,抛补、套期保值)  
\par $s$:即期汇率  
\par $\rho$:风险升水(与$B−A$正相关;政府挤出投资: $B↑/A↓→\rho ↑ →r↑→I↓$ )(若$\rho =0$,本币债券和外币债券完全替代;否则不可完全替代)  
\par $B$:本国政府债券存量   
\par $A$:央行持有的国内资产加息国家货币在远期外汇市场上贬值,即期升值



\paragraph{实际利率平价}

通过资本流动、收益率传导

实际汇率不变(引入费雪效应):
\begin{gather*}
    r^e-r^{\ast^e}=\left(R-\pi^e\right)-\left(R^\ast-\pi^{e^\ast}\right)=\left(R-R^\ast\right)-\left(\pi^e-\pi^{e^\ast}\right)=\frac{E^e-E}{E}-\left(\pi^e-\pi^{e^\ast}\right)
\end{gather*}


实际汇率变化:
\begin{gather*}
    =\left[\frac{e^e-e}{e}+\left(\pi^e-\pi^{e^\ast}\right)\right]-\left(\pi^e-\pi^{e^\ast}\right)=\frac{e^e-e}{e} 
\end{gather*}


当市场预期符合相对购买力平价时:
\begin{gather*}
    e^e=e,\ r^e=r^{\ast^e},\ R-R^\ast=\pi^e-\pi^{e^\ast},\ \frac{E^e-E}{E}=\pi^e-\pi^{e^\ast}=r-r^\ast
\end{gather*}

\subsubsection{理论}

暂时性$M_S$↑→r ↓ →i ↓(流动性偏好理论)→E ↑












\section{作用与影响}



\begin{enumerate}[1.]
    \item 风险
          \begin{enumerate}[(1)]
              \item 分类
                    \begin{enumerate}[a.]
                        \item 交易汇率风险:进出口贸易有收付外汇损失
                        \item 折算汇率风险(会计风险):外汇资产贬值损失;清偿外债损失
                        \item 经济汇率风险(经营风险):长久汇率扭曲,导致经济长期衰退,影响经济结构
                    \end{enumerate}
              \item 措施:
                    \begin{enumerate}[a.]
                        \item 进口支付尽量使用软货币(有贬值趋势或压力的货币)
                        \item 出口收入外汇尽量选择硬货币
                        \item 运用远期外汇买卖进行套期保值
                    \end{enumerate}
          \end{enumerate}
    \item 进出口
          \begin{enumerate}[(1)]
              \item 假设:进口需求和出口供给有价格弹性
              \item 本币贬值,本国商品国际竞争力提高,促进出口、抑制进口
              \item 本币升值,利于进口,不利出口
          \end{enumerate}
    \item 物价
          \begin{enumerate}[(1)]
              \item 进口商品和原材料:本币(名义)汇率贬值,进口商品国内价上涨
              \item 出口商品:本币(名义)汇率贬值+出口商品供给弹性小→国内抢购此商品(收购)→抬高国内价格
                    \par 汇率贬值→国外需求增加+国内需求不变→总需求增加→价格上涨
              \item 对物价总指数的影响,取决于在GNP中所占的比重
          \end{enumerate}
    \item 资本流动
          \begin{enumerate}[(1)]
              \item 本币对外贬值→持有本币计值金融资产意愿降低,相对吸引力下降→转兑成外汇,资本外流→外汇需求增加→本币汇率进一步下跌
              \item 对长期资本流动影响较小
          \end{enumerate}
    \item 金融资产选择
          \begin{enumerate}[(1)]
              \item 本币汇率下降,持有更多本币资产
          \end{enumerate}
\end{enumerate}



















\subsection{条件}

\begin{enumerate}[1.]
    \item 进出口商品的需求弹性、出口商品的供给弹性  
    \item 各国经济体制、市场条件、市场运行机制  
    \item 对外开放程度   
    \item 汇率制度
\end{enumerate}










\section{汇率制度}



\informationBox{
    * 影响因素:
    \begin{enumerate}[(1)]
        \item 经济论:经济规模、经济发展程度、经济开放程度、通货膨胀率
        \item 依附论:经济政治等对于他国的依赖程度
    \end{enumerate}
}




\subsection{原理}

\subsubsection{不可能三角}

定义:一国不可能同时拥有自由的资本流动、固定汇率和独立的货币政策(impossible trinity),又称国际金融三角困境(trilemma of international finance)





\subsection{分类}

\subsubsection{固定汇率制}




\begin{enumerate}[1.]
    \item 定义
          \begin{enumerate}[(1)]
              \item 一国的汇率基本固定,汇率的波动幅度被限制在较小的范围内,各国中央银行有义务维持本币币值基本稳定的汇率制度(fixed exchange rate regime)
              \item 经济开放程度⾼、经济规模⼩,或贸易商品结构和地区分布⽐较集中的国家,⼀般倾向于实⾏固定汇率制。
          \end{enumerate}
    \item 分类
          \begin{enumerate}[(1)]
              \item No separate legal tender
                    \begin{enumerate}[a.]
                        \item 美元化:本国货币被其他国家货币替代(Dollarization)
                              \begin{enumerate}[(a)]
                                  \item 国家体量小,对于美元依赖高;通胀严重
                              \end{enumerate}
                        \item 货币联盟汇率:本国货币与主要贸易伙伴的货币保持固定比价,而对其他经济体的货币随该货币锚而浮动
                    \end{enumerate}
              \item 货币局制度(currency board)
                    \begin{enumerate}[a.]
                        \item 经济体货币与另一货币保持强固定,固定汇率制两地区利率应趋同(美国和香港)
                        \item 如香港
                        \item 不能成为最后贷款人
                    \end{enumerate}
          \end{enumerate}
    \item 优点
          \begin{enumerate}[(1)]
              \item 限制政府任意的宏观经济政策
          \end{enumerate}
    \item 缺点:政策难协调
          \begin{enumerate}[(1)]
              \item 货币政策:balance of payments crisis:固定汇率制下外国资产不足以支持维持固定汇率制,看好预期改变,为了阻止资本外逃,央行大幅提高利率,但会冲击国内经济
              \item 财政政策:米德冲突:失业+经常账户盈余=财政政策扩张;失业+赤字=?;缺人+盈余=?;缺人+赤字=紧缩
          \end{enumerate}
\end{enumerate}


\subsubsection{浮动汇率制}


\begin{enumerate}[1.]
    \item 定义:政府不规定汇率波动的上下限允许汇率随外汇市场供求关系的变化而自由波动,各国中央银行仅根据需要自由选择是否进行干预的汇率制度(floating exchange rate regime)
    \item 分类
          \begin{enumerate}[(1)]
              \item (independent float, Clean float)
                    \par 恶性冲击、过度波动时干预
              \item 管理浮动制(managed float, dirty floating)
                    \par 中国:以市场供求为基础、参考一篮子货币进行调节、有管理的浮动汇率制度
          \end{enumerate}
    \item 优点:相对固定汇率制
          \begin{enumerate}[(1)]
              \item 贸易金融:汇率自动调节国际收支(automatic stablization)
              \item 央行:减少国际游资的冲击,减少国际储备需求(固定汇率制下,投机性资金集中抛售软货币,抢购硬货币谋利)
              \item 货币政策:货币政策有独立性,内外均衡易于协调
              \item 汇率政策避免资本流动不利影响对外均衡,财政货币政策对内均衡(固定汇率逆差、经济衰退时,只能紧缩财政货币政策,这对经济不利)
              \item 防治国外通胀传至国内
          \end{enumerate}
    \item 缺点
          \begin{enumerate}[(1)]
              \item 不利于国际贸易和投资的发展(成本、收益难以精准核算,不确定性大)
              \item 助长国际金融市场上的投机活动
              \item 经济波动加大
              \item 货币政策:可能引发货币之间竞相贬值(一国通过汇率贬值改善国际收支)
              \item 货币政策受限少,更容易诱发通胀
          \end{enumerate}
\end{enumerate}




\subsubsection{中间汇率制}


\begin{enumerate}[1.]
    \item 定义
    \item 分类
          \begin{enumerate}[(1)]
              \item 传统盯住(Conventional Peg)
              \item 波幅钉住(Peg with bands)
              \item 爬行钉住(Crawling Peg)
              \item 波幅爬行钉住
          \end{enumerate}
    \item 优点
    \item 缺点
\end{enumerate}











\chapter{金融工具}













\section{绝对价值评估}




\subsection{股票}




\subsubsection{现金流折现模型}



\paragraph{一般折现模型}

(divided discount model,DDM)

\begin{gather*}
  P_0 = \sum_{t=1}^{\infty}{\frac{C_t}{(1+r)^t}}
\end{gather*}
 
\par $P_0$:企业价值
\par $C_t$:$t$期每股现金流(现金流不确定,风险高,因此收益高于债券)
\par $r$:贴现率


\paragraph{零增长模型}

(Zero-growth Model)


假设$C$零增长,即$g=0$
\begin{gather*}
  P_0=\sum_{t=1}^{\infty}{\frac{C}{(1+r)^t}} 
     =\frac{C}{r}
\end{gather*}


\paragraph{固定增长模型}

成熟性公司(constant growth model)

\begin{align*}
  P_0& = \sum_{t=1}^{\infty}{\frac{C_0(1+g)^t}{(1+r)^t}}
      = \frac{C_{0}(1+g)}{r-g}
      =\frac{C_1}{r-g} \\
  P_t&=\frac{C_{t+1}}{r-g}
     =\frac{C_t\left(1+g\right)}{r-g}
     =P_{t-1}\left(1+g\right)
\end{align*}
\par $C_0$:当期每股现金流
\par $g$:现金流预期增长率,假设等比例上升($r>g$)


\paragraph{两阶段折现模型}

规模大的成长型公司(2-stage divided discount model)


\begin{align*}
  P_0&=\sum_{t=1}^{n}{\frac{C_0 (1+g_1 )^t}{(1+r)^t}} +\sum_{t=n+1}^{\infty}{\frac{C_0 (1+g_2 )^t}{(1+r)^t}} \\
  P_n&=\frac{C_{n+1}}{r-g_2} \\
  P_0&=\sum_{t=1}^{n}{\frac{C_t}{(1+r)^t}}
\end{align*}
\par $g_1$:发展期现金流预期增长率(growing)
\par $g_2$:成熟期现金流预期增长率(constant g, using constant model)
\par $n$:发展期末年份

\paragraph{三阶段折现模型}

规模小的成长型公司






\begin{landscape}
    \tiny
    \begin{tabu}to 0.8\columnwidth{|c|c|c|c|c|}
        \hline
             &                            & \multicolumn{2}{c|}{权益估值(equity valuation)} & 公司估值(film valuation)                             \\\hline
             &                            & equity holder                                     & equity holder              & debt holder+equity holder \\\hline
        P    &
             & P:股票价格
             & \begin{tabu}[t]{c}
                    E:所有权价值 \\
                    $P=\frac{E}{shares}$
                \end{tabu}
             & \begin{tabu}[t]{c}
                    V:企业价值 \\
                    $P=\frac{V-MVD=MVE}{shares}$
                \end{tabu} \\\hline
        C    &
             & \begin{tabu}[t]{c}
            D:每股股票分红
        \end{tabu}
             & \begin{tabu}[t]{c}
            FCFE:股东自由现金流
        \end{tabu}
             & \begin{tabu}[t]{c}
            FCFF:企业自由现金流
        \end{tabu}                               \\\hline
         r    &
              & \multicolumn{2}{l|}{
             \begin{tabu}to 10cm{X}
                 $r_s$:贴现率
                 \begin{gather*}
                     expected\ return =\frac{{\hat{D}}_1+{\hat{P}}_1-P_0}{P_0}={\hat{D}}_1+\frac{{\hat{P}}_1-P_0}{P_0}=DY+CGY\\
                     =DY+g=HPR→market\ price \\
                     required\ return = \frac{{\hat{D}}_1+{\hat{P}}_1-{\hat{P}}_0}{{\hat{P}}_0}=r_f+\beta\left[r_m-r_f\right]\rightarrow intrinsic\ price \\
                     actual\ return   = \frac{D_1+P_1-P_0}{P_0}
                 \end{gather*}
                 \par market\ price < intrinsic\ price $\Leftrightarrow$ expected\ return>required\ return $\Leftrightarrow$ undervalue $\Leftrightarrow$ buy
                 \par market\ price = intrinsic\ price $\Leftrightarrow$ expected\ return=required\ return $\Leftrightarrow$ fair\ value $\Leftrightarrow$ hold
                 \par market\ price > intrinsic\ price $\Leftrightarrow$ expected\ return<required\ return $\Leftrightarrow $overvalue $\Leftrightarrow$ sell
                 \par $r_m$:market return
                 \par $DY$:dividend yield
                 \par $CGY$:capital gain yield
             \end{tabu}
         } 
              & \begin{tabu}to 10cm{X}
             \tablelist{1.}{0.4\columnwidth}{
                \item 
                    \begin{gather*}
                        {WACC}_t = w_d\ast r_d(1-t)+w_p\ast r_p+w_s\ast r_s
                    \end{gather*}
                \item $WACC$:Weighted Average Cost of Capital,都使用市价,不用账面价值
                \item $w_d$:长期负债权重
                \item $w_p$:优先股权重
                \item $w_s$:普通股权重
                \item $r_d$:
                    \begin{enumerate}[(1)]
                        \item 用债券折现模型
                        \item 无股票交易用债券评级表找到估计的RP
                        \item 若未发债,根据Interest Coverage Ratio=EBIT/Interest划分评级
                    \end{enumerate}
                \item $r_p$:
                    \begin{gather*}
                        g=0, P=\frac{D_0 (1+g)}{r-g}
                        \Rightarrow r_p=\frac{D_p}{P_p}
                        \end{gather*}
                 \item $r_s$:机会成本(不派发红利+再投资)
                     \begin{enumerate}[(1)]
                         \item CAPM:$r_s=r_{RF}+\beta\left(r_M-r_{RF}\right)$,$r_{RF}$用同期限的国债利率
                         \item DCF:$r_s= D1/P01-f+g$,可用于non-constant growth
                         \item Bond-yield+RP:$r_s=r_d+RP$,RP不是CAPM的,是两个债券的差
                         \item Fama-French
                         \item average
                     \end{enumerate}
                     \par 当前股价不受持有时间的影响
                     \par 当考察对象是项目时,WACC要进行相应的调整
             }
         \end{tabu}  \\\hline
        \multirow{2}{*}{g}
             & 自身
             & $g_{Div.}$:股票分红增长率
             & $g_{FCFE}$:FCFE增长率
             & $g_{FCFF}$:FCFF增长率                                                                                                                  \\\cline{2-5}
             & 其他
             & \multicolumn{3}{l|}{
            \tablelist{1.}{0.4\columnwidth}{
                \item EPS:假设公司没有派发新股,没有回购股票
                \begin{enumerate}[(1)]
                    \item 历史数据:$g_{EPS,t}=\frac{{EPS}_t-{EPS}_{t-1}}{{EPS}_{t-1}}$
                    \item 分析师预测:$g_{EPS,t+1}=\frac{{EPS}_{t+1}-{EPS}_t}{{EPS}_t} $
                \end{enumerate}
                \item 基本面分析:
                \begin{enumerate}[(1)]
                    \item 假设ROE不变:投资带来的增长是可持续的
                          \begin{gather*}
                              g=b\times ROE=\frac{RE}{NI}\cdot ROE=retention\ rate\times ROE=\frac{NI-Dividend}{NI}\cdot ROE=\left(1-Payout\right)\times ROE=\left(1-\frac{DPS}{EPS}\right)\times ROE
                          \end{gather*}
                    \item 假设ROE变化:ROE的增长不可持续,竞争导致不可能一直有投资机会
                          \begin{gather*}
                              g_{t+1}=b\cdot{ROE}_{t+1}+\frac{{ROE}_{t+1}-ROE_t}{{ROE}_t}
                          \end{gather*}
                          \par Small changes in ROE translate into large changes in the expected growth rate
                \end{enumerate}
            }
        }   \\\hline
        适用 &
             & \begin{tabu}[t]{l}
            分红稳定上市公司 \\
            资本结构相对稳定
        \end{tabu}
             & \begin{tabu}[t]{l}
            所有企业(金融企业慎用) \\
            并购时用(不关心debt)   \\
            资本结构相对稳定
        \end{tabu}
             & \begin{tabu}[t]{l}
            所有企业(金融企业慎用) \\
            资本结构可能变化
        \end{tabu}
        \\
        \hline
    \end{tabu}
\end{landscape}

\subsubsection{期权估值模型}

初创企业







\subsection{债券}





\subsection{衍生品:远期}


远期合约价格=即期或现金价格+持有成本(carry cost)

\subsubsection{远期外汇}

\paragraph{外币标的}
\begin{gather*}
    F(t)=S(t) e^{(r-q)(T-t)}
\end{gather*}

\par $r$:期限为$T$的本币无风险利率
\par $q$:期限为$T$的外币无风险利率,或资产收益率

\paragraph{金银标的}
\begin{gather*}
  F(t)=[S(t)+U] e^{r(T-t)} 
      =S(t) e^{(r+u)(T-t)}
\end{gather*}
\par $U$:[t,T]期间储存费用的贴现
\par $u$:与商品价格挂钩的存储费用,视作负收益率  



\subsubsection{远期利率}
\begin{gather*}
    A=\frac{P(r_m-r_s)\frac{t}{360}}{(1+r_m)·\frac{t}{360}}
\end{gather*}
\par $A$:利差
\par $P$:名义本金额
\par $r_m$:参考利率(结算日前确定)
\par $r_s$:协议利率
\par $t$:计息天数






\subsection{衍生品:期货}


\subsubsection{假设}

\begin{enumerate}[(1)]
\item	不考虑交易费用和税收
\item	市场参与者能以相同的无风险利率借入和贷出资金
\item	进期合约没有违约风险
\item	允许现货卖空,交易市场为均衡市场
\item	期货合约的保证金账户支付同样的无风险利率
\end{enumerate}		

\subsubsection{模型}

\paragraph{无收益资产}	

\begin{gather*}
  F\left(t\right)=S\left(t\right)e^{r\left(T-t\right)} \\
  f\left(t\right)=\left[F\left(t\right)-K\right]e^{-r\left(T-t\right)}=S\left(t\right)-Ke^{-r\left(T-t\right)}=0,\ t_0\le t\le T
\end{gather*}

\par $T$:远期合约(期货)的到期时间,单位为年,合约生效日起算
\par $t$:现在的时间,单位为年,合约生效日起算
\par $S\left(t\right)$:标的资产在时间t时的价格
\par $S(T)$:标的资产在时间$T$时的价格(在$t$时刻未知)
\par $K$:远期、期货合约中的交割价格,初始合约价格
\par $f\left(t\right)$:远期合约多头在t时刻的价值,即t时刻的远期价值
\par $F\left(t\right)$:$t$时刻的远期合约和期货合约中的理论远期价格和理论期货价格,即远期价格和期货价格
\par $r$:$T$时刻到期的以连续复利计算的$t$时刻的无风险年利率

\paragraph{已知收益}

已知现金收益的资产:到期前会产生完全可预测的现金流的资产
\begin{gather*}
  F\left(t\right)=\left[S\left(t\right)-I\right]e^{r\left(T-t\right)}\\
  f\left(t\right)=S\left(t\right)-I-Ke^{-r\left(T-t\right)}=0
\end{gather*}
\par $I$:中间收入的现值

\paragraph{已知收益率}	

已知现金收益率的资产:到期前将产生与该资产现货价格成一定比率收益的资产
\begin{gather*}
  F\left(t\right)=S\left(t\right)e^{\left(r-q\right)\left(T-t\right)}\\
  f\left(t\right)=S\left(t\right)e^{-q\left(T-t\right)}-Ke^{-r\left(T-t\right)}=0
\end{gather*}
\par $q$:资产在$\left[t,T\right]$内的平均年收益率
\par $r-q$:资产在$\left[t,T\right]$内的无风险利率





\subsection{衍生品:期权}


\subsection{衍生品:互换}


\section{相对价值评估}



\subsection{股票}





\subsubsection{回归法}

相对估价转换为回归的方式,克服传统的不可比问题;回归中不同行业等可比

\paragraph{资本资产定价模型(capital asset pricing model,CAPM)}


\subparagraph{假设}	

\begin{enumerate}[(1)]
  \item 投资者是理性的,严格按照马科维茨模型的有效规则进行多样化投资,从有效边界的某处选择投资组合
  \item 资本市场完全有效,没有任何摩擦阻碍投资
  \item 投资者只关心某一特定时段内的收益率 (time horizons for different traders are different)
  \item 投资者只关心前两阶矩(first two moments)期望值和标准差
  \item 不同投资者的ε相互独立
  \item 所有投资者以无风险利率借贷资金
  \item 不考虑税收
  \item 投资者具有一致性预期性(homogeneous expectation)。估计出相同的期望、标准差、投资品的相关系数
\end{enumerate}	

\subparagraph{模型}

\begin{gather*}
  E(R_i) = \bar{R}_i = R_f+\beta_i (\bar{R}_m-R_f )+\varepsilon_i
\end{gather*}
\par $\bar{R}_i$:预期回报率,期望收益率(expected return)
\par $R_f$:无风险资产回报率(risk-free rate),无风险收益率,货币的时间价值
\par $\beta_i$:组合证券的β系数,衡量证券或投资组合相对于总体市场波动性(volatility),反映系统性风险
\par $\bar{R}_m=E(R_m )$:市场期望回报率(expected market return)
\par $\bar{R}_m-R_f$:股票市场溢价(equity market premium),超额收益(excess return)
\par $\beta_i (\bar{R}_m-R_f )$:风险溢价(risk premium),风险和不确定性
\par $\alpha_i=\varepsilon_i$:(extra return)

$$ \beta = \frac{Cov(R_i,R_m )}{Var(R_m)}  $$



\subparagraph{理论}

\begin{enumerate}[(1)]
  \item $\beta$系数高,投资者会提升股票的预期回报率
  \item 所有投资者期望持有相同的投资组合
\end{enumerate}






\paragraph{套利定价模型(arbitrage pricing theory,APT)}

\subparagraph{假设}	
\begin{enumerate}[(1)]
  \item 投资者有相同的投资理念
  \item 投资者是回避风险的,并且要求效用最大化
  \item 市场是完全的
\end{enumerate}

\subparagraph{模型}

单因素:资产收益率影响因素来自公司特有的因素和宏观经济因素(共同因素)
\begin{gather*}
  E(r_p )=r_p+\beta_p [F-E(F)]+\varepsilon_p
\end{gather*}
\par $E(r_p)$:预期收益率
\par $r_p$:公司最近一期收益率
\par $\beta_p$:证券对该因素的敏感性
\par $F$:共同因素(通胀率、失业率等)
\par $E(F)$:预期的共同因素
\par $\varepsilon_p$:公司因素引起收益的波动
\par 共同因子为系统性风险,其余为非系统性风险


多因素
\begin{gather*}
  E(r_i) = r_f + \sum_{j=1}^{k}{\beta_{ij} [E(r_{F_j})- r_f]} +\varepsilon_i
\end{gather*}
\par $E(r_i)$:预期收益率
\par $\beta_{ij}$:第i种股票收益率对因素$F_j$的敏感性
\par $F_j$:第j种因素(j=1,2,…,k)
\par $r_f$:无风险收益率
\par $\varepsilon_i$:第i种股票发行公司因素引起的收益波动





\subparagraph{Fama-French因子模型}

\subparagraph{假设}

\subparagraph{模型}	

\begin{gather*}
  E(R_i ) = R_F + \beta_{i1}(R_M-R_F) + \beta_{i2}R_{SMB} + \beta_{i3}R_{HML} + \varepsilon_{it}
\end{gather*}
\par $R_{SMB}$:市值因子,低市值-高市值(small minus big)
\par $R_{HML}$:账面市值比因子(high minus low)
\par $\sqrt{Var(\varepsilon_{it})}$:异质性风险(idiosyncratic risk)









\section{绝对风险评估}


\subsection{价格变动}


\subsubsection{情景分析}



\subsubsection{泰勒级数展开}

(Taylor Series Expansion)


\begin{align*}
    \Delta P& = \frac{\partial P}{\partial S}\Delta S+\frac{\partial P}{\partial t}\Delta t+\frac{\partial P}{\partial\sigma}\Delta\sigma+\frac{\partial P}{\partial r}\Delta r
    + \frac{1}{2} \frac{\partial^2P}{\partial S^2} \Delta S^2
    + \left( \frac{1}{2}\frac{\partial^{2}P}{\partial t^2}\Delta t^2 \right)
    + \left( \frac{\partial^{2}P}{\partial S\partial t}\Delta S\Delta t \right) +\cdots \\
    & = Delta\times\Delta S + Theta\times\Delta t+Vega\times\Delta\sigma+Rho\times\Delta r+\frac{1}{2}Gamma\times\Delta S^2+\cdots \\
    & = \Delta\times\Delta S+\Theta\times\Delta t+\nu \times\Delta\sigma+\rho\times\Delta r+\frac{1}{2}\Gamma\times\Delta S^2+\cdots 
\end{align*}
\par $\Delta P$:交易组合的价值变化
\par $\Delta S$:某一市场变量的变化
\par $Delta=\Delta =\frac{\partial P}{\partial S}$:$\Delta=0$则Delta neutral
\par $Gamma=\Gamma=\frac{\partial^2P}{\partial S^2}$:曲率(curvature)
\par $T$:时间。时间损耗(time decay)
\par $Theta=\Theta=\frac{\partial P}{\partial t}$:对于call或put期权通常为负
\par $\sigma$:标的资产价格波动率
\par $\Delta \sigma$:在$\Delta t$内波动率的变化
\par $Vega=\nu=\frac{\partial P}{\partial\sigma}$:
\par $r$:利率。外汇期权有两个利率
\par $Rho=\rho=\frac{\partial P}{\partial r}$:







\subsection{尾部风险:光谱型风险度量}

\subsubsection{VaR}


\subsubsection{ES}

\subsubsection{回顾测试}


\section{相对风险评估}


\subsection{相关性}



\subsection{波动率}




\section{均衡:资产组合理论}




















\chapter{金融市场}



 

\section{货币市场}




\section{证券市场}




\section{衍生品市场}

\chapter{金融机构}




\chapter{公司金融}




\chapter{家庭金融}


 






\chapter{金融监管}









\chapter{剩余价值理论}




剩余价值规律:剩余价值产生及其增殖的规律。资本主义的生产目的和动机是追求尽可能多的剩余价值,达到这一目的的手段是不断扩大和加强对雇佣劳动的剥削

基本矛盾:生产社会化与生产资料私有在资源配置效率上的不协调,是市场经济的局限性所在




\section{资本主义生产}

\informationBox{
    \begin{enumerate}[(1)]
        \item 实质:剩余价值的生产。不是获取使用价值。具有二重性,资本主义生产过程是劳动过程和价值增殖过程的统一
        \item 目的:剩余价值的增殖,或利润的最大化
        \item 条件:资本积累为资本主义生产奠定物质基础;劳动力成为商品(雇佣工人)
    \end{enumerate}
}





\subsection{资本总公式:$G-W-G'$}

\par 资本周转:资本持续不断的、周而复始的循环运动
\begin{enumerate}[(1)]
    \item 产业资本公式:$G-W(Pm,A)\cdots P\cdots W′-G′$
    \item 商业资本公式:$G-(W-)G'$
    \item 借贷资本公式:$G-G'$
\end{enumerate}


\subsubsection{产业资本公式}



\paragraph{模型:社会简单再生产}


1、假设:
\informationBox{
    * 社会剩余产品是用于消费而不是用于积累的,生产在维持原来的规模上重复进行
    \begin{enumerate}[(1)]
        \item 生产周期为一年
        \item 全部生产资料价值和消费资料在一个生产周期内一次性消耗掉
        \item 没有外贸和储备
    \end{enumerate}
}


2、社会总产出的实现:
\begin{gather*}
    \begin{cases}
        I(c+v+m)=I c+II c \\
        II (c+v+m)=I(v+m)+II(v+m)
    \end{cases}\\
    I(v+m)=IIc
\end{gather*}

\informationBox{
    \par * 社会生产部门:两大部类
    \par $I$:第一部类,生产生产资料的部门
    \par $II$:第二部类,生产消费资料的部门
    \par * 价值形态:三个部分
    \par $c$:产品中生产资料的转移价值。由具体劳动实现
    \par $m$:工人在剩余劳动时间里创造的剩余价值,成本价格小于商品的价值。活劳动创造新价值的过程由抽象劳动实现
    \par $v$:工人必要劳动创造的价值
    \par $c+v$:商品的生产成本或成本价格
    \par $c+v+m$:社会总产品,社会在一定时期内(通常为一年)所生产的全部物质资料的总和
    \par * 第一部类产出中用$I c$表示的部分,可以通过第一部类内部的相互交换而实现。第二部类产出中用$II(v+m)$表示的部分,可以通过第二部类内部的相互交换而实现。第一部类产出中用$I(v+m)$表示的部分,可以通过与第二部类产出中用$IIc$表示的部分相交换而实现,资本从商品形态向货币形态的转化。第一部类的消费需求得到补偿,第二部类的生产耗费得到替换。社会再生产的实现过程的实质在于两大部类之间能否保持一个平衡发展关系
}



\paragraph{模型:社会扩大再生产}


1、假设:
\informationBox{
    * 社会生产在社会总资本循环运动中不断扩大规模
    \begin{enumerate}[(1)]
        \item 内涵扩大再生产:依靠生产技术进步、提高劳动效率以及改善生产要素质量来扩大生产规模
        \item 外延扩大再生产:在生产技术、劳动效率和生产要素(生产资料和劳动力)质量不变的情况下,依靠增加生产要素数量来扩大生产规模
    \end{enumerate}
    * 此处为外延扩大再生产
    \begin{enumerate}[(1)]
        \item 生产周期为一年
        \item 全部生产资料价值和消费资料在一个生产周期内一次性消耗掉
        \item 没有外贸和储备
        \item 追加资本的资本有机构成与原资本的有机构成相一致
    \end{enumerate}
}


2、社会总产出的实现:
\begin{gather*}
    \begin{cases}
        I(c+v+m)>I(c)+II(c) \\
        II [c+(m-m/x)]>I(v+m/x)
    \end{cases}\\
    \begin{cases}
        I(c+v+m)=I(c+\Delta c)+II(c+\Delta c) \Leftrightarrow  I(v+m)=II(c+\Delta c)+I\Delta c \\
        II(c+v+m)=I(v+\Delta v+m/x)+II(v+\Delta v+m/x) \Leftrightarrow II[c+(m-m/x)]=I(v+\Delta v+m/x)+II\Delta v
    \end{cases}\\
    I(v+\Delta v+m/x)=II(c+\Delta c)
\end{gather*}


\informationBox{
    \begin{enumerate}
        \item 使用价值替换:社会总资本运动正常运行的关键。买不到生产资料和工人需要的消费品,社会再生产就不能正常进行
        \item 价值补偿:社会总资本运动正常进行的基础。产品卖不出去,得不到价值补偿,资金不能回笼,社会再生产就不能正常进行
    \end{enumerate}
}





























\paragraph{理论}


1、资本周转:资本持续不断的、周而复始的循环运动
\informationBox{
    产业资本循环:产业资本从货币资本的职能形式出发,顺次经过购买、生产、销售三个阶段,分别地采取货币资本、生产资本、商品资本三种职能形式,实现了价值的增值,并回到原来出发点的全过程
    \begin{enumerate}[1.]
        \item 过程:一个生产阶段,两个流通阶段(买和卖)。是生产过程和流通过程的统一,在空间上同时并存,在时间上相互继起,是三个职能形式资本循环的统一
              \begin{enumerate}[(1)]
                  \item 购买阶段:货币资本
                  \item 生产阶段:生产资本
                  \item 销售阶段:商品资本
              \end{enumerate}
        \item 影响因素:$\text{预付资本的总周转速度}=\frac{\text{固定资本年周转价值总额}+\text{流动资本年周转价值总额}}{\text{预付资本总量}}$
              \begin{enumerate}[(1)]
                  \item 资本周转时间
                        \begin{enumerate}[a.]
                            \item 生产时间:劳动时间,劳动过程中的正常中断时间,生产要素的储备时间
                            \item 流通时间:
                                  \begin{enumerate}[(a)]
                                      \item 购买时间:受生产要素供应条件制约
                                      \item 售卖时间:受市场需求和竞争状况制约
                                  \end{enumerate}
                        \end{enumerate}
                  \item 资本周转速度
                        \begin{enumerate}[a.]
                            \item 固定资本的周转速度
                            \item 流动资本的周转速度
                                  \informationBox{
                                      \par * 固定资本:以厂房、机器设备、生产工具等劳动资料形式存在的那部分生产资本(劳动资料)
                                      \par (1) 折旧
                                      \par a. 定义:对固定资本价值转移量的计算以及从商品销售中逐步提取和回收这部分价值的方式
                                      \par b. 原因:损耗
                                      \par (a) 有形损耗(物质磨损):使用;自然力
                                      \par (b) 无形损耗(精神磨损):生产技术进步,生产效率提高,生产同类机器设备的社会必要劳动时间减少;科技创新,更好的机器设备
                                      \par * 流动资本:以原料、燃料及动力、辅助材料等劳动对象形式存在的以及用于购买劳动力的那部分生产资本(劳动对象,劳动者)
                                      \par * 固定资产:多个生产过程中价值分批转移,分批回收;较长的有效使用期内不必更新;回收期、周转期长
                                      \par * 流动资产:一个生产过程中价值一次全部转移,通过产品出售一次全部回收;每一生产周期前需更新;回收期、周转期短
                                  }
                        \end{enumerate}
                  \item 生产资本的构成:$\text{资本的有机构成}=\frac{v}{c}$
                        \begin{enumerate}[a.]
                            \item 可变资本:用于购买劳动力的资本
                            \item 不可变资本:以生产资料形式存在的资本(在生产过程中不改变自己的价值量,是流动资本的一部分)
                        \end{enumerate}
              \end{enumerate}
    \end{enumerate}
}

2、资本总公式矛盾:从形式上看,货币资本在流通中增殖与等价交换原则矛盾


\informationBox{
    \par 价值增殖或剩余价值是在生产过程中创造出来的。在流通领域,无论是等价交换还是不等价交换,都不能产生价值增殖,但是价值增殖的实现必须以流通为前提和手段。流通中的购买阶段为剩余价值的生产做准备,流通中的销售阶段实现剩余价值。因此,剩余价值不产生于流通过程,又离不开流通过程,必须以流通过程为媒介。
}

3、解决关键:劳动力成为商品
\informationBox{
    \par 价值增殖只能发生在G—W阶段所购买的商品上,这一商品必须具备特殊的使用价值,而这一特殊商品只能是劳动力
    \par * 条件
    \begin{enumerate}[(1)]
        \item 可能性:劳动者有人身自由
        \item 必要性:劳动者丧失一切生产资料和生活资料,只拥有自己的劳动力
    \end{enumerate}
    \par * 价值:劳动力商品的价值由生产和再生产劳动力所需要的社会必要劳动时间决定
    \begin{enumerate}[(1)]
        \item 构成:劳动者自身生存;繁育后代;劳动力接受教育和训练
        \item 特点:包含历史和道德的因素
    \end{enumerate}
    \par * 使用价值:劳动力商品使用价值的特殊性,在于它是价值和大于自身价值的源泉。正是资本家对这种特殊商品的购买,从而对劳动力使用价值的运用,生产出归属于资本家的产品,才使得资本家的货币转化为资本,从而实现价值增殖
}




\subsection{资本积累}


资本积累:剩余价值转化为资本,即剩余价值的资本化
\informationBox{
    \par 资本积累是资本扩大再生产的重要源泉
    \par 剩余价值(利润)是资本积累的唯一源泉
}



\subsubsection{资本积聚}

定义:单个资本依靠自身的积累来使实际资本在价值形态和生产要素形态上实现量的扩大

\begin{enumerate}[1.]
    \item 产品经营
          \begin{enumerate}[(1)]
              \item 定义:企业围绕企业的产品与服务等主要业务,进行生产(含服务)管理、产品改进、质量提高、市场开发等一系列活动
          \end{enumerate}
    \item 资本经营
          \begin{enumerate}[(1)]
              \item 定义:以价值形态的资本为经营对象,通过调整、交易、优化重组等方式,以实现资本价值量的保值增值的一系列市场行为
              \item 方式
                    \begin{enumerate}[a.]
                        \item 利用证券市场或其他形式的产权交易市场进行资产的收购、出售、托管、租赁等
                        \item 利用金融市场进行投机性交易(赚取差价,提高公司市值)
                        \item 对资产存量或所积累的资产增量进行调整(调整规模,方向,结构)
                        \item 无形资产经营
                        \item 风险投资
                    \end{enumerate}
              \item 特点
                    \begin{enumerate}[a.]
                        \item 高智力性:经营的具体条件复杂,需运用不同思路研究、策划和设计
                        \item 非生产性:有别于产品经营,本身不创造财富和产品
                        \item 高收益性:高智力劳动,高风险性
                    \end{enumerate}
          \end{enumerate}
    \item 关系
          \begin{enumerate}[(1)]
              \item 联系
                    \begin{enumerate}[a.]
                        \item 目的:价值增殖
                        \item 产品经营是资本经营的基础,资本经营是产品经营发展到一定阶段的必然趋势
                    \end{enumerate}
              \item 区别
                    \begin{enumerate}[a.]
                        \item 内容对象:产品经营对象为产品及其生产销售过程;资本经营对象为企业资本
                        \item 经营方式:
                        \item 经营市场:产品经营依托商品或服务市场;资本经营依托资本市场
                        \item 收益风险:产品经营波幅和缓,持续时间长;资本经营可能短期大幅升跌
                    \end{enumerate}
          \end{enumerate}
\end{enumerate}






\subsubsection{资本集中}

\begin{enumerate}[1.]
    \item 定义:把若干个规模相对较小的资本合并重组为规模较大的资本
    \item 途径:
          \begin{enumerate}[(1)]
              \item 并购(兼并、收购):部分并购、整体并购其影响不仅是数量,也有质量,是行业标准方面的
              \item 联合原有的、分散的单个资本联合成新的更大的资本
              \item 上市向社会发行股票等方式,把社会闲散资金集中起来转化为资本
          \end{enumerate}
    \item 竞争和信用是资本集中的强有力的杠杆
\end{enumerate}




\subsubsection{关系}

\begin{enumerate}[(1)]
    \item 联系
          \begin{enumerate}[a.]
              \item 都能使单个资本的规模增大
              \item 二者相互促进
          \end{enumerate}
    \item 区别
          \begin{enumerate}[a.]
              \item 自身前提:资本积累以剩余价值的积累为前提;资本集中不以积累为必要前提
              \item 社会前提:资本积聚的实现受社会所能提供的实际生产要素增长的制约;资本集中较少受限
              \item 自身影响:资本积聚扩大单个资本规模,一般速度较慢;资本集中可以快速扩大资本规模
              \item 社会影响:资本积聚在增大单个资本的同时,增大社会总资本;资本集中一般不能直接增大社会总资本,可以改变资本的结构和质量
          \end{enumerate}
\end{enumerate}








\section{资本主义交换/流通}



\subsection{商品经济}



定义:是以交换为目的、包括商品生产和商品交换的经济形式


\informationBox{
    \par * 与自然经济相对应(自然经济是自给自足),是生产力发展到一定阶段的产物
    \par * 特征
    \begin{enumerate}[(1)]
        \item 自主性:生产经营独立自主,经济利益独立
        \item 平等性:商品交换以等价交换原则为基础
        \item 竞争性
        \item 开放性:商品经济以社会分工为基础;生产者间的经济联系紧密,范围扩大
    \end{enumerate}
    \par * 前提
    \begin{enumerate}[(1)]
        \item 社会分工(必要性):社会分工→生产专业化→交换需求
        \item 剩余产品的出现,并分属于不同的生产者所有(可能性)
    \end{enumerate}
    \par * 阶段
    \begin{enumerate}[(1)]
        \item 简单商品经济(小商品经济)
              \begin{enumerate}[a.]
                  \item 商品经济的初始形式
                  \item 以个体私有制和个体劳动为基础
                  \item 以手工业劳动为技术特征
              \end{enumerate}
        \item 市场经济(社会化商品经济):市场在资源配置中起决定作用的经济
              \begin{enumerate}[a.]
                  \item 价值规律成为支配社会经济发展的基本规律
                  \item 市场在资源配置中起基础性作用
                  \item 是商品经济发展到一定阶段的产物
              \end{enumerate}
    \end{enumerate}
}








\subsection{规律}

\subsubsection{价值规律}


\begin{enumerate}
    \item 内容:商品的价值量由生产商品的社会必要劳动时间决定,以此为基础进行商品等价交换
          \begin{enumerate}[(1)]
              \item 价格是商品价值的货币表现
          \end{enumerate}
    \item 形式
          \begin{enumerate}[(1)]
              \item 价格受供求关系影响,围绕价值上下波动
              \item 价格背离价值的运动总是围绕价值上下波动的
          \end{enumerate}
    \item 作用
          \begin{enumerate}[(1)]
              \item 调节:自发调节生产资料和劳动力在社会各部门之间的分配(生产前)
              \item 刺激:刺激生产者的积极性(生产中)
              \item 分化:优胜劣汰,导致生产者两极分化(生产后)
          \end{enumerate}
\end{enumerate}








\subsubsection{资源配置规律}



资源配置:在经济运行过程中,各种现实的资源在社会不同部门之间的分配和不同方向上的使用
\\

\begin{enumerate}[1.]
    \item 市场配置
          \begin{enumerate}[(1)]
              \item 定义:通过市场机制发挥作用,促使资源的分配和流动
                    \begin{enumerate}[a.]
                        \item 市场机制:价格(基础),供求,竞争,风险
                        \item 看不见的手
                    \end{enumerate}
              \item 优点
                    \begin{enumerate}[a.]
                        \item  只要依靠市场的自发调节,通过市场主体依照一定规则进行的市场交易活动,就能自动实现资源的优化配置(价格机制、竞争机制、供求机制、风险机制)
                    \end{enumerate}
              \item 缺点
                    \begin{enumerate}[a.]
                        \item 市场机制具有自发性,市场主体的分散决策,难以自动地实现整个国民经济的发展战略和目标
                        \item 市场配置不能直接对需求总量和结构进行调控,会造成宏观经济总量和结构的失衡
                        \item 市场配置对于外部不经济的调控显得乏力(环境污染、生态破坏、公共物品受损等)
                        \item 市场机制会刺激生产经营者的短期行为,导致产业结构的失衡和资源的浪费(盲目性)
                        \item 市场作用机制异化,如垄断反过来抑制市场机制的正常作用,失业、商品积压等也是市场的产物
                    \end{enumerate}
          \end{enumerate}
    \item 计划配置
          \begin{enumerate}[(1)]
              \item 定义:通过指令性计划和政府的各种预算、投资等直接进行分配资源的分配和组织流动
                    \begin{enumerate}[a.]
                        \item 看得见的手
                    \end{enumerate}
              \item 优点
                    \begin{enumerate}[a.]
                        \item 国家对经济进行统一调度,有利于经济的稳定和收入差距的减小,更大的体现平均
                        \item 国家的高度干预,使得资本的流转以及分配的权利都集中到了国家的手中,国家掌握了经济的主动权。绝大多数的企业和工厂都按照国家的指示来进行各种生产活动,就使得经济关系变得更加的简单
                        \item 国家掌握了社会财富,有利于集中财力物力人力干大事业。有利于国家在一定的时间和程度上实现资源的高利用,建成一些利国利民的基础设施和大型的生产设备和基地
                        \item 在短期内激发民众的热情,迅速使国力提升
                        \item 制定战略规划,前瞩性地引导国民经济发展
                    \end{enumerate}
              \item 缺点
                    \begin{enumerate}[a.]
                        \item 通过指令性计划配置资源,一刀切,企业成为了政府的附属物,没有自主权,缺乏内在的动力、主动性和创造性
                        \item 激励机制不足,对劳动者缺乏奖励,工作动力和热情难以持续
                        \item 中央计划当局难以把握全面准确信息,难以制定符合客观实际的计划
                        \item 计划经济生产部门之间是计划调拨,一旦一个生产环节脱钩,就会造成链式反应,整个经济陷于停顿
                        \item 国家基本上取消了市场,直接导致了竞争的消失
                    \end{enumerate}

          \end{enumerate}
\end{enumerate}






\section{资本主义分配(转型理论)}


本质:剩余价值m(p)在资本家之间的分配。
\\





\begin{enumerate}[1.]
    \item 剩余价值→利润:$W=c+v+m\Rightarrow W=K(c+v)+m$
          \begin{enumerate}[(1)]
              \item 当不把剩余价值看作是雇佣工人剩余劳动的产物,而是把它看作是全部预付资本的产物或增加额时,剩余价值便转化为利润
              \item $K = c + v$:商品的生产成本或成本价格
              \item 成本价格掩盖了不变资本和可变资本之间的区别,掩盖了它们在价值增殖过程中的不同作用
          \end{enumerate}
    \item 利润→平均利润:
          \begin{enumerate}[(1)]
              \item 假设
                    \begin{enumerate}[a.]
                        \item 市场充分竞争,未形成垄断
                        \item 价格是灵敏的,进入和退出的机制是自由的
                    \end{enumerate}
              \item 模型:
                    \begin{gather*}
                        \text{平均利润率}=\frac{\text{剩余价值总额}}{\text{社会总资本}}=\frac{\text{社会剩余价值总额}}{\text{产业资本总额}+\text{商业资本总额}}\\
                        \text{平均利润}=\frac{\text{预付资本}}{\text{平均利润率}}
                    \end{gather*}
                    \begin{enumerate}[a.]
                        \item 平均利润:投入各个不同部门的等量资本获得的等量利润(不是绝对平均,而是一种趋势)
                        \item 社会总资本:以社会分工和市场交换为条件,相互联系、相互依存、相互制约的全社会各单个资本的总和(社会资本)
                    \end{enumerate}
              \item 理论:
                    \begin{enumerate}[a.]
                        \item 部门内:提高部门资本有机构成,利润率下降
                        \item 部门间:平均利润是不同部门之间竞争的结果,竞争的手段是进行资本转移。社会总资本在各部门之间的分配。投入高利润部门资本比重越大,平均利润就高,反之则越低
                        \item 部门间竞争结果:不同部门的利润率出现平均化趋势
                    \end{enumerate}
          \end{enumerate}
    \item 价值→生产价格:$\text{生产价格}=K(\text{成本价格})+p(\text{平均利润})$
          \begin{enumerate}[(1)]
              \item 生产价格的形成是以平均利润率的形成为前提的。利润转化为平均利润,商品价值便转化为生产价格。此时,商品不再按照商品的价值出售,而是按照生产成本加平均利润的价格来出售了,即按照生产价格出售。
              \item 价值规律作用的形式也发生相应的变化。商品按生产价格销售,市场价格以生产价格为中心,受供求关系的影响而波动,所以,价值规律现不再是直接通过价值,而是通过生产价格起作用
              \item  对于个别,平均利润和剩余价值不相等,生产价格和价值不相等。对于全社会,总额相等
              \item 社会生产价格变动,最终取决于价值的变动
          \end{enumerate}
\end{enumerate}






\subsection{劳动力}









\subsubsection{工资}
定义:劳动力价值或价格的转化形式。本质是在资本主义社会里,资本家付给雇用工人的工资是劳动力的价值或价格,而不是劳动的价值或价格

\informationBox{
    * 劳动和劳动力的区别
    \begin{enumerate}[(1)]
        \item 劳动力是潜藏在人身体内的劳动能力。劳动是劳动力的使用,劳动力在生产中发挥作用时才是劳动
        \item 劳动不是商品,但劳动力在一定历史条件下则可以成为商品,它具有价值和使用价值
        \item 在资本家同工人的买卖关系中,工人出卖的是劳动力,而不是劳动,劳动根本不能成为商品
        \item 不成为商品的东西也就没有价值。所以工资只能是劳动力的价值或价格,而不是劳动的价格或价值。
    \end{enumerate}
    * 劳动不能成为商品,所以没有价值
    \begin{enumerate}[(1)]
        \item 如果劳动是商品,具有价值,价值又是凝结在商品中的无差别人类劳动,劳动的价值由劳动决定,同义反复
        \item 如果劳动是商品,出卖前就应该独立存在
        \item 如果劳动是商品,不等价交换违反价值规律,等价交换就无法获得剩余价值,否定资本主义生产关系存在的基础
        \item 如果劳动是商品,就等于说雇佣工人出卖了不属于自己的商品,因为劳动此时已归资本家所有
    \end{enumerate}
    * 资本主义工资在现象上表现为劳动的价值或价格是由资本主义生产关系决定的
    \begin{enumerate}[(1)]
        \item 从资本和劳动的交换关系来看,劳动力的买卖和其他商品的买卖一样,钱货两清
        \item 从工资的支付形式来看,资本家通常在劳动后支付工资,使得工资被看做劳动的价值或价格
    \end{enumerate}

}



基本形式
\begin{enumerate}[(1)]
    \item 计时工资:按一定的劳动时间来支付的工资,其实质是劳动力的月价值、周价值、日价值的转化形式
    \item 计件工资:按工人完成的产品数量或完成的工作量来支付的工资,它是计时工资的转化形式
\end{enumerate}





{\noindent
\begin{tabu}{|c|X|X|}
    \hline
                               & \multicolumn{1}{c|}{名义工资}                      & \multicolumn{1}{c|}{实际工资} \\\hline
    \multicolumn{1}{|c|}{定义} & \begin{tabu}{X}
        指工人出卖劳动力所得到的货币数量,即货币工资
    \end{tabu}
                               & \begin{tabu}{X}
        工人用货币工资实际买到的各类生活资料和服务的数量
    \end{tabu}                                                        \\\hline
    \multicolumn{1}{|c|}{对立} & \multicolumn{2}{l|}{\tablelist{(1)}{\columnwidth}{
            \item 在其他条件不变的情况下,名义工资越高,实际工资也就越高
    }}                                                                                                              \\\hline
    \multicolumn{1}{|c|}{统一} & \multicolumn{2}{l|}{\tablelist{(1)}{\columnwidth}{
            \item 两者常常不一致,即名义工资虽然不变甚至提高,实际工资却可能降低(原因:实际工资的多少不仅取决于名义工资的高低,而且还取决于物价的高低)
            \item 趋势:名义工资一般呈增加趋势。但是,实际工资的提高并不意味着工人受剥削程度的减轻(原因:相对工资(m+v)呈下降趋势);从资本主义发展的历史过程来看,实际工资有时降低有时提高
    } }                                                                                                             \\\hline
\end{tabu}
}











\subsection{资本}



\begin{equation*}
    \text{资本分配}\mindmap{
        \text{职能资本}\mindmap{
            \text{产业资本}\mindmap{
                \text{货币资本}\\
                \text{生产资本}\\
                \text{商品资本}
            }
            \text{商业资本}
        }
        \text{非职能资本}\mindmap{
            \text{生息资本}\mindmap{
                \text{高利贷资本}\\
                \text{借贷资本}
            }
        }
    }
\end{equation*}


高利贷资本是生息资本的古老形式,借贷资本是生息资本的资本主义形式。
\\



\begin{enumerate}[1.]
    \item 产业资本
          \begin{enumerate}[(1)]
              \item 功能:生产剩余价值
          \end{enumerate}
    \item 商业资本
          \begin{enumerate}[(1)]
              \item 定义:从产业资本中分离出来的专门从事商品买卖的,以获取商业利润为目的的资本
              \item 功能:实现剩余价值
              \item 特点:
                    \begin{enumerate}[a.]
                        \item 不直接增加剩余价值
                        \item 只在流通领域
                        \item 受生产和消费的限制
                    \end{enumerate}
              \item 成本:费用
                    \begin{enumerate}[a.]
                        \item 生产性流通费用:商品的分类、包装、保管和运输支出的费用。创造价值和剩余价值
                        \item 纯粹流通费用:用于商业簿记、邮资、通信、广告及商业职工的工资等的费。不生产价值和剩余价值
                    \end{enumerate}
              \item 利润
                    \begin{enumerate}[a.]
                        \item 本质:产业工人创造的、由产业资本家让渡给商业资本家的一部分剩余价值。从事商品买卖所获得的利润是产业资本家按照低于生产价格的价格把商品让渡给商业资本家,然后商业资本家再按照生产价格把商品卖给消费者。这种售价大于买价之差,就是商业资本家所获得的商业利润
                    \end{enumerate}
          \end{enumerate}
    \item 借贷资本
          \begin{enumerate}[(1)]
              \item 定义:从职能资本循环中独立出来的一种特殊资本形式,是为了取得利息而暂时借给另一个资本家使用的货币资本
              \item 借贷利息
                    \begin{enumerate}[a.]
                        \item 0<利息率<平均利润率
                        \item 当借贷资本的供求平衡时,利息率只能由社会习惯和法律等因素决定
                        \item 本质:产业工人创造的、由职能资本家让渡给借贷资本家的一部分剩余价值的特殊转化形式
                    \end{enumerate}
              \item 分类:银行资本
          \end{enumerate}
\end{enumerate}
























\subsection{土地}




\subsubsection{出租:地租}

\par 定义:土地所有者凭借土地所有权获得的一种非劳动收入
\par 本质:由农业工人创造的、被农业资本家让渡给土地所有者的超过平均利润以上的那部分剩余价值



\begin{enumerate}[1.]
    \item 绝对地租
          \begin{enumerate}[(1)]
              \item 定义:由于土地私权的存在,农业资本家租用任何土地都必须交纳的地租
              \item 实质:是农产品价值高于社会生产价格的差额,是由农业雇佣工人创造的剩余价值的一部分转化而来的,它体现的仍然是农业资本家和土地所有者对雇佣工人的剥削关系
              \item 形成原因:土地私有权的垄断(资本主义土地所有权的垄断)
              \item 形成条件:农业资本有机构成低于社会平均资本有机构成
                    \begin{enumerate}[a.]
                        \item 农业通常是劳动密集型的产业。资本有机构成比较低,因此,剩余价值比较高
                        \item 由于土地所有权的垄断,阻碍了资本向农业转移。高于平均利润的部分被留在农业部门,被土地所有者占有,形成绝对地租
                        \item 土地有限性:和工业品不同,农产品社会生产价格是由劣等地的生产条件决定的
                    \end{enumerate}
          \end{enumerate}
    \item 级差地租
          \begin{enumerate}[(1)]
              \item 定义:与土地等级(土地的优劣)相联系的地租形式
                    \begin{enumerate}[a.]
                        \item 土地自然生产力产生了资本生产率的等级差别
                        \item 是农产品的个别生产价格低于社会生产价格的差额
                        \item 级差地租I是级差地租II的基础
                    \end{enumerate}
              \item 级差地租I
                    \begin{enumerate}[a.]
                        \item  定义:由于土地肥沃程度和地理位置等不同而产生的级差地租
                    \end{enumerate}
              \item 级差地租II
                    \begin{enumerate}[a.]
                        \item 定义:由于在同一块土地上连续追加投资的资本生产率不同而产生的级差地租
                        \item 形成原因:土地的有限性所引起的土地经营的资本主义垄断(资本主义土地经营权的垄断)
                        \item 源泉:优等地和中等地的农业雇佣工人创造的超额利润(剩余价值)。土地数量有限,优等中等农产品供不应求。工业则只有优等。农产品市场价格是由劣等地的生产条件决定的
                    \end{enumerate}
          \end{enumerate}
\end{enumerate}











\subsubsection{出卖}

\begin{gather*}
    \text{价格}=\frac{\text{地租额}}{\text{利息率}}
\end{gather*}


土地价格是地租收入的资本化,它相当于这样一笔资本,把它存入银行,每年得到的利息与凭借土地得到的地租相等



\chapter{AD-AS模型}\label{chapter:AD-AS模型}


\section{AD:IS-LM-BP模型}

IS-LM模型:新古典综合学派



\subsection{假设}
国民经济均衡是商品市场和货币市场的同时均衡

\begin{enumerate}[1.]
    \item $\overline{T}$
    \item $\overline{G}$
    \item $\overline{M}_s$ 
\end{enumerate}



\subsection{模型}

\subsubsection{商品市场(IS)}

\begin{gather*}
    Y=PE=C(Y-\overline{T} )+I (r)+\overline{G}+NX(e)=C(Y-T)+I(i-\pi^e)+G+NX(e)
\end{gather*}
\par $Y$:实际支出;$PE$:计划支出($D$:总需求)
\par $Y$:总收入
\par $T$:税收
\par $Y-T$:可支配收入
\par $I$:计划投资
\par $G$:政府购买



\begin{tikzpicture}
    %%%%%%%%%  凯恩斯交叉  %%%%%%%%%
    \draw[eaxis] (7,7)--(12,7) node[below] {\small 收入,产出,$Y$};
    \draw[eaxis] (7,7)--(7,12) node[above] {\small 支出};
    \draw[elegant,black,domain=7:11.5] plot(\x,{
        \x
        }) node[right]{\small 实际支出};
    \draw[elegant,black,domain=7:11.5] plot(\x,{
        0.3*(\x-7)+9.1
        }) node[right]{\small$PE_1$};
    \draw[elegant,black,domain=7:11.5] plot(\x,{
        0.3*(\x-7)+8.05
        }) node[right]{\small$PE_2$};
    \draw[decorate,decoration={brace,raise=3pt}] (7,8.05)--(7,9.1);
    \draw[dashed] (6.8,8.5)--(6.8,8.5) node[left]{\small$\Delta I$};
    \draw[dashed] (8.5,8.5)--(8.5,7) node[below]{\small$Y_2$};
    \draw[dashed] (10,10)--(10,7) node[below]{\small$Y_1$};
    \draw[dashed] (9.25,7)--(9.25,7) node[below]{$\leftarrow $};
    \draw (8,8.75)--(8,8.75) node[left]{$\downarrow$};
    \draw[dashed] (9.25,6)--(9.25,6) node[below]{(a) 凯恩斯交叉};
    % angle
    \coordinate (o) at (7,7);
    \coordinate (a) at (7.05,7);
    \coordinate (b) at (7.025,7.025);
    \pic["$45^\circ $", draw=black, -, angle eccentricity=1.7, angle radius=0.5cm]
    {angle=a--o--b};
    %%%%%%%%%  投资函数  %%%%%%%%%
    \draw[eaxis] (0,0)--(5,0) node[below] {\small 投资,$I$};
    \draw[eaxis] (0,0)--(0,5) node[above] {\small 利率,$r$};
    \draw[elegant,black,domain=0.5:4.5] plot(\x,{
        -2*\x^0.6+5.5
        }) node[right]{\small$I(r)$};
    \draw[dashed] (1.5,-2*1.5^0.6+5.5)--(1.5,0) node[below]{\small$I(r_2)$};
    \draw[dashed] (3,-2*3^0.6+5.5)--(3,0) node[below]{\small$I(r_1)$};
    \draw[decorate,decoration={brace,raise=3pt}] (1.5,0)--(3,0);
    \draw[dashed] (2.25,0.2)--(2.25,0.2) node[above]{\small$\Delta I$};
    \draw[dashed] (6.5,-2*1.5^0.6+5.5)--(0,-2*1.5^0.6+5.5) node[left]{\small$r_2$};
    \draw[dashed] (6.5,-2*3^0.6+5.5)--(0,-2*3^0.6+5.5) node[left]{\small$r_1$};
    \draw (2.25,0)--(2.25,0) node[below]{$\leftarrow $};
    \draw (0,2.3)--(0,2.3) node[left]{$\uparrow$};
    \draw (2.25,-1)--(2.25,-1) node[below]{(b) 投资函数};
    %%%%%%%%%  IS曲线  %%%%%%%%%
    \draw[eaxis] (7,0)--(12,0) node[below] {\small 收入,产出,$Y$};
    \draw[eaxis] (7,0)--(7,5) node[above] {\small 利率,$r$};
    \draw[elegant,black,domain=7.5:11.5] plot(\x,{
                    -2*(\x-7)^0.6+5.5
                }) node[right]{\small$IS$};
    \draw[dashed] (8.5,6.5)--(8.5,0) node[below]{\small$Y_2$};
    \draw[dashed] (10,6.5)--(10,0) node[below]{\small$Y_1$};
    \draw[dashed] (8.5,-2*1.5^0.6+5.5)--(7,-2*1.5^0.6+5.5) node[left]{\small$r_2$};
    \draw[dashed] (10,-2*3^0.6+5.5)--(7,-2*3^0.6+5.5) node[left]{\small$r_1$};
    \draw (9.25,0)--(9.25,0) node[below]{$\leftarrow $};
    \draw (7,2.3)--(7,2.3) node[left]{$\uparrow$};
    \draw (9.25,-1)--(9.25,-1) node[below]{(c) IS曲线};
\end{tikzpicture}
    
\begin{gather*}
    r\uparrow 
    \Rightarrow I\downarrow
    \Rightarrow PE\downarrow
    \Rightarrow Y\downarrow
\end{gather*}


产品市场:购买力平价
\begin{gather*}
    D=Y=C(Y-T)+I(r^*)+G+CA(eP^*/P)=D(eP^*/P,Y-T,I,G)
\end{gather*}


\begin{tikzpicture}
    %%%%%%%%%  凯恩斯交叉  %%%%%%%%%
    \draw[eaxis] (7,7)--(12,7) node[below] {\footnotesize 收入,产出,$Y$};
    \draw[eaxis] (7,7)--(7,12) node[above] {\footnotesize 支出};
    \draw[elegant,black,domain=7:11.5] plot(\x,{
        \x
        }) node[right]{\footnotesize 实际支出};
    \draw[elegant,black,domain=7:11.5] plot(\x,{
        0.3*(\x-7)+9.1
        }) node[right]{\footnotesize $PE_1$};
    \draw[elegant,black,domain=7:11.5] plot(\x,{
        0.3*(\x-7)+8.05
        }) node[right]{\footnotesize $PE_2$};
    \draw[decorate,decoration={brace,raise=3pt}] (7,8.05)--(7,9.1);
    \draw[dashed] (6.8,8.5)--(6.8,8.5) node[left]{\footnotesize $\Delta NX$};
    \draw[dashed] (8.5,8.5)--(8.5,7) node[below]{\footnotesize $Y_2$};
    \draw[dashed] (10,10)--(10,7) node[below]{\footnotesize $Y_1$};
    \draw[dashed] (9.25,7)--(9.25,7) node[below]{$\leftarrow $};
    \draw (8,8.75)--(8,8.75) node[left]{$\downarrow$};
    \draw[dashed] (9.25,6)--(9.25,6) node[below]{(1) 凯恩斯交叉};
    % angle
    \coordinate (o) at (7,7);
    \coordinate (a) at (7.05,7);
    \coordinate (b) at (7.025,7.025);
    \pic["$45^\circ $", draw=black, -, angle eccentricity=1.7, angle radius=0.5cm]
    {angle=a--o--b};
    %%%%%%%%%  净出口曲线  %%%%%%%%%
    \draw[eaxis] (0,0)--(5,0) node[below] {\footnotesize 净出口,$NX$};
    \draw[eaxis] (0,0)--(0,5) node[above] {\footnotesize 汇率,$e$};
    \draw[elegant,black,domain=0.5:4.5] plot(\x,{
        0.4*(\x)^1.5+1
        }) node[right]{\footnotesize $NX(e)$};
    \draw[dashed] (1.5,0.4*1.5^1.5+1)--(1.5,0) node[below]{\footnotesize $NX(e_2)$};
    \draw[dashed] (3,0.4*3^1.5+1)--(3,0) node[below]{\footnotesize $NX(e_1)$};
    \draw[decorate,decoration={brace,raise=3pt}] (1.5,0)--(3,0);
    \draw[dashed] (2.25,0.2)--(2.25,0.2) node[above]{\footnotesize $\Delta NX$};
    \draw[dashed] (6.5,0.4*1.5^1.5+1)--(0,0.4*1.5^1.5+1) node[left]{\footnotesize $e_2$};
    \draw[dashed] (6.5,0.4*3^1.5+1)--(0,0.4*3^1.5+1) node[left]{\footnotesize $e_1$};
    \draw (2.25,0)--(2.25,0) node[below]{$\leftarrow $};
    \draw (0,2.4) node[left]{$\downarrow$};
    \draw (2.25,-1)--(2.25,-1) node[below]{(2) 净出口曲线};
    %%%%%%%%%  IS曲线  %%%%%%%%%
    \draw[eaxis] (7,0)--(12,0) node[below] {\footnotesize 收入,产出,$Y$};
    \draw[eaxis] (7,0)--(7,5) node[above] {\footnotesize 汇率,$e$};
    \draw[elegant,black,domain=7.5:11.5] plot(\x,{
        0.4*(\x-7)^1.5+1
                }) node[right]{\footnotesize$IS^*$};
    \draw[dashed] (8.5,6.5)--(8.5,0) node[below]{\footnotesize $Y_2$};
    \draw[dashed] (10,6.5)--(10,0) node[below]{\footnotesize $Y_1$};
    \draw[dashed] (8.5,0.4*1.5^1.5+1)--(7,0.4*1.5^1.5+1) node[left]{\footnotesize $e_2$};
    \draw[dashed] (10,0.4*3^1.5+1)--(7,0.4*3^1.5+1) node[left]{\footnotesize $e_1$};
    \draw (9.25,0)--(9.25,0) node[below]{$\leftarrow $};
    \draw (7,2.4) node[left]{$\downarrow$};
    \draw (9.25,-1)--(9.25,-1) node[below]{(3) $IS^*$曲线};
\end{tikzpicture}



直接标价法
\begin{gather*}
    e\downarrow 
    \Rightarrow NX\downarrow
    \Rightarrow PE\downarrow
    \Rightarrow Y\downarrow
\end{gather*}







\par IS下方:供给<需求:短缺经济
\par IS上方:供给>需求:过剩经济


\clearpage
\subsubsection{货币市场(LM)}

资产分析法(流动性偏好理论)
\begin{gather*}
    \left(\frac{M}{P}\right)^s= \frac{\overline{M}}{\overline{P}} ={\left(\frac{M}{P}\right)}^d=L(i,Y) 
\end{gather*}





\begin{tikzpicture}
    %%%%%%%%%  实际货币余额市场  %%%%%%%%%
    \draw[eaxis] (0,0)--(5,0) node[below] {实际货币余额,$M/P$};
    \draw[eaxis] (0,0)--(0,5) node[above] {利率,$r$};
    \draw[elegant,black,domain=0.5:4.5] plot(\x,{
            -2*\x^0.45+5.3
        }) node[right]{\small$L(r,Y_2)$};
    \draw[elegant,black,domain=0.4:4] plot(\x,{
            -2*\x^0.45+4
        }) node[right]{\small$L(r,Y_1)$};
    \draw[elegant,black] (2,4.5)--(2,0) node[below]{\small$\overline{M}/\overline{P}$};
    \draw[dashed] (6.5,0.4*1.5^1.5+0.5)--(0,0.4*1.5^1.5+0.5) node[left]{\small$r_1$};
    \draw[dashed] (6.5,0.4*3^1.5+0.5)--(0,0.4*3^1.5+0.5) node[left]{\small$r_2$};
    \draw (0,2) node[left]{$\uparrow$};
    \draw (1.8,2) node[left]{$\rightarrow$};
    \draw (2.25,-1) node[below]{(1) 实际货币余额市场};
    %%%%%%%%%  LM曲线  %%%%%%%%%
    \draw[eaxis] (7,0)--(12,0) node[below] {收入,产出,$Y$};
    \draw[eaxis] (7,0)--(7,5) node[above] {利率,$r$};
    \draw[elegant,black,domain=7.5:11.5] plot(\x,{
            0.4*(\x-7)^1.5+0.5
        }) node[right]{\small$LM$};
    \draw[dashed] (8.5,0.4*1.5^1.5+0.5)--(8.5,0) node[below]{\small$Y_1$};
    \draw[dashed] (10,0.4*3^1.5+0.5)--(10,0) node[below]{\small$Y_2$};
    \draw[dashed] (8.5,0.4*1.5^1.5+0.5)--(7,0.4*1.5^1.5+0.5) node[left]{\small$r_1$};
    \draw[dashed] (10,0.4*3^1.5+0.5)--(7,0.4*3^1.5+0.5) node[left]{\small$r_2$};
    \draw (7,2) node[left]{$\uparrow$};
    \draw (9.25,-1)--(9.25,-1) node[below]{(2) LM曲线};
\end{tikzpicture}


\begin{gather*}
    \text{收入}\uparrow
    \Rightarrow L(r,Y)\uparrow
    \Rightarrow r\uparrow
\end{gather*}



\clearpage
\subsubsection{国际市场BP}




外汇市场:货币分析法(汇率:利率平价理论)
\begin{gather*}
    r-r^*=(f-e)/e\\
    NX(e)=CF(r)
\end{gather*}
\par $r$:实际利率
\par $r^*$:世界利率(world interest rate)



\paragraph{理论:利率平价理论}










\paragraph{模型}



{\noindent
\begin{tikzpicture}
    % 资本完全流动
    \draw[eaxis] (0,0) -- (4,0) node[below] {\footnotesize 收入,产出,$Y$};
    \draw[eaxis] (0,0) -- (0,4) node[above] {\footnotesize 利率,$r$};
    \draw[elegant,black,domain=0.5:3] plot(\x,{
            0.7*\x^1.2+0.34
        }) node[right]{\footnotesize $LM$};
    \draw[elegant,black,domain=0:3.5] plot(\x,{
            1.92
        }) node[right]{\footnotesize $BP$};
    \draw[dashed] (2,1.92)--(2,-2.5);
    \draw (0,1.92) node[left]{$r=r^*$};
    % 资本完全不流动
    \draw[eaxis] (5,0) -- (9,0) node[below] {\footnotesize $Y$};
    \draw[eaxis] (5,0) -- (5,4) node[above] {\footnotesize $r$};
    \draw[elegant,black,domain=5.5:8] plot(\x,{
        0.7*(\x-5)^1.2+0.5
        }) node[right]{\footnotesize $LM$};
    \draw[elegant,black] (6.8,0)--(6.8,3.5) node[above]{\footnotesize $BP$};
    \draw (7,-6) node[above]{\footnotesize (2) 资本完全不流动};

    %%%  资本完全流动
    \draw[eaxis] (0,-5) -- (0,4-5) node[above] {\footnotesize 汇率,$e$};
    \draw[eaxis] (0,-5) -- (4,-5) node[below] {\footnotesize $Y$};
    \draw[elegant,black] (2,-5)--(2,-1.5) node[right]{\footnotesize $LM^*$};
    \draw (2,-6) node[above]{\footnotesize (1) 资本完全流动};
%%%  资本不完全流动
    %  BP'<LM'
    \draw[eaxis] (0,-5*2-1)--(0,4-5*2-1) node[above] {\footnotesize $r$};
    \draw[eaxis] (0,-5*2-1)--(4,-5*2-1) node[below] {\footnotesize $Y$};
    \draw[elegant,black,domain=0.5:3] plot(\x,{
        0.7*\x^1.2+0.5-5*2-1
        }) node[right]{\footnotesize $LM$};
    \draw[elegant,black,domain=0.5:3.5] plot(\x,{
            0.4*\x^1.1+1.15-5*2-1
        }) node[right]{\footnotesize $BP$};
    \draw (2,-12) node[above]{\footnotesize (3a) 资本不完全流动a};
    % BP'=LM'
    \draw[eaxis] (5,-5*2-1)--(5,4-5*2-1) node[above] {\footnotesize $r$};
    \draw[eaxis] (5,-5*2-1)--(4+5,-5*2-1) node[below] {\footnotesize $Y$};
    \draw[elegant,black,domain=5.5:8] plot(\x,{
        0.7*(\x-5)^1.2+0.5-5*2-1
        }) node[right]{\footnotesize $LM, BP$};
    \draw (7,-12) node[above]{\footnotesize (3b) 资本不完全流动b};
    %  BP'>LM'
    \draw[eaxis] (10,-5*2-1) -- (10,4-5*2-1) node[above] {\footnotesize $r$};
    \draw[eaxis] (10,-5*2-1) -- (14,-5*2-1) node[below] {\footnotesize $Y$};
    \draw[elegant,black,domain=10.5:13] plot(\x,{
        0.7*(\x-10)^1.2+0.5-5*2-1
        }) node[right]{\footnotesize $LM$};
    \draw[elegant,black,domain=10.5:13] plot(\x,{
            0.8*(\x-10)^1.5-5*2-1
        }) node[right]{\footnotesize $BP$};
    \draw (12,-12) node[above]{\footnotesize (3c) 资本不完全流动c};
\end{tikzpicture}
}




\clearpage
\subsubsection{均衡}
\begin{gather*}
    IS: Y=C(Y-T)+I(i)+G+NX(e)=C(Y-T)+I(i)+G+CF(r)\\ 
    LM: \frac{M}{P}=L(r,Y) 
\end{gather*}


{\noindent
\begin{tikzpicture}
    % 资本完全流动:小型IS-LM-BP
    \draw[eaxis] (0,0) -- (4,0) node[below] {\footnotesize 收入,产出,$Y$};
    \draw[eaxis] (0,0) -- (0,4) node[above] {\footnotesize 利率,$r$};
    \draw[elegant,black,domain=0.5:3.5] plot(\x,{
            -3*\x^0.3+5.5
        }) node[right]{\footnotesize $IS$};
    \draw[elegant,black,domain=0.5:3] plot(\x,{
            0.7*\x^1.2+0.5
        }) node[right]{\footnotesize $LM$};
    \draw[elegant,black,domain=0:3.5] plot(\x,{
            1.92
        }) node[right]{\footnotesize $BP$};
    \draw[dashed] (1.8,1.92)--(1.8,0) node[below]{\footnotesize 均衡收入};
    \draw (0,1.92) node[left]{\footnotesize 均衡利率};
    %%%  资本完全流动:M-F
    \draw[eaxis] (0,-5) -- (0,4-5) node[above] {\footnotesize $e$};
    \draw[eaxis] (0,-5) -- (4,-5) node[below] {\footnotesize $Y$};
    \draw[elegant,black,domain=0.5:3] plot(\x,{
        0.7*\x^1.2+0.5-5
        }) node[right]{\footnotesize $IS^*$};
    \draw[elegant,black] (1.8,-5)--(1.8,-1.5) node[right]{\footnotesize $LM^*$};
    \draw[dashed,domain=1.8:0] plot(\x,{
        -3.1
        }) node[left]{\footnotesize 均衡汇率};
    \draw[dashed] (1.8,-1.5)--(1.8,-0.5);
    \draw (1.8,-5) node[below]{\footnotesize 均衡收入};
    \draw (2,-1.5-10) node[above]{\footnotesize (1a) 资本完全流动:小型};
    % 资本完全流动:小型AD
    \draw[eaxis] (0,-5*2) -- (4,-5*2) node[below] {\footnotesize $Y$};
    \draw[eaxis] (0,-5*2) -- (0,4-5*2) node[above] {\footnotesize $P$};
    \draw[elegant,black,domain=0.5:3.5] plot(\x,{
        -3*(\x)^0.3+5.5-5*2
        }) node[right]{\footnotesize $AD$};
    \draw[dashed] (1.8,-0.5-5)--(1.8,-5-5) node[below]{\footnotesize 均衡收入};
    \draw[dashed] (1.8,1.9-5*2)--(0,1.9-5*2) node[left]{\footnotesize 均衡价格};
    %% 资本完全流动:大型IS-LM
    \draw[eaxis] (5,0) -- (4+5,0) node[below] {\footnotesize $Y$};
    \draw[eaxis] (5,0) -- (5,4) node[above] {\footnotesize $r$};
    \draw[elegant,black,domain=5.5:8.5] plot(\x,{
            -2*(\x-5)^0.3+4.3
        }) node[right]{\footnotesize $IS$};
    \draw[elegant,black,domain=5.5:8] plot(\x,{
            0.7*(\x-5)^1.2+0.5
        }) node[right]{\footnotesize $LM$};
    \draw[dashed] (5,1.92)--(10,1.92);
    \draw[dashed] (6.8,1.92)--(6.8,0) node[below]{\footnotesize 均衡收入};
    % CF
    \draw[eaxis] (5*2,0)--(5*2,4) node[above] {\footnotesize $r$};
    \draw[eaxis] (5*2,0)--(5*2+4,0) node[below] {\footnotesize $CF$};
    \draw[dashed] (10,1.92)--(11.55,1.92)--(11.55,-1.5);
    \draw[elegant,black,domain=10.5:13] plot(\x,{
        -(\x-10)+3.5
        }) node[right]{\footnotesize $CF(r)$};
    % NX
    \draw[eaxis] (5*2,0-5)--(5*2,4-5) node[above] {\footnotesize $e$};
    \draw[eaxis] (5*2,0-5)--(5*2+4,0-5) node[below] {\footnotesize $NX$};
    \draw[elegant,black] (11.55,-5)--(11.55,-1.5) node[right]{\footnotesize $CF$};
    \draw[elegant,black,domain=10.5:13] plot(\x,{
        -(\x-10)+3.5-5
        }) node[right]{\footnotesize $NX(e)$};
    \draw[dashed] (10,1.92-5)--(11.55,1.92-5);
    % DD-AA
    \draw[eaxis] (5,0-5) -- (4+5,0-5) node[below] {\footnotesize $Y$};
    \draw[eaxis] (5,0-5) -- (5,4-5) node[above] {\footnotesize $e$};
    \draw[elegant,black,domain=5.5:8.5] plot(\x,{
        -3*(\x-5)^0.3+5.5-5
        }) node[below]{\footnotesize $DD,IS^*$};
    \draw[elegant,black,domain=5.5:8] plot(\x,{
            0.7*(\x-5)^1.2+0.5-5
        }) node[above]{\footnotesize $AA,LM^*$};
    \draw[dashed] (5,1.92-5)--(10,1.92-5);
    \draw[dashed] (6.8,1.92-5)--(6.8,0-5);
    % 大型AD
    \draw[eaxis] (5,-5*2) -- (5+4,-5*2) node[below] {\footnotesize $Y$};
    \draw[eaxis] (5,-5*2) -- (5,4-5*2) node[above] {\footnotesize $P$};
    \draw[elegant,black,domain=5.5:8.5] plot(\x,{
        -3*(\x-5)^0.3+5.5-5*2
        }) node[right]{\footnotesize $AD$};
    \draw[dashed] (1.8+5,-5)--(1.8+5,-5-5) node[below]{\footnotesize 均衡收入};
    \draw[dashed] (1.8+5,1.9-5*2)--(0+5,1.9-5*2) node[left]{\footnotesize 均衡价格};
    \draw (10,-1.5-10) node[above]{\footnotesize (1b) 资本完全流动:大型};
\end{tikzpicture}



\begin{tikzpicture}
    % 资本完全不流动
    \draw[eaxis] (0,-5-1) -- (4,-5-1) node[below] {\footnotesize $Y$};
    \draw[eaxis] (0,-5-1) -- (0,4-5-1) node[above] {\footnotesize $r$};
    \draw[elegant,black,domain=0.5:3.5] plot(\x,{
        -3*(\x)^0.3+5.5-5-1
        }) node[right]{\footnotesize $IS$};
    \draw[elegant,black,domain=0.5:3] plot(\x,{
        0.7*(\x)^1.2+0.5-5-1
        }) node[right]{\footnotesize $LM$};
    \draw[elegant,black] (1.8,-5-1)--(1.8,3.5-5-1) node[above]{\footnotesize $BP$};
    \draw (2,-1-5-1) node[above]{\footnotesize (2) 资本完全不流动};
    %%%  资本不完全流动
    %  BP'<LM'
    \draw[eaxis] (0,-6-5*2-1) -- (0,-2-5*2-1) node[above] {\footnotesize $r$};
    \draw[eaxis] (0,-6-5*2-1) -- (4,-6-5*2-1) node[below] {\footnotesize $Y$};
    \draw[elegant,black,domain=0.5:3.5] plot(\x,{
        -3*\x^0.3+5.5-6-5*2-1
        }) node[right]{\footnotesize $IS$};
    \draw[elegant,black,domain=0.5:3] plot(\x,{
        0.7*\x^1.2+0.5-6-5*2-1
        }) node[right]{\footnotesize $LM$};
    \draw[elegant,black,domain=0.5:3.5] plot(\x,{
            0.4*\x^1.1+1.15-6-5*2-1
        }) node[right]{\footnotesize $BP$};
    \draw (2,-7-5*2-1) node[above]{\footnotesize (3a) 资本不完全流动a};
    % BP'=LM'
    \draw[eaxis] (5,-6-5*2-1) -- (5,-2-5*2-1) node[above] {\footnotesize $r$};
    \draw[eaxis] (5,-6-5*2-1) -- (4+5,-6-5*2-1) node[below] {\footnotesize $Y$};
    \draw[elegant,black,domain=5.5:8.5] plot(\x,{
        -3*(\x-5)^0.3+5.5-6-5*2-1
        }) node[right]{\footnotesize $IS$};
    \draw[elegant,black,domain=5.5:8] plot(\x,{
        0.7*(\x-5)^1.2+0.5-6-5*2-1
        }) node[right]{\footnotesize $LM, BP$};
    \draw (7,-7-5*2-1) node[above]{\footnotesize (3b) 资本不完全流动b};
    %  BP'>LM'
    \draw[eaxis] (10,-6-5*2-1) -- (10,-2-5*2-1) node[above] {\footnotesize $r$};
    \draw[eaxis] (10,-6-5*2-1) -- (14,-6-5*2-1) node[below] {\footnotesize $Y$};
    \draw[elegant,black,domain=10.5:13.5] plot(\x,{
        -3*(\x-10)^0.3+5.5-6-5*2-1
        }) node[right]{\footnotesize $IS$};
    \draw[elegant,black,domain=10.5:13] plot(\x,{
        0.7*(\x-10)^1.2+0.5-6-5*2-1
        }) node[right]{\footnotesize $LM$};
    \draw[elegant,black,domain=10.5:13] plot(\x,{
            0.8*(\x-10)^1.5-6-5*2-1
        }) node[right]{\footnotesize $BP$};
    \draw (12,-7-5*2-1) node[above]{\footnotesize (3c) 资本不完全流动c};
\end{tikzpicture}
}


\clearpage
均衡(A)下短期冲击(A→B)的长期调整(B→A):


\begin{enumerate}[1.]
    \item IS-LM-BP模型
    \begin{enumerate}[(1)]
        \item IS右移:财政政策
        \begin{gather*}
            A(\overline{Y})+ \left\{\begin{matrix} G\uparrow&\Rightarrow& B(Y\uparrow)\\ 
            T\downarrow &\Rightarrow&  B(C\uparrow\Rightarrow Y\uparrow) \end{matrix}\right\}  +B(r\uparrow) \Rightarrow B(Y\downarrow+r\downarrow) \Rightarrow A(\overline{Y}) 
        \end{gather*}
        \item  LM右移:货币政策
        \begin{gather*}
            A(\overline{Y})+M_{s}\uparrow\Rightarrow B(r\downarrow+Y\uparrow)\Rightarrow B(r\uparrow+Y\downarrow)\Rightarrow A(\overline{Y}) 
        \end{gather*}
    \end{enumerate}
    \item 蒙代尔-弗莱明模型(Mundell-Fleming model):资本完全流动,小型经济体
    \begin{enumerate}[(1)]
        \item 固定汇率制(货币政策无效)
        \begin{enumerate}[a.]
            \item IS*右移:财政政策( $G\uparrow /T\downarrow $ ),贸易政策(限制进口配额+关税)
            \begin{gather*}
                \left.\begin{matrix} A(\overline{Y})\\ A(\overline{Y})+G\uparrow/T\downarrow \end{matrix}\right\} \Rightarrow A(LM^*\text{右移}\Rightarrow Y\uparrow)\Rightarrow B(\overline{Y}) 
            \end{gather*}
            \item LM*右移:货币政策($ M_s\uparrow$ )
            \begin{gather*}
                A(\overline{Y})+M_s \uparrow \Rightarrow 
                \left\{\begin{matrix} 
                    B(e\uparrow \Rightarrow 套利\Rightarrow LM^*左移\Rightarrow Y\downarrow)\Rightarrow A(\overline{Y})\\ 
                    B(Y↑+e↓\Rightarrow EX↑) \Rightarrow B(\overline{Y})
                \end{matrix}\right.
            \end{gather*}
        \end{enumerate}
        \item 浮动汇率制(财政政策无效)
        \begin{enumerate}[a.]
            \item IS*右移:财政政策( $G↑/T↓$ ),贸易政策(限制进口配额+关税)
            \begin{gather*}
                A(\overline{Y})
                \Rightarrow \begin{Bmatrix} 
                    G\uparrow/T\downarrow
                    \Rightarrow e\downarrow
                    \Rightarrow NX\downarrow \\
                    \text{贸易限制}\Rightarrow NX\downarrow 
            \end{Bmatrix}
            \Rightarrow A(\overline{Y}) 
            \end{gather*}
            \item LM*右移:货币政策( $M_s\uparrow$ )
            \begin{gather*}
                A(\overline{Y})+M_s\uparrow \Rightarrow \text{资本流出}\Rightarrow e\uparrow\Rightarrow NX\uparrow\Rightarrow Y\uparrow \Rightarrow B(\overline{Y})  
            \end{gather*}
        \end{enumerate}
    \end{enumerate}
    \item DD-AA模型:资本完全流动,大型经济体
    \begin{enumerate}[(1)]
        \item 固定汇率制
        \begin{enumerate}[a.]
            \item P↑→X-M↓→D↓→Y↓(DD左移)
            \item P*→
            \item T↑→Y-T↓→D↓→Y↓
            \item I↑→D↑→Y↑
            \item G↑→D↑→Y↑
        \end{enumerate}
        \item 浮动汇率制
        \begin{enumerate}[a.]
            \item DD右移:财政政策($G\uparrow,T\downarrow$)
            \begin{align*}
                A(\overline{Y})+G\uparrow/T\downarrow
                &\Rightarrow 
                    B(Y\uparrow+r\uparrow
                    \Rightarrow CF\downarrow
                    \Rightarrow e\downarrow
                    \Rightarrow NX\downarrow) \\ 
               & \Rightarrow 
                    C(Y>\overline{Y}
                    \Rightarrow\pi^e \uparrow
                    \Rightarrow e^e\uparrow
                    \Rightarrow AA\text{左移}
                    \Rightarrow e\downarrow
                    \Rightarrow A(\overline{Y})
            \end{align*}
            \item AA右移:货币政策($M_s\uparrow$)
            \begin{align*}
                A(\overline{Y})
                &\Rightarrow 
                    B(Y\uparrow+r↓
                    \Rightarrow CF\uparrow
                    \Rightarrow e\uparrow
                    \Rightarrow NX\uparrow )
                \\&\Rightarrow 
                    C(P\uparrow
                    \Rightarrow DD\text{左移}+AA\text{左移}
                    \Rightarrow e\downarrow
                    \Rightarrow A(\overline{Y})+B(e)>A(e)
            \end{align*}
        \end{enumerate}    
    \end{enumerate}
\end{enumerate}





















\subsection{理论}

\begin{enumerate}[1.]
    \item 利率是既定的国民收入下由商品市场和货币市场共同决定的
    \item 政策有效性
          \begin{enumerate}[(1)]
              \item LM曲线越陡峭(利率对货币供给),IS曲线越平坦(投资对利率),货币政策越有效
              \item LM曲线越平坦(乘数效应大),IS曲线越陡峭(挤出效应小),财政政策越有效
                    \par 对LM的冲击比对IS的更普遍
          \end{enumerate}
    \item 挤出效应
          \begin{enumerate}[(1)]
              \item 政府购买乘数(government-purchases multiplier):
                    \begin{gather*}
                        \frac{\Delta Y}{\Delta G}=\frac{1}{1-MPC}>1 \\
                        \Delta G\rightarrow \Delta Y=Y_1-Y_2=\frac{\Delta G}{1-MPC}
                    \end{gather*}
                    \par G↑→i↑→挤出投资
              \item 税收乘数(tax multiplier):
                    \begin{gather*}
                        \frac{\Delta Y}{\Delta T}=-\frac{MPC}{1-MPC} \\
                        ΔT\rightarrow \Delta Y=Y_1-Y_2=\frac{\Delta T·MPC}{1-MPC}
                    \end{gather*}
          \end{enumerate}
\end{enumerate}




\section{AS}




\subsection{凯恩斯AS}


\subsection{古典AS}





\section{均衡}





\chapter{资产负债表}

衡量的是存量。

\section{企业资产负债表}



\section{央行资产负债表}

\section{国家资产负债}












\chapter{资金流量表}

衡量的是流量。




\section{企业现金流量表}



\section{国际收支平衡表}


定义:一定时期内一个国家(地区)和其他国家(地区)进行的全部经济交易的系统记录(statement for balance of payments,BOP)


\begin{tabu}{|c|c|c|c|}
    \hline
    账户名称 & 项目           & 借方debit-          & 贷方credit+           \\\hline
    经常账户 & 实体、资源     & 进口(买资源)       & 出口                   \\\hline
    金融账户 & 金融、资金     & 资金流出(买金融资产) & 资金流入(卖金融资产) \\\hline
    资本账户 & 免外债、被免债 & payment outflow      & payment inflow         \\\hline
\end{tabu}



\informationBox{
    \par * 在宏观经济中,$GNP\approx \text{国民收入}\approx GDP$
    \par * 净国际投资头寸(IIP):一国对外资产与负债的差额(net international investment position)
    \par * 关系:
    \begin{enumerate}[(1)]
        \item 贸易差额=货物出口-货物进口
        \item 经常项目差额CA=贸易差额+初次收入贷方-初次收入借方+二次收入贷方-二次收入借方
        \item 基本差额=CA+长期资本流入-长期资本流出
        \item 官方结算余额:中央银行净金融流动水平,表示需要官方储备去弥补的差额(official settlements balance),又称国际收支余额(balance of payments)。度量了国际借贷的规模和方向
        \item OBS=基本差额+私人短期资本流入-私人短期资本流出=CA+KA+Nonreserve FA+私人短期资本流动差额=-R
        \item 综合账户余额=官方结算余额+官方借款-官方贷款=官方结算差额+官方借贷=经常账户+金融账户(除官方储备)+资本账户+净误差与遗漏
        \item 经常账户余额+资本账户余额=金融账户余额
    \end{enumerate}
}





\subsection{经常账户}

定义:对外净出口的商品和服务的数额(current account, CA),又称经常账户余额


\informationBox{
    \par * 反映居民与非居民之间实际资源的国际流动
    \par * 构成:
    \begin{enumerate}[1.]
        \item 货物:一般商品 + 来料加工(goods)
        \item 服务:(services)
        \item 初次收入(primary income, PI)
              \begin{enumerate}[(1).]
                  \item 职工报酬
                  \item 财产收入:提供融资产和出租然资源所得的回报
                  \item 投资收益:提供融资产所得的回报,包括直接投资项下的利润利息收和再投资收入、证券投资收入(红利、利息等)和其他投资收入(利息)
              \end{enumerate}
        \item 二次收入:单方转移(官方援助/捐赠、侨民汇款、对国际组织的认缴款、战争赔款等)(secondary income, SI)
    \end{enumerate}
}


\subsection{金融账户}

定义:(financial account, FA)

\informationBox{
    * 构成:
    \begin{enumerate}[1.]
        \item 非储蓄性质
              \begin{enumerate}[(1)]
                  \item 直接投资:股本投资、建厂(foreign direct investment, FDI)
                  \item 证券投资(portfolio investment)
                  \item 金融衍生品(derivatives)和雇员认股权
                  \item 其他投资(other investment)
              \end{enumerate}
        \item 储蓄性质
              \begin{enumerate}[(1)]
                  \item 储备资产:一国货币当局所直接控制的、实际存在的可随时用来干预外汇市场、支付国际收支差额的资产(reserve assets)
                        \begin{enumerate}[a.]
                            \item 货币黄金:货币当局持有的黄金(gold)
                            \item 外汇储备(负表示官方储备增加)(主要)(foreign reserve)
                            \item 特别提款权:IMF 创设的种于补充成员国官储备的国际储备资产,虚拟合成,价值由一揽子货币决定(special drawing rights, SDR)
                            \item 在IMF的储备头寸:分配的储备(position)
                        \end{enumerate}
              \end{enumerate}
    \end{enumerate}
}


\subsection{资本账户}

定义:包括居民与非居民之间的资本转移和非生产性、非金融资产的交易(capital account, CA)

\informationBox{
    * 结构:
    \begin{enumerate}[(1)]
        \item 固定资产所有权的变更
        \item 债务债权的减免
        \item 非生产性(non-market,non-produced)有形资产(土地和地下资产)、无形资产(专利、版权、商标等)所有权转移
    \end{enumerate}
}

\subsection{净误差与遗漏}

定义:(net errors and omissions)
\informationBox{
    \par * 为负且扩大:资本外逃(capital flight)
    \par * 为正且扩大:热钱流入
}






\chapter{收入表}

衡量的是流量。

\section{企业利润表}


\section{所有者权益变动表}


\section{国民收入核算}


国民收入核算(national income accounting)。


\subsection{国内生产总值(GDP)}

定义:衡量现期生产的产品与服务的价值(gross domestic product,GDP)

\informationBox{
    * 局限性
    \begin{enumerate}[(1)]
        \item 国际比较
        \item 资源环境成本
        \item 经济增长效率
        \item 滞后性(核算程序)
    \end{enumerate}
}



\subsubsection{核算方法}

生产法:国内生产总值=总产出 - 中间投入

收入法:国内生产总值=劳动者报酬+生产税净额+固定资产折旧+营业盈余

支出法:国内生产总值=最终消费+资本形成总额+净出口


\informationBox{
    \par 计量规则:国内+生产(产品与服务)+总值(可统计)
    \par * 计入:
    \begin{enumerate}[(1)]
        \item 所有企业的总增加值(最终产品和服务的总价值)
        \item 租房、住房(被视为服务)、政府服务
        \item 短期出境单位
    \end{enumerate}
    \par * 不计:
    \begin{enumerate}[(1)]
        \item 二手货出售(非生产)
        \item 中间产品价值(重复统计)
        \item 耐用品租金(非生产)
        \item 地下经济(无法统计)
    \end{enumerate}
    \par * 特殊:
    \begin{enumerate}[(1)]
        \item 存货处理(是否可售)不可售不能理解为自己卖给自己,总支出没变,只不过在工资与利润之间分配
        \item 日常用GDP已经刨除了物价因素,是实际GDP
        \item 大多GDP数据都经过了季节性调整,因此GDP的变化必须寻找季节性周期以外的解释
    \end{enumerate}

}

\subsubsection{组成部分}

\begin{align*}
    Y & = C^D+I^D+G^D+EX         \\
      & = C+I+G+EX-(C^F+I^F+G^F) \\
      & = C+I+G+(EX-IM)          \\
      & = C+I+G+NX
\end{align*}
$C$:消费:国内居民私人消费的数额(consumption)
\informationBox{
    * 构成:
    \begin{enumerate}[(1)]
        \item 耐用品
        \item 非耐用品
        \item 服务
    \end{enumerate}
}



$I$:投资:私人企业为进行再生产而留下的用于购买厂房、设备的数额(investment)
\informationBox{
    * 构成:
    \begin{enumerate}[(1)]
        \item 企业固定投资(非住房固定投资)
        \item 住房固定投资
        \item 存货投资
    \end{enumerate}
    \par * 创造资本的新实物资产,资本可被用于未来生产
    \par * 不考虑部门内部的调剂
    \par * 买股票:发行市场算,流通市场不算
}


$G$:政府购买:政府购买商品和服务使用的数额(government purchases )
$T$:税收-转移支付
\informationBox{
    \par * 不包括转移支付
    \par 财政收支:政府参与国民收入分配和再分配活动的主要体现,参与和干预经济运行的主要渠道
    \par 常为财政赤字
\begin{enumerate}[1.]
    \item $G>T$:财政赤字,预算赤字(budget surplus)
    \begin{enumerate}[(1)]
        \item 增加税收 :缩减居民和企业的可支配收入  
        \item 向央行透支:影响货币稳定,诱发通胀  
        \item 借款:对内发公债,对外发行债券或借款  
        \begin{enumerate}[a.]
            \item 公债影响  
            \begin{enumerate}[(a)]
                \item 公债利率常为无风险基准利率  
                \item 金融机构调节资金流动性最主要工具  
                \item 央行公开市场操作的主要操作工具
            \end{enumerate}
        \end{enumerate}
    \end{enumerate}
    \item $G=T$:预算平衡(balanced budget)
    \item $G<T$:财政盈余,预算盈余(budget deficit)
    \begin{enumerate}[(1)]
        \item 偿还前期政府债券:减少非政府部门投资机会,供求引起债券价格上涨  
        \item 保留盈余以备后用:增加央行账户上的政府存款  
        \item 政府投资  
        \begin{enumerate}[a.]
            \item 国内:经济发展;经济结构调整  
            \item 国际:  
            \begin{enumerate}[(a)]
                \item 外汇储备  
                \item 主权财富基金:一国政府利用外汇储备资产创立的在全球范围内进行投资以提升本国经济和居民福利的机构投资者(Sovereign Wealth Funds ,SWFs),简称主权基金
            \end{enumerate}
        \end{enumerate}
    \end{enumerate}
\end{enumerate}
    
    
}


$EX$:外国在国内产品与服务上的支出

$NX$:净出口(net exports)=贸易余额(trade balance)
\informationBox{
    \par $$NX=Y-C-G-I=S-I$$
    \par $S-I$:资本净流出(net capital outflow)=国外净投资(net foreign investment)
    \par $NX=S-I>0$:贸易盈余(trade surplus)
    \par $NX=S-I<0$:贸易赤字(trade deficit)
    \par $NX=S-I=0$:贸易平衡(balanced trade)
    \par * 双边贸易余额无关紧要,一国与所有外国的总体贸易余额才重要
}


\subsection{国民生产总值(GNP)}

定义:一个国家的生产要素在一定时期内所生产并在市场上卖出的最终商品和服务的价值总量(gross national product,GNP)

\begin{align*}
    GNP=GDP+\text{来自国外的要素报酬}-\text{支付给国外的要素报酬}
\end{align*}


\subsection{国民净产值(NNP)}

(net national product)

\begin{align*}
    NNP= GNP-\text{折旧}
\end{align*}


\subsection{国民收入(NI)}

定义:一定时期内由该国的生产要素获得的收入(national income)

\begin{align*}
    NI=NNP+\text{净单边转移支付}-\text{统计误差}
\end{align*}

\informationBox{
    * 构成
    \begin{enumerate}[(1)]
        \item 雇员报酬:工人赚到的工资和福利津贴(compensation of employees)
        \item 业主收入:非公司型企业的收入(proprietors' income)
        \item 租金收入:房东得到的收入(包括自有住房)-折旧等支出(rental income)
        \item 公司利润:公司在向工人和债权人支付报酬后的收入(corporate profits)
        \item 净利息:国内企业支付的利息-国内得到的利息+国外赚取的利息(net interest)
        \item 生产和进口税:企业的某些税收(如销售税)减去充抵的企业补贴(taxs on production and imports)
    \end{enumerate}
}


\subsection{个人可支配收入}

(disposable personal income)

\begin{gather*}
    \text{个人可支配收入}
    =\text{个人收入}-\text{个人税收个人收入(personal income)} \\
    =\text{NI}-\text{生产和进口税}-\text{公司利润}-\text{社会保险费}-\text{净利息}+\text{股息}+\text{政府对个人的转移支付}+\text{个人利息收入}
\end{gather*}








\chapter{财政政策}


\chapter{货币政策}


\section{理论}



\subsubsection{传导理论}

\begin{enumerate}[1.]
  \item 金融价格传导论
        \begin{enumerate}[(1)]
          \item 凯恩斯
                \begin{enumerate}[a.]
                  \item 局部均衡到一般均衡分析$ M\uparrow \Rightarrow r\downarrow \Rightarrow I\uparrow \Rightarrow AD\uparrow \Rightarrow Y\uparrow  $
                \end{enumerate}
          \item 托宾Q:资产价格以及利率结构等因素
                \begin{enumerate}[a.]
                  \item 企业的市场价值与资本重置成本的比值q变化,将影响厂商的投资行为
                  \item $M\uparrow \Rightarrow Ps\uparrow \Rightarrow Q\uparrow \Rightarrow I\uparrow \Rightarrow AD\uparrow \Rightarrow Y\uparrow $
                \end{enumerate}
          \item 财富效应(莫迪利亚尼:消费生命周期假说)
                \begin{enumerate}[a.]
                  \item 影响家庭的资产负债表,影响家庭的消费决策,从而影响总需求和产出
                  \item $M\uparrow \Rightarrow Ps\uparrow\Rightarrow \text{财富}\uparrow\Rightarrow C\uparrow \Rightarrow AD\uparrow\Rightarrow Y\uparrow $
                \end{enumerate}
          \item 汇率传导:贸易开放度和进出口需求弹性
                \begin{enumerate}[a.]
                  \item 对于开放度较高、进出口需求弹性较大的经济体来说,汇率传导机制更为重要
                  \item 影响资本流动和资产价格进行传导
                  \item $M\uparrow\Rightarrow r\downarrow \Rightarrow E\uparrow \Rightarrow NX\uparrow \Rightarrow AD\uparrow\Rightarrow Y\uparrow $
                \end{enumerate}
        \end{enumerate}
  \item 货币传导论	
        \begin{enumerate}[(1)]
          \item $ M\uparrow \Rightarrow M/P>(M/P)^*\Rightarrow C\uparrow \Rightarrow AD\uparrow\Rightarrow Y\uparrow $
          \item $(M/P)*$:表示人们满意的实际货币余额
          \item 货币传导的作用机制主要是“实际余额效应”也称“皮古效应”
        \end{enumerate}
  \item 信贷传导论
        \begin{enumerate}[(1)]
          \item 企业资产负债表:$M\uparrow \Rightarrow Ps\uparrow \Rightarrow \text{资产价格}\uparrow \Rightarrow \text{逆向选择}\downarrow +\text{道德风险}\downarrow $
          \item 企业现金流:$M\uparrow \Rightarrow i\downarrow \Rightarrow \text{企业现金流}\uparrow\Rightarrow \text{逆向选择}\downarrow +\text{道德风险}\downarrow $
          \item 银行贷款:$M\uparrow \Rightarrow \text{银行准备金}\uparrow \Rightarrow \text{银行存款}\uparrow +\text{贷款}\uparrow \Rightarrow I\uparrow \Rightarrow AD\uparrow\Rightarrow Y\uparrow$
        \end{enumerate}
\end{enumerate}













\section{工具}

\subsection{一般性}


\begin{enumerate}[1.]
  \item 法定存款准备金率
        \begin{enumerate}[(1)]
          \item 效果:通过决定和改变乘数来影响货币供应量,其他货币政策工具都以存款准备金为基础
          \item 局限性:
                \begin{enumerate}[a.]
                  \item 效果强烈,不宜经常使用
                  \item 存款准备金率因金融机构、存款期限不同而不同,政策效果不易把握
                \end{enumerate}
        \end{enumerate}
  \item 再贴现政策
        \begin{enumerate}[(1)]
          \item  定义:中央银行对商业银行向中央银行申请再贴现所作的政策性规定。
          \item 内容:
                \begin{enumerate}[a.]
                  \item 再贴现率的确定与调整
                  \item 申请再贴现资格的规定与调整
                \end{enumerate}
          \item 优点:
                \begin{enumerate}[a.]
                  \item 能够产生告示效应,调节信用规模有效
                \end{enumerate}
          \item 局限性:
                \begin{enumerate}[a.]
                  \item 主动权并非只在中央银行,市场的变化可能违背政策意愿
                  \item 再贴现率的调节作用有限度
                  \item 中央银行通过再贴现充当最后贷款人,可能加大金融机构的道德风险
                \end{enumerate}
        \end{enumerate}
  \item 公开市场业务
        \begin{enumerate}[(1)]
          \item 定义:中央银行在金融市场上公开买卖有价证券,以此来调节金融机构的准备金和基础货币,进而影响市场利率和货币量的政策行为
          \item 优点:主动性强;灵活性强;调控效果和缓,震动性小;具有告示效应,影响范围广
          \item 局限性:中央银行具备强大的干预市场的实力;具有完善、发达的市场体系;需其他政策配合
        \end{enumerate}
\end{enumerate}






\subsection{选择性}

选择性政策工具是指中央银行可选择地对某些特殊领域的信用加以调节和影响的措施 

\begin{enumerate}[1.]
  \item 消费者信用控制
  \item 证券市场信用控制
  \item 不动产信用控制
  \item 优惠利率
  \item 预缴进口保证金
\end{enumerate}



\subsection{其他}

\begin{enumerate}[1.]
  \item 直接信用控制
        \begin{enumerate}[(1)]
          \item 定义:从质和量两个层面,以行政命令或其他方式,直接对金融机构尤其是商业银行的信用活动所进行的控制(direct credit control)
          \item 方式
                \begin{enumerate}[a.]
                  \item 利率最高限、信用配额、流动性比率和直接干预等
                \end{enumerate}
        \end{enumerate}
  \item 间接信用指导
        \begin{enumerate}[(1)]
          \item 定义:中央银行通过道义劝告、窗口指导等办法间接影响商业银行的信用创造
          \item 方式
                \begin{enumerate}[a.]
                  \item 道义劝告:一般包括情况通报、书面文件、指示及不负责人面谈意向等
                  \item 窗口指导:中央银行在其商业银行的往来中,对商业银行的季度贷款额度附加规 定,否则中央银行便削减甚至停止向商业银行提供再贷款
                \end{enumerate}
        \end{enumerate}
\end{enumerate}


\chapter{贸易政策}



\chapter{监管政策}


\section{模式}



\section{行业}


\subsection{银行业}



\subsection{证券业}


\chapter{行政政策}








\chapter{社会观}





\section{社会规律}

\subsection{运动规律:社会基本矛盾运动}




\begin{equation*} 
    \text{社会基本矛盾}\mindmap{
        \text{经济基础}\mindmap{
            \text{生产方式}\mindmap{
                \text{生产力}\mindmap{
                    \text{劳动者}\\ 
                    \text{劳动工具}\\ 
                    \text{劳动对象}\\ 
                    \text{其他生产要素}
                }
                \text{生产关系}\mindmap{
                    \text{生产资料所有制}\\ 
                    \text{人们在生产中的地位和相互关系}\\ 
                    \text{分配制度 }
                }
            }
        }
        \text{上层建筑}\mindmap{
            \text{政治上层建筑}\\ 
            \text{思想上层建筑} 
        }
    }
\end{equation*}


\subsubsection{生产力与生产关系}


物质资料的生产方式是人类社会存在和发展的基础。物质资料的生产方式是生产力和生产关系的统一。它对社会的发展起决定作用,表现在:生产活动是人类社会存在和发展的基础,生产方式决定着社会的性质和面貌,生产方式的变革决定着社会形态的更替。

\begin{enumerate}[1.]
    \item 生产力
          \begin{enumerate}[(1)]
            \item 含义
                  \begin{enumerate}[a.]
                    \item 定义:是人们改造自然和控制自然界的能力
                    \item 表现人与自然之间的关系
                  \end{enumerate}
            \item 构成:
                  \begin{enumerate}[a.]
                    \item 劳动者
                          \begin{enumerate}[(a)]
                            \item 劳动者:工人、管理人员、科技人员
                            \item 人的劳动是生产力的决定性要素
                          \end{enumerate}
                    \item 劳动资料
                          \begin{enumerate}[(a)]
                            \item 劳动资料:生产工具、建筑物等
                            \item 生产工具是划分经济时代的标志
                          \end{enumerate}
                    \item 劳动对象
                    \item 其他生产要素:科技、知识、信息、管理
                  \end{enumerate}
          \end{enumerate}
    \item 生产关系
          \begin{enumerate}[(1)]
            \item 含义
                  \begin{enumerate}[a.]
                    \item 狭义:人们在物质资料的生产过程中形成的社会关系
                    \item 广义:指一组关系,既包括生产、交换、分配、消费之间的关系,又包括人们在生产、交换、分配和消费各个环节中的关系
                          \begin{enumerate}[(a)]
                            \item 生产起决定作用,消费是最终目的,交换和分配是媒介
                            \item 生产决定分配、交换和消费
                            \item 分配、交换和消费反作用于生产
                          \end{enumerate}
                    \item 表现人与人、社会之间的关系
                  \end{enumerate}
            \item 关系:
                  \begin{enumerate}[a.]
                    \item 生产与消费
                          \begin{enumerate}[(a)]
                            \item 生产决定消费
                                  \begin{itemize}
                                    \item 生产决定消费的对象
                                    \item 生产决定消费的方式
                                    \item 生产是消费的动力
                                  \end{itemize}
                            \item 消费反作用于生产
                                  \begin{itemize}
                                    \item 只有通过消费,才能使产品成为现实的产品
                                    \item 消费是生产的目的和动力,消费又创造出现实的生产(新的消费热点)
                                    \item (消费调节生产)消费所形成的新的需求会对生产的调整和升级起导向作用
                                  \end{itemize}
                          \end{enumerate}
                    \item 生产与交换
                          \begin{enumerate}[(a)]
                            \item 生产决定交换
                                  \begin{itemize}
                                    \item 生产的性质决定交换的性质
                                    \item 生产的发展程度决定交换的发展程度
                                  \end{itemize}
                            \item 交换反作用于生产
                                  \begin{itemize}
                                    \item 生产进行前,要进行生产资料劳动力等生产要素的交换
                                    \item 生产出产品后,需要进行产品交换或商品交换
                                  \end{itemize}
                          \end{enumerate}
                    \item 生产与分配
                          \begin{enumerate}[(a)]
                            \item 产品分配:生产决定分配
                                  \begin{itemize}
                                    \item 分配对象:是生产的产品
                                    \item 参与生产的方式决定参与分配的方式:劳动者(工资),出资者(投资分红)
                                  \end{itemize}
                            \item 生产资料和社会成员在生产过程中的分配:分配反作用于生产
                                  \begin{itemize}
                                    \item 生产前:必须进行生产资料归属权的问题,社会成员在生产过程中处于的不同社会地位
                                    \item 生产中:分配方式直接影响劳动效率,直接影响生产要素配置的效率
                                  \end{itemize}
                          \end{enumerate}
                  \end{enumerate}
            \item 构成
                  \begin{enumerate}[a.]
                    \item 生产资料所有制
                          \begin{enumerate}[(a)]
                            \item 定义:由一定的生产力水平所决定的人们对生产资料的占有形式
                                  \begin{itemize}
                                    \item 组成:人们对生产资料的所有、占有、支配、使用等经济关系组成
                                    \item 地位:是生产关系的基础
                                  \end{itemize}
                            \item 产权
                                  \begin{itemize}
                                    \item 定义:主要指财产权利,即以财产所有权为主体的一系列财产权利的总和
                                    \item 财产权利关系的实质是人与人之间的经济关系
                                    \item 产权对财产主体有实现利益的要求
                                    \item 所有权义:人们占有某物并据此实现的经济利益关系。所有是所有制关系的基础,所有权是所有制的法律表现形式
                                  \end{itemize}
                          \end{enumerate}
                    \item 人们在生产中的地位和相互关系
                    \item 分配制度
                  \end{enumerate}
          \end{enumerate}
    \item 关系
          \begin{enumerate}[(1)]
            \item 对立:
                  \begin{enumerate}[a.]
                    \item 生产力反映人与自然界的关系,生产关系反映人与社会的关系
                  \end{enumerate}
            \item  统一:
                  \begin{enumerate}[a.]
                    \item 生产力决定生产关系
                          \begin{enumerate}[(a)]
                            \item 生产力的水平决定生产关系的性质
                            \item 生产力的变化决定生产关系的变化
                          \end{enumerate}
                    \item  生产关系反作用于生产力
                          \begin{enumerate}[(a)]
                            \item 生产关系与生产力相适应时,促进生产力的发展
                            \item 生产关系与生产力状况不适应时,阻碍生产力的发展
                          \end{enumerate}
                  \end{enumerate}
          \end{enumerate}
    \item 方法论
    \begin{enumerate}[(1)]
        \item 生产关系一定要适应生产力状况
    \end{enumerate}
\end{enumerate}
  

\subsubsection{经济基础与上层建筑}



\begin{enumerate}[1.]
    \item 经济基础
          \begin{enumerate}[(1)]
            \item 定义:由一定发展阶段生产力所决定的占统治地位的生产关系的总和
          \end{enumerate}
    \item 上层建筑
          \begin{enumerate}[(1)]
            \item 定义:一定社会的政治、法律制度和设施,以及该社会的各种思想观点和社会意识形态
          \end{enumerate}
    \item 关系
          \begin{enumerate}[(1)]
            \item 经济基础决定上层建筑。经济基础决定社会的政治、法律制度和设施,决定社会的各种思想观点和社会意识形态,即经济基础决定上层建筑
            \item 上层建筑对经济基础具有反作用。当上层建筑适合经济基础状况时,它促进经济基础的巩固和完善;当它不适合经济基础状况时,会阻碍经济基础的发展和变革
            \item 当上层建筑为先进的经济基础服务时,它就促进生产力的发展,推动社会进步,反之则阻碍
          \end{enumerate}
  \end{enumerate}
  






\subsection{发展规律:社会形态更替}

\subsubsection{社会历史发展总趋势}




\begin{enumerate}[1.]
    \item 总趋势是前进的、上升的
          \begin{enumerate}[(1)]
              \item 人类通过各种实践活动不断地解决社会基本矛盾,从而推动社会历史由低级向高级发展
          \end{enumerate}
    \item 发展过程是曲折的
          \begin{enumerate}[(1)]
              \item 社会发展是在生产力和生产关系、经济基础和上层建筑的矛盾运动中,在社会基本矛盾的不断解决中实现的
              \item 根本动力是社会基本矛盾
              \item 社会革命是重要动力
              \item ​在阶级社会里,社会基本矛盾的解决主要是通过阶级斗争实现的。阶级斗争是推动阶级社会发展的直接动力
              \item 当代社会,科学技术成为动力之一
          \end{enumerate}
\end{enumerate}





\informationBox{
    * 社会主义社会的基本矛盾
    \begin{enumerate}[(1)]
        \item 特点
        \begin{enumerate}[a.]
            \item 矛盾的性质不同:是非对抗性矛盾
            \item 矛盾的解决方式不同:不是通过阶级斗争的方式解决,只能通过社会主义的自我发展、自我完善加以解决
        \end{enumerate}
        \item 方法论:
            \begin{enumerate}[a.]
                \item 改革是社会主义的自我完善和发展,是推进中国特色社会主义各方面工作的强大动力、直接动力
            \end{enumerate}​
    \end{enumerate}
}






\section{社会历史主体:人民群众}



\chapter{价值观}






\chapter{琵琶}\label{chapter:琵琶}




















\part{习题}













\end{document}