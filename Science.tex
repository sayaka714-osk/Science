%!TEX TS-program = xelatex
%!TEX encoding = UTF-8

\documentclass[12pt]{book}


%%%%%%%  章节  %%%%%%

% Options: Sonny, Lenny, Glenn, Conny, Rejne, Bjarne, Bjornstrup
\usepackage[Lenny]{fncychap}
\ChTitleVar{\Large}

% 标题格式
\usepackage{titlesec}
\titleformat{\part}{\centering\Huge\bfseries}{第\Roman{part}部分}{1em}{}
\titleformat{\chapter}{\centering\huge\bfseries}{第\arabic{chapter}章}{1em}{}
\titleformat{\section}{\LARGE\bfseries}{第\arabic{section}节}{1em}{}
\titleformat{\subsection}{\Large\bfseries}{\arabic{subsection}.}{1em}{}
\titleformat{\subsubsection}{\large\bfseries}{\arabic{subsection}.\arabic{subsubsection}.}{1em}{}
\titleformat{\paragraph}{\normalsize\bfseries}{\arabic{subsection}.\arabic{subsubsection}.\arabic{paragraph}.}{1em}{}
\titleformat{\subparagraph}{\normalsize\bfseries}{\arabic{subsection}.\arabic{subsubsection}.\arabic{paragraph}.\arabic{subparagraph}.}{1em}{}

% 索引
\usepackage[xindy]{imakeidx}
\makeindex[columns=2, program=truexindy, intoc=true, options=-M texindy -I xelatex -C utf8, title={Index}]




%%%%%%%  页面设置  %%%%%%
\usepackage{geometry}   % 页面设置
\geometry{a4paper,left=2.5cm,right=2.5cm,top=2.5cm,bottom=2.5cm}
\usepackage{pdflscape}   % 页面横置
\usepackage{ragged2e}    % 两端对齐
\usepackage{indentfirst} % 首行缩进
%\usepackage{setspace}  % 间距
\setcounter{tocdepth}{7}  % 标题深度
\setcounter{secnumdepth}{7}
%\setlength{\baselineskip}{20pt} % 行距
\setlength{\parindent}{2em} % 首行缩进

% 页眉页脚
\usepackage{fancyhdr}
\pagestyle{fancy} % 设置页眉  
\lhead{}
\chead{}
\rhead{}
\cfoot{\thepage}
\rfoot{}
\lfoot{}
\renewcommand{\headrulewidth}{0pt}  %页眉线宽,设为0可以去页眉线


%%%%%%%  字体  %%%%%%
\usepackage{polyglossia} % 多语言

\usepackage{fontspec} % 字体
\usepackage{xeCJK}
\setmainfont{Times New Roman} 
\setCJKmainfont{SimSun} 

%%%%%%%%  表格  %%%%%%%%
\usepackage{supertabular}
\usepackage{tabularx}      % 表格自动换行
\usepackage{longtable}
\usepackage{tabu}
\usepackage{booktabs}      % 表格线条
\usepackage{makecell}
\usepackage{multirow}    % 单元格合并
\usepackage{caption}

% \columnwidth  当前分栏的宽度
% \linewidth    当前行的宽度
% \textwidth    整个页面版芯的宽度
% \paperwidth   整个页面纸张的宽度

\newcommand{\tablelist}[3]{
    \parbox{#2}{
        \begin{enumerate}[#1]
            #3
        \end{enumerate}
    }
}



%%%%%%%%%  编号  %%%%%%%%%
\usepackage{enumerate}   % 条目 
\usepackage{enumitem}
%\setitemsize[1]{itemsep=0pt,partopsep=0pt,parsep=\parskip,topsep=0pt}
\usepackage{amsmath}
\usepackage{paralist}
\let\itemize\compactitem
\let\enditemize\endcompactitem
\let\enumerate\compactenum
\let\endenumerate\endcompactenum
\let\description\compactdesc
\let\enddescription\endcompactdesc





%%%%%%%% 公式
\usepackage{bm} % 数学公式加粗
\usepackage{bbm}
\usepackage{amsfonts}
\usepackage{amssymb}
\usepackage{breqn}


%%%%%%%  绘图  %%%%%%
\usepackage{graphicx}    % 图

\usepackage{tikz}
\usetikzlibrary{trees,calc,quotes}
\usetikzlibrary{angles,patterns,datavisualization}
\usetikzlibrary{arrows,intersections}
\usetikzlibrary{graphs}
\newcommand{\treegraph}[1]{
\usetikzlibrary{trees}
\tikzstyle{every node}=[draw=black,thick,anchor=west]
\begin{tikzpicture}[
        grow via three points={one child at (0.5,-0.7) and
                two children at (0.5,-0.7) and (0.5,-1.4)},
        edge from parent path={(\tikzparentnode.south) |- (\tikzchildnode.west)}]
    #1
    ;
\end{tikzpicture}
}

\tikzset{eaxis/.style={->,>=stealth}}
\tikzset{elegant/.style={smooth,thick,samples=50,cyan}}

\usepackage{tikz-cd}
\usetikzlibrary{matrix,arrows,decorations.pathmorphing}

\usepackage{pgfplots}
\usepgfplotslibrary{groupplots}

%%%%%%%%  标记  %%%%%%%%%
\usepackage[breaklinks,colorlinks,linkcolor=black,citecolor=black,urlcolor=black]{hyperref}




%%%%%%%  颜色  %%%%%%

\usepackage{color}
\definecolor{codeKeyword}{RGB}{200,0,100}
\definecolor{codeString}{RGB}{200,100,40}
\definecolor{codeComment}{RGB}{0,100,0}
\definecolor{codeNumber}{RGB}{128,128,128}
\definecolor{codeBackground}{RGB}{242,242,242}

% Matlab highlight color settings
%\definecolor{mBasic}{RGB}{248,248,242}       % default
\definecolor{mKeyword}{RGB}{0,0,255}          % bule
\definecolor{mString}{RGB}{160,32,240}        % purple
\definecolor{mComment}{RGB}{34,139,34}        % green
\definecolor{mBackground}{RGB}{245,245,245}   % lightgrey
\definecolor{mNumber}{RGB}{128,128,128}       % gray

\definecolor{Numberbg}{RGB}{237,240,241}     % lightgrey

% Python highlight color settings
%\definecolor{pBasic}{RGB}{248, 248, 242}     % default
\definecolor{pKeyword}{RGB}{228,0,128}        % magenta
\definecolor{pString}{RGB}{148,0,209}         % purple
\definecolor{pComment}{RGB}{117,113,94}       % gray
\definecolor{pIdentifier}{RGB}{166, 226, 46}  %
\definecolor{pBackground}{RGB}{245,245,245}   % lightgrey
\definecolor{pNumber}{RGB}{128,128,128}       % gray


\usepackage{xcolor}

%%%%%%%%%%  模块  %%%%%%%%%%%%


\usepackage{tcolorbox}
\newcommand{\informationBox}[1]{
    \small
    \begin{tcolorbox}[colback=gray!10!white,colframe=gray!30!white]
        #1
    \end{tcolorbox}
}

% 思维导图
\newcommand{\mindmap}[1]{
    \begin{cases}
        #1
    \end{cases}\\
}


% 引用文本
\newcommand{\refdocument}[1]{
    {\kaishu 
    \begin{quotation}
        #1
    \end{quotation}
    }
}









%%%%%%%%%%%%%%%%%%%%%%%%%%%%%%%%%


\usepackage[UTF8]{ctex}
\usepackage{autobreak}
\usepackage[utf8]{inputenc} % Required for inputting international characters

\usepackage{adjustbox}   % 调整box大小

\usepackage{tipa}
\usepackage{CJKfntef}
\usepackage{pdfpages}

\usepackage{blindtext}
\usepackage{verbatim}
\usepackage{ascmac}
\usepackage{xpinyin} % 拼音









\begin{document}
\title{科学\\Science}  %%书名
\author{Maobin Xu} %%作者
%\date{} %%如果没有这句,会生成时间
\maketitle  %%生成书名


\tableofcontents  %%生成目录
\mainmatter %%表示文章的正文部分,在生成目录后将从第一页开始

\part{体系}

科学包括:数理与数据科学(Mathematics and Data Science)、自然与工程科学(Natural and Engineering Science)和人文与社会科学(Humanities and Social Science)。




{\tiny
\begin{equation*}
    \text{数理与数据科学}\mindmap{
        \text{数学理论}\mindmap{
            \text{确定性数学}\mindmap{
                \text{数理逻辑}\\
                \text{代数(数)}\mindmap{
                    \text{算数}\\
                    \text{数论}\\
                    \text{代数:线性代数}
                }
                \text{几何(形)}\mindmap{
                    \text{解析几何}\\
                    \text{拓扑学}\\
                    \text{非欧几何}
                }
                \text{分析(极限)}\mindmap{
                    \text{函数论}\\
                    \text{微积分}\\
                    \text{泛函分析}
                }
            }
            \text{不确定数学}\mindmap{
                \text{随机变量}\\
                \text{统计推断}\mindmap{
                    \text{参数估计}\\
                    \text{假设检验}\mindmap{
                        \text{参数检验}\\
                        \text{非参数检验}
                    }
                }
            }
        }
        \text{数据分析}\mindmap{
            \text{机器学习}\mindmap{
                \text{符号学习}\\
                \text{统计学习}\mindmap{
                    \text{有监督学习}\mindmap{
                        \text{回归分析}\\
                        \text{分类分析}
                    }
                    \text{无监督学习}\mindmap{
                        \text{聚类分析}\\
                        \text{降维分析}
                    }
                }
                \text{连接学习}\mindmap{
                    \text{深度学习/神经网络}
                }
            }
            \text{强化学习}
        }
    }
\end{equation*}



\begin{equation*}
    \text{自然与工程科学}\mindmap{
        \text{自然科学}\mindmap{
            \text{物理学}\\
            \text{化学}\\
            \text{生物学}
        }
        \text{工程科学}\mindmap{
            \text{计算机科学}\mindmap{
                \text{工具}\mindmap{
                    \text{LaTeX}\\
                    \text{Python}\\
                    \text{Stata}
                }
                \text{计算机视觉}\\
                \text{语音识别}\\
                \text{自然语言处理}\\
                \text{自动程序设计}
            }
        }
    }
\end{equation*}



}


\clearpage


人文与社会科学包括:经济学(Economics)、政治学(Politics)、文化学(Culture)和历史学(History)。

构成(是什么),理论(为什么),方法(怎么做)

{\tiny
\begin{equation*}
    \text{经济学}\mindmap{
        \text{构成}\mindmap{
            \text{实体经济}\mindmap{
                \text{商品市场}\mindmap{
                    \text{产品市场}\mindmap{
                        \text{商品}
                    }
                    \text{生产要素市场}\mindmap{
                        \text{劳动力}\\
                        \text{资本}
                    }
                }
            }
            \text{虚拟经济}\mindmap{
                \text{资金市场}\mindmap{
                    \text{基础要素}\mindmap{
                        \text{货币}\\
                        \text{信用}\\
                        \text{利率}\\
                        \text{汇率}\\
                        \text{金融工具}
                    }
                    \text{运作载体}\mindmap{
                        \text{金融市场}\\
                        \text{金融机构}\\
                        \text{非金融机构}\\
                        \text{家庭}
                    }
                    \text{运作机制}\mindmap{
                        \text{金融调控}\\
                        \text{金融监管}
                    }
                }
                \text{房地产市场}
            }
        }
        \text{理论}\mindmap{
            \text{完全竞争市场}\mindmap{
                \text{静态分析}\mindmap{
                    \text{长期}\mindmap{
                        \text{\hyperref[chapter:古典AD-AS模型]{古典AD-AS模型}}\\
                        \text{剩余价值理论}
                    }
                    \text{极短期:凯恩斯AD-AS模型}\mindmap{
                        \text{IS-LM-BP模型}\mindmap{
                            \text{产品市场}\mindmap{
                                \text{收入-支出模型}
                            }
                            \text{货币市场}\\
                            \text{国际市场}
                        }
                    }
                    \text{短期:新凯恩斯AD-AS模型}\mindmap{
                        \text{粘性价格理论}\\
                        \text{粘性工资理论}\\
                        \text{不完全信息理论}\\
                        \text{菲利普斯曲线}
                    }
                }
                \text{比较静态分析}\\
                \text{动态分析}\mindmap{
                    \text{经济增长}\mindmap{
                        \text{外生增长}\mindmap{
                            \text{索罗模型}\\
                            \text{无限期和世代交叠模型}
                        }
                        \text{内生增长}\mindmap{
                            \text{内生增长模型}
                        }
                    }
                    \text{经济周期}\mindmap{
                        \text{名义冲击}\mindmap{
                            \text{货币经济周期理论}\mindmap{
                                \text{理性预期假说}\\
                                \text{持续市场出清假说}\\
                                \text{自然率假说}
                            }
                            \text{新凯恩斯主义DSGE模型}
                        }
                        \text{实际冲击}\mindmap{
                            \text{真实经济周期模型(RBC)}
                        }
                    }
                }
            }
            \text{不完全竞争市场}\mindmap{
                \text{缓冲经济系统的扭曲}\mindmap{
                    \text{不确定性}\\
                    \text{不完全信息}\\
                    \text{外部性}
                }
                \text{默认经济系统的扭曲}\mindmap{
                    \text{市场结构}\mindmap{
                        \text{垄断竞争市场}\\
                        \text{寡头垄断市场}\\
                        \text{完全垄断市场}
                    }
                }
                \text{吸收经济系统的扭曲}\mindmap{
                    \text{博弈论}
                }
            }

        }
        \text{方法}\mindmap{
            \text{目标}\mindmap{
                \text{经济增长-产品市场}\mindmap{
                    \text{核算}\mindmap{
                        \text{资产负债表}\\
                        \text{资金流量表}\\
                        \text{收入表}
                    }
                }
                \text{充分就业-劳动力市场}\mindmap{
                    \text{自然失业率模型}
                }
                \text{物价稳定-货币市场}\\
                \text{国际收支平衡-国际市场}
            }
            \text{政府}\mindmap{
                \text{财政政策}\\
                \text{货币政策}\\
                \text{贸易政策}\\
                \text{行政政策}
            }
        }
    }
\end{equation*}
}


政治经济学(Political Economics):广义上,研究一定社会生产、交换、分配和消费等经济活动中的经济关系和经济规律。狭义上,在中国特指马克思主义政治经济学,研究资本主义生产方式以及和它相适应的生产关系和交换关系(不再是自然经济),生产关系是马克思主义政治经济学研究的核心,关注价值。

西方经济学:包括微观经济学和宏观经济学,关注价格。


{\tiny
\begin{equation*}
    \text{政治学}\mindmap{
        \text{政治发展}\mindmap{
            \text{马克思主义中国化}
        }
    }
\end{equation*}

\begin{equation*}
    \text{文化学}\mindmap{
        \text{哲学}\mindmap{
            \text{哲学史}\\
            \text{辩证唯物主义}\mindmap{
                \text{世界观}\mindmap{
                    \text{辩证唯物论}
                }
                \text{方法论}\mindmap{
                    \text{唯物辩证法}
                }
                \text{历史唯物主义}\mindmap{
                    \text{社会观}\\
                    \text{价值观(个人)}
                }
            }
        }
        \text{伦理学}\\
        \text{语音文字学}\\
        \text{文学}\mindmap{
            \text{纯文学}\\
            \text{新闻学}
        }
        \text{艺术}\mindmap{
            \text{音乐}\mindmap{
                \text{器乐}\mindmap{
                    \text{弹拨乐器}\mindmap{
                        \text{琵琶}
                    }
                    \text{声乐}
                }
                \text{表演}
            }
        }
    }
\end{equation*}













\begin{equation*}
    \text{历史学}\mindmap{
        \text{史前时期}\\
        \text{古代史}\\
        \text{近代史}\mindmap{
            \text{中国近代史}}
        \text{现代史}\\
        \text{形势与政策}
    }
\end{equation*}
\clearpage


}






\part{数理与数据科学}





\chapter{参数估计}

参数估计:利用样本信息对总体数字特征做出的估计(parameter estimation)
\\

\begin{enumerate}[1.]
    \item 原则
          \begin{enumerate}[(1)]
              \item 无偏性
                    \begin{gather*}
                        \lim_{n→\infty}{E(\hat{\theta})=\theta } \\
                        E\left(\bar{\bm{X}}\right)=\bm{\mu},\ E\left(\frac{1}{n-1}\bm{S}\right)=\bm{\Sigma}
                    \end{gather*}
              \item 有效性
                    \begin{gather*}
                        D(\hat{\theta}_1)<D(\hat{\theta}_2),\hat{\theta}_1比\hat{\theta}_2有效
                    \end{gather*}
              \item 一致性
                    \begin{gather*}
                        \lim_{n\rightarrow\infty}{P\left({\left|{\hat{\theta}}_n-\theta\right|<\varepsilon}\right)}=1
                    \end{gather*}
          \end{enumerate}
    \item 内容
          \begin{enumerate}[(1)]
              \item 点估计(point estimation)
              \item 区间估计(interval estimation):$z_{\frac{\alpha}{2}}\approx 1.96$
                    \begin{enumerate}[a.]
                        \item $1-\alpha$:置信度,置信区间以$1-\alpha$的概率覆盖总体未知参数。置信度越大,置信区间越宽
                        \item 样本均值:中心位置
                        \item 总体标准差:总体波动越小,置信区间越窄
                        \item 样本容量n:样本容量越大,置信区间越窄
                        \item 样本标准差s:样本标准差越大,置信区间越宽
                    \end{enumerate}
          \end{enumerate}
\end{enumerate}


\section{最小二乘估计}





普通最小二乘法(Ordinary Least Square,OLS)

\subsection{方法}

\subsubsection{均值估计}
总体均值估计$\approx$样本均值

\paragraph{一元}

\begin{gather*}
    \bar{X}=\frac{1}{n}\sum_{i=0}^{n}X_i
\end{gather*}



\paragraph{多元}

\begin{align*}
    \hat{\bm{\mu}}
    = & \overline{\mathbf{X}}
    =\frac{1}{n}\sum_{a=1}^{n}\mathbf{X}_{(a)}
    =\frac{1}{n}\left[\left[
            \begin{matrix}
                \begin{matrix}
                    X_{11} \\
                    X_{12} \\
                \end{matrix} \\
                \begin{matrix}
                    \vdots \\
                    X_{1p} \\
                \end{matrix} \\
            \end{matrix}\right]  +\left[
            \begin{matrix}
                \begin{matrix}
                    X_{21} \\
                    X_{22} \\
                \end{matrix} \\
                \begin{matrix}
                    \vdots \\
                    X_{2p} \\
                \end{matrix} \\
            \end{matrix}\right] +\cdots+ \left[
    \begin{matrix}
                \begin{matrix}
                    X_{n1} \\
                    X_{n2} \\
                \end{matrix} \\
                \begin{matrix}
                    \vdots \\
                    X_{np} \\
                \end{matrix} \\
            \end{matrix}\right]\right] \\
    = & \frac{1}{n} \left[
        \begin{matrix}
            \begin{matrix}
                X_{11}+X_{21}+\cdots+X_{n1} \\
                X_{12}+X_{22}+\cdots+X_{n2} \\
            \end{matrix} \\
            \begin{matrix}
                \vdots                      \\
                X_{1p}+X_{2p}+\cdots+X_{np} \\
            \end{matrix} \\\end{matrix}\right]
    =\left({\overline{X}}_1,{\overline{X}}_2,\cdots,{\overline{X}}_p\right)^\prime
\end{align*}


\subsubsection{方差估计}

总体方差(离差)估计$\approx$样本方差(离差)

\paragraph{一元(离差)}

\begin{gather*}
    S^2=\frac{1}{n}\sum_{i=1}^{n}{(X_i-\bar{X})^2}, S^2=\frac{1}{n-1}\sum_{i=1}^{n}{(X_i-X)^2}
\end{gather*}


\paragraph{多元(样本离差阵)}

\begin{align*}

    \mathbf{S}_{p\times p}
     & = \sum_{a=1}^{n}{\left(\mathbf{X}_{\left(a\right)}-\overline{\mathbf{X}}\right)\left(\mathbf{X}_{\left(a\right)}-\overline{\mathbf{X}}\right)^\prime}
    = \sum_{a=1}^{n}
    \begin{bmatrix}
        \begin{bmatrix}
            X_{a1}-{\bar{X}}_1 \\
            X_{a2}-{\bar{X}}_2 \\
            \vdots             \\
            X_{ap}-{\bar{X}}_p \\
        \end{bmatrix}
         & \left(X_{a1}-{\bar{X}}_1,X_{a2}-{\bar{X}}_2,\cdots,X_{ap}-{\bar{X}}_p\right) \\
    \end{bmatrix}                                                                                                                              \\
     & = \sum_{a=1}^{n}
    \begin{bmatrix}
        \left(X_{a1}-{\bar{X}}_1\right)^2
         & \left(X_{a1}-{\bar{X}}_1\right)\left(X_{a2}-{\bar{X}}_2\right)
         & \cdots
         & \left(X_{a1}-{\bar{X}}_1\right)\left(X_{ap}-{\bar{X}}_p\right) \\

        \left(X_{a2}-{\bar{X}}_2\right)\left(X_{a1}-{\bar{X}}_1\right)
         & \left(X_{a2}-{\bar{X}}_2\right)^2
         & \cdots
         & \left(X_{a2}-{\bar{X}}_2\right)\left(X_{ap}-{\bar{X}}_p\right) \\
        \vdots
         & \vdots                                                         \\
         & \                                                              \\
         & \vdots                                                         \\
        \left(X_{ap}-{\bar{X}}_p\right)\left(X_{a1}-{\bar{X}}_1\right)
         & \left(X_{ap}-{\bar{X}}_p\right)\left(X_{a2}-{\bar{X}}_2\right)
         & \cdots
         & \left(X_{ap}-{\bar{X}}_p\right)^2                              \\
    \end{bmatrix}                                                                                                                               \\
     & = \begin{bmatrix}
        s_{11} & s_{12} & \cdots & s_{1p} \\
        s_{21} & s_{22} & \cdots & s_{2p} \\
        \vdots & \vdots & \      & \vdots \\
        s_{p1} & s_{p2} & \cdots & s_{pp} \\
    \end{bmatrix}
    = \left(s_{ij}\right)_{p\times p}
\end{align*}



\subsubsection{系数估计}

\paragraph{一元}


\begin{gather*}
    Y_i = \beta_0+\beta_{1}X_{i}+{\hat{\varepsilon}}_i\\
    \min_{{\hat{\beta}}_1}{\sum_{i=1}^{n}{\hat{\varepsilon}}_i^2} =\sum_{i=1}^{n}{\left(Y_i-{\hat{\beta}}_0-{\hat{\beta}}_1 X_i\right)^2}\\
    \frac{\partial}{\partial \beta_{1}}  \sum_{i=1}^{n}{\hat{ε}_i^2} =-2\sum_{i=1}^{n}{(Y_i-\hat{\beta}_{0}-\hat{\beta}_{1} X_i) X_{i}}=0\\
    \hat{\beta}_{1} = \frac{\sum_{i=1}^{n}\left(X_i-\bar{X}\right)\left(Y_i-\bar{Y}\right)}{\sum_{i=1}^{n}\left(X_i-\bar{X}\right)^2}=\frac{\sum_{i=1}^{n}{x_i y_i}}{\sum_{i=1}^{n}x_i^2},\hat{\beta}_{0}=\bar{Y}-\hat{\beta}_{1}\bar{X}\\
    {\hat{\beta}}_1 \sim N\left(\beta_1,\frac{\sigma_2}{\sum_{i=1}^{n}{x_i^2}}\right),\hat{\beta}_{0} \sim N\left(\beta_0, \frac{\sum_{i=1}^{n}{X_i^2}}{ n \sum_{i=1}^{n}{x_i^2}} \sigma^2\right)\\
\end{gather*}


随机干扰项$\mu_{i}$的$\hat{\sigma}^2 =\frac{\sum_{i=1}^{n}{\varepsilon_i^2}}{n-2}$

\paragraph{多元}

\begin{gather*}
    \hat{\bm{\beta}} \equiv \left(\bm{X}^\prime\bm{X}\right)^{-1}\bm{X}^\prime\bm{Y}
\end{gather*}

随机干扰项$\varepsilon_i$的方差
$\hat{\sigma}^2  = \sum_{i=1}^{n}{ 	\frac{ \hat{\varepsilon}_i^2 }{n-k-1} 	} = \frac{RSS}{n-k-1} = \frac{\bm{\hat{\varepsilon}'\hat{\varepsilon}}}{n-k-1}$

$k$:变量个数

置信区间:
\begin{gather*}
    \left( \hat{\beta}_j-S_{\hat{\beta}_j} t_{\frac{\alpha}{2}}(n-k-1), \hat{\beta}_j+S_{\hat{\beta}_j} t_{\frac{\alpha}{2}}(n-k-1) \right)
\end{gather*}

缩小置信区间方法:增大样本容量n;提高模型拟合优度(减小残差平方和)

\subsection{性质}

高斯-马尔可夫定理(Gauss-Markov Theorem):(无需随机扰动项正态分布)最小二乘法是最佳线性无偏估计(Best Linear Unbiased Estimator,BLUE)

\subsubsection{小样本性质}

(small-sample properties)


\paragraph{线性性}

一元:
\begin{gather*}
    {\hat{\beta}}_1=\frac{\sum_{i=1}^{n}{x_iy_i}}{\sum_{i=1}^{n}x_i^2}=\frac{\sum_{i=1}^{n}{x_i\left(Y_i-\bar{Y}\right)}}{\sum_{i=1}^{n}x_i^2}=\frac{\sum_{i=1}^{n}{x_iY_i}}{\sum_{i=1}^{n}x_i^2}-\frac{\bar{Y}\sum_{i=1}^{n}x_i}{\sum_{i=1}^{n}x_i^2}=\sum_{i=1}^{n}{k_iY_i}, k_i=\frac{x_i}{\sum_{i=1}^{n}{x_i^2}}
\end{gather*}

多元:
\begin{gather*}
    \hat{\bm{\beta}}\equiv\left(\mathbf{X}^\prime\mathbf{X}\right)^{-1}\mathbf{X}^\prime\mathbf{Y}=\mathbf{CY}\text{(关于Y线性)}
\end{gather*}


\paragraph{无偏性}

一元:
\begin{gather*}
    E\left({\hat{\beta}}_1\middle| X\right)=E\left[\left(\beta_1+\sum_{i=1}^{n}{k_i\varepsilon_i}\right)\middle| X\right]=\beta_1+\sum_{i=1}^{n}{k_iE(\varepsilon_i|X)}=\beta_1
\end{gather*}


多元:
\begin{gather*}
    E(\hat{\bm{\beta}}|\bm{X}) =\bm{\beta},E(\bm{s}^2|\bm{X})=\sigma^2\\
    E\left(\hat{\bm{\beta}}-\bm{\beta}\middle| X\right) =E\left[\left(\bm{X}^\prime\bm{X}\right)^{-1}\bm{X}^\prime\bm{\varepsilon}\middle|\bm{X}\right] =\left[\left(\bm{X}^\prime\bm{X}\right)^{-1}\bm{X}^\prime\right]E\left(\bm{\varepsilon}\middle|\bm{X}\right)=\bm{0}
\end{gather*}


\paragraph{有效性/最小方差性}

一元:
\begin{gather*}
    Var(\hat{\beta}_1|X)=\frac{\sigma^2}{\sum_{i=1}^{n}x_i^2} \\
    {\hat{\beta}}_1^\ast=\sum_{i=1}^{n}{c_iY_i}=\sum_{i=1}^{n}{\left(k_i+d_i\right)Y_i},d_i\text{为不全为零的常数}\\
    Var(\hat{\beta}_1^\ast)\geqslant Var(\hat{\beta}_1)
\end{gather*}


多元:
\begin{gather*}
    Var\left(\hat{\bm{\beta}}\middle|\bm{X}\right)=\sigma^2\left(\bm{X}^\prime\bm{X}\right)^{-\bm{1}}
\end{gather*}


\subsubsection{大样本渐进性质}



(large-sample asymptotic properties)
\paragraph{一致性}

\begin{gather*}
    \lim_{n\rightarrow\infty}{\hat{\bm{\beta}}} = \bm{\beta}
\end{gather*}


\paragraph{渐近无偏性}

\paragraph{渐近有效性}


















\section{矩估计法}





矩估计法(Method of Moment,MM)

\subsection{方法}

\subsubsection{均值估计}

\begin{gather*}
    \begin{cases}
        E(X) = \frac{1}{n}\sum_{i=1}^{n}{X_i}     \\
        E(X^2) = \frac{1}{n}\sum_{i=1}^{n}{X_i^2} \\
        \vdots                                    \\
        E(X^k) = \frac{1}{n}\sum_{i=1}^{n}{X_i^k}
    \end{cases}
\end{gather*}

\subsubsection{系数估计}

\begin{gather*}
    E( \mathbf{X_i} \varepsilon_{i} ) =\mathbf{0} \\
    \hat{\bm{\beta}} \equiv  (\mathbf{X}^{\prime}\mathbf{X})^{-1} \mathbf{X}^{\prime} \mathbf{Y}
\end{gather*}










\section{极大似然估计}







极大似然估计:又称最大似然法(Maximum Likelihood,ML)

\subsection{方法}
\subsubsection{均值估计}

\begin{gather*}
    L\left(x_1,x_2,\cdots,x_n;\theta_1,\theta_2,\cdots,\theta_n\right) =\prod_{i=1}^{n}f\left(x_i;\theta_1,\theta_2,\cdots,\theta_k\right)\\
    lnL=\sum_{i=1}^{n}lnf\left(x_i;\theta_1,\theta_2,\cdots,\theta_k\right)\\
    \frac{\partial lnL}{\partial\theta_i}=0,\ i=1,2,\cdots,k\\
    \begin{cases}
        \hat{\theta}_1 = \hat{\theta}_1(x_1,x_2,\cdots,x_n) \\
        \hat{\theta}_2 = \hat{\theta}_2(x_1,x_2,\cdots,x_n  \\
        \vdots                                              \\
        \hat{\theta}_k = \hat{\theta}_k(x_1,x_2,\cdots,x_n) \\
    \end{cases}	\\
    L\left(\bm{\mu},\bm{\Sigma}\right) =\prod_{i=1}^{n}{f(\bm{X}_i,\bm{\mu},\bm{\Sigma})}
\end{gather*}


\subsubsection{系数估计}

\begin{gather*}
    L\left(\bm{\beta},\sigma^2\right) 	=P\left(Y_1,Y_2,\cdots,Y_n\right) \\
    \hat{\bm{\beta}} 	\equiv \left(\bm{X}^\prime\bm{X}\right)^{-1}\bm{X}^\prime\bm{Y}
\end{gather*}


$Y$服从正态分布

随机干扰项$\sigma_i$的方差
\begin{gather*}
    {\hat{\sigma}}^2
    =\frac{
    \sum_{i=1}^{n}{
    {\hat{\bm{\varepsilon}}}_i^2}
    }{n}
    =\frac{
        {\hat{\bm{\varepsilon}}}^\prime \hat{\bm{\varepsilon}}
    }{n}
\end{gather*}



















\section{贝叶斯估计}


\chapter{假设检验}





\section{参数检验}

适用于定量、连续变量

假设是针对总体分布中未知参数提出的

原理:小概率事件不太可能在一次事件中发生,若发生则有理由拒绝原假设

第一类错误:拒真,$P\{拒绝H_0|H_0为真\}$,能够计算

第二类错误:取伪,$P\{接受H_0|H_0不真\}$

一般将样本支持的放在备择假设,样本背离的放在原假设;药物治愈、培训成果等将无效放在原假设;等号放在原假设




\subsection{单个正态总体}





单个正态总体的参数检验。

\informationBox{
    对于匹配样本(paired-sample),样本量必须一样。通过差分变为单一样本,比较前后差异
}

\subsubsection{均值$\mu$的检验}

\paragraph{一元}

\begin{align*}
    H_0: \mu = \mu_0    & \leftrightarrow  	H_1: \mu \neq  \mu_0 \text{(双边假设检验)} \\
    H_0: \mu \leq \mu_0 & \leftrightarrow  	H_1: \mu > \mu_0 \text{(单边假设检验)}     \\
    H_0: \mu \geq \mu_0 & \leftrightarrow  	H_1: \mu < \mu_0 \text{(单边假设检验)}
\end{align*}


1、方差$\sigma^2 = \sigma^2_0$已知:$Z$检验

在$\mu = \mu_0$时,检验统计量及其分布为:
\begin{gather*}
    Z=\frac{\bar{X}-\mu_0}{\frac{\sigma}{\sqrt{n}}}\sim N(0,1)
\end{gather*}


置信区间为:
\begin{gather*}
    \left(\frac{\bar{X}-\sigma_0}{\sqrt{n}}Z_{\frac{\alpha}{2}}, \frac{\bar{X}+\sigma_0}{\sqrt{n}}Z_{\frac{\alpha}{2}} \right)
\end{gather*}


拒绝域为:
\begin{gather*}
    \left|Z\right|>z_{\frac{\alpha}{2}} \\
    Z >  z_\alpha\\
    Z < -z_\alpha
\end{gather*}


2、方差$\sigma^2$未知:$t$检验

在$\mu = \mu_0$时,检验统计量及其分布为:
\begin{gather*}
    T=\frac{\bar{X}-\mu_0}{\frac{S}{\sqrt{n}}}\sim t(n-1)
\end{gather*}


置信区间为:
\begin{gather*}
    \left( \bar{X}-\frac{S}{\sqrt{n}}t_{\frac{\alpha}{2}}(n-1), \bar{X}+\frac{S}{\sqrt{n}}t_{\frac{\alpha}{2}}(n-1) \right)
\end{gather*}


拒绝域为:
\begin{gather*}
    \left|T\right|>t_{\frac{\alpha}{2}}\left(n-1\right) \\
    T>t_\alpha\left(n-1\right) \\
    T<-t_\alpha\left(n-1\right)
\end{gather*}


\subsubsection{方差$\sigma^2$的检验}

\paragraph{一元}


\begin{gather*}
    H_0: \sigma^2=\sigma_0^2 \leftrightarrow H_1: \sigma^2\neq\sigma_0^2 \\
\end{gather*}


1、均值$\mu$已知

检验统计量及其分布为:
\begin{gather*}
    \chi^2=\frac{1}{\sigma_0^2}\sum_{i=1}^{n}{(X_i-\mu)^2}\sim \chi^2(n)
\end{gather*}


拒绝域为:
\begin{gather*}
    \chi^2>\chi_{\frac{\alpha}{2}}^{2}(n)\\
    \chi^2<\chi_{1-\frac{\alpha}{2}}^{2}(n)
\end{gather*}


2、均值$\mu$未知

检验统计量及其分布为:
\begin{gather*}
    \chi^2=\frac{1}{\sigma^2}\sum_{i=1}^{n}{X_i- \bar{X}} =\frac{(n-1)S^2}{\sigma^2} \sim \chi^2(n-1)
\end{gather*}


置信区间为:
\begin{gather*}
    \left( \frac{(n-1)S^2}{\chi_{\frac{\alpha}{2}}^{2} (n-1)}, \frac{(n-1)S^2}{\chi_{1-\frac{\alpha}{2}}^{2} (n-1)} \right)
\end{gather*}


拒绝域为:
\begin{gather*}
    \chi^2>\chi_{\frac{\alpha}{2}}^2(n-1) 或 χ^2<χ_{1-\frac{\alpha}{2}}^2(n-1)
\end{gather*}













\subsection{两个正态总体}





\subsubsection{均值$\mu$的检验}

\paragraph{一元}
$X\sim N\left(\mu_1,\sigma_1^2\right),\ Y\sim N\left(\mu_2,\sigma_2^2\right)$,两样本相互独立

\begin{align*}
    H_0:\mu_1=\mu_2    & \Leftrightarrow H_1:\mu_1\neq\mu_2 \\
    H_0:\mu_1\le\mu_2  & \Leftrightarrow H_1:\mu_1>\mu_2    \\
    H_0:\mu_1\geq\mu_2 & \Leftrightarrow H_1:\mu_1<\mu_2
\end{align*}


1、$\sigma_1^2,\sigma_2^2$已知,$\sigma_1^2=\sigma_2^2$

检验统计量及其分布:
\begin{gather*}
    U=\frac{\bar{X}-\bar{Y}}{\sqrt{\frac{\sigma_1^2}{n}+\frac{\sigma_2^2}{m}}}\sim N(0,1)
\end{gather*}


拒绝域为:
\begin{gather*}
    \left|U\right|>u_{\frac{\alpha}{2}}
\end{gather*}


2、$\sigma_1^2,\sigma_2^2$未知,$\sigma_1^2=\sigma_2^2$

检验统计量及其分布:
\begin{gather*}
    T=\frac{\bar{X}-\bar{Y}-(\mu_1-\mu_2)}{\sqrt{\frac{\left(n_1-1\right)S_1^2+\left(n_2-1\right)S_2^2}{n_1+n_2-2}}\sqrt{\frac{1}{n_1}+\frac{1}{n_2}}}\sim t(n_1+n_2-2)
\end{gather*}


拒绝域为:
\begin{gather*}
    \left|T\right|>t_{\frac{\alpha}{2}}(n_1+n_2-2)
\end{gather*}


3、$\sigma_1^2\neq\sigma_2^2$

\paragraph{多元}


\begin{gather*}
    \mathbf{X}_{a}=(X_{a1},X_{a2},\cdots,X_{ap})^\prime \sim N_p(\bm{\mu}_1,\bm{\Sigma}_1), a=1,2,\cdots,n_1\\
    \bm{Y}_a=\left(Y_{a1},Y_{a2},\cdots,Y_{ap}\right)' \sim N_p(\bm{\mu}_2,\mathbf{\Sigma}_2), a=1,2,\cdots,n_2 \\
    \mathbf{X}_{a}与\mathbf{Y}_a\text{相互独立}, n_1>p, n_2>p\\
    \bar{\bm{X}}=\frac{1}{n_1}\sum_{a=1}^{n_1}{X_a}, \bar{\bm{Y}}=\frac{1}{n_2}\sum_{a=1}^{n_2}{Y_a}
\end{gather*}



\begin{align*}
    H_0:\bm{\mu}_1=\bm{\mu}_2    & \Leftrightarrow H_1:\bm{\mu}_1\neq\bm{\mu}_2 \\
    H_0:\bm{\mu}_1\le\bm{\mu}_2  & \Leftrightarrow  H_1:\bm{\mu}_1>\bm{\mu}_2   \\
    H_0:\bm{\mu}_1\geq\bm{\mu}_2 & \Leftrightarrow H_1:\bm{\mu}_1<\bm{\mu}_2
\end{align*}


1、$\bm{\Sigma}_1,\bm{\Sigma}_2$已知,$\bm{\Sigma}_1=\bm{\Sigma}_2$

检验统计量及其分布:
\begin{gather*}
    T_0^2 =\frac{n_1\cdot n_2}{n_1+n_2}\left(\bar{\bm{X}}-\bar{\bm{Y}}\right)^\prime\bm{\Sigma}^{-1}\left(\bar{\bm{X}}-\bar{\bm{Y}}\right)=\chi^2\left(p\right)
\end{gather*}

2、$\bm{\Sigma}_1,\bm{\Sigma}_2$未知,$\bm{\Sigma}_1=\bm{\Sigma}_2$

3、$\bm{\Sigma}_1\neq\bm{\Sigma}_2$

\subsubsection{方差$\sigma^2$的检验}

\paragraph{一元}

\begin{gather*}
    H_0:\sigma^2=\sigma_0^2\Leftrightarrow H_1:\sigma^2\neq\sigma_0^2
\end{gather*}


1、均值$\mu$未知

检验统计量及其分布:
\begin{gather*}
    F=\frac{S_1^2}{S_2^2}\sim F(n_1-1,n_2-2)
\end{gather*}


拒绝域为:
\begin{gather*}
    F>F_{\frac{\alpha}{2}}(n_1-1,n_2-1)
    \text{或}
    F<F_{\frac{\alpha}{2}}(n_1-1,n_2-1)
\end{gather*}


\paragraph{多元}

\begin{gather*}
    H_0:\bm{\Sigma}_1^2=\bm{\Sigma}_2^2	\Leftrightarrow H_1:\bm{\Sigma}_1^2\neq\bm{\Sigma}_2^2
\end{gather*}



















\subsection{多个正态总体}






多个正态总体:方差分析(analysis of variance,ANOVA)

并不是两两比较均值,而是通过方差衡量均值

\subsubsection{均值$\mu$的检验}

\paragraph{一元}
$k$个正态总体$N\left(\mu_1,\sigma_1^2\right),\ \cdots,N\left(\mu_k,\sigma_k^2\right)$,$k$个样本相互独立,第$i$个样本有$n_i$个个体

${\bar{X}}_i=\frac{1}{n_i}\sum_{j=1}^{n_i}X_i^{(j)}$,
$\bar{\bar{X}}=\frac{1}{n}\sum_{i=1}^{k}\sum_{j=1}^{n_i}{X_i^j},n=n_1+n_2+\cdots+n_k$

$
    SST=\sum_{i=1}^{k}\sum_{j=1}^{n_i}{\left(X_i^j-\bar{\bar{X}}\right)^2}, SSA=\sum_{i=1}^{k}n_i\left(X_i-\bar{\bar{X}}\right) , SSE=\sum_{i=1}^{k}\sum_{j=1}^{n_i}{\left(X_i^j-\bar{\bar{X}}\right)^2}
$

\begin{gather*}
    H_0:\mu_1=\mu_2=\cdots=\mu_k\Leftrightarrow H_1:\mu_1,\mu_2,\cdots,\mu_k\text{不全相等} \\
\end{gather*}


1、$\sigma_1^2=\sigma_2^2=\cdots=\sigma_k^2$

检验统计量及其分布:
\begin{gather*}
    F =\frac{MSA}{MSE} =\frac{SSA/(k-1)}{SSE/(n-k)} =\frac{\sum_{i=1}^{k}{m(\bar{X}_i-\bar{\bar{X}})/(k-1)}}{\sum_{i=1}^{k}\sum_{j=1}^{m}{(X_{ij}-\bar{X}_i)^2/(n-k)}} \sim F(k-1,n-k)
\end{gather*}

\par F值越大越背离原假设
\par MSA:组内均方(mean square)
\par MSE:组间均方

\paragraph{多元}
各总体样品的均值向量

$\mathbf{T}=\mathbf{A}+\mathbf{E}$,$\mathbf{A}$为组间离差阵,$\bm{E}$为组内离差阵,$\bm{T}$为总离差阵

\begin{gather*}
    H_0:\bm{\mu}_1=\bm{\mu}_2=\cdots=\bm{\mu}_k	\Leftrightarrow H_1:\bm{\mu}_1,\bm{\mu}_2,\cdots,\bm{\mu}_k不全相等 \\
\end{gather*}


1、$\bm{\sigma}_1^2=\bm{\sigma}_2^2=\cdots=\bm{\sigma}_k^2$

检验统计量及其分布:
\begin{gather*}
    \bm{\Lambda}=\frac{\left|\bm{E}\right|}{\left|\bm{T}\right|}=\frac{\left|\bm{E}\right|}{\left|\bm{A}+\bm{E}\right|} \sim\bm{\Lambda}(p,n-k,k-1)
\end{gather*}


\subsubsection{方差$\sigma^2$的检验}

\paragraph{一元}

\begin{gather*}
    H_0:\sigma^2=\sigma_0^2 \Leftrightarrow	H_1:\sigma^2\neq\sigma_0^2
\end{gather*}


\paragraph{多元}

\begin{gather*}
    H_0:\bm{\Sigma}_1=\bm{\Sigma}_2=\cdots=\bm{\Sigma}_k\Leftrightarrow H_1:\left\{\bm{\Sigma}_i\right\}\text{不全相等}
\end{gather*}















\section{非参数检验}

适用于定类、定序数据、离散变量

假设是关于总体其他的统计特征(如总体分布、独立性等)















\part{自然与工程科学}



\part{人文与社会科学}


\chapter{商品}




\begin{equation*}
    \text{商品}\mindmap{
        \text{价值}\mindmap{
            \text{原因—劳动}\mindmap{
                \text{生产}\mindmap{
                    \text{具体劳动}\\
                    \text{抽象劳动}
                }
                \text{交换}\mindmap{
                    \text{个别劳动}\\
                    \text{社会劳动}
                }
            }
            \text{衡量—价值量}\mindmap{
                \text{决定}\mindmap{
                    \text{劳动时间}\mindmap{
                        \text{个别劳动时间}\\
                        \text{社会劳动时间}
                    }
                    \text{劳动复杂程度}\mindmap{
                        \text{简单劳动}\\
                        \text{复杂劳动}
                    }
                }
                \text{指标—劳动生产率}
            }
        }
        \text{使用价值}
    }
\end{equation*}



定义:商品是用来交换的劳动产品
\informationBox{
    * 自给的农产品:交换(×),劳动产品(√)
}


\section{价值构成}



 {\noindent
  \begin{tabu}{|c|X|X|}
      \hline
                                 & \multicolumn{1}{c|}{价值}  & \multicolumn{1}{c|}{使用价值} \\ \hline
      定义
                                 & \begin{tabu}{X}
          凝结在商品中的无差别的一般人类劳动
      \end{tabu} &
      \begin{tabu}{X}
          物品和服务能够满足人们某种需要的属性,即物品和服务的有用性 \\
          * 商品使用价值与一般物品使用价值的区别                     \\
          \tablelist{(1)}{0.5\columnwidth}{
              \item 来源:是劳动产品的使用价值
              \item 过程:必须通过交换让渡
              \item 去处:对他人、社会有用,不是对生产者
          }
      \end{tabu}                                                              \\\hline
      \multicolumn{1}{|c|}{对立} & \multicolumn{2}{l|}{
          \tablelist{(1)}{\columnwidth}{
              \item 生产者和消费者不能同时占有价值和使用价值
              \item 价值:反映人与社会的关系
              \item 使用价值:反映人与自然的物质关系
      }}                                                                                      \\ \hline
      \multicolumn{1}{|c|}{统一} & \multicolumn{2}{l|}{
          \tablelist{(1)}{\columnwidth}{
              \item 使用价值是价值的基础、物质载体
              \item 通过交换,生产者消费者互相让渡使用价值和价值
          }
      }                                                                                       \\ \hline
  \end{tabu}
 }





\subsection{交换价值}

定义:商品能够通过买卖具有和其他商品相交换的属性

\informationBox{
    \par * 表现:一种使用价值同另一种使用价值相交换的量的比例关系
    \par * 关系
    \begin{enumerate}[(1)]
        \item 使用价值是交换价值的物质承担者
        \item 价值是交换价值的基础,交换价值是价值的表现形式(价值自己无法表示自己的价值大小,但交换价值的表示可能不一致)
    \end{enumerate}
    *  商品流通
    \begin{enumerate}[(1)]
        \item 可能性:价值量相同
        \item 必要性:使用价值不同
    \end{enumerate}
}


\section{价值原因:劳动}

\subsection{商品生产}

(劳动二重性)


{\noindent
\begin{tabu}{|c|X|X|}
    \hline
                               & \multicolumn{1}{c|}{具体劳动}                   & \multicolumn{1}{c|}{抽象劳动}                                     \\ \hline
    定义                       & \multicolumn{1}{l|}{在一定具体形式下进行的劳动} & \multicolumn{1}{l|}{撇开劳动的特定具体形式的无差别的一般人类劳动} \\\hline
    \multicolumn{1}{|c|}{对立}
                               & \tablelist{(1)}{0.5\columnwidth}{
        \item 反映人与自然的关系(自然属性)
        \item 创造使用价值,不是唯一源泉
        \item 形式千差万别
    }
                               &
    \tablelist{(1)}{0.45\columnwidth}{
        \item 反映人与社会的关系(社会属性)
        \item 创造价值,是唯一源泉
        \item 无质的差别
    }                                                                                                                                                \\ \hline
    \multicolumn{1}{|c|}{统一} & \multicolumn{2}{l|}{
        \tablelist{(1)}{\columnwidth}{
            \item 是同一劳动的两个方面(不是两次或两种劳动),在时空上不可分割
            \item 具体劳动是抽象劳动的基础
            \item 只有通过商品交换,具体劳动才能还原为抽象劳动
    }}                                                                                                                                               \\ \hline
\end{tabu}
}



\subsection{商品交换}

{\noindent
    \begin{tabu}{|c|X|X|}
        \hline
                                   & \multicolumn{1}{c|}{个别劳动}                 & \multicolumn{1}{c|}{社会劳动}                                 \\\hline
        定义                       & \multicolumn{1}{l|}{商品生产者各自独立的劳动} & \multicolumn{1}{l|}{经过市场交换得到社会承认的个别劳动的汇总} \\\hline
        \multicolumn{1}{|c|}{对立} & \tablelist{(1)}{0.45\columnwidth}{
            \item 表现
            \begin{enumerate}[a.]
                \item 不同所有制的劳动
                \item 私有制条件下,个人劳动即私人劳动
                \item 同一所有制内部,不同的有独立利益的企业
            \end{enumerate}
            \item 个人生产不是直接的社会生产

        }                          & \tablelist{(1)}{0.5\columnwidth}{
            \item 劳动的社会性是劳动的本质特征,是人类社会的本质特征
            \item 在商品经济中,由于社会分工的存在,商品生产者之间是相互联系、相互依存的
        }                                                                                                                                          \\\hline
        \multicolumn{1}{|c|}{统一} & \multicolumn{2}{l|}{
            \tablelist{(1)}{\columnwidth}{
                \item 个人只能为社会和在社会中生产
                \item 通过市场交换个别劳动得到市场认可,成为社会劳动
            }
        }                                                                                                                                          \\\hline
    \end{tabu}
}



\section{价值衡量:价值量}

定义:生产商品所耗费的劳动量,即凝结在商品中的一般人类劳动量

\subsection{决定}

1、劳动时间

{\noindent
\begin{tabu}{|c|X|X|}
    \hline
                               & \multicolumn{1}{c|}{个别劳动时间}                  & \multicolumn{1}{c|}{社会劳动时间} \\\hline
    \multicolumn{1}{|c|}{定义} & \begin{tabu}{X}
        独立的生产者在个别的生产条件下生产商品所耗费的劳动时间\end{tabu}
                               & \begin{tabu}{X}
        在现有的正常的生产条件下,在社会平均的劳动熟练程度和劳动强度下制造某种使用价值所需要的劳动时间 \\\\
        \tablelist{(1)}{0.5\columnwidth}{
            \item 生产同种商品的不同生产者之间形成的,涉及的是同种商品生产上的劳动耗费
            \item 生产不同商品的生产者之间形成的社会必要劳动时间,涉及社会总劳动时间在各种商品上的分配
        }
    \end{tabu}                                                             \\\hline
    \multicolumn{1}{|c|}{对立} & \tablelist{(1)}{0.5\columnwidth}{
        \item 供给角度
        \item 单个相同商品的价值
        \item 考察价值的决定
    }                          & \tablelist{(1)}{0.5\columnwidth}{
        \item 需求角度
        \item 部门商品的总价值
        \item 考察价值的实现
        \item 反映不同商品的使用价值量被社会接受的程度
    }                                                                                                                   \\\hline
    \multicolumn{1}{|c|}{统一} & \multicolumn{2}{l|}{\tablelist{(1)}{\columnwidth}{
            \item 二者共同决定商品的价值
    }}                                                                                                                  \\\hline
\end{tabu}
\ \\
}




2、劳动的复杂程度

{\noindent
\begin{tabu}{|c|X|X|}
    \hline
                               & \multicolumn{1}{c|}{简单劳动}                      & \multicolumn{1}{c|}{复杂劳动} \\\hline
    \multicolumn{1}{|c|}{定义} & \begin{tabu}{X}
        不经过专门训练和学习就能胜任的劳动\end{tabu}
                               & \begin{tabu}{X}
        需要经过专门训练和学习,具有一定技术专长的劳动 \\\\
        \tablelist{(1)}{0.5\columnwidth}{
            \item 科技劳动:科技本身不创造价值,掌握和运用科技的劳动者的活劳动创造价值
            \item 管理劳动
        }
    \end{tabu}                                                         \\\hline
    \multicolumn{1}{|c|}{对立} & \tablelist{(1)}{0.5\columnwidth}{
        \item 收入分配少
    }                          & \tablelist{(1)}{0.5\columnwidth}{
        \item 收入分配多
    }                                                                                                               \\\hline
    \multicolumn{1}{|c|}{统一} & \multicolumn{2}{l|}{\tablelist{(1)}{\columnwidth}{
            \item 复杂劳动要转化为简单劳动来比较
            \item 简单劳动、复杂劳动是相对的,随着科技发展和文化教育水平的提高,复杂劳动正变为简单劳动
    }}                                                                                                              \\\hline
\end{tabu}
}




\subsection{指标}

\subsubsection{劳动生产率}

定义:劳动者在一段时间内生产某种使用价值的效率

\informationBox{
    * 劳动生产率=产品量/劳动时间
    \begin{enumerate}[(1)]
        \item 单位时间内生产的产品数量
        \item 生产单位产品所耗费的劳动时间
    \end{enumerate}
    * 规律
    \begin{enumerate}[(1)]
        \item 同一劳动在同样的时间内提供的价值量相同
        \item 劳动生产率同商品的使用价值量成正比,同商品的价值量成反比
    \end{enumerate}
    * 影响因素
    \begin{enumerate}[(1)]
        \item 劳动者:劳动者的平均熟练程度
        \item 劳动工具:科技发展水平,及其在生产中的应用程度
        \item 劳动对象:劳动对象的状况
        \item 管理:生产过程的社会结合(分工协作、劳动组织、生产管理)形式(生产过程各要素的契合程度)
        \item 自然条件(农业受的影响大)
    \end{enumerate}

}


















\chapter{劳动力}




\chapter{资本}



定义:不断在运动中谋求自身增殖的价值,是价值的一种特殊形式
\\

\informationBox{
    \begin{enumerate}[1.]
        \item 一般性:生产要素,市场经济的一种基本要素;
        \item 特殊性:不同社会经济制度下的资本反映着不同的社会生产关系
              \begin{enumerate}[(1)]
                  \item 增殖性:是区别于一般商品和货币的根本特征
                        \par * 货币流通公式与资本流通公式的区别
                        \begin{enumerate}[a.]
                            \item 流通形式:货币:$W1-G-W2$;资本:$G-W-G'$
                            \item 流通目的:货币:获得使用价值;资本:价值增殖
                            \item 流通限度:货币:有限的;资本:无限的($G—W—G'—W—G''……$)
                        \end{enumerate}
                  \item 运动性
                  \item 返还性:主观目的上
                  \item 风险性:要有前瞻性
              \end{enumerate}
    \end{enumerate}
}



\section{实体资本}

定义:能定期带来收入的,以实物或货币形式表现的资本
\\

\begin{enumerate}[1.]
    \item 货币资本形态
          \begin{enumerate}[(1)]
              \item 货币资本:是资本最一般的和初始的形态
          \end{enumerate}
    \item 实物资本形态
          \begin{enumerate}[(1)]
              \item 实物资本:以物质形态表现的资本,包括投入生产过程和流通过程的一切物的要素和待售的产出品,又称物质资本
              \item 分类:生产要素(生产资本)和待售产出品(商品资本)
          \end{enumerate}
    \item 无形资本形态
          \begin{enumerate}[(1)]
              \item 无形资产:以知识形态存在的特有经济资源
              \item 构成:专利权、商标权、版权、著作权、特许经营权、商誉、技术秘密等
          \end{enumerate}

\end{enumerate}



\section{虚拟资本}

定义:能定期带来收入,以有价证券形式表现的资本
\\

\begin{enumerate}[1.]
    \item 形式:
          \begin{enumerate}[(1)]
              \item 信用形式的虚拟资本
              \item 收入资本化形式的虚拟资本
                    \par * 现代分类
              \item 货币证券:银行券、银行票据(期票、本票等)
              \item 资本证券:股票、公司债券
          \end{enumerate}
    \item 特点:
          \begin{enumerate}[(1)]
              \item 经济性
                    \begin{enumerate}[a.]
                        \item 价值符号及它们的交换也是以劳动价值为基础的,没有价值及价值交换,谈不上它的经济性
                        \item 无论是纸币,还是股票等各种有价票证,它们的发行和流通基础就是价值和信誉,它们代表的是实体价值,是实体价值的代表,并且还可以为实体经济服务
                    \end{enumerate}
              \item 虚拟性
                    \begin{enumerate}[a.]
                        \item 交换物在形态上是虚拟的而非实物的,脱离了价值实体,成了实体价值的影子,是虚拟现实
                        \item 资金作为价值的表现,只有当它没有与实物商品进行交换,而只与它的同类即价值符号进行交换时。它才能被归为虚拟经济范畴
                    \end{enumerate}
          \end{enumerate}
    \item 产生条件:
          \begin{enumerate}[(1)]
              \item 资本虚拟化:指在当代发达市场经济体制之下,与实体资本相对应的虚拟资本数量不断膨胀,种类不断演化,并与个别实体资本逐渐脱离关系的过程和趋势。具体表现为各类金融市场的不断扩张,包括股票、债券、期货、期权以及其他金融衍生品市场等。
              \item 前提:货币的虚拟化
                    \begin{enumerate}[a.]
                        \item 货币摆脱贵金属束缚,不以有价值的实际资产作为货币材料
                        \item 在商业信用和银行信用不断发展的基础上,产生了代替金银及其铸币进行流通的信用流通工具,货币进一步虚拟化
                        \item 在高度发达的信用制度基础上发展的
                    \end{enumerate}
              \item 根据:借贷资本信用关系
                    \begin{enumerate}[a.]
                        \item 借贷资本是通过资本使用权的有偿转让,凭借债券来获得定期收入的
                        \item 利息是资本所有权的果实,借贷资本成为独立的收入来源
                    \end{enumerate}
              \item 基础:社会信用制度的逐步完善
                    \begin{enumerate}[a.]
                        \item 银行制度的发展,直接推动了货币的虚拟化
                        \item 货币的虚拟化也可以视为银行制度发展的重要动力,两者之间存在互动的关系
                        \item 在货币虚拟化基础上发展出来的资本虚拟化,同样离不开社会信用制度的发展和完善
                    \end{enumerate}

          \end{enumerate}
\end{enumerate}




\section{关系}



\begin{enumerate}[1.]
    \item 统一:
          \begin{enumerate}[(1)]
              \item 实体资本是虚拟资本的客观基础(虚拟资本的存在和运动必然要以它所表现的实体资本为基础)
                    \begin{enumerate}[a.]
                        \item 实体资本的运动状况决定虚拟资本的运行状况,证券发行者的生产经营状况决定着证券投资者的收益
                        \item 实体资本运用的规模影响着虚拟资本的发行规模,社会再生产规模决定着证券投资规模
                        \item 实体资本的循环周期影响着虚拟资本的周期波动
                    \end{enumerate}
              \item 虚拟资本对实体资本有制约作用(价值发现和风险化解)
                    \begin{enumerate}[a.]
                        \item 影响实体资本运用的过程和规模
                        \item 其流向影响着实体资本的分配比例和结构
                        \item 扩大了实体资本的活动范围
                    \end{enumerate}
          \end{enumerate}
    \item 对立:
          \begin{enumerate}[(1)]
              \item 虚拟资本价格不由实体资本价值决定,由预期收入和平均利率决定(不一定由经营状况决定)
              \item 虚拟资本价格变动相对独立,可能不随实体资本价值的变动而变动,而与其背离
          \end{enumerate}
\end{enumerate}



























\chapter{货币}




\begin{enumerate}[1.]
    \item 定义:固定充当一般等价物的特殊商品(money)
    \begin{enumerate}[(1)]
        \item 属性:
        \begin{enumerate}[a.]
            \item 名义价值(市场价值):该货币与其他货币的比值
            \item 实际价值(内在价值):货币材料(作为商品)的价值
        \end{enumerate}
        \item 货币与商品伴生,在商品交换过程中产生的,是商品交换发展到一定阶段的必然产物
        \item 具有一般商品的特性,又执行一般等价物的职能
    \end{enumerate}
    \item 规律:外在形式不断变化,货币本质、职能不变
    \begin{enumerate}[(1)]
        \item 静态:
        \begin{enumerate}[a.]
            \item 内在价值支撑信用
            \item 国家制度支撑主权货币信用
            \item 内在价值和主权信用支撑国际货币信用
        \end{enumerate}

        \item 动态:价值属性不断消失(价值和使用价值分离,弹性供给),支付属性不断增强(制度共识+价值共识)
        \begin{enumerate}[a.]
        \item 货币越来越摆脱具体的物质实体,成为单纯代表着按照一定单位对社会财富占有数量的数字
           \item 货币本身的价值基础越来越脱离作为价值实体的无差别人类劳动,越来越依赖机构与制度(中央银行制度)的作用来赋予货币所能代表的价值
        \end{enumerate}
    \end{enumerate}
   
\end{enumerate}





\section{货币分类}







\subsection{价值形式}

其发展出于降低交易成本、更方便的需要


\begin{enumerate}[1.]
    \item 简单的、个别的或偶然的价值形式
          \begin{enumerate}[(1)]
              \item 1只绵羊=2把斧子
              \begin{enumerate}[a.]
                  \item 1只绵羊:
                        \begin{enumerate}[(a)]
                            \item 相对价值形式(相对:相对于等价形式;劳动生产率变化,价值量也会变化)
                            \item 主动地位
                        \end{enumerate}
                  \item 2把斧子:
                        \begin{enumerate}[(a)]
                            \item 等价形式:使用价值成为价值的表现形式,具体劳动成为抽象劳动的表现形式,私人劳动成为社会劳动的表现形式
                            \item 被动地位
                        \end{enumerate}
              \end{enumerate}
              \item 交换行为、比例、对象等具有偶然性
          \end{enumerate}
    \item 总和的或扩大的价值形式
          \begin{enumerate}[(1)]
              \item 优点:等价物的数量扩大
              \item 缺点:没有形成统一等价物
              \item 价值第一次表现为无差别的劳动凝结
          \end{enumerate}
    \item 一般价值形式
          \begin{enumerate}[(1)]
              \item 优点:分离出公认的一般等价物(质的变化),产生物物交换的媒介
              \item 缺点:等价物商品太多,实物货币的不足
          \end{enumerate}
    \item 货币形式
          \begin{enumerate}[(1)]
              \item 优点:金属货币(其优点也是相对的)固定充当一般等价物,更加便利
              \item 最发达的价值形式,是商品的交换价值的最高形式
              \item 商品与商品的对立表现为商品与货币的对立
          \end{enumerate}
\end{enumerate}













\subsection{物质形式}


\begin{equation*}
    \text{物质形式}\mindmap{
        \text{实体货币}\mindmap{
            \text{商品货币}\mindmap{
                \text{实物货币} \\
            \text{金属货币}
            }
            \text{现钞}\mindmap{
                \text{纸质货币}
            }
        }
        \text{虚拟货币}\mindmap{
            \text{存款货币}\\
            \text{电子货币}\\
            \text{数字货币}
        }
    }
\end{equation*}


\subsubsection{实体货币}

定义:货币币材可触摸的有形货币(physical currency),又称为真实货币
\informationBox{
    * 物理属性
}



\begin{enumerate}[1.]
    \item 商品货币:有内在价值的货币(commodity money/currency)
          \begin{enumerate}[(1)]
              \item 实物货币
                    \begin{enumerate}[a.]
                        \item 定义:以自然界存在的某种物品(非贵金属)或人们生产的某种物品来充当货币(physical money)
                        \item 不足:不易分割、储存、运输
                    \end{enumerate}
              \item 金属货币
                    \begin{enumerate}[a.]
                        \item 定义:以金属如铜、银、金等作为材料的货币(metallic money)
                        \item 分类:称量货币,铸币(coins)
                        \item 优点:价值稳定;易于分割;易于贮藏
                    \end{enumerate}
          \end{enumerate}
    \item 现钞:包括纸币、硬币的法令货币(fiat money/currency),又称法偿货币(legal tender)
    \par 分类:可兑换的信用货币;不可兑换的信用货币
          \begin{enumerate}[(1)]
              \item 纸质货币
                    \begin{enumerate}[a.]
                        \item 定义:包括国家答应的纸质货币符号、商人发行的兑换券和银行发行的纸质信用货币等(paper money/currency),简称纸币
                              \begin{enumerate}[(a)]
                                  \item 纸币本身没有价值,仅是价值符号
                                  \item 纸币只有代表金量才有价值
                                  \item 纸币由国家通过法定途径发行,以国家信用做担保,一旦进入流通会受到流通规律的支配(脱离一定的表现材料,不再依赖生产货币的劳动,更多地依赖机构和制度)
                              \end{enumerate}
                        \item 分类:纸币、银行券(bank note)等
                        \item 原因:代替金属货币执行流通手段职能
                              \begin{enumerate}[(a)]
                                  \item 金银自然属性与充当价值符号的社会属性的矛盾
                                  \item 商品转化为作为流通手段的货币只是价值形态的变化,金属货币转手仍不方便
                                  \item 商品生产流通的扩大化要求货币量相应增加,蕴藏、开采、提炼跟不上
                                  \item 信用货币代替金属货币充当支付手段和流通手段的信用证券
                              \end{enumerate}
                    \end{enumerate}
          \end{enumerate}
\end{enumerate}



\subsubsection{虚拟货币}

定义:无实体的“货币”、在某些领域执行货币职能的价值代表物(virtual currency)

\informationBox{
    \par * 广义:央行和私人都能发行
    \par * 狭义:非官方发行
    \par * 分类(发行机构)
    \begin{enumerate}[(a)]
        \item 由游戏平台、网络社交平台发行的虚拟货币:不涉及金融机构,实际上被当作商品出售
        \item 基于银行账户相关联的记账式虚拟货币:电子货币,金融机构发行
        \item 没有发行机构的虚拟货币:比特币等
    \end{enumerate}
    \par * 分类(经济功能)
    \begin{enumerate}[(a)]
        \item 支付工具型:Q 币,法币预付充值
        \item 交易媒介型:魔兽金币,虚拟社区居民之间交易
        \item 促销工具型:折扣积分
        \item 激励合作型:论坛积分,提高网络资源共享程度
    \end{enumerate}
}






\begin{enumerate}[1.]
    \item 存款货币
\begin{enumerate}[(1)]
    \item 定义:能够发挥货币交易媒介和资产职能的银行存款(deposit money)
    \item 经济属性:后期的存款货币为电子货币
    \item 分类:可以直接进行转账支付的活期存款和企业定期存款、居民储蓄存款等
\end{enumerate}
\item 电子货币
    \begin{enumerate}[(1)]
        \item 定义:以金融数字化网络为基础,通过计算机网络系统,以传输电子信息的方式实现支付功能的电子数据(electronic money)
        \item 技术属性
        \item 分类:
        \begin{enumerate}[a.]
            \item 卡基电子货币:借记卡,贷记卡,储值卡等(card-based)
            \item 数基电子货币:网银,电子现金(soft-based)
        \end{enumerate}
        \item 电子货币是信用货币与虚拟货币的过渡阶段(市场形式的转变),是记账式虚拟货币,法币的电子化,传统金融机构负债的电子化
        \item 对经济的影响:现金先行约束(CIA)的放松 → 减少预防性的现金留存 → 改变货币结构 → 提高金融体系中的金融资源 → 促进投资,进而促进经济增长  
    \end{enumerate}
    \item 数字货币
    \begin{enumerate}[(1)]
        \item 定义:具备某些货币属性的数字化价值代表物(digital money/currency)
        \item 分类:
        \begin{enumerate}[a.]
            \item 央行发行的数字货币(中心化)
            \item 私人发行的数字货币(去中心化)
        \end{enumerate}
    \end{enumerate}
\end{enumerate}




\subsection{流动性}

\informationBox{
    \par 流动性强弱变化导致货币范围变化
    \par 国家金融制度越发达,金融产品越丰富,货币层次就越多
    \par 不同国家货币层次包含内容不同
    \par 金融产品创新和金融环境改变,需重新划分层次
    \par 层次划分只能在一定程度上反映货币流通状况
}

\subsubsection{IMF}

\begin{enumerate}[1.]
    \item 通货:中央银行或财政部发行流通于银行体系以外的现钞(currency,$M_0$)
    \item 可转让存款:存款性公司发行的活期存款、银行本票、旅行本票,邮政储蓄机构发行的可转让存款,非存款性公司发行的旅行支票
    \item 货币市场基金份额
    \item 债务凭证
    \item 存款公司发型的大额存单、商业票据等
\end{enumerate}

\subsubsection{我国}

\par $M_0$ =流通中的现金
\par $M_1$ (货币)=$M_0$+企业活期存款
\par $M_2$ = $M_1$+准货币(企业单位定期存款+城乡居民储蓄存款+证券公司的客户保证金存款+其他存款)
\par 狭义货币$M_1$反映整个社会对商品和劳务服务的直接购买力,广义货币$M_2$反映整个社会潜在的购买力


\informationBox{
其他:
\begin{enumerate}[1.]
    \item 资产负债
    \begin{enumerate}[(1)]
        \item 金属货币:持有者的资产,不是任何人的负债
        \item 纸质货币:持有者的资产,发行者的负债
        \item 可兑换的银行券:持有者的资产,发行者的负债
        \item 中央银行发行的现钞:持有者的资产,中央银行的负债
        \item 商业银行发行的存款货币:持有者的资产,商业银行的负债
        \item 虚拟货币(游戏公司或社交平台发行):持有者的资产,不是销售方的负债
        \item 虚拟货币(电子货币):持有者的资产,发行者的负债
        \item 虚拟货币(数字货币):持有者的资产,不是任何人的负债
    \end{enumerate}
    \item 货币价值
    \begin{enumerate}[(1)]
        \item 商品货币:名义价值与内在价值相一致
        \item 金属货币(银行券):名义价值远小于内在价值
        \item 信用货币(银行券):无内在价值
        \item 信用货币:无内在价值,所以必须垄断货币
    \end{enumerate}
    \item 发行数量
    \begin{enumerate}[(1)]
        \item 商品货币:发行量有限
        \item 信用货币:发行量无限(不考虑经济约束)
        \item 虚拟货币(游戏公司或社交平台发行):发行量无限
        \item 虚拟货币(数字货币):发行量固定
    \end{enumerate}
    \item 实体货币、虚拟货币
    \begin{enumerate}[(1)]
        \item 相互隔绝,不存在兑换关系:在虚拟社区或者游戏之内等封闭环境当中(单机)
        \item 实体货币向虚拟货币的单向兑换:与一般产品和服务的出售没有本质差异
        \item 双向兑换(范围狭小,局部特定人群):兑换的价格或随行就市或相对固定。游戏币等,可以视为可转手的产品或服务的预付款,与一般产品和服务的出售没有本质区别
        \item 双向兑换(较大范围被接受):
        \begin{enumerate}[a.]
            \item 兑换比价波幅大:比特币。小范围内流通;投资品而不是货币
            \item 兑换比价固定:支付宝“零钱”
        \end{enumerate}
    \end{enumerate}
\end{enumerate}
}












\section{作用与影响}

\informationBox{
    \begin{enumerate}[1.]
        \item 货币中性观
              \begin{enumerate}[(1)]
                  \item 认为货币对经济运行没有实质性影响
                  \item 政策意义:货币供给应与潜在经济增长相适应
              \end{enumerate}
        \item 货币非中性观
              \begin{enumerate}[(1)]
                  \item 认为货币对经济运行能够产生实质性影响
                  \item 政策意义:货币供应应逆周期动态调整
              \end{enumerate}
        \item 共识
              \begin{enumerate}[(1)]
                  \item 短期是非中性的,长期是中性的
                  \item 长短界限依赖于工资物价调整的速度(工资、物价黏性)以及信息传递的速度(信息黏性)
              \end{enumerate}
    \end{enumerate}
}


\subsection{货币职能}


\subsubsection{计价标准}

定义:用货币去计算并衡量商品或劳务的价值,从而为商品和劳务的交换标价(standard of value),又称计价单位(unit of account)

\informationBox{
    \begin{enumerate}[(1)]
        \item 价值尺度:货币表现商品的价值(质)、衡量商品价值量的大小(量)的尺度
        \item 商品流通中自发产生的
        \item 观念上的货币
    \end{enumerate}  
}



\subsubsection{交易媒介}

定义:货币在商品交易中作为交换手段、计价标准和支付手段,从而提高交易效率,降低交易成本,便利商品交换的职能(media of exchange)

\informationBox{
    \par 货币独有的、最基本的职能
    \par 交换手段和支付手段决定了货币的交易性需求,与预防性需求和投机性需求一起构成货币总需求
}


\begin{enumerate}[1.]
    \item 交换手段
          \begin{enumerate}[(1)]
              \item 定义:货币在商品交换中作为中介,通过一手交钱一手交货作为商品流通的媒介(means of exchange),又称流通手段
              \item 需要现实的货币
              \item 优点:
                    \begin{enumerate}[a.]
                        \item 打破时空界限,加速流通→形成商品生产者全面的社会联系
                        \item 分离出公认的一般等价物
                    \end{enumerate}
              \item 缺点:
                    \begin{enumerate}[a.]
                        \item  买卖脱节,蕴含流通危机
                    \end{enumerate}
              \item 商品流通:以货币为媒介的商品交换
          \end{enumerate}
    \item 支付手段
          \begin{enumerate}[(1)]
              \item 定义:货币作为延期支付的手段来结清债权债务关系(means of payment)
              \item 形式:清偿债务、支付赋税、工资、租金、利息、捐款、赔偿等
              \item 原因:赊购赊销
              \item 优点:促进商品经济发展
              \item 缺点:潜伏支付危机
              \item 货币支付手段职能是信用货币产生的基础
          \end{enumerate}
\end{enumerate}


\subsubsection{资产职能}

定义:货币可以作为人们总资产的一种存在形式,成为实现资产保值增值的一种手段,又称价值储藏手段(store of value)、贮藏手段

\begin{enumerate}[1.]
    \item 资产职能决定了货币的投机性需求
    \item 货币与其他金融工具的关系
    \begin{enumerate}[(1)]
        \item 联系:货币是一种金融工具
        \item 区别:货币收益性最低,流动性最高
    \end{enumerate}
    \item 作用:“蓄水池”
    \begin{enumerate}[(1)]
        \item 社会财富的一般代表
        \item 自发调节流通中的货币量
    \end{enumerate}
    \item 充当贮藏手段的货币必须是足值货币
    \item 现实中纸币可以以储蓄的形式贮藏,但不同于金属货币的贮藏,纸币最终还需要进入流通领域,而贮藏金属货币是退出流通领域,所以纸币没有贮藏手段的功能 
\end{enumerate}


\subsubsection{世界货币}

定义:货币在世界市场充当一般等价物的职能

\begin{enumerate}[1.]
    \item 原因:跨国贸易的需要
    \item 职能
    \begin{enumerate}[(1)]
        \item 支付手段:平衡贸易差额
        \item 购买手段:单方面向外国购买商品
        \item 贮藏:作为社会财富的代表在国家间转移
    \end{enumerate}
\end{enumerate}



\subsection{经济影响}

\begin{enumerate}[1.]
    \item 积极:
          \begin{enumerate}[(1)]
              \item 克服物物交换困难,提高交换效率,商品流通(交换手段)
              \item 便于价值衡量和交换比率确定(计价标准)
              \item 通过支付充抵部分交易金额,节约流通费用(支付手段)
              \item 流动性高,丰富了贮藏手段和投资形式(资产职能)
              \item 支付手段和存款等促进社会资金集中,便于社会化大生产(支付手段、资产职能)
          \end{enumerate}
    \item 消极:
          \begin{enumerate}[(1)]
              \item 交换过程:买卖分离,易发生商品买卖脱节和供求失衡
              \item 支付手段:债务链条复杂,易产生债务危机
              \item 跨期支付:财政超分配和信用膨胀,易造成通货膨胀
          \end{enumerate}
\end{enumerate}



\subsection{其他影响}
(一)积极:

(1) 扩大了人类活动范围
(2) 激发人类想象力和创造力
(3) 激发了人们创造财富的欲望
(二)消极:

(1) 货币拜物教扭曲人类的思想与行为


\subsection{影响条件}
(1) 携带方便
(2) 贮藏安全:身边→专门机构→云端;实体→电子(加密)
(3) 易于与其他形式资产转换,可分割性
(4) 币值稳定
(5) 货币流通量的调节机制,供给弹性











\chapter{信用}




\chapter{利率}


利率:借贷期满的利息总额与贷出本金总额的比率

一、产生原因

\begin{enumerate}[(1)]
    \item 收益资本化:利息转化为收益的一般形态(capitalization of return)
          \begin{enumerate}[a.]
              \item 过程:利息本来以借贷为前提,源于产业利润的利息,逐渐被人们从借贷和生产活动中抽象出来,被赋予与借贷、生产活动无关的特性,逐渐被人们认为与资本的所有权联系,认为是资本所有权的必然产物
              \item 结果:各种有收益的事物都可以通过收益与利率的对比进行资本定价
          \end{enumerate}
    \item 货币的时间价值:同等金额的货币其现在的价值要大于其未来的价值(time value of money)
          \begin{enumerate}[a.]
              \item 机会成本:对货币的占用具有机会成本;对当前消费推迟的时间补偿
              \item 风险溢价:需要对通胀损失、投资风险进行补偿
          \end{enumerate}
\end{enumerate}


二、产生结果:利息


\begin{enumerate}[(1)]
    \item 定义:借贷关系中资金借入方支付给资金贷出方的报酬(interest)
          \begin{enumerate}[(a)]
              \item 是货币时间价值的具体体现
          \end{enumerate}
    \item 实质
          \begin{enumerate}[a.]
              \item 非货币因素
                    \begin{enumerate}[(a)]
                        \item 时差利息论:利息来源于同种和同量物品价值上的差别,这种差别由二者在时间上的差别完成;生产的费时性决定了现在物品和未来物品的差额,利息实质上来源于这种差额
                        \item 等待论:利息为纯息,利润为毛利息,利息是节欲和等待的报酬
                        \item 马克思:利息实质来源于劳动创造的价值,体现剥削或分配关系
                    \end{enumerate}
              \item 货币因素
                    \begin{enumerate}[(a)]
                        \item 流动性偏好利息论:利息是一定时期内放弃货币流动性的报酬
                    \end{enumerate}
              \item 现代
                    \begin{enumerate}[(a)]
                        \item 利息是投资者让渡资本使用权而索取的补偿或报酬。补偿包括对放弃投资于无风险资产机会成本的补偿和对风险的补偿。
                        \item 风险资产的收益率=无风险利率+风险溢价
                    \end{enumerate}
          \end{enumerate}
\end{enumerate}


三、计算
\informationBox{
\begin{enumerate}[(1)]
    \item 现金流贴现分析:计算现值的过程(discounted cash flow analysis)
          \par 贴现率:贴现(discount)时所使用的利率(discount rate),又称折现率
          \par 现金流难以预测,影响也相对短暂;折现率更为重要
    \item 收益率(yield)、回报率(returns)本质都是利率
          \par $$RET=\frac{C+P_{t+1}-P_t}{P_t}=\frac{C}{P_t}+\frac{P_{t+1}-P_t}{P_t}$$
          \par $C$ :年利息
          \par $\frac{C}{P_t}$ :当期收益率:每年的利息收入与证券购买价格的比率(股息收益率)
          \par $\frac{P_{t+1}-P_t}{P_t}$ :资本利得(损失)率:证券价格变动相对于购进价格的比率
          \par 复合收益率:考虑利息再投资的收益率
\end{enumerate}
}



\begin{enumerate}[(1)]
    \item 单利(simple interest)
          \begin{gather*}
              APR=\frac{I}{PV}·\frac{1}{n} =\frac{FV-PV}{PV}·\frac{1}{n}=\frac{r(n)}{n}
          \end{gather*}
          \par $APR$:年化百分比利率,常为投资报价利率(annual percentage rate)
          \par $I$:利息额
          \par $n$:借贷年数
          \par $PV$:现值(present value)
          \par $FV$:终值(final value)
          \par $r(n)$:n年的总利率
    \item 复利
          \begin{enumerate}[a.]
              \item 一般复利
                    \begin{gather*}
                        EAR=\left(1+\frac{APR}{m}\right)^{m\cdot n}-1=\left[1+r\left(n\right)\right]^{1/n}-1
                    \end{gather*}
                    \par $EAR$ :有效年利率,常为投资结算利率(effective annual rate)
                    \par $r$ :一般复利年利率
                    \par $m$ :每年复利次数
                    \par $n$ :借贷年数
                    \par $r(n)$ : $n$年的总利率
                    \par $\left(1+\frac{r}{m}\right)^{m·n}$ :终值复利因子
                    \par $d(m,n)=\frac{1}{\left(1+\frac{r}{m}\right)^{m·n}}$ :现值复利因子/贴现因子/折现因子(discount factor),随时间严格递减,体现了货币的时间价值
              \item 连续复利(continuous compounding)
                    \begin{gather*}
                        I=PV(e^{EAR\times n}-1)
                    \end{gather*}
                    \par $r$:连续复利年利率
                    \par 一般复利利息=连续复利利息:可求出等价的连续复利利息
                    \par 连续复利:$1+EAR=e^{APR}$
          \end{enumerate}
\end{enumerate}






\section{利率分类}



\subsection{市场利率}


\begin{enumerate}[1.]
    \item 计息时间
          \begin{enumerate}[(1)]
              \item 年利率:\%(annual interest rate)
              \item 月利率:‰(monthly interest rate)
              \item 日利率:‱(daily interest rate)
          \end{enumerate}
    \item 决定方式
          \begin{enumerate}[(1)]
              \item 市场利率:按照市场规律自发变动的利率(market interest rate)
              \item 官定利率:一国货币管理部门或者中央银行所规定的利率(official interest rate)
              \item 公定利率:由非政府部门的民间组织为维护公平竞争所确定的属于行业自律性质的利率(trade-regulated interest rate)
          \end{enumerate}
    \item 利率地位
          \begin{enumerate}[(1)]
              \item 基准利率:多种利率并存的条件下起决定作用的利率(benchmark interest rate)
                    \par 西方基准利率通常是中央银行的再贴现利率以及商业银行和金融机构之间的同业拆借利率;我国基准利率是央行对商业银行及其它金融机构的存、贷款利率,又称法定利率;货币市场的基准利率:上海银行间同业拆放利率(Shanghai interbank offered rate,SHIBOR)
              \item 一般利率:金融机构在金融市场上形成的各种利率(general interest rate)
          \end{enumerate}
    \item 信用期限
          \begin{enumerate}[(1)]
              \item 短期利率:一年期以内的信用活动适用的利率(short-term interest rate)
              \item 长期利率:一年期以上的信用活动适用的利率(long-term interest rate)
          \end{enumerate}
    \item 业务管理
          \begin{enumerate}[(1)]
              \item 普通利率
              \item 优惠利率
              \item 惩罚利率
          \end{enumerate}
    \item 市场交易
          \begin{enumerate}[(1)]
              \item 即期利率:对不同期限的金融工具以复利形式标示的利率(spot interest rate)
              \item 远期利率:给定即期汇率中隐含的未来一定期限的利率(forward interest rate)
                    \par  远期的借贷款合约可能按远期利率达成,此意义上远期利率为市场利率
          \end{enumerate}
\end{enumerate}


\subsection{政策利率}

\begin{enumerate}[1.]
    \item 币值变化
          \begin{enumerate}[(1)]
              \item 名义利率:物价水平不变即货币的实际购买力不变时的利率(real interest rate)
                    \par 名义利率是持有货币的机会成本 → 预期货币需求取决于名义利率 → 价格水平取决于现期货币量和预期的未来货币量
              \item 实际利率:包括物价变动因素的利率(nominal interest rate)
          \end{enumerate}
    \item 浮动范围
          \begin{enumerate}[(1)]
              \item 固定利率:借贷期内利息按照借贷双方事先约定的利率计算,不随市场资金供求状况而调整(fixed interest rate)
                    \par 借贷期长、市场利率波动较大的情况下不宜采用固定利率
              \item 浮动利率:借贷期内根据市场利率的变化定期调整利率(floating interest rate)
                    \par 浮动利率多用于期限较长的借贷和国际金融市场上的借贷
          \end{enumerate}
\end{enumerate}










\section{决定理论}





\subsection{马克思}

\begin{enumerate}[1.]
    \item 理论
          \begin{enumerate}[(1)]
              \item 利息量的多少取决于利润总额
              \item 利息率取决于平均利润率,介于零和平均利润率之间,取决于利润率和总利润在贷款人和借款人之间的分配比例
              \item 法律、习惯等的影响
          \end{enumerate}
    \item 利率特点
          \begin{enumerate}[(1)]
              \item 长期内平均利润率处于下降趋势,利率相同
              \item 利润率下降缓慢,利率比较稳定
              \item 利率决定有一定偶然性
          \end{enumerate}
\end{enumerate}



\subsection{可贷资金利率理论}
新剑桥学派:认为利率是借贷资金的价格,批判综合实际利率理论和流动性偏好利率(loanable-funds theory of interest)

\begin{gather*}
    F_s=F_d\\
    S+\Delta M_s=I+\Delta M_d
\end{gather*}
\par $F_s$:可贷资金的供给
\par $S$:某一时期的储蓄流量
\par $ΔM_s$:货币供给的增量
\par $I$:同期投资流量
\par $\Delta M_d$:人们希望保有的货币余额

\subsubsection{实际利率理论}

古典学派:储蓄、投资决定利率(非货币因素,实际因素)

\par 投资流量(企业厂房、设备、存货等)导致的资金需求是利率的减函数
\par 储蓄流量(家庭为主)导致的资金供给是利率的增函数

\subsubsection{流动性偏好理论}

凯恩斯:利率与流动性偏好负相关(货币因素,短期)

M垂直,外生变量
\par 利率取决于货币供求数量的对比
\par 货币当局决定货币供给
\par 人们的流动性偏好决定货币需求


\subsection{IS-LM模型}

\subsection{影响因素}

\subsubsection{宏观}

\begin{enumerate}[1.]
    \item 宏观经济周期(内生)
          \begin{enumerate}[(1)]
              \item 危机阶段:资金供不应求,利率走高
              \item 萧条阶段:资金需求降低,利率走低,甚至零利率
              \item 复苏阶段:资金需求增加,利率走高
              \item 繁荣阶段:利率持续提高
          \end{enumerate}
    \item 制度(外生):利率管制程度
          \begin{enumerate}[(1)]
              \item 利率管制:直接制定利率或规定上下限
              \item 利率市场化:通过市场和价值规律机制,在某一时点上由供求关系决定的利率运行机制(interest rate liberalization)
          \end{enumerate}
\end{enumerate}

\subsubsection{微观}


\par * 时间+风险
\par * 解释经验事实:
\par A 不同期限债券的利率随时间变化一起波动
\par B 短期利率低,收益率曲线更倾向于向上倾斜;如果短期利率高,收益率曲线可能向下倾斜
\par C 收益率曲线通常是向上倾斜的

\paragraph{预期/期望假说}

(expectation hypothesis)

\begin{enumerate}[1.]
    \item 假设:
          \begin{enumerate}[(1)]
              \item 不同债券完全可替代,投资者不偏好某种债券
              \item 所有债券都会有相同的收益率
          \end{enumerate}
    \item 理论:
          \begin{enumerate}[(1)]
              \item A 投资者在不同期限债券之间套利,使得不同期限债券价格相互影响、同升同降
                    \par $\left(1+r_n\right)^n=\left(1+f_1\right)\left(1+f_2\right)\left(1+f_3\right)..\ldots\left(1+f_n\right)$
              \item B 短期利率$r_1$,长期债券的到期收益率很大程度上取决于投资者对未来各年度远期利率$f_i$的预期。若$r_1$高位,则$f_i$下降的可能性更大
                    \par 远期利率:
                    \begin{gather*}
                        f_n = \frac{1}{\frac{P_{n-1,t}}{P_{n,t}}}-1=\frac{\left(1+r_n\right)^n}{\left(1+r_n\right)^{n-1}(n-1)}\\
                        R_F = \frac{R_2T_2-R_1T_1}{T_2-T_1}
                    \end{gather*}
                    \par $R_1$:期限为$T_1$的零息利率
                    \par $R_F$:$T_1$、$T_2$之间的远期利率
              \item C 无法解释
          \end{enumerate}
\end{enumerate}


\paragraph{市场分割理论/期限偏好理论}

(market segmentation theory)


\begin{enumerate}[1.]
    \item 假设:
          \begin{enumerate}[(1)]
              \item 不同期限的债券不是替代品,不同投资者会对不同期限的债券具有特殊偏好(一般偏好期限较短、利率风险小的证券
              \item 长期债券产生正的流动性溢价,若市场偏好长期债券,则流动性溢价为负)若短期利率超过长期利率,可能预示着经济衰退
          \end{enumerate}
    \item 理论:$f_n=E\left(r_n\right)+\text{流动性溢价}$
          \begin{enumerate}[(1)]
              \item A 无法解释
              \item B 无法解释
              \item C 需要高利率补偿长期债券的购买
          \end{enumerate}
\end{enumerate}


\paragraph{期限选择与流动性升水理论}

综合预期假说和市场分割理论(liquidity premium theory)



\begin{enumerate}[1.]
    \item 假设:
          \begin{enumerate}[(1)]
              \item 不同期限的债券是替代品,但并不完全可替代
              \item 要让投资者持有风险较大的长期债券,必须向其支付流动性升水以补偿其增加的风险
          \end{enumerate}
    \item 理论:
          \begin{enumerate}[(1)]
              \item AB同预期假说
              \item C 当考虑期限选择和流动性升水时,因为需要提供收益补偿,收益率曲线向下的概率会大大降低,向上倾斜的概率会大大增加
          \end{enumerate}
\end{enumerate}










\section{作用与影响}

\subsection{宏观}

\begin{enumerate}[(1)]
    \item 储蓄投资的规模和结构
    \item 借贷资金供求
    \item 资产价格:加息利空房地产价格和证券行市
    \item 社会总供求的调节:短期易于调节总需求,短期低利率刺激总需求(消费投资),增加总供给压力,长期倾向于增加总供给缓解供求压力,影响消费行为
    \item 资源配置效率:高利率会淘汰低弱企业,优质企业资金可得性增加,经济增长率会下降,但会导致资源消耗速度下降,资源配置效率提高
    \item 金融市场:利率是金融工具定价的基本要素
\end{enumerate}


\subsection{微观}

\begin{enumerate}[(1)]
    \item 促进企业加强核算,提高经济效益
    \item 调节个人的行为决策
\end{enumerate}



\subsection{条件}


\begin{enumerate}[(1)]
    \item 基础性条件
          \begin{enumerate}[a.]
              \item 独立的市场决策主体
                    \begin{enumerate}[(a)]
                        \item 市场主体独立决策并独立承担责任,权责利的有机结合
                        \item 微观经济主体是理性经济人
                    \end{enumerate}
              \item 市场化的利率决定机制
                    \begin{enumerate}[(a)]
                        \item 利率高低能够真实反映资金稀缺程度及其机会成本影响
                        \item 影响微观主体的投资和融资决策
                        \item 通过利率信号筛选优质项目,引导和配置资金口厂商的投资决策
                    \end{enumerate}
              \item 合理的利率弹性
          \end{enumerate}
    \item 经济制度与经济环境
          \begin{enumerate}[a.]
              \item 市场化改革:塑造独立决策、独立承担责任的市场行为主体
              \item 市场投资机会与资金的可得性:影响利率弹性和微观主体对利率的敏感性
              \item 产权制度:影响微观主体的激励和约束
          \end{enumerate}
\end{enumerate}













\chapter{汇率}




\chapter{金融工具}




\chapter{金融市场}




\chapter{金融机构}




\chapter{非金融机构}




\chapter{家庭金融}




\chapter{金融调控}




\chapter{金融监管}









\chapter{古典AD-AS模型}\label{chapter:古典AD-AS模型}


\chapter{剩余价值理论}




剩余价值规律:剩余价值产生及其增殖的规律。资本主义的生产目的和动机是追求尽可能多的剩余价值,达到这一目的的手段是不断扩大和加强对雇佣劳动的剥削

基本矛盾:生产社会化与生产资料私有在资源配置效率上的不协调,是市场经济的局限性所在




\section{资本主义生产}

\informationBox{
    \begin{enumerate}[(1)]
        \item 实质:剩余价值的生产。不是获取使用价值。具有二重性,资本主义生产过程是劳动过程和价值增殖过程的统一
        \item 目的:剩余价值的增殖,或利润的最大化
        \item 条件:资本积累为资本主义生产奠定物质基础;劳动力成为商品(雇佣工人)
    \end{enumerate}
}





\subsection{资本总公式:$G-W-G'$}

\par 资本周转:资本持续不断的、周而复始的循环运动
\begin{enumerate}[(1)]
    \item 产业资本公式:$G-W(Pm,A)\cdots P\cdots W′-G′$
    \item 商业资本公式:$G-(W-)G'$
    \item 借贷资本公式:$G-G'$
\end{enumerate}


\subsubsection{产业资本公式}



\paragraph{模型:社会简单再生产}


1、假设:
\informationBox{
    * 社会剩余产品是用于消费而不是用于积累的,生产在维持原来的规模上重复进行
    \begin{enumerate}[(1)]
        \item 生产周期为一年
        \item 全部生产资料价值和消费资料在一个生产周期内一次性消耗掉
        \item 没有外贸和储备
    \end{enumerate}
}


2、社会总产出的实现:
\begin{gather*}
    \begin{cases}
        I(c+v+m)=I c+II c \\
        II (c+v+m)=I(v+m)+II(v+m)
    \end{cases}\\
    I(v+m)=IIc
\end{gather*}

\informationBox{
    \par * 社会生产部门:两大部类
    \par $I$:第一部类,生产生产资料的部门
    \par $II$:第二部类,生产消费资料的部门
    \par * 价值形态:三个部分
    \par $c$:产品中生产资料的转移价值。由具体劳动实现
    \par $m$:工人在剩余劳动时间里创造的剩余价值,成本价格小于商品的价值。活劳动创造新价值的过程由抽象劳动实现
    \par $v$:工人必要劳动创造的价值
    \par $c+v$:商品的生产成本或成本价格
    \par $c+v+m$:社会总产品,社会在一定时期内(通常为一年)所生产的全部物质资料的总和
    \par * 第一部类产出中用$I c$表示的部分,可以通过第一部类内部的相互交换而实现。第二部类产出中用$II(v+m)$表示的部分,可以通过第二部类内部的相互交换而实现。第一部类产出中用$I(v+m)$表示的部分,可以通过与第二部类产出中用$IIc$表示的部分相交换而实现,资本从商品形态向货币形态的转化。第一部类的消费需求得到补偿,第二部类的生产耗费得到替换。社会再生产的实现过程的实质在于两大部类之间能否保持一个平衡发展关系
}



\paragraph{模型:社会扩大再生产}


1、假设:
\informationBox{
    * 社会生产在社会总资本循环运动中不断扩大规模
    \begin{enumerate}[(1)]
        \item 内涵扩大再生产:依靠生产技术进步、提高劳动效率以及改善生产要素质量来扩大生产规模
        \item 外延扩大再生产:在生产技术、劳动效率和生产要素(生产资料和劳动力)质量不变的情况下,依靠增加生产要素数量来扩大生产规模
    \end{enumerate}
    * 此处为外延扩大再生产
    \begin{enumerate}[(1)]
        \item 生产周期为一年
        \item 全部生产资料价值和消费资料在一个生产周期内一次性消耗掉
        \item 没有外贸和储备
        \item 追加资本的资本有机构成与原资本的有机构成相一致
    \end{enumerate}
}


2、社会总产出的实现:
\begin{gather*}
    \begin{cases}
        I(c+v+m)>I(c)+II(c) \\
        II [c+(m-m/x)]>I(v+m/x)
    \end{cases}\\
    \begin{cases}
        I(c+v+m)=I(c+\Delta c)+II(c+\Delta c) \Leftrightarrow  I(v+m)=II(c+\Delta c)+I\Delta c \\
        II(c+v+m)=I(v+\Delta v+m/x)+II(v+\Delta v+m/x) \Leftrightarrow II[c+(m-m/x)]=I(v+\Delta v+m/x)+II\Delta v
    \end{cases}\\
    I(v+\Delta v+m/x)=II(c+\Delta c)
\end{gather*}


\informationBox{
    \begin{enumerate}
        \item 使用价值替换:社会总资本运动正常运行的关键。买不到生产资料和工人需要的消费品,社会再生产就不能正常进行
        \item 价值补偿:社会总资本运动正常进行的基础。产品卖不出去,得不到价值补偿,资金不能回笼,社会再生产就不能正常进行
    \end{enumerate}
}





























\paragraph{理论}


1、资本周转:资本持续不断的、周而复始的循环运动
\informationBox{
    产业资本循环:产业资本从货币资本的职能形式出发,顺次经过购买、生产、销售三个阶段,分别地采取货币资本、生产资本、商品资本三种职能形式,实现了价值的增值,并回到原来出发点的全过程
    \begin{enumerate}[1.]
        \item 过程:一个生产阶段,两个流通阶段(买和卖)。是生产过程和流通过程的统一,在空间上同时并存,在时间上相互继起,是三个职能形式资本循环的统一
              \begin{enumerate}[(1)]
                  \item 购买阶段:货币资本
                  \item 生产阶段:生产资本
                  \item 销售阶段:商品资本
              \end{enumerate}
        \item 影响因素:$\text{预付资本的总周转速度}=\frac{\text{固定资本年周转价值总额}+\text{流动资本年周转价值总额}}{\text{预付资本总量}}$
              \begin{enumerate}[(1)]
                  \item 资本周转时间
                        \begin{enumerate}[a.]
                            \item 生产时间:劳动时间,劳动过程中的正常中断时间,生产要素的储备时间
                            \item 流通时间:
                                  \begin{enumerate}[(a)]
                                      \item 购买时间:受生产要素供应条件制约
                                      \item 售卖时间:受市场需求和竞争状况制约
                                  \end{enumerate}
                        \end{enumerate}
                  \item 资本周转速度
                        \begin{enumerate}[a.]
                            \item 固定资本的周转速度
                            \item 流动资本的周转速度
                                  \informationBox{
                                      \par * 固定资本:以厂房、机器设备、生产工具等劳动资料形式存在的那部分生产资本(劳动资料)
                                      \par (1) 折旧
                                      \par a. 定义:对固定资本价值转移量的计算以及从商品销售中逐步提取和回收这部分价值的方式
                                      \par b. 原因:损耗
                                      \par (a) 有形损耗(物质磨损):使用;自然力
                                      \par (b) 无形损耗(精神磨损):生产技术进步,生产效率提高,生产同类机器设备的社会必要劳动时间减少;科技创新,更好的机器设备
                                      \par * 流动资本:以原料、燃料及动力、辅助材料等劳动对象形式存在的以及用于购买劳动力的那部分生产资本(劳动对象,劳动者)
                                      \par * 固定资产:多个生产过程中价值分批转移,分批回收;较长的有效使用期内不必更新;回收期、周转期长
                                      \par * 流动资产:一个生产过程中价值一次全部转移,通过产品出售一次全部回收;每一生产周期前需更新;回收期、周转期短
                                  }
                        \end{enumerate}
                  \item 生产资本的构成:$\text{资本的有机构成}=\frac{v}{c}$
                        \begin{enumerate}[a.]
                            \item 可变资本:用于购买劳动力的资本
                            \item 不可变资本:以生产资料形式存在的资本(在生产过程中不改变自己的价值量,是流动资本的一部分)
                        \end{enumerate}
              \end{enumerate}
    \end{enumerate}
}

2、资本总公式矛盾:从形式上看,货币资本在流通中增殖与等价交换原则矛盾


\informationBox{
    \par 价值增殖或剩余价值是在生产过程中创造出来的。在流通领域,无论是等价交换还是不等价交换,都不能产生价值增殖,但是价值增殖的实现必须以流通为前提和手段。流通中的购买阶段为剩余价值的生产做准备,流通中的销售阶段实现剩余价值。因此,剩余价值不产生于流通过程,又离不开流通过程,必须以流通过程为媒介。
}

3、解决关键:劳动力成为商品
\informationBox{
    \par 价值增殖只能发生在G—W阶段所购买的商品上,这一商品必须具备特殊的使用价值,而这一特殊商品只能是劳动力
    \par * 条件
    \begin{enumerate}[(1)]
        \item 可能性:劳动者有人身自由
        \item 必要性:劳动者丧失一切生产资料和生活资料,只拥有自己的劳动力
    \end{enumerate}
    \par * 价值:劳动力商品的价值由生产和再生产劳动力所需要的社会必要劳动时间决定
    \begin{enumerate}[(1)]
        \item 构成:劳动者自身生存;繁育后代;劳动力接受教育和训练
        \item 特点:包含历史和道德的因素
    \end{enumerate}
    \par * 使用价值:劳动力商品使用价值的特殊性,在于它是价值和大于自身价值的源泉。正是资本家对这种特殊商品的购买,从而对劳动力使用价值的运用,生产出归属于资本家的产品,才使得资本家的货币转化为资本,从而实现价值增殖
}




\subsection{资本积累}


资本积累:剩余价值转化为资本,即剩余价值的资本化
\informationBox{
    \par 资本积累是资本扩大再生产的重要源泉
    \par 剩余价值(利润)是资本积累的唯一源泉
}



\subsubsection{资本积聚}

定义:单个资本依靠自身的积累来使实际资本在价值形态和生产要素形态上实现量的扩大

\begin{enumerate}[1.]
    \item 产品经营
          \begin{enumerate}[(1)]
              \item 定义:企业围绕企业的产品与服务等主要业务,进行生产(含服务)管理、产品改进、质量提高、市场开发等一系列活动
          \end{enumerate}
    \item 资本经营
          \begin{enumerate}[(1)]
              \item 定义:以价值形态的资本为经营对象,通过调整、交易、优化重组等方式,以实现资本价值量的保值增值的一系列市场行为
              \item 方式
                    \begin{enumerate}[a.]
                        \item 利用证券市场或其他形式的产权交易市场进行资产的收购、出售、托管、租赁等
                        \item 利用金融市场进行投机性交易(赚取差价,提高公司市值)
                        \item 对资产存量或所积累的资产增量进行调整(调整规模,方向,结构)
                        \item 无形资产经营
                        \item 风险投资
                    \end{enumerate}
              \item 特点
                    \begin{enumerate}[a.]
                        \item 高智力性:经营的具体条件复杂,需运用不同思路研究、策划和设计
                        \item 非生产性:有别于产品经营,本身不创造财富和产品
                        \item 高收益性:高智力劳动,高风险性
                    \end{enumerate}
          \end{enumerate}
    \item 关系
          \begin{enumerate}[(1)]
              \item 联系
                    \begin{enumerate}[a.]
                        \item 目的:价值增殖
                        \item 产品经营是资本经营的基础,资本经营是产品经营发展到一定阶段的必然趋势
                    \end{enumerate}
              \item 区别
                    \begin{enumerate}[a.]
                        \item 内容对象:产品经营对象为产品及其生产销售过程;资本经营对象为企业资本
                        \item 经营方式:
                        \item 经营市场:产品经营依托商品或服务市场;资本经营依托资本市场
                        \item 收益风险:产品经营波幅和缓,持续时间长;资本经营可能短期大幅升跌
                    \end{enumerate}
          \end{enumerate}
\end{enumerate}






\subsubsection{资本集中}

\begin{enumerate}[1.]
    \item 定义:把若干个规模相对较小的资本合并重组为规模较大的资本
    \item 途径:
          \begin{enumerate}[(1)]
              \item 并购(兼并、收购):部分并购、整体并购其影响不仅是数量,也有质量,是行业标准方面的
              \item 联合原有的、分散的单个资本联合成新的更大的资本
              \item 上市向社会发行股票等方式,把社会闲散资金集中起来转化为资本
          \end{enumerate}
    \item 竞争和信用是资本集中的强有力的杠杆
\end{enumerate}




\subsubsection{关系}

\begin{enumerate}[(1)]
    \item 联系
          \begin{enumerate}[a.]
              \item 都能使单个资本的规模增大
              \item 二者相互促进
          \end{enumerate}
    \item 区别
          \begin{enumerate}[a.]
              \item 自身前提:资本积累以剩余价值的积累为前提;资本集中不以积累为必要前提
              \item 社会前提:资本积聚的实现受社会所能提供的实际生产要素增长的制约;资本集中较少受限
              \item 自身影响:资本积聚扩大单个资本规模,一般速度较慢;资本集中可以快速扩大资本规模
              \item 社会影响:资本积聚在增大单个资本的同时,增大社会总资本;资本集中一般不能直接增大社会总资本,可以改变资本的结构和质量
          \end{enumerate}
\end{enumerate}








\section{资本主义交换/流通}



\subsection{商品经济}



定义:是以交换为目的、包括商品生产和商品交换的经济形式


\informationBox{
    \par * 与自然经济相对应(自然经济是自给自足),是生产力发展到一定阶段的产物
    \par * 特征
    \begin{enumerate}[(1)]
        \item 自主性:生产经营独立自主,经济利益独立
        \item 平等性:商品交换以等价交换原则为基础
        \item 竞争性
        \item 开放性:商品经济以社会分工为基础;生产者间的经济联系紧密,范围扩大
    \end{enumerate}
    \par * 前提
    \begin{enumerate}[(1)]
        \item 社会分工(必要性):社会分工→生产专业化→交换需求
        \item 剩余产品的出现,并分属于不同的生产者所有(可能性)
    \end{enumerate}
    \par * 阶段
    \begin{enumerate}[(1)]
        \item 简单商品经济(小商品经济)
              \begin{enumerate}[a.]
                  \item 商品经济的初始形式
                  \item 以个体私有制和个体劳动为基础
                  \item 以手工业劳动为技术特征
              \end{enumerate}
        \item 市场经济(社会化商品经济):市场在资源配置中起决定作用的经济
              \begin{enumerate}[a.]
                  \item 价值规律成为支配社会经济发展的基本规律
                  \item 市场在资源配置中起基础性作用
                  \item 是商品经济发展到一定阶段的产物
              \end{enumerate}
    \end{enumerate}
}








\subsection{规律}

\subsubsection{价值规律}


\begin{enumerate}
    \item 内容:商品的价值量由生产商品的社会必要劳动时间决定,以此为基础进行商品等价交换
          \begin{enumerate}[(1)]
              \item 价格是商品价值的货币表现
          \end{enumerate}
    \item 形式
          \begin{enumerate}[(1)]
              \item 价格受供求关系影响,围绕价值上下波动
              \item 价格背离价值的运动总是围绕价值上下波动的
          \end{enumerate}
    \item 作用
          \begin{enumerate}[(1)]
              \item 调节:自发调节生产资料和劳动力在社会各部门之间的分配(生产前)
              \item 刺激:刺激生产者的积极性(生产中)
              \item 分化:优胜劣汰,导致生产者两极分化(生产后)
          \end{enumerate}
\end{enumerate}








\subsubsection{资源配置规律}



资源配置:在经济运行过程中,各种现实的资源在社会不同部门之间的分配和不同方向上的使用
\\

\begin{enumerate}[1.]
    \item 市场配置
          \begin{enumerate}[(1)]
              \item 定义:通过市场机制发挥作用,促使资源的分配和流动
                    \begin{enumerate}[a.]
                        \item 市场机制:价格(基础),供求,竞争,风险
                        \item 看不见的手
                    \end{enumerate}
              \item 优点
                    \begin{enumerate}[a.]
                        \item  只要依靠市场的自发调节,通过市场主体依照一定规则进行的市场交易活动,就能自动实现资源的优化配置(价格机制、竞争机制、供求机制、风险机制)
                    \end{enumerate}
              \item 缺点
                    \begin{enumerate}[a.]
                        \item 市场机制具有自发性,市场主体的分散决策,难以自动地实现整个国民经济的发展战略和目标
                        \item 市场配置不能直接对需求总量和结构进行调控,会造成宏观经济总量和结构的失衡
                        \item 市场配置对于外部不经济的调控显得乏力(环境污染、生态破坏、公共物品受损等)
                        \item 市场机制会刺激生产经营者的短期行为,导致产业结构的失衡和资源的浪费(盲目性)
                        \item 市场作用机制异化,如垄断反过来抑制市场机制的正常作用,失业、商品积压等也是市场的产物
                    \end{enumerate}
          \end{enumerate}
    \item 计划配置
          \begin{enumerate}[(1)]
              \item 定义:通过指令性计划和政府的各种预算、投资等直接进行分配资源的分配和组织流动
                    \begin{enumerate}[a.]
                        \item 看得见的手
                    \end{enumerate}
              \item 优点
                    \begin{enumerate}[a.]
                        \item 国家对经济进行统一调度,有利于经济的稳定和收入差距的减小,更大的体现平均
                        \item 国家的高度干预,使得资本的流转以及分配的权利都集中到了国家的手中,国家掌握了经济的主动权。绝大多数的企业和工厂都按照国家的指示来进行各种生产活动,就使得经济关系变得更加的简单
                        \item 国家掌握了社会财富,有利于集中财力物力人力干大事业。有利于国家在一定的时间和程度上实现资源的高利用,建成一些利国利民的基础设施和大型的生产设备和基地
                        \item 在短期内激发民众的热情,迅速使国力提升
                        \item 制定战略规划,前瞩性地引导国民经济发展
                    \end{enumerate}
              \item 缺点
                    \begin{enumerate}[a.]
                        \item 通过指令性计划配置资源,一刀切,企业成为了政府的附属物,没有自主权,缺乏内在的动力、主动性和创造性
                        \item 激励机制不足,对劳动者缺乏奖励,工作动力和热情难以持续
                        \item 中央计划当局难以把握全面准确信息,难以制定符合客观实际的计划
                        \item 计划经济生产部门之间是计划调拨,一旦一个生产环节脱钩,就会造成链式反应,整个经济陷于停顿
                        \item 国家基本上取消了市场,直接导致了竞争的消失
                    \end{enumerate}

          \end{enumerate}
\end{enumerate}






\section{资本主义分配(转型理论)}


本质:剩余价值m(p)在资本家之间的分配。
\\





\begin{enumerate}[1.]
    \item 剩余价值→利润:$W=c+v+m\Rightarrow W=K(c+v)+m$
          \begin{enumerate}[(1)]
              \item 当不把剩余价值看作是雇佣工人剩余劳动的产物,而是把它看作是全部预付资本的产物或增加额时,剩余价值便转化为利润
              \item $K = c + v$:商品的生产成本或成本价格
              \item 成本价格掩盖了不变资本和可变资本之间的区别,掩盖了它们在价值增殖过程中的不同作用
          \end{enumerate}
    \item 利润→平均利润:
          \begin{enumerate}[(1)]
              \item 假设
                    \begin{enumerate}[a.]
                        \item 市场充分竞争,未形成垄断
                        \item 价格是灵敏的,进入和退出的机制是自由的
                    \end{enumerate}
              \item 模型:
                    \begin{gather*}
                        \text{平均利润率}=\frac{\text{剩余价值总额}}{\text{社会总资本}}=\frac{\text{社会剩余价值总额}}{\text{产业资本总额}+\text{商业资本总额}}\\
                        \text{平均利润}=\frac{\text{预付资本}}{\text{平均利润率}}
                    \end{gather*}
                    \begin{enumerate}[a.]
                        \item 平均利润:投入各个不同部门的等量资本获得的等量利润(不是绝对平均,而是一种趋势)
                        \item 社会总资本:以社会分工和市场交换为条件,相互联系、相互依存、相互制约的全社会各单个资本的总和(社会资本)
                    \end{enumerate}
              \item 理论:
                    \begin{enumerate}[a.]
                        \item 部门内:提高部门资本有机构成,利润率下降
                        \item 部门间:平均利润是不同部门之间竞争的结果,竞争的手段是进行资本转移。社会总资本在各部门之间的分配。投入高利润部门资本比重越大,平均利润就高,反之则越低
                        \item 部门间竞争结果:不同部门的利润率出现平均化趋势
                    \end{enumerate}
          \end{enumerate}
    \item 价值→生产价格:$\text{生产价格}=K(\text{成本价格})+p(\text{平均利润})$
          \begin{enumerate}[(1)]
              \item 生产价格的形成是以平均利润率的形成为前提的。利润转化为平均利润,商品价值便转化为生产价格。此时,商品不再按照商品的价值出售,而是按照生产成本加平均利润的价格来出售了,即按照生产价格出售。
              \item 价值规律作用的形式也发生相应的变化。商品按生产价格销售,市场价格以生产价格为中心,受供求关系的影响而波动,所以,价值规律现不再是直接通过价值,而是通过生产价格起作用
              \item  对于个别,平均利润和剩余价值不相等,生产价格和价值不相等。对于全社会,总额相等
              \item 社会生产价格变动,最终取决于价值的变动
          \end{enumerate}
\end{enumerate}






\subsection{劳动力}









\subsubsection{工资}
定义:劳动力价值或价格的转化形式。本质是在资本主义社会里,资本家付给雇用工人的工资是劳动力的价值或价格,而不是劳动的价值或价格

\informationBox{
    * 劳动和劳动力的区别
    \begin{enumerate}[(1)]
        \item 劳动力是潜藏在人身体内的劳动能力。劳动是劳动力的使用,劳动力在生产中发挥作用时才是劳动
        \item 劳动不是商品,但劳动力在一定历史条件下则可以成为商品,它具有价值和使用价值
        \item 在资本家同工人的买卖关系中,工人出卖的是劳动力,而不是劳动,劳动根本不能成为商品
        \item 不成为商品的东西也就没有价值。所以工资只能是劳动力的价值或价格,而不是劳动的价格或价值。
    \end{enumerate}
    * 劳动不能成为商品,所以没有价值
    \begin{enumerate}[(1)]
        \item 如果劳动是商品,具有价值,价值又是凝结在商品中的无差别人类劳动,劳动的价值由劳动决定,同义反复
        \item 如果劳动是商品,出卖前就应该独立存在
        \item 如果劳动是商品,不等价交换违反价值规律,等价交换就无法获得剩余价值,否定资本主义生产关系存在的基础
        \item 如果劳动是商品,就等于说雇佣工人出卖了不属于自己的商品,因为劳动此时已归资本家所有
    \end{enumerate}
    * 资本主义工资在现象上表现为劳动的价值或价格是由资本主义生产关系决定的
    \begin{enumerate}[(1)]
        \item 从资本和劳动的交换关系来看,劳动力的买卖和其他商品的买卖一样,钱货两清
        \item 从工资的支付形式来看,资本家通常在劳动后支付工资,使得工资被看做劳动的价值或价格
    \end{enumerate}

}



基本形式
\begin{enumerate}[(1)]
    \item 计时工资:按一定的劳动时间来支付的工资,其实质是劳动力的月价值、周价值、日价值的转化形式
    \item 计件工资:按工人完成的产品数量或完成的工作量来支付的工资,它是计时工资的转化形式
\end{enumerate}





{\noindent
\begin{tabu}{|c|X|X|}
    \hline
                               & \multicolumn{1}{c|}{名义工资}                      & \multicolumn{1}{c|}{实际工资} \\\hline
    \multicolumn{1}{|c|}{定义} & \begin{tabu}{X}
        指工人出卖劳动力所得到的货币数量,即货币工资
    \end{tabu}
                               & \begin{tabu}{X}
        工人用货币工资实际买到的各类生活资料和服务的数量
    \end{tabu}                                                        \\\hline
    \multicolumn{1}{|c|}{对立} & \multicolumn{2}{l|}{\tablelist{(1)}{\columnwidth}{
            \item 在其他条件不变的情况下,名义工资越高,实际工资也就越高
    }}                                                                                                              \\\hline
    \multicolumn{1}{|c|}{统一} & \multicolumn{2}{l|}{\tablelist{(1)}{\columnwidth}{
            \item 两者常常不一致,即名义工资虽然不变甚至提高,实际工资却可能降低(原因:实际工资的多少不仅取决于名义工资的高低,而且还取决于物价的高低)
            \item 趋势:名义工资一般呈增加趋势。但是,实际工资的提高并不意味着工人受剥削程度的减轻(原因:相对工资(m+v)呈下降趋势);从资本主义发展的历史过程来看,实际工资有时降低有时提高
    } }                                                                                                             \\\hline
\end{tabu}
}











\subsection{资本}



\begin{equation*}
    \text{资本分配}\mindmap{
        \text{职能资本}\mindmap{
            \text{产业资本}\mindmap{
                \text{货币资本}\\
                \text{生产资本}\\
                \text{商品资本}
            }
            \text{商业资本}
        }
        \text{非职能资本}\mindmap{
            \text{生息资本}\mindmap{
                \text{高利贷资本}\\
                \text{借贷资本}
            }
        }
    }
\end{equation*}


高利贷资本是生息资本的古老形式,借贷资本是生息资本的资本主义形式。
\\



\begin{enumerate}[1.]
    \item 产业资本
          \begin{enumerate}[(1)]
              \item 功能:生产剩余价值
          \end{enumerate}
    \item 商业资本
          \begin{enumerate}[(1)]
              \item 定义:从产业资本中分离出来的专门从事商品买卖的,以获取商业利润为目的的资本
              \item 功能:实现剩余价值
              \item 特点:
                    \begin{enumerate}[a.]
                        \item 不直接增加剩余价值
                        \item 只在流通领域
                        \item 受生产和消费的限制
                    \end{enumerate}
              \item 成本:费用
                    \begin{enumerate}[a.]
                        \item 生产性流通费用:商品的分类、包装、保管和运输支出的费用。创造价值和剩余价值
                        \item 纯粹流通费用:用于商业簿记、邮资、通信、广告及商业职工的工资等的费。不生产价值和剩余价值
                    \end{enumerate}
              \item 利润
                    \begin{enumerate}[a.]
                        \item 本质:产业工人创造的、由产业资本家让渡给商业资本家的一部分剩余价值。从事商品买卖所获得的利润是产业资本家按照低于生产价格的价格把商品让渡给商业资本家,然后商业资本家再按照生产价格把商品卖给消费者。这种售价大于买价之差,就是商业资本家所获得的商业利润
                    \end{enumerate}
          \end{enumerate}
    \item 借贷资本
          \begin{enumerate}[(1)]
              \item 定义:从职能资本循环中独立出来的一种特殊资本形式,是为了取得利息而暂时借给另一个资本家使用的货币资本
              \item 借贷利息
                    \begin{enumerate}[a.]
                        \item 0<利息率<平均利润率
                        \item 当借贷资本的供求平衡时,利息率只能由社会习惯和法律等因素决定
                        \item 本质:产业工人创造的、由职能资本家让渡给借贷资本家的一部分剩余价值的特殊转化形式
                    \end{enumerate}
              \item 分类:银行资本
          \end{enumerate}
\end{enumerate}
























\subsection{土地}




\subsubsection{出租:地租}

\par 定义:土地所有者凭借土地所有权获得的一种非劳动收入
\par 本质:由农业工人创造的、被农业资本家让渡给土地所有者的超过平均利润以上的那部分剩余价值



\begin{enumerate}[1.]
    \item 绝对地租
          \begin{enumerate}[(1)]
              \item 定义:由于土地私权的存在,农业资本家租用任何土地都必须交纳的地租
              \item 实质:是农产品价值高于社会生产价格的差额,是由农业雇佣工人创造的剩余价值的一部分转化而来的,它体现的仍然是农业资本家和土地所有者对雇佣工人的剥削关系
              \item 形成原因:土地私有权的垄断(资本主义土地所有权的垄断)
              \item 形成条件:农业资本有机构成低于社会平均资本有机构成
                    \begin{enumerate}[a.]
                        \item 农业通常是劳动密集型的产业。资本有机构成比较低,因此,剩余价值比较高
                        \item 由于土地所有权的垄断,阻碍了资本向农业转移。高于平均利润的部分被留在农业部门,被土地所有者占有,形成绝对地租
                        \item 土地有限性:和工业品不同,农产品社会生产价格是由劣等地的生产条件决定的
                    \end{enumerate}
          \end{enumerate}
    \item 级差地租
          \begin{enumerate}[(1)]
              \item 定义:与土地等级(土地的优劣)相联系的地租形式
                    \begin{enumerate}[a.]
                        \item 土地自然生产力产生了资本生产率的等级差别
                        \item 是农产品的个别生产价格低于社会生产价格的差额
                        \item 级差地租I是级差地租II的基础
                    \end{enumerate}
              \item 级差地租I
                    \begin{enumerate}[a.]
                        \item  定义:由于土地肥沃程度和地理位置等不同而产生的级差地租
                    \end{enumerate}
              \item 级差地租II
                    \begin{enumerate}[a.]
                        \item 定义:由于在同一块土地上连续追加投资的资本生产率不同而产生的级差地租
                        \item 形成原因:土地的有限性所引起的土地经营的资本主义垄断(资本主义土地经营权的垄断)
                        \item 源泉:优等地和中等地的农业雇佣工人创造的超额利润(剩余价值)。土地数量有限,优等中等农产品供不应求。工业则只有优等。农产品市场价格是由劣等地的生产条件决定的
                    \end{enumerate}
          \end{enumerate}
\end{enumerate}











\subsubsection{出卖}

\begin{gather*}
    \text{价格}=\frac{\text{地租额}}{\text{利息率}}
\end{gather*}


土地价格是地租收入的资本化,它相当于这样一笔资本,把它存入银行,每年得到的利息与凭借土地得到的地租相等



\chapter{凯恩斯AD-AS模型}\label{chapter:凯恩斯AD-AS模型}





\chapter{资产负债表}

衡量的是存量。

\section{企业资产负债表}



\section{央行资产负债表}

\section{国家资产负债}












\chapter{资金流量表}

衡量的是流量。




\section{企业现金流量表}



\section{国际收支平衡表}


定义:一定时期内一个国家(地区)和其他国家(地区)进行的全部经济交易的系统记录(statement for balance of payments,BOP)


\begin{tabu}{|c|c|c|c|}
    \hline
    账户名称 & 项目           & 借方debit-          & 贷方credit+           \\\hline
    经常账户 & 实体、资源     & 进口(买资源)       & 出口                   \\\hline
    金融账户 & 金融、资金     & 资金流出(买金融资产) & 资金流入(卖金融资产) \\\hline
    资本账户 & 免外债、被免债 & payment outflow      & payment inflow         \\\hline
\end{tabu}



\informationBox{
    \par * 在宏观经济中,$GNP\approx \text{国民收入}\approx GDP$
    \par * 净国际投资头寸(IIP):一国对外资产与负债的差额(net international investment position)
    \par * 关系:
    \begin{enumerate}[(1)]
        \item 贸易差额=货物出口-货物进口
        \item 经常项目差额CA=贸易差额+初次收入贷方-初次收入借方+二次收入贷方-二次收入借方
        \item 基本差额=CA+长期资本流入-长期资本流出
        \item 官方结算余额:中央银行净金融流动水平,表示需要官方储备去弥补的差额(official settlements balance),又称国际收支余额(balance of payments)。度量了国际借贷的规模和方向
        \item OBS=基本差额+私人短期资本流入-私人短期资本流出=CA+KA+Nonreserve FA+私人短期资本流动差额=-R
        \item 综合账户余额=官方结算余额+官方借款-官方贷款=官方结算差额+官方借贷=经常账户+金融账户(除官方储备)+资本账户+净误差与遗漏
        \item 经常账户余额+资本账户余额=金融账户余额
    \end{enumerate}
}





\subsection{经常账户}

定义:对外净出口的商品和服务的数额(current account, CA),又称经常账户余额


\informationBox{
    \par * 反映居民与非居民之间实际资源的国际流动
    \par * 构成:
    \begin{enumerate}[1.]
        \item 货物:一般商品 + 来料加工(goods)
        \item 服务:(services)
        \item 初次收入(primary income, PI)
              \begin{enumerate}[(1).]
                  \item 职工报酬
                  \item 财产收入:提供融资产和出租然资源所得的回报
                  \item 投资收益:提供融资产所得的回报,包括直接投资项下的利润利息收和再投资收入、证券投资收入(红利、利息等)和其他投资收入(利息)
              \end{enumerate}
        \item 二次收入:单方转移(官方援助/捐赠、侨民汇款、对国际组织的认缴款、战争赔款等)(secondary income, SI)
    \end{enumerate}
}


\subsection{金融账户}

定义:(financial account, FA)

\informationBox{
    * 构成:
    \begin{enumerate}[1.]
        \item 非储蓄性质
              \begin{enumerate}[(1)]
                  \item 直接投资:股本投资、建厂(foreign direct investment, FDI)
                  \item 证券投资(portfolio investment)
                  \item 金融衍生品(derivatives)和雇员认股权
                  \item 其他投资(other investment)
              \end{enumerate}
        \item 储蓄性质
              \begin{enumerate}[(1)]
                  \item 储备资产:一国货币当局所直接控制的、实际存在的可随时用来干预外汇市场、支付国际收支差额的资产(reserve assets)
                        \begin{enumerate}[a.]
                            \item 货币黄金:货币当局持有的黄金(gold)
                            \item 外汇储备(负表示官方储备增加)(主要)(foreign reserve)
                            \item 特别提款权:IMF 创设的种于补充成员国官储备的国际储备资产,虚拟合成,价值由一揽子货币决定(special drawing rights, SDR)
                            \item 在IMF的储备头寸:分配的储备(position)
                        \end{enumerate}
              \end{enumerate}
    \end{enumerate}
}


\subsection{资本账户}

定义:包括居民与非居民之间的资本转移和非生产性、非金融资产的交易(capital account, CA)

\informationBox{
    * 结构:
    \begin{enumerate}[(1)]
        \item 固定资产所有权的变更
        \item 债务债权的减免
        \item 非生产性(non-market,non-produced)有形资产(土地和地下资产)、无形资产(专利、版权、商标等)所有权转移
    \end{enumerate}
}

\subsection{净误差与遗漏}

定义:(net errors and omissions)
\informationBox{
    \par * 为负且扩大:资本外逃(capital flight)
    \par * 为正且扩大:热钱流入
}






\chapter{收入表}

衡量的是流量。

\section{企业利润表}


\section{所有者权益变动表}


\section{国民收入核算}


国民收入核算(national income accounting)。


\subsection{国内生产总值(GDP)}

定义:衡量现期生产的产品与服务的价值(gross domestic product,GDP)

\informationBox{
    * 局限性
    \begin{enumerate}[(1)]
        \item 国际比较
        \item 资源环境成本
        \item 经济增长效率
        \item 滞后性(核算程序)
    \end{enumerate}
}



\subsubsection{核算方法}

生产法:国内生产总值=总产出 - 中间投入

收入法:国内生产总值=劳动者报酬+生产税净额+固定资产折旧+营业盈余

支出法:国内生产总值=最终消费+资本形成总额+净出口


\informationBox{
    \par 计量规则:国内+生产(产品与服务)+总值(可统计)
    \par * 计入:
    \begin{enumerate}[(1)]
        \item 所有企业的总增加值(最终产品和服务的总价值)
        \item 租房、住房(被视为服务)、政府服务
        \item 短期出境单位
    \end{enumerate}
    \par * 不计:
    \begin{enumerate}[(1)]
        \item 二手货出售(非生产)
        \item 中间产品价值(重复统计)
        \item 耐用品租金(非生产)
        \item 地下经济(无法统计)
    \end{enumerate}
    \par * 特殊:
    \begin{enumerate}[(1)]
        \item 存货处理(是否可售)不可售不能理解为自己卖给自己,总支出没变,只不过在工资与利润之间分配
        \item 日常用GDP已经刨除了物价因素,是实际GDP
        \item 大多GDP数据都经过了季节性调整,因此GDP的变化必须寻找季节性周期以外的解释
    \end{enumerate}

}

\subsubsection{组成部分}

\begin{align*}
    Y & = C^D+I^D+G^D+EX         \\
      & = C+I+G+EX-(C^F+I^F+G^F) \\
      & = C+I+G+(EX-IM)          \\
      & = C+I+G+NX
\end{align*}
$C$:消费:国内居民私人消费的数额(consumption)
\informationBox{
    * 构成:
    \begin{enumerate}[(1)]
        \item 耐用品
        \item 非耐用品
        \item 服务
    \end{enumerate}
}



$I$:投资:私人企业为进行再生产而留下的用于购买厂房、设备的数额(investment)
\informationBox{
    * 构成:
    \begin{enumerate}[(1)]
        \item 企业固定投资(非住房固定投资)
        \item 住房固定投资
        \item 存货投资
    \end{enumerate}
    \par * 创造资本的新实物资产,资本可被用于未来生产
    \par * 不考虑部门内部的调剂
    \par * 买股票:发行市场算,流通市场不算
}


$G$:政府购买:政府购买商品和服务使用的数额(government purchases )
\informationBox{
    \par * 不包括转移支付
    \par $T$:税收-转移支付
    \par $G>T$:预算赤字(budget surplus)
    \par $G=T$:预算平衡(balanced budget)
    \par $G<T$:预算盈余(budget deficit)
}


$EX$:外国在国内产品与服务上的支出

$NX$:净出口(net exports)=贸易余额(trade balance)
\informationBox{
    \par $$NX=Y-C-G-I=S-I$$
    \par $S-I$:资本净流出(net capital outflow)=国外净投资(net foreign investment)
    \par $NX=S-I>0$:贸易盈余(trade surplus)
    \par $NX=S-I<0$:贸易赤字(trade deficit)
    \par $NX=S-I=0$:贸易平衡(balanced trade)
    \par * 双边贸易余额无关紧要,一国与所有外国的总体贸易余额才重要
}


\subsection{国民生产总值(GNP)}

定义:一个国家的生产要素在一定时期内所生产并在市场上卖出的最终商品和服务的价值总量(gross national product,GNP)

\begin{align*}
    GNP=GDP+\text{来自国外的要素报酬}-\text{支付给国外的要素报酬}
\end{align*}


\subsection{国民净产值(NNP)}

(net national product)

\begin{align*}
    NNP= GNP-\text{折旧}
\end{align*}


\subsection{国民收入(NI)}

定义:一定时期内由该国的生产要素获得的收入(national income)

\begin{align*}
    NI=NNP+\text{净单边转移支付}-\text{统计误差}
\end{align*}

\informationBox{
    * 构成
    \begin{enumerate}[(1)]
        \item 雇员报酬:工人赚到的工资和福利津贴(compensation of employees)
        \item 业主收入:非公司型企业的收入(proprietors' income)
        \item 租金收入:房东得到的收入(包括自有住房)-折旧等支出(rental income)
        \item 公司利润:公司在向工人和债权人支付报酬后的收入(corporate profits)
        \item 净利息:国内企业支付的利息-国内得到的利息+国外赚取的利息(net interest)
        \item 生产和进口税:企业的某些税收(如销售税)减去充抵的企业补贴(taxs on production and imports)
    \end{enumerate}
}


\subsection{个人可支配收入}

(disposable personal income)

\begin{gather*}
    \text{个人可支配收入}
    =\text{个人收入}-\text{个人税收个人收入(personal income)} \\
    =\text{NI}-\text{生产和进口税}-\text{公司利润}-\text{社会保险费}-\text{净利息}+\text{股息}+\text{政府对个人的转移支付}+\text{个人利息收入}
\end{gather*}











\chapter{社会观}


\chapter{价值观}






\chapter{琵琶}\label{chapter:琵琶}




















\part{习题}













\end{document}