

\section{飘}



\par 书名:飘(套装全2册)(译林名著精选)
\par 作者:[美]玛格丽特·米切尔
\par 译者:李美华
\par 出版社:译林出版社
\par 出版时间:2017-03
\par ISBN:9787544761628



\subsection{明天又是另外一天了(代译序)}

\par 仅仅写了一部作品就名扬天下并在文坛上占有一席之地的作家是绝无仅有的,而美国女作家玛格丽特·米切尔便是这样一位绝无仅有的作家。她唯一的作品《飘》一经问世便成了美国小说中最畅销的作品。自1936年出版之日起,《飘》这部美国内战时期的罗曼史便打破了所有的出版记录。1937年,小说获得普利策奖,三年后被改编成电影,连电影也成了美国电影史上的经典之作。
\par 作者玛格丽特出生在美国南方城市亚特兰大,是个典型的南方姑娘。出生于1900年的她并没有经历过美国南北战争,但是,由于亚特兰大在美国内战期间曾经被北方军攻陷,落入北方军将领舍曼之手,所以,这段历史成了亚特兰大市民十分热衷的话题。玛格丽特从小听到许多有关这段历史的谈论,这使她萌发了创作一部以美国南北战争为题材的小说的想法。一经决定,亚特兰大也就理所当然地被作者定为小说的创作背景。小说初稿早在1929年就已经完成,但玛格丽特并未马上付印,而是几经修改,终于使小说成了一本举足轻重的世界名著,至今魅力仍经久不衰。正如有的出版商所说,《飘》的读者群是一代接一代的。老一辈读者有之,中年一代亦不乏其人,年轻读者的数量更是大得惊人。
\par 《飘》是一部有关战争的小说,但作者玛格丽特没有把着眼点放在战场上。除了亚特兰大失陷前五角场上躺满伤病员那悲壮的一幕外,其他战争场景并没有花费作者过多的笔墨。作为第一部从南方女性角度来叙述美国内战的小说,玛格丽特着重描写了留在后方家里的妇女饱受战乱之苦的体验和感受,从战争伊始对战争怀有的崇敬心理、对战争全然的支持,到因战争而带来的失去亲人的痛苦、不得不屈服于失败的命运以及战后立志重建家园的艰辛历程。战争失败了,有的人因此而意志消沉,失去了原有的斗志,无法调整好自己的心态,面对战后支离破碎的生活。反之,另外一些人则克服了失败的心理,凛然面对严酷的现实,成了生活中不畏困难、重新前进在生活旅途上的强者。
\par 这其中就有女主人公郝思嘉。应该说,小说中最具吸引力的人物非她莫属。出身种植园主家庭的思嘉年轻漂亮,个性鲜明。然而,不幸的是,在她尚属青春年少的十六岁花季时,思嘉就遭遇了情场失意的痛苦。她爱上了风度翩翩的邻居卫希礼,可卫希礼却娶了善解人意的表妹媚兰。使思嘉更加不幸的是,战乱接踵而至,整个南方社会不得不投身战争岁月。在残酷的战争和艰辛的生活这双重重压之下,历经磨难的郝思嘉成了一位二十八岁的成熟女性。
\par 郝思嘉的父亲郝嘉乐是个爱尔兰移民,身无分文的他只身来到美国,通过玩一手好牌和喝酒的海量赢得了一片红色的土地,几经创业把其发展成一个收入颇丰的种植园。思嘉的母亲出身于海滨城市萨凡纳的名门望族,因为情场失意赌气嫁给了比她大将近二十岁的郝嘉乐。作为他们的大女儿,思嘉既沿袭了父亲豪爽、粗犷、不拘小节、脾气暴躁的性格,自小又受到母亲良好家教和道德观念的教诲。所以,她的性格是个矛盾的统一体。她既想做个像她妈妈那样有大家闺秀风范的淑女,骨子里又有背叛妈妈的道德框框的反骨。正是血管里流着的这种充满矛盾的血液造就了思嘉敢爱敢恨、认定自己的目标便勇往直前、不择手段的性格特点。
\par 小说《飘》出版后,美国评论界对郝思嘉的评价莫衷一是,有人把郝思嘉说成是一个毫不足取的女性。美国诗人约翰·P.毕晓普曾经说过:“在任何情况下,郝思嘉都是毫不足取的。她吝啬迷信,还自私自利,简直无人可比。她显然属于她那一阶层的一员,但她只有在少女时代才在表面上有点该阶层的言谈举止;至于他们的情感,她却从来没有共享过。人是要有精神的,这一点于她是不可理解的,至于说思想,她知道得最多的就是那种属于小农意识的卑劣的狡诈伎俩。基于这一点,除了她那珍贵的皮肤、土地和钱财以外,她什么也不看重。而这些正是使她的狡诈伎俩可以永远延续下去的东西。她手里抓着这个,眼里又觊觎另一个,为此,她杀了一个前来偷盗的北方士兵,洗劫了他的尸体、结了好几次婚、购买锯木厂、剥削囚犯的劳动、行使欺骗术、无情地把好几个人送上了西天。”
\par 可以说,毕晓普用洗练的概述把郝思嘉为人鄙视的一面做了精确的描述。然而,作为纷繁复杂的社会的一员,人的性格绝对不可能是单一的。所以,既没有绝对的好人,也没有全然的坏人。人只能是个多面体,人的性格也只能是多种性格特点的总和。主人公郝思嘉就是这样的多面体之一。在郝思嘉身上,我们可以很清楚地看到传统与反传统的冲突在她身上的体现。毋庸置疑,她的性格有不足取的一面,但同样也有为人欣赏的一面。尽管她有这样那样的缺点,但她还是受到广大读者的欢迎,而使这一点成为可能的正是她性格中为人欣赏的那一面。
\par 郝思嘉虽出生于南方种植园主家庭,但她从小就是个与众不同的女孩,对南方上流社会那些条条框框有着天生的反感。她讲究实惠,认准了自己的目标就不顾一切地去实现它,根本不管她采用的方法和她置身其中的社会准则相符不相符。所以,她和媚兰、希礼以及亚特兰大上流社会的那些“老卫兵”们格格不入,招致了“老卫兵”们的颇多指责和评判。可是,思嘉的信条没有改变,这就是:不管战争把原有的美好生活变得多么面目全非,不管社会发生了什么样的变化,她还是要不惜一切代价生存下去,她必须竭尽全力保住塔拉——那是在她陷入困境时会给她予力量的土地。而在她为生存而奋斗的过程中,她性格中为人称道的一面也就凸显了出来。
\par 玛格丽特在书中刻画了诸多南方妇女形象。通过对比,郝思嘉毫不虚伪、充分表现“真我”的性格特点便在读者面前一览无遗。在故事发生的那个年代,上流社会对妇女的要求是颇为苛刻的。女孩子要让先生们欣赏,很大的一面就是要伪装自己,把真正的自我隐藏起来。不管这个女孩多么聪明,多么有主见,她在先生们面前都要表现得很柔弱、很无知。她们最好是胆小如鼠的懦弱女子,一见到老鼠就跳到凳子上;一听见令人惊愕的事就要晕过去;在别人家吃东西要像小鸟一样少,哪怕是别人的宴会上有许多美味佳肴而自己也很想品尝也白搭;对先生们说话要表现得尽量无知,即使她们认为先生们其实很愚蠢,她们也还得假装崇拜他们的样子,要不时违心地对先生们夸上几句。这么做的目的无非是为了能合乎上流社会的习惯和所谓的美德,为了能找一个体面、尊贵、有钱的丈夫;而一旦结了婚,她便成了男人的附庸,成了生儿育女的机器,而结了婚的女人自己亲自打点生意,就算她的丈夫是个很不精明的生意人,那也是离经叛道的行为,是绝对行不通的。然而,郝思嘉对这些做法嗤之以鼻,对所有这一切发起了义无反顾的挑战。
\par 作者对思嘉的反叛行为最集中的描述就是她怂恿卫希礼和她私奔以及她婚后自己经营锯木厂这两件事情上。年方二八的郝思嘉爱上了貌似风流倜傥的邻居卫希礼。遗憾的是,卫希礼却要和他的表妹媚兰结婚了。思嘉为了得到自己的所爱,采取了大胆的行动。在宣布卫希礼和媚兰要结婚的野餐会上,思嘉想办法单独面见希礼,坦言自己对他的爱情,怂恿他和自己私奔。遭到拒绝后,思嘉毫不犹豫地给了他一巴掌。而后,为了报复,她不假思索地嫁给了媚兰的哥哥查理。读者可以想象,在当时传统习俗根深蒂固的美国南方,一个女孩子要作出这样的举动要有多大的勇气。郝思嘉在这个问题上表现了她敢爱敢恨的个性,一如她一开始对白瑞德的恨意。她不像别的女孩,把爱深埋在心里,不敢对自己所爱的人言明。在她看来,哪怕有一线希望,也应该争取得到自己的幸福。
\par 作者对郝思嘉表现真我的个性刻画还体现在另外一件事情上。那就是,郝思嘉在嫁给第二任丈夫弗兰克后,自己借钱买下一家锯木厂。让全体亚特兰大人目瞪口呆的是,她居然自己亲自经营锯木厂,根本不理睬对她此举持反对意见的弗兰克。按照亚特兰大传统的思维,嫁给弗兰克后的思嘉应该安分守己,让开店的弗兰克养活自己,自己在家里当个相夫教子的太太。可是,思嘉的举动却使亚特兰大人瞠目结舌。她不但在弗兰克生病时接管了店铺的生意,让弗兰克在邻里乡亲面前抬不起头来,而且私自买下了锯木厂,当上了名副其实的女商人。这个举动虽然算不上大逆不道,可对于女人来说也是非常出格的。更令亚特兰大人气愤的是,她凭着自己的姿色和独特的经营方式,挤垮了同行中的男性竞争对手,成了木材行业里的佼佼者。思嘉的举动成了别人议论的中心,闲言碎语、造谣中伤铺天盖地而来。然而,思嘉对这一切置之不理,照样我行我素,朝自己认准的目标前进。其实,思嘉在这一点上的做法正是现代社会中商场竞争的写照。竞争应该对每个人都是平等的,男人也罢,女人也罢。强者存,弱者汰。从这点上说,十九世纪的郝思嘉倒是有了超前的竞争意识和竞争能力。
\par 思嘉性格为人称道的另外一点是她的责任心。尽管她不喜欢她的妹妹,尽管她对自己的孩子照顾不周,尽管她对黑人态度严厉,但她在最困难的时候并没有抛下大家不顾,而是千方百计统筹安排,带领大家咬紧牙关,挺过饥饿交加的最艰难的时期。她义无反顾地把一切承揽在自己的肩上,而这负荷本来是要有两个男人才负担得了的。可她父亲傻了,母亲去世了,身为大女儿的她成了一家之主,她有责任承担这一义务,而她也确实义不容辞地履行了这一职责。为了避免失去家园、无家可归的悲惨命运,她违心地嫁给了她一点都不爱的弗兰克,用自己的幸福为代价换来了挽救塔拉的三百美元。她后来处心积虑地经营锯木厂,千方百计地赚钱,一方面是为了自己不再会有挨饿受冻的威胁,另一方面也是为了塔拉能够维持下去,为了有朝一日塔拉能够恢复过去的风采,也为了家里人能够安安稳稳地生活。她虽然也暗暗诅咒这种职责,恨不得能把这些负荷通通甩掉,但是,正如希礼所说的,她永远也做不到这一点。
\par 她的责任心不但表现在她对自己亲人的照顾上,同样也表现在她对媚兰的态度上。媚兰是查理的妹妹,也就是思嘉的小姑。希礼参军后撇下媚兰孤身一人面对没有男人保护的孤寂,面对生孩子的痛苦,面对战争带来的恐惧。在这样的时刻,陪伴她的只有思嘉。其实,媚兰代替自己占据了希礼的妻子这个位置,思嘉有足够的理由不去关照她。可她虽然打心眼里不喜欢媚兰,甚至暗暗诅咒她死,但她答应过希礼要照顾媚兰。为了履行自己的诺言,她不顾自己的生命危险保护她,陪伴她。因为她不仅仅是希礼的妻子,而且还是她的小姑。从思嘉对媚兰的态度,读者似乎也能预见到媚兰死后,思嘉肯定又会承担起照顾希礼和他的儿子的义务,因为她已经在媚兰临终前答应了她。
\par 如果说思嘉对媚兰的照顾完全是因为顾及希礼的情面,是为了她所爱的人的话,那思嘉对白蝶姑妈的照顾就跟爱情没有任何瓜葛了。白蝶是查理的姑妈,思嘉自从来到亚特兰大后就已经把照顾白蝶当成自己的责任了。媚兰怀孕后,按理留下来帮助媚兰的应该是上了年纪的白蝶姑妈,可白蝶姑妈老早就扔下媚兰逃难去了。因为她没有能力照顾媚兰,也没有勇气面对北方军的来临。而在思嘉嫁给白瑞德后,白蝶姑妈的生活来源也全都靠思嘉。没有思嘉,她根本没有能力生存下去。因为她的产业全被战争给毁了,而身为侄女和侄女婿的媚兰和希礼自顾不暇,根本没有经济能力来资助她。所以,总的说,思嘉是一个很有责任心的人。姑且不管她这么做时乐意不乐意,但她毕竟做了,尽了一份责任。所以,她在这方面为人称道的一面是不应该被抹杀的。
\par 思嘉的性格中最能给人鼓舞的一点还是她面对现实、不畏困难的精神。综观郝思嘉的一生,从故事开篇情场失意开始,打击一个连着一个。如果不是能够面对现实这一点支撑着她,她早就会被挫折、困难打倒了。年仅十六岁的郝思嘉就经历了失恋的痛苦,紧接着是丧夫的伤痛。年仅十七岁的她就已经成了有一个儿子的寡妇。如果说这一切都还只是个人生活上的不幸的话,那席卷整个南方的战乱给她带来的痛苦就是人所共知的了。我们来看看这么一幕:亚特兰大失陷前夕,郝思嘉拖着刚刚生过孩子的奄奄一息的媚兰和自己被炮火及北方军吓坏的孩子逃离亚特兰大,历经千辛万苦回到塔拉。思嘉从小崇拜妈妈,一有困难就去寻求妈妈的保护伞。此时的她之所以一心要回家,是因为她认为到家了就可以卸下自己肩头的担子,天塌下来自有爸爸妈妈去顶住,回到家后的她又可以过上少女般无忧无虑的日子。殊不知,正当思嘉为塔拉没有被无情的战火摧毁感到庆幸时,一场更大的灾难正等着她。回到家的她愕然发现,妈妈在前一天刚刚去世,爸爸因为妈妈的辞世已经傻了。家里十来张嘴要吃饭,而塔拉种植园留给她的却几乎一无所有。
\par 注视着默默望着她的一双双眼睛,面对一张张面黄肌瘦的脸,思嘉没有绝望,没有气馁,她既没有沉溺在过去美好的岁月中,也没有自暴自弃,得过且过。她下决心要让塔拉存在下去,要让塔拉的人挺过这个艰难时世。她亲自下地摘棉花;拎着篮子在烈日下到邻居废弃的果园里挖剩下的蔬菜;骑着唯一的一匹孱弱的小马到邻居家借种子,了解外界的情况;甚至杀了一个前来偷盗的北方士兵。在塔拉受到要挟、大家面临无家可归的威胁时,她带着嬷嬷来到亚特兰大,想利用自己的魅力从白瑞德手中借钱挽救塔拉。此计不成,她转而向小有资财的弗兰克展开攻势,终于让他拜倒在她的石榴裙下。虽然思嘉把她妹妹的男朋友夺了过来,而这也招致了许多人的指责和非难,但是,她不畏困难,敢于面对困难、想尽方法克服困难的勇气着实令人钦佩。
\par 我们再来看看小说的结尾。真心爱慕思嘉的白瑞德最终因为失望而决定离开思嘉,而此时的思嘉刚刚才意识到自己真正爱的人其实不是卫希礼,而是白瑞德,只是自己一直不知道而已。她希望他们能够重新开始,从此幸福美满地生活在一起。可是,白瑞德觉得自己虽然与思嘉生活在一起,但两人的心从来没有合二为一过,爱女的夭折更是使他产生了绝望心理。面对瑞德的离她而去,思嘉虽然也感到伤心、难过,但她没有撒泼耍赖,而是坚强地接受了这一令人难以接受的事实。“我明天再想这事好了,到塔拉去想。那时我就承受得了了。明天,我要想个办法重新得到他。毕竟,明天又是另外一天了。”这就是思嘉在碰到困难时屡试不败的法宝。
\par “明天又是另外一天了。”这是思嘉的座右铭。她相信,所有的一切痛苦和挫折都将成为过去,明天将会是另一个开始。只要自己付出努力,一切都会好起来的。思嘉一生坎坷,历经磨难,支撑她挺过一道道难关、克服一个个困难的就是这一信条。小说作者原来是要用“明天又是另外一天了”作为小说的书名的。据有关资料记载,作者写本书时最先写好的即是最后一章。可见,作者着重要表现的就是思嘉的这一精神。
\par 有人认为,《飘》出版的年代是三十年代,正是美国历史上的大萧条时期。由于经济滑坡,全国人口失业率激增,许多人生活没有保障,过着艰难的日子。人们于是很想逃避现实,试图回到过去的岁月当中去。他们发现自己正在为生存打一场恶战,这场恶战和内战以后重建时期郝思嘉为生活而打的战役如出一辙。它们同样艰辛,同样困难。虽然郝思嘉采取的作战方式并不是全都合乎道德规范的,但是她至少没有躺下等死,而是竭尽全力去拼搏,去奋斗。人们从郝思嘉身上多少获得了面对现实、克服困难的勇气。这是该小说一出版就成了畅销书的原因之一。此种想法不无道理。其实,郝思嘉不畏困难、面对现实的精神和勇气也正是小说历经一个多世纪而魅力仍经久不衰的原因所在。
\par 面对现实、克服困难这一信条不但适用于大萧条时期,而且适用于任何年代。生活对每个时代的每个人来说都是不易的。谁要是在困难面前低头,那他就是生活的弱者;而如若他不畏困难,勇敢地面对现实,想办法解决困难,那他就是生活的强者。写到这里,我不禁想起翻译本书过程中发生的一件事。纯粹是出于偶然,在我翻到郝思嘉从亚特兰大长途跋涉回家后却面临母逝父傻的不幸时,我的生活中也遭遇了丧母的创痛。母亲在一九九八年底被病魔夺去了生命。当时的感觉一如郝思嘉的感觉,我深切体会到失去亲人的切肤之痛。母亲病重、住院及去世期间,我曾一度中断了本书的翻译工作。而一段时间后使我重坐案头埋头翻译的不是别的,正是郝思嘉克服困难的这种勇气。记得我当时也经常翻到小说的最后,默默诵读思嘉克服困难的法宝,把丧母之痛深深埋在心底,重新投入到自己的工作、生活中去。我想,九泉之下的母亲要是知道我能化悲痛为力量,一定会感到欣慰的。
\par 一晃一年又过去了,如今《飘》终于要付印出版了。我心里除了高兴,亦有不少感慨。记得结束最后一遍修改时正是深秋时节一个凉风习习的夜晚。历时两年的工作终于告一段落,我不禁从胸腔里抒出一口长气,顿时感到一种从未有过的轻松感,一阵喜悦弥漫了我的周身。我信步来到阳台上,举目四望。城市的这一角华灯点点,霓虹灯五颜六色。远处的街道上,车流人流影影绰绰。更远处的大海上,点点渔火飘荡在海面上。停泊在厦门港的大客轮上灯火闪烁,把条鹭江点缀得分外妖娆。我仰头遥望天空,无数星辰眨着眼睛回望着我,似乎在告诉我,明天又是一个大晴天。不知怎的,小说最后那句话又在我耳际回响:“明天又是另外一天了!”是呀,今天即将过去,不管有过什么成绩,或是有过什么痛楚,一切都只属于过去。而明天,已经是另外一天了!只有把每一个明天当作新的起点,为实现自己的目标更加努力奋斗的人才算得上是智者。
\par \rightline{李美华}
\par \rightline{于厦门大学凌峰楼}


\subsection{第一部}


\subsubsection{第一章}

\par 郝思嘉其实长得并不漂亮。然而,男人们被她的魅力迷住时,却极少意识到这一点。塔尔顿家那一对孪生兄弟就是如此。她的脸上显然融合了她的母亲——沿海一位法兰西血统的贵族和她的父亲——爱尔兰后裔的特点,既标致娇柔,又红润粗犷。这张脸实在迷人,非常引人注目,尖尖的下巴,方形的下颚,双眼则呈淡绿色,一点茶褐色也没有。黑黑的睫毛圈在眼睛周围,尾部还微微有点翘,带着点欢快俏皮的模样。眼睛上方,两道墨黑的浓眉向上翘起,在她那像木兰花一样洁白的皮肤上画出两道颇为抢眼的斜线。南方的太太小姐们都非常珍视这种肤色。她们总是戴着帽子、围着面纱、戴着露指长手套,小心地呵护着自己的皮肤,以免让佐治亚州炎热的太阳光晒黑。
\par 一八六一年四月一个阳光明媚的下午,在她父亲的塔拉种植园里,郝思嘉和斯图尔特·塔尔顿、布伦特·塔尔顿兄弟俩一块坐在阴凉的游廊里,坐态显得优美极了。她身穿一件簇新的绿色花布长裙,裙环撑开了宽及十二码的飘曳裙摆。这和她脚上的绿色摩洛哥皮平跟拖鞋极为相配,鞋子是不久前她父亲在亚特兰大给她买的。裙子完美地衬出她那仅有十七英寸的腰身,这也是三个县的女孩中最纤细的了。合体的紧身胸衣托出她虽只有十六岁却已发育成熟、丰满隆起的乳房。虽然她那宽大飘曳的长裙显得端庄朴素,头发也平滑地梳在脑后,挽成一个发髻,一双白皙而小巧的手规矩地叠放在大腿上,但是,她真正的性情并未得到很好的掩饰。在那张极其恬美的脸上,她那绿色的双眸显得骚动不宁、狡黠任性,而且生气勃勃,与她那副似乎很有教养的行为举止极为不符。她那副仪态纯粹是平日里在她母亲的温和训导以及她的黑人嬷嬷的严厉管教之下形成的,而这一切都是别人强加给她的。只有她的双眸才是与生俱来、能显示她本性的地方。
\par 塔尔顿家的斯图尔特和布伦特兄弟俩一边一个,懒洋洋地躺在放在她两边的躺椅上。他们肆意谈笑着,眼睛透过有薄荷属植物装点的高大玻璃窗斜睨着太阳光。他们随意地跷着二郎腿,修长的双腿穿着长及膝盖的长筒靴,腿部肌肉因长期骑马而异常发达。兄弟俩都是年方十九,身高六英尺二英寸,身材高挑,肌肉发达,脸膛被太阳晒得黝黑,头发则是茶褐色的。他们眼神欢快,目光傲慢,身穿一样的蓝色上装、芥末色马裤,像足了棉花丛中的两株棉桃。
\par 屋外,午后的阳光斜照在院子里,把山茱萸的树影投射到忽隐忽现的亮光中。虽然大自然刚泛出一片新绿,这些山茱萸却已结满了一团团、一簇簇洁白的花蕾。兄弟俩的马拴在车道边。马儿高大剽悍,毛色和它们主人的头发一样呈暗红色。马的脚边围着一群身子瘦长、颇不安分的猎犬,它们正在吵吵闹闹、狂吠不已。不管斯图尔特和布伦特兄弟俩走到哪里,这群猎犬总是伴随其左右的。较远处还躺着一只有着黑色斑点的随车狗。它似已成了一名贵族,鼻子凑在前爪上,耐心地等着兄弟俩回家吃饭。
\par 在猎犬、马儿和哥儿俩之间,除了他们一贯的交情外,似乎还有更深一层的血缘关系。猎犬和马儿同样都是身体健康、没有思想的年轻动物。它们毛发光滑、壮健漂亮、勇猛活跃。而哥儿俩跟他们的坐骑一样骁勇而顽皮,顽皮得甚至到了危险的地步。但是,谁要是摸清了他们的脾气,知道如何驾驭他们,他们的性情却又会好得出奇。
\par 尽管一生下来就在种植园里过着安逸的生活,从娘胎里一落地便由别人从头到脚伺候着,可是,游廊上三个人的面孔并不像是娇生惯养、无精打采的。相反,倒是像那些长年累月在室外劳作、很少费神去思考书本中的无聊之事的乡下农人,既精力充沛,又警觉活跃。在佐治亚北部的克莱顿县,生活还处于起始阶段。若用奥古斯塔、萨凡纳和查尔斯顿的标准来衡量的话,还多少有点原始。在南部开发较早的地方,那些老成持重的人对身居内陆的佐治亚人老大瞧不起。但在佐治亚北部,只要一个人在重要的事情上精明能干,那么,就算他没有受过一流的教育,也不是什么丢脸的事。而这些重要的事无非就是:棉花种得好、骑马骑得棒、枪法准确、舞步轻盈、对女士们表现得举止优雅、态度殷勤,还有,喝起酒来像个男人。
\par 在这些事情上,兄弟俩自然是出类拔萃的,可他们在学习书本知识方面表现出来的无能也同样远近闻名。在县里,他们家比任何人都更有钱,拥有的马匹和黑奴也更多。可要说到肚里的墨水,那么,他们那些穷苦的白人邻居当中,大多数人都比这哥儿俩要强得多。
\par 这个四月的下午,斯图尔特和布伦特兄弟俩之所以能够悠闲地躺在塔拉种植园的游廊上,原因正出于此。他们刚刚被佐治亚大学开除出门。这已经是两年中第四所把他们逐出校门的大学了。他们的两个哥哥——汤姆和博伊德也跟他们一块打道回府了。这所学校既然不欢迎他们的两个孪生弟弟,他们也就不愿意再留在那了。斯图尔特和布伦特把这次被校方开除当作绝棒的笑话,而思嘉小姐也跟他们一样觉得有趣极了。自从一年前离开了费耶特维尔女子学院,她就再也没有心甘情愿地打开过一本书。
\par “我知道你们俩根本不会把被开除当回事的,汤姆当然也不在乎,”她说,“可是博伊德呢?他一心想让自己接受良好的教育,可你们俩却一而再、再而三地把他也从大学里拖了出来,先是弗吉尼亚大学,接着是亚拉巴马大学,再是南卡罗来纳大学,现在又是佐治亚大学。他这个愿望是再也实现不了啦。”
\par “噢,他可以到费耶特维尔的帕马利法官那里去学法律,”布伦特漫不经心地回答说,“再说,这也没多大关系。不管怎么样,我们都得在学期结束之前回家来的。”
\par “为什么?”
\par “因为战争呀,傻瓜!战争随时都可能爆发,你总不至于认为烽火四起的时候我们还会待在学校里吧?”
\par “你们知道的,哪会有什么战争呀。”思嘉说着,感到有点心烦。“都只是说说罢了。上星期卫希礼和他父亲还跟我爸爸说,我们在华盛顿的委员们就南部邦联事宜和林肯先生达成了——哦——令人欣慰的一致意见。无论怎么说,北方佬也害怕我们会跟他们打起来。不会有什么战争的,我可不想再听到这些言论了,烦死人了。”
\par “不会有什么战争!”兄弟俩愤愤不平地叫了起来,就好像被别人骗了一样。
\par “哦,亲爱的,当然会爆发战争的,”斯图尔特说,“也许北方佬真的怕我们,但是,前天博勒加德将军用炮火把他们从萨姆特堡给轰跑了,这样,他们就不得不应战,否则,他们在世人面前就成了懦夫。哦,南部邦联——”
\par 思嘉做了个鬼脸,显出极不耐烦的样子。
\par “如果你们再提‘战争’这两个字,我就马上进屋去把门关上。这辈子我还从来没有对哪个词像对‘战争’这么厌恶过,令我更厌恶的两个字就只有‘脱盟’了。爸爸从早到晚都在谈论战争,来我们家看他的所有先生也都在大声叫嚷着什么萨姆特堡、州权、亚伯·林肯\footnote{亚伯拉罕·林肯的昵称。},我已经烦透了,烦得我几乎要尖叫起来。而所有的男孩也都在谈论这件事,谈论他们那个老骑兵连。就因为所有的男孩除了此事就不会谈点别的,自今春以来的晚会从来没有过什么乐趣。我很高兴佐治亚州是等到圣诞节过后才退出联邦政府的,要不它就把那些圣诞晚会都给毁了。假如你们再提‘战争’这两个字,我就马上进屋去。”
\par 她是认真的,不是说着玩的。对于不是以她为中心的谈话,她从来就不会忍受太久。但说这些话时,她脸上却挂着微笑,还刻意使脸上的酒窝显得深些。她飞快地眨着眼睛,那欢快俏皮的黑睫毛便一张一合的,就像蝴蝶在扇动着美丽的翅膀一样。她这么做,存心是要让两个男孩对她着迷,而他们也确实被她迷住了。他们赶紧向她道歉,说自己让她心烦了。他们并不因为她对战争毫无兴趣就看不起她,却反而把她看得更重。战争毕竟是男人的事,不是女人的事。他们认为,她的这种态度只不过证明她更有女人味罢了。
\par 思嘉略施小计,成功地使他们停止谈论战争这令她厌烦的话题,而后便又饶有兴趣地谈起眼前的事来。
\par “你们俩被开除了,你们的妈妈有什么看法?”
\par 三个月前,兄弟俩被弗吉尼亚大学勒令退学。一想起当时他们回到家时他们母亲的态度,两个男孩便显得很不安。
\par “哦,”斯图尔特说,“她还没有机会对此说什么。今天早上她还没起身,汤姆和我们就溜出来了。他到方丹家去,我们就上这来了。”
\par “你们昨晚到家时,她难道没说什么吗?”
\par “昨晚我们可是交上好运了。我们还没到家,妈妈上个月在肯塔基新买定的那匹种马被送了过来。家里简直闹翻天了。这是一匹雄健的好马,思嘉,你该叫你父亲马上到我们家去看看。在被送到这来的路上,这高大的畜生竟然把马夫的肉给咬掉了一块,还把我妈妈派到琼斯伯勒火车站去接车的两个黑鬼给踩了。就在我们到家前,它正试图把马厩踢翻呢,我妈妈原有的那匹叫草莓的种马也被它折腾得半死。我们到家时,妈妈正在马厩里拿着一袋糖试图哄它安静下来。她做得好极了。可黑奴们都躲得远远的,眼睛瞪得老大,他们都被吓坏了。可妈妈却跟马说着话,好像它和人一样、正从她手里吃东西呢。对马呀,还真没有人像我妈妈这么有办法的。她一看到我们就说:‘我的天哪,你们四个人又回家来干什么?你们简直比埃及的祸患还糟糕!'\footnote{根据《圣经·旧约》“出埃及记”的记载,由于上帝将一连串的祸害降到了埃及人的头上,埃及法老不得不允许受奴役的以色列人离开埃及。}这时,马又开始喷着鼻息又嘶又叫的,还用后腿站了起来。她赶忙说:‘快离开这!没看到这个高大的宝贝正躁动不安吗?我明早再跟你们四个算账!’我们就全都上床睡觉去了。今天一大早,还没等她逮住我们,我们就开溜了,只剩下博伊德去应付她。”
\par “你们认为她会不会打博伊德呀?”像县里其他人一样,思嘉也看不惯个子矮小的塔尔顿太太对待她那些已经长大成人的儿子们的方式。她不但会打骂他们,有机会竟然还会用马鞭抽他们。
\par 比阿特丽斯·塔尔顿是个很忙碌的女人,她手里不但有一片很大的棉花种植园、上百个黑奴以及八个孩子,而且还拥有全州最大的马匹饲养场。她脾气非常暴躁,动不动就被她四个经常惹是生非的儿子搞得苦恼不堪。她虽然不允许别人鞭打马匹或是黑奴,可是对她这些儿子,她倒觉得,不时给他们来那么一两下决不会伤着他们什么。
\par “她当然不会打博伊德。他是老大,又是我们这伙人中个子最小的,她从来就没有真正打过他。”斯图尔特说,说话间对自己六英尺二英寸的高个头颇为得意。“所以我们才留下他去向她解释一切。见鬼,妈妈不该再打我们的!我们都已经十九岁了,汤姆也已经二十一了,可她却还把我们当成只有六岁的孩子。”
\par “明天卫家的野餐会,你妈妈会不会骑着这匹新买的马去参加呢?”
\par “她当然想骑着它去,可是爸爸说这太危险了。再说,我们家那些女孩子也不会让她这么做。她们说,至少她们得让她像个贵夫人那样,坐着马车去参加晚会。”
\par “希望明天不会下雨,”思嘉说,“这一整个星期几乎天天都在下。若是野餐变成室内聚餐,那就太扫兴了。”
\par “噢,明天会天晴的,一定会热得像六月天一样,”斯图尔特说,“你瞧那轮落日,我还没见过比这更红的呢。我们总是可以通过落日来判断天气的。”
\par 他们望着郝家那一片绵延不断、新犁过的棉花地,一直延伸到被落日映红的天边。太阳正徐徐落向弗林特河对岸的山峦后面,把那一片天空照得通红。四月的暖意也随着太阳的降落而变为一种让人感到颇为舒服的微微的凉意。
\par 这一年的春天来得特别早。温暖急骤的春雨潇潇而下,粉红的桃花和雪白的山茱萸便竞相怒放,把墨黑的河流两岸及远处的山峦装点得分外漂亮。春耕已经接近尾声。佐治亚州的土壤本来就是红色的,上面新犁出的垄沟便被那轮血红的落日映照得更加绚丽夺目。翻起的潮湿的泥土正焦急地等着棉花种子投入它的怀抱。一条条垄沟映着落日,顶部的凸处呈现出粉红和浅红,沟底的凹处则是朱红、猩红和赭红。种植园里白色的砖房恰如宽广无垠的红色大海上的一座岛屿。海面汹涌澎湃、波涛起伏,翻腾的巨浪和那顶部呈粉色的波涛撞到一起,稍顿片刻,旋即变成拍岸浪花,四散开去。这里的垄沟既不太长,也不很直,而在平坦的佐治亚州中部那土壤呈黄色的田野上,或是沿海种植园里那芬芳的黑色土地上,你就能看到既长且直的垄沟了。可在佐治亚北部绵延起伏的丘陵地带,田地则被犁成无数弯弯曲曲的垄沟,以防肥沃的土壤被雨水冲到低处的河底去。
\par 这是一片原始的红土地。大雨过后是一片猩红,干旱期间则砖屑飞扬。这里是世界上最适合棉花生长的地方。这块土地上,白色的房屋星星点点,犁过的田地静穆安详,黄色的河流流速缓慢,一派令人愉悦的景象。但这也是一片对比强烈的土地,既有最烈的太阳光,也有最阴凉的所在。种植园里的开阔地和绵延数英里的棉花地对着和煦的阳光点头微笑,一副平和满足的样子。它们的边沿则是一片片未开垦的林地。即使在最热的中午,那里也是既阴暗又凉爽的,而且还带着某种神秘感和些许邪恶感。古老的松树飒飒作响,似乎在耐心地等待着什么,同时叹息着对人们发出威胁:“当心!当心!你们曾经属于我们。我们一定能再把你们夺回来。”
\par 在田地里忙活的人们和骡子日暮归来了,游廊上三个人的耳边便回荡着脚步声、马具上链条的叮当声以及黑人毫无顾忌的尖声谈笑声。屋里传来思嘉的母亲埃伦轻柔的话语,她正在呼唤给她提放钥匙的篮子的黑人小女孩。小女孩尖声的童音回答着:“是,夫人。”脚步声便朝着后面熏肉房的方向渐渐远去,那里是埃伦给归来的人手分发食物的地方。而后又是一阵陶瓷及银制餐具的响声传来,塔拉的男管家波克已经在摆桌子准备用餐了。
\par 听到这些声音,兄弟俩意识到他们该动身回家了。但他们不愿意回去面对他们的母亲,于是一直逗留在塔拉的游廊上,心里盼望着思嘉会邀请他们留在那吃饭。
\par “我说思嘉,我们说说明天的事。”布伦特说,“因为我们一直不在,不知道野餐会和舞会的事,明天晚上我们没有理由不跳个够。你还没有答应别人吧?”
\par “哦,我当然已经答应别人了。我怎么知道你们都会回来呢?我才不想为了只伺候你们俩而把自己变成舞会上受冷落的小可怜虫。”
\par “你会成为受冷落的小可怜虫!”两个男孩乐得捧腹大笑。
\par “我说宝贝,你得答应跟我跳第一支华尔兹,跟斯图跳最后一支。你还得跟我们一起吃晚饭。我们再坐在楼梯平台上,就像上次舞会时那样。再让吉茜嬷嬷来给我们算命。”
\par “我可不喜欢吉茜嬷嬷算命。你们知道的,上次她说我会和一个头发乌黑发亮、胡子又长又黑的先生结婚。我才不喜欢黑头发的先生呢。”
\par “你喜欢红头发的,对不对,宝贝?”布伦特咧嘴笑了,“来吧,答应我们,跟我们跳所有的华尔兹舞曲,并且和我们一起吃晚饭。”
\par “如果你答应我们,我们就告诉你一个秘密。”斯图尔特说。
\par “什么秘密?”思嘉听到这话,像小孩一样兴奋地叫了起来。
\par “是不是我们昨天在亚特兰大听到的,斯图?如果是的话,你知道我们答应过不说出去的。”
\par “不错,是白蝶小姐告诉我们的。”
\par “什么小姐?”
\par “就是卫希礼的远房亲戚,住在亚特兰大的韩白蝶——也就是韩查理和韩媚兰的姑妈。”
\par “我知道的,她是个傻乎乎的老太太,我一辈子也没见过第二个她那样的。”
\par “我们昨天在亚特兰大等火车回家,她的马车正巧经过车站,她就停下来和我们说话。她告诉我们,明天晚上卫家的舞会上要宣布一个人订婚的消息。”
\par “噢,这个我知道。”思嘉失望地说。“就是她那个傻侄儿——韩查理和卫哈尼的事。他们迟早要结婚的,这事大家都知道好几年了,虽然查理自己似乎对此事兴致不高。”
\par “你觉得他很傻吗?”布伦特问道,“去年圣诞节时,你可是尽让他围着你转呢。”
\par “他要缠着我,我也没办法呀。”思嘉不屑地耸耸肩,“我认为他女人气太足了,婆婆妈妈的。”
\par “再说,也不是要宣布他要订婚,”斯图尔特得意洋洋地说,“而是希礼和查理的妹妹媚兰小姐!”
\par 思嘉虽然脸上不动声色,嘴唇却刷地变白了——就像是毫无防备被人猛击了一拳似的。刹那间,她只是惊异万分,根本反应不过来究竟发生了什么事。她一动不动地盯着斯图尔特,从来不动脑筋的他便想当然地认为她只是对此事颇感吃惊,并且觉得很有趣罢了。
\par “白蝶小姐告诉我们,由于媚兰小姐身体一直不太好,他们本来打算明年再宣布的。可是现在到处都在谈论战争,他们两家人都认为还是趁早结婚的好。所以决定在明天晚餐时宣布。好了,思嘉,我们已经把秘密告诉你了,你得答应明天晚上和我们一块吃饭。”
\par “我当然会答应的。”思嘉机械地回答说。
\par “还要跟我们跳所有的华尔兹舞?”
\par “行。”
\par “你真是太好了!我敢打赌,其他男孩一定会气得跳起来的。”
\par “让他们去气好了,”布伦特说,“我们俩可以对付他们的。哦,思嘉,早晨的野餐也跟我们坐一块吧。”
\par “什么?”
\par 斯图尔特重复了他的请求。
\par “当然。”
\par 兄弟俩你看看我,我看看你,高兴极了,可又觉得有点奇怪。虽然他们自认为是思嘉心目中喜爱的意中人,可他们从来没有像今天这样轻而易举地得到这份殊荣。通常,她总是要他们一再请求,她则一再搪塞,既不说行,也不说不行。他们若不高兴,她就乐得哈哈大笑;而一旦他们生气,她就故意冷落他们。可是现在,她却几乎答应明天一整天都跟他们待在一起——野餐和他们坐在一起,所有的华尔兹舞曲都跟他们一起跳(他们当然会安排好所有的舞曲都播华尔兹!),晚宴的时间也归他们所有。能够这样,那么,被大学开除也是值得的。
\par 他们的成功使他们兴致大增。他们继续逗留在那儿,谈论着野餐和舞会、卫希礼和韩媚兰,还不时打断对方的话,说说笑话,相互逗乐,同时明显地暗示思嘉邀请他们吃晚饭。过了好一阵,他们才觉察到思嘉已经没什么话可说了。不知怎么的,谈话气氛已经变了。到底是怎么回事,这哥儿俩也是丈二和尚摸不着头脑,反正一整个下午的欢快气氛已经悄然而逝、无影无踪了。对他们所说的话,思嘉似乎并不很在意,虽然她还能明白无误地回答他们。这其中一定有什么他们不明白的东西,兄弟俩觉察到这一点,感到颇为不解和不安。但他们还是在那儿又赖了好一会,可最终还是看了看手表,很不情愿地站起身来。
\par 从新犁过的田地望过去,太阳已经渐渐西沉,马上要落到山后面去了。河那边高大的树木隐隐现出黑魆魆的轮廓。家燕在院子里急速地冲来冲去,一群群鸡、鸭、土鸡等踱着方步,大摇大摆地从田野里四散归来。
\par 斯图尔特大叫了一声:“吉姆斯!”过了一会,一个和他们年龄相仿、身材高大的黑人小伙子气喘吁吁地从屋子边上应声跑了过来,向车道边拴着的马跑去。吉姆斯是他们的贴身男仆,就像他们的狗一样,不管他们走到哪里,他都跟到哪里。自孩童时代起,他就是他们的玩伴。他们十岁生日那年,他就被送给兄弟俩作贴身仆人了。一见到他,塔尔顿家的猎狗便从一片红色的尘土中立起身来,等候着它们的主人。兄弟俩弯腰行了行礼,和思嘉握手道别,告诉她,明天一大早他们就会到卫家去等她。然后,他们匆匆忙忙走上人行道,飞身上马,沿着两旁栽满雪松的车道飞驰而去,一边还摘下帽子回头挥舞着向她道别。吉姆斯则尾随其后。
\par 他们在尘土飞扬的马路上拐了一个弯,塔拉便从视野里消失了。布伦特在一丛山茱萸树下停了下来。斯图尔特也驻马不走了。黑人男孩在离他们几步远处也跟着停了下来。缰绳一松,马儿乘机伸长脖子去吃春天嫩绿的青草,那耐心十足的猎狗又在松软赤红的尘土中重新躺下来,看着渐渐降临的暮色中盘旋飞翔的家燕,眼里露出渴望的神情。布伦特那张天真的宽脸庞上一脸的困惑不解,并且颇有点愤愤不平的样子。
\par “我说,”他说,“你不认为她应该留我们吃饭吗?”
\par “我原以为她会这么做的,”斯图尔特说,“我一直在等她开口,她却没有。你说这是怎么回事?”
\par “我也不知道是怎么回事。我只是觉得她似乎应该留我们吃饭的。毕竟今天是我们回家后的第一天,她也好一阵子没看到我们了。我们还有一大堆事要告诉她呢。”
\par “我认为,我们来的时候,她倒是很高兴看到我们的。”
\par “我也这么认为。”
\par “后来,也就是半小时前,她就有点变沉默了,就像得了头痛一样。”
\par “我也觉察到这一点了,但我当时没在意。你认为是什么使她感到不高兴呢?”
\par “我也不知道。你说,会不会是我们说了什么话让她生气了?”
\par 他们都低头想了一会。
\par “我可想不出什么来。再说,思嘉生气的时候可是大家都看得出来的。她不像有些女孩子那样全藏在心里。”
\par “是的,这正是我喜欢她的地方。她生气的时候也不会冷落你或是怀恨在心——她会直接告诉你。但是,应该是我们做错了什么或是说错了什么才使她闭口不言的,她看上去就像生病了一样。我敢发誓,我们来的时候她是很高兴看到我们的,而且还有留我们吃饭的意思。”
\par “你认为会不会是因为我们被开除的缘故呢?”
\par “见鬼,绝不是的!别傻了。我们告诉她的时候,她还笑得不亦乐乎呢。再说,思嘉并不比我们俩更看重念书。”
\par 布伦特在马鞍上转过身来,对他的男仆吆喝了一声。
\par “吉姆斯!”
\par “少爷,什么事?”
\par “你有没有听到我们和思嘉小姐的谈话?”
\par “没呢,布伦特少爷!你怎么会认为俺敢偷听白人老爷的谈话呢?”
\par “偷听,我的天哪!你们这些黑鬼,没有什么事是你们不知道的,你分明是在撒谎。我亲眼看见你在游廊的拐角处鬼鬼祟祟的,还蹲在墙边的茉莉花丛的阴影下。说吧,你有没有听到我们说过什么话使思嘉小姐不高兴了——或是什么会伤她感情的话?”
\par 被这么一问,吉姆斯便不再找借口申辩自己没听到他们的对话了。他皱着眉头。
\par “没呢,少爷。俺没听到你们说过什么会让她生气的话。俺觉得她是很高兴见到你们的,而且好像也想见到你们,她高兴得就像小鸟一样呢。但是,你们和她谈起卫希礼先生和媚兰小姐要结亲时,她就开始不出声了,就像一只看到空中有老鹰在盘旋的小鸟一样。”
\par 兄弟俩互相对视了一眼,点了点头。可还是百思不得其解。
\par “吉姆斯是对的。可我还是不明白为什么,”斯图尔特说,“上帝!卫希礼只不过是她的一个朋友罢了。她并不喜欢他,她喜欢的是我们。”
\par 布伦特点头表示同意。
\par “你会不会认为,”他说,“也许希礼还没有告诉她他明天晚上要宣布订婚的事,作为老朋友,他却没有在告诉别人以前先告诉她,所以她不高兴了。女孩子对比别人先知道这类事情是挺在乎的。”
\par “噢,也许吧。可是,就算他没告诉她是明天要宣布,那又怎么样呢?他们本来就要保守这个秘密,好给人们来个惊喜。而且,一个男人总有权保守自己订婚的秘密的,对不对?要不是媚兰小姐的姑妈把这事泄露给我们,我们也不会知道的。但思嘉也不是现在才知道他要和媚兰小姐结婚呀。我们都知道好几年了。卫家和韩家的人总爱跟他们的表亲通婚。人人都知道他十有八九要和她结婚的,就像卫哈尼要和媚兰小姐的哥哥查理结婚一样。”
\par “好吧,我同意这样解释不通。但她没有留我们吃晚饭,我感到很遗憾。老实说,我不想回家去听妈妈就我们被开除的事瞎唠叨。这可不是第一次了。”
\par “也许博伊德这时候已经使她心平气和了呢。你知道,那个小狐狸可是个了不起的说客。他总是能够使她平心静气的。”
\par “不错,博伊德的确能做到这点,但也得给他时间。他得绕很多圈子,一直到把妈妈给弄糊涂了,她才会让他好好保护嗓子,留待以后上法庭辩护时用。可他还没有时间来开始好好地唱这出戏呢。我敢打赌,妈妈一定还在忙乎那匹新买的马,她甚至根本没意识到我们又回家来了。今晚她坐下来吃饭看到博伊德时才会注意到这一点。而且晚饭还没吃完,她就会越想越气、火冒三丈的。一定要等到十点,博伊德才能找到机会告诉她,校长用那种方式跟你我谈过话后,我们中间不管是谁再留在学校里都是很没面子的。一直要到半夜,他才能设法让她把怒气转移到校长身上。那时,她就会问博伊德干嘛不一枪把他毙了。不行,我们得等到子夜过后再回家。”
\par 兄弟俩你看看我,我看看你,闷闷不乐的。对降服野马、打架闹事以及邻居们对他们的满腔愤慨,他们一点也不害怕。可是,对他们那红头发的母亲直言不讳的数落及毫不犹豫地往他们屁股上抽的马鞭,他们俩却颇为发怵。
\par “哎,我说,”布伦特说,“我们干脆到卫家去算了。希礼和那些女孩子一定会很乐意请我们吃饭的。”
\par 斯图尔特看上去便显得有点不安。
\par “不,我们还是别上那去。他们正急着准备明天的野餐会呢。再说——”
\par “噢,我把这给忘了,”布伦特急忙说,“那我们就别上那去了。”
\par 他们对马吆喝了一声,一言不发地往前骑了一阵。斯图尔特褐色的双颊泛起了一片尴尬的红晕。直到去年夏天,斯图尔特还在追求卫家的英蒂,双方家人以至全县的人都已认可了这件事。县里人都认为,或许冷静而有自制力的卫英蒂对他会起到一种镇静的作用。至少,他们非常希望如此。斯图尔特兴许是找对了对象,可布伦特对此却很不满意。虽然布伦特也喜欢英蒂,但他认为她太普通、太温顺了,他根本无法使自己也爱上她,好和布伦特做伴。兄弟俩第一次趣味不投。自己的兄弟居然会看上一个在他看来一点也不出众的女孩,布伦特对此颇有怨气。
\par 去年夏天,在琼斯伯勒橡树丛中的一次政治演讲会上,他俩突然注意到了郝思嘉。其实他们认识她已经有好些年头了。从孩提时代起,她就是个招人喜欢的玩伴。因为,不论是骑马还是爬树,她都几乎跟他们不相上下。可现在,他们都惊奇地发现,她已经出落成一个妙龄少女了,而且可以说是所有人中最有魅力的一个。
\par 他们第一次注意到,她那绿色的双眸秋波粼粼的,一笑起来便现出深深的酒窝。手脚既小巧又娇嫩,腰肢更是纤细动人。他们的花言巧语使她不时发出一串串银铃般的笑声。一想到她兴许会把他们视为出色的一对,他们更是使尽浑身解数表现自己。
\par 这是兄弟俩一生都无法忘怀的日子。自此以后,每当谈起这事,他们都感到很纳闷,怎么过去从来没有注意到思嘉这么有魅力呢?其实,他们自己绝对无法找到正确的答案,因为那天思嘉是存心要引起他们注意的。她的本性根本无法容忍一个男人爱上别的女人而不是她自己。在演讲会上,看到卫英蒂和斯图尔特在一起,这是她那要征服男人的本性决不能容忍的。可是,吸引了斯图尔特一人后她还不满足,于是又去勾引布伦特,结果还真的完完全全地把他们给俘获了。
\par 现在他们俩都深深爱上了她。过去,布伦特曾半真半假地追过拉夫乔伊的芒罗。可现在,卫英蒂和莱蒂·芒罗都早已被抛到脑后了。如果思嘉接受了他们中的一个,那被拒绝的另一个又该怎么办,兄弟俩从来没想过这个问题。反正,船到桥头自然直。目前,他们都对同一个女孩产生了爱意,为此他们感到很满足,因为他们之间不存在任何忌妒心理。这种情况,他们的邻居们都感到很有趣,可他们的母亲却为此颇为烦恼,因为她一点也不喜欢郝思嘉。
\par “如果那个狡猾的小妖精真的接受了你们中的一个,那也是你们罪有应得,”她说,“她兴许还会同时接受你们俩,那样的话,你们就得搬到犹他州去。或许那里的摩门教徒会收留你们——但我很怀疑他们会不会这么做……\footnote{摩门教是19世纪30年代创立于美国的一个教派,初期实行一夫多妻制。美国犹他州是摩门教徒聚居地。}我担心的是,你们很快就会为了那个狡黠奸诈、双眼泛绿的小尤物而喝得烂醉如泥,因争风吃醋而大打出手,甚至会用枪瞄准对方,让他脑袋开花。不过,这也许并不是什么坏事。”
\par 自那次演讲会后,斯图尔特在英蒂面前便感到很不自在。这并不是因为英蒂曾经指责过他,或是用眼神或手势暗示过她已经知道他突然间就已经移情别恋了。她是个颇有教养的淑女。但斯图尔特还是觉得愧对于她,跟她在一起便万分不自在。他知道,他已经使英蒂爱上自己了。在内心深处,他觉得自己太没有绅士风度。他至今还是特别喜欢她,因她冷静、良好的教养、她的博学多识以及她身上具备的所有优点而敬重她。但是,真见鬼,她老是让人觉得兴味索然、毫无生气,而且老是一成不变的。不像思嘉,不但欢快活跃,而且连魅力也是千变万化的。跟英蒂在一起,你绝不会忘记自己在什么地方,可跟思嘉在一起却一点这种感觉也没有。这就足以驱使一个男人意乱情迷了。再说,这其中也有无尽的魅力呢。
\par “哎,那我们到凯德·卡尔弗特家去,在那吃晚饭得了。思嘉说凯思琳从查尔斯顿回家来了。也许她会带回一些我们还没听到过的有关萨姆特堡的消息。”
\par “凯思琳可不会。我敢和你打赌,她甚至连萨姆特堡就在那港湾里都不知道呢,更不用说那里曾经驻扎着北方佬,直到我们用炮火把他们给轰跑。她就只知道她要去参加的那些舞会和她招引的那些花花公子。”
\par “哦,去听她唠叨唠叨也挺有趣的。这也是能避开妈妈的好去处,等她上床睡觉以后再说。”
\par “哦,见鬼!我倒挺喜欢凯思琳,她蛮有趣的,我也想去听听卡罗·雷特和其他查尔斯顿人说说话;但是,如果我还能容忍和她那北方佬的继母坐在一起再吃一餐饭,我就不是人。”
\par “别对她太苛刻了,斯图尔特。她人挺好的。”
\par “我没有对她太苛刻。我只是为她感到难过,可我不喜欢让我为其感到难过的人。她老是大惊小怪、小题大做的,总想把事情做好,让你有宾至如归的感觉,可最终总是话也说不对、事也做不好。她老让我烦躁不安!她还认为南方人都是野蛮人,居然还这么对妈妈说了。她怕南方人。我们一在场,她看上去就怕得要死。她让我想起蹲在椅子上的瘦骨嶙峋的老母鸡,虽然双眼还有光泽,但是目光呆滞,充满恐惧,一有动静就会扇动翅膀、咯咯大叫。”
\par “噢,这你不能怪她。你确实把凯德的腿给打伤了。”
\par “咳,我那时喝醉了,要不然我也不会开枪的,”斯图尔特说,“再说凯德并没有记恨我。凯思琳、雷福德和卡尔弗特先生也没有。只有他那个北方佬的继母哭哭啼啼的,说我是个野蛮人,还说体面人跟我们这些未开化的南方人在一起一点也不安全。”
\par “这你不能怪她。她是个北方佬,礼貌举止方面并不周全;而且你也确实用枪打伤了她丈夫和前妻生的儿子。”
\par “哦,去她的!那也没有理由侮辱我!你是妈妈的亲生儿子,那次托尼·方丹开枪打伤你的腿时,她有没有大为光火呢?没有,她只是派人去把老方丹医生请来给你包扎伤口,问医生是什么使托尼把枪打偏了。还说她猜想是醉酒使他的枪法不准了。你记得吗?这话简直把托尼气疯了。”
\par 两个男孩不禁哈哈大笑。
\par “妈妈真是个人物!”布伦特赞赏地说,言语中流露出对母亲的敬爱之情。“你若希望她把事情做对,她就不会让你的希望落空,而且决不会让你在别人面前难堪。”
\par “不错,可今晚我们回家时,她却很可能会在爸爸和那些女孩子面前说出令我们难堪的话来,”斯图尔特闷闷不乐地说,“我说,布伦特,我想,这就意味着我们去不成欧洲了。你知道的,妈妈说过,如果我们再被大学开除的话,我们就不能去欧洲观光了。”
\par “让它见鬼去吧!我们才不在乎呢,对不对?欧洲有什么好看的?我敢打赌,那些外国佬根本拿不出一件我们佐治亚州没有的东西来。我敢说,他们的马绝不会比我们的跑得快,女孩子也不会比我们这儿的漂亮。我知道得很清楚,他们的黑麦威士忌酒绝对没有爸爸的够味。”
\par “卫希礼说,那里景色优美的地方很多,音乐也非常动听。希礼喜欢欧洲。他老谈论它呢。”
\par “咳——你知道卫家的人是怎么回事的。他们好像对音乐、书本和自然风光挺着迷的。妈妈说,这都是因为他们的祖父是从弗吉尼亚来的缘故。她说,弗吉尼亚人挺看重这些东西的。”
\par “让他们去迷这些东西好了。我嘛,只要有好马骑,有好酒喝,有好姑娘让我追,再有一个不起眼的姑娘供我取乐,这就行了。谁能够去欧洲游玩,我才不管呢……不能遍游欧洲,那又怎么样?假设我们现在在欧洲,那这里打起仗来怎么办?我们就不能马上赶回来了。我宁愿去打仗而不去欧洲。”
\par “我也是,不定哪天……哦,布伦特!我知道我们可以到哪儿吃饭了。我们骑马穿过沼泽地到埃布尔·温德那里去,告诉他我们兄弟四个都回来了,随时准备参加集训。”
\par “这主意不错!”布伦特兴奋地叫起来,“我们还能听到有关骑兵连的所有消息,知道他们最后决定用什么颜色的布料来做制服。”
\par “如果是那种华丽的服装,我是绝对不会去参加骑兵连的。穿着那种宽大的红裤子,我感觉自己像个傻瓜似的。它们看起来就像红法兰绒布做的女人内裤一样。”
\par “你们都打算去温德先生家吗?如果是,那晚饭你们就吃不舒服了,”吉姆斯说,“他们的厨子死了,又没有再买新的。他们叫了个干农活的黑奴做饭,那些黑鬼告诉俺,她是全州最糟糕的厨子了。”
\par “老天!他们干吗不另外买个厨子呢?”
\par “那些白人穷鬼能买几个黑鬼呢?他们拥有的黑奴最多不会超过四个呢。”
\par 吉姆斯的声音里明显带着瞧不起人的口气。塔尔顿家有一百个黑奴,所以吉姆斯的社会地位很稳固。像所有大种植园主拥有的黑奴一样,他也看不起只有少数几个黑奴的小农场主。
\par “就凭你这样,我就该剥了你的皮,”斯图尔特厉声喝道,“你不能把埃布尔·温德称为‘白人穷鬼’。当然,他并不富有,但他不是什么穷鬼;我绝不允许任何人,不管是黑人还是白人,说他的坏话。这县里没有比他更合适的人了,要不骑兵连怎么会选他当中尉呢?”
\par “俺也一直想不通呢,少爷。”吉姆斯回答着,并不因为主人生气而感到不安。“俺觉得他们应该从有钱的白人老爷中选长官,而不是从住在沼泽地的白人穷鬼中选。”
\par “他不是白人穷鬼!你是不是有意要把他和斯莱特里一家那样真正的白人穷鬼比较呢?埃布尔只是不富有而已。他是个小农场主,不是大种植园主。但是,如果所有小伙子都看重他,选他当中尉,那么,任何黑人都不能说他的坏话。骑兵连是知道它在做些什么的。”
\par 骑兵连是三个月前组建的,成立那天正好是佐治亚州退出联邦政府的同一天。从那时起,新兵们就一直在待命参战。骑兵连的名称还没定下来,虽然已有了不少提议。在这点上,每个人都有自己的看法,而且不愿意放弃,在制服的颜色和样式上也一样。“克莱顿野猫”“火焰食者”“北佐治亚轻骑”“义勇军”“内陆步枪队”(虽然骑兵连的武器装备只有手枪、马刀和长猎刀而没有步枪),还有“克莱顿灰衣连”“血光霹雳”“豪爽精英”等,每个名称都有一帮人拥护。在名称还没最后确定以前,大家只是把这一组织称为骑兵连,尽管最后采用了夸大其词的名称,他们一直都以与他们组建初衷有关的“骑兵连”而闻名。
\par 军官是由其成员选出来的,因为全县除了几个参加过墨西哥战争\footnote{1846年为购买新墨西哥地区而进行的美墨谈判破产,美国军队进入有争议的地区,并正式对墨西哥宣战。1848年美墨战争结束,美国以一笔补偿费从墨西哥获得了得克萨斯、新墨西哥、加利福尼亚、犹他、内华达、亚利桑那和科罗拉多及怀俄明的部分地区。}和森密诺尔战争\footnote{森密诺尔人为美国印第安人中得摩斯科格人的一部分,后从印第安人的政治组织——克瑞克联盟中退出,并迁离佐治亚州。}的老兵以外,再也没有别人有作战经验。再说,骑兵连也不屑于起用一个老兵来当头,除非他们个人特别喜欢他而且信任他。虽然大家都喜欢塔尔顿家的四个男孩以及方丹家的三个男孩,但是很遗憾,他们都不能选这些人,因为塔尔顿家的男孩动不动就喝醉,而且爱开玩笑。方丹家的呢,性情又太易怒,太暴躁。卫希礼被选为上尉,因为他是全县最出色的骑手,而且他头脑冷静,可以指望他来维持点军纪。雷福德·卡尔弗特被任命为第一中尉,因为大家都喜欢雷福。而沼泽地一位猎人的儿子、身为小农场主的埃布尔·温德则被选为第二中尉。
\par 埃布尔是个精明、严肃的大块头,他丁字不识,心肠却很好。他比其他男孩年纪更大,在太太小姐们面前,他的举止并不比其他男孩逊色,甚至还略胜一筹。骑兵连的人并不势利,他们中太多人的父辈和祖辈也都是从小农场主阶层发展而来的富户。再说,埃布尔还是骑兵连中最好的射手。他在七十五码远处还能射中松鼠的眼睛。除此以外,他对野外宿营知道得很多,雨天怎么生火、如何追踪猎物以及用何方法才能找到水源等等。骑兵连队员对他真的是心悦诚服,而且,还因为大家都喜欢他,所以就选他当了军官。他极为慎重地接受了这一殊荣,一点也不自高自大,就好像这是他的职责一样。可是,他并不是一生下来就是个绅士的,这一事实就算种植园主家的先生们能够忽略,太太小姐们和黑奴们却做不到。
\par 起初,骑兵连只招募种植园主的儿子,算是一支乡绅队伍。每人都得提供自己的坐骑、武器、装备、制服及贴身男仆。但在克莱顿这样开发历史不长的县里,有钱的种植园主并不多。为了组建一个战斗力强的骑兵连,有必要从小农场主、偏僻丛林的猎人、沼泽地的狩猎户、家境贫寒的山地白人中招募队员;个别情况下还招穷苦白人,只要他们的家境在他们那个阶层中处于中上水平就行了。
\par 如果战争来临,后面这些年轻人也跟他们富有的白人邻居一样急于跟北方佬干上一仗;可是钱这一微妙的问题便随之而来。很少有农人拥有马匹,他们农场里的农活是用骡子应付的,而且没有多余的骡子,至多不超过四头。即使骡子为骑兵连所接受,它们也腾不出时间去参战,更何况骑兵连根本不接受骡子。至于穷苦的白人,他们有一头骡子就觉得自己很富有了。偏僻丛林的猎户和沼泽地的狩猎人既没有马也没有骡子。他们完全靠地里的庄稼和沼泽地的猎获物过活,商业行为基本上是物物交换,一年里连五块钱现金都很少看到。马和制服根本就是他们可望而不可即的东西。可他们对自己的贫穷却傲气十足,就像种植园主对自己的财富感到无比自豪一样。他们的白人邻居略带慈善性质的捐助,他们从来都不接受。所以,为了顾及所有人的情绪,并且把骑兵连建成强有力的部队,郝思嘉的父亲、卫约翰、巴克·芒罗、吉姆·塔尔顿、休·卡尔弗特,事实上,全县除了安格斯·麦金托什以外,所有的大种植园主都出钱以便全面装备骑兵连,包括人和马匹。结果是,每个种植园主都同意出钱给自己的儿子以及一定数量的其他人买装备。一经这么处理,较不富有的骑兵连队员便可以坦然接受捐助的马匹和制服,自尊心又不会受到伤害。
\par 骑兵连在琼斯伯勒每两周集训一次,期盼着战争打起来。还没完全安排好弄到足够的马匹,但有马的人已经在县政府后面的空地上表演他们想像中的骑术动作。马蹄扬起了一大片尘土,他们虽然喊哑了嗓子却还在大喊大叫,手里挥舞着从起居室墙上取下来的革命战争时期用过的马刀。那些还没有马匹的人则坐在布拉德铺子前面的街沿石上,一边观看骑在马上的战友们表演,一边嚼着烟草谈天说地,或者干脆进行射击比赛。大多数南方人出生后就手不离枪的,狩猎生活更是使他们个个都成了神枪手。
\par 从种植园主的家及沼泽地的小木屋里,拿出了一堆堆各式各样的武器。它们是:第一批移民翻过阿勒根尼山脉时还是簇新的打松鼠用过的长杆枪、佐治亚州刚开发时曾经打过许多印第安人的前装枪、一八一二年桑密诺尔及墨西哥战争中服务过的马枪、决斗用的镶银手枪、袖珍大口径短筒小手枪、双管猎枪,以及亮闪闪的、上好木头制作的漂亮崭新的英式步枪。
\par 训练总是以在琼斯伯勒的沙龙聚会而告终。傍晚时分,斗殴事件频繁发生,军官们不得不加强警戒,以防在和北方佬交战以前造成人员伤亡。就是在一次这类吵架事件中,斯图尔特·塔尔顿用枪打伤了凯德·卡尔弗特,托尼·方丹则打伤了布伦特。那时兄弟俩刚从弗吉尼亚大学被开除回家,正好在组建骑兵连,他们便兴致勃勃地参加了。枪伤事件发生以后,就在两个月前,他们的母亲帮他们打点好行装,打发他们到州立大学去求学,责令他们待在那。因不在家错过了军训,他们感到很痛心。只要他们能和朋友们一起骑马、叫喊、用步枪射击,那么,即使失去了受教育的机会也是值得的。
\par “我们穿过乡野到埃布尔家去好了,”布伦特建议说,“我们可以从郝家的河床和方家的牧地穿过去,很快就可以到的。”
\par “除了负鼠和蔬菜,俺们不会有啥吃的呢。”吉姆斯争辩说。
\par “你不用有什么吃了,”斯图尔特咧嘴笑了,“因为你要回家去告诉妈妈,我们俩不回家吃饭了。”
\par “不,俺才不去呢!”吉姆斯惊恐地叫了起来。“不,俺不去!俺才不想为你们所做的事让比阿特丽斯小姐打我呢,这可不是好玩的。首先,她会问俺,俺是咋的让你们俩被开除的。其次,她会问俺,为啥今晚不把你们带回家去好让她揍你们一顿。然后她就会把火发到俺身上,就像鸭子扑在绿花金龟上一样。俺知道的头一件事就是,这啥事都要怪俺。如果你们不带俺到温德先生那去,那俺就一整夜躺在树林里,也许巡逻队会把俺抓去。可俺宁愿让巡逻队抓住也不愿在比阿特丽斯小姐生气时被她逮住。”
\par 兄弟俩茫然不解、怒气冲天地看着这个一脸倔强的黑人小伙子。
\par “这个傻瓜,竟然宁愿被巡逻队抓去,这又会给妈妈留下好几星期的话柄了。我敢发誓,黑人是越来越麻烦了。有时我都会想,废奴主义者的观点兴许是对的。”
\par “可让吉姆斯去面对我们自己不想面对的局势也是不对的。我们只好带他走了。可是,你给我听着,你这厚颜无耻的黑蠢货,你如果在温德先生家的黑奴面前端架子,或者提到我们家总是有炸鸡火腿什么的,而他们除了兔子和负鼠外什么也没有,我就——我就告诉妈妈。而且也不让你跟着我们去打仗了。”
\par “架子?俺会在那些便宜买来的黑鬼面前端架子?不呢,少爷,俺的举止比他们高明多了。在行为举止方面,比阿特丽斯小姐难道不是用教你们的同样的方式教俺的吗?”
\par “在我们任何一个人身上,她的教法都没达到目的。”斯图尔特说,“好啦,我们上路吧。”
\par 他让他那高大、赤红的马后退了几步,双腿一夹腿肚子,马儿便带着他轻松地越过围栏,进入郝家种植园松软的田地里。布伦特的马也越了过去,然后是吉姆斯的,他还紧紧贴着马鞍的前桥和马的鬃毛呢。吉姆斯不喜欢骑马跳越围栏,但为了跟上主人,比这更高的他都跳过。
\par 夜色越来越浓了,他们在垄沟里择道而行,顺着山坡向河床走去。斯图尔特对他兄弟叫道:
\par “哎,布伦特!你难道不觉得思嘉本来是要请我们吃饭的吗?”
\par “我也一直在想她本来是会这么做的,”斯图尔特也叫道,“你认为为什么……”


\subsubsection{第二章}

\par 兄弟俩离去时,思嘉仍站在塔拉的游廊上。等到飞驰而去的马蹄声渐渐消失之后,她才像个夜游的人一样回到椅子上坐下。内心的痛苦使她紧绷着脸,嘴巴也因强装微笑而感到不适,因为她不想让这孪生兄弟俩看透她心中的秘密。她疲惫不堪地坐下来,把一只脚放在椅子上、压在另一条腿下,内心涌起一阵阵悲苦。这悲苦愈演愈烈,直至她那颗心再也无法承受。她的心不时地在微微抽痛,双手发冷,一种即将被毁灭的感觉压迫着她,脸上便现出一副痛苦不已却又茫然无措的神情,就像一个娇生惯养的孩子,从来就是想要什么就有什么的,可现在,生活中第一次遇到了不顺心的事,于是就表现出这种茫茫然不知所措的神情来。
\par 希礼要和韩媚兰结婚!
\par 噢,这不可能是真的!兄弟俩一定是弄错了。他们又跟往常一样在跟她开玩笑吧。希礼不可能、绝不可能爱上她的。媚兰那小个子女人像耗子一样,谁也不可能爱上她。思嘉带着鄙夷想着媚兰单薄瘦弱、孩子气十足的身材以及正儿八经的心形脸孔,这副尊容普通极了,简直到了难看的地步。而且希礼应该也有好几个月没跟她见面了。自去年在十二棵橡树举办家庭晚会以来,希礼到亚特兰大去的次数总共不会超过两次。不,希礼不可能在爱着媚兰,因为——噢,她不可能搞错的!——因为他在爱着她!她,郝思嘉,才是他爱着的人——她知道这一点!
\par 思嘉听到嬷嬷笨重的脚步声传来,把过道的地板也震得直摇晃,她赶紧把压在腿下的那只脚放下来,重新调整脸部表情,使之显得更平静一些。让嬷嬷怀疑出了什么事,那是绝对不行的。嬷嬷总是认为,郝家的人从外表到内心全都属于她,他们的秘密也就是她的秘密;哪怕只有一丁点疑点也足以使她像猎犬一样紧追不放。从以往的经验,思嘉知道,如果嬷嬷的好奇心没有马上得到满足,她就会把事情捅到埃伦那,到时候思嘉就只好被迫向她妈妈供述一切,或是编造一个能自圆其说的谎言。
\par 嬷嬷从过道里出现了。她是个身材高大的老妇人,却和大象一样有双精明的小眼睛。她黑色的皮肤亮闪闪的,是个地地道道的非洲人。她为郝家尽心尽力,是埃伦的左右手,却是她三个女儿的眼中钉,也是屋里其他仆人眼里的母老虎。嬷嬷是个黑人,但她的行为准则和自尊心跟她的主人们相比并不逊色,甚至准则还更高、自尊心还更强。她是在埃伦的母亲索兰格·罗比亚尔的闺房里长大的,而埃伦的母亲是个举止优雅、冷静严肃、鼻子高挺的法国太太,不论是她的孩子还是家里的仆人,只要他们礼仪不周,就绝对逃脱不了公正的惩罚。嬷嬷原是埃伦的奶妈,埃伦出嫁后随她从萨凡纳来到内地。只要是嬷嬷所爱的人,她都要加以调教。由于她对思嘉的爱特别深,又为思嘉感到无比自豪,所以,她对思嘉的调教实际上从来就没有中断过。
\par “那两个先生回家去啦?你为啥没留他们吃晚饭呢,思嘉小姐?俺已经告诉波克给他们多摆两副刀叉了。你的礼貌都到哪儿去了?”
\par “哦,我太讨厌听他们谈论战争了。若晚饭期间他们还要继续谈论此事,特别是爸爸也会来凑热闹,大喊大叫什么林肯先生。那我怎么受得了。”
\par “虽然俺和埃伦小姐在你身上花了不少工夫,可你的礼仪并没比一个干农活的人好多少。你怎的没披披巾坐在这呢?夜风正当面吹过来!俺不是跟你说过,肩上没披东西,夜里的凉意会让你受凉发烧的。进屋去吧,思嘉小姐。”
\par 思嘉故意无动于衷地转过身去,不看嬷嬷。嬷嬷一心想着披巾的事,没注意到思嘉的脸,思嘉为此感到很庆幸。
\par “不,我想坐在这看夕阳。夕阳太美了。你去把我的披巾拿来吧。求求你了,嬷嬷,我要坐在这儿等爸爸回来。”
\par “你的声音听上去像是着凉了。”嬷嬷怀疑地说。
\par “哦,没这回事。”思嘉不耐烦地说,“你去帮我拿披巾吧。”
\par 嬷嬷一摇一摆地走进过道,思嘉耳边便响起她在楼梯口轻声呼唤楼上的女仆的声音。
\par “喂,罗莎!把思嘉小姐的披巾扔下来给俺。”之后,又更大声地叫道:“没良心的黑鬼!简直一点用也没有。看来俺得自己爬上去拿了。”
\par 思嘉听到楼梯一阵吱呀作响,便轻轻地站起身来。嬷嬷回来时又会对她待人接物方面的失礼唠叨个不停的,思嘉觉得,在她痛苦得几乎心碎欲裂的时候还有人为这种小事唠叨个没完,这于她是无法容忍的。她犹犹豫豫地站起来,心里想着该到哪里去躲避一下,以便让内心的痛苦得到一点缓解。恰在此时,她心头忽然掠过一个想法,心里不禁升起了一线希望。她父亲下午骑马到卫家的种植园——十二棵橡树去了。他是去提议购买迪尔西的。迪尔西是他的贴身男仆波克的妻子,可还属于其他主人。她是十二棵橡树的女仆总管和接生婆,六个月前两人结婚后,波克不论白天还是黑夜都在缠着他的主人,要他去买迪尔西,好让他们两人生活在同一个种植园里。郝嘉乐被他缠得实在没有办法,那天下午只好出门去办此事了。
\par 思嘉寻思着,爸爸一定会知道这个可怕的消息是真的还是假的。就算今天下午他实际上并没有听说什么,他也会注意到某些苗头,比如说觉察到卫家的喜悦之情呀什么的。只要晚饭前我能单独见到他,我就能知道事实真相——发现这只不过是那孪生兄弟俩一个令人讨厌的恶作剧罢了。
\par 该是嘉乐回来的时候了,而假如思嘉想单独见到他,她就只能到车道拐上马路的地方去接他。她轻轻地缓步走下房子前面的台阶,一边还小心翼翼地转过头往后看,以确保嬷嬷没有从楼上的窗户监视她。还好,从飘动的窗帘缝里,她没看到那张戴着雪白的头巾式帽子的宽大的黑脸庞带着不以为然的神情在窥视她,于是,她大胆地提起绿色的花裙子,顺着小路飞快地向车道跑去。她脚上穿着小巧、用缎带镶边的鞋子,这鞋能让她跑多快,她就尽量跑多快。
\par 砾石铺设的车道两边,墨黑的雪松枝条纵横交错,在上方形成了一个拱形,偌长的车道便变成了一条光线暗淡的隧道。一跑到雪松那长满节瘤的枝条下面,她就知道自己已经不用担心屋子那边有人会看见她了。于是,她放慢了脚步。此时的她已是气喘吁吁的,因为她的紧身胸衣束得太紧,她不能跑太远的路。但她还是尽可能快地往前走。很快她便来到车道尽头,拐上马路。但她并没有停下脚步,而是拐过一个弯,让一大片树林把她挡住,使自己和房子完全隔了开来。
\par 她满脸泛红,喘着粗气,在一个树桩上坐下来等她父亲。已经过了父亲该回家的时间了,但他今天推迟了反而使她很高兴。这一耽搁便让她有时间缓口气,让脸上的表情复归平静,这样她父亲就不会产生怀疑了。她时刻都在期待着听到他哒哒的马蹄声,看到他像平时那样危险地飞速冲上山坡急驰而来。可是,时间一分一秒地过去,嘉乐还是没有露面。她顺着路线寻视着她父亲的身影,与此同时,心里的痛苦又重新涌上心头。
\par “噢,这不可能是真的!”她心里想着,“他怎么还不回来呢?”
\par 她顺着弯弯曲曲的马路望去,早上下过雨后,马路上呈现一片猩红色。她的思绪已经沿着蜿蜒曲折的路径,飘下山坡,直至流速缓慢的弗林特河,再穿过杂草灌木盘根错节、土壤潮湿而松软的河床,飘上下一道山坡,来到希礼住的十二棵橡树。现在这一整条路径也就剩下这个含义了——这条路可通向希礼以及他那座房子,房子就像希腊神庙一样坐落在一座小山上,白色的柱子高高耸立着,漂亮极了。
\par “噢,希礼!希礼!”她心里想着,连心跳也加快了。
\par 自从塔尔顿家的男孩告诉了她无意中听来的消息后,一种令人寒心、茫然无措、大难临头的感觉一直压迫着她,而现在,这种感觉被抛到脑后去了,代之而起的是已经在她心里燃烧了两年的那股爱火。
\par 现在想起来还真觉得有点奇怪。在她的成长过程中,希礼对她似乎从来没有产生过什么吸引力。孩童时代,她看着他来来去去,但对他从来没有过什么想法。可是,两年前的一天,希礼刚从欧洲旅游观光回来后到她家礼节性拜访。自那天起,她便爱上了他。事情就这么简单。
\par 那天,他骑着马沿着长长的车道走过来时,她正好在前门的游廊上。他身着灰色的绒面呢上衣,系着黑色的领带,镶有饰边的衬衫被衬托得完美极了。即使现在,她也还能想起那天他服饰的每个细节,靴子闪闪发亮,领带夹有个浮雕宝石做成的希腊美女美杜莎的头像,还有他一看到她就脱下来拿在手里的巴拿马式帽子。他飞身下了马,把马缰扔给一个黑人小孩,站在那抬头对着她微笑,一双慵懒的灰眼睛睁得大大的。灿烂的阳光照在他淡黄色的头发上,好似给他戴上了一顶银白发亮的帽子。他开口说道:“哦,你都长大了,思嘉。”他轻步走上台阶,吻了吻她的手。哦,还有他那声音!她永远也无法忘记,听到他的声音时自己的心跳得有多快,就好像是第一次听到了这种不紧不慢、浑厚洪亮、悦耳动听的声音一样。
\par 就在那一刹那,她就很想要他,就像她要食物吃、要马儿骑、要一张柔软的床好让自己躺在上面一样,既简单明了,又不可理喻。
\par 两年来,他伴着她在全县四处活动,参加舞会、炸鱼野餐、郊游,还到法院去看审案。虽然不像塔尔顿家的孪生兄弟俩或是凯德·卡尔弗特那么频繁,也没有像方丹家年轻的男孩那样对她纠缠不清,但是,希礼没有哪个星期不来塔拉拜访的。
\par 诚然,他从未向她求过爱,那双清澈的灰眼睛也从来没有过那种思嘉在其他男人眼里司空见惯的热切的光芒。然而——然而——她知道他爱她。这一点,她绝不可能弄错的。知觉强于理性,况且,从经验获得的学识告诉她,他是爱她的。她经常会出其不意地发现,他的眼睛并没有露出无精打采或是远不可及的神色,而是带着一种令她费解的渴望和忧伤的神情看着她。他为什么不告诉她呢?她也不明白这一点。但在他身上,她不明白的事情还多着呢。
\par 他一直都很殷勤礼貌,但又深不可测,远不可及。没人知道他心里到底在想什么,思嘉就更不用说了。在这一带,人们总是想到什么就马上说出来的,所以,希礼这种含蓄的个性总是令人感到很恼怒。在县里平常的娱乐活动中,如打猎、赌博、跳舞和关心政治等等,他都不比别的年轻人逊色,还是他们中最出色的骑手;但是他和其他所有人都不一样,他并不把这些愉快的活动当做生活的终结和人生的目的。他爱好书本和音乐,喜欢写诗,在这些兴趣爱好方面,他是茕茕孑立、无人可及的。
\par 噢,他那一头金发为什么那么漂亮;他看似高高在上,为什么又那么殷勤有礼;他老爱谈论欧洲、书本、音乐、诗歌以及她一点也不感兴趣的东西,这令人烦得要死,她却又偏偏很想听,这又到底是为什么?无数个夜晚,当思嘉在房子前面半明半暗的游廊上和他闲坐之后,躺在床上总是辗转反侧,好几小时都无法入眠,只好用这一想法自我安慰:下一次他看到她时,他一定会开口求爱的。可是下一次来了又走了,结果还是什么也没有——什么也没有,只有她心里的那股爱火越燃越旺、愈烧愈热。
\par 她爱他,她要他,但她却不理解他。她性格直率、头脑简单,就像每天吹过塔拉的清风以及绕之流过的黄色小河一样纯朴自然,至死也无法把一件复杂的事情弄明白。可是现在,她生平第一次遇上了一个性格复杂、莫测高深的人了。
\par 希礼天生就不是那种把闲暇时间用来做事情的人,一旦有空,他就把时间用来思考问题。他会用这种时间来编织与现实世界没有任何关联的色彩斑斓的梦想。他会沉溺于一个比佐治亚州更加美妙的内心世界,极不情愿回到现实生活中来。他冷眼旁观着世间的生灵,既谈不上喜欢他们,也谈不上讨厌他们。他漠然观察着凡间生活,既说不上激动振奋,也说不上伤心失望。他按照这个世界原有的样子接受了这个世界以及他在其中所处的位置,而后耸耸肩,转而沉浸在他喜好的音乐、书本以及他那更美好的世界当中去。
\par 他的心灵世界对思嘉来说,那是完全陌生的,可他为什么偏偏就能俘获她的心呢?这她自己也不明白。他这个谜一般的人物激起了她的好奇心,就像一扇既没有门锁也没有钥匙的门一样。他身上她无法理解的东西却使她更加爱他,而他那奇特、有节制的求爱只是更加坚定了她要把他完全占为己有的决心。她从来就没有怀疑过,总有一天他会向她求爱的,这是因为她不但年轻气盛,家里人又对她溺爱有加,为此,她从来就没尝过失败的滋味。可现在却传来了这个可怕的消息,真像是晴天霹雳。希礼要跟媚兰结婚了!这绝不可能是真的!
\par 怎么说呢,就在上星期,他们俩在日暮时分一起从费尔希尔骑马回家,他曾对她说过:“思嘉,我有些重要的事情要告诉你,可我真不知道该怎么开口。”
\par 她拘谨地垂下眼睑,内心却是一阵狂喜,心想这一幸福的时刻终于来临了。可接着他又说:“现在不行!我们都快到家了,没时间说了。噢,思嘉,我真是个胆小鬼!”他用马刺驱了马一下,便跟她一起策马上了山坡回到塔拉。
\par 思嘉坐在树桩上,回想着这些曾使她感到无比幸福的话,突然间联想到另外一层意思,一层令人感到可怕的意思。他要告诉她的也许就是他即将要订婚的消息!
\par 噢,要是爸爸现在回家来该有多好啊!她一刻也忍受不了这种忧虑不安、吊在半空中的感觉了。她极不耐烦地再次朝路上望去,可光秃秃的路面还是再次使她的希望落空了。
\par 太阳已经落到地平线下了,天边那一抹红霞已经渐渐褪为粉色。头顶上的天空也慢慢地由原来的天蓝色变成了像知更鸟的蛋一样柔和的青绿色,乡间那种神秘、寂静的夜色便悄悄地降临了,把她笼罩在其中。整片乡野已是一派朦朦胧胧的景致。红色的垄沟以及开裂的路面已经看不出原有的带神秘色彩的猩红色,变成了普普通通的褐土。路对过的牧场里,马匹、骡子和牛群把头伸出围栏,安安静静地站在那,等着人们把它们赶回牲口棚里去进食。它们一点也不喜欢把牧场和小溪隔开的灌木丛那黑魆魆的影子,于是都对着思嘉抽动耳朵,似乎很感激这人的陪伴。
\par 在这种奇特的半明半暗之中,长在河边沼泽地里的高大的松树在昏暗的天空映衬下已是一片漆黑。尽管在阳光下它们是令人倍感温暖的绿油油的植物,现在却好似一堵由黑色巨人组成的无法穿越的人墙,把它们脚下那条黄色的小河流给藏匿得无影无踪。河对面的小山上,卫家那些高大的白色烟囱渐渐隐没在房子周围橡树丛的浓密阴影中,只有远处星星点点的晚餐灯光告诉人们那里有一座房子。春天温暖、潮湿的气息一阵阵袭来,带来了新犁过的土地微湿的气味以及所有新泛绿的植物散发到空气中的香味,她便全然置身于这一片温暖的气息当中了。
\par 对思嘉来说,日落、春天及新绿都不是什么奇迹。她漫不经心地接受了这些东西所蕴含的美,就像她平常呼吸空气及喝水一样。除了女人的脸蛋、马匹、丝绸服饰及看得见、摸得着的东西以外,她从来没有在别的事情上意识到美的存在。然而,此时此刻,塔拉种植园精心耕耘的田地上这种安详寂静、半明半暗的景致却给她忧虑不安的心灵带来了某种宁静。她深爱着这片土地,就像她爱她母亲在祈祷时灯光映照下的那张脸一样,可她甚至从来都没有意识到自己内心有这份爱。
\par 寂静、蜿蜒的路上还是没有嘉乐的身影。如果她再等下去,嬷嬷一定会来找她,把她硬拉回屋去的。正当她瞪大眼睛朝越来越暗的路面上望去时,她听见从牧场的小山脚下传来一阵马蹄声,接着看见马匹和牛群因受惊而四散开来。郝嘉乐回家来了,他正纵马穿过乡野飞驰而来。
\par 他骑着那匹膘肥体壮、马腿修长的猎马,正往山坡急驰而上,远远看去就像一个小男孩骑在一匹高大的马上一样。他那长长的白发被风吹到脑后,一边挥着鞭子,一边还大声吆喝着驱马前行。
\par 虽然她心里充满了焦虑与不安,但此时还是带着无比的自豪深情地望着她父亲,因为嘉乐是个出色的骑手。
\par “我真的弄不明白,为什么他喝了一点酒后就老爱纵马跳过围栏,”她心里寻思着,“即使去年在此处摔破了膝盖以后也还是不改。你总认为他该吸取教训的。更何况他还对妈妈发过誓,说再也不跳了。”
\par 思嘉一点也不怕她的父亲,甚至认为他还比她那些妹妹们更像她的同龄人。因为他经常跳越围栏,而且保守这个秘密不让他妻子知道,这给了他一种小男孩般的得意及做了坏事后得到的快乐。而这与她智斗嬷嬷得胜后的快乐如出一辙。她于是站起身来望着他。
\par 高大的马到了围栏边,略鼓鼓劲,便毫不费力地一越而过,就像鸟儿在空中掠过一样轻松,马背上的骑手也兴高采烈地大声叫喊着。他在空中挥舞着鞭子,白色的鬈发在脑后飘动。嘉乐并没看见在树影中的女儿,他在路上勒住马缰,满意地拍了拍马脖子。
\par “这县里没有哪匹马比得上你了,就是全州也没有。”他自豪地对他的坐骑说。虽然在美国已经待了三十九年,可是,他讲话时爱尔兰米斯郡的口音还很重。然后,他匆匆忙忙用手抚平头发,弄平皱巴巴的衬衣,整理好已经歪到耳朵后面的领带。思嘉知道,这些匆忙的整装都是为了有副绅士的仪容去面对他的妻子,让她认为,他拜访完邻居后是稳稳当当地骑马回家来的。思嘉还知道,这无疑给了她一个极好的机会上前跟他搭话,又不必暴露她的真正目的。
\par 她于是故意放声大笑起来。果然不出她所料,嘉乐被这笑声吓了一大跳;等到认出是她,红润的脸上便浮上一种局促不安的神情及充满挑战的意味。因为他的膝盖僵硬了,下马时颇为费劲。他让马缰滑到手臂上,脚步沉重地向她走去。
\par “哦,小宝贝,”他说着便在她脸上拧了一把,“这么说,你就像上星期你妹妹苏埃伦那样一直在监视我,而且要到你妈妈那去告发我,对吗?”
\par 他嘶哑、低沉的声音里带着点愤愤不平,但也有点连哄带骗的口吻。思嘉伸出手去把他的领带理好,同时开玩笑地啧啧舌头。他呼到她脸上的气息夹杂着波旁威士忌味和淡淡的薄荷香味,身上还发出嘴嚼烟草味、上了油的皮具味及马匹的气味——她一贯是把这些混合在一起的气味和她父亲联系在一起的,而且也本能地喜欢上别的男人身上的这些气味。
\par “不,爸爸,我才不像苏埃伦那样爱打小报告呢。”她向他保证着,退后一步用审慎的目光打量着他整理好的服饰。
\par 嘉乐个子不高,身高只有五英尺多一点,但是膀阔腰圆、脖颈粗壮,他坐着时,不知道的人还会认为他是个大块头呢。他体格健壮、双腿却又粗又短,总是穿着能买到的最好的皮靴,而且站着时总爱两腿分立,就像个狂妄自大的小男孩。大多数严肃认真、个子矮小的人都会显得有点可笑;可在场院里,矮小而好斗的公鸡总是受人尊重的,嘉乐的情形也一样。谁也不会莽撞地把郝嘉乐当成滑稽可笑的小个子。
\par 他已年届六十,满头鬈发已是一片银白。可他那张精明的脸上一条皱纹也没有,严厉、蓝色的小眼睛充满青春的活力,就像一个除了打扑克时要抓几张牌以外,从不费心去考虑比这更抽象的问题的年轻人一样,无忧无虑的。他的脸型极富爱尔兰人的特点,这种脸型在他很久以前就已离开的祖国到处可见——圆圆脸、面色红润、鼻子短小、嘴巴宽大,一副生性好斗的样子。
\par 郝嘉乐外表易怒暴躁,其实心地却是最好的。连黑奴受到训斥不高兴时,他也会看不下去,即使这黑奴是罪有应得也是如此。他还不忍听见小猫叫唤或是孩子啼哭;但他又很害怕自己的这些弱点会被别人发现。其实,不管是谁,遇见他五分钟之后就会发现他善良的心地,可他自己对这一点却一无所知;要是他知道这一点,他的虚荣心就一定会受不了,因为他喜欢认为,自己高声发号施令的时候,每个人都会胆战心惊、唯命是从。他从来就没有意识到,偌大的种植园里,只有一个声音是违背不得的——那就是他妻子埃伦柔和的声音。这是个他永远也无法知道的秘密,因为每个人——上至埃伦,下至最笨的干农活的黑奴都出于好意串通一气——让他相信他的话就是法律。
\par 思嘉对他的脾气和吼声比谁都更不会害怕。她是他最大的孩子。嘉乐知道,继那三个已躺在家庭墓地里的儿子之后,他已不可能再有别的儿子了,为此,他不知不觉养成了一种习惯,用非常坦率的态度对待她,而她竟也觉得,这使她快乐极了。她比她的妹妹们都更像她父亲,因为原名叫卡罗琳·艾琳的卡丽恩生性娇弱,成天想入非非,而教名为苏珊·埃利诺的苏埃伦却总爱为自己所谓的优雅举止和淑女风范自鸣得意。
\par 再说,思嘉和她父亲还各自遵守着一项无形中订立的秘密和约。如果嘉乐发现她懒得走半英里路从大门进去而图省事从围栏上爬过去,或是跟男朋友在屋前的台阶上待得太迟的话,他虽然会私下严厉地训斥她一番,但不会对埃伦或是嬷嬷提及此事。而一旦思嘉发现他在对妻子发过誓后还跳越围栏,或是知道他打牌时输掉了多少钱(她总是可以从别人的闲聊中知道这些),她也不会在吃晚饭时像苏埃伦那样傻乎乎地说出来。思嘉和父亲心照不宣,都认为把这些事说给埃伦听只会让她伤心,而他们是说什么也不会去伤害她那温柔的心肠的。
\par 思嘉在渐渐暗淡的微光中看着她的父亲,不知为什么,在他面前,她便觉得得到了某种安慰。他身上所具有的活力及朴实、粗鲁的气质深深吸引着她。她是个最不善于分析问题的人了,所以她并未意识到她自己在某种程度上就拥有同样的气质,尽管埃伦和嬷嬷十六年来一直在努力去除这些特点。
\par “你现在看上去倒是挺像样的,”她说,“我想,除非你自己吹牛皮,要不没有人会怀疑你又玩了你那些把戏的。但我确实觉得,自你去年在此跳越同样的围栏摔伤膝盖后——”
\par “得了,我才不要我自己的女儿来教训我什么该跳,什么不该跳呢。”他大声嚷嚷着,又在她脸上拧了一把。“反正是我自己的脖子,你管它呢。再说,我的小宝贝,你没围披巾跑到这来干什么?”
\par 看到他正用这种惯用的伎俩来逃避令人不快的谈话,她便悄悄地把一只手臂伸到他的臂弯里,说:“我在等你呢。我不知道你会这么迟回来。我正在想,买迪尔西的买卖有没有做成。”
\par “买是买成了,可那价格简直要让我倾家荡产。我买下了她和她的小女孩普里西。卫约翰几乎想白送给我们,可我郝嘉乐做买卖从来不用交情来占便宜,买她们俩,我硬是让他收下三千块钱。”
\par “我的天哪,爸爸,三千块哪!再说,你也没必要买普里西的!”
\par “哦,难道轮到我的女儿来对我评头论足了?”嘉乐大声辩解道,“普里西是个漂亮的小女孩,所以——”
\par “我知道她的。她是个又淘气又愚笨的小黑鬼。”思嘉平静地说,并未受他高声嚷嚷的影响。“你买下她的唯一的原因是迪尔西求你买下她。”
\par 嘉乐看上去垂头丧气的,非常尴尬,每当别人发现他做了软心肠的事时,他总是如此。思嘉看到他轻易就被别人识破真相,不禁哈哈大笑起来。
\par “情况的确如此,那又怎么样呢?如果迪尔西老是惦记着孩子,那买了她又有什么用?哦,我决不会再让一个黑奴和别处的女人结婚了,这代价太高啦。请吧,我们进去吃饭吧。”
\par 夜色越来越浓了,空中最后一抹淡绿也已退去,一股微微的凉意代替了春天的暖意。可思嘉磨蹭着,不知怎样挑起希礼这个话题又不让嘉乐怀疑她的动机。这并非易事,因为思嘉骨子里就没有思维敏锐的特质;而嘉乐这方面跟她极为相像,他从来就能看穿她那些苍白无力的托词,就像她能看穿他的一样。而且,在揭穿别人的托词方面,他极少时候能够做得圆滑得体。
\par “十二棵橡树那边的人全都好吧?”
\par “还好。凯德·卡尔弗特也在那。谈妥了迪尔西的事后,我们大家便在游廊上坐下来喝棕榈酒。凯德刚从亚特兰大回来,他们那都在谈论战争,简直闹翻天了。而且——”
\par 思嘉叹了口气。一旦嘉乐谈起战争和脱盟的话题,他就一定会一连谈好几个小时也不歇嘴的。她赶紧用别的话把话题岔开。
\par “他们有没有谈起明天的野餐会呢?”
\par “我想,他们谈起过的。哦——她叫什么来着——去年也在那里的那个可爱的小东西,你知道她的,就是希礼的表妹——噢,对了,叫韩媚兰,就叫这个名字——她和她哥哥查理已经从亚特兰大到这来了,而且——”
\par “噢,这么说她真的来啦?”
\par “是来了,她是个可爱文静的姑娘,从来不标榜自己,很守妇道的。走吧,我的女儿,别拖拖拉拉的。你妈妈会找我们的。”
\par 听到这个消息,思嘉的心直往下沉。她曾一再希望住在亚特兰大的韩媚兰会被什么事给耽搁住。她那可爱、文静的性情跟自己的截然不同,可连自己的父亲都在称赞她,这逼得她只好把话说白了。
\par “希礼也在家吗?”
\par “在的。”嘉乐放开女儿的手臂,转过身用锐利的目光看着她的脸。“如果你到这来等我就为了这个,你干吗不直说而绕这么大的圈子呢?”
\par 思嘉想不出来该说些什么,她感到自己的脸因不安而刷地变红了。
\par “哦,说吧。”
\par 她还是什么也没说,真恨不得能摇着父亲撒娇,让他闭嘴。可这又是不允许的。
\par “他在家,还非常友好地问你是否安好。他的妹妹们也一样,他们说,希望明天不会有什么事阻住你,令你参加不了野餐会。我能保证不会有什么事的。”他机灵地说着,“告诉我,女儿,你和希礼之间到底是怎么回事?”
\par “没什么。”她简短地回答着,拉了拉他的胳膊,“我们进去吧,爸爸。”
\par “这下是你催我要进去了,”他说,“可我打算站在这,直到把你的事弄明白再说。我看近来你有点奇怪,他没玩弄你吧?他有没有向你求婚呢?”
\par “没有。”她简短地回答着。
\par “他也不会的。”嘉乐说。
\par 她不禁怒火中烧,但嘉乐挥挥手,让她安静。
\par “别说了,小姐!今天下午我从卫约翰那听到了绝密消息,希礼要和韩媚兰结婚了。明天就要宣布。”
\par 思嘉的手从他的胳膊上滑落了下来。这么说,这是真的了!
\par 一阵痛苦袭上心头,她的心似被一只野兽的尖牙利齿无情地撕咬着一样难受。这期间,她感觉到父亲正用充满怜爱、焦虑不安的目光注视着她,因为他现在正面临着一个他根本无法回答的问题。他爱思嘉,但她老是问他一些孩子气的问题,逼他说出答案,这却使他非常不舒服。埃伦什么答案都知道,思嘉应该把碰到的麻烦向她诉说才是。
\par “你这不是在让自己出丑——也让我们大家出洋相吗?”他大声叫起来,连音调也提高了。他激动时就免不了会这样。“你难道一直在追一个并不爱你的人吗?县里哪个男孩子你不能嫁?”
\par 思嘉心里非常气愤,自尊心又受到了伤害,这多少抵消了一些痛苦。
\par “我没有追他。这——这只是使我感到奇怪罢了。”
\par “你在说谎!”嘉乐说,之后,他凝视着她那张受到打击的脸,声音里又掺进了无限慈爱:“对不起,我的女儿。可你毕竟还是个孩子,再说,好的男孩多得是。”
\par “妈妈跟你结婚时才十五岁呢,我已经十六了。”思嘉说着,连声音也哽咽了。
\par “你妈妈的情况不一样,”嘉乐说,“她可不像你一会风一会雨的。来吧,我的女儿,振作起来,下星期我带你到查尔斯顿去看你的尤拉莉姨妈,去听听他们那有关萨姆特堡的高谈阔论,一星期后你就会把希礼忘得一干二净的。”
\par “他总把我当小孩看,”思嘉心里想着,痛苦和愤怒使她连话都说不出来了,“好像只要他拿个新的玩具在我面前晃来晃去,我就会把摔肿的伤痛忘掉一样。”
\par “别对我噘着嘴了,”嘉乐警告道,“假如你明理一些,你早该嫁给斯图尔特·塔尔顿或是布伦特·塔尔顿了。好好想想吧。和双胞胎中的任何一个结婚,我们两个种植园就能连在一起了。吉姆·塔尔顿和我会给你们盖一座漂亮的房子,就在那片松树林里,两个种植园相连的地方——”
\par “你不要再把我当小孩看了行不行!我不想去查尔斯顿,也不要什么房子,更不想和孪生兄弟中的任何一个结婚。我只要——”她虽然打住了,可已经太迟了。
\par 嘉乐的声音平静得出奇,他说得很慢,就像从一个极少使用的词库里挑着词用一样。
\par “你要的只有希礼,可你不会得到他了。即使他有想和你结婚的意思,凭着我和卫约翰之间的交情,我虽然会同意,可也还会担着一份心。”看到她一脸的惊愕不解,他又接着说:“我要让我的女儿幸福,可你和他在一起不会幸福的。”
\par “噢,我会的!我会的!”
\par “你不会的,我的女儿。只有性格相近的人结为伉俪才会幸福。”
\par 思嘉心头突然掠过一个危险的念头,她很想大声喊出来:“可你不是一直都很幸福吗,但你和妈妈的性格并不相近啊。”但她忍住了,担心自己的鲁莽会招来父亲的耳光。
\par “我们家的人和卫家的人是不一样的,”他斟酌着词句慢慢地接着说下去,“卫家的人和我们的邻居也都不一样——跟我所知道的所有家庭都不一样。他们都是些奇怪的人,所以他们老和他们的表亲结亲,把这种怪异行为局限在他们家族内部,那是再好不过的了。”
\par “可是,爸爸,希礼一点也不——”
\par “你别急嘛,小姑娘!我不是说他不好,因为我也喜欢他。我说怪异,意思并不是说他们疯疯癫癫的。他这种古怪跟其他人不一样,既不像卡尔弗特家的人那样为了一匹马可以把全部家当都赌掉,也不像塔尔顿家的人那样一喝酒就醉得一塌糊涂,更不像方丹家的人,都是些头脑发热的小畜生,想到别人怠慢他们就会要人家的命。这种古怪行为当然是很容易理解的,要不是上帝仁慈,郝嘉乐也会有这些毛病的!我也不是说你成了希礼的妻子以后,他会和别的女人私奔,或是会对你施以暴力。他若果真如此的话,你也许还会更幸福,因为至少你就能逐步理解他了。但是他的怪异是在其他方面,是根本无法理解的。我是喜欢他,可对他说的话,十句有八句我都摸不着头脑。好了,小姑娘,跟我说实话,他对书本、诗歌、音乐、油画以及诸如此类荒唐可笑的东西如此狂热,对此你能理解吗?”
\par “噢,爸爸,”思嘉不耐烦地叫起来,“如果我跟他结了婚,我会改变这一切的!”
\par “噢,你会,你现在行吗?”嘉乐很恼火,严厉地看了她一眼,“你对男人的生活了解得太少了,更不用说希礼了。没有哪个妻子能改变丈夫的,哪怕是一丁点也不行,你可别忘了这一点。至于改变一个卫家的人——那简直是痴心妄想,我的女儿!他们全家都是那样的,从来就是如此。而且很可能永远都会如此。我告诉你,他们天生就是怪人。你瞧瞧他们那个样子,一会奔到纽约,一会又跑到波士顿,就为了去听歌剧,去看油画。还从北方佬那里成箱成箱地订购法国书和德国书!他们成天坐在那读书、做梦,谁知道他们在搞什么名堂,他们就不能跟其他规规矩矩的人一样,把时间花在打猎和玩扑克牌上吗?这样岂不是更好。”
\par “县里可再也没有哪个人骑马骑得比希礼更好的了,”思嘉说,为这种诋毁希礼太女人气的话感到很愤怒,“或许,除了他父亲,再没有别人了。说到玩牌,上星期在琼斯伯勒,你不是还输给希礼两百美元?”
\par “卡尔弗特家的男孩又在瞎说了,”嘉乐不置可否地说,“要不你不会知道这个数目的。希礼可以跟最好的骑手赛马,也能和一流的扑克玩家玩牌——那也就是我了,小姑娘!我也并不否认,真喝起酒来,他甚至能把塔尔顿家的灌倒在桌子底下。这些事他通通都会,可他并没把心放在上面。我为什么会说他怪呢,原因就在这。”
\par 思嘉不吱声了,心却在往下沉。对父亲最后说的这一点,她根本想不出什么理由来反驳,因为她知道嘉乐是对的。这些寻欢作乐的事情,希礼都做得很出色,可他的心却根本不放在这些事情上。对这些别人都特别感兴趣的事,他从来都只是出于礼貌才装出点兴趣来。
\par 嘉乐即刻看透了她沉默的原因,他拍拍她的胳膊,得意地说:“好了,思嘉!你也承认我说的这点是对的吧。若嫁了个像希礼这样的丈夫,你又能做些什么呢?他们全都是神经错乱的人,卫家所有的人都一样。”然后,他又哄着她说:“刚才我提到塔尔顿家的人,我并不是在推销他们。他们都是挺不错的小伙子,但是你如果对凯德·卡尔弗特有意的话,这于我并没有什么不一样。卡尔弗特家的也都是好人,全家人都是,尽管老头儿娶了个北方佬。在我离开这个世界以后——你别说话,亲爱的,先听我说!我会把塔拉留给你和凯德——”
\par “你要把凯德放在银盘上送给我,我才不要呢。”思嘉愤怒地大叫起来,“我希望你不要再把他推销给我了!我才不要塔拉或是什么老旧的种植园呢。种植园有什么大不了的,特别是在——”
\par 她正想说“在你得不到你想要的男人之后”,嘉乐却早被她对自己提供的礼物如此轻慢给激怒了,在这世界上,除了埃伦以外,种植园就是他的最爱。他不禁大吼起来。
\par “郝思嘉,你站在那就是要告诉我塔拉——那片土地——没什么大不了的吗?”
\par 思嘉固执地点点头。她太伤心了,根本顾不上她会不会惹他爸爸生气。
\par “土地是这世界上唯一了不起的东西,”他大声叫喊着,短而粗壮的胳膊奋力挥舞着,显得愤怒极了,“它是这世间唯一永恒的东西,这点你千万别忘了!它是唯一值得为之工作、为之奋斗——为之献身的东西。”
\par “噢,爸爸,”思嘉厌恶地说,“你就像个爱尔兰人一样在说教!”
\par “难道我曾为此感到不光彩过吗?不,我为此感到很骄傲。你可别忘了,你也是半个爱尔兰人,小姐!对每个哪怕只有一丁点爱尔兰血统的人来说,他们赖以生存的土地就像他们的母亲一样。此时此刻,我倒是为你感到羞耻。我要把世界上最美的土地送给你——除了老家的米斯县,就数它漂亮了——可你都做了些什么?你竟然对它嗤之以鼻!”
\par 嘉乐大喊大叫着发泄怒气,正说得来劲,这时,思嘉愁眉不展的脸上那种悲苦的神情使他停了下来。
\par “当然,你还年轻。但是你慢慢会爱上土地的。如果你是爱尔兰人,你就无法摆脱这种爱。你还只是个孩子,只会为你那些男朋友而烦恼。等你更大一些,你就会明白这……好了,你能不能打定主意跟凯德或是塔尔顿家那两个孪生兄弟,抑或是埃文·芒罗家的少爷呢,瞧我怎样把你风风光光地嫁出去!”
\par “噢,爸爸!”
\par 到了这时候,嘉乐对这谈话已经完全感到厌烦了,而且这个问题居然落到他肩上,他也为此极端地烦恼。再说,他把县里最出色的男孩都提出来了,还要把塔拉送给思嘉,可她看上去还是悲悲戚戚的,他为此感到很愤愤不平。嘉乐喜欢别人拍着双手、用亲吻来接受他的礼物。
\par “好了,别再噘着嘴了,小姐。你跟谁结婚,这并不重要,只要他跟你情投意合,是个上等人,又是南方人,而且又体面,这就行了。女人都是先结婚然后才有爱情的。”
\par “噢,爸爸,那是爱尔兰的老观念了!”
\par “可这是个相当不错的观念!你瞧瞧这里的人,尽在忙乎什么为爱而结婚这类美国的玩意儿,就像那些下人和北方佬一样!最美满的婚姻就是那些父母做主为女儿选择的婚姻了。因为像你这样的傻孩子怎么能够把好人和坏蛋区分开来呢?你看看卫家的人,到底是什么使他们能够几代相传、赫赫扬扬呢?不就是因为他们总跟他们的同类人结婚,老跟他们家一向相中的表亲通婚吗?”
\par “噢。”思嘉叫出声来,嘉乐的话使她认识到,这一可怕的事实是在所难免的了。痛苦又重新袭上她的心头。嘉乐看她低着头难过的样子,不安地把脚在地上蹭来蹭去。
\par “你不会是在哭吧?”他笨拙地摸着她的下巴,想把她的脸扬起来,自己也愁眉紧锁,满脸充满怜爱。
\par “没有!”她愤愤然地叫起来,把脸扭向一边。
\par “你这是在说谎,可我为此感到很自豪。我很高兴,你身上还有股傲气,小姑娘。明天的野餐会上,我也想看到你这股傲气。我可不想让全县的人都议论你、嘲笑你,说你钟情于一个除了友情对你别无他想的男人。”
\par “他当然是对我有所想的。”思嘉心里想着,内心痛苦极了。“噢,他对我所想可多了!我知道他确实对我有意。这我感觉得到。如果我再有一点点时间,我知道我就可能使他对我说——噢,假如卫家的人不是老觉得他们必须跟他们的表亲结婚,那该多好!”
\par 嘉乐拉起她的胳膊,挽在自己的手臂上。
\par “现在我们要进去吃晚饭了,这些事就只有你知我知。我不会把这些告诉你妈妈,让她担忧的——你也不会这么做的。我的女儿,把鼻子揩一揩。”
\par 思嘉用她那块破手帕揩了揩鼻子,他们手挽着手迈步向昏暗的车道走去,马在后面慢慢地跟着。快进家门时,思嘉正想开口说话,忽然看见她妈妈站在游廊上的阴影中。她戴着帽子,围着披巾,还戴着露指长手套。嬷嬷站在她后面,阴沉着脸,就像马上要下雷雨一样。她手里拿着一个黑色皮袋,那是郝埃伦用来放置救护黑奴时用的绷带和药品的。嬷嬷的嘴唇又厚又大,往下垂着。她生气的时候,下唇就可以拉得两倍长。而现在下唇就被拉长了,思嘉知道,嬷嬷又碰上什么不顺心的事,心里正窝着火呢。
\par “郝先生。”看到他们俩从车道上走过来,埃伦叫了起来——埃伦属于非常正统的那代人,即使在结婚十七年、生了六个孩子之后也还是一样——“郝先生,斯莱特里家有人病了,艾米产下了一个婴儿,可小孩却快咽气了,必须给他受洗。我和嬷嬷正要到那去,看看能帮什么忙。”
\par 她提高了自己的声调,似乎是在征求意见,等着嘉乐同意她去实施自己的计划似的。这纯粹是客套,却让嘉乐心里很受用。
\par “我的上帝!”嘉乐怒气冲冲地说,“那些白人穷鬼干嘛偏偏在吃晚饭的时候就把你叫走,我还要告诉你亚特兰大那里发生的有关战争的高论呢。去吧,郝太太。如果外面出了什么麻烦,而你又没有在场帮忙的话,晚上你躺在枕头上也会睡不安稳的。”
\par “夜里她老是东奔西跑地去照顾那些自己也可以照顾自己的黑鬼和白人穷鬼,她从来就没有睡安稳过。”嬷嬷用一种单调的声音嘟哝着,一边走下台阶,朝等在边道上的马车走去。
\par “吃饭时替我照看一下吧,亲爱的。”埃伦说,用戴着连指手套的手轻轻拍了拍思嘉的面颊。
\par 虽然思嘉在拼命抑制着眼泪,但是她妈妈这种从来就带着某种魔力的触摸,以及她那沙沙作响的丝绸衣裙上装着马鞭草的小香袋里散发出来的淡淡的薄荷香味,还是使思嘉激动不已。对思嘉来说,郝埃伦身上有一种使人激动、令人讶异的东西,和她住在同一个屋檐下,既让思嘉对她感到敬畏,又为她的魅力所倾倒,并且还让她的心灵得到些许安慰。
\par 嘉乐帮助妻子上了马车,嘱咐车夫驾车小心点。已经照管了嘉乐的马匹达二十年之久的托比嘟着嘴生着闷气,自己的本行活儿还要别人对他指手画脚告诉他该怎么做,他心里不受用呢。马车上路了,嬷嬷坐在托比旁边,两人都是一副非洲人遇到不顺心的事时生着闷气的嘴脸。
\par “如果我没有帮斯莱特里这家穷鬼这么多忙,他们就得在其他地方花钱,”嘉乐怒气冲冲地说,“他们也许就会愿意把他们那几顷贫瘠的河滩地卖给我,然后只好搬离这个县了。”可接着他又变得兴高采烈的,满心期待着来个他驾轻就熟的恶作剧:“来吧,我的女儿,我们去告诉波克,我没有把迪尔西买回来,反而把他卖给卫家了。”
\par 他把马缰扔给站在附近的一个黑人小孩,然后沿着台阶拾级而上。他早把思嘉那颗悲痛欲碎的心抛到九霄云外去了,一心就想着要去折磨他的贴身仆人。思嘉跟在他后面,慢慢走上台阶,两脚却像灌了铅一样步履维艰。她寻思着,其实她和希礼的结合未必就会比她父亲和郝埃伦的结合更别扭。她父亲总是大叫大嚷,而且一点也不敏感,怎么就偏偏和像她母亲那样的女人结婚,对此她总是百思不得其解。因为不论在出身、教养还是性格方面,绝对没有比他们两人更截然不同的了。


\subsubsection{第三章}

\par 郝埃伦虽然只有三十二岁,可用她那个年代的标准来衡量的话,她已经是个中年妇女,一个生过六个孩子却已安葬过其中三个的母亲了。她身材高挑,站着比她那脾气火暴的小个子丈夫足足高出一个头。但她总穿着带裙环的飘曳长裙,走起路来又是那么轻巧、优雅,所以她的高个头并不特别显眼。她穿着黑色的塔夫绸紧身上衣,上方露出的脖颈皮肤呈米色,既圆润又颀长。她的头发很多,挽在脑后罩在一个头发网里。脖子似乎也因头发的影响而微微地往后仰。她母亲是法国人,外祖父母是在一七九一年的革命中逃离海地的\footnote{1791年,原为法国殖民地的海地自由有色人首先发动武装起义,赢得了大批黑人奴隶的拥护。在黑人领袖杜桑·卢维杜尔的领导下,起义军赶走了法国、西班牙和英国的殖民者,于1801年宣布独立。}。从母亲那里,她继承了向上斜行的黑眼睛、墨黑的睫毛及乌黑的头发;她父亲曾是拿破仑手下的一名士兵,她那又长又直的鼻子和棱角分明的方形下巴就是从她父亲身上遗传来的。但她脸颊的线条非常柔和,这使她下巴的棱角显得不会那么生硬。埃伦脸上还有一股傲气,但她并不会目中无人。此外,她还有宽厚仁慈、庄重忧郁及不苟言笑等特点,这一切却都是从生活中获得的了。
\par 要是她的眼里再有一些光彩,微笑时带有相应的热情,或是自自然然地发出轻柔、动听的声音,让它萦绕在家人和仆人耳边,那她就是个绝色美人了。她讲话带有佐治亚州沿海人的特点,轻柔但有点模糊不清,元音发声流畅,辅音发音也很亲切,只有一点点法国口音。她吩咐仆人做事或训斥孩子时,从来不提高嗓门,但在塔拉,她的话总是马上就会被服从,而大家对她丈夫的咆哮、吼叫却老是默不作声地不予理睬。
\par 从思嘉能记事时起,她母亲就一直是这个样子。不论是赞扬人或是训斥人,她的声音总是既温柔又悦耳。尽管嘉乐那乱糟糟的家里每天都有这样那样的急事,可她处理起事情来总是有条不紊,效率很高。她总是头脑冷静,背从来就没弯过,甚至在她三个儿子还在襁褓中就夭折时也是如此。思嘉从来没见过她母亲坐着时靠在椅背上,也从未见过她坐下来的时候手里没拿着针线活,只有吃饭或照顾病人的时候,或者为种植园理账的时候才例外。有客人的时候,她手里忙活的是精美的刺绣,没客人的时候,则是嘉乐皱巴巴的衬衫、女儿的衣裙或是给黑奴做的衣服。她妈妈的手指上总是套着顶针,衣裙响过之处,总见她身边跟着一个黑人小女孩,小女孩这辈子唯一的职责就是拆掉疏缝针脚,拿着青龙木做成的针线盒从一个房间走到另一个房间。埃伦要在屋子里走来走去,指挥仆人烹饪、打扫房屋以及为种植园所有的人缝制衣服,只要她走到哪里,小女孩就跟到哪里。
\par 她妈妈总是那么稳重、平静,思嘉从未见过她这种心境被扰乱过。不管是在白天还是黑夜,她全身上下总是装扮得整整齐齐的。埃伦着装去参加舞会,或是会客,抑或是到琼斯伯勒去听审案的时候,常常要两个女仆和嬷嬷花两个小时才能把她打扮得合自己的意。可在情况紧急的时候,她打扮的速度之快也是令人暗暗称奇的。
\par 思嘉的卧室就在过道对过,她妈妈的房间对面。从婴儿时期起,思嘉对这类声音就极为熟悉:凌晨时分黑人光着脚轻声在硬木地板上匆匆走过,在妈妈的房门上急促地敲几下,然后传来了惊恐万分的黑人压低嗓子说话的耳语声——他们总是在禀报那一长排白色的小屋里谁又生病啦、某人又生下孩子啦、谁又撒手人寰啦等等。小时候,她经常蹑手蹑脚溜到门边,从最小的门缝里往外偷看。她会看见埃伦从那黑魆魆的房间里出来,黑人举着一根蜡烛,埃伦便出现在闪烁不定的烛光中,而嘉乐却还在节奏分明地鼾声大作,一点也没有受到惊扰。埃伦腋下夹着药箱,头发整洁地梳成惯有的发式,紧身上衣的扣子也扣得整整齐齐。
\par 埃伦轻手轻脚走过过道时,总是语气坚决又充满同情地低声说道:“嘘,别这么大声。你会吵醒郝先生的。他们的病并不重,一时半刻不会死的。”每当听到她妈妈这样的低语声,思嘉心里便受到莫大的抚慰。
\par 然后她再小心翼翼地回到床上,知道埃伦晚上不在家而一切又还是那么井然有序,这种感觉好极了。
\par 有时候,老方丹医生和小方丹医生都出诊去了,没法找到他们来帮忙。在一整夜照顾了刚生下孩子的产妇和婴儿或是料理后事之后,到了早晨,埃伦还是像往常一样坐在餐桌的主人席上照料一切。虽然她那黑色的眼睛周围有了一圈倦容,但声音和举止一点也不会露出劳累过度的样子。她那高贵、温柔的外表下有种钢铁般的意志,而正是这种意志使全屋子的人感到敬畏。嘉乐和女儿们一样也不例外,虽然他是宁死也不承认这一点的。
\par 有的晚上,思嘉会蹑手蹑脚地走到妈妈身边,去亲吻她那高个子妈妈的脸蛋。她端详着妈妈的嘴巴,那稍稍嫌短的上唇柔嫩极了,这么一张嘴是极易受到外界的伤害的。她真不知道妈妈是否曾经有过女孩子那样的咯咯傻笑,或是对要好的女朋友通宵达旦地低声倾诉心中的秘密。哦,不,这是不可能的。妈妈一直就是这个样子,是力量的支柱、智慧的源泉。不管是什么问题的答案,她都是无所不知的。
\par 可思嘉在这点上错了。多年以前,在景色迷人的滨海城市萨凡纳,埃伦也像任何一个年仅十五岁的少女一样莫名其妙地发笑,和朋友彻夜长谈,低声说着知心话,向好友倾吐所有的秘密。可是,有一个秘密她是缄口不言的。那就是比她大二十八岁的郝嘉乐闯入她生活的那一年——也就是她那年轻潇洒、眼珠乌黑的表哥菲利普·罗比亚尔从她的生活中消失的同一年。菲利普长着一双会勾人的眼睛,行为方式放荡豪爽。自他永远离开了萨凡纳以后,他也把埃伦心中所有的激情给带走了。而当罗圈腿的小个子爱尔兰人跟她结婚时,她留给他的就只剩下一副温柔的躯壳了。
\par 但对嘉乐来说,这已经足够了。他实实在在地成了她的丈夫,这种幸运简直令人不可思议,更令他激动不已。若说她身上什么东西没有了,他也从未觉察到。他是个精明的人,他知道,像他这样一无门第二无钱财的爱尔兰人,能够娶上沿海最富有、最显赫的家族之一的千金为妻,这本身就已经是个奇迹。因为嘉乐全是靠白手起家的。
\par  
\par 嘉乐是二十一岁那年从爱尔兰来到美国的。和许多境况比他好或是比他差、比他先来或是比他后到的爱尔兰人一样,他是匆促起程的。他背上的行囊里只有几件换洗衣服,付过船费后,身上也就剩下两个先令。他还是个被悬赏捉拿的要犯,而他认为他所犯的罪根本就不值这个价。在地球这边的地狱里,可没有什么对英国政府或是对魔鬼本人来说值一百英镑的奥兰治党人\footnote{奥兰治党人:1795年成立于北爱尔兰的拥护新教及英国王权的秘密社团成员。}。但是,假如政府对死了一个为英国在外的地主代收租金的人那么在乎的话,那也就是郝嘉乐该离家远行而且必须是突然离开的时候了。千真万确,他曾骂那个租金代收人是“奥兰治党人的狗杂种”,但据嘉乐看来,那人也并不因此而有权利用口哨吹出《博恩河\footnote{博恩河:爱尔兰米斯郡东北部河流,1690年英王威廉三世在博恩河战役中击败苏格兰国王詹姆斯二世。}水》这首曲子的开头几小节来侮辱他。
\par 博恩战役是早在一百多年前就已发生过的事,可对郝家和他们的邻居来说,就好像发生在昨天一样。由于惊恐万状的斯图亚特\footnote{斯图亚特王朝:斯图亚特家族在苏格兰(1370年起)和英格兰(1603—1649, 1660—1625)建立的封建王朝。1649年被英国资产阶级革命推翻。1660年复辟。}王子仓皇出逃,他们的希望也变成了失望,梦想也化为泡影,随之同去的还有他们的土地和财富。只剩下奥兰治的威廉及其戴着橘黄色帽章的令人憎恶的军队大肆砍杀爱尔兰斯图亚特王朝的追随者的人头。
\par 就因为这及其他一些原因,这次吵架只是被控应负责严重的后果而已,嘉乐的家人并没有把他这次吵架的不幸后果看得特别严重。多年来,在英国军事警察眼里,郝家的名声一直不好,因为涉嫌在进行反政府的秘密活动。嘉乐并不是郝家第一个半夜三更起程离开爱尔兰的人。他的两个哥哥——詹姆斯和安德鲁,他对他们已经没有什么印象了,只记得他们都是沉默寡言的人,老是在夜里颇不寻常的时刻来来往往,秘密执行任务,有时还会一连好几个星期不见踪影,让他们的母亲为他们担忧不已。好几年前,郝家的猪圈里埋藏着步枪,这个小小的军火库被发现之后,他们便到了美国。现在,他们已是萨凡纳成功的商人。一提到她最年长的两个儿子,他母亲就会插话:“只有亲爱的上帝才知道那可能在哪里。”年轻的嘉乐就是被派去投奔他们的。
\par 离别时,他母亲匆匆吻了吻他的面颊,在他耳边热切地说些天主教徒的祝福之词,他父亲则温和地告诫他:“记住你是谁,千万不要学人家的样。”他五个身材高大的哥哥也都含笑跟他道别,那笑容里虽满含羡慕之情,可也颇有点神气之态,因为在这个其他成员全都身强力壮的家庭中,嘉乐简直就像个婴儿,只有他是个小个子。
\par 他的五个哥哥和他父亲的身高都超过六英尺,块头也很大,可是,年已二十一岁但身材却很矮小的嘉乐自己也明白,凭上帝的才智,至多也只能让他长到五英尺四英寸半。他从来也不为自己身材矮小而无谓地长吁短叹,也从来没发现这在他争取得到自己想要的东西的过程中是个障碍,而这正是嘉乐的特点。更确切地说,嘉乐这副结实、矮小的体格正是使嘉乐之所以成为嘉乐的原因。他很早就知道,置身于身材高大的人群中,小个子的人要生存就得吃苦耐劳。而嘉乐就是个很能吃苦耐劳的人。
\par 他那些身材高大的哥哥们都是些坚强不屈却又文静温和的人,家里世代相传的往昔的荣耀已经一去不复返,这激起了他们内心的怨恨,但他们并没有说出来,而是用一种苦涩的幽默来表达不满。假如嘉乐也是个身材高大的人的话,他也会和郝家其他人走同一条路,暗中悄悄地参与反政府的活动。他妈妈满含爱意地称他是“多嘴多舌的顽固分子”。嘉乐正是这样的人,火暴的性子一触即怒,动不动就摩拳擦掌,既易怒又好斗,这点几乎人人都看得出来。他在高大的郝家人中昂首阔步、狂妄自大,就像在场院里大摇大摆地走在一群交趾大公鸡当中的矮脚鸡一样。他们也很爱他,总是充满温情地引诱他上钩,好听他大喊大叫;还会用他们的大拳头捶他几下。当然,他们一旦使小弟弟老实规矩了就罢手,决不多动他一根毫毛。
\par 嘉乐来到美国前所受的教育极少,可他自己根本就不知道。就算有人告诉他,他也不会在意的。他妈妈曾教过他读写。他的字倒写得满清楚,计算也相当出色,可他的书本知识也就到此为止了。拉丁文他只知道望弥撒时吟唱的祈祷文,历史知识也就是爱尔兰所受的各种各样的冤屈。除了摩尔\footnote{摩尔:即托马斯·摩尔(1779—1852),爱尔兰诗人。}的诗歌外,他对其他诗歌一无所知,懂的音乐也只有爱尔兰年复一年传下来的歌谣。他对那些书本学识比他强的人万分尊重,但他从来都没有感觉到自己在这方面非常薄弱。是呀,他要这些干什么呢?在这个新的国家,不是连最无知的爱尔兰人都已经发了大财吗?在这个国家,不是只要求一个人身强力壮、不怕辛劳吗?
\par 詹姆斯和安德鲁把他收留在他们在萨凡纳的店里。他虽然所受的教育不多,可他们并不觉得这有什么好遗憾的。他清晰的笔迹、精确的账目及讨价还价的精明劲赢得了他们的尊敬。假如年轻的嘉乐文学知识渊博,对音乐又有很高的鉴赏力的话,反倒会使他们对他嗤之以鼻。本世纪初期,美国对爱尔兰人还是很友善的。詹姆斯和安德鲁最初只是把装在有帆布篷顶的大马车里的货物从萨凡纳拉到佐治亚内地城镇去而已,可现在也发达了,开了自己的店铺,嘉乐也就跟着他们一起发达。
\par 他喜欢南方,而且据他自己看来,很快便变成了南方人。南方——南方人,这个中的含义是很深的,他永远也无法理解;但是有他这种凡事都全身心投入的天性,他于是就根据自己理解的方式接受了这里的观点和习俗,并且把它们变成了自己的东西。对他来说,就是打牌、赛马、最新的新闻以及决斗的全部规则、州权、对所有北方佬的诅咒、蓄奴制和棉花大王、对白人穷鬼的鄙夷以及对太太小姐们过分的殷勤。他甚至学会了嚼食烟草。他是完全没有必要刻意训练自己喝威士忌的酒量的,因为他天生就是海量。
\par 可是,嘉乐还是嘉乐。他的生活习惯和观念变了,但他的行为举止却没有改变。就算他有能力去改变,他也不会这么做。他很羡慕那些有钱的粮棉种植园主们那种不紧不慢的高雅举止。他们从自己那长满青苔的王国里纵马来到萨凡纳,自己骑在受过严格训练的良种马上,尾随其后的是坐着举止同样优雅的太太小姐们的马车及黑奴乘坐的马车。可嘉乐跟这种高雅是无缘的。那种慵懒、含糊的话语他听起来很入耳,可他舌头转出的总是自己的土腔。他也喜欢他们处理重大事情时的那份随意——把财产、种植园或是黑奴压在一张牌上,若无其事、情绪极好地注销赌输的赌注,就像他们把分币散发给黑人小孩一样干脆。但嘉乐体验过贫穷,他永远也学不会情绪极好、大大方方地输钱。这些沿海的佐治亚人确实是令人愉悦的一类人,他们虽然也容易发脾气,但在气头上说话也还是轻声慢语的。他们还会自相矛盾,可这也同样令人着迷。嘉乐喜欢他们。但这年轻的爱尔兰人身上有一股生气勃勃、烦躁不满的活力。他初来乍到,在自己的祖国,刮的风既潮湿又寒冷,薄雾笼罩的沼泽地一点也无法令人兴奋起来。这把他和这些生活在地处亚热带、空气污浊的沼泽地里的慵懒、出身高贵的上流人士完全区分了开来。
\par 他向他们学习他认为有用的东西,其余的他就置之不理了。他发现打扑克是所有南方习俗中最有用的,打扑克,还有喝威士忌的酒量。正是嘉乐打扑克和喝琥珀色酒的天赋为他赢得了他最珍视的三样财产中的两样——他的贴身男仆和种植园。第三样就是他的夫人了,能得到她,他只能归功于上帝仁慈的恩赐。
\par 名叫波克的男仆皮肤黝黑发亮,仪表堂堂,在着装上如何才能得体这方面受过严格训练。他是嘉乐和一个来自圣西门斯岛的种植园主赌了一夜扑克后赢来的。此人那虚张声势的勇气倒是可以和嘉乐相匹敌,可喝新奥尔良酒却喝不过嘉乐。尽管波克原来的主人事后要用双倍的价钱把他买回去,但嘉乐固执地拒绝了,因为这是他拥有的第一个黑奴,而这黑奴是“沿海该死的最好的男仆”,这是他向自己心中的目标迈出的第一步。嘉乐想成为拥有黑奴的主人及有地产的绅士。
\par 他已下定决心,决不像詹姆斯和安德鲁那样,所有的白日就在讨价还价中度过,而所有的夜晚则就着烛光跟账本上一列列长长的数字打交道。他深切地感受到和“做生意”联系在一起的来自社会的污辱,而他的兄弟们却一点感觉也没有。嘉乐要做个种植园主。他曾是个佃农,他的国人曾经拥有过那片土地并曾苦苦追寻过那片土地。带着这种爱尔兰人对土地的渴望,他想亲眼看见自己拥有的郁郁葱葱的田地绵延伸展到远方。这就是他几近无情的专一目标,他希望有自己的房子、自己的种植园、自己的马匹和黑奴。在他已经离开的那片国土上,购置地产有两重风险,一是苛捐杂税会使有地之人变得跟颗粒无收没什么两样;二是随时都可能会被突然没收。而在这新兴的国度就没有这些风险。所以,他打算置办地产。但是,随着时间的推移,他发现有这种抱负和把它变为现实是两码事。佐治亚州沿海被一个根深蒂固的贵族阶层牢牢地控制着,他想要得到自己想要的东西,希望非常渺茫。
\par 后来,命运之神和一手纸牌联手把一座种植园拱手送到了他面前,他后来把它叫做塔拉,与此同时,这也让他从沿海迁移到佐治亚内陆。
\par 那是春天里一个炎热的夜晚,在萨凡纳一个沙龙里,坐在旁边的一个陌生人偶然的谈话使嘉乐竖起了耳朵。这个陌生人是萨凡纳本地人,他在内地乡村地带住了十二年后刚回来。这片土地是嘉乐来美国的前一年从印第安人手里割让过来的。当时州政府正针对佐治亚中部辽阔地区发行土地彩票,此人碰巧中了彩。他便到那去建了一所种植园;可现在房子被烧毁了,他也已经厌倦了那个可恶的地方,极乐意把种植园及早脱手。
\par 嘉乐的心里从来没有停止过想拥有种植园的念头。他于是托人介绍,和这人进行洽谈。听陌生人说本州的北部地区挤满了来自卡罗来纳及弗吉尼亚州的新来者时,他的兴趣就越来越浓了。嘉乐在萨凡纳生活的日子足以让他知道沿海人的观点——州里其他所有地区都是落后的丛林地带,每一丛灌木后都躲藏着印第安人。在为郝家兄弟打点生意的时候,他曾到过从萨凡纳河逆流而上到一百英里远的奥古斯塔,他还继续往内陆地区旅行,到过从该城往西的一些老城镇。他知道那个地区跟沿海一样有很多人定居,但从陌生人的描述中,他得知他的种植园在萨凡纳西北部内陆两百五十英里处,离查特胡奇河也没多少路了。嘉乐知道,那条河以北的土地还掌握在柴罗基族\footnote{柴罗基族:美国东南部最大的一支印第安部族。}人手里,但别人提到会有印第安人骚扰时,陌生人对此予以嘲笑,他还大肆描述着在这片新兴的土地上,繁荣的城镇正在发展,种植园也不断涌现。听到这些,嘉乐大为惊奇。
\par 一个小时后,谈话渐渐少了。嘉乐提议打牌,这一诡计与他那双天真无邪、明亮湛蓝的大眼睛是极为不符的。夜渐渐深了,酒也喝得差不多了,其他人都已歇手不打,最后只剩下嘉乐和陌生人两人在赌。陌生人压上所有的筹码,接着又压上了种植园。嘉乐也推出所有筹码,把钱包放在筹码上。假如钱包里的钱正巧是属于郝氏兄弟商行的话,嘉乐的良心也不会太不安,不至于第二天一早在望弥撒前就得向上帝忏悔。他知道他想要什么,而每当嘉乐想要什么东西时,他总是采用最直接的方法来得到它。再说,他就是这么相信命运,相信自己手里四张两点的牌。他一刻也没有想过,如果桌子对面坐着的是一个比他更高明的高手,那他该怎么去偿还输掉的钱。
\par “你也并没有占到什么大便宜,我很高兴不用再为这个地方上税了。”那人手里拿着的全是一点,他叫人拿来笔和墨水,叹了口气,“大房子一年前被烧毁了,田地里长满了灌木丛和松树苗。但已经是你的了。”
\par “除非你已经不喝爱尔兰威士忌酒了,要不,绝不要一边打牌,一边喝酒。”同一天晚上,波克伺候他上床睡觉时,他严肃地对波克说。这个男仆人出于对新主人的敬慕,已经开始努力用爱尔兰的土音对主人的问题做出必要的回答。他的土音是一种吉契口音和米斯郡口音的混合,这种口音谁听了都会感到困惑不解,只有这两个人不会。
\par 浑浊的弗林特河静静地流淌着,两岸是松树形成的松墙,水边有被藤蔓缠绕着的橡树。河流像一条弯曲的臂膀,把嘉乐新得到的土地从两边环绕住。对嘉乐来说,站在房子原来所在的小山上,这道高高的绿色屏障是他拥有这片土地的证据,这是有目共睹、令人愉悦的,就像是他自己亲手立起的标明自己领地的围栏一样。房子被烧毁的地方,地基石已经是漆黑一片。他站在那,俯视着直达路边的长长的林荫道,兴奋地赌咒发誓,心灵深处的喜悦使他连感谢上帝的祷告也顾不上说了。这两排幽暗的树木是他的了,这片荒废的草坪也是他的了,虽然草坪上只零零落落地长着一些开着白花的小木兰树,树下的杂草已经有齐腰高了。还有那荒芜的田地,田里散布着许多小松树和矮树丛,红色的地面起伏可见,从四面伸展开去,直至远处,而这一切都已经属于郝嘉乐——这一切之所以都成了他的财产,是因为他有一颗清醒的爱尔兰人的头脑,有勇气把一切都压在一手纸牌上。
\par 嘉乐闭上了眼睛,在这未开垦的土地的静寂中,他感觉像回到家一样。就在他的脚下,将建起一栋刷成白色的砖房。路对过则要竖起崭新的围栏,把肥硕的牛群和纯种马匹圈在里面,在太阳光照射下,沿着山坡顺势而下直至河床的肥沃土地将像绒鸭的绒毛一样泛着白光——那是棉花,绵延数百英亩的棉花!郝家的家运又要再次兴盛了。
\par 嘉乐自己还有一小笔赌金,又从他那对此一点也不热心的两个哥哥那里借了些钱,以土地为抵押又贷了一笔款,他用这些钱买来了第一批干农活的黑奴。来到塔拉后,他在只有四个房间的监工房里独自一人住了下来,直到塔拉立起了雪白的高墙。
\par 他把田地清理干净,种上棉花,又从詹姆斯和安德鲁那里再借了些钱买来更多黑奴。郝家是个大宗族,不管是家道兴旺还是家道中落,他们都互相支持。这并不是为了夸大那份家人中存在的亲情,而是无情的岁月使他们认识到,要在世上求生存,一个家族就必须在世人面前紧紧抱成一团。他们借钱给嘉乐,接下来的几年,这钱就连本带利都收了回来。渐渐地,嘉乐又买下近旁更多的土地,种植园不断扩大。最后,白色的房子由梦想变成了现实。
\par 房子是由黑奴动手建造的,这是一座外表笨拙、毫无规划、随意扩延的建筑,耸立于山顶上,俯瞰着斜坡上郁郁葱葱的牧场。另一侧顺坡延伸至河边。嘉乐高兴极了,因为房子簇新时已经有了一副历经多年沧桑的样子。老橡树曾经亲眼看见过印第安人在它们的枝蔓下路过,现在则用它们粗大的树干紧紧环抱着屋子,枝条垂挂在屋顶上方,形成了浓密的树荫。草坪从杂草手里收回了主权,苜蓿草和百慕大草正长得厚密而青翠,嘉乐总是关照人好好保养草坪。从两旁长满雪松的林荫道到黑奴居住的那排白色的小屋,整个塔拉上空弥漫着一种浑然一体、稳定坚固、恒远持久的祥和气氛;每次嘉乐纵马转过路上那道弯,看见从青翠的枝条中隐现出来的自家屋顶时,心里的自豪感便油然而生,每次看到都好像是第一次看到时一样。
\par 这一切都是他一手操办的,就是这个个子矮小、头脑冷静、脾气暴躁的嘉乐干的。
\par 嘉乐和县里所有的邻里乡亲关系都相当不错,只有麦金托什一家和斯莱特里一家例外。麦金托什家的土地和嘉乐田地的左侧接壤,斯莱特里家的三英亩贫瘠的土地则在他土地的右侧沿着河床的沼泽地向前延伸,处于河流和卫约翰的种植园之间。
\par 麦金托什一家兼有苏格兰和爱尔兰血统,他们还是奥兰治党人。在嘉乐看来,就算他们拥有天主教徒所有的高尚品德,就凭这血统也会让他们在地狱里永世不得翻身。千真万确,他们是在佐治亚生活了七十年,在这以前,还在卡罗来纳住了整整一代人;但家族中第一个踏上美国国土的人是从阿尔斯特\footnote{阿尔斯特曾是奥兰治党成立地。}来的,这对嘉乐来说已经足够了。
\par 这家人个个沉默寡言,还顽固得要命。他们故步自封,很少跟别人来往,只跟他们在卡罗来纳的亲戚通婚。不喜欢他们的人并非只有嘉乐一人,因为县里的人都友善待人、友好来往,对缺少这些品德的人,没有人会受得了。曾经有传闻说他们同情废奴主义者,可这也并未使麦金托什一家更受人欢迎一些。老奥格斯一个黑奴也没释放过,而且还犯了不可饶恕的违反社会约定的错误,他把一些黑奴卖给了途经此地到路易斯安那州的甘蔗地去的奴隶贩子。可是,传闻并未因此而消失。
\par “毫无疑问,他是个废奴主义分子,”嘉乐对卫约翰说,“但是对奥兰治党人来说,当原则和苏格兰人的吝啬相冲突时,原则就无用武之地了。”
\par 斯莱特里一家则是另一回事。因为他们是穷苦白人,奥格斯·麦金托什的倔强不屈、与人格格不入的脾性倒是硬从邻里家庭中赢得了些许勉勉强强的尊重,可斯莱特里一家连这点尊重也没有。老斯莱特里既懒惰无能,又总是牢骚满腹。尽管嘉乐和卫约翰一再提议要购买他那几英亩薄地,他却死抓着不放。他的妻子成天蓬头垢面的,总是一脸病容、无精打采的样子,生了一群总是苦着脸、看上去像兔子一样的孩子——这个群体的数目还在每年一个地增加。汤姆·斯莱特里没有黑奴,他和最年长的两个男孩伺弄着那几英亩棉花地,他的太太和其余的孩子则照管着那个所谓的菜园子。但是,不知怎么回事,棉花总是歉收,菜园子呢,由于斯莱特里太太不停地生孩子,也很少时候能够满足她那一大群孩子的需要。
\par 汤姆·斯莱特里在邻居家的游廊上磨磨蹭蹭,讨棉花种子或是一块咸肋肉以“周济他一下”,这早已是司空见惯的事。斯莱特里没什么本事,可就是这样,他也恨透了他的那些邻居,尤其痛恨那些“财主们盛气凌人的黑鬼们”。县里大户人家的黑鬼们把自己看成是比穷苦白人更上等的人,他们那不加掩饰的蔑视刺痛了他,而他们生活中更为稳固的地位更激起了他的忌妒。跟他自己悲惨的境遇相比,他们不愁吃、不愁穿,病了、老了还有人照顾。他们为自己主人的好名声感到无比自豪,多半还为自己属于这些本身就是身份和地位的象征的人感到很骄傲。可他呢,却被所有的人瞧不起。
\par 汤姆·斯莱特里本可以以三倍的价钱把他的农场卖给县里任何一个种植园主,他们只会权当花钱为本地区清除碍人耳目的一家人而已。但他很满足于留在此地,靠每年一包棉花的收入和邻里的施舍苦撑着过日子。
\par 嘉乐和县里其余的人都保持着和睦亲密的关系。每当骑在高大的白马上的小个子纵马沿车道奔驰而来时,卫家、卡尔弗特家、塔尔顿家及方丹家,全都对他微笑致意,一边还招手让人拿来高玻璃酒杯,杯子底部放上一茶匙糖,还有一小片捣碎的薄荷叶,再往里倒上波旁威士忌酒。嘉乐很有人缘,孩子们、黑奴和狗都能一眼断定,在他大吼大叫、举止粗暴的外表下面藏着一颗善良的心。他是个极好的倾诉对象,又富有同情心,乐意掏腰包帮助别人。邻居们天长日久也都发现了这一点。
\par 他每到一家,猎狗们都狂吠不已,黑人小孩则欢叫着跑过去迎接他,为争得为他牵马的特权而争吵不休,并在他善意的辱骂中畏缩不安、再则咧嘴而笑。白人小孩则吵闹着要坐在他的大腿上玩骑马,他则在大人们面前对北方政客的狼藉声明大加谴责;他朋友们的女儿则把他当成知己,把自己的恋爱都告诉他;这一地区的小伙子们,不敢跪在地上向父亲承认欠下的赌债,但也发现他是个能帮忙的朋友。
\par “这么说,这笔赌债你已经欠了一个月了,你这个小无赖!”他会这么大吼道,“我的老天,你干吗不早点向我要钱呢?”
\par 他那粗鲁的说话方式是人所共知的,绝不会冒犯别人,只会使那些年轻人忸怩作态地笑着回答说:“喔,先生,我实在不想麻烦你,可我父亲——”
\par “你父亲是个好人,但挺严厉,这一点不可否认。那就把钱拿去吧,不要再提这件事了。”
\par 种植园主的夫人们是最后认可嘉乐人品的人。嘉乐曾把卫太太描述成“一个具有沉默寡言的非凡天赋的贵夫人”。一天晚上,当嘉乐骑马的马蹄声在车道上渐渐远去时,她告诉她丈夫说:“他虽言谈粗鲁,可是个绅士。”直至此时,嘉乐的绅士地位才最终得到承认。
\par 他一点也不知道这种认可花了他将近十年功夫,因为他从来都没想到,起先邻居们都是斜睨着眼瞧他的。他自己心里可从来没有怀疑过,从他来到塔拉的那一刻起,他就已经是个绅士了。
\par 嘉乐四十三岁的时候,体格结实、面色红润,看上去就像一个狩猎图中的乡绅。这时,他意识到,尽管塔拉很可贵,县里的人们也都心旌坦荡,全都对他敞开门户表示欢迎,但总还是美中不足。他需要一个妻子。
\par 塔拉急需一个女主人。由于厨房需要,从场院里忙活的黑奴中提升了一个胖厨娘,可她从来没有准时开过饭。侍女原是干农活的,总是让家具堆满灰尘,家里似乎从来都没有现成的干净被单。一有客人上门,家里总是大呼小叫、一派忙乱。波克是唯一受过训练的供屋里使唤的黑奴,由他总管着其他黑奴。可是,这么多年来见识了嘉乐这种乐天派的生活方式后,连他也变懒散马虎、粗心大意了。作为贴身男仆,他把嘉乐的卧室拾掇得井井有条;作为男管家,他端庄而体面地在饭桌上伺候主人。可在其他事情上,他却极少过问、任其自流。
\par 凭着非洲人那万无一失的本能,黑奴们全都发现嘉乐是个光打雷不下雨的角色,他们竟然毫无廉耻地利用他。他总是威胁着要把黑奴卖到南方去以及要把某人狠狠地抽一顿,但从来就没有一个黑奴从塔拉被卖出去过,鞭打也只发生过一次,那是因为嘉乐心爱的马在狩猎了一整天后竟然没人给它好好洗刷,为此才执行鞭打的。
\par 嘉乐蓝色的眼睛目光锐利,他当然注意到了他的邻居们的屋子理得多么井然有序,穿着沙沙作响的裙装、头发梳得一丝不乱的女主人轻松适然地管理着仆人们。他哪里知道,这些女人从早上一睁眼直到子夜时分马不停蹄地照管着煮饭、喂孩子、做针线、洗衣服,忙得不可开交。他只看到了外表的结果,而这些结果却给他留下了深刻的印象。
\par 一天早晨,他正着装准备骑马到镇里去看审案。波克拿来了他最喜欢的褶边衬衫,但是侍女缝补得太蹩脚了,以至除了他的贴身男仆外谁也穿不出去。这时他已经很明白,他急需一位太太。
\par “嘉乐先生,”波克见嘉乐发火,一边毕恭毕敬地帮他卷起衬衫的袖子,一边说,“你需要个太太,一个有很多供屋里使唤的黑奴做陪嫁的太太。”
\par 嘉乐嘴里骂着波克放肆,心里却知道他是对的。他需要个妻子,需要孩子,而如果他不能很快娶妻生子的话,那就会为时太晚了。可他不想随便和某个人结婚,就像卡尔弗特先生那样,居然把教他那些没娘的孩子的北方女家庭教师变成了自己的太太。他的太太必须是个小姐,而且必须是出身名门的小姐,应该像卫太太那样有高贵的神态和优雅的举止,而且应该有能力打理好塔拉这个家,就像卫太太那样,把自己的家管理得井然有序。
\par 但是,要娶上县里名门望族的小姐为妻有两个困难。首先是已到结婚年龄的小姐不多,其次,也是更为重要的一个,尽管嘉乐在此地已经住了将近十年,但还只是个“新来的客户”,而且还是个外国人。没有人知道他的家庭背景。在佐治亚内陆地区,虽然上流社会不像沿海贵族阶层那样坚不可摧,但也没有人会愿意把女儿嫁给一个没人知道其祖上背景的人。
\par 嘉乐知道,县里的绅士们确实很喜欢他,他成天跟他们一起打猎、喝酒、谈政治。可是尽管如此,几乎没有一个人的女儿是他可以与之成婚的。他可不打算让自己成为别人餐桌上的笑料谈资,说某某某又遗憾地拒绝了郝嘉乐向他的女儿求爱。知道这一点并没有使嘉乐觉得自己比邻居们矮一截。什么也无法使嘉乐觉得自己不如别人,无论是在什么方面。县里的人只会让女儿和名门望族的公子结婚,这只是一种怪习俗。这种名门望族必须在南方住了二十二年以上,而且应该是拥有地产和黑奴,沉迷于当时风靡一时的恶习的家族。
\par “收拾一下,我们要到萨凡纳去,”他对波克说,“只要我听到你说一声‘嘘’或‘呸’,我就把你给卖了,因为这些话我自己也很少说了。”
\par 在婚姻问题上,詹姆斯和安德鲁兴许能提些建议,也许他们的老朋友当中有些人的千金能符合他的要求,而又能接受他做丈夫。詹姆斯和安德鲁耐心地听完他的打算,可并没给他多少鼓励。他们在萨凡纳没有亲戚可帮他们的忙,因为在他们来美国时,他们都早已成家了。他们那些老朋友的女儿也都早已结婚成家,生儿育女了。
\par “你又不是很富有的人,再说,你也不出生于名门望族。”詹姆斯说。
\par “我已经赚到钱了,我也有能力成为大户人家。我可不想随便娶个太太。”
\par “你的心也未免太高了。”安德鲁干巴巴地说。
\par 但他们还是尽力帮助嘉乐。詹姆斯和安德鲁都已年过花甲,在萨凡纳混得还不错。他们有很多朋友,于是整个月领着嘉乐一家一家登门造访,参加宴会、舞会及野餐会。
\par “只有一个我看得上眼的,”嘉乐最后说,“可我来到此地落脚时,她甚至还没出生呢。”
\par “你看得上眼的是谁呀?”
\par “埃伦·罗比亚尔小姐。”嘉乐尽量随意地说着,因为埃伦·罗比亚尔那微微上斜的黑眼睛早已令他心旌摇荡了。尽管她有无数令人费解的举止,而且对一个年仅十五岁的少女来说,这些举止是令人觉得颇为奇怪的,但她还是把他迷住了。此外,她身上还有一种令人难以忘怀的绝望之情,他看在眼里,记在心上,不禁格外温柔地待她,而他对世界上任何人都没这么温柔过。
\par “可你的年纪已经可以做她的父亲了!”
\par “可我正当壮年呢!”嘉乐被刺痛了,大声叫起来。
\par 詹姆斯说话很平静。
\par “嘉乐,在萨凡纳,你跟任何女孩结婚都比跟她结婚的可能性更大。她父亲是罗比亚尔家族的,那些法国人都傲慢得不得了。她的母亲呢——愿上帝保佑她的灵魂——也是个出身名门的大家闺秀。”
\par “我才不管那么多呢,”嘉乐激动地说,“再说,她妈妈已经去世了,罗比亚尔老先生又喜欢我。”
\par “他是喜欢你,但做他的女婿,那就不一样了。”
\par “那姑娘也不会接受你的。”安德鲁插话说,“她一直爱着她的表兄,行为放荡的纨绔子弟菲利普·罗比亚尔,这事已经一年了,虽然她家里人日夜劝她算了,可她还是不听。”
\par “这个月他应该到路易斯安那去了。”嘉乐说。
\par “你怎么知道?”
\par “我当然知道。”嘉乐回答说。把这一有价值的信息提供给他的是波克,而菲利普则是应他自己家里人的特别要求才出发到西部去的。但嘉乐不肯透露这些消息。“我认为她爱他并没有到忘不了他的地步。十五岁毕竟太年轻了,对于爱知道得并不多。”
\par “可他们宁愿为她选择她那危险的表兄,而不会要你。”
\par 所以,当消息传出来,说皮埃尔·罗比亚尔的千金将嫁给从内地来的小个子爱尔兰人时,詹姆斯和安德鲁的吃惊程度并不亚于任何人。萨凡纳家家户户都在议论纷纷,推测着已到西部去的菲利普·罗比亚尔到底怎么啦,可这种闲言碎语根本得不出什么结论。为什么罗比亚尔最可爱的女儿会嫁给一个大嗓门、红脸庞、个子几乎刚够得着她耳际的小个子男人,这对大家来说都是个解不开的谜。
\par 嘉乐自己也不太明白到底是怎么回事。他只知道这是奇迹发生了。那天,皮肤白皙、外表冷静的埃伦把手挽在他手臂上,对他说:“我愿意嫁给你,郝先生。”此时此刻,他是完完全全感到自己很卑微了,这也是他一生中唯一的一次。
\par 大吃一惊的罗比亚尔一家多少知道这是怎么回事,但只有埃伦和她的黑人嬷嬷才知道整件事情的来龙去脉。那天晚上,姑娘像个伤心欲碎的孩子一样一直哭到天亮,早上起床后却已成了个主意已定的女人。
\par 嬷嬷给她年轻的女主人送来一个包裹,此时她已经预感到大事不妙。包裹是从新奥尔良寄来的,写地址的字迹很陌生。包裹里有一张埃伦的小画像,埃伦当即哭着把它扔到地上。还有四封她写给菲利普·罗比亚尔的亲笔信,新奥尔良一个牧师写来的一封短信,告知她的表哥已在一次酒吧闹事中不幸亡故了。
\par “是他们把他赶走的,是爸爸、波琳和尤拉莉把他赶走的。是他们把他赶走的。我恨他们。我恨他们所有的人。我再也不想看见他们了。我要离开他们。我要走,到一个我再也看不到他们的地方去,一个见不到这个城市或是会让我想起——想起——他们任何人的地方去。”
\par 嬷嬷一整个晚上也在黑暗中陪着她年轻的女主人掉了一夜眼泪。天快亮的时候,她提出反对意见:“可是,亲爱的,你不能这么做。”
\par “我要这么做。他是个好人。我要这么做,要不然我就到查尔斯顿的女修道院去。”
\par 也就是进修道院的威胁最终赢得了茫然失措、心碎欲裂的皮埃尔·罗比亚尔的首肯。虽然他家里人都信天主教,他却是个虔诚的基督教长老会教徒。他的女儿竟要变成修女,这甚至比让她嫁给郝嘉乐还更糟。毕竟,除了没有门第之外,此人还是挺合他的意的。
\par 就这样,埃伦从罗比亚尔家嫁出去了。她义无反顾地离开萨凡纳,并且再也不想见到它。她和她那年已中年的丈夫、嬷嬷,以及二十个“屋里的黑奴”起程来到了塔拉。
\par 第二年,他们的大孩子出世了,他们用嘉乐母亲的名字给她起名叫思嘉。嘉乐颇感失望,因为他想要个男孩。但看着他那头发乌黑的小不点女儿,他也够高兴的了。他在塔拉大宴黑奴,自己也喝得酩酊大醉,醉得大喊大叫,却也幸福无比。
\par 如果埃伦曾为自己突然决定嫁给他而后悔过,那也没有人会知道。嘉乐当然也不会知道。每次嘉乐看着她时,心里几乎都会有一种自豪感油然而生。离开萨凡纳那座风尚高雅的海滨城市时,她已经把它及有关它的一切记忆都抛至脑后。从她到达县里的那一刻起,佐治亚北部就已经是她的家了。
\par 当她离开了她父亲的房子时,她也就永远离开了那个家。这座房子流线型的线条就像女性的胴体一样美丽,也像张满帆全速前进的船只一样气势宏伟;房子被刷成淡淡的粉色,建成法国殖民地的样式,墙基离地很高,样子极为精致。屋前有盘旋而上的台阶通向屋子,两边是锻铁制成的栏杆,雅致得就像镶上了花边一样。这是一座色调暗淡但却富丽堂皇的房子,漂亮雅致,但却高高在上,可望而不可即。
\par 她不仅离开了这座高雅精致的住所,也告别了这幢建筑背后所代表的所有文明。她发现自己置身于一个完全陌生、迥然不同的世界,就像是到了一个新大陆一样。
\par 佐治亚北部地区岩石丛生,崎岖不平,住在这里的是些勤劳勇敢、吃苦耐劳的人们。这里地处高原地带,正好坐落在蓝岭山脉脚下。埃伦从这里举目四望,周围尽是绵延起伏的红色小山包,到处还耸立着浅黑色的裸露在外的花岗岩石峰及长得不甚茂密的松林。这一切对已经看惯了沿海景色的她来说,似乎都是粗野荒芜、尚未开化的。在沿海地区,海岛上的丛林静穆漂亮,岛上覆盖着灰色的青苔和缠绕不清的绿色植物。白色的沙滩沿着海岸向前伸展,沐浴在亚热带炎热的阳光下。房屋与房屋之间,平坦、狭长的沙质空地上,星星点点地散布着矮棕榈树及棕榈树。
\par 在这个地区,冬天寒冷彻骨,夏天酷热逼人。而这里的人身上都有一股生机和活力,这是颇让她感到奇怪的。这些人善良友好、殷勤有礼、慷慨大方,心眼实在是好极了。他们健康强健,极富男子气概,但很容易发怒。她已经离开了沿海地带,那里的人们总是用不经意的态度对待所有的事情,甚至对决斗和世代结仇的冤家也是如此,而且为这一点感到无比自豪;可这些居住在佐治亚北部的人们却有一点粗暴。在沿海地带,生活显得安详宁馨。可在此地,生活却充满朝气、生机勃勃,同时还富有新意。
\par 埃伦在萨凡纳认识的所有人似乎都是一个模子印出来的,他们的观点和传统都极为相似,可这里的人们却多种多样、各不相同。佐治亚北部地区的居民来自不同的地方——佐治亚其他地区、卡罗来那和弗吉尼亚以及欧洲和北方。他们中有些人才刚来不久,是到此地寻找契机以发财致富的,正像嘉乐一样。有些人则像埃伦一样出身于世家,可发现原先家里的生活令人无法容忍,于是到这边远地带来寻求一个避难所。有很多人迁居来此却什么原因也没有,只是血管里从拓荒者祖先那里继承下来的不安分的血液流速加快的结果。
\par 这些来自不同地方、有着不同背景的人们,给县里的整体生活汇入了一种不拘礼节的特质,这对埃伦来说是全新的一面,她从来就无法使自己习惯这种不拘礼节。凭本能,她知道沿海的人们在各种境况中是如何行事的。可佐治亚北部的人会怎么做,从来是无法预先知道的。
\par 加速了这一地区所有事物的进程的正是当时席卷整个南方的繁荣昌盛的浪潮。整个世界都急需棉花,而县里新开垦的土地非但不贫瘠,而且肥沃极了,成了盛产棉花之地。棉花是这个地区的心脏命脉,下种和摘棉是这片红土地的两件大事,就像是心脏的一张一弛一样。财富来源于那弯弯曲曲的垄沟,与之俱来的也就是傲慢自大了——这种傲慢自大就是建立在葱翠碧绿的灌木和一亩亩雪白的棉花上的。如果棉花能使他们这一代人富起来,那他们的下一代又会变得何等富有!
\par 确信明天会更美好,这使他们对生活兴趣顿涨、热情大增,县里的人们都在尽心尽力地享受生活,这一点是埃伦永远也无法理解的。他们有足够的钱和黑奴,这使得他们有时间玩乐,而他们也喜欢玩乐。他们从来就没有忙得放不下手头活计的时候,总是有时间举行炸鱼野餐会、打猎、赛马,几乎每个星期都要举行野餐会和舞会。
\par 埃伦从来没有想过要成为他们的一员,也无法成为他们的一员——她把自己的绝大部分都留在萨凡纳了——但她尊重他们,渐渐地还学会去欣赏这些人身上直来直去、坦率真诚的个性,他们对自己几乎毫无保留,并且能用实事求是的眼光去看待别人。
\par 她成了县里最受爱戴的邻居。她是个生活节俭、心地善良的女主人,一个称职的好妈妈,一个尽职尽责的好太太。她本要把自己的整个身心献给教堂,但现在却把一切用于照顾孩子、料理家务及伺候丈夫上。也就是这个男人带她离开了萨凡纳,使她远离与那里有关的所有记忆,但他从来没有问过任何问题。
\par 思嘉一岁的时候,用嬷嬷的话说,她比任何同龄的小女孩都更健康,更活泼。这时,埃伦的第二个女孩出世了。她被命名为苏珊·埃莉诺,但大家总是叫她苏埃伦。又过了一年,卡丽恩也来到了世上,在家谱上,她的名字是卡罗琳·艾琳。接着是三个男孩,可全都在没学会走路以前就夭折了——他们都被安葬在离房子一百码远的墓地里,在那弯弯曲曲的雪松下面,坟上各自立着一块墓碑,上面刻的字全是“小郝嘉乐”。
\par 从埃伦来到塔拉的第一天起,这地方就开始发生变化了。虽然她还只有十五岁,可已经能够担负起种植园女主人的全部责任。婚前,女孩子最重要的是要可爱、温柔、漂亮、会打扮,而婚后,人们却希望她们能够掌管黑人白人加在一起有上百号人口甚至还更多的大家庭里大大小小一应事物。在这方面,她们也是受过训练的。
\par 埃伦和所有有过良好教养的年轻小姐一样,也曾为结婚做过这方面的准备。更何况她还有嬷嬷、这个能使最懒惰的黑奴也变得有劲起来的帮手。很快,她便使嘉乐的家变得井然有序、尊贵体面、高雅漂亮,使塔拉有了一种从来没有过的美感。
\par 建这房子时本来就没有什么建筑计划,方便的时候就随时在任何一角加盖房间。但在埃伦的精心料理下,房子有了一种魅力,足以弥补设计方面的不足。从大路通向房子的那两排雪松既阴暗又清凉——而没有这两排雪松,任何佐治亚种植园主的家都不能算是完美的——但是和别的树木的暗绿相比,它们的色调又更为明快些。垂挂到阳台上来的紫藤,在刷成白色的砖墙映衬下显得生气勃勃,紫藤和门边绉绸似的粉色长春花丛交杂在一起,加上院子里白花怒放的木兰花,把房子一些难看碍眼的线条给遮掩起来了。
\par 在春夏两季,草坪上的百慕大草和苜蓿草葱翠绿的,诱人极了,本应在屋后空地上闲荡漫步的火鸡和白鹅都禁不住诱惑跑到这来。鸡鹅群中的老者受碧绿青草及甘美的栀子花蕾和白日草苗圃的引诱,不时领着同伙偷偷溜到前院来。为了防止草坪受它们的蹂躏,游廊上安排了一个黑人小孩当哨兵。他手里拿着一块破破烂烂的毛巾,坐在台阶上履行职责,这也成了塔拉整幅画面的一个部分——可他却是极为不幸的,因为他不许用石头或棍子扔这些家禽,也不能大声吓唬它们,只能用毛巾和嘘声驱赶它们。
\par 埃伦派了好几十个黑人小孩做这项工作,这是塔拉男性黑奴必须履行的第一个职责。十岁以后,他们就被送到种植园里的皮匠老爹爹那去学手艺,或到造车人兼木匠的艾莫斯那去,有的被送到照管牛群的菲利普那里,要不就到管骡子的卡菲那里。如果在这些手艺方面全都没有什么天分的话,他们就只好去干农活了,而在黑奴看来,他们也就因此而完全丧失了社会地位。
\par 埃伦的生活并不安逸,也谈不上幸福,但她从来不指望生活过得安逸。而如果不幸福的话,那也是女人的命。这个世界是男人的世界,她接受了这一点。是男人拥有财产,由女人来管理而已。管好了是男人的功劳,女人还得称赞他的聪明能干。男人手上扎了一根刺便大喊大叫,像只公牛一样,而女人连生小孩的时候也得拼命忍住呻吟,生怕会搅扰男人。男人说话粗鲁、肆无忌惮,还经常喝得烂醉如泥。女人只能对他的言语不慎毫不在意,还得把醉鬼弄到床上去,同时不能有半句怨言。男人粗暴无礼,说话没遮没拦;女人却总是宽厚善良、通情达理,还老要原谅别人。
\par 她是在有着大户人家淑女风范的传统中长大的,良好的家教教会了她如何忍辱负重,同时又能魅力犹存,她打算把三个女儿也调教成出身名门的大家闺秀。在两个小女儿身上,她倒是成功了,因为苏埃伦急于使自己柔情万种、魅力十足,所以对她妈妈的教诲总是颇为用心、言听计从,卡丽恩生性羞涩,也极易引导。但是脾气个性极像嘉乐的思嘉却觉得,通往大家闺秀的路简直荆棘丛生、乱石密布,难走极了。
\par 使嬷嬷颇为生气的是,思嘉喜欢的玩伴既不是她那两个娴静拘谨的妹妹,也不是家教极好的卫家的女孩,而是种植园里的黑人小孩和左邻右舍的小男孩。她爬树、扔石头的本领一点也不比那些男孩差。埃伦的女儿居然会玩弄这些把戏,这使嬷嬷极为不安,她经常恳求她“行事要像个小姐”。但埃伦对此忍耐有加,并且用长远的观点看待这件事。从孩提时代起她就知道,小时候的玩伴会变成日后的男友,而女孩子的首要任务就是结婚。她于是告诉自己,这孩子只是生性活泼、精力充沛罢了,以后还是有时间教会她如何吸引男人的技巧和优雅举止的。
\par 为达到这个目的,埃伦和嬷嬷全力以赴。随着思嘉渐渐长大,在这方面成了出色的学生。可在其他方面,她学到的东西就很少了。虽然家里请过几任家庭教师,在附近的费耶特维尔女子学院也待过两年,但她所受的教育还是很少。然而,县里的所有女孩中,没有哪个人的舞姿比她更优美的了。她知道怎样微笑才能使脸上的酒窝上下跳动,怎么脚尖略朝里走才能使宽大且带裙环的裙子飘曳迷人,怎么抬头看着男人的面孔,然后垂下眼睑,飞快地眨着眼睛,好像她因情感细腻而感忧虑不安似的。最重要的是,她知道用一张像婴儿一样恬静、柔和的脸掩饰了骨子里绝顶的聪明与才智。
\par 埃伦总是轻声细语地告诫她,嬷嬷则没完没了地对她百般挑剔,她们齐心协力,把那些能使她成为真正为人所求的妻子的优良品质灌输到她脑海里去。
\par “你必须更温柔些,亲爱的,还要更稳重些,”埃伦告诉她的女儿,“先生们说话的时候,你不能打断他们,即使你确确实实认为你比他们懂得多也不行。先生们不喜欢锋芒毕露的女孩。”
\par “老是愁眉苦脸、拉长着下巴,而且总是说‘我偏要’‘我偏不’的年轻小姐经常是找不到老公的。”嬷嬷闷闷不乐地预言,“年轻小姐应该垂下眼睛,说‘噢,先生,我知道啦’或是‘先生,听你吩咐好了’。”
\par 她们把一个大家闺秀应该知道的一应事宜都教给她了,可她只学会了表面彬彬有礼的举止。至于应该和这些表面举止联系在一起的内在的优雅素质,她却没有学到家,她也不明白为什么要学这些东西。有外表的东西就足够了,因为有名门小姐气质的长相,这使她大受欢迎,而这就是她所要的一切了。嘉乐吹牛说,她是五个县中数一数二的美女,但这话也并非毫无根据,因为这一带的街坊邻居当中,几乎所有的小伙子都向她求过婚,还有许多远至亚特兰大和萨凡纳的求婚者。
\par 现在思嘉已经十六岁了,她看上去既可爱迷人又风骚轻佻,这都是嬷嬷和埃伦的功劳。可是实际上,她却固执任性、爱慕虚荣、个性强硬。她继承了她那爱尔兰父亲的极易激动的性情,从母亲那却没遗传到什么东西,埃伦那毫不自私、宽容忍耐的品德,她也只继承了最表层的一丁点而已。埃伦从未意识到她只是表面上如此表现罢了,因为在她妈妈面前,思嘉总是把她最完美的一面表露出来。埃伦在场的时候,她总是把一应越轨行为都掩盖起来,而且尽力控制自己的脾气,尽可能表现得性情很好。要不然的话,妈妈责备的目光就足以使她羞愧得掉眼泪。
\par 但是嬷嬷对她可丝毫不存什么幻想,一直留神着揭开她虚饰的外表露出真面目。嬷嬷的眼睛比埃伦的厉害多了,从小时候到现在,思嘉还从来不记得有哪一次能够欺骗嬷嬷很久却不被发现的。
\par 这两个慈爱的良师对思嘉的情绪饱满、生机勃勃及迷人的魅力倒不发愁。这些都是南方的太太小姐们感到无比自豪的特点。她们担心的是她身上表现出来的嘉乐那种刚愎自用、性急鲁莽的个性。有时候,她们还担心,在找到合意可心的丈夫以前,她无法把那些有破坏性的特点很好地掩饰起来。但是,思嘉打定主意要结婚——而且要和希礼结婚——她也愿意表现得娴静、顺从、浮躁,只要这些都是能吸引男人的个性特点就行了。她不知道男人为什么会这样。她只知道这些方法行得通。她从来就没有多大兴趣试图去弄清这个中的缘由,因为她根本不知道人的头脑里是怎么想的,连她自己的都弄不清楚。她只知道,如果她这么做、这么说了,男人们也都会准确无误地继续做得更多、说得更远。这就像一个数学公式,一点也不难,因为在学生时代,数学是她比较拿手的一门科目。
\par 如果说她对男人的心灵世界知道得不多,那她对女人的就了解得更少了,因为她对她们的兴趣更小。她从来都没有女性的朋友,也从来不觉得需要这方面的朋友。对她来说,所有的女人,包括她的妹妹,都是她在追逐同样的猎物——男人中的自然对手。
\par 所有的女人都是,只有她妈妈是个例外。
\par 郝埃伦是与众不同的,思嘉把她尊为圣物,是和其他人截然不同的。思嘉还是孩子的时候,她就把她妈妈和圣母马利亚混在一起。现在,思嘉已经长大了,但她觉得没有理由改变她这种看法。对她来说,埃伦代表着绝对的安全感,而这是只有上帝和母亲才有能力给予的。她知道她妈妈是正义、真理、慈爱温柔和广博智慧的化身——是个了不起的贵夫人。
\par 思嘉很想学她妈妈的样子。唯一的困难就是,要做到公正、真诚、温柔及无私,人就得错过很多生活乐趣,无疑还有很多男朋友。但是人的一生也太短促了,失去这些令人愉快的事多可惜。她跟希礼结婚后,当她上了年纪有时间的时候,总有一天她会打算做埃伦那样的女人的。可是,到那时候……


\subsubsection{第四章}

\par 当晚吃饭的时候,因为母亲不在,思嘉在饭桌上打点着,但她只是装装样子而已。她听说的有关希礼和媚兰的可怕消息使她的心绪躁动不宁。她非常希望她妈妈能从斯莱特里家回来,因为,家里要是没有她,思嘉就感到茫然若失、孤独无助。斯莱特里一家及他们那没完没了的疾病有什么权利让埃伦离开自己的家呢?而此时此刻的她、思嘉,又是多么需要她。
\par 餐桌上气氛沉闷、毫无生气。嘉乐如打雷般的大嗓门在她耳边响个不停,最后,她觉得自己再也无法忍受了。他已把下午跟她的谈话忘到九霄云外去了,一个人滔滔不绝地讲着从萨姆特堡传来的最新消息,不时还在桌上擂着拳头、在空中挥舞着手臂以示强调。嘉乐已经养成习惯,在饭桌上总是他在唱主角。思嘉则常常想着自己的心事,很少听进他的话。可是今晚,她却无论如何也无法抵御住他的声音,虽然她尽力竖起耳朵,想听见能说明埃伦已归来的车轮声。
\par 当然,她不打算告诉妈妈她的满腹心事,因为如果埃伦知道自己的女儿居然会想要一个已经跟另一个女孩订婚的男人,她一定会大吃一惊、伤心不已的。但是,置身于她生平碰到的第一个悲剧当中,她需要她妈妈在她身边,这能带给她安慰。只要埃伦在她身边,她总是感到很安全,因为事情再糟,只要埃伦在那,她总能使事情好转起来。
\par 一听到车道上传来车轮转动的吱嘎声,她马上从椅子上站起身来,可车轮声却绕过屋子直往后院去了,她只好重新坐下。这不可能是埃伦,因为她总是从房子前面的台阶那里下车的。接着,从黑漆漆的后院传来黑人的说话声和尖笑声。从窗户看出去,思嘉看见几分钟前离开饭厅的波克手里高举着一个燃烧着的松节,有人正从车上下来,但只看得见模糊的身影。笑声和谈话声在黑夜中此伏彼起,听上去欢快亲切、无忧无虑,轻声细语如温柔的喉音,尖声喊叫则像乐声。接着就听见脚步声走上后面游廊的台阶,进了通往主房的过道,停在餐厅外的过道里。一小阵耳语声之后,波克走了进来,他身上惯有的一本正经的模样不见了,双眼不停地转动着,露出了洁白的牙齿。
\par “嘉乐先生,”他上气不接下气地说,满脸放光,一副当新郎官的得意之态,“您新买的女奴到了。”
\par “新买的女奴?我没新买什么女奴呀。”嘉乐说着,瞪着眼睛佯装不知。
\par “有的,您买了,嘉乐先生!哦,她现在正等在外面想和您说话呢。”波克回答着,一边笑,一边还激动地搓着双手。
\par “那就把你的新娘带进来吧。”嘉乐说,波克于是转身叫他的妻子进来,她刚从卫家的种植园来到这里,成为塔拉这个大家庭的一员。她走了进来,后面跟着她十二岁的女儿,瑟瑟缩缩地伏在她妈妈的身边,几乎被她妈妈宽大的花布裙给完全挡住了。
\par 迪尔西身材高大,身膀挺直。她古铜色的脸一动不动,没有皱纹,年龄在三十到六十岁之间。从相貌看,她显然有印第安人的血统,这比黑人的特点还更突出。她那红色的皮肤、高而窄的前额、高耸的颧骨、底部扁平的鹰钩鼻梁,还有下面黑人所特有的厚嘴唇,一切都表明了她是两种血统的混血儿。她沉着冷静,走起路来有一种高贵气质,甚至超过了嬷嬷的,因为嬷嬷的气质是后天学来的,而迪尔西的则是与生俱来的。
\par 她说话的时候,声音并不像大多数黑人那样含糊不清,措辞也较为谨慎。
\par “小姐们,晚上好。嘉乐先生,对不起,打扰您了。但我还是要到这来再次谢谢您买下了我和我的孩子。很多先生曾经想买我,但他们不想连我的孩子也一同买下。就为了您使我不用忍受和孩子分离的痛苦,我也得谢谢您。我一定全心全意地为您效劳,让您看看我不是个忘恩负义的人。”
\par “哦——哦。”嘉乐尴尬地清清喉咙,在大庭广众之下,自己的慈善之举被别人说穿了,他为此感到颇不好意思。
\par 迪尔西转身面对着思嘉,一种看似微笑的表情使她眼角现出了一些皱纹。“思嘉小姐,波克告诉过我,您曾叫嘉乐先生把我买下来,所以,我打算把我的普里西给你做贴身侍女。”
\par 她把手伸到身后,把那小女孩拉到前面来。她是个皮肤呈褐色的小不点,双腿骨瘦如柴,就像小鸟一样,头上用细绳绑着无数的小辫子,硬邦邦地直竖起来。她的目光锐利、机敏,不会漏过任何东西,脸上则是一副装傻的模样。
\par “谢谢你,迪尔西,”思嘉回答道,“但恐怕嬷嬷会有意见的。自我出生起,她就是我的贴身女仆了。”
\par “嬷嬷年纪大了。”迪尔西说,那副平静的神态一定会使嬷嬷大发雷霆的。“她是个好嬷嬷,可你现在是个年轻小姐了,需要个好的侍女,而我的普里西已经伺候英蒂小姐有一年了。她的针线活和梳头的本领都不比成年人差。”
\par 在母亲的督促下,普里西突然行了个屈膝礼,对思嘉咧嘴笑了,搞得别人禁不住也要对她报以回笑。
\par “真是个伶俐的小女孩。”思嘉想着,然后大声说道:“谢谢,迪尔西,妈妈回来后再谈这件事好了。”
\par “谢谢小姐。晚安。”迪尔西说完,转身和孩子一起离开了餐厅,波克讨好地跟在后面。
\par 饭后的杯盘碗盏收拾完后,嘉乐又重新开始演说,可就连他自己也不甚满意,他的听众对他的言辞就更无赞赏可言了。他大扯着喉咙预言战争即将爆发,老用反问句问别人诸如南方是不是还能再容忍北方佬的侮辱这类问题,可只是得到了略显无聊的“是的,爸爸”或“不,爸爸”这类回答。卡丽恩正坐在大灯下的一块跪垫上全神贯注地看一本爱情小说,书中的女主人公自情人死后就做了修女。卡丽恩沉浸在小说中,不禁潸然泪下,眼前似乎出现了她自己头戴白色修女帽的模样,免不了有些兴奋。苏埃伦一边在绣她笑称为“嫁妆箱”的刺绣品,一边寻思着明天的野餐会上有没有可能把斯图尔特·塔尔顿从她姐姐身边引开,用她所具有而思嘉却没有的女性魅力来迷住他。而思嘉呢,则在为希礼而心烦意乱。
\par 爸爸明明知道她伤心欲碎,他怎么还能够没完没了地谈论萨姆特堡和北方佬呢?正如许多年轻人一样,她认为人们竟然如此自私,居然全然不顾她内心的痛苦;而且,在她几乎心碎时,世界却一如既往、毫无变化,这简直使她吃惊极了。
\par 她心里已经像是刮过了一阵旋风,很奇怪,他们坐在其中的餐厅居然还是平静如水,一无二致。过去是什么样子,现在还是什么样子。沉重的红木桌子和餐具柜、既大又重的实心银器、光亮的地板上铺着的鲜艳的碎毡小地毯,所有的一切都还原地不动,就像什么也没发生过一样。这个餐厅既亲切又舒适,通常,思嘉很喜欢晚饭后和家人聚在那里的颇为宁静的几个小时;可是今天晚上,她看到它就厌恶,要不是害怕她父亲会大声质问她,她早就开溜了,她要从黑暗的过道溜到埃伦的小办公室去,坐在那张旧沙发上,把心里的痛苦都给哭出来。
\par 屋里所有的房间中,思嘉最喜欢那间。每天早晨,埃伦就在那个房间里,坐在高高的写字台前,一边理着种植园里所有的账目,一边听着监工乔纳斯·威尔克森的汇报。有时候,一家人还在那里悠闲地消磨时间。埃伦手拿鹅毛笔在账簿上记着账,嘉乐坐在那把旧摇椅上,姑娘们则赖在那张坐垫已经凹陷进去的沙发上。沙发太破旧了,没法摆在屋子前面。思嘉很希望自己现在能和埃伦一起待在那里,这样她就可以把头伏在妈妈的腿上,安安静静地哭上一阵。妈妈难道就此不回来了吗?
\par 就在这时,砾石车道上传来了车轮碾过路面的刺耳的声音,接着,埃伦柔声遣退车夫的低语声飘进房来。埃伦快步走进餐厅时,所有人都热切地抬头看着她。她的裙摆款款飘动,脸上现出疲惫而忧伤的神情。随着她走进房间,一阵美人樱香囊的淡淡香味扑鼻而来。这香味似乎总是从她裙子的褶皱处散发出来,思嘉的意念里总是把这种香味和她妈妈联系在一起。嬷嬷跟在后面几步远处,手里拿着皮袋子,下嘴唇拉得老长,前额往下耷拉着。嬷嬷边摇摇摆摆地往前走,边唧唧咕咕地自顾自唠叨着,但还会注意不让自己的嘀咕太大声,以免被别人听懂,但又要有一定的音量,以表示自己心里是绝对持不赞成态度的。
\par “我这么迟才回来,真对不起。”埃伦说着便把斜削的肩膀上的方格披巾拉下来,递给思嘉,走过她身边时,还拍了拍她的脸蛋。
\par 她的归来使嘉乐像着了魔一样脸上大放异彩。
\par “小孩受洗了吗?”他问。
\par “受洗是受洗了,但他死了,可怜的孩子,”埃伦说,“我曾担心艾米也活不成,可我现在认为她能活下去了。”
\par 姑娘们把脸转向她,既吃惊又迷惑不解,只有嘉乐达观地摇摇头。
\par “哦,小孩还是死了好,不用说,可怜的没有父——”
\par “时间不早了,我们现在最好还是祈祷吧。”埃伦打断嘉乐的话,语气非常自然。要不是思嘉很了解她妈妈,她就不会注意到这句插话的用意了。
\par 要能知道艾米·斯莱特里的孩子父亲是谁,那倒是件挺有趣的事。但是思嘉知道,如果等着从她妈妈那里听到这件事的话,那她是永远也无法知道真相的。思嘉怀疑是乔纳斯·威尔克森,因为她经常看见他和艾米黄昏时沿着大路散步。乔纳斯是个北方佬,至今还孤身一人。他只是个监工,这个事实使他永远无法步入县里上流社会的生活圈。只要有点社会地位的家庭,就不会让女儿跟他结婚,他所能交往的人就只有斯莱特里一家以及和他们一样地位低贱的人。因为在受教育方面比斯莱特里一家高出好几个级别,他不想和艾米结婚也是很自然的事,不管他在黄昏时有多经常跟她一起散步。
\par 思嘉叹了口气,因为她的好奇心强着呢。许多事情就发生在她妈妈的眼皮底下,可对她来说,却好像根本没发生过一样。只要是埃伦认为不正当的事,她就对它们不屑一顾。她也试图把思嘉也调教成这样,但并没有成功。
\par 埃伦已走到壁炉架边去取念珠,它们总是放在炉架上的镂花小首饰盒里。这时,嬷嬷语气强硬地说话了:
\par “埃伦小姐,祈祷前你得先吃些晚饭。”
\par “谢谢,嬷嬷,可我不饿。”
\par “俺得亲自去给你弄饭,你必须先把饭吃了。”嬷嬷说。她的前额因生气现出了不少皱纹。她开始走向过道到厨房去。“波克!”她大声叫道,“叫厨娘生火。埃伦小姐回来了。”
\par 地板在她肥胖的身体重压下吱呀作响,她在前面过道里的自言自语也越来越大声,餐厅里所有的人都能听得清清楚楚。
\par “俺已经说了不止一次了,给那些白人穷鬼帮忙没半点好处。他们都是些懒惰虫、忘恩负义的窝囊废、没出息的贱骨头。埃伦小姐犯不着自己累死累活去伺候他们,他们不配。要不然的话,他们就会有黑奴伺奉他们了。俺早说过——”
\par 她顺着那长长的露天过道走去,声音也慢慢远去。这露天过道上面有顶篷,直通向厨房。要让主人知道在所有的事情中,她持的是什么立场,在这方面,嬷嬷很有自己的一套。她知道,如果白人对在嘟哝自语的黑人哪怕表示一点点在意,那也是有失体面的。她也知道,白人主人为了维护面子,就必须对她说什么置之不理,就算她在隔壁房间近乎大喊大叫也是白搭。仅此一点就可以使她避免受责骂,无疑别人也会对她对事情所持的看法留有印象。
\par 波克走进餐厅,手里端着一个托盘、银制餐具及餐巾。他后面紧跟着年仅十岁的黑人男孩杰克。他一只手在匆匆忙忙地扣白麻布上衣的扣子,另一手拿着一根拂尘。这拂尘是用报纸剪成的细纸条绑在一根比他人还高的芦苇秆上制成的。埃伦原有一根用漂亮的孔雀毛制成的拂尘,但只在特殊场合才动用。由于波克、厨娘和嬷嬷都固执地迷信孔雀毛不吉利,所以每次动用前都要先在家里进行好一番争执。
\par 嘉乐为埃伦拉开椅子。埃伦一坐下来,四个声音立即便在她耳边回响着。
\par “妈妈,我新舞裙上的花边松了,可明晚在十二棵橡树的舞会上我要穿,你能不能给我缝缝呀?”
\par “妈妈,思嘉的新裙子比我的漂亮,我穿粉红色的就像丑八怪一样。干嘛不让她穿我粉色的那件,我来穿她绿色的裙子呢?她穿粉色的也不错。”
\par “妈妈,明天晚上我能不能也呆到舞会结束呢?我都已经十三岁了——”
\par “郝太太,你信不信——嘘,孩子们,别闹了,要不我得去拿鞭子抽你们一顿了!凯德·卡尔弗特今晨去了亚特兰大,他说——你们能不能安静点,好让我能听到我自己的声音?——他说那里都闹翻天了,人们的话题总离不开战争、民兵训练、组建骑兵部队。他还说,从查尔斯顿传来的消息说,他们对北方佬的侮辱已经再也无法容忍了。”
\par 埃伦一脸倦容,对这一片吵闹声,嘴角泛起一丝微笑。她首先对丈夫说话,就像身为妻子应该做的那样。
\par “如果查尔斯顿那些好人们都这么认为,我敢说,我们很快也会有同样的看法的。”她说,因为她有个根深蒂固的观念,除了萨凡纳以外,整个美洲大陆大多数名门望族都出在那座不大的海滨城市查尔斯顿,而这一观念正是查尔斯顿人普遍的共识。
\par “不,卡丽恩,明年才行,亲爱的。那时你就能待着参加舞会,也能穿大人的衣服了。到那时,我这粉色脸蛋的小家伙会多么快活啊!别把嘴翘得老高的,亲爱的,你可以去参加野餐会,记住,你也可以呆到晚餐结束,但要等到十四岁以后才能参加舞会。
\par “把你的衣服给我,思嘉。祈祷完我会把花边缝好。
\par “苏埃伦,我可不喜欢你说话的口气,亲爱的。你粉色的衣服很漂亮,配你的肤色很合适,就像思嘉的衣服也很配她的肤色一样。不过,明晚你可以戴我的石榴石项链。”
\par 站在她妈妈身后的苏埃伦得意地对思嘉皱了皱鼻子,因为思嘉也正盘算着请妈妈把项链借给她。思嘉对她伸了伸舌头。苏埃伦是个牢骚满腹、自私自利的令人讨厌的妹妹,要不是有埃伦管束,思嘉肯定会经常刮她耳光的。
\par “我说,郝先生,再跟我谈谈卡尔弗特先生说的有关查尔斯顿的消息吧。”埃伦说。
\par 思嘉知道,她妈妈一点也不关心战争和政治,认为它们都是男人的事,聪明的女人决不会关心这些事的。但这能让嘉乐发表自己的观点,也就能使他高兴,埃伦对丈夫的兴致总是考虑得很周到的。
\par 嘉乐也就接着谈他的新闻,嬷嬷把一道道菜放在主人面前,有顶端烤得金黄的松饼、油炸鸡脯肉,还有一盘切开的黄澄澄的红薯,不但在冒着热气,融化的黄油还在往下滴。嬷嬷拧了小杰克一把,他便赶忙去履行自己的职责,站在埃伦背后慢慢地前后摇动着那纸条绑成的拂尘。嬷嬷站在桌边,看着食物一叉一叉地从盘子里被送到嘴里,仿佛一旦看到什么懈怠的迹象,她就打算把食物硬塞进埃伦嘴里似的。埃伦也在很用心地吃着,但思嘉可以看出,她太累了,根本就不知道在吃什么,只是嬷嬷那张毫不宽容的脸迫使她不得不吃下去而已。
\par 埃伦吃完了所有的食物,站起身来。此时嘉乐才谈到一半呢。他正对北方佬的不光彩行径发表看法,说他们要解放黑奴,却又不肯为黑奴的自由花一个子儿。
\par “我们要祈祷了吗?”他问,口气颇为不情愿。
\par “是的。已经这么迟了——哦,实际上已经十点了。”正好钟在嘤嘤嗡嗡地报着时。“平时卡丽恩到这时早该睡着了。波克,把灯拉下来,嬷嬷,把我的祈祷书拿来。”
\par 在嬷嬷沙哑的低语声催促下,杰克把拂尘放在角落里,着手收拾桌上的盘子。嬷嬷则在餐具柜的抽屉里摸着寻找埃伦那本用旧了的祈祷书。波克踮起脚尖,抓住灯链上的环,把灯慢慢拉下来,直到桌子上方都笼罩在灯光中,而屋顶退为一片片暗影。埃伦弄好裙子,双膝跪在地上,把祈祷书打开放在面前的桌面上,十指交叉放在书上。嘉乐跪在她身边,思嘉和苏埃伦跪在桌子对面,那是她们祈祷时一贯跪的位置。她们把多皱的衬裙折了好几层垫在膝下,这样,跪在硬地板上就更不会痛了。卡丽恩年纪太小,跪在桌边不舒服,她于是跪在一张椅子前面,肘部放在椅子上。她喜欢这种姿势,因为祈祷时她很少不睡着的。而这种姿势可以躲开她妈妈的注意。
\par 一阵脚步和衣裙沙沙作响的声音,屋里的黑奴们都在门边跪了下来。嬷嬷边跪下嘴里边大声嘟哝着,波克直挺挺地跪在地上,侍女罗莎和蒂娜穿着宽大、亮丽的印花布裙,显得优雅极了,厨娘虽戴着雪白的帽子,可满脸憔悴、脸色蜡黄,杰克哈欠连天、一脸蠢相,尽可能躲得远远的,不让嬷嬷的手指够着他,怕她掐他。他们的黑眼睛都发出期待的亮光,因为和家里的白人一起祈祷是一天中的一件大事。应答祈祷中那古老而生动的词句及带着东方色彩的比喻对他们来说没有什么意义,但这使他们心中的某种欲望得到了满足,所以他们吟唱着应答词的时候总是摇头摆脑的:“上帝,怜悯怜悯我们吧,”“主啊,怜悯怜悯我们吧。”
\par 埃伦闭上眼睛开始祈祷,她的声音抑扬顿挫的,既像在催眠,又像在抚慰。埃伦感谢上帝给她的家、家人及黑奴带来健康和幸福的时候,黄色的光圈中人人都低着头。
\par 在她为住在塔拉屋檐下的所有人以及她父亲、母亲、姐妹、三个夭折的孩子及“所有在炼狱中可怜的灵魂”都祈祷完以后,她把白色的念珠放在修长的十指之间,然后双手交叉地捻着念珠,开始念玫瑰经。这就像吹过了一阵和风,白人和黑人的喉咙里同时作出了应答:
\par “圣母马利亚,上帝之母,为我们这些罪人祈祷吧,不论是现在,还是将来我们临终的时刻。”
\par 思嘉虽然伤心痛苦,强忍眼泪,但她还是深深地感受到一种宁静与安详,就像往日这种时候给她带来的感觉一样。白天的失望之情及对明天的恐惧心理减退了一些,留下了一种希望的感觉。这种安慰剂并不是因为她的心灵飞到上帝身边给她带来的,因为宗教对她来说只是一种口头上的信仰;而是她看到了妈妈脸上的那种安详的神情,她妈妈正抬头看着上帝的神座及上帝的圣者和天使,祈求上帝为所有她所爱的人祝福。每次埃伦对天说话的时候,思嘉总是确确实实感觉到天是听得见的。
\par 埃伦祷告完后,总是找不到念珠的嘉乐偷偷摸摸地用手指数着遍数开始祷告。他的声音单调低沉、索然无味,思嘉的思绪也随着他嘤嘤嗡嗡的声音而四散开去。她知道她必须好好审视审视自己的良心。埃伦教导过她,每天结束时,她都有责任认认真真地审视自己的良心,承认自己所犯的无数错误,祈求上帝原谅自己,并给予自己不再重复这些错误的力量。但此时的思嘉却在审视自己的心灵。
\par 她低下头,把头靠在十指交叉的双手上,这样她妈妈就看不到她的脸了。她的思绪便又伤感地回到希礼身上。他真正爱的其实是她,思嘉,可他怎么可能计划和媚兰结婚呢?而且他还知道她爱他爱得有多深,他怎么能够刻意伤她的心呢?
\par 紧接着,她的脑际突然掠过一个新颖的念头,这个念头就像流星一样闪闪发亮,在她脑际一晃而过。
\par “哦,希礼一点也不知道我在爱着他!”
\par 这意外的念头让她大吃一惊,她几乎喘出口大气来。有好一会,她喘气不匀,脑袋瓜都僵化了,就像瘫痪了一样,但紧接着思绪又接着向前驰骋。
\par “他怎么会知道呢?他在身边时,我总是表现得很拘谨,一副正统的淑女样,大有拒人于千里之外的神态,他很可能会认为我根本不在乎他,只把他当成一个朋友。没错,所以他从来不说什么!他觉得他的爱是毫无希望的。这就是为什么他看上去如此——”
\par 她的思绪迅速回到往昔的岁月,那时她曾发现他曾用那种奇怪的神情望着她,他那灰色的眼睛完全遮盖了他内心的想法,就像他心灵之窗的窗帘一样。可有时候,他的眼睛大睁着,没遮没拦的,清澈坦然,眼里还有一种痛苦而绝望的神情。
\par “他一定伤透了心,因为他认为我爱的是布伦特,或是斯图尔特,抑或是凯德。他很可能是这么想的。假如他得不到我,那还不如和媚兰结婚,好让他的家里人高兴。可是,如果他知道我真的爱他的话——”
\par 她那变化无常的情绪从悲哀的最低谷一下飞登到幸福的顶峰。这就是希礼沉默不语、行为古怪的原因。原来他不知道!她极愿意去相信这一点,而虚荣心也促使她相信这一点,进而把相信变成确信。如果他知道她爱他,他一定会奔到她身边来的。她只要——
\par “噢!”她不由得心花怒放,手指抠着低垂的前额。“我有多傻呀,直到现在才想到这一点!我必须想法让他知道。如果他知道我爱他,他就不会和她结婚了!他怎么可能和她结婚呢?”
\par 她突然意识到嘉乐已经祈祷完毕,她妈妈正看着她呢。她不禁吃了一惊,赶忙开始祈祷,机械地数着念珠,声音里融入了很深的感情。这使嬷嬷睁开眼睛,探究似的瞥了她一眼。她祈祷完后,轮到苏埃伦,接着是卡丽恩,也都开始祈祷,可她的思绪因那令人着魔的新想法而继续向前驰骋。
\par 就是现在也还不算太迟!县里私奔之事太经常发生了,已经订婚的男方或女方却突然和另外一个人出现在圣坛前结为夫妇。而希礼的订婚甚至都还没宣布!是的,时间还有的是!
\par 如果希礼和媚兰之间没有爱,只是很久以前的一个约定的话,那他违约和她结婚怎么就没有可能呢?假如他知道她,思嘉,爱他的话,他一定会这么做的。她得想个法子让他知道。她一定会想出办法的!然后——
\par 思嘉突然从兴致勃勃的梦想中回到现实中来,因为她竟然疏忽了应答祷文,她妈妈正责备地看着她。她重新加入祷告行列,一边却睁开眼睛飞快地扫了一眼整个房间。跪着祷告的人影、柔和的灯光、黑奴们在昏暗的阴影处摇头摆脑,即便是一小时前她看到就恨之入骨的那些熟悉的东西,转瞬间又都蒙上了她的感情色彩,房间似乎又重新变成个可爱的地方。此时此刻的此情此景,她是永远也无法忘怀的!
\par “至诚的圣母马利亚。”她妈妈吟诵道。歌颂圣母的玫瑰经开始了,思嘉乖乖地应答道:“为我们祈祷吧。”同时,埃伦便用温柔的女低音歌颂着圣母的美德。
\par 从孩提时代起,对思嘉来说,这一刻便是敬慕她妈妈的时刻,而不是敬慕圣母的时刻。也许这是对圣母的亵渎,但大家重复着那些古老的词句时,思嘉虽闭着眼睛,但似乎还能透过眼睛看见埃伦仰头朝上的面孔,而不是神圣的圣母马利亚的面孔。“病人的康复之神”“智慧的源泉”“罪人的庇护人”“神秘的玫瑰”——它们都是无比美丽的词句,因为它们都是埃伦所具有的美德。可是今晚,由于思嘉兴奋异常,她便在这整个仪式中,从被他们轻声念颂的词句中,从应答祷文的囔囔声中,感受到一种她以往从未体验过的美感。她的心里在真诚地感谢上帝,因为在她的脚下已经开辟好一条道路——可以使她从她悲哀的境地中走出来,直通希礼的臂弯。
\par 最后一声“阿门”念完时,大家都站起身来,身体多少都有点僵硬了。蒂娜和罗莎一起把嬷嬷从地上拉起来。波克从壁炉架上拿下一个长长的点火纸捻,在灯火上点燃,走进过道。在蜿蜒而上的楼梯对面有个胡桃木餐具柜,因为太大而不便放在餐厅里用,只好放在这里。它宽大的柜顶放着好几盏灯,还有一排插满蜡烛的烛台。波克带着一种夸大的尊贵神情点燃一盏灯和三根蜡烛,就像是国王寝宫的第一内侍在为国王和王后点灯照明,让他们入寝室就寝。他把灯高举过头顶,领着这队人马走上楼梯。埃伦挽着嘉乐的手臂跟在波克后面,姑娘们各自拿着一根蜡烛,跟在他们后面上了楼。
\par 思嘉进了房间,把蜡烛放在抽斗柜上,用手在黑暗的衣橱里摸着寻找要缝的舞裙。她把裙子搭在手臂上,悄悄地穿过走道。父母的卧室门微微开启着,还不等她敲门,埃伦的声音便传到她耳里,声音很低,但很坚定。
\par “郝先生,你必须解雇乔纳斯·威尔克森。”
\par 嘉乐却大声叫起来:“可我上哪去再找一个不会欺骗我的监工呢?”
\par “必须解雇他,马上,明天早晨就得让他走人。大个子萨姆是个不错的工头,他可以接管监工的职责,直到你雇到另外一个监工为止。”
\par “啊,哈!”嘉乐的声音又响了。“这么说,我可是明白了!是那可敬的乔纳斯睡了——”
\par “一定要解雇他。”
\par “这么说,他就是艾米·斯莱特里生的孩子的父亲,”思嘉寻思着,“噢,原来如此。你还能指望一个北方佬男人和一个白人穷鬼的女儿做出什么别的事情来呢?”
\par 接着,她特意停了一会,让父亲那唾沫乱溅的话有时间慢慢消失,然后敲了敲门,把裙子递给她妈妈。
\par 到思嘉脱了衣服,吹灭蜡烛躺在床上时,明天如何行动也已经详详细细地计划好了。这个计划并不复杂,她像嘉乐一样,头脑里只有要达到的目标,于是,她的双眼就只盯着这个目标,也只考虑能达到目标的最直接的几个步骤。
\par 首先,她得表现得“傲气十足”,就像嘉乐所要求的那样。从她到十二棵橡树时起,她将表现出快活且最富有生气的自我。不要引起任何人怀疑她曾因希礼和媚兰订婚之事而消沉沮丧过。而且,她将和在场的每一个男人调情逗乐。这对希礼是很残酷,但这会增加他对她的渴望之情。她不会疏忽每一个已到婚龄的男人,老到苏埃伦的男友、长着姜黄色胡须的老弗兰克·肯尼迪,小到媚兰的哥哥,腼腆、内向、爱脸红的韩查理。他们都将蜂拥在她身边,就像蜜蜂围着蜂巢转一样。希礼也一定会从媚兰身边被吸引到她的崇拜者这个圈子中来。然后,她将设法摆脱众人,单独和他在一起呆几分钟。她希望一切将按计划进展,要不然的话,事情就麻烦多了。假如希礼没有走那第一步,那她就只能亲自迈出这一步了。
\par 最后,当他们终于单独待在一起时,其他男人围着她转的那一幕在他脑海里还历历在目,他就会得到一个新的印象,那就是,那群人中的每个人都想要她,于是,他的眼里又会现出那种忧伤而绝望的神情。接着,她就会让他知道,尽管她很受欢迎,可全世界所有的男人中,她还是会选择他,这样她就能让他重新高兴起来。她羞涩、甜蜜地承认这点时,在他心里,她的地位就会比原先高出一千倍。当然,她这么做时应该表现出大家闺秀的风范。她连做梦都没有想过,自己会大胆地对他说她爱他的话——那是绝对不行的。但是怎么告诉他,这只是个细节,她一点也不为此而心烦。她曾经对付过这种情形,现在也能够再次获得成功。
\par 她躺在床上,朦胧的月光洒在她身上,她想像着整个场景。当他意识到她确确实实是爱他时,脸上就会现出惊喜的神情。此时此刻,她似乎看到了他的这种表情,而且还听到了他叫她嫁给他的话语。
\par 自然,她得说,嫁给一个已经和另一个姑娘订婚的男人,这种事情她连想都不敢想。但他会一再坚持,最后,她就会让自己被他说服。然后,他们就会决定,当天下午就逃到琼斯伯勒去,并且——
\par 噢,明天这个时候,她可能就已经成为卫希礼太太了!
\par 她从床上坐起身来,双手抱着膝。有好一会,她陶醉在身为卫希礼太太——希礼的新娘的幸福中!可紧接着,她的心里掠过一丝凉意。如果事情没有按此计划发生呢?假如希礼没有恳求她跟他一块私奔呢?但她坚决地把这种想法硬从脑海中赶走了。
\par “现在我可不考虑这一点,”她坚定地说,“假如我现在考虑这一点,这会使我感到沮丧的。如果他爱我,事情就没有理由不按我想让它们发生的方式进行。而且,我知道他是爱我的!”
\par 她扬起下巴,长着一圈黑睫毛的淡绿色的眼睛在月光下闪闪发亮。埃伦从没告诉过她,希望和让希望变成现实是完全不同的两码事;生活也还没教会她捷足未必先登的道理。生活如此美好,失败是不可能的,漂亮的衣裙和清秀的面孔便是征服命运的武器,这个年方二八的少女躺在银色的月影之中,抱了无比的勇气盘算着。


\subsubsection{第五章}


\par 已是早上十点了。对四月天来说,气候已经算是暖和的了。金色的阳光透过宽大的窗户上蓝色的窗帘洒入思嘉的房间,显得特别耀眼。米白的墙壁闪闪发亮,深色的红木家具在阳光中呈现出深红色,就像葡萄酒一样。地板像玻璃似的反射出白光,只有铺着碎毡小地毯的地方显现出鲜明的色彩。
\par 夏天的脚步已经款款移近,这是佐治亚州夏日来临的第一个迹象。春之高潮虽不情愿,却也只好让位给夏之酷暑了。一股怡人的暖意漫进房里,夹杂着各种怡人的香气,有各种各样的花香、已泛新绿的树香及新翻过的红土潮湿的气味。从窗户看出去,思嘉可以看到砾石车道两边的黄水仙正开得绚丽夺目,黄茉莉花团锦簇,花束四处散开,却又谦恭地垂向地面,就像内有裙环的飘曳长裙一样。反舌鸟和㭴鸟为争夺她窗下那棵木兰树的所有权,又在进行那场旷日持久的争夺战了。它们叽叽喳喳地争吵着,㭴鸟声音刺耳,态度蛮横,反舌鸟声音恬美,鸣声哀戚。
\par 这么一个阳光明媚的早晨,思嘉常常会被吸引到窗前,把手支在宽大的窗台上,呼吸着塔拉各种芳香的气息,聆听着塔拉的各种声音。可是今天,她无心欣赏这灿烂的阳光和蔚蓝的天空,头脑中只掠过这么一个想法:“感谢上帝,还好没下雨。”床上放着那件苹果绿波纹绸舞裙,叠得整整齐齐的放在一个大纸盒里,淡褐色的花边从中间往下垂着。舞裙已经准备好送到十二棵橡树去,以便舞会开始前好换上。可思嘉看到它却耸了耸肩。如果她的计划获得成功,今晚她就用不着穿它了。等舞会开始,她和希礼早就上路到琼斯伯勒结婚去了。麻烦的问题是——野餐会上,她穿什么衣服好呢?
\par 什么衣服最能衬出她的妩媚,使她对希礼产生不可抗拒的魅力呢?从八点开始,她就一直在试穿衣服,可没一件令她感到满意的。此刻的她正穿着花边长裤、亚麻紧身胸衣和有三层波浪形花边的亚麻衬裙站在房里满心沮丧、烦躁不安呢。衣服扔得到处都是,散落在她周围。地上、床上、椅子上,全是一堆堆色彩鲜艳的衣服和零零落落的缎带。
\par 那件玫瑰色的玻璃纱裙子配粉红腰带挺合适,但去年媚兰到十二棵橡树来的时候,她已经穿过了,媚兰一定会记得的。她还可能会不怀好意地把这一点说出来。这件黑色的毛葛细斜纹裙,袖子蓬松,配着公主花边领,倒是能极好地衬出她那雪白的肌肤,但会使她看上去稍显老气一些。思嘉急切地在镜子中端详着自己年方二八的脸孔,就像是想找出皱纹或下巴已经松弛的赘肉似的。在媚兰那张孩子气十足的可爱的脸孔面前,若是自己显得稳重、老气,那是绝对不行的。那件淡紫色条纹的薄纱裙,镶着宽大的花边,边上还有镂网状小孔,漂亮倒是蛮漂亮,但和她这种体型不相配。卡丽恩身材小巧,脸上无甚表情,这件裙子倒是蛮适合她的。但若思嘉穿起来就会使她看上去像个小女生。在沉着冷静的媚兰面前,她看上去却像个小女生,那也是万万行不通的。这件绿色的方格塔夫绸裙镶着荷叶边,每片荷叶边末梢还用绿色的天鹅绒滚边,应当是最合适的了。实际上,这是她最喜欢的裙子,因为穿着它会让她的眼睛颜色更深,成了祖母绿的颜色。但是它的胸前有一块显眼的油污。当然,她可以把胸针别在这点油污上,可是万一媚兰眼睛很尖呢?剩下的就是五颜六色的棉布裙了,思嘉觉得它们都不是这种场合能穿的节日盛装。还有就是舞裙以及昨天穿过的有枝叶花型的平纹布绿裙子。可这是下午穿的裙子,不适合穿去参加野餐会,因为它只有一点蓬袖,而且领口开得很低,都可以在舞会上穿了。但除此之外也毫无办法,只好穿它了。即使在早晨就光着脖颈、袒胸露臂的,可她终究也不会为此而难为情的。
\par 她站在镜子前面,一边扭过身子看自己的侧面,一边想着,她的身材绝对没有哪一部分会让她感到见不得人的。她的脖子虽短,但浑圆柔润,胳膊丰满迷人。她的胸部在紧身胸衣的衬托下高高隆起,漂亮极了。她从来就不用像许多十六岁的女孩那样,要在紧身胸衣的衬垫上缝上一排排小小的丝褶边,以使身材现出理想的曲线和丰满的体型。她遗传了埃伦细长、白皙的双手和小巧的双脚,为此她很高兴。她也希望能有埃伦那样的身高,但自己的身高已经令她很满意了。可惜腿不能露出来,她边寻思着,边拉起衬裙遗憾地看着双腿,它们在长裤里面同样现出丰满而匀称的线条。这双腿确实漂亮极了,连费耶特维尔女子学院的姑娘们都承认这一点。至于腰肢——费耶特维尔、琼斯伯勒乃至三县中也没有谁的腰肢能如此纤细的。
\par 想到腰身,她的思绪也就回到实际问题上来。绿色的平纹布裙子腰部是十七英寸,而嬷嬷给她束腰时是让她穿腰部十八英寸的毛葛细斜纹布裙的。嬷嬷应该把她的腰部束得更紧些。她推开门,侧耳听了听,听到嬷嬷在楼下过道里沉重的脚步声。她知道,自己可以提高嗓门而不会受到责备,因为埃伦正在熏肉房里给厨娘分派今天的食物呢。于是她不耐烦地大声叫嬷嬷。
\par “有些人认为俺会飞呢。”嬷嬷嘟哝着拖着脚步走上楼来,气喘吁吁地走进房间,一副时刻准备战斗的表情。她那双黑色的大手上端着一个熏肉盘,上面有两个涂满黄油的甘薯,一堆荞麦饼还在滴着汁液,还有一大块涂满肉卤的火腿。看到嬷嬷手里拿着这些东西,思嘉脸上微微烦躁的神情变成了准备坚定不移地交战的神色。思嘉只顾着激动地试穿衣服,倒把嬷嬷那条雷打不动的规矩给忘了。那就是,郝家的姑娘们去参加任何聚会以前必须先在家里吃得饱饱的,这样,在聚会上就没法再吃点心了。
\par “你端来也没用。我不吃。你可以拿回厨房去。”
\par 嬷嬷把盘子放在桌子上,两手叉腰站在那里。
\par “不,你必须吃!俺可忘不了上次野餐会发生的事。俺那时病了,你去之前没有给你端来食盘。今天你可得把每一样东西都给俺吃下去。”
\par “我不吃!来吧,帮我把腰束紧些,我们已经迟了。我听到马车已经被赶到屋子前面去了。”
\par 嬷嬷换上了哄人的口吻:
\par “来吧,思嘉小姐,你最好还是吃一点。卡丽恩小姐和苏埃伦小姐都把她们那份全部吃完了。”
\par “她们当然会吃完的,”思嘉轻蔑地说,“她们就像兔子一样没什么主见。我才不吃呢!我对这些食盘里的食物讨厌透了。我可不会忘记上次去卡尔弗特家之前,我吃了满满一盘东西,等到他们端出用大老远从萨凡纳带来的冰淇淋时,我却一勺也吃不下了。我今天要玩个痛快,想吃多少就吃多少。”
\par 听到这些极富挑战性的左道邪说,嬷嬷气得低头皱起了眉头。一位年轻小姐能做什么,不能做什么,在嬷嬷看来,这其中的差别就像是黑人和白人之间的差别一样非常明显,中间是没有缓和余地的。苏埃伦和卡丽恩就像是她那强有力的手里的泥土一样,都会恭恭敬敬地听从她的劝诫。但想教导思嘉,让她知道她有很多心血来潮的冲动是与大家闺秀的风范格格不入的,这却总是像进行一场艰苦的战斗一样颇费口舌。嬷嬷制服思嘉总是来之不易,而且总是用上了一些阴谋诡计,而这些诡计是没有一个白人会知道的。
\par “你如果不在乎别人怎么议论咱们这个家,俺还在乎呢,”她声音低沉、大声说道,“俺可不想站在旁边,听着野餐会上每个人都在谈论你如何没教养。俺一再告诉你,从一个人像小鸟那样吃东西的方式就能知道她是不是位出身名门的小姐。俺可不打算让你在卫先生家像个做农活的下人那样狼吞虎咽。”
\par “妈妈也是个贵夫人,可她也吃的。”思嘉反驳说。
\par “你要是结婚了,你也可以吃,”嬷嬷也针锋相对,“埃伦小姐像你这把年纪的时候,出去从来不吃东西的,你姨妈波琳和尤拉莉也一样。但她们婚后就都吃了。大吃特吃的姑娘家往往嫁不出去。”
\par “我才不信呢。上次野餐会你病了,我事先也没吃东西,卫希礼还对我说,他喜欢看见一个姑娘有这么个好胃口。”
\par 嬷嬷摇摇头,表示不吉利。
\par “先生们说的和心里想的可不是一回事。俺就没看见希礼先生向你求过婚。”
\par 思嘉一下便怒容满面的,正想说几句厉害的话,却又忍住了。嬷嬷击中了她的要害,她已无话可说了。看到思嘉满脸执拗倔强的表情,嬷嬷端起食盘,改变了战术,转用黑人那种软的一套伎俩。她边起脚向门边走去,边叹着气。
\par “那好吧。厨娘装这盘食物时俺还告诉她:‘从一个人的吃相,你就可以断定她是不是大家闺秀。’俺还对厨娘说,‘俺还从来没见过哪个白人小姐比韩媚兰上次来拜访希礼先生时吃得更少的了。'——俺是说,她来拜访英蒂小姐的时候。”
\par 思嘉满脸狐疑,飞快地看了她一眼,但嬷嬷宽大的脸上只有一副无辜和遗憾的神情,好像为思嘉不是像韩媚兰那样的大家闺秀感到很可惜似的。
\par “把食盘放下,过来把我的腰再束紧些,”思嘉烦躁地说,“然后我会试着吃一些,如果我现在先吃,我的腰就会束得不够紧了。”
\par 嬷嬷把一副胜利者的得意姿态掩盖起来,将食盘放下。
\par “俺的小羊羔要穿哪件裙子?”
\par “那件。”思嘉回答着,用手指着那堆蓬松的绿色平纹布花裙子。嬷嬷马上又进入状态准备战斗了。
\par “不行,你不能穿那件。早晨穿它不合适。下午三点钟以前决不能露出胸部。再说,那件裙子既没领子也没袖子,你一定会生出痱子来的。去年你到萨凡纳的海滩去,就长了一身痱子回来。俺可没忘记,这一整个冬天俺都在用酸奶给你擦,好不容易才好了。你再要穿那件,俺就去告诉你妈妈。”
\par “如果我穿戴好以前你去对妈妈说一个字,我就一点东西也不吃了,”思嘉冷冷地说,“只要我穿好了,妈妈要让我回来换衣服也来不及了。”
\par 嬷嬷看到自己这一招不灵,只好叹了口气表示放弃。虽然两样都不是什么好事,但既然两者只能取其一,那与其让她像猪那样狼吞虎咽地大吃大喝,还不如让她在早晨的野餐会上穿下午装来得好。
\par “抓住什么东西,吸一口气。”她命令道。
\par 思嘉照办了,她摆正姿势,两手紧紧抓牢床架杆。嬷嬷用力往后拉着、扯着,束着鲸骨腰带的腰围便越发纤细了。嬷嬷眼里露出了又骄傲又喜欢的神情。
\par “再没有人的腰能像我的小羊羔这般细的了,”她赞赏地说,“每回俺给苏埃伦小姐束腰时,一束到细于二十英寸一点点,她就像是要晕过去了。”
\par “噗!”思嘉喘了口气,说话有些费劲了,“我这辈子还没晕过去过呢。”
\par “噢,有时晕那么一两回也不打紧。”嬷嬷劝她说,“有时你也真不懂分寸,思嘉小姐。俺一再告诉你,看见蛇呀、老鼠呀什么的,你不晕过去就不太好。俺不是说你在家里也要这样,而是你和别人一起出去的时候。俺已经告诉过你——”
\par “噢,快点!别啰唆了。我会找到丈夫的。即使我不尖叫,不晕过去,你瞧瞧我是不是就找不到丈夫。天哪,我的腰束得太紧了!帮我穿上裙子吧。”
\par 嬷嬷把下摆宽及十二码的绿色枝叶花型平纹布裙子小心地放下,罩住像山一般的衬裙,然后把绷紧、低胸的上衣的背钩钩上。
\par “在太阳底下,你得把披巾披在肩上,太热时也不要把帽子脱掉,”她用命令的口吻说,“要不然的话,你回家来的时候就会变得跟老斯莱特里太太一样,看上去像棕色人种了。来吧,过来吃吧,宝贝,可别吃得太快了。再重新束腰可就不管用了。”
\par 思嘉听话地在食盘前坐下,心里想着,她往胃里咽下一些食物后,到底还能不能呼吸。嬷嬷从脸盆架上拉下一块大毛巾,小心地系在思嘉脖子上,抖开折叠的部分铺在她腿上。思嘉先吃火腿,硬把它咽下,因为她喜欢火腿。
\par “我真恨不得已经结婚了,”她一边厌恶地对付着吃甘薯,一边不满地说,“老是要矫揉造作的,从来就不能做我自己想要做的事,我简直烦透了。我得装出吃得不会比小鸟多一点点,想跑时却又只能走路,刚跳完一支华尔兹舞曲,就得说我感觉快晕过去了。实际上,我还能连跳两天两夜却一点也不会累。对这一切,我都厌烦透了。还有,对一个见识还不如我一半的男人,却必须对他说‘你真了不起!’去欺骗他,还得假装我啥都不懂,好让男人告诉我这,告诉我那,让他这么做时感觉到他自己很重要,所有这些都使我讨厌极了……我实在是一口也吃不下了。”
\par “再吃一块热饼吧。”嬷嬷毫不宽容地说。
\par “为什么女孩子要找个丈夫就得表现得这么愚蠢呢?”
\par “俺觉得,是因为先生们不知道他们想要的是什么。他们只知道他们认为想要的东西。把他们认为想要的东西给了他们也就省了很多事,不至于做一辈子老姑娘。而他们认为,他们想要的就是胆小得像耗子一般、胃口又像小鸟一样、一点儿见识也没有的姑娘。如果一位先生怀疑哪位小姐见识比他多的话,他是不会想跟她结婚的。”
\par “如果婚后男人发现他们的妻子比他们更有见识的话,你想想,他们难道不会感到吃惊吗?”
\par “哦,那已经太迟了。他们已经结了婚。再说,先生们也希望他们的太太有见识。”
\par “总有一天,我要做所有我想做的事,说我想要说的话,就算别人不喜欢,我也不会在乎的。”
\par “不,那可不行,”嬷嬷严厉地说,“只要俺还有一口气,你就不能那么做。你吃饼吧。用卤汁泡一泡,宝贝。”
\par “我想,北方的女孩子就不用像这样表现得如同傻瓜一样。去年在萨拉托加的时候,我就注意到很多女孩子都表现得非常有见识,在男人面前也一样。”
\par 嬷嬷哼了一声。
\par “北方的女孩子!是的,俺也认为她们会直截了当地说出她们的想法,但俺可没发现在萨拉托加有多少人向她们求婚。”
\par “可北方佬也得结婚哪,”思嘉争辩道,“她们也不是光长大就好了。她们也得结婚生子。她们的数量可多啦。”
\par “男人跟她们结婚是为了她们的钱。”嬷嬷肯定地说。
\par 思嘉把麦饼放在卤汁里浸了浸,然后放到嘴里。也许嬷嬷说的话也有一定的道理。肯定是有一定的道理的,因为埃伦也用不同但更委婉的词句说过类似的话。实际上,她所有女伴的妈妈都让她们的女儿们记住,必须做个柔弱无助、依赖性强、有着小鹿般眼睛的可人儿。确实,要培养并保持这么一种姿态得花很多精力。也许她真的是太鲁莽了。她偶尔也会和希礼辩论,坦率地发表自己的看法。或许这一点以及她那些健康的乐趣,诸如散步呀,骑马呀什么的,导致他把注意力从她身上转移到脆弱温顺的媚兰那里去了。也许,如果她改变一下自己的策略的话——但是她觉得,要是希礼也屈从于这些预先谋划好的女人家的花招的话,她就再也不会像现在这样敬重他了。如果一个男人居然傻到会拜倒在这样一个咯咯傻笑、胆小得会晕过去、会说“噢,你真了不起!”的女孩子的石榴裙下的话,这样的男人是不值得要的。可他们所有的人似乎都喜欢这一套。
\par 如果说她过去对希礼采用的策略用错了——哦,那也只是过去的事,都已经结束了。今天,她可是要采用迥然不同的策略,正确的策略。她要他,而她只有几个小时的时间来得到他。如果晕过去,或假装晕过去能成为获得成功的诀窍的话,那她也会采用晕过去这一招的。如果咯咯傻笑、卖弄风情或没有头脑能吸引他,她也会愉快地打情卖俏,甚至表现得比凯思琳·卡尔弗特更没有头脑。如果有必要采取更大胆的措施的话,她也会采用的。今天可是时候了。
\par 没有人告诉思嘉,她自己生气勃勃的个性尽管令人吃惊,但这比她可能采用的任何伪装都更吸引人。如果有人告诉她这一点的话,她一定会很高兴,但又会觉得不可置信。而且,她置身其中的文明社会也会觉得不可置信的,因为,从古至今,以至从今往后,从来没有一个时代会对女性的自然风范加以奖赏的,哪怕是极小的奖赏也没有。
\par  
\par 马车载着思嘉,沿着红土大路向卫家的种植园驶去。她母亲和嬷嬷都没有随行,思嘉因此而觉得很快乐,但也因这快乐而感到有点内疚。野餐会上就不会有人微微皱起眉头或拉长下嘴唇来影响她把计划付诸实施了。当然,明天苏埃伦是一定会大讲特讲的,但如果一切都如思嘉所希望的那样进展顺利的话,她和希礼订婚,或是同他私奔,给家里人带来的刺激一定会超过原来的不快心情。是的,埃伦不得不要待在家里,这使她很高兴。
\par 一大早,嘉乐喝够了白兰地后,便把乔纳斯·威尔克森给解雇了。埃伦留在塔拉,要在他走以前把种植园的账目理清楚。思嘉吻别她母亲时,她正坐在小办公室里的宽大写字台前,上面放着插满了票据、账单的分类文件架。乔纳斯·威尔克森手里拿着帽子站在她旁边,紧绷着灰黄色的脸,对心里的愤怒几乎不加什么掩饰。这么随随便便地就失去了县里最好的监工工作,他感到气愤极了。而这一切只不过是因为一次无足轻重的风流韵事。他已经跟嘉乐反复说明,艾米·斯莱特里的孩子也可能是其他一打男人中任何一个人的孩子,这于她是很容易的事,就像可能怀上他的孩子一样容易——这点嘉乐也同意,但就埃伦来说,这并无法改变他的境遇。乔纳斯恨所有的南方人。他们对他虽客客气气,但这种客气极为冷淡,并且表露出对他低微的社会地位的轻视来,根本没有对此加以很好地掩饰。他最恨的就是埃伦了,因为她是他痛恨的南方人身上所有特点的集中体现。
\par 嬷嬷作为种植园的总管,也留下给埃伦帮忙。坐在车夫托比旁边一起随行的是迪尔西,姑娘们的舞裙装在一个长盒子里,放在她腿上。嘉乐骑着他那高大的猎马走在马车旁边。他喝过酒后很兴奋,而且对自己这么快就解决了威尔克森这件令人不快的事感到很高兴。他把所有的责任都推给了埃伦,至于她因此没法去参加野餐会以及不能和朋友们相聚而感到很失望,他头脑里可没有一点谱。这是一个和煦的春日,他的田地漂亮极了,鸟儿也在欢唱,他觉得自己生气勃勃的,恣意玩笑,就像年轻人一样,根本就不会想到别人。不时地,他就会蹦出一首《低靠背车上的假腿人》或其他爱尔兰小调,或是哀悼罗伯特·埃米特的忧伤歌曲《她已经远离了她那年轻的英雄长眠的土地》。
\par 他非常高兴,想到他可以花上一整天时间大谈特谈北方佬和战争,他就兴奋非凡。他也为三个漂亮的女儿感到骄傲,此时此刻,她们正穿着带裙环的靓丽、飘曳的长裙坐在马车上,打着可笑的镶着花边的阳伞。他根本就没有想起他前一天和思嘉的谈话,因为他已经把这忘到九霄云外去了。他只想到她很漂亮,是他的一种荣誉,而且今天,她的眼睛绿得就像爱尔兰的青山。这想法使他的自我感觉也好了许多,因为这比喻还很有诗意呢,于是他便对女儿们大声唱起了稍稍走调的《绿衣裳》。
\par 思嘉带着爱意轻蔑地看着他,就像母亲瞧着自鸣得意的小儿子一样。她知道,天黑以前,他又将喝得烂醉如泥了。乘着夜色回家的路上,他又将像往常一样,试图跳越十二棵橡树和塔拉之间的每一道围栏。她不禁希望,凭着上帝的仁慈及他那匹马的好悟性,他不会因此而折断自己的脖子。他将放弃过桥的方法,让马游过河,大喊大叫着回到家,让波克把他弄到小办公室的沙发上躺下。在这种时刻,波克总是掌着灯在前面的过道里等着他。
\par 他将会把他的绒面呢新衣服弄得一团糟,第二天早晨便破口大骂,对埃伦详细地叙述他的马如何在黑夜中摔到河里去了——这种一听便知的谎言瞒不了任何人,但大家都会接受,这使他觉得自己很聪明。
\par “爸爸是个可爱、自私、不负责任的可人儿。”思嘉心里想着,涌起了一股对他的爱意。今天早上,她既兴奋又高兴,以至把整个世界包括嘉乐都包容进她爱的行列中。她很漂亮,她深知这一点;今天还没过完,她就要把希礼占为己有了;太阳温暖,阳光柔和,佐治亚春日的景色展示在她眼前。路两旁的黑莓以其最柔软的新绿掩盖住了被冬天的雨水冲刷出来的一道道红色、突兀的冲沟,耸立于红土之上的光秃秃的花岗岩巨石上覆盖着星星点点的金樱子,周围点缀着只有丁点紫色的野生紫罗兰。河边树木葱郁的小山上,洁白耀眼的山茱萸争相怒放,好像白雪还残留在绿叶上一样。正开着花的酸苹果树花团锦簇的,从嫩白色逐渐变成最深的粉色。树下,阳光把松树点缀得斑斑点点的,野生的忍冬青形成了一块夹杂着猩红、橘黄和玫瑰色的多色地毯。微风中夹着一丝灌木发出的淡淡的甜香味,所有东西的气味都好极了,使人食欲大开。
\par “我死也不会忘记今天有多么美丽,”思嘉心里想着,“也许今天就是我结婚的日子呢!”
\par 她心里一阵激动,想着就在今天下午,或是今晚月色当空时,自己就可能和希礼一块骑着马飞快地穿越这鲜花绽放的美丽景致,到琼斯伯勒去找牧师。当然,以后她也得由一个亚特兰大的牧师重新举行结婚仪式,但这应该是埃伦和嘉乐要操心的事了。埃伦乍一听到自己的女儿居然会和另一个女孩的未婚夫私奔这消息时,一定会羞愧得脸色惨白的。想到这点,她心里不禁有点心虚。但她知道,埃伦看到她幸福快乐时,一定会原谅她的。嘉乐也会声嘶力竭、大声叫骂,因为他昨天还表示不想让她和希礼结婚,不过,如果自己的家庭能和卫家联姻,他也会高兴得不知说什么好。
\par “可这已经是我结婚以后要考虑的问题了。”她一边想,一边甩甩头,把这一重忧虑从脑海中抹去。
\par 十二棵橡树的烟囱刚刚从河对过的小山上冒出头来,在这样一个春天里和煦的阳光下,除了令人心动的快乐,是不可能感受到别的什么的。
\par “我一辈子都将住在那,将会看到五十个像这样的春天,也许还会更多。我要告诉我的孩子们以及孙子孙女们,这个春天有多美,比他们将要看到的任何一个春天都更可爱。”这最后一个想法使她快乐至极,不禁和嘉乐一起唱起了《绿衣裳》的最后一段,并博得了嘉乐的大声喝彩。
\par “我真不明白你今天为什么这么高兴。”苏埃伦生气地说,因为她心里还在想着,她若穿上思嘉绿色的绸舞裙,一定比它的合法主人看上去漂亮得多。对出借自己的衣服和帽子,思嘉为什么总是那么小气自私呢?妈妈又为什么老护着她,说绿色不是适合苏埃伦的颜色呢?“你和我一样清楚,希礼订婚的事今晚就要宣布了。今天早晨爸爸就已经说过了。我知道,你已经对他倾心好几个月啦。”
\par “你也就只知道这些罢了。”思嘉说着伸了伸舌头,并不因此而放弃自己的好心情。明天早晨这个时候,苏埃伦小姐还不定会有多惊奇呢!
\par “苏西,你知道不是这样的,”卡丽恩吃了一惊,不禁申辩道,“思嘉中意的是布伦特。”
\par 思嘉转过身,绿色的双眸含笑看着她的小妹妹,真弄不明白为什么每个人都这么可爱。全家人都知道,十三岁的卡丽恩那颗心已经放在布伦特身上了,可布伦特除了把她看成是思嘉的小妹妹外,从来就没有对她动过一丝念头。埃伦不在跟前时,郝家的人总会开她的玩笑,甚至把她气哭。
\par “亲爱的,我一点也不在乎布伦特。”思嘉宣布说,为自己的慷慨感到很高兴。“他对我同样不在乎。我说,他正在等你长大呢!”
\par 卡丽恩圆圆的小脸变得通红,心里既高兴又不太相信。
\par “噢,思嘉,是真的吗?”
\par “思嘉,你知道的,妈妈说过,卡丽恩还太小,不能想男朋友的事,可你却在给她灌输这种思想。”
\par “行,那你去告密好了,看看我会不会在乎,”思嘉回答说,“你要阻止西西,因为你知道再过一两年,她就要长得比你漂亮啦。”
\par “今天你们说话可得给我小心点,否则我就要抽你们鞭子了,”嘉乐警告道,“好了,别出声!我听到的是不是车子的声音?那应该是塔尔顿家的或是方丹家的了。”
\par 他们快到通往含羞草庄园和费尔希尔的那条岔路了,这条路从一座丛林茂密的小山上沿坡而下。这时,马蹄声和车轮声越来越清楚,树丛后还传来女性说话的声音,吵吵嚷嚷的,正在愉快地争论着什么。嘉乐骑马走在前面,在两条路交汇处勒住马缰,示意托比把马车停下来。
\par “这是塔尔顿家的太太小姐们。”他告诉他的女儿们,红润的脸上神采飞扬的,因为除了埃伦,县里的太太中他最喜欢的就是红头发的塔尔顿太太了。“又是她亲自赶车。哦,她真是个会弄马的好手!她手的动作像羽毛一样轻柔,却又像牛皮鞭一样有力,就为这些,就漂亮得令人禁不住想吻一下了。更可惜的是,你们没有一个人有这么一双好手,”他带着慈爱而责备的眼神看了女儿们几眼,继续说道,“卡丽恩害怕那些可怜的动物,苏呢,手一抓住马缰就像熨斗一样硬邦邦的,你呢,小姑娘——”
\par “哦,不管怎么说,我还从来没被马掀翻过,”思嘉愤愤不平地说,“再说,塔尔顿太太每次打猎时都被马摔下来。”
\par “而且像男人一样把锁骨都给折断啦,”嘉乐说,“但是既没有昏过去,也不会大惊小怪的。好了,别再说了,她已经来啦。”
\par 看到塔尔顿家的马车时,他站在马镫上,利索地挥手脱下帽子致意。车上坐满了姑娘们,她们身着靓丽的服装,撑着阳伞,围着飘曳的面纱。正如嘉乐所说的那样,塔尔顿太太亲自坐在驾驶座上驾车。她的四个女孩,还有她们的嬷嬷及放舞裙的长纸盒全都挤在车上,根本就没有车夫的位子了。再说,只要自己手里没有缰绳,比阿特丽斯就绝不乐意别人驾车的,不管是白人还是黑人。她看似脆弱,但骨架极好,皮肤雪白,好像那火红的头发把她脸上的颜色都给弄到生气勃勃、红得发亮的一堆堆发丝里去了,然而,她不但非常健康,而且还有不知疲倦的精力。她一共生了八个孩子,个个都像她一样有着火红的头发和勃勃的生气。县里的人都说,她把她的孩子们抚育成人的方式是最成功的,因为她对她的孩子们就像对她养的小马驹一样,既加之以慈爱的纵容,又施之以严格的纪律。塔尔顿太太的座右铭是:“既要约束他们,又不要对他们管得过死。”
\par 她很爱马,总是把马挂在嘴边。她比县里任何男人都更了解马匹,驭马的才能比他们任何人都好。马儿从围场上蜂拥到屋前的草场上,就像她的八个孩子们从她那杂乱无章的房子里拥到小山上一样。她在种植园里走动时,马匹、儿子、女儿以及猎狗都紧紧跟在她后面。她相信她的马,特别是她那匹通人性的红色母马内利。如果屋里的事情让她忙得超过了她每天骑马的时间,她就会把糖碗塞到一个黑人男孩的手里,对他说:“给内利一把糖吧,告诉她我马上就来。”
\par 除了少数的场合以外,她总是穿着骑马装,因为不管她有没有骑马,她总是希望能骑一骑,因此一起床就穿上骑马装。每天早晨,内利总是被配上马鞍,在屋子前面走来走去,等着塔尔顿太太能从家务活中抽出一小时来。可费尔希尔是个不易管理的种植园,几乎没法抽出时间来。时间一小时又一小时地过去,内利也没有人骑,只好在那走来走去,塔尔顿太太则把骑马装的下摆捋到齐手臂处,连骑马装的式样也看不出来了,只在底下露出六英寸长的亮闪闪的靴子。
\par 今天,她穿着已经不流行的窄裙环的暗黑色丝绸裙子,看上去好像还穿着骑马装似的,那是因为裙子的裁剪极为朴素;插着黑色长羽毛的黑色小帽斜扣在头上,遮住了一只热情洋溢、不断闪烁的棕眼睛。而这帽子也只不过是她打猎时用的破旧不堪的帽子的翻版。
\par 看到嘉乐,她挥了一下鞭子,令她那两匹正踏着舞步前进的红马停了下来。车后座上的四个姑娘探出身子,大声打着招呼,使马队也吃了一惊。过路人看到,一定会觉得塔尔顿家的人好像是好几年没见到郝家的人了,其实他们仅仅分开了两天。但他们两家都是友善可亲的家庭,又都喜欢他们的邻居,特别是郝家的姑娘们。准确的说,他们喜欢苏埃伦和卡丽恩。在县里,除了没有头脑的凯思琳·卡尔弗特,没有一个姑娘会真正喜欢思嘉的。
\par 夏天,县里几乎平均一星期就会举办一次野餐会或舞会。对红头发的塔尔顿家的人来说,他们有足够的能力来让自己尽兴。每次野餐会和舞会都会令他们激动万分,好像他们是第一次参加一样。她们漂亮而丰满,一齐挤在马车里,于是裙环和裙子的荷叶边便交叠在一起,阳伞在她们头顶上互相碰来碰去。她们戴着意大利太阳帽,上面围着一圈玫瑰花,就像花冠一样,还垂挂着黑色的天鹅绒帽带。她们红色头发的细微差别都由这些帽子代表了,赫蒂是纯粹的红色,卡米拉是草莓般的白里透红,兰达则是像铜一样的茶褐色,还有小贝齐,她的是像胡萝卜长在地面部分的颜色。
\par “真是一群出色的姑娘,太太。”嘉乐献着殷勤,策马和邻家的马车一道前行。“但要超过她们的妈妈,那就差得远啦。”
\par 塔尔顿太太转动红棕色的眼珠,咂了咂嘴,做出一副滑稽的感激状。姑娘们大叫起来:“妈妈,别再飞媚眼了,不然我们要去告诉爸爸了!”“我敢起誓,郝先生,有你这么一个英俊的美男子在身边,她从来就没给过我们露脸的机会!”
\par 思嘉也和其他人一样,被这些俏皮话逗笑了,然而,一贯如此,塔尔顿家的人对他们的妈妈这种自由自在的态度总是使她颇为吃惊。她们的所作所为似乎只把妈妈看成是她们中的一员,是个年仅十六岁的姑娘。对思嘉来说,对自己的妈妈说这种话,几乎是一种亵渎。然而——然而——塔尔顿家的姑娘们和她们妈妈的关系中有一种令人愉快的和谐气氛,尽管她们批评她、指责她、取笑她,她们还是很敬慕她。但是思嘉忠诚地赶快告诉自己,这并不是说自己更喜欢像塔尔顿太太这样的妈妈,而不喜欢埃伦,但是,能和妈妈打闹笑骂也挺有趣的。她知道,即使有这种想法也是对埃伦的不敬,不禁为此感到很内疚。她知道,坐在车里的四个被火红头发覆盖着的脑袋瓜,从来不会被这类令人讨厌的想法弄得心绪不宁。像以往一样,每当感到自己和邻居们不一样时,她心里便会涌起一股令人恼怒的慌乱情绪。
\par 虽然她思维敏捷,但不善分析。但她隐隐觉得,尽管塔尔顿家的姑娘们像马儿一样难以驾驭,像发情的野兔一样野性十足,但是她们头脑简单、无忧无虑,而这也是她们从父母那遗传来的一种特性。她们的父母都是佐治亚人,是佐治亚北部人,和拓荒者那辈只隔了一代人。他们对自己和周边环境都有非常确定的信念。他们凭本能就能知道自己是怎么样的人,卫家的人也是如此,虽然方式完全不一样。在他们身上,没有时常使思嘉心里气愤不平的这种冲突,也就是说话柔声细气、教养过分良好的沿海贵族血统和精明朴实的爱尔兰农民血统混合在一起的冲突。思嘉既想如同崇拜偶像一样敬重、爱慕她的妈妈,也想去拨弄她的头发,跟她开开玩笑。她知道,她必须想方设法把两者统一起来。同样缘于这种相互冲突的情感,使她既想在男孩子面前表现得像个温文尔雅、出身高贵的大家闺秀,又想做个不在乎跟别人亲几个吻的孟浪女郎。
\par “今早埃伦上哪去啦?”塔尔顿太太问道。
\par “我们刚解雇了我们的监工,她留在家里跟他理清账目呢。塔尔顿先生和小伙子们呢?”
\par “噢,他们早在几小时前就骑马到十二棵橡树去了——要去尝尝那种用果汁呀,酒呀混合在一起的甜饮料,看看酒的成分够不够,我敢说,就好像是他们从现在起直到明天早晨都不会沾一口似的!我要叫卫约翰留他们在这过夜,就算他只能让他们睡马厩也没关系。五个喝得烂醉的人,我可没办法应付。三个我还能应付自如,但是——”
\par 嘉乐赶忙打断她,换个话题。他能够感觉到自己的女儿在身后窃笑,因为她们都会记得,去年秋天从卫家办的上一次野餐会回来时自己是什么样子的。
\par “你今天为什么不骑马呀,塔尔顿太太?真的,没有内利,你看上去就不太像以往的你了。你真是斯坦特\footnote{“斯坦特”是希腊神话里特洛伊战争中的传令官,声音非常洪亮,相当于五十个人同时喊叫的音量。}。”
\par “斯坦特,真是个无知的男子汉!”塔尔顿太太模仿着他那爱尔兰土腔大叫道,“你是要说森特\footnote{“森特”则是希腊神话里半人半马的怪兽,但又有骑术高明之人的意思。嘉乐把两者弄混了。}吧。斯坦特是个嗓门像铜锣的人。”
\par “是斯坦特或是森特问题都不大,”嘉乐回答说,对自己的错误表现得若无其事,“你也是有这种像铜锣一样的嗓子的,太太,你催赶猎狗的时候,用的就是这种声音。”
\par “真是你的声音,妈妈,”赫蒂说,“我告诉过你,你每次一看到狐狸,你就叫得像个科曼契人\footnote{科曼契人是居住在美国西南部的印第安人的一个分支。}似的。”
\par “但是,不像嬷嬷给你洗耳朵时你叫的那么大声,”塔尔顿太太回嘴道,“而你已经——十六岁呢!哦,说到我今天没有骑马,内利今早产崽啦。”
\par “它真的产崽啦!”嘉乐叫了起来,兴趣十足,眼里闪耀着爱尔兰人对马的热情。思嘉重新把她妈妈和塔尔顿太太相比,不免又大吃一惊。对埃伦来说,母马从不产小马,母牛也不会生小牛。事实上,母鸡也几乎不会下蛋。埃伦完全不管这些事情。塔尔顿太太可没有这些节制。
\par “是头小母马,对不对?”
\par “不,是头蛮不错的小公马,双腿有两码长呢。你得骑上马去看看它,郝先生。它真是匹塔尔顿家的马。毛发就像赫蒂的鬈发一样红。”
\par “长得也很像赫蒂呢。”卡米拉说,接着就尖叫着躲进一大堆裙子、裤子和颤动着的帽子中不见了,原来赫蒂确实长着一张长脸,听到这话便开始拧她了。
\par “我的这群小母马,今天早晨可高兴啦,”塔尔顿太太说,“自从今天早晨听到关于希礼和他那亚特兰大的小表妹的消息后,就把她们给乐坏了。她叫什么名字来着?媚兰?上帝保佑这孩子,她真是个可爱的小东西,但我从来就记不住她的名字,也记不清她长什么样。我们家的厨娘是卫家管家的老婆,昨晚他上我们家来,带来了这个消息,说是今晚要宣布这桩婚事。厨娘今天早晨把这消息告诉我们了。姑娘们都为此激动万分,可我不明白为了什么。这事大家都知道好几年了,也就是说,如果他没有和梅肯县的伯尔家族的表亲联姻的话,他就会跟她结婚的。就像卫哈尼会和媚兰的哥哥查理成婚一样。对了,郝先生,你告诉我,卫家的人如果和本家族以外的人结婚,是不是就不合法呢?因为——”
\par 其余的笑谈思嘉可没听进去。有一瞬间,就好像太阳避到了云层后面,把整个世界留在了阴影中一样,把一切的一切的色彩都给抹去了。刚刚泛出新绿的草地看上去一副病容,山茱萸苍白无力,刚刚还美丽非凡的开着粉色红花的酸苹果树,现在则色泽暗淡、毫无生气。思嘉的手指抠着马车的内壁,有一刻,连手里的阳伞也因拿不稳而晃动起来。知道希礼订婚是一回事,可听到别人这么随便地谈论此事又是另一回事。紧接着,她那十足的勇气又回到身上来了,于是太阳重新露脸,景色又欣然怡人。她知道希礼爱她。那是确定无疑的事。想到今晚根本就不会宣布什么订婚时塔尔顿太太会有多惊奇——私奔事件发生时她又会如何地吃惊,思嘉不禁露出快慰的微笑。她一定会告诉邻居们,思嘉是个多么顽皮的家伙,居然能若无其事地坐在那听着她谈论媚兰,而她和希礼一直就在——想到这里,她现出了深深的酒窝。赫蒂一直在热切地观察着她妈妈的话会有什么效果,此时却往靠背上一靠,不解地微微蹙起了眉头。
\par “我可不在乎你怎么说,郝先生,”塔尔顿太太强调说,“老是和自己的表亲结婚是不好的。希礼要和韩家的孩子结婚简直糟透了,但哈尼嫁给脸色苍白的韩查理倒是——”
\par “哈尼若不和查理结婚,她就抓不住其他人了,”兰达不留情面地说,因自己很受欢迎感到有恃无恐,“除了他,她从来没有过别的男朋友。虽然他们订婚了,他也不是特别喜欢她。思嘉,你记得去年圣诞节时他是怎么追你的吗——”
\par “别这么刻薄,小姐,”她妈妈说,“表亲不应该结婚,即使是父母的堂表兄妹的孩子也不行。这会削弱血统的。这跟马儿可不一样。你可以让一匹母马和自己的同胞兄弟交配,或是让种马和自己生的母马交配,而且只要你知道马种,结果就不错,但对人可不合适。血统也许很好,但精力不济——”
\par “得了,太太,这点上我倒想跟你辩一下了!你能不能跟我说说比卫家更出色的家族呢?他们可是自布赖恩·博鲁还是孩子的时候就开始互相通婚了。”
\par “应该是他们停止的时候了,因为已经有了不好的迹象。噢,希礼倒看不出多少问题,因为他是个英俊漂亮、精力充沛的小伙子,尽管他——但看看卫家那两个面色苍白的可怜的姑娘吧!当然,她们是好姑娘,但面色太苍白了。再看看瘦小的媚兰。骨瘦如柴、弱不禁风、无精打采的。自己一点见解都没有。‘不,太太!'‘是的,太太!’她就只会说这些。明白我的意思吗?那家人需要新鲜的血液呢,像我的红头发姑娘或你家思嘉这样富有朝气的良好血统。哦,请不要误会我的意思。照他们自己的生活方式,卫家人倒是好人。你知道,坦率地说,我很喜欢他们!可他们生养过密,又总是近亲结婚,对不对?在干燥的跑道上,在结实的跑道上还能走得不错。但请注意,我相信,卫家在泥泞的跑道上就动弹不得了。我相信,他们的精力在繁殖过程中都耗尽啦,有紧急情况时,我可不相信他们能够应变不测。他们是个只能在好天气里跑的家族。至于我,我可要一匹在任何天气情况下都能跑的好马!他们总是近亲通婚,这也已经使他们跟这里其他人不一样了。他们总是爱摆弄钢琴,还一头埋进书本里。我确实相信,希礼是宁愿读书而不愿去打猎的!是的,我确确实实相信这一点,郝先生!只要看看他们的骨架就行了,太瘦小啦。他们需要的是力大无穷的母马和种马——”
\par “啊——啊——哦。”嘉乐嘴里说着,突然意识到,这么一个最最有趣且于他完全对味的话题对埃伦来说可是完全不同的,他不禁感到颇为内疚。事实上,他知道,如果埃伦知道他们当着她的女儿们的面谈论这么坦率的话题,她就再也不会泰然自若了。但塔尔顿太太还跟往常一样,讲到她最喜欢的话题,也就是繁殖问题时,对其他话题就充耳不闻了,不管是马的繁殖还是人的繁殖。
\par “我知道我在说些什么,因为我有几个近亲结婚的表亲。我告诉你吧,他们的孩子全都像牛蛙一样个个都是暴眼睛,可怜的孩子。所以,我家要我和一个远房表兄结婚时,我就像小马一样奋起反抗。我说:‘不,妈妈,我可不干。我的孩子会得跗骨内肿和喘息病的。’噢,我妈妈听我说到跗骨内肿时晕了过去,但我坚持我的立场,我奶奶也支持我。你知道,她对马匹交配知道得很多,说我是对的。她还帮我和塔尔顿先生一块逃跑呢。呶,你看看我这些孩子们!全都又高大又健康,他们中没有一个病恹恹或是发育不全的,虽然博伊德只有五英尺十英寸高。可卫家——”
\par “我不是故意要改变话题的,太太。”嘉乐赶紧打断她的话,因为他已经注意到卡丽恩现出了一副困惑不解的神情,苏埃伦脸上则表现出极强的好奇心。他害怕她们会问埃伦一些令人尴尬的问题,那就会露馅,显出他这个护送者是多么不称职了。他注意到,他的思嘉倒像个淑女似的想着其他事情,心里颇为高兴。
\par 赫蒂·塔尔顿解了他的围。
\par “天哪,妈妈,我们还是赶路吧!”她不耐烦地叫起来,“太阳正烤着我呢,我都可以感觉到脖子上的痱子冒出来了。”
\par “等等,太太,再打扰你一会,”嘉乐说,“关于卖马给骑兵连的事,你决定怎么办?现在战争随时可能爆发,小伙子们都想把事情定下来。这是克莱顿县的骑兵连,我们也想给他们配备克莱顿县的马。可你太固执了,还是不愿把你的好马卖给我们。”
\par “也许根本就不会有什么战争。”塔尔顿太太敷衍着说,她的思路已经完全从卫家古怪的结婚习惯中转移了。
\par “哦,太太,你不能——”
\par “妈妈,”赫蒂又插话了,“你和郝先生不能到十二棵橡树再谈马的事吗?”
\par “你说对了,赫蒂小姐,”嘉乐说,“我只耽搁你一分钟。我们一会就能到十二棵橡树,那里,所有男人,老老少少都想知道马的事。啊,看到像你妈妈这样出色、漂亮的太太对她的马匹这么小气,真让我痛心!我说,你的爱国心哪去了,塔尔顿太太?南部邦联对你来说难道一点意义也没有吗?”
\par “妈妈,”小贝齐说,“兰达坐在我的裙子上,把裙子都弄皱了。”
\par “好了,把兰达推开,别插嘴。哦,听我说,嘉乐先生,”她反驳道,眼睛变得咄咄逼人,“别拿南部邦联来压我!我想南部邦联对我和对你意义是一样的,我有四个儿子在骑兵连,而你一个也没有。但我的儿子们会自己照顾自己,而我的马却不会。如果我知道要骑我的马的人是我认识的小伙子,也就是那些有良好教养的绅士的话,我会很乐意无偿献出马匹的。不会的,我一秒钟也不会犹豫的。但是,让我漂亮的马儿给那些只习惯骑骡子的乡巴佬和白人穷鬼骑!那可没门,先生!想到它们被人骑得鞍部有擦伤和肿痛,却又没有被好好饲养,我就会做噩梦。你想想,我会让那些无知的傻瓜骑我这些娇生惯养的宝贝,马嘴给勒得一道一道的,还不住地抽打它们直到它们垂头丧气、一点生气也没有吗?哦,想到这些,我现在浑身都起鸡皮疙瘩了!不,郝先生,你想要我的马也是一片好意,但你最好到亚特兰大去买些年迈的老马给你那些乡巴佬用吧。他们死也不会知道这会有什么差别的。”
\par “妈妈,我们难道不能继续上路吗?”卡米拉问道,加入了不耐烦的行列,“你知道得很清楚,不管怎么样,你最终都会把你的宝贝给他们的。爸爸和男孩们谈论一番南部邦联需要它们等等道理后,你就会大哭一场,然后让它们走。”
\par 塔尔顿太太咧嘴笑了,抖了抖缰绳。
\par “我才不会做这种事呢。”她说着用马鞭轻轻碰了碰马。马车便轻快地跑了起来。
\par “真是个好样的女人。”嘉乐说。他戴上帽子,在自己的马车旁站好位置。“继续上路吧,托比。我们会慢慢说服她,把马匹弄到手的。当然,她是对的。她是对的。一个男人如果不是绅士,那他就没有资格骑马。步兵连才是他该去的地方。但更遗憾的是,这县里种植园主的儿子不多,不够组建步兵连。你说呢,小姑娘?”
\par “爸爸,请你骑在我们后面或是前面吧。你扬起了一片尘土,我们都被呛死了。”思嘉说,她觉得再也无法忍受说话声了。这搅扰了她的思绪,她正急于让自己的思绪和脸部表情在到达十二棵橡树以前现出迷人的模样呢。嘉乐顺从地用靴刺踢了踢马肚子,转眼消失在一片红色的尘土中,追随塔尔顿家的马车去了。在那里,他又可以继续有关马的话题了。


\subsubsection{第六章}

\par 他们过了河,马车上了山坡。尽管十二棵橡树还没映入眼帘,但思嘉已经可以看见高高的树顶上空悠悠然缭绕着一股轻烟,飘来一阵阵燃烧着的山核桃木块和烤猪肉和羊肉混杂在一起的香味。
\par 从昨晚就开始生火慢慢让其燃烧的烧烤坑,此时已吐出红玫瑰般长长的火舌。上方转动着的烧烤架上烤着肉,肉汁滴落到炭火上,发出嘶嘶的声音。思嘉知道,微风吹过来的芳香是从大房子后边高大的橡树林里过来的。卫约翰总是在那里举办野餐会,那是一个缓缓下行的山坡,直通到玫瑰花园里。这是个舒服、阴凉的所在,比别人的,比如说,卡尔弗特家举办野餐会的那个地点就舒服多了。卡尔弗特太太不喜欢烧烤的食物,声称那烧烤味几天几夜都还萦绕着屋子,所以她的客人们只好在离房子有四分之一英里远的一个平坦、不遮阴的地方烧烤,备受酷暑的煎熬。至于在全州以热情好客闻名的卫约翰一家,当然知道该如何举办野餐会。
\par 餐桌是由桌面搁在支架上而搭成的,长长的野餐桌总是放置在树木最浓密的树荫下,上面铺着卫家上好的台布,没有靠背的长凳子摆在两边;周围空地上还零零星星放着椅子、跪垫和坐垫,这是给那些不喜欢长凳子的人准备的。长长的烧烤坑离这还有一段距离,烧烤的浓烟不会飘到这里来。烤坑里烤着肉,大铁锅里是调味汁和不伦瑞克炖菜,香味扑鼻,令人垂涎欲滴。卫先生总是让至少十二个黑人端着托盘穿梭于烧烤坑和餐桌之间,伺候客人。在仓房后面,往往还有另外一个烧烤坑,这里是客人的仆人、车夫和侍女用餐的地方。他们吃的是玉米饼、甘薯,还有黑人都很喜欢的那道猪内脏——猪小肠。如果时令碰巧,还会有西瓜供他们一饱口福。
\par 鲜嫩的肉香扑鼻而来,思嘉不禁皱了皱鼻子,吸进这诱人的香味。她希望等肉烤好时,自己多少会有些食欲。像以往一样,她吃得这么饱,束腰的带子又系得这么紧,她真担心自己随时都可能会打嗝。那就糟透了,因为只有老头老太们打嗝才不用担心会引起众人的反感。
\par 他们到了坡顶,白色的房子便以完美、和谐的姿态展示在她面前,高大的柱子、宽敞的走廊、平缓的屋顶,美得就像一个靓丽的妇人。她对自己的魅力信心十足,因而对所有人都慷慨大方、宽厚仁慈。思嘉甚至比喜欢塔拉还更喜欢十二棵橡树,因为她有一种高贵的美,持重而尊贵,而这是嘉乐的房子所没有的。
\par 宽大、弯曲的车道上停满了上着鞍的马和马车,正在下马或下车的客人跟朋友们打着招呼。每逢聚会,黑人们都会激动非常。他们笑容满面,把马儿牵到场院去卸车下鞍。一群群孩子,有黑人也有白人,在刚冒出新绿的草地上大喊大叫、跑来跑去,有玩跳格子游戏的,有玩捉人游戏的,还有的在吹牛皮说自己今天能吃多少东西。从房子前面直通到后院的过道里挤满了人。郝家的马车在屋子前面的台阶前面停了下来,思嘉看见穿着用裙环撑开的裙子的姑娘们像花枝招展的蝴蝶一样,在一楼到二楼的楼梯上上上下下、飞来飞去的,不时还停下来倚在精致的楼梯扶手上,笑着对那些在底下过道里的年轻男子叫喊着。
\par 从敞开的法式窗户看进去,她可以看见年纪较大的太太们坐在客厅里,穿着黑色的绸布裙,一副稳重肃穆的样子。她们坐在那里,一边摇着扇子,一边聊着孩子、病痛以及谁又和谁结婚了,为了什么而结婚等等。卫家的管家汤姆手里端着一个银制托盘,正在过道里快速穿行着,他一边笑着弯腰行礼,一边把杯子递给穿着浅黄褐色和灰色长裤、质地良好的褶边亚麻布衬衫的年轻小伙子们。
\par 阳光灿烂的屋前游廊上也挤满了客人。是呀,整个县的人都来了,思嘉心想。塔尔顿家的四个男孩和他们的父亲一块斜靠在高大的柱子上,和往常一样,双胞胎兄弟斯图尔特和布伦特没有分开,肩并肩地站在一起,博伊德和汤姆则和他们的父亲在一块。卡尔弗特先生在近旁站在他那北方佬妻子的身边,她就是在佐治亚待了十五年之后,似乎也还是不属于这里。大家都对她很礼貌,也很客气,因为他们都为她感到难过,但没有一个人会忘记,她不但投胎投错了地方,还当过卡尔弗特先生的孩子们的家庭教师,这就错上加错了。卡尔弗特家的两个男孩雷福德和凯德,正和他们打扮得花枝招展的金发妹妹凯思琳在一起,拿脸盘黝黑的乔·方丹及他那漂亮的未来新娘萨莉·芒罗开着玩笑。亚历克斯·方丹和托尼·方丹正跟迪米蒂·芒罗低声耳语着,逗得她发出一阵阵银铃般的笑声。还有远至十英里外的拉夫乔伊及费耶特维尔和琼斯伯勒来的家庭,也有一些来自亚特兰大和梅肯的客人。房子被人群挤得水泄不通,谈话声、笑闹声、女人的尖叫声此起彼伏、不绝于耳。
\par 游廊的台阶上站着卫约翰,他满头银发、身板挺直,浑身散发出安详的魅力和热情,就像佐治亚夏天的阳光一样,永不缺乏怡人的温暖。他身边站着卫哈尼\footnote{英语Honey一词有“宝贝儿”的意思,这里用以指哈尼嘴巴很甜,把谁都叫成“宝贝儿”。},人们这么叫她是因为她对谁都冠之以“宝贝儿”这一称呼,对她父亲这么叫,对干农活的黑人也这么叫。此时她正烦躁地笑着和刚到的客人打招呼。
\par 哈尼神情不安却明显想吸引在场的每个人的注意力,她那样子和她父亲泰然自若的神情形成了鲜明的对比。思嘉便寻思着,也许塔尔顿太太说的话毕竟是有些道理的。卫家的男人继承了祖上的容貌,这一点也没错。卫约翰和卫希礼灰色的眼睛上方睫毛浓密,呈深金色,但哈尼和她姐姐英蒂的脸上,睫毛既稀疏又毫无色彩。哈尼没几根睫毛的长相很奇怪,就像一只兔子似的,而英蒂呢,就只好用相貌平平来形容她了。
\par 英蒂此时连人影都看不见,思嘉知道,她很可能正在厨房给仆人作最后的指示呢。“可怜的英蒂,”思嘉想,“自从她妈妈去世后,她就被家务缠身,以致除了斯图尔特·塔尔顿外,一直没有机会去交别的男朋友。可要是他认为我比她漂亮,那也绝不是我的过错。”
\par 卫约翰走下台阶,把手臂伸给思嘉。她下车时,看到苏埃伦在傻笑。思嘉便知道,她是在人群中看到了弗兰克·肯尼迪了。
\par “我要是找不到比那穿着裤子的老处女\footnote{因弗兰克年已四十,又像老处女般婆婆妈妈,所以思嘉戏称他为“穿裤子的老处女”。}更好的男朋友,那才怪呢!”思嘉轻蔑地想着。她双脚着地时,微笑着向卫约翰致谢。
\par 弗兰克·肯尼迪赶忙跑到马车边,帮助苏埃伦下车。苏埃伦拼命抑制住自己的感情,思嘉看了那模样简直想甩她一巴掌。弗兰克·肯尼迪可能比县里任何人拥有更多的土地,也可能心地非常善良,但与他自身的条件相比,这些东西便显得无足轻重了。他年已四十,身材瘦小,整日惴惴不安的,留着稀疏、姜黄色的胡子,还像个老处女那样爱大惊小怪的。然而,想到自己的计划,思嘉掩饰了轻蔑之情,对他莞尔一笑,跟他打着招呼,搞得手里挽着苏埃伦的他愣了一会神,两眼瞪着思嘉,一副高兴而茫然的神情。
\par 思嘉的眼睛在人群中搜寻着希礼的身影,甚至在和卫约翰愉快地进行简短的交谈时也没有停止搜寻,但他不在游廊上。十几个声音同时叫着跟她打招呼,斯图尔特和布伦特也向她走了过来。芒罗家的姑娘们冲过来,对她的衣服评头论足的,她很快便成了一大片声音的中心。声音越来越大,似乎要努力盖过喧闹声。可希礼在哪里呢?还有媚兰和查理?她环顾周围,视线往过道里那群笑闹着的人群望过去,可又尽量不露出找人的样子。
\par 她一边谈笑,一边飞快地打量着屋子和院子。这时,她的视线落在了一个陌生人的身上。他独自一人站在过道里,用一种冷淡而不礼貌的神情看着她。这使她陡然升起一股强烈的复杂感受,一方面是因自己吸引了这个男人而带来的女性的快意,另一方面是自己衣服领口太低而产生的尴尬之情。他看上去已有了一定的年纪,至少有三十五岁。他个子很高,身板结实。思嘉心里想,自己从来没看见过肩膀这么宽、肌肉这么发达的男人,对上流社会的人来说,几乎是发达得过分了。当他们的目光对视时,他对她笑了笑,修剪得很密的黑胡子下面露出像动物一样洁白的牙齿。他脸盘黝黑,黑得像个海盗一样,双眼又大胆又乌黑,就像个海盗在判定是否要放弃劫掠一艘西班牙大帆船的行动或是糟蹋少女的举动时的眼睛一样。他对她展露笑容时,脸上有种冷淡而满不在乎的神情,嘴角却露出玩世不恭的样子,思嘉不禁倒吸了一口冷气。她觉得她应该感到自己被这样的一种表情冒犯了,可她却没有这种感觉,不禁对自己颇为恼火。她不知道他是谁,但不可否认,他那黝黑的脸上有良好血统的迹象。这从他丰满、红润的嘴唇上方的鹰钩鼻以及高高的额头和分得很开的眼睛就看得出来。
\par 思嘉硬是把目光从他身上移开,并且没有对他报以回笑。这时,有人在叫他,他于是转过身去。
\par “瑞德!白瑞德!快上这来,我要让你见见佐治亚州心肠最硬的姑娘。”
\par 白瑞德?这名字听起来挺熟悉,似乎和某种令人愉快的谣传有联系,但她全部心思都在希礼身上,便把这个想法从脑海中抹去了。
\par “我得上楼去梳梳头发。”她对斯图尔特和布伦特说,他们正想把她从人群中带走,让她脱不开身。“你们俩等着我,别跟别的女孩跑了,要不我会很生气的。”
\par 她看得出来,今天她若和任何别的人打情骂俏,那就没人管得住斯图尔特了。他一直在喝酒,一副傲慢无比、蓄意打架的神情。她从经验知道,这就意味着挑衅生事了。她在过道里停了一会,跟朋友们说话,和英蒂打招呼。英蒂刚从房子后面过来,头发凌乱,额头上还挂着小小粒的汗珠。可怜的英蒂!头发淡而无色,睫毛也毫无色彩,突出的下巴意味着脾气固执,这已经是够糟的了。此外,她虽还不到二十岁,却已经像个老处女一样。她不知道,如果她把斯图尔特从她身边抢过来,英蒂是不是会非常不满。很多人都说,她还在爱着他,可是卫家的人到底在想什么,这是从来都不会有人知道的。即使她对此不满,她也从来不会露出什么迹象,还是用她惯常对思嘉的那种有点冷淡却又和善客气的态度对待她。
\par 思嘉愉快地跟她说着话,开始沿着宽大的楼梯往上走。这时,她听到背后有个羞答答的声音在叫她,她转过身,看到叫她的是韩查理。他长得满英俊的,皮肤白皙的前额上留着一绺蓬松的淡棕色鬈发。双眼呈深棕色,清澈而温和,就像大牧羊犬的眼睛一样。他穿着芥末色裤子,黑色上衣和褶状衬衫,衬衫最上方是最宽最时髦的黑色领带。这身打扮把他的体型衬托得极好。她转过身来时,他脸上现出一片淡淡的红晕,因为和女孩子在一起,他总是很腼腆。像许多腼腆的男人一样,他对像思嘉这样性情活泼、生气勃勃、总是无拘无束的女孩大为赞赏。过去她都只是客客气气地敷衍他,所以,她跟他打招呼时那种快乐、粲然的微笑以及伸到他面前的一双手,几乎使他的心脏都停止了跳动。
\par “哎呀,韩查理,你这潇洒的家伙!我敢打赌,你从亚特兰大一路到这来,就是为了让我伤心的吧!”
\par 查理激动得连说话都几乎结巴起来。他把她那温暖的小手握在自己手里,眼睛直视着那双欢呼雀跃的绿色眸子。女孩子老用这种方式和别的男孩子说话,可从来没对他说过。他一直都不知道这是为什么,可女孩子总是把他当小弟弟看待,对他很友好,但从来不费心去跟他调笑。他总是希望有女孩子和他打情骂俏,就像她们和那些不如他英俊、不及他富有的男孩玩闹那样。但这种情况偶尔发生在他身上时,他总是想不出来该说些什么,于是因自己哑口无言困窘得痛苦不堪。接着他就会彻夜不眠地想着自己本可以使用的生动迷人的言辞;但他极少能再获机会,因为女孩子们试过一两次之后就不再理他了。
\par 甚至和哈尼在一起,他也是与众不同、沉默寡言的,虽然没有明说他也知道,明年秋天他继承了财产时,他就要跟她结婚了。有时,他甚至有种有失风度的感觉,认为哈尼那卖弄风情和主人姿态并不完全是因为他的缘故才做出来的,因为她想男朋友都想疯了。他想,对任何给她机会的男人,她都会使出这套本事的。查理对和她结婚的前景并不感到激动,因为她激不起他身上任何爱得死去活来的浪漫情感,而他那些酷爱的书籍却使他确信,这些情感对一个爱人来说是恰如其分的。他一直在渴望着爱慕他的是个美丽漂亮、精神抖擞而又充满活力、调皮捣蛋的尤物。
\par 现在,郝思嘉居然跟他逗乐,说他让她伤心了!
\par 他试图想出些话来说,但什么话也想不出来,只好默默地暗自感谢思嘉,因为她一直叽叽喳喳地说个不停,这使他大为宽慰,因为他根本就没有必要说话了。这简直太令人不可思议了。
\par “哎,你就在这等我回来好了,我要跟你一块去吃烧烤。你可别跟别的姑娘去瞎混了,我的忌妒心可强得很呢。”这些令人不可置信的话从那两片鲜红的嘴唇里飞出来,飘到他耳里;说话时那张脸蛋现出两个酒窝,绿色双眸上墨黑的睫毛欢快而娴静地眨巴着。
\par “我不会的。”他终于设法透过气来,做梦都没想到她心里想的其实是,他看上去就像一头等着屠夫来屠宰的小牛犊一样。
\par 她用折扇轻轻敲了敲他的手臂,转过身走上楼梯,目光又一次落在那个叫白瑞德的人身上,他正独自一人站在离查理几英尺远的地方。显然他已经听到了全部对话,因为他正像只公猫一样对她邪恶地咧嘴笑着。他的视线也重新落在她身上,目光里完全没有她通常熟悉的那种淡漠之情。
\par “真是活见鬼!”思嘉愤愤不平地对自己说,用上了嘉乐最喜欢的诅咒词。“他看上去好像——好像他知道我没穿衬衫是什么样子的。”她甩甩头,走上楼梯。
\par 在卧室里放外衣披巾等东西的地方,她看到凯思琳·卡尔弗特正坐在镜子前打扮,咬着嘴唇以使嘴唇看上去更红润。她的腰带上别着新鲜的玫瑰花,这和她的脸颊非常相配,矢车菊般蓝色的眼睛因激动而眨巴着,就像在跳舞似的。
\par “凯思琳,”思嘉一边说着,一边试着把自己裙子的胸部拉上一些,“楼下那个叫白瑞德的讨厌的家伙是谁呀?”
\par “亲爱的,难道你不知道吗?”凯思琳兴奋地低声说道,一面留神着隔壁房间,因为迪尔西和卫家的嬷嬷们正在那聊天呢。“我简直无法想像有他在这,卫先生有何感想,他是到琼斯伯勒去拜访肯尼迪先生的——是有关买棉花的事——当然,肯尼迪先生只好把他带到这来了。他不能自己离开而扔下他不管。”
\par “他出了什么事了吗?”
\par “亲爱的,他一点也不受欢迎!”
\par “这是真的吗?”
\par “是真的。”思嘉默默琢磨着这些话,因为她过去从来没有和一个不受欢迎的人在同一个屋檐下呆过呢。这确实令人兴奋。
\par “他做错什么了吗?”
\par “噢,思嘉,他的名声是坏到极点啦。他名叫白瑞德,从查尔斯顿来的。他那些亲戚们倒都是为人极好的人,但他们连话都不跟他说。卡罗·瑞德去年夏天把有关他的事告诉我了。他跟她家没有任何亲戚关系,但他的什么事她都知道,其实每个人都知道。他曾被西点军校开除出来。真难以想像!那是由于做了什么坏事,连卡罗也不便知道。后来又出了他不肯跟一个女孩结婚的事。”
\par “请你跟我说说吧!”
\par “亲爱的,难道你一点都不知道吗?卡罗去年夏天全都告诉我了,如果卡罗的妈妈知道卡罗知道这事,她妈妈一定会没命的。是这样,这个白先生带了查尔斯顿的一个女孩坐着轻便马车出去兜风。我一直不知道这个女孩是谁,但我已经怀疑上某个人了。她不可能是个好姑娘,要不她不会在没人陪伴的情况下在下午很迟的时候还跟他出去。哦,亲爱的,他们几乎在外面待了一整夜,最后却走着回家来了,说是马跑了,并且把轻便马车给毁了,他们在树林里迷了路。嗯,你猜猜——”
\par “我不会猜。告诉我吧。”思嘉饶有兴致地说,希望听到最糟糕的结果。
\par “第二天他就拒绝跟她结婚!”
\par “哦。”思嘉说道,希望落空了。
\par “他说他没对她——哦——做过什么事,他不明白他为什么要跟她结婚。当然,她哥哥把他叫了出来,白先生说,他宁愿挨枪子也不愿和一个傻瓜结婚。他们于是进行了一场决斗,白先生把那女孩的哥哥打死了。白先生只好离开查尔斯顿,现在谁都不欢迎他。”凯思琳得意洋洋地结束了叙述,也结束得正是时候,因为迪尔西回到房间来查看她看管的衣服来了。
\par “她有没有怀上孩子呢?”思嘉在凯思琳耳边低声问道。
\par 凯思琳拼命摇头。“但她还是一样被毁了。”她倒吸了一口气。
\par “真希望我已经和希礼达成了一致意见,”思嘉突然想道,“他若不和我结婚,就不是个绅士。”但不知怎么的,对白瑞德拒绝和一个傻瓜结婚,她隐隐对他产生了尊重感。
\par  
\par 在屋子后面一丛高大的橡树的树荫里,思嘉坐在一张红木制成的高脚凳上,裙子如云的荷叶边和褶边把她包围在其中,脚上露出两英寸长的绿色摩洛哥舞鞋——一个淑女所能向别人显示的最大限度——在裙子底下若隐若现。烧烤野餐已经进入了高潮,温暖的空气中到处弥漫着谈笑声、银器和瓷器的碰撞声,还飘荡着烤肉浓浓的香味和卤汁的芳香味。时不时地,由于微风的风向改变,从长长的烤坑里吹来一股股烟,飘到人群中来,太太小姐们叫着假装表示很沮丧,用力扇着棕榈叶做的扇子。
\par 大多数年轻小姐都和男伴们坐在面朝桌子的长凳上,但思嘉意识到,在那里,一个姑娘只有两边可分别让一个男子就座,所以选择坐在旁边,这样她就可以让尽可能多的男人围在她身边了。
\par 那些已婚妇女坐在树枝搭成的凉亭里,她们黑色的衣裙在周围的色彩和欢快气氛中是礼貌而有教养的象征。主妇们不分年龄,总是和目光炯炯有神的姑娘们、小伙子们及周围的笑闹声分开,自成一群,因为在南方是没有老处女的。方家的老祖母自恃年高,明目张胆地打着饱嗝,年仅十七岁的艾丽斯·芒罗正拼命抑制着第一次怀孕带来的恶心反应。她们这群人从老到少,凑在一起没完没了地讨论家谱及助产问题,而这些问题便形成了这类聚会的极为令人愉悦的有益的话题。
\par 思嘉对她们投去蔑视的目光,觉得她们真像一群肥胖的乌鸦。结过婚的女人一点情趣也没有。她一点也没意识到,如果她和希礼结了婚,她就会自然而然地被归到凉亭里和走廊上,和那些稳重的主妇们坐在一起,穿着单调乏味的丝绸衣裙,这些衣裙就像她们本人一样既稳重又乏味,一点情趣和嬉闹劲都没有。就像许多女孩一样,她的想像力只能把她带到圣坛前,再也不往前走一步了。再说,她现在心里很不痛快,没心情去胡思乱想。
\par 她垂下眼睛,看着盘子里的食物,一点一点、动作优雅地嚼着一块已被敲扁的饼干,可一点食欲也没有。嬷嬷见了肯定会赞不绝口的。尽管她男朋友多得过剩,可她从来没有像现在这样难受过。连她也不明白是怎么回事,她昨晚的计划,在希礼这方面是完全失败了。她吸引了成打成打的男孩子,但没有把希礼吸引过来。昨天下午的恐惧又重卷而来,使她的心一会狂跳不已,一会又慢下来,脸色也一会红一会白的。
\par 希礼并没有试图加入围着她的这群人的行列。事实上,自从来到这以后,她就没有单独跟他说过一句话,除了第一次碰面时打个招呼外,连跟他说话的机会都没有。她走进后花园时,他走上前来欢迎她,但当时媚兰正挽着他的手臂,她的个头还不及他的肩膀高呢。
\par 她身材瘦小,体格虚弱,外表看上去就像个穿着母亲宽大、带裙环的裙子的孩子一样——她那羞涩、几乎可以说是害怕的神情,配上那双大而棕色的眼睛,又加强了这种印象的效果。她长着一头拳曲的黑发,被一丝不苟地梳平罩在发罩里,一根散发也没露出来,这堆黑色的头发加上长长的寡妇式的发髻,更衬出她那张心型的脸。她的颧骨太宽,下巴太尖,这是一张可爱但却又怯生生的脸,而且是普普通通、毫无特色的脸。再说,她又没有女性吸引人的那套技巧,好让看到她的人忘记掉她的大众化脸谱。她看上去——哦——像泥土一样简单平凡,像面包一样没什么害处,像泉水一样透明无色。然而,尽管她相貌平平,身材瘦小,但她的举止有种稳重端庄的样子,一般比她年长得多的人才会有这种神情,而它在年仅十七岁的她身上出现则是极为奇怪的。
\par 她穿着灰色的玻璃纱裙子,扎着樱桃色锦缎腰带,裙子翻卷的褶边掩饰了她那孩子般未发育成熟的身子。黄色的帽子配着长长的樱桃色帽带,把她米色的皮肤衬得闪闪发亮。镶着长长金边的略重的耳环从梳得整整齐齐、网在发罩里的头发边上垂挂下来,在她棕色的眼睛边晃来晃去。她的眼睛发出的光亮,就像是冬日里森林深处的池塘上,棕色的树叶从平静的水中发出的那种静止的光亮一样。
\par 她跟思嘉打招呼时,露出了羞涩的微笑。她恭维思嘉那绿色的裙子有多漂亮。思嘉因为太渴望单独和希礼说话,好不容易才勉强报以礼貌的回答。自那时起,希礼就一直坐在媚兰脚边的一张凳子上,和其他客人分开,静静地和媚兰说话,露出那种思嘉喜欢的、慢条斯理而慵懒的微笑。更糟糕的是,在他的微笑之下,媚兰的眼里露出了一丝亮光,以致连思嘉也只好承认,她看上去几乎可以说是很漂亮了。媚兰抬头望着希礼时,她那平淡的脸上神采奕奕的,就像内心燃着一团火似的。如果说一颗正在恋爱的心会从脸上表现出来的话,那韩媚兰此时此刻就把自己的心迹展露无遗了。
\par 思嘉试图把视线从这两人身上移开,可是她做不到。每看完他们一眼,她便加倍地和身边对她献殷勤的骑士们嬉笑打闹,放声大笑、说些莽撞的话,戏弄取笑别人,对他们的赞美之词摇头否认,直至耳环晃动不停,跳起舞来。她多次重复“胡说”这词,宣称他们说的话里没有一句是真话,发誓说她再也不相信男人们告诉她的任何话了。但希礼似乎一点也没注意到她。他只是抬头看着媚兰,继续说着话,媚兰则低头瞧着他,那表情流露出这么一个事实:她是属于他的。
\par 所以,思嘉非常难过。
\par 从外表看来,她是最没有理由难过的女孩了。无疑,她是野餐会上的王后,是大家注意力的中心。她在男人当中引起的轰动,加上其他女孩内心的怒火,若是在别的时候,那是会使她欣喜若狂的。
\par 韩查理因思嘉对自己的注意,胆子变得大了起来。他稳稳地坐在她右边,塔尔顿家的孪生兄弟俩合力要把他支开,他却不肯离开。他一手拿着她的扇子,另一手端着一盘连动都没动过的烧烤食物,固执地不和哈尼四目相对,而哈尼似乎都快要哭出来了。凯德懒洋洋地斜靠在她的左边,拉着她的裙子吸引她的注意力,眼里满含怒意地盯着斯图尔特。他和孪生兄弟俩的关系已经非常紧张,有了一触即燃的势头,双方已经言语粗鲁地口角过了。弗兰克·肯尼迪咋咋呼呼的,像是一只带鸡崽的母鸡,在橡树荫和桌子之间跑来跑去,取来美味可口的食物吸引思嘉,就好像是干这活的十几个仆人不在场似的。结果,苏埃伦的愠怒终于达到了极限,再也不能像淑女般尽力掩饰了,不禁对思嘉怒目而视。小卡丽恩可能都已经哭过了,尽管那天早晨思嘉用话语鼓励了她,可布伦特除了对她说“你好,西西”并拉了拉她的发带外啥也没做,把注意力全集中在思嘉身上了。平常,他极为和善,会用一种随意的敬重对待卡丽恩,让她感到自己好像长大了。卡丽恩暗地里梦想着有那么一天,自己能挽起头发、穿着长裙,把他当成正式男朋友来接待。可现在,似乎是思嘉已经拥有他了。芒罗家的姑娘们正掩饰着皮肤黝黑的方家男孩对她们的背叛带来的懊恼,可她们对托尼和亚力克斯站在那群人边上那副模样大为恼火。因为他们都在等候着,一旦有其他人站起来离开原位,他们便想千方百计去占一个靠近思嘉的位置。
\par 她们微微耸了耸眉毛,把对思嘉行为的不满传给海蒂·塔尔顿。给思嘉的评价也就只有“放荡”这个词了。三位年轻的小姐同时举起花边阳伞,说她们已经吃饱了,谢谢,然后挽着离他们最近的男人的手臂,娇嗔地吵着要去看玫瑰园、春天的景色及凉亭。这种适时的战略撤退被在场的一位女士和先生看在眼里。
\par 看到三个男人被拖离了仰慕她的魅力的行列,被迫去查看那些女孩子们从孩提时代起就再熟悉不过的界石,思嘉不禁笑出声来。她目光锐利地扫了希礼一眼,想看看他是否注意到了这一点。但他正把弄着媚兰腰带的末梢,抬头对着她微笑呢。痛苦折磨着思嘉的心灵。她觉得自己恨不得把媚兰那乳白色的皮肤抓出血来,从中得到快乐。
\par 当她把目光从媚兰身上移开时,她和白瑞德的目光对视了。他此时没有和别人混在一起,只是站在一边和卫约翰说着话。他一直在看她,当她看到他时,他放声大笑。思嘉有个颇为不安的感觉,觉得这个不受欢迎的男人是在场的人中唯一一个知道她野性十足的外表下隐藏着其他想法的人,而且,这使他可以讥讽她以获得快乐。她也可以带着快感把他的皮肤抓破呢。
\par “只要我能应付到下午,等这烧烤野餐结束的话,”思嘉想着,“那时所有姑娘们都得上楼去小睡一会,好在晚上能够精力充沛的起舞。我便待在楼下,和希礼说话。他一定已经注意到我今天有多吸引人了。”她又用另一个希望来抚慰自己:“当然,他得殷勤礼貌地对待媚兰,因为,她毕竟是他的表妹,而且她一点也不招人喜欢。如果他再不关照她,她就会成为受冷落的可怜虫了。”
\par 想到这里,她又重新鼓起勇气,加倍努力地引诱查理,他那发亮的棕色眼睛正热切地望着她呢。对查理来说,今天可是非同寻常的一天,就像梦境中的日子一样,他毫不费劲就爱上了思嘉。在这种新的情感面前,哈尼已经退到一片模糊不清的雾霾中去了。哈尼是只声音尖利的麻雀,而思嘉则是晶莹亮丽的蜂鸟。她取笑他,偏袒他,问他问题却又自己回答,这样,他什么话也不用说,却反倒显得很聪明。其他男孩都感到困惑不解,因她明显对他感兴趣而懊恼不已。因为他们都知道查理生性腼腆,就算连续说两个词都做不到。气氛分外紧张,仅仅出于礼貌,他们才没有把越来越大的火气发出来。每个人都是一肚子火,要不是希礼,这就该是思嘉明白无误的胜利了。
\par 最后一叉猪肉、鸡肉和羊肉都被吃完了,思嘉希望,该是英蒂站起身来建议太太小姐们到屋里去休息的时候了。已经下午两点了,太阳温暖地当空照着。但是,花了三天时间准备烧烤野餐的英蒂已经精疲力竭,此时,她正高高兴兴地坐在凉亭里,对着一个从费耶特维尔来的耳背的老绅士大声说着话呢。
\par 人们都露出了一种慵懒的困倦状。黑人们荡来荡去,拾掇着放食物的长桌。谈笑声已不及先前活跃了,这里一群、那里一堆的人们渐渐静下来。大家都在等着女主人宣布上午的活动到此结束。棕榈扇摇得越来越慢了,有几个老先生因天气闷热,再加上吃得太饱,已经在打盹。烧烤已经结束,正值天最热的时候,大家都愿意去休息休息。
\par 在上午的聚会和晚上的舞会之间这段空隙,他们似乎成了一个平静的群体。只有年轻的小伙子们还有那静不下来的精力,而不久前,他们就是把这种精力灌注到人群当中去的。他们在人群中从这里逛到那里,用软软的声音慢吞吞地说话,就像纯种雄马一样既漂亮又危险。大中午的,大家都感到很倦怠,可暗地里却隐藏着足以在一秒钟内坏到想杀人的那种脾气,而且那坏脾气很快便能发出来。男人和女人,他们都是既漂亮又野性十足,在他们愉悦的外表下都有点狂暴,只是较驯服而已。
\par 又过了些时候,太阳越来越热了,思嘉和其他人都再次把目光投向英蒂。谈话渐渐停止,在这间歇时,树林里的每个人突然都听到嘉乐用狂怒的口音说话的声音。他站在离野餐桌稍远的地方,正和卫约翰争得热火朝天。
\par “真是活见鬼,老兄!祈求能和北方佬和平解决吗?在我们炮轰了萨姆特堡的无赖以后?还能和平解决?南方必须用武力证明,它是不能被侮辱的,而且,它脱盟不是因为联邦政府的友善,而是出于它自身的力量!”
\par “噢,我的天哪!”思嘉想着,“他真这么做了!现在我们大家只好坐到半夜了。”
\par 一瞬间,懒洋洋的人群中那种困倦之态稍纵即逝,某种东西像电一样,在空气中迅速传播开来。先生们从长凳和椅子上一跃而起,用力地挥舞着手臂,大声嘶叫着以争得自己的声音能够盖过别人声音的权利。由于卫先生怕太太小姐们会厌烦,所以一整个早上都没谈论起政治和即将发生的战争。可现在嘉乐已经嚷出了“萨姆特堡”这几个字,在场的每个男人便都忘记了主人的告诫。
\par “当然,我们要打的——”“北方佬这些贼人——”“我们一个月内就能把他们消灭掉——”“哎,一个南方人可以消灭二十个北方佬——”“给他们一个教训,让他们不要忘得太快——”“和平解决?他们不会让我们和平的——”“不会的,看看林肯先生是怎么侮辱我们的特派员的!”“是的,他让他们闲荡了好几个星期——发誓说他要让萨姆特堡的军队撤离!”“他们要打仗,我们会让他们讨厌战争的——”在所有的声音中,嘉乐叫得最响。思嘉能听到的就只有被一遍又一遍叫嚷的“州权、上帝!”嘉乐过得可是愉快极了,但他的女儿可不愉快。
\par 脱盟,战争——这些字眼由于一再重复,思嘉早就对它们厌烦透顶了,但现在她恨透了说到这些字眼的声音,因为这些字眼就意味着男人们要几个小时站在那互相高谈阔论,而她就没有机会和希礼面谈了。当然,不会发生战争的,这些男人都知道这一点。他们只是喜欢谈话,喜欢听自己谈话而已。
\par 韩查理没有和其他人一起站起来。他发现自己相对来说是单独和思嘉待在一起,便把身体靠近些,低声向思嘉承认自己大着胆子新燃起的爱情之火。
\par “郝小姐——我——我已经决定,如果我们真的打起仗来,我就到南卡罗来纳州去,参加那里的部队。听说韦德·汉普顿先生正在那里组织骑兵部队,当然我要去和他在一起。他是个很出色的人,又是我父亲最好的朋友。”
\par 思嘉寻思着:“我该怎么做呢——欢呼三声吗?”因为查理的表情说明,他正向她透露他心中的秘密呢。她想不出来该说些什么,所以只是看着他,心想男人们怎么会这么蠢,居然会认为女人们会对这些事情感兴趣。他把她的表情当成是颇为吃惊之后又感到满意的表现,于是很快地、大胆地接着说下去——
\par “如果我去了——你——你会不会难过,郝小姐?”
\par “我一定会每天晚上把头埋在枕头里哭泣的。”思嘉说,意思是想让自己显得能说会道,但他只理解了这话的表层意思,高兴得脸都红了。她的手是藏在裙子的褶边里的,可他小心翼翼地把自己的手移到她的手上,抓住了它,完全被自己的大胆和她的默许给征服了。
\par “你会为我祈祷吗?”
\par “真是个傻瓜!”思嘉尖刻地想着,偷偷地向周围瞄了一眼,希望自己能从这种谈话中被解救出来。
\par “你会吗?”
\par “哦——会的,是真的,韩先生。至少每天晚上念三遍《玫瑰经》!”
\par 查理飞快地向周围看了一眼,倒吸了一口冷气,腹部的肌肉都僵硬了。他们几乎就是单独在一起了,他可能永远也不会再有这种机会的。即使上帝再送给他这么一个机会,可他也许会失去勇气的。
\par “郝小姐——我得告诉你些事。我——我爱你!”
\par “呣?”思嘉心不在焉地说着,却试图透过争论不休的男人们看到希礼坐在媚兰脚边和她说话的地方。
\par “是的!”查理低声说着,心里一阵狂喜,可她既没笑出声来,也没有尖叫或晕过去,他总是想像年轻的姑娘们在这种境况下是会这么做的。“我爱你!你是最——最——”他生平第一次有了说话的能力。“漂亮的女孩。在我认识的人中,你是最可爱、最善良的,你的举止是最可爱的,我全心全意地爱你。我不指望你会爱上像我这样的人,我亲爱的郝小姐。如果你能给我一些鼓励,我会做这世界上任何事来使你爱上我。我会——”
\par 查理停了下来,因为他想不出什么事情是很难完成的,可以真正向思嘉证明他对她的感情有多深,所以他只简单地说:“我要跟你结婚。”
\par 听到“结婚”这两个字,思嘉猛然回到现实中来。她一直在想着结婚,想着希礼,她恼怒看着查理,并没有把恼怒很好地掩饰起来。这个像小牛般的傻瓜为什么偏偏要在这个特别的日子把他的感情硬挤进来呢?今天她可是忧虑交加,都快要发疯了。她朝那棕色、恳求的眼睛望进去,却看不到一个初恋的男孩应有的风采、理想实现后的那种崇敬之情以及正像火焰一样从他身上一掠而过的幸福和温情。思嘉对男人们向她求婚的事已经习以为常了,这些人都比韩查理有魅力得多,而且也比他更有手腕,不会在这野餐会上提出求婚,此时的她心里有更重要的事情要做呢。她只看到一个二十岁的男孩,脸红得像甜菜根一样,看上去傻里傻气的。她真希望自己能够告诉他,他看上去有多傻。但是埃伦教她在这种紧急场合要说的话自动地溜到嘴边,长久以来的习惯培养的力量使她垂下眼睑,喃喃自语地说:“韩先生,你要我做你的妻子,你给我的这种荣幸我不是不知道,但这太突然了,我都不知道说什么好。”
\par 要消除男人的虚荣心,又让他对此留有希望,这方法是太好了。查理上钩了,好像这是个新的诱饵,他成了第一个吞食这诱饵的人。
\par “我会永远等下去的!除非你已经很确定,要不我不会要你跟我结婚的。郝小姐,请你告诉我,我至少可以有这种希望!”
\par “呣。”思嘉说着,锐利的目光却注意到,没有加入谈论战争的人的行列的希礼正抬头对着媚兰微笑呢。只要这个抓着她的手的傻瓜安静一会,也许她就可以听到他们在说些什么了。她必须听到他们在说些什么。媚兰到底跟他说了些什么,使他眼里露出了感兴趣的神情呢?
\par 她虽竖起耳朵,极力想听清楚他们的话,但查理的话却使她听不清楚了。
\par “哦,别出声!”她用嘘声制止他,捏了捏他的手,连看都不看他一眼。
\par 思嘉的冷淡使查理吃了一惊,起先也为此感到很不好意思,可后来看到她双眼盯着的是他的妹妹,不由得笑了。思嘉是担心别人会听到他的话。她生性害羞,怕难为情,万一这些话被别人听到,她会很苦恼的。查理感到心中陡然升起一股男性的激情,这是他从未体验过的,因为这是他平生第一次使一个女孩感到难为情。这是股令人陶醉的激情。他调整了一下脸上的表情,露出他想像中认为是漫不经心、根本无所谓的神情,只谨慎地回捏了思嘉的手一下,表明他早已是个老于世故的人,可以理解并且接受她的责备。
\par 她甚至连他捏了她一下都没感觉到,因为她可以清楚地听到媚兰那甜甜的声音,而这也是她最大的魅力所在:“恐怕我不能同意你对萨克雷先生作品的看法。他是个愤世嫉俗的人。恐怕他不是像狄更斯先生那样的绅士。”
\par 对男人说这种话,真是傻透了。思嘉心里想着,不禁松了一口气,几乎要笑出声来。咳,她至多不过是个女学者,而谁都知道,男人们对女学者是怎么看的。……要想让一个男人感兴趣,并且使他一直都有兴趣,办法就是谈论有关他的事情,然后慢慢把话题引到自己身上——接着便不改话题,一直谈下去。如果思嘉发现媚兰说这类话,她倒是有理由感到恐慌的,比如“你真是太了不起了!”或者“你怎么会想到这些事的呢?换了我,哪怕我想试着想一想,我的小脑袋瓜也会爆炸的!”可坐在那里的她,在身边坐着一个男士的时候,谈话却如此严肃,就像在教堂里一样。对思嘉来说,前途似乎更光明了。实际上,这光明的前途甚至使她神采飞扬的眼睛转向查理,纯粹是出于快乐地微笑着。看到她明显表示出对他的爱意,他不禁欣喜若狂,抓起她的扇子热情地替大扇起来,把她的头发都扇得凌乱地飘舞着。
\par “希礼,你还没发表你的高见呢。”吉姆·塔尔顿从大叫大嚷的男人堆中转过身来说道。希礼对媚兰说了声对不起,然后站起身来。那里的男人中谁都没有他那么英俊潇洒,思嘉看到他那若无其事的优美姿态,被阳光照得闪闪发亮的金发和胡子,心里不禁这么想。连更年长的人此时也都停下来听他说话。
\par “我说,先生们,如果佐治亚要参战,我一定会和它一起并肩作战的。要不我干吗要参加骑兵连呢?”他说。他灰色的眼睛睁得大大的,眼里懒洋洋的神情不见了,取而代之的是全神贯注的样子,这是思嘉从来没有见过的。“但是,和我父亲一样,我也希望北方佬能让我们和平解决,那就不会有什么战争了——”他笑着举起手,因为方丹家和塔尔顿家的男孩已经开始发出一片喧哗声了。“是的,是的,我知道我们被侮辱了,也被骗了——但是,如果我们处在北方佬的处境,要脱离联邦的是他们,那我们会怎么做呢?很可能也会这么做。我们也不可能喜欢这种情形的。”
\par “他又来了,”思嘉想,“老是把自己置于别人的境地。”对她来说,每个争论都只有一方是正确的。有时候,真是没法理解希礼。
\par “我们都别太头脑发热,也别打什么仗。世上大多数的痛苦都是战争引起的。而战争一旦结束,谁也不知道这些战争是怎么回事。”
\par 思嘉吸了吸鼻子。很幸运,在勇敢方面,希礼的名声是不可辩驳的,要不就有麻烦了。她正这么想的时候,响起了一连串不同意希礼的声音,既愤愤不平,又火冒三丈。
\par 凉亭底下,那位从费耶特维尔来的耳背的老先生用力打了英蒂一下。
\par “在吵什么呀?他们都在说些什么?”
\par “战争!”英蒂把两手捧成杯状凑在他耳边大声喊道。“他们要和北方佬打仗!”
\par “打仗,真的吗?”他大叫起来,手摸寻着手杖,猛地从椅子上站起身来。这么充沛的精力在他身上已经有好几年没见过了。“我来告诉他们有关战争的事吧。我参加过战争。”麦克雷先生不是经常有机会谈战争的事的,他的女性街坊邻里就是这么谐谑他的。
\par 他笨拙而快速地走到人群中,一边挥舞着手杖,一边大声叫嚷着。因为他听不见周围的声音,毫无疑问,他的声音很快便占有了整个领地。
\par “你们这些好战的年轻小伙子们,听我说。你们不会想打仗的。我打过仗,我知道这一点。我曾去参加过森密诺尔战争,还像个大傻瓜似的去参加了墨西哥战争。你们都不知道战争是什么样子的。你们以为战争就是骑着一匹漂亮的马儿,还有女孩子向你们直扔鲜花,像个英雄似的凯旋归来。可是,不是这样的。不是的,先生!打仗得挨饿,因在潮湿的地方睡觉,还要得麻疹和肺炎。如果没得麻疹和肺炎,那也会得肠胃病。是的,先生,战争使人得的肠胃病就是——痢疾以及诸如此类的——”
\par 太太小姐们都涨红了脸。麦克雷先生是个会使人想起较粗野的那个年代的人,就像方丹家的老奶奶和她那令人感到不好意思的大声打嗝的毛病一样,那是个大家都想忘记的年代。
\par “快去把你爷爷带回来。”老人的一个女儿对站在附近的一个年轻姑娘嘘声说道。“我说,”她对周围焦躁不安的主妇们低声说道,“他现在是日见日糟了。你信不信,就在今天早晨,他对玛丽说——而她还只有十六岁呢:‘我说,小姐……'”声音越来越小,变成了低语声。此时,那个孙女已经悄悄溜了出去,试图劝诱麦克雷先生回到树荫下的座位上。
\par 在树下瞎转的人群中,女孩子们激动地微笑着,先生们热情地谈论着,只有一个人似乎是平静如常的。思嘉的视线转到白瑞德身上,他正倚靠在一棵树上,双手深深地插在裤袋里。他单独一人站着,因为卫约翰已经离开他身边了。谈话越来越热烈,他却一言不发。剪得短短的胡子下,两片红润的嘴唇噘着,黑色的眼里隐隐现出一丝因感到有趣而露出的轻蔑之态——轻蔑,就像他是在听孩子们的自吹自擂一样。这是一种表示意见非常不一致的微笑。他静静地听着别人说话。此时,有着一头乱蓬蓬的红头发、两眼却炯炯有神的斯图尔特·塔尔顿正一再重复着下面的话:“我说,我们一个月内就能把他们全消灭掉!绅士们打起仗来总是比乌合之众更出色的。一个月——我说,打一仗——”
\par “先生们。”白瑞德用一种平平的声调慢吞吞地说道,这声音便证明了他是查尔斯顿人。他仍然倚靠在树上,没有改变姿势,也没有把手从裤袋里拿出来。“我可以说句话吗?”
\par 他的举止和他的眼睛一样带有某种轻蔑神态,这种轻蔑神态被一种礼貌神情掩盖着,不知怎的,也给他自身的举止蒙上了一丝嘲讽意味。
\par 人群都转过身去看着他,用一种对待外人所惯有的礼貌迎候他的话。
\par “你们这些先生们有没有人想过,梅森—迪克森线以南,一座大炮工厂都没有?南方的铸铁厂也少得可怜?还有毛纺厂、棉纺厂或是制革厂都一样?你们有没有想过,我们一艘战舰也没有,而北方佬的舰队一个星期内就可以把我们的港口轰得底朝天,我们也就没有办法再把棉花卖到国外去了?但是——当然——你们这些绅士们已经想到这些事了。”
\par “哦,他意思是说,这些男孩子都是一群傻瓜!”思嘉愤愤不平地想,一股热血涌上心头,使她双颊涨得通红。
\par 显然,她并不是唯一一个想到这一点的人,因为有几个男孩的下巴已经开始扬起来了。卫约翰随意却是迅速地回到说话的人身旁,似乎要让在场的所有人知道,这个人是他的客人,而且,在场的还有太太小姐们。
\par “我们大多数南方人的麻烦就在于,”白瑞德继续说下去,“我们要不就是走的地方不够多,要不就是从我们的旅行中获益不够多。哦,当然,你们这些绅士们走的地方都很多。可你们都看到了什么呢?欧洲、纽约和费城,当然,太太小姐们也去过萨拉托加(他向凉亭下的那群人微微行了个礼)。你们看到了旅馆、博物馆、舞会以及赌场。你们回到家里来,相信没有一个地方像南方这样。至于我,我生在查尔斯顿,但过去的几年中我一直待在北方。”他咧嘴笑了,露出洁白的牙齿,似乎他已意识到在场的每个人都知道他为什么不再住在查尔斯顿,而且,即使他们知道这一点,他也一点都不在乎。“我看到了许多你们全都没看到的东西。为了食物和几个美金,成千上万的移民都很乐意为北方佬打仗,而且,工厂、铸造厂、铁矿和煤矿——这些东西我们都没有。唉,我们就只有棉花、黑奴和傲气。他们一个月内就能把我们杀得精光。”
\par 有一会工夫,气氛极为紧张,但大家都沉默不语,一片寂然无声。白瑞德从上衣口袋里掏出一块上好的亚麻布手帕,悠闲地抽打着袖子上的灰尘。接着人群中响起了一片不祥的嘟哝声,凉亭底下也传来一阵嗡嗡声,非常清楚明白,就像是一个刚受到骚扰的蜂窝一样。尽管思嘉觉得双颊上还流动着愤怒的热血,但她注重实际的头脑里却萌生出这样一个想法,这个人说的话是对的,听起来也颇为在理。不错,她从来没见过工厂,或是知道有哪个人有见过工厂。但是,即使这是对的,他说这样的话也太没有绅士风度了——居然在大家都玩得很尽兴的聚会上这么说。
\par 低头垂眉的斯图尔特走上前来,身后跟着布伦特。当然,孪生兄弟俩很有教养,即使被激得气愤非凡,也不至于在烧烤野餐会上当众大吵大闹。同样,所有的太太小姐们也都很激动,也很高兴,因为她们能真正亲眼看见某个场景或是吵架场面的机会太少了。通常,她们都是从第三者那里听来的。
\par “先生,”斯图尔特闷声闷气地说,“你这是什么意思?”
\par 瑞德礼貌地看着他,眼里却带着讥讽的神情。
\par “我意思是说,”他回答道,“拿破仑说的——也许你听说过他吧?——有一次他说过:‘上帝是站在最强大的军队那一边的!'”说着他转身面对着卫约翰,真诚、礼貌地对他说:“你答应过要让我参观参观你的藏书的,先生。如果我现在要你带我去看,是不是太过分了?恐怕今天下午我就得早点赶回琼斯伯勒去,有点生意要我去打点。”
\par 他转过身来,面对人群,双脚咔嚓一声立正,像个知名舞蹈家一样鞠了一躬。对他这样一个身材高大的人来说,这样的举动显得优雅极了,但也显得傲慢极了,就像是打了别人一记耳光似的。然后他和卫约翰一起穿过草坪,一头黑发的脑袋在空中移动着,令人不安的笑声飘了过来,桌子边的人群都听见了。
\par 大家都吃了一惊,人群中一片寂静,接着便又响起了嘤嘤嗡嗡的声音。凉亭底下,英蒂有气无力地从座位上站起来,向正在生气的斯图尔特·塔尔顿走去。思嘉听不见她在说些什么,但她直看向他低垂着的脸的眼神给了思嘉某种像是受良心谴责的刺痛感。媚兰看着希礼的时候同样也有这种神情,只是此刻的斯图尔特没看到罢了。这么说,英蒂确实爱他。有一会,思嘉心想,一年前的政治集会上,她若没有公然和斯图尔特调情,他也许早就和英蒂结婚了。但是,紧接着那刺痛感便消失了,取而代之的是一种慰藉感。要是其他女孩没法留住自己的男朋友,那也不是她的过错。
\par 斯图尔特终于低头对英蒂笑了,这是一种非常勉强的笑,他还对她点了点头。很可能英蒂刚才一直在请求他不要跟着白瑞德去生事。树底下响起了一阵礼貌的骚动,客人们纷纷站起身来,拍着屁股上沾着的碎屑。已婚妇女们呼叫着奶妈和小孩,把成群的孩子召到一块,准备离开。一群群姑娘们也谈笑着开始向屋子走去,要到楼上卧室里聊聊天,睡个午觉。
\par 除了塔尔顿太太,所有的太太们都离开了后院,把橡树下的树荫和凉亭留给男人。她是被嘉乐、卡尔弗特先生和其他想从她那里得到给骑兵连的马匹的人留住的。
\par 希礼闲荡到思嘉和查理坐的地方,脸上露出若有所思又颇感有趣的微笑。
\par “他是个傲慢的魔鬼,对不对?”他朝白瑞德走去的方向看过去,说道,“他看上去像是波吉亚的一员\footnote{意大利一显赫家族。}。”
\par 思嘉迅速思考着,但记不起县里、亚特兰大或是萨凡纳有哪一家叫这个名字的。
\par “我不知道这些人。他是他们的亲戚吗?他们是谁?”
\par 查理脸上现出了奇怪的表情,他感到不可置信,同时又感到很不好意思,这些情感和心里的爱在打架。当他意识到对一个姑娘来说,可爱、温柔、漂亮就已足够,教育多少并不影响她的魅力时,爱便占了上风。他于是简练地回答说:“波吉亚一家是意大利人。”
\par “噢,”思嘉说着,失去了兴趣,“外国人。”
\par 她的脸漾着最迷人的微笑转而面对希礼,但出于某种原因,他并没有看她。他在看着查理,脸上既有理解的成分,又有些微的怜悯。
\par  
\par 思嘉站在楼梯平台上,小心翼翼地从楼梯扶手上往下面的过道里窥视着。过道里空无一人。楼上的卧室里传来没完没了的低声说话的嗡嗡声,此起彼伏的,不时被一阵阵笑声以及“哎,你没那么做,真的!”和“接下来他怎么说?”之类的话所打断。在六个大卧室里,姑娘们躺在床上和长沙发椅上休息。她们脱了衣服,褪下紧身胸衣,放下头发,垂至腰际。下午小睡一会是乡间的习惯,而在从一大早就开始直至以晚上的舞会告终的全天聚会中,这种休息就特别有必要。姑娘们会谈笑半个小时,然后仆人们会来把百叶窗关好。在温暖怡人、半明半暗的氛围中,谈话会渐渐变成低语声,最后归于一片宁静,只听得见轻柔、均匀的呼吸声。
\par 思嘉确定媚兰已经和哈尼及赫蒂·塔尔顿一起躺在床上后,她才一个人悄悄地溜到过道里,迈步走下楼梯。从楼梯平台上的窗户望出去,可以看到一群男人坐在凉亭下,端着高脚杯在喝酒。她知道他们会一直在那呆到傍晚。她在人群中搜寻着希礼的身影,可他没跟他们在一起。然后她侧耳听了听,听到了他的声音。正如她所希望的,他还在前面的车道上和要离开的太太和孩子们告别呢。
\par 她的心跳到了嗓子眼里,迅速走下楼梯。要是她遇上卫约翰先生,那该怎么办呢?别的姑娘们都在睡午觉,好使自己晚上看上去更漂亮些,她却在屋里溜来溜去,她有什么借口来解释自己的行为呢?哎,那也还是得冒冒险。
\par 她走到最底下一级楼梯时,听到仆人们在管家的吩咐下正在餐厅里走来走去忙活着,他们正把桌子和椅子移出去,为舞会作准备。宽大的过道对过是藏书室,门正开着,她悄无声息地快步走了进去。她可以在那一直等到希礼跟那些人道完别,在他进屋时把他叫住。
\par 藏书室的光线半明半暗的,因为窗帘已经拉上好挡住太阳光。这个昏暗的房间里,四周高高的墙上摆满了黑压压的书籍,这使她感到很沮丧。这不是一个她会选择来约会的地点,她原希望这次约会不会在这样的地方。这么多的书籍总是使她感到很沮丧,就像喜欢读很多书的人会令她感到同样沮丧一样。也就是说,所有这样的人——只有希礼除外。半明半暗中,沉重的家具耸立在她身边:座位很深、扶手宽大的高背椅,这是特为卫约翰家的男人们专制的,它们前面放着带天鹅绒跪垫的天鹅绒矮椅,这是给姑娘们坐的。长长的房间另一头的壁炉前面,放着一张有七条腿的沙发,那是希礼最喜欢的位子。它的靠背很高,就像一只高大的动物在睡觉一样。
\par 她关上门,只留下一条缝,努力使自己的心跳速度慢下来。她想确确切切地回忆起昨晚计划好要对希礼说的话,可却什么也想不起来。她是不是曾经想得好好的,现在却把它忘了呢——还是说,她只计划好让希礼对她说些什么呢?她记不起来了,不禁打了一个寒噤,心里吓了一大跳。如果心跳声不是在她耳朵里响个不停的话,她兴许能想出来要说些什么。但当她听到他最后道完别后走进前面的过道里时,她那已经跳得很快的心却跳得更快了。
\par 她所能记得的一切就是她爱他——爱他的一切,从他那满头金发、傲慢地扬着的头,到他修长的黑靴子,爱他的笑声,甚至在他的笑使她感到迷惑不解的时候也一样,还爱他令人茫然不解的沉默。噢,要是他此刻能走到她这儿来拥抱她,那该多好啊,这样,她就什么也不用说了。他应该爱她的——“也许,如果我祈祷的话——”她紧紧地闭着双眼,开始对自己嘀咕起来:“万福马利亚,无限仁慈——”
\par “哎啊,思嘉!”响起了希礼的声音,他的声音直传过来,在她耳边回响着,弄得她慌乱不已。他正站在过道里透过半开着的门往里窥视着,脸上带着疑惑的微笑。
\par “你在躲谁呀——查理还是塔尔顿兄弟?”
\par 她喘了一口大气。这么说,他已经注意到围着她转的那些男人了!他站在那眨着眼睛,全然不知她内心的激动,那可爱劲真是无法用言语来形容。她什么话也说不出来,伸出一只手把他拉进房间。他走了进来,感到困惑不解,但兴味十足的。她身上有种紧张感,眼里的神采是他过去从未见过的,即使在昏暗的光线下,他还能看到她双颊泛着两片玫瑰色红晕。他顺手带上门,拉住她的手。
\par “什么事?”他说,几乎是在喃喃低语。
\par 一接触到他的手,她便浑身颤抖起来。现在就要发生了,正如她所梦想的一样。上千种互不连贯的念头掠过脑际,可她却一个也抓不住,没法把它用言语表达出来。她只能浑身发抖,注视着他的脸。他干嘛不开口呢?
\par “什么事?”他重复了一遍,“有秘密要告诉我?”
\par 突然,她又有了说话的能力,埃伦几年来的教诲似乎突然一扫而空,嘉乐那爱尔兰血统里直截了当的个性从他女儿的嘴里表现出来了。
\par “是的——一个秘密。我爱你。”
\par 有一刻,他们都沉默不语,空气极为紧张,似乎两人都停止了呼吸。然后,她不再颤抖了,幸福和骄傲感贯穿了全身的血脉。她过去为什么没这么做呢?这比她所接受的教育——如何耍弄淑女般的花招要简单多了。接着,她的目光便捕捉住了他的视线。
\par 他的眼里有种大为惊愕的神情,既有不可置信,又有些别的东西——那是什么呢?对了,那天嘉乐心爱的猎马摔断了腿,他不得不要把它杀掉时,嘉乐也是这副样子的。她现在干吗要想到这些呢?多么傻气的想法。为什么希礼看上去这么怪,而且什么也不说?接着,他脸上就像是戴上一副训练有素的面具似的,很有风度地笑了。
\par “你今天在这里已经把每一个男人的心都收去了,你还觉得不够吗?”他说,声音里带着惯有的调笑、奉承的意味,“你是不是要把所有人的心都收去?行了,你一直就拥有我的心,你知道的。你已经开始懂事了。”
\par 一定有什么弄错了——全都弄错了!这不是她计划中的那种方式。她脑海里一再浮现的那些疯狂且支离破碎的想法中,有一个开始成形了。不知怎的——出于某种原因——希礼的表现似乎觉得她也只是跟他调情呢。但他知道不是这样的。她知道他是明白这一点的。
\par “希礼——希礼——告诉我——你应该——噢,你现在别取笑我了!我拥有了你的心了吗?噢,亲爱的,我爱——”
\par 他的手迅速盖住了她的嘴巴。面具被脱去了。
\par “你不该说这些话的,思嘉!你不该的。你不是认真的。你会为说了这些话而恨自己的,而且你也会因为我听了这些话而恨我!”
\par 她把头一扭,看着别的地方。一股暖流迅速流遍了她的全身。
\par “我不可能恨你的。我告诉你,我爱你,我也知道你一定在乎我的,因为——”她停下不说了。她从来没有在一个人的脸上看到过比这更痛苦的神情。“希礼,你在乎吗——你在乎的,对不对?”
\par “是的,”他阴沉着脸说,“我在乎。”
\par 假如他说他讨厌她,她也不会比听到这更惊恐。她拉了拉他的袖子,一句话也不说。
\par “思嘉,”他说,“我们不能离开这,忘掉我们曾经说过这些话好吗?”
\par “不,”她低声说道,“我忘不了的。你这是什么意思?你难道不想跟我——跟我结婚吗?”
\par 他回答道:“我要跟媚兰结婚了。”
\par 不知怎的,她发现自己坐在低矮的天鹅绒椅子上,希礼则坐在她脚边的跪垫上,把她的两手紧紧地握在自己的手里。他在说话——可这些话却是毫无意义的字句。她的大脑一片空白,仅仅几分钟前还在她脑海里翻江倒海的所有想法,此刻却无影无踪了。他的话什么印象也没给她留下来,就像打在玻璃上的雨一样。这些话直往这根本听不进任何东西的耳朵里灌,语速很快,温柔体贴,又充满怜悯,就像个父亲对受伤的孩子说的话。
\par 媚兰的名字唤回了她的意识,她定定地看着他那水晶般的灰色眼睛。她从这双眼里看到了一直使她感到困惑不解的那种冷漠神情——和自己恨自己的神态。
\par “父亲今晚就要宣布订婚的事了。我们很快就会结婚。我应该早点告诉你的,但我以为你知道呢。我以为每个人都知道——知道好几年了。我做梦也没想到你——你有这么多男朋友。我以为斯图尔特——”
\par 她身上慢慢开始恢复了生气、感情和理解力。“但你刚才还说你在乎我的。”
\par 他温暖的双手把她的手都握痛了。
\par “亲爱的,你要让我说出些会伤害你的话来吗?”
\par 她的沉默逼着他说下去。
\par “我怎么才能让你明白这些事呢,亲爱的?你又年轻又不爱动脑筋,你不知道结婚意味着什么。”
\par “我知道我爱你。”
\par “像我们这样很不一样的人,要使婚姻成功,光有爱是不够的。你会想要一个男人的全部,思嘉,他的身体、他的心、他的灵魂以及他的思想。而如果你得不到这些,你就会很痛苦。而我不能给你我的一切。我也不能给任何人我的一切。我也不想要你的所有思想和灵魂。那样你就会受到伤害,然后你就会渐渐地转而恨我——非常非常地恨我!你会恨我读的书和我喜爱的音乐,因为它们使我离开了你,可你是一刻也不会答应的。而我——也许我——”
\par “你爱她吗?”
\par “她很像我,我们有部分血统是一样的,而且我们能互相理解。思嘉!思嘉!我难道不能使你明白,除非两个人是同类人,要不婚姻是不可能平安无事的?”
\par 也有其他人说过这句话:“一个人应该和同类人结婚,否则不会幸福。”谁说过呢?她听到这句话以后,似乎已经过去上百万年了,但这话还是没什么意义。
\par “但你说过你在乎的。”
\par “我不该这么说的。”
\par 她头脑里有一股火慢慢腾起,愤怒开始把其他任何事都抛置脑后。
\par “哦,可你说了,你真是无赖到家了——”
\par 他的脸都白了。
\par “我说了,我当时真是个无赖,因为我要跟媚兰结婚了。我对你做错了事,对媚兰错得更厉害。我不该说的,因为我知道你不会明白的。我怎么能够做到不在乎你呢——你对生活充满激情,而这正是我没有的。你敢爱敢恨,爱得疯狂,恨得切齿,而这些于我是不可能的。哦,你就像火、风和一切野性十足的东西一样有力,而我——”
\par 她想到了媚兰,似乎突然间看见了她静静的棕色眼睛,带着那种远离现实的神情,戴着镶黑色花边的露指长手套的那双安分的小手,还有她那温和而默不吭声的性格。接着,她的愤怒爆发了,这股愤怒和驱使嘉乐去杀人、促使其他爱尔兰祖先去做使他们掉脑袋的事情的愤怒同出一辙。罗比亚尔家族的人能够以全然的沉默来忍受这个世界可能出现的任何情形,可现在,她却没有一丝这种良好血统的特质。
\par “你干嘛不早说,你这胆小鬼!你害怕跟我结婚!你宁愿和那个愚蠢的小傻瓜生活在一起,她除了会说‘是的’或‘不是’外就根本开不了口,还只会养一群像她一样说话拐弯抹角的小鬼头!为什么——”
\par “你不该这么说媚兰!”
\par “我不该?操你妈!你是谁,要你来告诉我我不该?你这懦夫,你这无赖,你这——你使我相信你会跟我结婚——”
\par “公平一点,”他申辩着,“我曾——”
\par 她可不要什么公平,虽然她知道他说的是事实。他从来未跨越过跟她的友情界限。想到这一点,她心里又升起了新的怒意,这是自尊心和女性的虚荣心受到伤害而引起的怒意。她在追他,而他却一点都不接受。他居然更喜欢一个像媚兰那样脸色苍白的小傻瓜,而不要她。噢,要是她接受了埃伦和嬷嬷的训诲、一点也不向他透露她喜欢他,那就好多了——任何事情都比面对这令人难堪的羞耻要强得多!
\par 她一跃而起,双手紧握着。他也站起身来,身材比她高出许多,脸上满是无声的苦痛,就像一个被迫面对痛苦现实的人一样。
\par “我到死也会恨你的,你这无赖——你这卑鄙小人——卑鄙小人——”她要说的是什么字眼呢?她想不出足够粗鲁的字眼来了。
\par “思嘉——请——”
\par 他向她伸出手去,可就在他这么做时,她却用尽全力甩了他一巴掌。啪的一声,在这平静的房间里就像鞭子的声音一样。突然间,她的愤怒消失了,心里只有孤寂和凄凉。
\par 她的巴掌在他苍白、疲倦的脸上留下了鲜红的手指印。他什么也没说,把她软弱无力的手放到嘴边吻了吻,然后,没等她重新开口说话便离开了,随手轻轻地关上了门。
\par 她颓然坐下,盛怒之下做出的举动使她双膝发软。他走了,可他那张被打的脸至死也会留在她的记忆里,使她不得安宁。
\par 她听见他轻轻却又沉闷的脚步声由近而远,渐渐消失在长长的过道里,她所有举动的后果也展现在她面前。她永远永远地失去他了。他从现在起就会恨她了。每次一见到她,他就会想起,在他一点鼓励也没给她的情况下,她是怎么主动向他示爱的。
\par “我的境遇跟卫哈尼的一样糟。”她突然这么想到,一边还想起每一个人(尤其是她自己)是如何带着轻蔑的态度嘲笑哈尼先前的行径的。她好像看见了哈尼挽着男孩们的胳膊时别扭地扭动着的身子,听到了她咯咯的傻笑声。这一想法刺激着她,使她重新生起气来,气自己,气希礼,气整个世界。因为她恨自己,所以她也恨他们所有的人,带着十六岁时的初恋遭到挫败和羞辱的怒意去恨他们。她的爱里只融进了一丝真正的温柔。大多数时候,这都是出于虚荣以及对自己的魅力充满自信、洋洋自得才融进去的。现在,她已经失去了,比这种失落感更甚的是另一种恐惧感,她担心自己当众出了洋相。她的洋相会不会比哈尼的更明显呢?大家都在嘲笑她吗?想到这里,她浑身不禁开始发起抖来。
\par 她的手放下时碰到了在旁边的一张小桌子,手指摸到了一个陶瓷玫瑰花钵,上面有两个小天使在傻笑着。房间里静如止水,她几乎要尖叫出来,打破这种沉静。她得做些什么,要不她就要疯了。她一把抓起花钵,恶狠狠地朝房间对过的壁炉摔过去。花钵擦过高高的沙发椅背,摔在壁炉架上。随着一小声脆响,花钵四分五裂。
\par “这,”沙发深处传来了一个声音,“太过分了。”
\par 从来没有什么东西比这声音更令她吃惊、更令她害怕的了。她顿时嗓子眼发干,一句话也说不出来。她抓住椅背,双膝却在发软。这时,躺在沙发上的白瑞德站起身来,用夸张的礼貌态度向她鞠了一躬。
\par “我的午睡居然被这被迫洗耳恭听的插曲打扰了,这已经糟透了,可为什么我的生命还得受到威胁呢?”
\par 他是活人。不是鬼魂。但是,圣人保佑我们,他什么都听到了!她使足浑身的力气,装出一副尊贵的样子来。
\par “先生,你应该让别人知道你在这里。”
\par “真的吗?”他露出洁白的牙齿,大胆的黑眼睛看着她直笑。“可你才是入侵者呢。我被迫留下来等肯尼迪先生,因为我感到自己在后院也许不受欢迎,我便考虑得周到一些,让不受欢迎的自己到这来,我还以为在这不会有人打扰我呢。可是,唉!”他耸了耸肩,轻声笑了起来。
\par 一想到这个粗鲁、傲慢的男人听到了一切——听到所有那些话,而现在的她是宁愿死也不愿把它们说出口的。想到这里,她的情绪又开始坏起来。
\par “偷听者——”
\par “偷听者经常听到非常有趣、非常有启发性的话,”他咧嘴笑了,“从长期偷听的经验中,我——”
\par “先生,”她说,“你真不是个君子!”
\par “非常恰当的说法,”他轻松地回答说,“而你,小姐,你也不是淑女。”他似乎觉得她很有趣,因为他又低声笑了起来。“在说过我刚才无意听到的话,做过我无意看到的事后,谁也没法再做个淑女了。然而,对我来说,很少有淑女是富有魅力的。我知道她们在想些什么,但她们从来就没有勇气或教养说出她们在想的东西来。这样,久而久之,就成了令人厌烦的人了。可你,我亲爱的思嘉小姐,却是个富有罕见的活力的女孩,这活力很是令人钦佩,我在此向你致敬了。我无法理解那儒雅的希礼先生究竟有什么魅力能吸引你这么一个性情暴躁的女孩。他应该跪下双膝感谢上帝,能有你这么一个有——他是怎么说的来着?——‘生活激情’的女孩,可是他是个没什么活力的可怜虫——”
\par “你连给他擦靴子都不配!”她愤怒地大叫起来。
\par “你这一辈子都要恨他了!”他在沙发上坐下,她又听到了笑声。
\par 如果她能把他杀了,她也会这么做的。可与此相反,她尽可能地收起自己的尊严,走出房间,随手把厚重的门砰的一声带上了。
\par  
\par 她飞快地走上楼梯,来到楼梯平台时,她觉得自己都要晕过去了。她停了下来,两手抓住扶手,由于愤怒、羞辱、尽力,心跳得特别快,好像都要绷破紧身胸衣跳出来一般。她试图深吸几口气,但嬷嬷给她系得太紧了。如果她真晕倒了,他们在这平台上发现了她,他们会怎么想呢?噢,他们什么都想得出来,希礼、那可恶的白瑞德,还有那群忌妒心强得很的讨厌的姑娘们!她生平第一次希望自己也像其他女孩一样随身带着嗅盐\footnote{碳酸铵和香料的混合物,可用来治疗昏厥和头痛等症。},可她从来就没有过一个嗅盐盒。她总是为自己从不头晕而自豪的。现在,她绝对不能让自己晕倒!
\par 慢慢地,不适感开始消失了。再过一会,她就会没事的,然后她就可以悄悄地溜进紧连着英蒂的房间的小梳妆室,解开紧身胸衣,轻手轻脚地到正在睡觉的女孩们身边的一张床上躺下来。她努力使心平静下来,使脸上的表情更加镇定自若,因为她知道,她现在看上去一定像个疯女人。如果哪个女孩还没睡着的话,她们就会知道有什么事不对劲了,可谁也不能、不能知道曾发生过什么事。
\par 从平台上宽大的凸窗望出去,她可以看到,在树底下和凉亭里的阴凉处,先生们还在椅子上懒洋洋地或躺或坐。她多忌妒他们哪!做个男人多好,从来就不用去经受她刚刚经历过的痛苦!在她两眼发热、头昏眼花地站在那看着他们时,她听到屋子前面的车道上传来一阵急促的马蹄声、沙砾飞溅的声音以及有人激动地向一个黑奴问话的声音。沙砾声又响了起来,一个男人骑着马的身影出现在她视线里。他穿过碧绿的草坪,直向树底下慵懒的人群奔去。
\par 是个迟到的客人;可他为什么骑着马穿过草坪呢?这可是英蒂引以为荣的东西呢。她认不出这人是谁,但他飞身下马,一把抓住卫约翰的胳膊时,她可以看出,他身上到处洋溢着激动之情。人群向他围拢过去,高脚杯和棕榈叶扇子被扔在椅子上和地上。尽管离得很远,她还能听到喧闹声、问话声、叫喊声,感觉到男人们身上有一种狂热的紧张感。接着,在混乱的嘈杂声中响起了斯图尔特·塔尔顿兴高采烈的叫喊声:“噢——哎——喂!”就好像他在猎场上一样。她第一次听到了南方反叛者的呼喊声,可她却不知道。
\par 她正观望着,看到塔尔顿家的四个男孩,接着是方丹家的男孩离开了人群,开始奔向马厩。一边跑,一边还叫喊着:“吉姆斯!你,吉姆斯!快给马上好鞍!”
\par “有人的家起火了。”思嘉想。可不管有没有起火,她的事便是在被别人发现以前回到卧室去。
\par 她的心现在已经平静些了。她蹑手蹑脚地走上楼梯,来到静悄悄的过道里。一股宜人的困倦笼罩着屋子,好像它也跟姑娘们一样在轻松适然地睡大觉一样,到了晚上才会音乐弥漫、烛光点点,把美丽全然展示在人们面前。她小心翼翼地把梳妆室的门推开,悄悄溜了进去。她手背在身后,还抓着门把,却听到卫哈尼的声音从对面通往卧室的门缝里传了出来。声音很低,几乎就是耳语声。
\par “今天,思嘉的行为已经放荡到一个姑娘所能表现的极限了。”
\par 思嘉觉得自己的心又开始狂跳起来,她无意识地把手捂住心窝,就好像她要抓住它,使它平静下来似的。“偷听者经常会听到非常有启发性的话”,记忆中的话冒了出来。她要不要再溜出去呢?还是让她们知道她在这里,好让哈尼尴尬万分呢?因为这也是她罪有应得。但接下来的声音却使她停了下来。听到媚兰的声音,就是一队骡子也没法把她拉走了。
\par “哦,哈尼,别这样!别这么不友好。她只是生气勃勃、性情活泼罢了。我当时倒觉得她极有魅力呢。”
\par “噢,”思嘉心里想着,指甲都抠进紧身上衣里去了,“那个说话拐弯抹角的小傻瓜还为我说话呢!”
\par 这比哈尼那明目张胆的恶毒还难以忍受。除了她的母亲以外,思嘉从未信任过别的女人,也不相信她们除了私心之外还能有别的动机。媚兰知道她已经安全稳妥地拥有希礼了,所以能够表现出这样的基督精神。思嘉觉得,这正是媚兰夸耀自己胜利的方式,同时又能赢得心眼好的美誉。思嘉和男人谈论别的女孩时也经常使用同样的伎俩,要让愚蠢的男人相信她心地善良、毫无私心,这方法从来就没有失败过。
\par “哎,小姐,”哈尼刻薄地说,声音也提高了,“你一定是眼瞎了。”
\par “别说了,哈尼,”萨莉·芒罗嘘声说道,“全屋子的人都会听到你说话的!”
\par 哈尼放低了声音,却还继续说下去:
\par “我说,你看到她是怎样和能到手的每一个男人调情的吗——连肯尼迪先生也不放过,而他是她亲妹妹的男朋友。我从没见过像她这样的人!毫无疑问,她还在追查理。”哈尼不自然地咯咯笑出声来。“你知道,查理和我——”
\par “你是认真的?”几个声音在激动地低声问道。
\par “哦,别告诉任何人,姑娘们——还没呢!”
\par 咯咯咯的笑声更多了,有人在挤哈尼,弄得床上的弹簧叽叽作响。媚兰在嘟嘟哝哝地说,哈尼若能成为她的嫂嫂,她不知会有多高兴。
\par “哎,思嘉要是成了我的嫂嫂,我就会不高兴了。要说我曾经见识过放荡的女孩的话,她就是一个,”传来了赫蒂·塔尔顿痛心的声音,“但她实际上就等于和斯图尔特订婚了。布伦特说她根本不在乎他,可是,当然,布伦特也迷恋她呢。”
\par “如果你们问我的话,”哈尼神秘兮兮地强调说,“她真正在乎的人只有一个。那就是希礼!”
\par 低语声顿时混杂在一起,有询问的,有打断别人说话的,思嘉因恐惧和羞辱而感到全身发冷。哈尼是个笨蛋,对于男人,她只是个傻瓜、蠢蛋,但她对其他女人有一种女性的本能,这点思嘉低估她了。在图书室里跟希礼和白瑞德在一起时所蒙受的屈辱和受伤的自尊都是令人烦恼的事。男人在严守秘密方面是值得信赖的,即使像白瑞德这样的人也一样,但有卫哈尼像猎犬一样在猎场上狂吠不已,六点以前,全县的人就都会知道这件事了。就在昨天晚上,嘉乐还说过,他不想让全县的人嘲笑他的女儿呢。现在他们会怎样嘲笑她呀!冷汗从她的腋窝顺着肋骨往下直流。
\par 媚兰很有分寸、平静而略带责备的声音盖过了其他人的声音。
\par “哈尼,你知道不是这样的。这也太不友好了。”
\par “真是这样的,梅利\footnote{梅利,即媚兰的爱称。}。你总是忙着在人们身上寻找优点,而他们实际上却是没有这些优点的。要是你没有这么做的话,你就会看明白了。若确实是这样,我也很高兴。这是她活该。郝思嘉所做过的事无非就是制造事端,试图把别人的男朋友抢过来。你知道得很清楚,她从英蒂手里抢走了斯图尔特,自己却不想要他。而今天,她还试图抢走肯尼迪先生,还有希礼和查理——”
\par “我得回家去!”思嘉想,“我得回家去!”
\par 要是她能像变戏法似的被送回塔拉,回到安全之地去,那该多好啊。要是她只跟埃伦在一起,只要看着她,拉着她的裙子,伏在她的膝上哭着把一切都告诉她,那又有多好啊。如果她再听到一个字,她就会冲进去,把哈尼那凌乱而苍白的头发成把成把地扯下来,并且当面啐韩媚兰一口,就为了她显示了她那自以为是的宽厚仁慈。但她今天已经表现的够普通的了,甚至像白人穷鬼一样——这也正是她的所有烦恼所在。
\par 她把手紧紧地压在裙子上,这样它就不会发出窸窣的声音了,然后像头动物一样悄悄退出去。“家”,她一边想着,一边飞快穿过过道,经过紧闭着的门和静悄悄的房间门口,“我必须回家去。”
\par 她已经到了前面的游廊上,这时,一个新的想法突然使她停了下来——她不能回家去!她不能逃跑!她必须熬过这一切,忍受姑娘们的恶意和怨恨以及她自己的屈辱和伤心。逃跑只会给她们徒添向她进攻的弹药。
\par 她握紧拳头,一拳砸在身旁高大、白色的柱子上,希望自己是大力士参孙,这样她便能够推倒十二棵橡树的所有建筑,毁灭里面的每一个人。她要让他们后悔。她要给他们点颜色瞧瞧。她还不太清楚该怎样给他们点颜色瞧瞧,但不管怎样,她得这么做。他们伤害了她,她要把他们伤得更深。
\par 这一刻,本来的希礼已经被抛至脑后。他已经不是她爱着的高挑、慵懒的男孩,而是卫约翰一家的一个部分、一群人中的一个。十二棵橡树,全县的人——她恨他们所有的人,因为他们会嘲笑她。年方十六的人,虚荣心比爱还更强,在她的胸腔里满是仇恨,再也没有其他情感的位置了。
\par “我不回家,”她寻思着,“我要待在这,我要让他们后悔。而且我决不告诉妈妈。不,我谁也不告诉。”她鼓起勇气回到屋里,打算重新爬上楼梯,到另外一间卧室去。
\par 她转过身时,看到查理从长长的过道另一头走进屋子。看到她,他快步朝她走来。他头发蓬乱,激动得整张脸就像天竺葵一样。
\par “你知道发生什么事了吗?”还没走到她面前,他就大叫起来。“你听说了吗?保罗·威尔逊刚刚从琼斯伯勒骑马带来的消息!”
\par 他顿了顿,走到她面前,上气不接下气的。她一言不发,只是盯着他看。
\par “林肯先生已经招募人了,士兵——我指的是自愿者——他们已有七万五千人了!”
\par 又是林肯先生!男人们难道从来不考虑考虑真正重要的事情吗?这里这个傻瓜居然指望她伤心欲碎、简直是身败名裂的时候会对林肯先生的胡闹激动万分。
\par 查理凝视着她。她的脸像白纸一般白,眯着的眼睛像祖母绿一样闪着光。他从来没在任何女孩的脸上看到这么大的火气,也没见过谁的眼里发出过这种光彩。
\par “我太笨了,”他说,“我应该委婉一些告诉你的。我忘了太太小姐们是很脆弱的。对不起,我让你不开心了。你不会晕倒吧,对不?要不要我给你拿杯水来?”
\par “不用。”她说,硬挤出一丝别扭的微笑。
\par “我们到长凳上坐下好吗?”他问,挽住她的胳膊。
\par 她点了点头。他小心地扶着她走下屋前的台阶,领着她穿过草地,来到前院那棵最大的橡树下的铁制长凳边。“女人真是又脆弱又娇嫩,”他心里想,“只要一提到战争和艰难境况,就能使她们晕过去。”这个想法使他觉得自己男子汉气概十足,扶着她坐下时也就加倍地轻柔。她神情古怪地看着周围,苍白的脸上有一种野性的美,这使他的心跳都加快了。会不会是他可能去参战这个想法导致她这么悲痛呢,这可能吗?不可能,相信这点也未免太自负了。但她干吗这么奇怪地看着他呢?她找绣花手帕时双手又为什么会颤抖呢?还有她那浓密乌黑的睫毛——它们正不停地一张一合的,就像他读过的浪漫故事中女孩子的眼睛一样,带着羞怯和爱意在眨动着。
\par 他清了三次喉咙想说话,但每次都没说出口。他垂下了眼睛,因为她绿色的双眸跟他的眼睛对视时目光非常锐利,就好像她没有在看他似的。
\par “他很有钱,”她迅速思考着,一个想法和计划掠过她的脑际。“他也没有父母亲会烦我,又住在亚特兰大。如果我马上和他结婚,这会让希礼看到我一点也不在乎他——我只是跟他调情而已。这还会使哈尼寻死觅活的。她再也找不到别的男朋友,大家会当着她的面笑得死过去。而这也会伤到媚兰,因为她太爱查理了。这还会使斯图尔特和布伦特伤心——”她并不太明了自己为什么想伤害他们,只知道他们有恶毒的妹妹,这是原因之一。“我可以坐着豪华的马车回到这来做客,又能有很多漂亮的衣服和自己的房子,到时候他们全都会难过的。他们就再也不会笑话我了。”
\par “当然,这也就意味着战争了,”又尴尬地努力过几次后,查理终于说出话来,“但你别发愁,思嘉小姐,一个月内就会结束的,我们要打得他们鬼哭狼嚎的。真的,小姐!鬼哭狼嚎!说什么我也不会错过这次机会的。恐怕今晚不会开什么舞会了,因为骑兵连要在琼斯伯勒集合。塔尔顿家的男孩已经去传递消息了。我知道太太小姐们会感到失望的。”
\par 她说“噢”,还想他说些更好的消息,但这已经够了。
\par 她开始平静下来,慢慢恢复了理智。她所有的情感都似蒙上了一层严霜,她认为自己再也不会感受到任何温暖的东西了。干吗不接受这个英俊、羞涩的男孩呢?他并不比别的人差,何况她也不在乎。不,她再也不会在乎什么事了,就算她活到九十岁,她也不会在乎什么了。
\par “我现在还不能决定,是去参加韦德·汉普顿先生的南卡罗来纳军团呢,还是去参加亚特兰大城卫队。”
\par 她又说了声:“噢。”他们的眼睛又对视了,她那眨动的睫毛成了毁灭他的祸根。
\par “你会等我吗,思嘉小姐?只要知道你在等着我,直到我们把他们彻底消灭掉,这——这简直是太棒了!”他屏住呼吸等着她说话。看着她嘴角两片嘴唇噘着的样子,他第一次注意到了这嘴角通常看不到的部分,心想要是能吻吻它,那将意味着什么呀。她那因汗湿而变得黏糊糊的手掌悄悄地伸到他手里。
\par “我不想等。”她说,眼睛似蒙上了一层面纱。
\par 他坐在那抓着她的手,嘴巴张得老大。思嘉的眼睛从睫毛下向上看着他,心里很超脱,心想他看上去就像一只被鱼叉叉住的青蛙。他结结巴巴地开口说了好几次,却又闭上嘴不说了,然后又张嘴欲说点什么,脸上又泛起了天竺葵般的色彩。
\par “你会爱我,这可能吗?”
\par 她一句话也没说,只是低头看着自己的大腿,查理再次又狂喜,又尴尬的。也许男人是不应该对女孩问这样的问题的。也许要她这么一个少女回答这样的问题是不合适的。查理过去从来没有过这种勇气,能使自己处于这样的境地,所以一时不知所措,不知该怎么做才好。他真想大喊大叫、放声歌唱,去亲吻她,在草地上欢呼雀跃,然后跑去告诉每一个人,不管是黑人还是白人,告诉他们,她爱他。但他只是紧紧握着她的手,直到把她的戒指压进肉里去。
\par “你会很快跟我结婚,对吗,思嘉小姐?”
\par “呣。”她说,手指拨弄着裙子上的一个褶皱。
\par “我们要不要和媚兰的婚礼同时举行——”
\par “不。”她很快说道,眼睛望着他,一副不祥的神情,发出隐隐约约的光。查理又一次意识到自己又犯了一个错误。当然,女孩子总是想自己单独举行婚礼的——不愿跟别人分享这种荣耀。她对他的严重错误忽略不顾,真是太仁慈了!要是现在是晚上,他能受到黑夜的鼓舞吻她的手,说些他早就想说的话,那该多好啊。
\par “我什么时候可以去跟你的父亲提亲呢?”
\par “越快越好。”她说,同时希望他会松手,把似要把她的戒指压碎的压力解除,而不用等她开口叫他这么做。
\par 他跳了起来,有一会,她都认为他会不顾身份欢蹦乱跳呢。他容光焕发地看着她,一颗纯洁无邪的心从眼里显露无遗。她过去从来没见过别人用这种眼神看过她,而且再也不会有别的男人这么看她了,但在这种心不在焉的奇怪心境下,她认为他看上去像头小牛犊。
\par “我现在就去找你的父亲,”他说,满脸都是笑,“我没法再等了。你能让我对你说声抱歉吗——亲爱的?”这爱称说出来很不容易,但一旦说出口,他便高兴地又重复了一遍。
\par “可以,”她说,“我就在这等着。这里很凉快,舒服极了。”
\par 他穿过草坪,在房子周围不见了。她则独自一人坐在沙沙作响的橡树下。男人们骑着马从马厩里鱼贯而出,黑人奴仆紧紧跟在他们的主人身后。芒罗家的男孩飞奔而过,手里挥着帽子,方丹家和卡尔弗特家的则叫喊着向路上飞奔而去。塔尔顿家的四个男孩在草坪对过纵马经过她面前,布伦特大声喊道:“妈妈要把马给我们了!噢——哎——喂!”草皮被马蹄卷起,他们离开了,又把她独自一人留在那。
\par 白色的屋子前,高高的柱子伫立在她面前,似乎要带着尊贵、冷淡的态度离她而去。现在这里再也不会是她的房子了。希礼永远不会把她当成新娘抱过门槛了。噢,希礼,希礼!我都做了些什么呢?在心灵深处,她的心受到受伤的自尊和冷漠的实用心理的层层重压,那里有某种东西在撕咬着她痛苦的心。一种成人的情感正在生成,这比她的虚荣心和固执的自私心理还更强烈。她爱希礼,她知道她爱他。此时此刻,看到查理消失在弯弯曲曲的砾石铺筑的人行小路上,她觉得自己从来没有像现在这样在乎过。


\subsubsection{第七章}


\par 不到两个星期,思嘉便成了一位妻子,又过了不到两个月,她已成了寡妇。她曾经如此匆匆忙忙,这般不费心思便承担起这些契约上所规定的义务,如今很快就又解脱了。但她再也无法体验未婚时那种无忧无虑的自由了。寡妇身份倒是紧接着婚姻接踵而至,但使她感到沮丧的是,当妈妈的日子也跟随而来了。
\par 在以后的岁月里,当思嘉回想起一八六一年四月最后那些日子时,对那些细节,她的记忆从来就不是太清楚。时间和所发生的事重叠交叉,像一场并非现实、没有理性的梦魇一样,混杂在一起。到她去世的那一天,对那些日子的记忆一定会有空白点的。对她接受查理和举行婚礼之间的那段时间的记忆,更是特别模糊。两个星期!订婚时间这么短,这在和平时期是绝对不可能的。那时本来应该有一年半载的礼节性的间隔期。但是南方已经燃起了战火,各种事件就像被一股劲风刮过似的以迅雷不及掩耳之势相继发生,过去日子里那种不紧不慢的步调一去不复返了。埃伦双手绞在一起,建议往后推一推,好让思嘉或许能够更加慎重地把这件事再考虑考虑。但思嘉对她的恳求充耳不闻,满脸不高兴。她要结婚!而且必须快点,两个星期内就得结婚。
\par 希礼的婚礼已从秋天提前到五月一日,这样,只要骑兵连一旦被召参战,他便可以随军开拔。知道这一点后,思嘉把她的婚礼定在他的婚礼前一天。埃伦表示反对,但查理以新近才发现的口才恳求她同意,因为他急于要到南卡罗来纳去参加韦德·汉普顿的团队。嘉乐也站在两个年轻人这一边。他因战争热已是激动万分,对思嘉找了这么一个如意佳婿感到很高兴。战争在进行当中,他还站在一对年轻恋人的爱情之路上碍手碍脚的,他成什么人了?埃伦被搞得心烦意乱的,最后也只好和南方其他的妈妈们一样让步了。他们从容不迫的世界被搅得乱七八糟的,而在把他们裹胁向前的强大力量面前,他们的恳求、祈祷和建议根本无济于事。
\par 整个南方都陶醉在一股热情和激动的情绪当中。每个人都知道,只要打一仗就可以结束战争,而每个年轻小伙子都赶在战争结束以前去报名参军——而且在冲到弗吉尼亚去给北方佬痛击一番以前,赶紧跟自己心爱的人结婚。县里有几十对新人借战争之机举行了婚礼,但也没什么时间可用来为分别痛苦一番,因为每个人都太忙了,也太激动了,无暇顾及那些一本正经的思想和眼泪。女人们在做制服、织袜子、卷绷带,男人们则忙着军训和练射击。每天都有一火车一火车的士兵经过琼斯伯勒到北部的亚特兰大和弗吉尼亚去。有些支队穿着猩红、浅蓝或浅绿的制服,是社会—民兵连队中精选出来的,看上去非常令人赏心悦目;有些小股部队却穿着家纺的衣服,戴着浣熊皮帽;还有其他没穿制服的,他们只穿绒面呢和上好的亚麻布做的衣服;全都是未经过全面的严格训练的半吊子,武器装备也不全,可都激动得发狂,大喊大叫的,好像是在去野餐的路上一样。看到这些人的样子,县里的男孩全都着慌了,害怕还没等他们到达弗吉尼亚战争就会结束,所以,为骑兵连出发参战的准备便也紧锣密鼓地加速进行着。
\par 在这一片混乱当中,思嘉的婚礼也在准备过程中。还没等她明白是怎么回事,她已经穿上埃伦的婚纱、戴上她的面纱,挽着父亲的手臂,顺着塔拉宽大的楼梯拾级而下,去面对一座挤满宾客的房子了。后来,她就像回忆梦境中的情景一样,还记得墙上几百支蜡烛烛光点点,她妈妈的脸上带着慈爱,有点迷惑不解的样子,嘴唇无声地蠕动着,在为女儿的幸福祈祷。嘉乐满脸通红,一是喝了白兰地的缘故,二则是为女儿和一个既有钱、名声又好而且是个世家大户的人结婚而感到很自豪——希礼手里挽着媚兰正在台阶底部站着。
\par 她看见他脸上的表情时,心想:“这不可能是真的!不可能的。这只是一场噩梦。我会醒过来,发现这全都只是一场噩梦。现在,我可不能想,要不我会在这么多人面前尖叫出来的。现在我可不能想,我要在以后能忍受的时候再想这件事——在我看不到他的眼睛的时候。”
\par 一切都好像在梦境中一样,通道两旁站满了满脸是笑的人们、查理猩红色的脸和结结巴巴的声音,还有她自己的回答,都清晰得令人吃惊,但又显得非常冷淡。还有后来人们对他们的祝贺、亲吻、祝酒以及舞会——一切,一切的一切都像做梦一样。连希礼吻她面颊的感觉以及媚兰温柔的低语:“现在我们成了真正的姐妹了。”都是那么地不真实。那令人神魂颠倒的魅力使查理那丰满而易动感情的姑妈韩白蝶小姐目瞪口呆。可就是这引起的激动之情也带上了一丝梦魇的意味。
\par 但是,当舞会和祝酒终于结束,当黄昏最后到来时,来自亚特兰大的宾客能挤就全都挤进塔拉和监工房里,睡在床上、沙发上及地上的地铺上,所有的邻居也都回家去休息了,准备第二天去忙活在十二棵橡树举行的婚礼。这时,那梦境般的恍惚在现实面前便像水晶玻璃一样破碎了。这个现实便是,面露羞赧之色的查理穿着睡衣从她的梳妆室里出现了,他躲避着她向他投来的诧异的目光。此时的她正躺在床上,床单拉得很高。
\par 当然,她知道结过婚的人是共睡一张床的,但她过去从来没有想过这件事。这于她的母亲和父亲似乎是很自然的事,可她从未把这条规则用在自己身上。现在,她突然意识到自己都为自己做了些什么,这从烧烤野餐会以来还是头一次。这个她从来没真正想跟他结婚的陌生男人要和她一起睡在同一张床上,而她的心却因为自己匆促的行动和永远失去希礼而痛苦得快要碎了,想到这一点,她觉得这一切太令人无法忍受了。当他犹犹豫豫地向床边走去时,她用沙哑的声音低声说道:
\par “如果你走近我,我就大声叫起来。我会的!我会的——用我最大的声音叫起来!从我这滚开!你不要碰我!”
\par 这样,查理的新婚之夜便在角落里的一张扶手椅上度过了,但他并没有感到特别的不高兴,因为他理解,或者说,他认为他理解他的新娘羞涩和微妙的情感。他愿意等她的畏惧感慢慢减退,只是——只是——他叹了口气,一边挪动身子以找到一个舒适的睡姿,因为他很快就要离开家参加战争去了。
\par 尽管她自己的婚礼犹如梦魇一般,但希礼的还更糟。在几百支蜡烛的烛光映照下,思嘉身着婚礼后第二天穿的苹果绿裙装,站在十二棵橡树的游廊上,身边挤着和前一天晚上一样的那群人,看着韩媚兰那张普通的小脸蛋在变成希礼太太的过程中大放异彩,成了美人。现在,永远失去希礼了。她的希礼。不,现在不是她的希礼了。他曾经是她的吗?这一切在她脑子里全混在一起了,而她的头脑又是如此疲倦,如此迷茫。他曾经说过他爱她,但又是什么把他们分开了呢?要是她能记得就好了。通过和查理结婚,她堵住了县里爱传播流言蜚语的人们的嘴,可对于现在,那又有什么要紧了呢?有一度似乎是很重要的,可现在却好像一点也不重要了。唯一重要的是希礼。现在他走了,而她却已经和一个她不但不爱,而且打心眼里就瞧不起的男人结了婚。
\par 噢,她有多后悔呀。她经常听说有人总跟自己过不去,但迄今为止她还把这只当做一种修辞手法。现在,她终于知道这个说法的含义了。她疯狂地希望自己能摆脱查理,安全地回到塔拉,重新做一个未婚姑娘。和这愿望混杂在一起的想法便是:她知道这只能怨自己一个人。埃伦曾试图阻止她,可她不听她的。
\par 这样,在希礼举行婚礼的那天,她整个晚上都茫然地跳舞,机械地说话,脸上还带着微笑,还为这个毫不相干的问题感到纳闷:人们怎么就这么傻,会认为她是个幸福的新娘子,却看不出她的心其实都要碎了。哦,感谢上帝,他们看不出来!
\par 那天晚上,嬷嬷帮她脱了衣服,然后向她告别离开后,查理害羞地从梳妆室出现了,心里还在想着自己是否要在马毛椅上度过第二个夜晚。这时,她不禁放声大哭起来。查理爬上床,坐在她身边,想去安慰她。她一言不发地哭着,直哭到眼泪干了,最后才躺在他肩膀上无声地啜泣着。
\par 要不是发生了战争,那就会有一星期时间让他们在全县拜访客人,还会有为这两对新人举办的舞会和烧烤野餐,然后他们就会出发到萨拉托加或白硫磺泉去蜜月旅行。如果没有战争,思嘉还得穿上婚礼后第三天、第四天及第五天穿的衣服到方丹家、卡尔弗特家及塔尔顿家去参加为庆祝她的婚礼而举办的晚会。但现在既没有晚会也没有蜜月旅行了。婚礼举行后一个星期,查理出发去参加韦德·汉普顿上校的部队去了,而两个星期以后,希礼和骑兵连也出发了,使整个县犹如丧失亲人一般。
\par 在那两个星期中,思嘉从来没有单独见过希礼,也没有私下和他说过一句话。他在前往火车站的路上,曾在塔拉稍作停留,和他们告别。即使在这个可怕的时刻,她也没有私下和他谈过话。媚兰戴着帽子,围着披巾,有了一种新近才有的主妇般的尊贵神情,挽着他的手臂,稳重而严肃。塔拉的所有成员,不管是黑人还是白人,全都出来送希礼去参战。
\par 媚兰说:“你应该吻吻思嘉,希礼。她现在是我嫂嫂了。”于是希礼弯下腰,用冰凉的嘴唇吻了吻她的面颊。他拉长着脸,一副严峻的样子。思嘉从这一吻中几乎没有得到什么快乐,因这一吻是在媚兰的怂恿下才有的,所以,她心里闷闷不乐。媚兰分别时紧紧拥抱了她,几乎让她透不过气来。
\par “你会到亚特兰大来看我和白蝶姑妈的,对不对?噢,亲爱的,我们太想你来了!我们想对查理的妻子了解得多一些。”
\par 又过了五个星期。这期间,查理从南卡罗来纳寄来了羞羞答答、欣喜若狂、充满爱意的信件,诉说他的爱,战争结束后对未来的计划、为了她要成为战斗英雄的理想以及对他的上司——韦德·汉普顿的崇拜。到第七个星期,来了一封由汉普顿上校亲自发来的电报,而后是一封信,一封善意、尊贵的慰问信。查理死了。上校本来早就要拍电报的,但是查理认为自己的病只是小毛病,不想让他的家人担心。这个不幸的男孩不但被他认为自己已经得到的爱欺骗了,而且也被他想在战场上获得荣誉的极大希望欺骗了。他得了麻疹,又并发了肺炎,只到了南卡罗来纳的营地,连北方佬的影子都没看见,便无声无息地迅速离开了人世。
\par 到了产期,查理的儿子出世了,因为当时很时髦把男孩的名字用父亲的指挥官的名字来命名,所以孩子被叫做韦德·汉普顿。思嘉知道自己怀孕时曾经绝望地哭过,并且希望自己也死去算了。但在她的十月怀胎期,身体不适的时候很少,而且不怎么痛苦就生下了他,恢复得也很快。嬷嬷私下曾告诉她,这是极为正常的——女人们应该多受罪。她对孩子并没多少爱,虽然她可以掩饰这一实情。她本不想要他,所以不喜欢他的到来,可现在他还是来到了人间,但他似乎不可能是她的孩子,不可能是她的骨肉。
\par 生下韦德后,她的身体恢复得很快,时间短得让人感到很丢脸。虽然如此,她在精神上却觉得神情恍惚,像生了病一样。充满活力的她变得萎靡不振的,即使整个种植园的人都努力想让她恢复过来也无济于事。埃伦成日里皱着眉头、忧心忡忡的,嘉乐比往日更会诅咒发誓了,还从琼斯伯勒给她带来毫无用处的礼物。连老方丹医生在他的硫磺补剂、糖浆及药草都没法使她振作起来之后,也只好承认连他都感到困惑不解了。他私下告诉埃伦,思嘉一会烦躁不安,一会无精打采,是因为她伤透了心。但是,如果思嘉想说话的话,她就会告诉他们,这其中的烦恼与此大相径庭,而且比这复杂得多。她没有告诉他们,这是因为生活太无聊了,而且,确确实实当了妈妈以后,她感到很茫然,最重要的是,由于希礼不在,这才使她看上去有这么一副愁眉苦脸的样子。
\par 她感到非常无聊,而且这种无聊的心境从来就没有消失过。自从骑兵连去参战之后,县里就不再有什么娱乐和社交活动。所有有趣的年轻小伙子都走了——塔尔顿家四个男孩,卡尔弗特家两个,方丹家的,芒罗家的,还有从琼斯伯勒、费耶特维尔及拉夫乔伊来的每个年轻而有魅力的男子。只有老人、残疾人和妇人才留了下来,她们成天就只是编织、做针线,为部队种植更多的棉花和玉米,饲养更多的猪呀羊呀牛呀什么的。除了苏埃伦年届中年的男朋友弗兰克·肯尼迪带领的军需部队每个月打这经过去收集供给外,从来就看不到一个真正的男人。军需部队的男人并不是会令人非常激动的人,而弗兰克那羞怯的讨好奉承使她更加烦恼,最终发现自己很难对他礼貌相待。要是他和苏埃伦能早日完婚就好了!
\par 就算军需部队的人有趣得多,这对她的心境也无济于事。她是个寡妇,心已经进了坟墓。至少,大家都认为她的心已进了坟墓,并且希望她能有相应的举动。这使得她很烦躁不安,尽管她努力去做,但她还是回忆不起任何有关查理的事,唯一记得的就是她告诉他要和他结婚时他脸上现出的那副死前的小牛犊的神情。可即使是这幅画面也在慢慢地被淡忘。但她是个寡妇,她得注意自己的行为举止。未婚女孩的快乐于她是不合适的。她非得庄重肃穆、冷淡孤傲不可。埃伦看到弗兰克手下的中尉在花园里给思嘉荡秋千并使她尖声大笑之后,特别详细地强调了这一点。埃伦非常苦恼地告诉她,一个寡妇要成为别人闲言碎语的对象,别提有多容易了。和一个普通妇人相比,寡妇的言谈举止要加倍地谨慎。
\par “只有上帝知道,”思嘉一边乖乖地听着她妈妈温柔的声音,一边想,“婚后的女人根本没什么乐趣可言。所以,寡妇还不如死了的好。”
\par 寡妇还得穿着可怕的黑衣裙,连镶上一点点镶边、使它看上去更有生气一些都不行,还不能戴鲜花、扎缎带、配花边,甚至首饰也不能戴,只有用亡夫的头发做的缟玛瑙胸针和项链才能戴。帽子上的黑绉面纱必须长达膝部,只有守寡三年以后,才能缩短至肩部。寡妇从来就不能快快乐乐地说话,肆无忌惮地大笑。即使微笑的时候也必须是忧伤且带悲剧色彩的微笑。而且,最可怕的是,无论如何,她们都不能对有绅士陪伴表现出一点点兴趣。如果哪位绅士如此没教养,敢暗示对她感兴趣,她也必须以一种尊贵且经过斟酌的词句提到自己的丈夫,好让他死心。“哦,是的,”思嘉消沉地想,“有些寡妇最后在人老珠黄、青筋凸显的时候也有再婚的。虽然,只有老天才知道,她们在邻居的众目睽睽之下是如何应付的。而且,这一般都是一些拥有一个大种植园和一打孩子的绝望的老寡妇。”
\par 结婚就已经够糟的了,但成了寡妇——噢,那生活就永远结束了!人们谈到查理走后,小韦德·汉普顿给了她多大的安慰时,他们有多傻啊!他们说,现在她活下去就有奔头了,他们真是太傻了!大家都在说,她有了这个遗腹子,留下了爱情的印记,这真是太好了,她自然也不想去纠正他们的想法。但这一想法离她自己的心思是相距最远的。她对韦德的兴趣很少,有时还很难记得他确确实实是她的骨肉。
\par 每天早晨醒来后,在睡眼惺忪的那一刻,她会重新成为郝思嘉。屋外阳光灿烂,照在她窗外的木兰花上,模仿鸟在欢唱,煎咸肉的好闻的香味悄悄地飘入鼻腔。她便无忧无虑、年轻快乐了。接着她便会听到因肚子饿而躁动不安的号啕大哭,这总是——总是使她大吃一惊,一边还想:“哦,屋里有个婴儿呢!”这以后,她才会记得这是她的孩子。这太令人茫然不解了。
\par 而希礼!噢,最重要的是希礼!她平生第一次对塔拉心怀恨意,恨那从小山坡上往下通到河边的长长的红土路,恨那栽满泛出新绿的棉花丛的红色的田地。这里的每一寸土地、每一棵树木、每一条小溪、每一条小路、每一条马道都使她想起他。他已属于另一个女人,而且已经去打仗了。但垂暮时分,他的幽灵还在困扰着她,还站在走廊的阴影中用慵懒的灰色目光对着她微笑。每次听到从十二棵橡树沿着河边的道路迤逦而来的马蹄声,她无不忘情地想起——希礼!
\par 她现在恨透了十二棵橡树,而她一度曾爱过它。她恨它,但又总被它吸引到那去,这样她就能听到卫约翰和姑娘们谈论他了——听他们读他从弗吉尼亚寄来的信。它们令她伤心,但她还得听。她不喜欢脖子僵硬的英蒂和又愚蠢又爱唠叨的哈尼,也知道她们同样不喜欢她。但她无法不接近她们。每次从十二棵橡树回来后,她便闷闷不乐地躺在床上,连晚饭也不起来吃。
\par 她拒绝吃东西,这比任何别的事都令埃伦和嬷嬷更担心。嬷嬷端来了令人看了垂涎欲滴的食盘,暗示说现在她已经是寡妇了,高兴吃多少就可以吃多少。但思嘉却一点食欲也没有。
\par 当方丹医生严肃地告知埃伦,伤心常常会导致体质衰弱,而妇女因衰弱消瘦会引发死亡时,埃伦脸都白了,因为她心里也有这种担心。
\par “没什么法子了吗,医生?”
\par “换个环境,在这个世界上,这对她是最好的办法了。”医生说,心里急于摆脱一个他无法医治的病人。
\par 这样,毫无兴致的思嘉便带着她的孩子出发了,先去萨凡纳拜访了她娘家及罗比亚尔家的亲戚,然后又去查尔斯顿埃伦的姐妹波琳和尤拉莉家。但她比埃伦预计的提早一个月便回到了塔拉,对她的提早归来也未作任何解释。萨凡纳的亲戚对她都很好,但詹姆斯和安德鲁及他们的妻子都已上了年纪,成天只满足于安安静静地坐着谈论往昔的日子,这思嘉一点也不感兴趣。罗比亚尔家也是一样,而查尔斯顿更是一团糟,思嘉这么想。
\par 波琳姨妈和她的丈夫住在河边的一个种植园里,那里比塔拉偏僻多了。她丈夫是个小个子老头,有一套正规而冷淡的礼数和一副生活在往昔岁月里的神情,看上去漫不经心的。他们最近的邻居也在二十英里外,通往那里的是一条黑漆漆的路,从还是丛林的柏树沼泽地和橡树林里穿过去。橡树上幕状的灰色苔藓摇摆不定,给了思嘉一种毛骨悚然的感觉,而且总是令她想起嘉乐讲的有关爱尔兰的鬼魂在闪着微光的灰色雾霭中游荡的故事。那里什么事也没有,成天就只是编织,晚上则听凯里姨夫大声朗读布尔沃·利顿开导人的作品。
\par 尤拉莉幽居在查尔斯顿炮台处的一座大房子里,前面是一座围墙很高的花园,而她的生活更是兴味索然。思嘉习惯了绵延起伏的红色山丘那宽广无边的景色,在这里觉得就像在蹲监狱一样。这里比波琳姨妈那多一些社交活动,但思嘉不喜欢登门拜访的人,他们那神态、传统及对家世的注重都让思嘉反感。她知道,他们全都认为她父母的婚姻是门不当户不对的,不知道罗比亚尔家的人怎么会嫁给一个新来乍到的爱尔兰人。思嘉感觉到尤拉莉姨妈背地里为她辩解。这使她火冒三丈,因为她和她父亲一样并不在乎家世。她对嘉乐及他所获得的成功感到很自豪,因为那是在没有人帮助、只靠他自己精明的爱尔兰头脑获得的。
\par 查尔斯顿人还动不动就把炮轰萨姆特堡的事引以为荣!老天,他们难道没有意识到,就算他们没有傻乎乎地开枪燃起战火,其他一些傻瓜也照样会去做的吗?听惯了佐治亚山地人们欢快的声音后,平原地带人平平的慢吞吞说话的声音在她看来似乎很造作。她觉得,如果她再听到把“棕榈树”说成“棕哦榈树”、“房子”说成“房——子”、“不”说成“不哦——”、“妈妈和爸爸”说成“妈啊妈和爸啊爸”这类声音,她就会叫喊起来。这使她极为烦躁,以致在一次正式的拜访中,她模仿了嘉乐的爱尔兰土腔,使得她姨妈很苦恼。这以后,她便回到了塔拉。与其让查尔斯顿的口音弄得痛苦不堪,还不如被对希礼的思念折磨来得好。
\par 埃伦日夜忙活着,让塔拉生产出双倍的产品来支援南部邦联。当她的大女儿从查尔斯顿回到家里时,看到她身体瘦弱、面色苍白、说话尖刻,她不禁吓坏了。她自己也体验过伤心的痛苦,夜复一夜,她躺在鼾声大作的嘉乐身边,试图想出能让思嘉减轻苦恼的办法。查理的姑姑——韩白蝶小姐给她写了好几封信,敦促她让思嘉到亚特兰大去长住一阵,现在埃伦第一次慎重地考虑起这个问题来。
\par 她和媚兰两人孤零零地住在一所大房子里,“既然连查理也走了,她们便没有了男性的保护,”白蝶小姐在信中写道,“当然,还有我的哥哥亨利,但他不跟我们住在一起。但也许思嘉已经告诉过你有关亨利的情况。我身体不好,不能在信中写更多有关他的情况了。如果思嘉能到这来跟我们住在一起,梅利和我会感觉更自在、更安全的。三个寂寞的女人在一起总比只有两个强。而且,也许亲爱的思嘉能发现什么能减轻她的悲伤的东西,就像梅利在做的,到医院里去照料我们的勇敢的孩子们——当然,梅利和我也很想看看亲爱的小宝贝……”
\par 这样,思嘉的箱子连同她的丧服又重新被打点一番,和韩韦德及他的保姆普里西一道,带着满脑子埃伦和嬷嬷对她行为准则的告诫及嘉乐给她的换成南部邦联纸币的一百美元,出发到亚特兰大去了。她并不特别想去亚特兰大。她认为白蝶姑妈是老太太中最愚蠢的人,而且,要和希礼的妻子住在同一个屋檐下,这个念头就已经够令人厌恶了。但是县里能勾起她回忆的事太多了,现在已不可能再待下去,所以,换一换环境总是受欢迎的。



\subsection{第二部}



\subsubsection{第八章}

\par 一八六二年五月的一天早晨,思嘉乘着列车北上时,心里还在想,亚特兰大可能不至于像查尔斯顿和萨凡纳那样单调乏味。所以,虽然她不喜欢白蝶小姐和媚兰,但还是好奇心十足,想知道自从上次到过亚特兰大后,这个城镇又有了哪些新变化。那还是去年冬天的事,那时战争还没开始呢。
\par 和其他城镇相比,她对亚特兰大的兴趣总是更大一些,因为,在她还是个孩子的时候,嘉乐就告诉过她,她和亚特兰大刚好是同年。她年龄再大一些后,发现嘉乐其实多少夸大了事实,而这正是他的习惯,只要这种夸大能形成一个故事;但亚特兰大只比她大九岁,和她所听说过的任何一个城镇相比,这个地方还是年轻得令人咋舌。萨凡纳和查尔斯顿因为有了些年头而颇显尊贵,一个正在第二个世纪之路上挺进,另一个正迈入第三个世纪。在她年轻的眼里,它们就像上了年纪的老祖母一样,总是在阳光下心平气和地摇着扇子。可亚特兰大和她却是属于同一个年代的,因为不成熟而显得很粗鲁,而且和她自己一样任性而急躁。
\par 嘉乐告诉她的故事也并非没有根据,即她和亚特兰大是在同一年受洗命名的。在思嘉出生前的九年中,这个城市先是被叫做特米纳斯,然后又被叫做马撒斯维尔,直到思嘉出生的这一年,才改叫亚特兰大。
\par 嘉乐刚搬到佐治亚北部时,亚特兰大根本就不存在,甚至连个小村子也不像。这地方全是茫茫的荒野。但在第二年,也就是一八三六年,州里授权修建一条西北走向的铁路,横穿柴拉基几族人新近退出的领地。计划中的铁路目的地为田纳西和西部,这是毋庸置疑的,但在佐治亚的起点却不知怎的还没定,直到一年过后,一位工程师在红土上立了一根桩,标出了铁路线的最南端,由此也就有了前身为特米纳斯的亚特兰大。
\par 当时佐治亚北部还没有铁路,在其他地方也极为罕见。但在嘉乐和埃伦结婚前的那些年中,这个塔拉以北二十五英里远的小拓荒地慢慢发展成一个小村庄,铁路也渐渐向北延伸。后来,铁路建设的年代真正开始了。从老城镇奥古斯塔修了第二条向西延伸横跨全州的铁路,和通往田纳西的新路连接。从老城市萨凡纳则修了第三条铁路,起先只通到佐治亚的中心地带梅肯,后来再向北延伸,穿过嘉乐所在的县到亚特兰大,和另外两条路相连接,为萨凡纳的港口提供了一条通往西部的交通干线。从年轻的亚特兰大这个连接点,又建了第四条西南走向的铁路,通到蒙哥马利和莫比尔。
\par 亚特兰大因铁路而诞生,也随着铁路的发展而发展。四条铁路修好后,亚特兰大便跟西部、南部、沿海,经由奥古斯塔又和北部和东部相连了。它成了可通往四面八方的十字路口,这个小村子顿时充满了勃勃生机。
\par 在一段时间内——比思嘉度过的十七年长不了多少——亚特兰大从只有打入地下的一根标桩发展成了一个拥有一万人口的繁荣的小城市,成了全州关注的中心。更加古老、宁静的城市总是用母鸡孵出了小鸭那种惊奇感看待喧闹繁忙的新兴城镇。为什么这个地方和佐治亚其他城镇都不一样呢?为什么它会发展得这么快?他们终究还是认为,这个城镇根本没什么可值得推荐给别人的——只有铁路和一群干劲冲天的人们。
\par 最早在这个相继叫做特米纳斯、马撒斯维尔及亚特兰大的镇子定居下来的人们是一群干劲冲天的人。颇不安分但精力充沛的人们从佐治亚其他较古老的地区及更边远的州被吸引到这个城市里来,它的中心便是铁路连接点,再向四周蔓延开来。他们满怀热情而来,在那五条在车站附近交叉在一起的泥泞不堪的红土路周围建起了商店。他们在怀特霍尔和华盛顿大街两边建起了温馨的家园,沿着那被几代印第安人穿着鹿皮鞋的脚踩出一条叫做桃树街的高高的山脊上安家落户。他们为这地方感到很骄傲,也为它的发展感到很自豪,更为他们自己使它向前发展而感到很荣耀。那些老城镇把亚特兰大叫做什么都行,他们爱怎么叫就怎么叫。亚特兰大才不在乎呢。
\par 思嘉喜欢亚特兰大的原因正是萨凡纳、奥古斯塔和梅肯谴责它的原因。正如她自己一样,这是个佐治亚州新旧混合的城镇,而在旧势力与固执任性、朝气蓬勃的新势力的冲突中,旧势力总是退居第二。再说,在这个在她受洗的同一年诞生——或者至少是受洗命名——的城镇中,还有一些个人的令人激动的东西。
\par  
\par 前一天晚上,狂风肆虐,大雨倾盆。但当思嘉到亚特兰大的时候,温暖的太阳又重新露出了笑脸。街上满是沟沟壑壑,就像是积满红色泥泞的弯弯曲曲的小河。可太阳却勇敢地试图把它们晒干。车站周围的开阔地上,进进出出、连续不断的人流和车辆把那松软的泥土碾出了点点脚印、道道车辙,地面被搅得一塌糊涂,看上去就像猪打过滚的泥沼,这里那里,不时有车辆陷入车辙和凹槽中。源源不断的军用马车和救护车从火车上装卸物资和伤员,它们费尽艰辛地进来,再千辛万苦地挣扎着出去,使这片泥地和混乱状态更加惨不忍睹。司机大声咒骂,骡子陷入泥泞,泥浆飞溅,一直溅到几码开外。
\par 思嘉站在火车上较低的台阶上,黑色的孝服衬出她那脸色苍白、身材漂亮的身影,黑色的绉绸面纱几乎飘至她的脚后跟。她极不甘愿把便鞋和褶边弄得泥迹斑斑的,所以犹豫着不敢迈步。她在喧闹混杂的马车和货车声中举目四望,寻找着白蝶小姐,可连那丰满、脸蛋粉红的老太太的影子也没看见。但当思嘉的目光焦急地四处搜寻时,有个上了年纪、面容清癯的黑人穿过泥泞地向她走来。他看上去焦虑不安,手里拿着帽子,模样颇为体面,一副很权威的样子。
\par “你是思嘉小姐,对吗?俺是彼德,白蝶小姐的车夫。别在那泥浆里走,”思嘉拉起裙子,准备往下走时,他严肃地命令道,“你真是跟白蝶小姐一样坏,她就像个孩子一样,老把双脚弄得湿漉漉的。俺来抱你吧。”
\par 虽然他看上去身体瘦弱,又上了年纪,但他还是轻而易举地把思嘉抱了起来。看到普里西手里抱着小孩站在火车的平台上,他停下脚步:“那孩子是你的保姆吧?思嘉小姐,她太年轻了,没法伺候查理唯一的孩子!咱们还是以后再说这件事吧。你这孩子,跟我来吧,可别把孩子摔着了。”
\par 思嘉乖乖地依言而行,让自己被抱到马车上去,也接受了彼德大叔批评她和普里西的独断的方式。他们穿过泥泞地,普里西则板着脸踩着泥浆跟在后面。这时,思嘉想起了查理说过的有关彼德大叔的事。
\par “他和爸爸一起经历了墨西哥的所有战役。爸爸受伤时,他便看护他——事实上,是他救了爸爸的命。彼德大叔实际上抚养了媚兰和我,因为爸爸妈妈去世时,我们还很小。差不多那时候,白蝶姑妈和她哥哥,也就是亨利叔叔吵了一架,所以也来和我们住在一起,照顾我们。她是个最没用的人了——就像个可爱、老长不大的大小孩一样,彼德大叔就是这样看待她的。为了保住一条命,她对什么事都下不了决心,所以彼德只好帮她拿主意。决定我十五岁时应该有笔数目更大的津贴的就是他,他还坚持我大学四年级必须去上哈佛,彼德大叔想让我在这所大学拿学位。梅利到了可以梳起头发去参加晚会的年龄时,也是他的决定。他还告诉白蝶姑妈,什么时候天气太冷,不宜出门访客,什么时候该披上披巾……他是我见过的最精明的老黑人,而且差不多是最忠诚的了。他唯一的麻烦是他拥有我们三个人,从肉体到灵魂,他也知道这一点。”
\par 彼德爬上车座,拿起马鞭时,查理的话就进一步得到了证实。
\par “白蝶小姐不太舒服,所以才没来接你。她还担心你会不理解,但我告诉她,她和梅利小姐会弄得满身是泥,把新衣服也给毁掉了。还告诉她我会向你解释的。思嘉小姐,你最好把孩子抱过来,那个黑人小孩会把孩子摔着的。”
\par 思嘉看了看普里西,叹了口气。普里西并不是最胜任的保姆。她新近才从一个穿着简单的裙子、扎着硬邦邦的辫子的瘦骨嶙峋的黑人小孩变成一个穿着长长的女式衣裙、戴着上过浆的白色无檐女帽的尊贵的成年人,这种等级的升越是件令人陶醉的事。要不是战事紧急,军需部对塔拉的要求使埃伦不可能让嬷嬷或迪尔西闲下来,甚至连罗莎和蒂娜也分不开身,她是决不会这么快就升到这种显赫的地位的。普里西过去从未到过离十二棵橡树或是塔拉超过一英里的地方,坐火车的旅程加上她升为保姆的喜悦,这些几乎使她那颗小小的黑人脑壳无法承受。从琼斯伯勒到亚特兰大的全长二十英里的旅程使她激动万分,思嘉不得不一路自己抱着孩子。现在,看到这么多建筑物和人,普里西完全陷入了混乱心态。她从一边转到另一边,指东指西,动来动去,把孩子颠得痛苦地号啕大哭起来。
\par 思嘉太希望嬷嬷那肥胖、苍老的手臂能在跟前了。嬷嬷的手只要一触到孩子,孩子便会止住哭声。但嬷嬷人在塔拉,思嘉自己则对此无能为力。就算她从普里西手里抱过韦德,那也不会有什么用的,他还是会像在普里西抱他时一样大声哭闹。此外,他还会用力拉扯她帽子上的丝带,无疑还会弄皱她的衣服。所以她假装没有听到彼德大叔的建议。
\par “也许什么时候我得学些有关孩子的知识,”马车颠簸着摇摇晃晃驶出车站周围那片泥沼时,思嘉烦躁不安地想着,“但我绝不会喜欢哄孩子的。”韦德的脸因哭闹而变成青紫时,她生气地厉声说道:“把你口袋里那个糖水奶头给他,普里西。只要能让他安静下来,什么都行。我知道他饿了,可我现在什么事也做不了。”
\par 普里西拿出那天早晨嬷嬷给她的糖水奶头,孩子的哭声渐渐止住了。重新恢复了平静后,再加上看到了新的东西,思嘉的情绪开始慢慢好起来。彼德大叔最终把马车顺利地赶出坑坑洼洼的泥泞地,上了桃树街。她感到几个月以来的兴致终于涌上心头。这个城镇发展多快啊!离她上次到这里来只不过才一年多一点,可她所知道的小小的亚特兰大居然变化这么大,这简直是不可能的。
\par 在过去的一年中,她的心思全放在自己的不幸上去了。别人一提到战争,她就感到厌烦透顶。她不知道,从开战的那一刻起,亚特兰大就被改变了。和平时期,那些铁路使这个城市成了商业贸易的十字路口,而在战时,同样的这些铁路便被赋予了重要的战略地位。虽然远离前线,这个城市及它所拥有的铁路连接了南部邦联的两支部队——在弗吉尼亚的一支及在田纳西和西部的一支。亚特兰大同样也成了联系这两支部队以及南部物资供给区的连接点。现在,为了适应战争的需要,亚特兰大已经变成一个制造中心、医疗基地以及南部供给品的主要仓库之一。
\par 思嘉环顾四周,想找到自己如此熟悉的那个小镇。可那早已无影无踪了。她现在看见的这个城市就像是一个婴儿在一夜之间猛长,突然就长成了一个忙忙碌碌、四肢伸展着的巨人。
\par 亚特兰大喧闹忙乱,犹如蜂窝一般。它自知自己对南部邦联很重要,为此感到无比自豪。各项工作正在紧锣密鼓、日夜不停地进行着,要把农业区变成工业区。战前,马里兰州以南没什么棉纺厂、毛纺厂、兵工厂及机械商店——所有的南方人都曾为这一点感到无比自豪。南方会出政治家和士兵,种植园主和医生,律师和诗人,但没有工程师和机械师,那是当然的。让北方佬去享有这些低档的头衔吧。可是现在,南部邦联的港口都被北方佬的炮舰封锁住了,只有一点点从欧洲来的物资才偷偷越过封锁线被运进来,南方正竭尽全力试图生产出自己的战争物资。北方可以号召全世界为它提供物资和士兵,而受北方优厚报酬的诱惑,成千上万的爱尔兰人和德国人蜂拥而至,纷纷参加了联邦军队。而南方只能依靠自己的力量。
\par 在亚特兰大,也有一些工厂老牛拉破车似的生产出能制造战争物资的机器——说它老牛拉破车,是因为在南方没什么机器可供他们模仿制造,几乎每一个轮子和嵌齿都得按照从英国越过封锁线弄进来的图纸来生产。现在,亚特兰大的街上便有了一些陌生的面孔。一年前,有些当地人听到哪怕是西部的口音也会警觉地竖起耳朵,现在,就是对来自欧洲的外国口音也毫不在意了。这些欧洲人都是穿过封锁线到这来制造机器并生产出南部邦联所需的军需品的。这些都是些有技术的人,没有他们,南部邦联就很难生产出手枪、步枪、大炮及炸药。
\par 工作在日夜不停地进行着,把战争物资装上铁路干线,运到两个作战前线,人们似乎可以感觉到这个城市的心脏跳动的脉搏声。每时每刻都有火车飞奔着进出车站。新建工厂的烟灰铺天盖地而来,沾在一座座白色的房屋上。到了晚上,市民们上床睡觉后,很久了都还能看见火炉的火光、听到铁锤敲击的铿锵声。一年前还是空地的地方,现在呢,有的已经变成了生产马具、马鞍和马掌的工厂;有的成了制造步枪和大炮的兵工厂;还有的成了生产用以代替被北方佬毁坏的铁轨和火车车厢的轧钢厂和铸造厂,还出现了各种各样制造马刺、马勒的小部件、带扣、帐篷、扣子、手枪和刺刀的行业。铸造厂已经开始感到铁的供应吃紧了,因为能越过封锁线进来的没有多少,或根本就没有,而在亚拉巴马州的矿山却几乎就在闲置着,因为矿工们都到前线去了。现在,亚特兰大的所有草坪上,根本看不到铁栅栏、铁制凉亭和铁门,甚至连铁的雕塑也没有,因为它们早就被送到轧钢厂的炼钢炉里去熔化了。
\par 桃树街及附近的大街上,沿街全是部队各个部门的总部,每个办公室都挤满了穿着军服的人。军需部、通信部、邮寄部、铁路运输部及宪兵司令部。市郊是马匹的补给点,宽大的畜栏里一群群马匹和骡子在转来转去,旁边的街道则是医院。彼德大叔向思嘉介绍这些情况时,她总感到亚特兰大是座充斥伤病员的城市,因为既有不计其数的普通医院,又有传染病院和疗养院。每天,列车开到五角场便又吐出大批伤病员。
\par 小镇已经不见了,这个城市快速发展的新面孔被赋予了永远使不完的精力和活跃气氛。思嘉刚从乡下那种悠闲、安静的环境中来,看到这里一派繁忙景象,几乎使她透不过气来,但她喜欢这样。这个地方这种令人激动的气氛使她感到振奋。她似乎实实在在地感觉到,这个城市正在稳步加快的心脏搏动正和自己的一块跳动。
\par 在市里的主要街道上,他们穿过坑坑洼洼的路面慢慢前行。这时,她饶有兴趣地注意到所有新的建筑物和新面孔。人行道上挤满了穿着制服的男人,戴着各种军衔和服役兵团的徽章;窄小的街道挤满了各种车辆——马车、小货车、救护车,还有部队的有篷运货车,骡子碾过车辙凹槽在艰难地前进,好咒骂的司机则在不停地谩骂;穿着灰色制服的信使在飞溅的泥浆中带着货单和电报急件从一个总部冲到另一个总部;正在康复的士兵拄着拐杖一瘸一拐地走着,通常两边还各有一个满心焦虑的女士;军训场上传来军号声、擂鼓声和喊口令的叫声,刚入伍的新兵正在那里被训练成士兵;思嘉第一次看到北方佬的军服时,心都跳到嗓子眼里了。彼德大叔用马鞭指着一队穿着蓝色制服的神情沮丧的人给她看,一小队南方部队的士兵正端着上好刺刀的枪押送他们到车站去,再让他们坐火车到战俘营去。
\par “噢。”思嘉心里涌起了一股真正的愉快之情。自野餐会以来,这还是第一次呢。“我会喜欢这里的!这里太有生气,太令人激动了!”
\par 这个城市甚至比她所意识到的还要有生气,因为,新的酒吧几十家几十家地不断开张;紧接着部队而来的是妓女的蜂拥而至,妓院里的女人生意兴隆,使常上教堂的信徒们目瞪口呆。每家旅馆、供膳寄宿处和私人住宅都挤满了客人,他们到这来是为了更接近住在亚特兰大各大医院里受伤的亲戚的。这里每星期都举办晚会、舞会和义卖会,还有数不清的战时婚礼。新郎是正在休假的军人,穿着色泽明亮、有灰色和金色镶边的制服,新娘则穿着偷越封锁线带进来的华丽衣服,通道上放着交叉在一起的军刀,大家喝着同样遭封锁的香槟酒为他们祝福,却又要含泪告别。晚上,两旁整齐地栽着树木的阴沉沉的街道上回响着跳舞的脚步声,大厅里回荡着钢琴声,女高音混杂着做客的士兵悦耳却忧郁的声音在唱着“军号吹响了停战声”及“你的信到了,但到得太迟了”——这些哀怨的民谣引得那些对真正痛苦的眼泪还一无所知的心软的人们流下了激动的泪水。
\par 他们穿过老往下陷的泥泞沿街继续前进时,思嘉嘴里不断冒出许多问题来,彼德一一为她解答,用马鞭指指这,指指那,为能展示自己的所知而感到无比荣耀。
\par “那是军火库。是的,小姐,他们把枪呀什么的都放在那。不,小姐,那不是商店,它们是封锁办事处。法律,思嘉小姐,你不知道封锁办事处是什么吗?那是那些外国人待的地方,他们从我们南部邦联手里买走棉花,用船运到查尔斯顿和威尔明顿出口,再把军火给我们运进来。不,小姐,俺也不敢肯定他们是哪一国的外国人。白蝶小姐说他们是英国人,但他们说的话没一个人听得懂。是的,小姐,烟雾灰尘太大了,尘土穿过白蝶小姐的丝绸窗帘往里钻。这是从铸造厂和轧钢厂飘来的。还有晚上从那传来的声音!简直吵得人没法睡觉。不,小姐,俺不能停下来让你看一看。俺已经向白蝶小姐答应过把你直接带回家的……思嘉小姐,向她们回个礼,那是梅里韦瑟小姐和埃尔辛小姐在向你点头致意呢。”
\par 思嘉依稀记得,曾有两个叫梅里韦瑟和埃尔辛的太太从亚特兰大到塔拉来参加她的婚礼,她还记得她们是白蝶小姐的好朋友。所以她很快转过身,对着彼德大叔指的方向点头致意。那两人正坐在一家干货店外面的马车里。店主和两个伙计站在人行道上,手里抱着一匹匹棉布在推销。梅里韦瑟太太是个高大结实的女人,她的紧身胸衣束得很紧,以至胸部向前突起,就像是船头一样。她那铁灰色的头发被一绺拳曲的假刘海装饰着,褐色的刘海傲气十足,似乎不屑与她的其余头发相配。埃尔辛太太较为年轻,是个单薄瘦弱的女人,过去曾经是个美人,所以,在她身上还残留着一丝已经淡化的青春活力,还有一种挑剔专横的神情。
\par 这两位太太,加上怀廷太太这第三位,是亚特兰大的三根顶梁柱。她们掌管着三座教堂、牧师、唱诗班和教民,而她们自己也是教民之一。她们组织义卖会,主持针线组的活动,还在舞会和野餐会上陪伴未婚少女。她们知道谁跟谁很般配,谁和谁则配不来,谁又暗地里喝酒了,谁又怀孕了,连什么时候生她们都知道。凡在佐治亚、南卡罗来纳及弗吉尼亚三个州有点头脸的人的家谱,她们三个都是权威,而对其他州,她们根本就不予费心,因为她们相信,有点头脸的人物没有一个是从这三个州以外的其他州来的。她们知道什么才是有教养的行为举止,什么不是,而且从来都能让她们的观点为别人所知——梅里韦瑟太太利用她那最高的嗓门,埃尔辛太太则用讲究的慢吞吞的渐渐消失的声音,怀廷太太用的是忧伤的耳语,显示出她很讨厌谈及这类事情。这三位太太打心眼里互相不喜欢,也互相不信任,就像古罗马的第一任三位执政官庞贝、恺撒和克拉瑟斯一样,而她们紧密的联盟很可能也出于同样的原因。
\par “我告诉过白蝶,我得把你要到我的医院里来。”梅里韦瑟太太笑着说,“你可别答应米德太太和怀廷太太哟!”
\par “我不会的。”思嘉说。她根本不知道梅里韦瑟太太在说些什么,但有人欢迎自己,需要自己,她心里感到了一丝温暖。“我希望很快就能再见到你。”
\par 马车继续向前跋涉。中途停了一会,让两位手臂上挎着一篮子绷带的太太踏着满是泥泞的街上摆放的几块踏脚石摇摇晃晃地穿街而过。就在同一时候,思嘉的视线被人行道上一个身穿鲜艳服饰的人影吸引住了——那服饰在街上显得太艳丽了——她披着佩兹利细毛披巾,流苏直垂到脚后跟。她转过身,看到一个高个子漂亮女人,有着一张大胆而显冒失的脸,一头蓬乱的红头发,红得像是假的。这是她第一次看到她敢肯定“做过头发”的女人。于是她注视着她,完全被迷住了。
\par “彼德大叔,那是谁呀?”她低声问道。
\par “俺不知道。”
\par “你知道的。这我看得出来。她是谁?”
\par “她名叫贝尔·沃特琳。”彼德大叔说,他的下嘴唇开始拉长了。
\par 思嘉马上注意到他没有在名字后加上“小姐”或“太太”两个字。
\par “她是谁?”
\par “思嘉小姐,”彼德阴沉着脸说,马鞭在马身上抽了一鞭,把马吓了一跳,“你问这些跟我们毫无关系的问题,白蝶小姐会不高兴的。她是这城里不值一提的贱货,说了也没用的。”
\par “天哪!”思嘉心里想着,却已被训斥得哑口无言。“那一定是个坏女人!”
\par 她过去从没见过坏女人,所以她扭过头,盯着她的背影看,直至她消失在人群中。
\par 商店和新建的战时建筑连得不那么紧密了,建筑与建筑之间有了一些空地。最后,商业区被甩在后面了,居住区映入眼帘。思嘉像是老朋友一样把它们一一认了出来:莱登家的房子,既尊贵又雄伟;有小小的白色柱子和绿色百叶窗的邦内尔家的房子;麦克卢尔家族那幽深的佐治亚红砖房伫立在低矮的箱状树篱后面。他们现在走得更慢了,因为游廊上、花园里及人行道上都有太太向她打招呼。有些人她只知道一点,其他的她记不太清楚了,但大多数她根本就不认识。白蝶一定是到处广播了她即将到来的消息。小韦德只好一次又一次被抱起来,以便敢冒险越过淤泥走到他们的马车车厢前的太太们可以对着他惊叫。她们全都对她叫着,说她必须参加她们的编织组、针线组或是护理会,不能参加别人的,她则漫不经心地左右答应着。
\par 他们经过一座有凌乱不堪的绿色护墙板的房子时,坐在门前台阶上的一个黑人小女孩叫了起来:“她来了,”米德医生和他太太,连同年仅十三岁的小菲尔便出现了,他们跟她打着招呼。思嘉想起来了,他们也来参加过她的婚礼。米德太太登上马车车厢,伸长脖子看孩子,但医生却不顾烂泥,跋涉到马车边上。他又高又瘦,留着铁灰色的尖胡子,衣服挂在消瘦的身体上,好像是被飓风刮到那似的。亚特兰大把他当成所有力量和智慧的源泉,而他多少具有他们所相信的某些优点,这是一点也不奇怪的。要不是他那发表神谕式的说话习惯和稍带浮夸式的举止的话,他倒是个好人。
\par 医生和她握了握手,并用手指在韦德肚子上戳了戳,逗着他,接着便宣布,白蝶姑妈已经发过誓,答应思嘉只到米德太太的医院和卷绷带组去帮忙。
\par “噢,天哪,可我已经答应了有上千个太太了!”思嘉说。
\par “梅里韦瑟太太,一定是她!”米德太太愤愤不平地叫了起来。“这个讨厌的婆娘!我相信,她每次火车来时都去接车!”
\par “我答应是因为我一点也不知道这是怎么回事。”思嘉承认道。“医院护理会到底是什么呀?”
\par 医生和他太太都对她的无知感到有点惊讶。
\par “当然,你一直待在乡下,被埋没了,自然不会知道,”米德太太为她辩解说,“我们有为不同的医院和不同时间服务的护理会。我们护理伤病员,给医生帮忙,制作绷带,缝制衣服。当他们治疗到可以出院时,我们便把他们接到自己家里,好让他们恢复健康,直到他们能够回部队去。我们还照看穷苦伤病员的妻子和孩子——是的,比穷苦还糟。米德医生就在我的护理会的学院医院里做事,每个人都说他太出色了,而且——”
\par “行啦,行啦,米德太太,”医生嗔怪地说,“别在人前夸我了。我能做的实在是太少了,而你又不让我去参军。”
\par “不让!”她愤愤不平地叫了起来,“我?是这个城市不让你去,你自己知道得很清楚。听我说,思嘉,当人们听说他打算去弗吉尼亚当军医时,所有的太太都签名请愿,要求他留在这。这个城市不能没有你,那是当然的。”
\par “好了,好了,米德太太,”医生说,显然听了这表扬感到很舒服,“也许有了个儿子在前线,目前来说就已经够了。”
\par “我明年也要去的!”小菲尔叫道,激动得跳来跳去,“去当鼓手。我现在正在学习如何击鼓。你想听我击鼓吗?我跑去把鼓拿来。”
\par “不,现在不用。”米德太太说,把他往身边拉了拉,脸上突然现出一种紧张的神情。“明年不行,亲爱的,也许后年吧。”
\par “但那时战争就已经结束了!”他耍着性子喊了起来,从她身边挣扎开去。“你答应过的!”
\par 在他头顶上,他父母亲的目光对视了一下,思嘉看到了这一幕。很显然,达西·米德正在弗吉尼亚,因此他们对留下的这个小儿子格外依恋。
\par 彼德大叔清了清嗓子。
\par “俺离开家里时,白蝶小姐正不舒服。如果俺不赶快回去,她会晕过去的。”
\par “再见。我下午过去看你,”米德太太叫道,“你帮我转告白蝶,如果你不到我的护理会,她的日子会更不好过。”
\par 马车继续起程,沿着泥泞的路向前滑行。思嘉靠在坐垫上,脸上露出了微笑。她现在的感觉比几个月来的感觉都更好。在亚特兰大,人头攒动、步履匆匆,还有一股促人激动的潜流,这太令人高兴、令人振奋了,所以比远在查尔斯顿郊外的那孤单寂寞的种植园好多了,那里只有短尾鳄的叫声才会打破夜晚的宁静。这里也比查尔斯顿更好,那里的人们只会躲在高高的院墙后面的花园里做梦;这里甚至比宽大的街道两旁种满棕榈树、濒临泥泞浑浊的河流的萨凡纳还要好。是的,短时间内甚至比塔拉还好,虽然塔拉也很可爱。
\par 这个街道泥泞窄小、位于起伏的红色山峦之间的城市有着某种令人激动的东西,某种天然的粗野的东西,这和她隐藏在埃伦和嬷嬷教给她的优雅外表下的某种天生的粗野天性正好吻合。她突然感到,这里正是自己应该归属的地方,自己不属于濒临黄色的河流边上的安详、宁静、平坦的老城市。
\par 房子与房子之间隔得越来越开了,思嘉探出头,看到了白蝶小姐那石板屋顶的红砖房。这几乎是这城镇北边的最后一座房子了。再过去,桃树街便越来越窄,在大树下蜿蜒远去,消失在浓密而宁静的森林中。整洁的木片栅栏刚刚漆过,雪白雪白的。栅栏围着的前院里,点缀着已要过季的最后几朵黄色的长寿花。屋前的台阶上站着两位一袭黑衣的女人。她们身后还有一个大个子黄皮肤女人,她双手放在围裙下,一脸粲然的微笑,露出了洁白的牙齿。丰满的白蝶小姐正激动地迈着小脚摇摇晃晃地向前走来,一只手放在丰满的胸部,以让那跳动不安的心平静下来。思嘉看到媚兰站在她身边,心里涌起了一股厌恶感。她于是意识到,亚特兰大的美中不足之处就是这个穿着黑色丧服的小个子女人。她那茂密的鬈发硬是被平平地梳在脑后,显出一副沉稳的模样,心形的脸上挂着表示欢迎且充满爱意的幸福微笑。
\par  
\par 南方人不嫌麻烦地收拾好箱子,来到二十英里外去探亲访友时,待在那的时间很少不超过一个月的,通常都比一个月更长。南方人去走亲戚时,热情得就像是他们才是主人一样,亲友们来过圣诞节,可自此后却一直待到七月份,这一点也不奇怪。经常,新婚夫妇作例行的巡回探亲访友时,会在某个温馨的家庭一直待到第二个孩子出世才离开。而上了年纪的姑姑姨姨、叔叔伯伯本是来赴星期天的晚宴的,却一待好几年,直至他们入土,这也是经常的事。客人来访并不会有什么麻烦,因为房子宽大,仆人成群,多加几张吃饭的嘴,在那富裕的地方真乃小事一桩。男女老幼都爱去探亲访友:度蜜月的新婚夫妇、为炫耀新生婴儿的年轻妈妈、正在康复的病人、丧失了亲人的人,还有的是年轻姑娘们,有的是父母亲急于把她们支走,以免落入不明智的婚姻的危险中去,有的则是已到了步入老姑娘的危险年龄却还没有说上亲事,希望在其他地方亲友的指导下,找到合适的婆家。来访的客人给南方慢吞吞的生活步调注入了一股令人激动的新鲜感,所以他们总是受欢迎的。
\par 同样,思嘉到亚特兰大来,对自己要在这待多久,心里一点谱也没有。如果这里也证明跟萨凡纳和查尔斯顿一样无聊乏味,那她一个月后就回家去。如果在这待得还愉快,她就将无限期地留在这。但是她刚到达,白蝶姑妈和媚兰就发起了一场战役,劝她永远和她们待在一起,把这里当成自己的家。她们把一切可能的论据都提出来了。为她们自己起见,她们也需要她,因为她们爱她。在这所大房子里,她们感到又孤单又寂寞,晚上常常感到很害怕,而她是这么勇敢,可以给她们勇气。她又是这么美丽迷人,在她们如此悲伤的时候,可以让她们振作起来。既然查理死了,她和她儿子的住所就该和他的亲人们在一起。再说,根据查理的遗嘱,现在这房子的一半已经属于她了。最后,南部邦联也需要每一双能为其做针线、编织、卷绷带和护理伤病员的手。
\par 查理的叔叔亨利是个单身汉,住在车站附近的亚特兰大旅馆里。他也就这个话题跟她严肃地谈了话。亨利叔叔五短身材,大腹便便,是个性情暴躁的老绅士。他脸色粉红,留着银白色的长发,让人看了颇感吃惊;他完全没有耐心,却又有女人般的羞涩胆怯和自卖自夸的特点。正是这后一个原因使他和他妹妹白蝶小姐关系不太好。从孩提时代起,他们的性情就截然相反,而他对她抚养查理的方式持反对态度,这便使他们更加疏远——他认为她“把一个军人的儿子培养成了一个该死的女人气十足的胆小鬼!”多年以前,他便这样侮辱过她,以至现在白蝶小姐从来都不提他,只是有时才谨慎地小声嘀咕着,而且说得极有保留,不知道的人还以为,这个诚实的老律师至少是个杀人犯呢。那次侮辱事件是在这种情况下发生的:白蝶小姐想从她的个人财产中取出五百美元去投资一座并不存在的金矿,由于他是她财产的受托管理人,所以不允许她支取,还言辞激烈地说她不会比一只绿花金龟更有头脑,说他若和她在一起再待上五分钟以上,他就会烦躁不安。从那天起,她便只跟他正式会面,每月一次,由彼德大叔赶着马车送她到他的办公室去取家用钱。每次这种短暂的会面之后,白蝶总是躺倒在床上,那天的剩余时间便是泪眼汪汪、闻着鼻盐在床上度过的。媚兰和查理跟他们叔叔的关系都好得不得了,他们也曾经不时主动提出来要减轻她所受的这种折磨,但白蝶总是紧闭她那张婴儿般的小嘴,拒绝接受。亨利是她的灾星,但她得忍着他。从这点上,查理和媚兰只能推断,她从这种偶尔才有的激动状态中能得到深深的快乐,而这激动也是她被人庇护的生活中唯一的激动。
\par 亨利叔叔马上便喜欢上了思嘉,他说,这是因为他看得出来,尽管她也傻乎乎地故作姿态,但还多少有点头脑。他不但是白蝶和媚兰财产的受托管理人,也是查理留给思嘉的财产的受托管理人。思嘉现在已是个富有的年轻女人,这对思嘉来说是个颇为令人高兴的惊喜。因为查理不但把白蝶姑妈的房子的一半留给了她,还留给了她田产和城里的产业。车站附近铁路沿线的商店和仓库也是她所继承的遗产的一部分,自开战争以来,它们就已升值了三倍。就在亨利叔叔把她财产的账目交给她时,他也提出来要她把亚特兰大作为永久住所。
\par “韦德到年龄的时候,他就会成为富有的年轻人,”他说,“根据亚特兰大的发展趋势,他的产业二十年后会增值十倍。孩子必须在他产业的所在地被抚养成人,这才是对的,这样,他就能够学会如何管理他的财产了——是的,还有白蝶的和媚兰的财产。不久以后,他就要成为韩姓家族留在这的唯一的男人,因为我不会永远待在这。”
\par 至于彼德大叔,他则想当然地认为,思嘉来了是会长住下去的。在他看来,查理唯一的儿子在自己无法监督的地方抚养成人,这是令人难以相信的。对所有这些理由,思嘉只是笑而不答。在弄清楚自己对亚特兰大和夫家亲属长期相处到底喜欢到何种程度以前,她不愿表态。她也知道,先得说服嘉乐和埃伦。再说,她一旦离开塔拉,心里便想得厉害,想那红色的田野、生长茂密的绿油油的棉花以及晨曦中舒心怡人的宁静气氛。嘉乐曾说,她对土地的爱是从血统中带来的,她现在才第一次隐隐约约地意识到这句话的含义。
\par 所以,对她要住多久这个问题,眼下她总是巧妙地避开,不给确切的答复,而是颇为轻松地融入这座红砖房里的生活中去,融入这所位于桃树街宁静的末端的房子的生活中去。
\par 跟查理的亲属生活在一起,看着他生于斯长于斯的家,思嘉现在对这个在短期内接二连三地把她变成妻子、寡妇和母亲的年轻人的了解多了一些。很容易便可以看出他为什么如此害羞、不懂世故,却又如此理想主义。如果说查理继承了他父亲——一位勇敢坚强、大胆无畏、脾气暴躁的士兵——的某些个性的话,那在孩提时代也早被把他抚养成人的女性氛围给扼杀了。他对孩子气的白蝶很忠心,跟媚兰也很亲近,比通常哥哥对妹妹的态度还亲,而这世界上又再也找不到比这两位女士更温柔可爱、更不谙世事的人了。
\par 六十年前,白蝶姑妈受洗时被命名为萨拉·简,但在很久很久以前的一天,因为她那双脚步轻盈、永不安定、嗒嗒乱跑的小脚,她那溺爱孩子的父亲便把这一绰号安在她身上\footnote{Pittypat英语里有小脚丫的意思。}。自那以后,便没有人叫过她别的名字。这第二次命名以后的岁月里,她身上却发生了很多变化,使这一爱称变得不太合适。原来那个步履轻快、蹦蹦跳跳的小姑娘不见了,如今只有那两只与她现在的体重极不相称的小脚和欢快天真、漫无目的的说话声还有原来的样子。她身材矮胖,面色粉红,头发银白。由于紧身胸衣束得太紧,总是有点气喘吁吁的。她把两只小脚硬塞进过小的便鞋中,走路顶多能走一个街区远。她那颗心一激动便跳得飞快,而她也总是随它去,一点也不会觉得不好意思。稍受刺激,她便会晕过去。大家都知道,她的昏厥一般情况下都只是小姐般的装模作样而已,但他们太爱她了,肯定不会这么说出来。每个人都很爱她,像孩子一样惯着她,不愿跟她认真——大家都这样,只有她的哥哥亨利除外。
\par 在这世界上,她喜欢闲聊胜过任何事,甚至超过对餐桌上食物的喜爱。她可以一连好几个小时用一种对人无害的友好方式谈论别人的事情。她对人名、日期和地点根本记不住,常常把亚特兰大上演的一出剧里的演员和另一出剧里的演员混为一谈,而这也不会造成任何人因此而被误导,因为没有人会蠢到把她说的话当真,也没有人告诉过她真正骇人听闻或是羞耻可恶之事,因为,虽然年已六十,她那老处女的心态还是应该受到保护的。她的朋友们于是都好心地联合起来,对她就好像对一个需要保护和爱抚的孩子一样。
\par 媚兰很多方面都很像她的姑妈。她像她那样生性羞怯,会突然脸红,还很谦虚,但她确确实实“有点见识——这我得承认”,思嘉心里不甘愿地这么想。像白蝶姑妈一样,媚兰有着一张受着保护的孩儿脸,从来就只知道单纯和善良、真理和爱心。她像个孩子,即使看到艰苦和邪恶的东西,她也辨别不出来。因为她总是非常幸福,非常快乐,所以她想要她周围的每个人也都幸福快乐,至少是想让他们对自己感到满意。为了这一点,她总是看到别人最好的一面,而且会很善意地说出来。在再笨的仆人身上,她也能发现一点忠诚的品德以作补偿。相貌再丑陋、再不可爱的女孩,她也能在她身上发现礼数上的优雅举止和高贵的气质。再没用、再无聊的男人,她也会从他可能有的潜在能力看待他,而不从其现在的样子去看待他。
\par 因为她那颗慷慨善良的心真诚、自然地表现出这些品德,所以大家都聚集在她周围。若一个人总能在别人身上发现一些令人仰慕的优点,而这些优点就连他们自己也都是做梦都不敢想的,那么,有谁能抵挡这样一个人的魅力呢?因为她不具备那种用以俘获男人的心所需要的存心与私心,所以没什么男朋友。可是,她在城里的女性朋友和男性朋友比任何人都多。
\par 媚兰所做的只不过是所有南方姑娘都接受了教育应该去做的——使她们周围的那些人感觉自在,并对自己感到满意。正是这种令人愉悦的女性整体风范,使得南方社会如此令人愉快。女人们知道,男人们若拥有一块土地,对此又感到心满意足、毫无抵触、安全稳妥,又能满足未被揭穿的虚荣心,那这块土地就很可能成为女人们非常令人愉快的居所。为此,从躺在摇篮中起直到走入坟墓为止,女人总是努力使男人们满意,而心满意足的男人则用殷勤和爱意慷慨地回报她们。事实上,男人愿意把世间所有的一切都给予他们的太太,只有聪明这点荣誉除外。思嘉其实是在施展着和媚兰一无二致的魅力,只不过加上了精心研究过的艺术技巧和完美无缺的技艺罢了。两个姑娘的区别在于,媚兰对人说善意讨好的话是出于使别人感到快乐的目的,哪怕是暂时的也成,而思嘉这么做,从来都是为了达到自己的目的。
\par 从他最爱的两个人身上,查理没有受到任何能使他变得坚强的影响,对艰苦境遇或说现实社会也没有学到一星半点的知识,抚养他长大成人的家就像鸟窝一样温暖。和塔拉相比,这个家是如此宁静、老式、温和。对思嘉来说,这座房子在大声呼喊着需要白兰地、烟草和马卡油\footnote{一种发油。}这些雄性的气味,需要粗哑的声音和不时的诅咒叫骂声,需要枪支、威士忌,需要马鞍、马勒和趴在脚边的猎狗。她很想念在塔拉总能听到的吵架声。只要埃伦一转身,这些声音便会响起来——嬷嬷和波克争吵,罗莎和蒂娜拌嘴,还有她自己和苏埃伦的尖刻争论以及嘉乐大声威胁的声音。难怪从这么一个家中长大的查理会成了个女人气十足的胆小鬼。在这里,从来不会有什么激动,也从来不会有人提高说话的嗓门,每个人的意见都只是和别人的意见稍微有点不一样而已,而最后,厨房里那个灰白头发的黑人独裁者便随心所欲、为所欲为了。思嘉曾希望逃离了嬷嬷的监督后可以把马缰放松些,结果却伤心地发现,彼德大叔有关淑女风范的行为标准比嬷嬷的还更严格,对主人查尔斯的遗孀就更是如此。
\par 在这样一个家庭中,思嘉恢复了原来的样子,几乎是连她自己都还没意识到,她的精神就已经恢复正常了。她才只有十七岁,她有的是健康的体魄和旺盛的精力,而查理的家人又竭尽全力使她快乐。如果他们觉得这还不太够,那也不是他们的过错。因为,每当有人提到希礼的名字,她的心就在颤动,谁也无法驱除她心中的这种痛苦。而媚兰又是这么经常地提起他!但媚兰和白蝶都在不辞辛劳地计划着如何抚慰她的悲伤。她们认为,她正受着这种悲伤的折磨呢。她们把自己的悲痛藏起来,好转移她的注意力。她们为她的食物、下午午睡要睡多长时间以及坐马车外出兜风等事情忙个不停。她们不但对她崇拜得过分,崇拜她的满身活力、苗条的身材、小巧的手和脚,白皙的皮肤,而且还经常说出来,用轻拍、拥抱和亲吻来加强她们的亲昵。
\par 思嘉并不在乎拥抱和爱抚,但她对那些恭维倒是感到很舒服。在塔拉,没有人对她说过这么多好话。实际上,嬷嬷老是要杀杀她那自负的气焰。小韦德不再是个烦人的小家伙,因为全家人,包括黑人和白人,还有邻居都很爱他,大家不停地争着让他坐在膝上。媚兰特别溺爱他。即使在他尖叫号哭最厉害的时候,媚兰还是认为他很可爱,而且会说出来,还会加上一句:“噢,你这亲爱的小宝贝!我真希望你是我自己的孩子!”
\par 有的时候,思嘉发现很难掩饰自己的情感,因为她还是认为白蝶姑妈是那些老太太中最为愚蠢的,她的模糊不清和愚蠢的空想使她烦得受不了。她对媚兰的不喜欢则是一种带着妒意的不喜欢,这种不喜欢的程度与日俱增。有时候,当媚兰满脸微笑,带着充满爱意的自豪感谈到希礼或是大声读着他的来信时,她只得突然离开房间。但总的说,这种情况下的生活已经相当快乐了。亚特兰大比萨凡纳或是查尔斯顿和塔拉都更有趣,它还为人们提供了这么多的战时工作,她根本就无暇去思想或是忧郁不乐。可是,有时候,当她吹灭蜡烛,把头埋进枕头中时,她也会叹息着想:“要是希礼还没结婚就好了!要是我不用到那瘟疫般的医院去做护理工作,那又有多好!噢,要是我有几个男朋友就好了!”
\par 她很快就厌恶了护理工作,但她无法逃脱这一职责,因为她同时属于米德太太和梅里韦瑟太太的护理会。这就意味着她一星期得有四天要待在闷热难耐、臭气熏天的医院里,把头发包在一块毛巾里,从脖子到脚则被一块闷热的围裙围起来。亚特兰大的每个妇女,年老的也罢,年轻的也罢,全都参加护理工作,而且干得热情洋溢,这对思嘉来说,简直可以说是一种狂热。她们想当然地认为,她也像她们一样充满爱国热情,要是知道她对战争根本没什么兴趣,她们一定会大吃一惊的。希礼可能会阵亡,这是一直在折磨她的念头。除此以外,战争引不起她丝毫的兴趣。至于护理工作,那是因为她不知如何摆脱才去做的。
\par 确实,护理工作一点也不浪漫。对她来说,这只意味着痛苦的呻吟、神智不清、死亡和难闻的气味。医院里挤满了污迹斑斑、胡子拉碴、虫蝇围绕的男人,他们散发出难闻的气味,身上带的伤可怖骇人,足以使一个基督徒翻胃想呕。医院里发出坏疽的恶臭,臭气直冲她的鼻孔,离门很远便能闻到,一种难闻又带点甜丝丝的气味萦绕在她手上、头发上,连在梦中都困扰着她。苍蝇、蚊子和小虫子成群结队地盘旋在病房上空,嘤嘤嗡嗡地唱着歌,把病人们折磨得诅咒谩骂,无力地呻吟着;思嘉抓着自己被蚊子叮咬的地方,摇着棕榈扇,直到肩膀发疼。于是,她真恨不得所有的男人都死光才好。
\par 然而,媚兰似乎对那些气味、伤口和上身赤裸的男人们毫不在意。思嘉觉得,这对一个最胆小、最羞怯的女人简直奇怪极了。有时候,米德医生切除长了坏疽的肌肉时,媚兰端着脸盆和手术器械站在旁边,脸色也会发白。有一次,做完一次这样的手术后,思嘉发现媚兰在用亚麻布围起来的盥洗室里悄悄地往一块毛巾里呕吐。但是,只要她出现在伤员面前,她便显得极为和蔼、富有同情心,而且很快活,医院里的男人们都叫她慈善天使。思嘉本来也很喜欢这个头衔,但这就意味着要去动那些身上爬满虱子的男人;在烟草块被吞下去时,把手指伸到那些不省人事的病人口里,看看他们是否噎住了;给他们的腿缠上绷带,还要从溃烂的肌肉里往外抓蛆。不,她不喜欢护理!
\par 如果允许她对那些正在康复的男人施展魅力的话,那也许还能忍受,因为他们很多人也很吸引人,出身也很好。但她正在守寡,不能这么做。城里的年轻姑娘们负责康复病区,因为不允许她们去做护理工作,生怕她们会看到不适于少女看到的情景。她们不受已婚或是守寡的遏制,向康复病人发起猛攻。思嘉黯然神伤地注意到,即使是最不吸引人的姑娘,也能轻而易举地使自己跟别人订婚。
\par 除了那些病入膏肓和伤势特重的男人外,思嘉的世界全然是个女性世界,这使她恼怒到极点。她既不喜欢自己的同性,也不相信她们,更糟的是,她总是被女性世界搞得很厌倦。可每星期有三个下午,她还得参加媚兰的朋友们的针线组和卷绷带组。这些姑娘们全都认识查理,在这些聚会上对她都很友好、很有礼貌,特别是范妮·埃尔辛和梅贝尔·梅里韦瑟,城里两位贵妇人的女儿。但她们都对她毕恭毕敬,好像她已是个老妇人,这辈子已经完了。她们不断谈论舞会和男朋友,这使她既忌妒她们的快乐,又为自己的寡妇身份妨碍了自己参加这类活动感到怨恨不已。这是为什么呢?她比范妮和梅贝尔迷人三倍呢!噢,生活多么不公平呀!每个人都认为她的心已经进了坟墓,而事实上一点也没有,这又有多不公平啊!她的心在弗吉尼亚和希礼在一起呢!
\par 然而,虽然有这些不痛快,亚特兰大还是使她很高兴。随着一星期一星期悄悄地过去,她在这儿耽搁的时间也越来越长了。

\subsubsection{第九章}

\par 那个仲夏日的早晨,思嘉坐在卧室的窗口,郁郁不乐地看着窗前经过的货车和马车。车上坐满了姑娘、士兵和作伴的年长妇女。他们高高兴兴地沿桃树街向郊外驶去,那天晚上要为医院举行义卖会,他们是去林区寻找枝叶装点会场。红色的路上,阴影和强烈的阳光交相辉映,上方是搭成拱形的树枝。众多马蹄过处,扬起了一小片红色的尘土。走在其他马车前面的一辆货车,上面坐着四个身强力壮的黑人,他们带着斧头,要去砍冬青树,扯回一些藤蔓植物。货车的后部,高高堆着一些盖着餐巾的大篮子、橡树条篓筐,里面装着野餐用的午餐,还有十几个西瓜。两个黑人还带着班卓琴和口琴,他们正唱着一首经过修改的激动人心的乐曲——“如果你想过得快活,就去参加骑兵”。在他们后面,欢快的车队鱼贯而行:姑娘们穿着凉快的花布裙子,披着精美的披巾,戴着无边女帽和露指长手套以保护她们的肌肤,头顶还遮着小巧的阳伞;在一片欢笑声、马车与马车之间的叫喊声及玩笑声中,上了年纪的妇女们心平气和地微笑着;康复病人挤在身体健壮的陪伴妇女和身材苗条的姑娘们中间,搞得女士们对他们大呼小叫、喧闹不休;骑马的军官们则在马车旁让马悠闲地像蜗牛一样缓缓前行——车轮骨碌碌、马蹄嗒嗒嗒,金色的饰带熠熠生光、小巧的阳伞摇来摆去;扇子沙沙响、黑人在歌唱。每个人都驶出桃树街去采集青枝绿叶,还要在那野餐、吃西瓜。“每个人,”思嘉愁眉不展地想着,“只有我除外。”
\par 他们经过时全都向她挥着手,叫喊着打着招呼,她也试图举止优雅地回礼,但是太费劲了。一丝隐隐的痛楚从她心中涌起,慢慢传到了她的喉咙口,在这便会变成一块硬块,而这硬块很快便会化作眼泪。除了她,每个人都去野餐了。而每个人都要去参加今晚的义卖会和舞会,只有她不行。也就是说,除了她、白蝶、梅利和城里其他正在服丧的不幸的人们。可梅利和白蝶似乎并不在乎。她们甚至连想都没想到要去。思嘉可想到了。而她也确实很想去,特别地想去。
\par 这真是太不公平了。跟城里的姑娘相比,她比谁都加倍努力地工作,为义卖会准备东西。她也织袜子、婴儿帽、软毛毯和围巾,编织了成码成码的花边,在毛发盘和髭须杯上画过画。她还在半打沙发枕套上绣上了南部邦联旗帜(星星绣的有点不像了,确实,有些几乎成了圆形的,其他的则有六个角,甚至七个角。但总体效果还是好的)。昨天,她在一个军械库的旧库房里用黄色、粉色和绿色的干酪包布\footnote{干酪包布是一种粗布。}装饰排列在墙边的货摊,直到干得筋疲力尽。在妇女医院护理会的监督下,这显然是苦差事,而且一点乐趣也没有。在梅里韦瑟太太、埃尔辛太太和怀廷太太旁边,由她们来指挥你干这干那,就好像你是个黑人一样,那是绝对不会有什么乐趣的,还得听她们吹嘘她们的女儿有多受人欢迎。最糟的是,在帮助白蝶和厨娘制作抽彩用的多层蛋糕时,她的手指还被烫了两个泡。
\par 可是现在,像个做农活的黑人般干完活后,她只得有教养地退回家中,而那里的乐趣才刚刚开始。噢,她就得有个死去的丈夫、隔壁房间里还有个呀呀乱叫的婴儿,还得远离一切令人快乐的事,这太不公平了。仅仅在一年多以前,她还在大尽舞兴,穿着靓丽的衣服,而不是这黑乎乎的丧服,而且,实际上等于和三个男孩订了终身。她现在还只有十七岁,还有许多舞曲等着她去跳。噢,这太不公平了!真正的生活就在她眼皮底下、在夏日炎热的气候中一条阴凉的路面上与她擦身而过——一种伴随着灰色制服、嗒嗒的马蹄声、带花的玻璃纱衣裙和班卓琴声的生活。对那些她最熟识的男人,也就是她在医院护理过的男人,她对他们报以微笑,跟他们招着手,但这么做时却要努力使自己不至太热情,可很难使自己不把酒窝露出来,很难使自己看上去整颗心已经进入坟墓——因为实际上并非如此。
\par 她正对外面的人点着头、招着手,这时,白蝶突然走进房间打断了她。白蝶像往常一样,由于爬楼梯而气喘吁吁的,她唐突地把思嘉从窗边拉了回来。
\par “你疯了吗,宝贝,居然在你的卧室窗口对外面的男人招手?我宣布,思嘉,我是太吃惊了!你妈妈会怎么说呢?”
\par “哦,他们不知道这是我的卧室。”
\par “但他们会怀疑这是你的卧室的,那也同样很糟糕。宝贝,你不能做这种事。大家会说闲话,会说你放荡的——不管怎么说,梅里韦瑟太太知道这是你的卧室。”
\par “我想,她会把这告诉所有的男孩的,这只老母猫。”
\par “宝贝,别说了!多利·梅里韦瑟是我最好的朋友。”
\par “哦,那她也同样是只猫——噢,我很抱歉,姑妈,你别哭!我一时忘了这是我卧室的窗口了。我以后不这样了——我——我只是想看看他们经过。我希望我也能去。”
\par “宝贝!”
\par “是的,我希望如此。坐在家里简直腻味透了。”
\par “思嘉,答应我以后别再说这种话了。人们会说闲话的。他们会说你对可怜的查理连应有的尊重都没有——”
\par “噢,姑妈,你别哭!”
\par “噢,现在我把你也弄哭了。”白蝶啜泣着,那样子却似乎是高兴的,一边还把手伸到裙子口袋里去掏手帕。
\par 那一丝隐隐的痛楚终于传到了思嘉的喉咙口,她大声哭了起来——并不是像白蝶所想的是为可怜的查理而哭泣,而是街上那最后的车轮声和欢笑声已渐渐远去了。一阵衣裙的沙沙声响处,媚兰从她的房间里匆匆走了进来,眉头紧锁,一副担忧的样子,手里还拿着一把梳子,平常梳得整整齐齐的黑头发从发罩里放了下来,微微拳曲的头发波浪般披散在脸颊周围。
\par “亲爱的!怎么回事?”
\par “查理!”白蝶哭泣着,完全陷入因痛苦所带来的快感中,把头埋在梅利的肩膀上。
\par “噢,”梅利说着,提到她哥哥的名字,她的嘴唇也抖动了。“坚强些,亲爱的。别哭。哦,思嘉!”
\par 思嘉已经扑倒在床上放声大哭,为她逝去的青春而哭,为青春所能带来的快乐而她却被拒之门外而哭。她带着孩子般的愤愤不平和伤心绝望大声哭着,孩提时她曾经用哭泣就能得到自己想要的东西,而现在,她知道,哭泣再也帮不了她了。她把头埋在枕头里,一边哭泣,一边还用脚踢着有绒毛的床罩。
\par “我还是死了的好!”她极动情地哭着。在思嘉发泄这些痛苦以前,白蝶那易落的眼泪已经止住了,梅利于是飞奔到床边去安慰她的嫂嫂。
\par “亲爱的,别哭了!你想想查理有多爱你,你就可以得到安慰了!想想你那亲爱的宝贝吧。”
\par 思嘉因被误解而感到愤恨不已,这和自己被一切事情排斥在外的那种凄凉感掺杂在一起,使她哽咽得说不出话来。这反倒是一种幸运,因为如果她能说出话来的话,她就会像嘉乐那样直截了当地把真心话哭叫出来。媚兰拍着她的肩膀,白蝶则踮起脚尖,却又脚步沉重地在房里走来走去,把窗帘拉了下来。
\par “别拉!”思嘉从枕头上抬起一张涨得通红的脸,大声叫道。“我还没断气呢,那才要你把窗帘拉下来呢——可我最好还是死掉的好。噢,请你们都出去吧,让我独自待着!”
\par 她又重新把头埋进枕头里,站在她身边的两个人低声商量了一会,蹑手蹑脚地出去了。她们下楼梯时,她听到媚兰低声对白蝶说:
\par “白蝶姑妈,我希望你以后不要再对她提起查理了。你知道的,这对她的影响总是很大。可怜的思嘉,她脸上的表情很怪,我知道她是拼命想忍住不哭的。我们不该使她更难过的。”
\par 思嘉愤恨万分,无力地踢着床罩,想骂几句脏话来发泄发泄。
\par “去他娘的!”她终于说了出来,多少感到好受了一些。媚兰怎么能够心满意足地待在毫无乐趣可言的家里,为她哥哥戴着黑绉纱呢?她才只有十八岁呀。媚兰似乎根本不知道,或者根本就不在乎,生活正踏着嗒嗒的马蹄声匆匆而过呢。
\par “可她只是根芦柴棒,”思嘉心里想着,用拳捶打着枕头,“她从来没有像我那样受欢迎过,所以她不会想要我想要的东西。而且——而且她得到了希礼,而我——我谁也没有得到!”想到这一新的悲哀,她不禁又重新放声大哭起来。
\par 她忧郁哀伤地待在房间里,一直待到下午。后来,她看到了野餐归来的人们,马拉货车上堆满了松树枝、藤类和蕨类植物。但这并没有使她快活起来。每个人都是一副倦容,但都很高兴,他们又向她招手打招呼,她闷闷不乐地回着礼。生活毫无希望,当然就不值得过下去。
\par 解脱终于降临了,这是她根本没有预料到的。午饭后午睡的时候,梅里韦瑟太太和埃尔辛太太坐着马车来了。在这种时候有人来访,媚兰、思嘉和白蝶都感到很吃惊,她们赶忙起床,匆匆忙忙钩上紧身胸衣的背钩,梳理好头发,下楼来到客厅里。
\par “邦内尔太太的孩子们得了麻疹。”梅里韦瑟太太出其不意地说,显然是在说明,她认为邦内尔太太居然让这种事发生,那她本人就得为此负全部责任。
\par “而麦克卢尔家的姑娘们又被叫到弗吉尼亚去了。”埃尔辛太太用她那慢吞吞的声音说道,她忧虑地摇着扇子,好像不管是这件事,还是别的事都无关紧要似的。“达拉斯·麦克卢尔受伤了。”
\par “太可怕了!”主人们一齐叫道,“可怜的达拉斯他——”
\par “不。子弹只是穿过了肩膀,”梅里韦瑟太太欢快地说,“可这事发生在这种时候,没有比这更糟的了。姑娘们要到北边去把他接回家来。但是,老天在上,我们可没有时间坐在这聊天。我们得赶紧回到军械库去把装饰工作做完。白蝶,我们今晚需要你和媚兰来代替邦内尔太太和麦克卢尔家姑娘们。”
\par “噢,可是,多利,我们不能去的。”
\par “别对我说‘不能’,韩白蝶,”梅里韦瑟太太厉声说道,“我们需要你去看管着那些负责点心饮料的黑人们。那原来是邦内尔太太做的。媚兰,你就去照顾麦克卢尔家姑娘们的摊子。”
\par “噢,我们只是不能——可怜的查理死了才一——”
\par “我知道你们是怎么想的,但是,为了我们的事业,做出再大的牺牲也不为过。”埃尔辛太太用一种软绵绵但却是一锤定音的声音插话说。
\par “噢,我们是很愿意帮忙,但是——你们为什么不找些可爱漂亮的姑娘去照顾摊子呢?”
\par 梅里韦瑟太太大声地哼了一声,就像在吹号一样。
\par “我真不知道,这些日子姑娘们脑袋瓜里想的是什么。她们一点责任感都没有。所有还没有答应看管货摊的姑娘们都有数不清的借口。噢,她们骗不了我!她们只是不想被阻碍,好去接近那些军官,原因无非就是这个。她们还担心她们那些新衣服在货摊后面没人看得见。但愿那个闯封锁线的人——他叫什么名字来着?”
\par “白船长。”埃尔辛补充说。
\par “我希望他多带些医院器械过来,少带些有裙环的裙子和花边来。如果今天我看到一件裙子,我肯定要看到他弄进来的二十件裙子。白船长——我听到这个名字就不舒服。好了,白蝶,我没有时间跟你争了。你必须来。大家都会理解的。再说你在后边的房间里,没有人会看见你的,而梅利也不会太引人注目的。可怜的麦克卢尔家的姑娘们的摊子在通道的尽头,也不很漂亮,没人会注意到你的。”
\par “我认为我们必须去。”思嘉说,尽量掩饰着自己的急迫心情,脸上则露出一副真诚、单纯的样子。“我们也只能为医院做这点事了。”
\par 两个来访的太太都没提到她的名字。她们转过身,目光锐利地看着她。即使在她们人手最紧的时候,她们也没有考虑过要让一个守寡才一年的寡妇在社交场合露面。思嘉大睁着眼睛,带着一副孩子般的神情迎视着她们的目光。
\par “我想,我们应该去帮忙,把这次义卖会搞成功,我们大家都得去。我认为我必须和梅利一起去看管货摊,因为——哦,我觉得我们两人出现在那比只有一个人看上去会好一些。你不这样认为吗,梅利?”
\par “哦。”梅利无助地说道。还在服丧的时候就在公开的社交集会上抛头露面,这种事她连听都没听说过,所以觉得茫然失措的。
\par “思嘉是对的。”梅里韦瑟太太看到媚兰有退让的迹象,便这么说道。她站了起来,把裙环整理好。“你们俩——你们大家都得来。好了,白蝶,别又开始摆你的借口了。想想医院多需要钱买新的病床和药品吧。我知道查理会喜欢你们为他已经为之献身的事业出力的。”
\par “那好吧,”白蝶说,在一个比她个性更强的人面前,她总是感到毫无办法,“假如你认为大家能理解的话。”
\par  
\par “这太棒了,简直是真的!太棒了,太让人无法相信了!”思嘉不引人注意地悄悄走进那被装饰成粉黄两色的货摊时,心里在欢唱着。这货摊原是属于麦克卢尔家的姑娘们的。她实际上就等于在参加晚会了。在被隔离了一年之后,在过了一年戴着黑绉纱,连话也不敢大声说的日子之后,在她烦闷得几乎要发疯的时候,她实际上又在参加晚会了,而且是亚特兰大举办过的最大型的晚会。她可以看到许多人,许多灯,可以听音乐,还能亲眼见识那个出名的白船长上次闯封锁线弄进来的漂亮的花边、女上衣和褶边。
\par 她在货摊柜台后边的一张小凳子上坐下来,上下打量着这长长的大厅。直到今天下午为止,这里还只是个空荡荡的丑陋难看的操练场呢。那些太太小姐们把它打扮成现在这副美丽的模样,今天她们做了多少工作呀。它看上去漂亮极了。今晚,亚特兰大的每一根蜡烛、每一个烛台全都集中到这里来了吧,她暗自思忖着,有可以插十二根蜡烛的银烛台;烛台座上围着可爱迷人的小雕像的瓷烛台;还有老式的铜制蜡烛架,它们直立在那儿,一副颇为尊贵的样子,上面放满了形色各异的蜡烛,散发着月桂果的芳香,长长地排列在大厅里的枪架上有,装饰着鲜花的桌子上有,货摊柜台上也有,就连大开的窗户上的窗台上也有。一阵阵夏日温暖的和风不大不小,正好把烛光吹得闪闪烁烁的。
\par 在大厅的中央,那盏又大又难看的吊灯原来是由锈迹斑斑的链条从屋顶倒挂下来的,现在已经被缠绕在一起的常春藤和野葡萄藤完全给改变了。受灯光的映照,叶子已经软恹恹的。墙边排着松树板凳,散发出阵阵香味,把大厅的角落变成供年长妇女和上了年纪的老太太闲坐的好去处。到处挂满常春藤、葡萄藤和菝葜藤缠绕在一起的雅致的藤条,挂在墙上环形的彩饰架上,装饰在窗户上,还缠绕在色彩明快的干酪布搭成的货摊的扇状彩饰上。而在青枝绿叶丛中,在旗帜和旗布上,到处闪烁着南部邦联那点缀在红蓝背景下的星星。
\par 为乐师们准备的高出地面的平台特别艺术化。它被一排排青枝绿叶及装点着星星的旗布遮住,完全看不见了。思嘉知道,城里每盆盆景都搬到这来了:锦紫苏、天竺葵、八仙花、夹竹桃、秋海棠——连埃尔辛太太视若珍宝的四盆橡胶植物也被摆在了四角尊贵的位置上。
\par 从平台看过去,在大厅的另一头,太太们已经把自己隐蔽起来了。这面墙上挂着戴维斯总统和南部邦联的副总统史蒂芬斯的巨幅画像,佐治亚人称斯蒂芬斯为我们自己的“小亚历克”。画像上方是一面很大的旗帜,画像下方的长桌上则是从城里的花园里“劫掠”来的鲜花:有凤尾草、成排的玫瑰,有红的、黄的和白的,还有剑兰那傲气十足、像剑一般的叶片,一簇簇五颜六色的旱金莲,笔直高挺的蜀葵那深紫和米色的花朵从其他花后探出头来。在它们中间,蜡烛就像祭坛里的火苗一样安详地燃烧着。那画像上的两张脸往下俯视这一场景,对两个掌握着如此伟业的男人,没有比这两张脸的差别更大的了。戴维斯脸颊扁平,目光冷酷,像个苦行僧一样,两片傲气的薄嘴唇紧抿着;斯蒂芬斯则两眼凹陷,黑色的眼睛炯炯有神,似乎除了疾病与痛苦外,什么也不知道,而且已经用诙谐和火焰征服了它们——这是两张深受爱戴的脸。
\par 整场义卖会的责任就落在护理会那些上了年纪的太太手里,她们像装备齐全的船一样,庄重地开了进来,催着那些迟到的年轻太太和笑声吟吟的姑娘们到各自的货摊上去。然后,她们便一阵风似的走进后面的房间里去了,那里正在摆放点心饮料呢。白蝶姑妈气喘吁吁地跟在她们后面。
\par 乐师们爬上平台,他们都是黑人,满脸漾着笑,胖胖的脸上因出汗已经闪闪发亮了。他们开始调试小提琴,郑重其事地提早用弓在琴上拉着、拨着。梅里韦瑟太太的车夫老利瓦伊此刻正拨着琴弓以引起其他乐手的注意。自亚特兰大被命名为马撒斯维尔起,他就一直是每场义卖会、舞会和婚礼晚会的乐队指挥。除了主持义卖会的太太外,来的人还不多。尽管如此,所有的眼睛都朝利瓦伊望过去。接着,小提琴、低音大提琴、手风琴、班卓琴和骨片琴一齐低声演奏起《洛雷纳》来——音乐声太低,不适合跳舞。舞会要等货摊上的东西都卖光后才开始。优美、抑郁的华尔兹舞曲传到思嘉耳里,她觉得自己的心跳都加快了:
\refdocument{
    \par “时间年复一年慢慢地流逝,洛雷纳!
    \par 草地上又落满了洁白的雪花。
    \par 太阳早已西斜落山,洛雷纳……”
}
\par 一——二——三,一——二——三,下蹲——摆——三,转——二——三。多美的华尔兹舞曲啊!她微微伸出双手,闭上眼睛,和着那伤感、萦绕在脑际的节奏摆动起来。悲哀的旋律里,某种东西和洛雷纳失去的爱情及她自己的激动心情纠缠在一起,使她喉咙里似被一块硬块堵住似的。
\par 接着,就像被华尔兹乐曲吸引来的一样,从被月光照得斑斑驳驳的街上,各种声响飘了进来,马蹄声、车轮声、馨香的空气中飘荡的笑声以及黑人那虽然柔和却刻薄的争吵声,他们正在争拴马的地方呢。楼梯上一派忙乱而欢快适然的景象,姑娘们肆意的说话声和陪伴她们的男人低沉的声音夹杂在一起。虽然那天下午才刚刚分手,可认出朋友时,姑娘们还是欢快地叫喊着打招呼,兴高采烈地尖叫着。
\par 转瞬间,大厅便生机盎然了。厅里挤满了姑娘们——姑娘们拥了进来,她们穿着像蝴蝶一样靓丽的衣裙,裙环把下摆撑得宽宽的,镶着花边的长裤在裙子底下若隐若现;她们裸露着浑圆、小巧又白皙的双肩,镶着花边的荷叶边上方,柔软、小巧的乳房的轮廓隐约可见;带花边的披巾随意地从手臂上垂挂下来;用金属片装饰和绘着画的扇子,用天鹅绒毛和孔雀羽毛做的扇子,被姑娘们用细细的丝绒缎带系在腰间,摇摇晃晃的。满头黑发的姑娘们则把头发从耳际平滑地梳在脑后,挽成颇有分量的发髻,使得她们的头也稍稍后仰,一副傲慢无礼的样子;有着一头金色鬈发的姑娘们则任由头发披散在脖颈周围,带有饰物的金色耳坠荡来荡去的,和金色的鬈发一起翩翩起舞。花边、丝绸、镶边和缎带全都是穿过封锁线暗地里运进来的,因此也就更加珍贵,穿戴起来便更加神气。花枝招展的华丽服饰被加进了一种傲气,人们把这也当做对北方佬的一种附加的刻意冒犯。
\par 并不是城里所有的鲜花都被放在南部邦联的领袖面前当贡品。最小巧、最芳香的花成了姑娘们身上的饰物。茶玫瑰被夹在粉红的耳朵后,栀子花和含苞欲放的玫瑰花蕾被串成小小的花环,戴在如瀑布般从两侧垂下来的发丝上,还有优雅地插在锦缎腰带上的鲜花,不等天亮,这些花就会作为珍贵的纪念品,成为灰色军服胸袋里的物件。
\par 人群中军服攒动,不计其数——这么多思嘉认识的人都穿着军服,有她在医院里的吊床上遇见过的,有在街上遇见过的,还有在操练场上遇见过的。这些军服如此华丽,显出勇敢的气度,纽扣闪闪发亮,袖口和领口上缠绕在一起的金色镶边令人眩目,裤子上有红黄蓝三色条纹用以区别军队中不同的军种,把灰色的制服衬托得完美无瑕。深红和金色的腰带闪来闪去,军刀熠熠生辉,碰到亮闪闪的靴子,使上面的靴刺咯咯作响。
\par 这么多英俊的男人,思嘉想着,心里一股骄傲感油然而生。他们互相打着招呼,向朋友们招手致意,弯腰亲吻着上了年纪的太太们的手。他们看上去全都那么年轻,虽然他们留着弯弯的髭须及黑色和棕褐色的连鬓胡子,但还是那么英俊,那么鲁莽妄为。他们手臂还吊着悬带,头上缠着的绷带在被太阳晒得棕褐色的脸上白得令人讶异。有些人还拄着拐杖,姑娘们只好小心地放慢脚步,好和她们跳跃着前进的陪伴者合拍。那些姑娘们多自豪啊!军服中还有一种眩目的色彩使姑娘们的华丽服饰黯然失色,在人群中分外醒目,就像热带地区的一只小鸟一样——那是个路易斯安那义勇军,他穿着宽大的蓝白相间的条纹裤,米色有绑腿的高筒靴,红色紧身小上衣,是个脸色黝黑、满脸是笑、像个小猴子似的人,一只手臂还吊在黑色的丝悬带里。他是梅贝尔·梅里韦瑟的专任男朋友——勒内·皮卡德。全医院的人都出动了,至少是能走动的每个人,还有所有在休假和休病假的人、铁路部门和邮政部门的每个人、医院和军需部的所有的人,只要是这里和梅肯之间的,全都来了。太太们会多高兴呀!医院今晚一定可以筹到一笔巨款。
\par 下面的街上传来轻声敲着的鼓声、沉重的脚步声,还有马车夫羡慕的叫喊声。军号响过,一个低沉的声音叫着口令,解散了队列。转瞬间,穿着色彩明快的制服的城卫队和民兵的队员们拥进了房间,把狭窄的楼梯挤得直摇晃。他们弯腰鞠躬,向人们打招呼,和别人握着手。他们都是城卫队员,为在战争时期能够参加城卫队而感到无比自豪,他们向自己许诺,只要战争能打到明年,明年这个时候他们就要到弗吉尼亚去参战。花白胡须的老头子,穿着沾了在前线浴血奋战的儿子辈的官兵们的光的制服,也在队列里行进,同样感到无比自豪,只希望自己能更年轻一些。在民兵中,也有许多中年人和一些年纪更大一些的人,但零零星星也有一些适合参军年龄的人,他们就不像比他们年长或年幼的人那么神气活现了。已经有人在窃窃私语,询问他们为什么没和李将军\footnote{指罗伯特·李(1807—1870),美国内战中南部邦联弗吉尼亚军总司令。}一起作战。
\par 他们这么多人怎么可能都挤进大厅里去呢!几分钟以前看上去还是偌大的地方,如今已挤得水泄不通了。夏夜的空气中散发着香囊、古龙水及发油的香味,加上燃烧的月桂香蜡烛味,温馨而宜人。花香阵阵,由于众多脚步踩踏在训练用的老旧的地板上,还微微扬起了一片尘土。吵闹声和喧哗声使人们几乎什么也听不见。老利瓦伊似乎也感觉到这一时刻的兴奋和激动,《洛雷纳》演奏到半途中便被他停了下来。他用弓尖利地敲击了一下,然后拼命一拉,乐队便一齐奏起了《美丽的蓝旗》。
\par 上百个声音一齐随乐曲唱了起来,唱得很大声,就像是在欢呼一样。城卫队的号手爬上平台,正好在合唱开始时跟上了音乐的节拍。高昂、清越的颤音盖过了众人的合唱,使人们裸露的胳膊上顿时起了鸡皮疙瘩,一股铭心刻骨的情绪给人们带来了一阵阵寒冷彻骨的寒意:
\refdocument{
    \par “万岁!万岁!南方的权利万岁!
    \par 只有一颗星的
    \par 美丽的蓝旗万岁!”
}
\par 他们又一齐唱起了第二段。思嘉正和别人一块唱着,突然听到身后响起了媚兰高亢、甜美的女高音,声音既清晰又真诚,就像号声一样令人心旌摇荡。思嘉转过身,看到梅利双手交叉着放在胸前。她站在那,双目紧闭,眼角渗出了泪花。音乐结束时,她神情古怪地微笑着望着思嘉,一边用手帕轻轻拭泪,一边噘着嘴表示歉疚。
\par “我太高兴了,”她低声嘟哝着,“为士兵们感到无比自豪,我便情不自禁地流泪了。”
\par 她眼里有一种深沉,几乎是不可思议的神采。有一刻,把她那张毫无特色的小脸蛋映照得熠熠生辉,使它看上去变得挺漂亮。
\par 歌曲结束时,所有女人的脸上都带着同样的表情。她们的脸上挂着骄傲的泪花,粉嫩的脸蛋如此,满布皱纹的老脸也不例外。她们嘴角挂着微笑,眼里则闪着深沉且热情洋溢的光芒。女人们转而面对她们的男人,姑娘们面向她们的心上人,母亲面对她们的儿子,妻子面对她们的丈夫。她们全都因为那看不见的美而显得很漂亮,而当一个女人受到全然的保护和被全心全意地爱着,并且以上千倍的热情回报这种爱时,这种看不见的美甚至能使最普通的脸也变得漂亮起来。
\par 她们爱自己的男人,她们相信他们,她们便信任他们,至死不渝。有这么一支穿着灰色军服的坚强的部队屹立在她们和北方佬之间,灾难怎么可能落到她们头上呢?有史以来,什么时候有过像他们这样的男人呢?他们英勇崇高、不甘寂寞、风度翩翩,却又温情无限。他们所从事的事业如此公正、正义,这项事业除了战无不胜之外还可能会有什么别的结果吗?她们热爱这项事业,就像爱她们的男人一样,她们用自己的双手全心全意地为这种事业服务,她们谈论这一事业,思考这一事业,做梦也想着这一事业——如果需要的话,她们会为它牺牲她们的男人,而且会为这种损失感到无比自豪,就像男人们自豪地举着战旗一样。
\par 在她们内心深处,这正是献身的高潮,骄傲的高潮,是南部邦联的高潮,因为最后的胜利马上就要到来了。石墙杰克逊\footnote{指汤姆森·杰克逊(1824—1863),内战期间为南部邦联的重要将领之一。1861年他组建了著名的“石墙旅”。后来,“石墙”成了杰克逊的绰号。}在山谷的胜利和里士满附近发生的“七天战役”\footnote{指1862年6月25日至7月1日间在弗吉尼亚首府查门德地区南北双方的一场恶战。此次战役中南部邦联损失惨重,但罗伯特·李却在此战后声名大噪。}中,北方佬的挫败已经清楚地说明了这一点。有像李和杰克逊这样的领导,除了这样的结果,还可能是别的结果吗?再来一次胜利,北方佬就会跪在地上要求投降,男人们就可以骑着马回家,接下来就是接吻和欢笑了。再来一次胜利,战争就永远结束了!
\par 当然,一些家庭会空着一些椅子没人坐,还有的孩子永远也见不着父亲的面孔了。弗吉尼亚的寂寞的小溪边和田纳西宁静的高山上会留下一些没有墓碑的坟墓。但是,为了这样一个事业,这种代价难道会太大吗?太太小姐们的丝绸,还有茶、糖等不容易买到,但那只是笑料谈资了。再说,那些冲劲十足的偷闯封锁线的人正从北方佬满脸不高兴的鼻子底下把这些东西带进来呢,这使得买这些东西的价钱贵了好几倍。但很快,拉斐尔·西麦斯和南部邦联的海军就会去收拾北方佬的炮舰,各港口就会门户大开的。英国也会来帮南部邦联赢得战争的胜利的,因为英国的棉纺厂正无事可干,等着南方的棉花呢。自然,英国贵族是同情南部邦联的,这正如贵族会同情贵族一样,他们也不喜欢像北方佬那类只爱美元的人。
\par 这样,女人们便把丝绸衣裙弄得窸窣作响,笑出声来,心里充满自豪地看着她们的男人。她们知道,面临危险和死亡而成的姻缘总是和奇妙的激情同时并存的,而因了这种激情,这种爱就加倍地美妙。
\par 起初,思嘉望着人群,心也在怦怦乱跳。因为身临晚会现场,浑身也有了种不习惯的激动情绪。但是,当她半明不白地看到周围人的脸上那心高气盛的神情时,她的高兴劲渐渐消失了。在场的每个女人都因某种情感而神采奕奕的,而这种情感她却毫无感觉。这使她茫然失措,心情沮丧。不知怎的,大厅似乎不那么漂亮了,姑娘们打扮得也没有那么美丽了,而还在每张脸上熠熠生辉的那股献身事业、已达白热化程度的热情似乎——哦,这似乎只是太傻了!
\par 她突然意识到,她并没有像其他女人一样,享有无上的自豪感、牺牲自己的愿望以及她们为事业所拥有的一切,这不禁使她因吃惊而张大了嘴巴。万分恐惧之下,她不禁想到:“不——不!我不能想这些事!它们是错误的——有罪的。”她知道,这事业对她来说一点意义也没有,听到其他人眼里闪耀着那种不可思议的神情谈论它,她感到厌烦极了。对她来说,这事业好像根本就不神圣。战争似乎不是神圣的事,而是令人讨厌的事。它不仅毫无理性地杀戮男人,而且花费钱财,还容易使高档物品紧缺。她明白,对没完没了的编织、卷绷带及捡棉绒等差事,她已经厌烦透了,它们使她指甲的表层都变粗了。还有,噢,她对医院也讨厌透了!对那正在恶化的坏疽的味道和没完没了的呻吟,她真是既讨厌又烦心,厌恶透顶,那些情绪低落的脸上将死的神情也使她感到害怕。
\par 这些背叛性、亵渎性的思绪掠过她脑际的时候,她偷偷地看了看周围,担心有人会发现她这些想法正明白无误地写在脸上。噢,为什么她就没有这些女人那样的感觉!她们献身事业的热情发自内心、全心全意、情感真挚。她们所说的话和所做的事都是认真的。而如果有谁怀疑她——不,没有人会知道的!虽然她对事业毫无感觉,她还是必须继续装出满腔的热情和自豪感,还得扮演一个勇敢承受痛苦的南部邦联军官的遗孀,一个心已进入坟墓的女人。如果丈夫的死为事业的胜利出了一份力,她还得有他死而无憾的感觉。
\par 噢,为什么她和这些充满爱心的女人格格不入、一点也不一样呢?她从来就无法像她们一样无私地热爱任何东西或任何人。这种感觉多么孤单无助啊——而不论从身体或是情感上说,她过去可都是从来没感到孤单寂寞的呀。起先,她试图把这种想法遏制住,然而,她骨子里包含的那股固执的诚实个性不允许她这么做。所以,在义卖会进行过程中,当她和媚兰为光顾她们货摊的顾客服务时,她的思想却在不停地忙活着,试图对自己证明自己是对的——对付这种差事,她极少时候会感到很困难的。
\par 其他女人都在傻乎乎、歇斯底里地谈论着爱国主义和事业,男人们也好不到哪里去,他们正在谈论着关键的问题和州权。只有她,郝思嘉,才有良好、冷静的爱尔兰人的理性。她不会为了这事业把自己变成傻瓜,也不会去承认自己的真实想法而让自己变成傻瓜。她很冷静,在这种情况下,她会讲求实际,谁也不会知道她是怎么想的。如果他们知道她的真实想法,那参加义卖会的人会感到多么震惊呀!如果她突然爬上音乐台,宣布她认为战争必须停止,这样每个人就都可以回家去照看棉田,而且重新开办晚会,重新有男朋友和许许多多浅绿色的衣裙,那人们又会多么惊恐啊!
\par 有一会,她的自我辩解使她振作了一些,但她还是厌恶地环顾着大厅。正如梅里韦瑟太太说过的,麦克卢尔家姑娘们的货摊一点也不显眼,而且有时很长时间都没有人走到她们这个角落来。这样,思嘉便无所事事,只是忌妒地看着快乐的人群。媚兰感觉到她忧郁的心情,但是认为她是在思念查理,所以并不跟她说话。媚兰自己忙着在货摊上摆弄着物品,让它们看起来更诱人一些。思嘉却坐在那,闷闷不乐地环视着大厅。就连戴维斯先生和斯蒂芬斯先生画像底下摆成一排一排的鲜花也没有使她高兴起来。
\par “它看上去就像个祭坛,”她对之嗤之以鼻,“瞧他们对那两个人的热乎劲,他们最好还是把他们当成是上帝和他的儿子!”接着,她突然被自己的大不敬吓了一大跳,急忙在自己身上画十字表示歉疚,恰到好处地恢复了正常的神态。
\par “哦,这是真的,”她和自己的良心争辩着,“大家都将他们奉若神明,可他们什么都不是,只是普普通通的人,而且相貌一点也不吸引人。”
\par 当然,对自己看上去相貌如何,斯蒂芬斯先生是无能为力的,因为他一辈子都是个残废,可戴维斯先生——她抬头看着那张浮雕刻就的整洁、骄傲的脸。最使她不安的就是他的山羊胡子了。男人要不就把胡子剃干净,留着上唇的髭须,要不就留络腮胡子。
\par “那一小束胡子看上去是他唯一能做的了。”她心想,看不出他那张脸上带有冷静、不容怀疑的智慧,而这智慧正承受着一个新国家的负荷。
\par 不,她现在一点也不快乐,而起先她还为能置身于人群中而欢呼雀跃呢。现在看来,仅仅在场是不够的。她在义卖会现场,但她并不是其中的一员。没有人注意到她,她是在场的唯一一个既没有男朋友又没有丈夫的年轻女性。她这一辈子曾经是舞场的中心。这太不公平了!她还只有十七岁,双脚正在地上踏着拍子呢,她想跳舞。她只有十七岁,却有个躺在奥克兰墓地中的丈夫和躺在白蝶姑妈家的摇篮里的婴儿,而且,每个人都认为她必须认命。她的酥胸比在场的任何一个姑娘的都更白,腰也更细,脚也更小巧,但他们大家都认为,她最好是躺在查理身边,墓地上刻着“某某某的爱妻”。
\par 她既不是一个能去跳舞、同男人调情的姑娘,也不是个能和其他太太坐在一起、对跳舞和姑娘们的调情说三道四的妻子。而她年纪并不大,还没有老到当寡妇的年龄。寡妇应该是年纪很大的妇人——非常非常老,老到不想跳舞,不想跟人调情,也不想被别人仰慕。噢,她还只有十七岁,却不得不一本正经地坐在这,做个有尊严、合礼数的寡妇的典范,这是不公平的。有男人,还有有魅力的男人来到她们的货摊时,她就得把声音放低,谦逊地垂下眼睑,这是不公平的。
\par 亚特兰大的每个姑娘都被男人里三层外三层地包围着。即使是最普通的女孩都很有市场,就像漂亮姑娘一样——而且,噢,最糟糕的是,她们都穿着如此漂亮、如此好看的衣服!
\par 她像只乌鸦似的坐在这,穿着闷热、乌黑、袖子长及手腕的塔夫绸衣服,扣子直扣到下巴,连一点花边和镶边都没有,什么首饰也不能戴,只有埃伦服丧用的缟玛瑙胸针,只能眼睁睁地看着俗气的姑娘们吊着英俊男人们的膀子。这一切都是因为查理得了麻疹。他居然没死在战斗中英勇奋战的时候。若是那样的话,她还可以对此吹吹牛皮。
\par 她抗议似的把胳膊肘撑在货柜上,望着熙熙攘攘的人群,心里嘲笑着嬷嬷经常重复的告诫,说是不能撑着胳膊肘,要不会把胳膊肘弄难看的,还会起皱纹。胳膊肘变丑了又有什么关系呢?她也许再也没有机会把它们露出来炫耀了。她热切地望着来来往往的人群:带着玫瑰花苞的花环;黄色的波纹绸;有十八片荷叶边、边沿饰有小巧、黑色的天鹅绒缎带的粉色缎子;淡蓝色的塔夫绸,裙摆有十码宽,瀑布般的花边如水花飞溅;若隐若现的酥胸;魅力十足的鲜花。梅贝尔·梅里韦瑟挽着那义勇军的胳膊,朝隔壁的货摊走去。她穿着苹果绿的薄纱裙,裙摆很宽,把她的腰衬得如此纤细,就像小得没有了似的。裙子上镶满了奶油色的香蒂叶荷叶\footnote{法国北部一小镇,以生产花边出名。}花边,这是最近一次偷闯封锁线的人从查尔斯顿弄进来的。梅贝尔穿着它如此招摇,就好像偷闯封锁线的是她,而不是那个著名的白船长。
\par “要是我穿着那衣服,那会有多漂亮啊!”思嘉想着,心里忌妒得要命。“她的腰像头牛一样粗。那种绿色是最适合我的颜色,它能使我的眼睛看上去——为什么金发女人要试着穿这种颜色的衣服呢?她的皮肤看上去就像是陈奶酪一样发绿。想想看,我再也不能穿这种颜色的衣服,即使出了服丧期也不能穿了。不,即使我想法再嫁了也穿不了了。那时我就只好穿难看的灰色、褐色和淡紫色的了。”
\par 有一瞬间,她想到了所有的一切都不公平。寻欢作乐、穿漂亮衣服、跳舞、卖弄风情的时间真是太短暂了!只有几年时间,太短暂了!接着你便结婚了,穿着色彩灰暗的衣服,还生儿育女,这便毁了你的腰身。这以后,舞会上你就只有和有节制的已婚妇女一起坐在角落里,只能跟你的丈夫跳舞,或是和会踩你脚的老先生们跳舞。如果你不这么做,其他妇人就会对你评头论足,接着你的名声就被毁了,你的家庭也会蒙受耻辱。把做姑娘时的全部时间都花在学习如何表现得迷人而有魅力上面,花在如何抓住男人的心上面,然后却只有一两年时间用来应用这些知识,这真是太浪费了。她想起在埃伦和嬷嬷手里受训的时候,知道对她的训练很彻底,效果也很好,因为总是硕果累累。是有一套规则要遵守,而一旦你依规则而行的话,你的努力就会被冠之以成功的花环。
\par 对付老太太,你就得温柔坦率,尽量表现得天真无邪,因为老太太待人尖刻,她们就像猫一样妒意十足地望着姑娘们。只要姑娘们的言语或眼神稍有不慎,她们马上就会扑过去。对付老先生,女孩子就得冒冒失失、天真活泼,几乎要有点轻佻,但又不能太过分,这样,那些老傻瓜们的虚荣心就会被逗得痒痒的。这会使他们觉得自己精力充沛、还像年轻人一样,他们便会在你的脸上拧一把,声称你是个调皮鬼。当然,在这种情况下,你总是要面泛红晕,要不他们就会更加没有分寸、更加高兴地拧你,然后就会告诉他们的儿子,说你很放荡。
\par 对年轻姑娘和少妇们,你就得满口甜言蜜语的。每次见到她们都得去吻她们,即使是一天十次也得如此。你得用手环抱着她们的腰肢,还得忍受她们也如此对待你,不管你多么的不喜欢。你得不分青红皂白地对她们的衣服和孩子表示羡慕,开诸如男朋友之类的玩笑,称赞她们的丈夫,咯咯发笑也得有节制,还得矢口否认你的魅力比她们强。此外,最重要的是,你决不能说出你对某件事情的真正看法,她们若告诉你她们真正是怎么想的,你也就只能说这么多,决不能说得更多。
\par 对其他女人的丈夫,你就得正正经经地对他们不予理睬,就连你自己抛弃过的男朋友也不行,就算他们再吸引人也白搭。如果你对这些年轻的丈夫们太好的话,他们的太太就会说你很放荡,你因此就会名声不好,再也找不到自己的男朋友。
\par 但是对年轻的单身汉——那就不一样了!你可以对他们柔声大笑,他们就会飞奔到你身边,想弄明白你为什么发笑。你呢,可以不告诉他们,而且笑得更厉害,让他们围在你身边,一心想找出你笑的原因。你可以用眼神向他们许诺无数件令人激动的事,让一个男人设法跟你单独在一起。而一旦和你单独在一起,并且试图吻你时,你可以表现得受了很深、很深的伤害,或是非常非常生气的样子。你可以让他因为自己的卑鄙行径而向你道歉。因为你如此可爱地原谅了他,他就会试图再次吻你。有时候,当然不能太经常,你确实也可以让他吻你(埃伦和嬷嬷没有教她这一招,但她知道这很有效。)接着你就哭起来,声称自己也不知道这是怎么啦,而他再也不会尊重你了。然后他就会给你擦干眼泪,通常还会向你求婚,以显示他有多尊重你。接下来还有——噢,能对单身汉做的事太多了,她全都知道:斜目而视的细微差别、用扇子遮遮掩掩的似笑非笑、把屁股扭来扭去好让裙子像铃铛一样摆来摆去,还有眼泪、笑声、奉承和亲切感人的同情心。噢,所有的技巧从来都没有失效过——除了对希礼。
\par 不,学会了所有这些精明的技巧,又只用了这么短时间,然后就得永远永远地把它们丢在一边,这似乎是不对的。若是永远不结婚,就这么穿着淡绿色的裙子,一辈子这么活泼可爱下去,而且一直有英俊的男人求婚,那该有多好啊!但是,如果你耽搁太久的话,你就会像英蒂那样变成老姑娘,大家都用暗自窃喜、充满敌意的样子对你说“可怜的人儿”。不,即使不能再有什么乐趣,毕竟还是去结婚以保持自尊来得更好。
\par 噢,生活真犹如一团乱麻!在所有的人当中,她干嘛这么傻,偏偏要嫁给查理,以致才十六岁就把美好的生活给断送了呢?
\par 人群开始往墙边靠,她愤愤不平、毫无希望的思绪也被打断了。太太们小心地托着裙环,以免别人不小心把裙环碰翻起来,不合适地露出太多的长裤。思嘉踮起脚尖,从人群头顶上望过去,看到民兵队长登上乐台。他大声喊着口令,一半民兵成员很快排成了一列。有一会工夫,他们表演了生气勃勃的操练,这使他们的额头上都冒出了汗珠,也赢得了观众们的欢呼声和掌声。思嘉也和其他人一道责无旁贷地鼓着掌。队员们解散后,向前奔往卖甜饮料和柠檬汁的摊点。她转向媚兰,觉得最好还是尽快开始装出自己对事业充满热情的样子来。
\par “他们看上去棒极了,不是吗?”她说。
\par 媚兰正忙活着柜台上的编织品。
\par “他们中大多数人要是穿着灰色的军装,身在弗吉尼亚,那会棒得多,”她说,根本没有费心去压低声音。
\par 有几个民兵队员的母亲正满心自豪感地站在附近,无意中听到了这些话。吉南太太脸刷地红了,接着又转白,因为她二十五岁的儿子威利也在民兵队伍中。
\par 这些话从大家都喜欢的媚兰嘴里说出来,思嘉简直惊呆了。
\par “噢,梅利!”
\par “你知道的,这一点也没错,思嘉。我并不是指那些小男孩和老先生。但许多民兵队员是完全能够去扛枪打仗的,那正是他们此刻本该做的事。”
\par “可是——可是——”思嘉支吾着,她过去从没考虑过这个问题。“总该有人留在家乡——”威利·吉南解释他之所以留在亚特兰大时是怎么告诉她的呢?“总得有人留在家乡保护这个州,使它免受侵略。”
\par “没有人在侵略我们,也没有人会来侵略我们,”梅利冷淡地说,两眼朝一群民兵队员望过去,“阻挠侵略者入侵的最好办法就是到弗吉尼亚去,到那去打北方佬。至于说民兵留在这防止黑人们造反的论调——唉,那是我听到过的最愚蠢的事了。我们的人为什么要造反呢?这只是懦夫们的好借口罢了。如果各州所有的民兵都到弗吉尼亚去的话,我敢打赌,北方佬一个月就会被打败的。就是这样!”
\par “哦,梅利!”思嘉再次叫起来,目瞪口呆的。
\par 梅利温柔的黑眼睛因生气而亮闪闪的。“我的丈夫并不害怕到那去,你的丈夫也没有害怕。我宁愿他俩都牺牲在那,而不愿他们待在家里——哦,亲爱的,真对不起。我是多么考虑不周,多么残忍啊!”
\par 她动情地抚摩着思嘉的胳膊,思嘉则盯视着她。但她想的不是死去的查理,而是希礼。假设他也会牺牲呢?她马上转过身,米德医生走到她们的货摊时,她机械地微笑着。
\par “哦,姑娘们,”他向她们打着招呼,“你们能来真是太好了。我知道,你们今晚出来一定是做出了很大的牺牲。但这都是为了事业的缘故。我要告诉你们一个秘密。我有个令人吃惊的办法,能在今晚为医院筹到更多的钱,但是恐怕有些太太小姐们会因此大为震惊。”
\par 他停下不说了,一手捋着山羊胡子,一边笑出声来。
\par “噢,什么办法?快告诉我们!”
\par “我再一想,我相信我也会让你们捉摸不透的。但如果教会会员要把我逐出城去的话,你们这些姑娘们得站出来为我说话。不管怎样,这也是为了医院。你们会明白的。过去从来没有人做过这种事。”
\par 他自鸣得意地朝一群年长妇女走去了。两位姑娘刚转过身想谈谈这个秘密可能会是什么时,有两位老先生在她们的货摊上弯腰看着东西,大声宣布要买十英里长的梭织花边。唉,有老先生来毕竟也比根本没有先生来更好,思嘉边这么想着,边量着花边,然后一本正经地把花边夹在下巴下折好。两个上了年纪的浮荡之人付了钱,朝卖柠檬汁的货摊走去了,又有其他客人取而代之,来到货摊前。她们的货摊不如别的货摊客人那么多,因为别的货摊有的回荡着梅贝尔·梅里韦瑟的尖声大笑,有的因范妮·埃尔辛的咯咯笑声和怀廷家姑娘们的智言妙语而有一片欢快景象。梅利像个售货员一样静静地、安详地把毫无用处的东西卖给先生们,而这些先生们根本就不可能会去用这些东西,思嘉也学着梅利的样子照样做着。
\par 其他人的货摊前全都围着一群一群的人,只有她们的没有。在其他货摊前,女孩子在叽叽喳喳说着话,男人们则在买东西。有几个到她们这来的人跟她们讲的是他们怎样和希礼一起上的大学,他是个多出色的士兵,或是用尊敬的语气谈到查理,说他的死对亚特兰大来说是个多大的损失。
\par 接着乐队突然演奏起《约翰尼·布克,帮帮这黑人》这首旋律欢快的歌曲,思嘉觉得她都要尖叫起来了。她想跳舞。她太想跳舞了。她从地上望过去,脚和着音乐踏着步点,绿色的双眸因十分热切而熠熠生辉。在大厅的另一头,一个刚来的人站在门口,他看到了她们,开始想把她们认出来。他目不转睛地注视着那张闷闷不乐、颇有反抗精神的脸上那双斜行的眼睛。当他认出了那种对别人传递出的邀请时,他不禁对自己笑了,这一点任何男人都看得出来的。
\par 他穿着黑色绒面呢布料做的衣服,个子很高,比站在他身边的军官们都高出一截。他肩膀很宽,从肩部到腰间渐渐变窄,腰却挺细,脚更是小得可笑,脚上的靴子擦得锃亮。他那身肃穆的黑西装,配上上好的有褶边的衬衫和潇洒地绑在高帮鞋面里的裤子,看上去怪怪的,跟他的身材和脸型显得极不协调,因为他打扮得像个纨绔子弟,而高大魁梧的身材穿着一身花花公子的衣服,看似懒散雅致,其实这其中潜伏着危险。他的头发乌黑发亮,黑色的髭须不宽,修剪得很短,和近旁那些骑兵们修剪得漂漂亮亮、如飞鹰般的胡须相比,看上去几乎有点外国气派。他看上去像是个欲望十足、毫无廉耻的人,而实际上也确实如此。他一副狂妄自大的神态,那傲慢无礼的样子令人感到颇为不快。他盯着思嘉看时,大胆的眼里闪耀着一丝邪恶的光芒。最后,思嘉感觉到了他的注视,也朝他看去。
\par 在她的头脑里,记忆之河开始流淌,但此时此刻,她还记不起来他是何许人。可他是几个月来第一个对她有兴趣的男人,她于是对他嫣然一笑。他点头致意时,她微微回了个礼。但当他挺直腰,迈着特别轻巧自如的印第安人般的步态向她走来时,她惊恐地用手遮住了嘴巴,因为她知道这人是谁了。
\par 她就像被雷击中似的,站在那动弹不得,而他正穿过人群朝她走来。她茫然地转过身,弯下身子想逃到点心房去,但她的裙子被货摊上的一个钉子给钩住了。她愤怒地猛地一拉,用力扯着。可转瞬间,他已经站在她身边了。
\par “让我来吧,”他说,弯下身子解开荷叶边。“我绝没想到你会记得我,郝小姐。”
\par 他的声音出奇地悦耳,像位绅士那样抑扬顿挫的,既洪亮又带有查尔斯顿人那种平平的、慢吞吞的长音。
\par 她抬头哀求似的看着他,上次见面时的羞辱使她涨红了脸。她看到了一双她所见过的人中最乌黑的眼睛,眉飞色舞的,既欢快又毫无怜悯心。在这出现的世界上所有的人中,只有这个可怕的人曾亲眼看见了她和希礼的那一幕,这至今还让她做噩梦呢;这个可恶可耻的人曾毁了女孩子的名声,好人都不愿接受他;就是这个卑鄙小人曾经说过她不是个淑女,而且还很有理由。
\par 听到他的声音,媚兰转过身来。思嘉平生第一次因为她小姑的存在而真诚地感谢上帝。
\par “哦——是——是白瑞德先生,对吗?”媚兰淡淡地一笑,把手伸给他,“我见过你——”
\par “在宣布你订婚的那个幸福的时刻,”他接着她的话说下去,弯腰吻她的手。“你还记得我,真是太谢谢了。”
\par “大老远从查尔斯顿到这来,来做什么呢,白瑞德先生?”
\par “生意上一件烦人的事,卫太太。从现在起我得经常进出你们这个城市了。我发现我不但要把货物弄进来,还得负责把它们卖掉。”
\par “弄进来——”梅利开口说道,眉头皱了起来,接着便高兴地笑了。“哦,你一定是我们经常听说的那位声名远扬的白船长吧——闯封锁线的人。噢,这里的每个姑娘都在穿你弄进来的衣服呢。思嘉,你难道不为此感到高兴——怎么啦,亲爱的?你是不是要晕倒了?快坐下。”
\par 思嘉一屁股坐在凳子上,呼吸非常急促,她甚至担心紧身胸衣的系带会绷断。噢,发生这种事有多可怕啊!她从来没想到会再碰到这个人。他从货柜上抓起她黑色的扇子,开始满心焦虑地给她扇着,非常非常的焦虑。他一脸严肃,可眼睛却还是欢呼雀跃着的。
\par “这里面很热,”他说,“难怪郝小姐会发晕。我能不能送你到窗户边去?”
\par “不用。”思嘉说,口气很不礼貌,梅利不禁盯着她看。
\par “她不再是郝小姐了,”梅利说,“她现在是韩太太。她是我嫂嫂了。”梅利疼爱地微微瞟了她一眼。思嘉觉得,白船长那黝黑、海盗般的表情真要使她窒息了。
\par “我敢肯定,这对两个漂亮的太太来说都是莫大的收获。”他说,微微鞠了一躬。这是所有男人都会说的话,可他这么说时,她似乎觉得他的所指恰恰相反。
\par “我相信,在今晚这幸福的时刻,你们的丈夫都在这吧?和熟人再见见面是件令人高兴的事。”
\par “我丈夫在弗吉尼亚,”梅利说着自豪地扬起了头,“但查理——”她的声音哽住了。
\par “他死在军营中了。”思嘉平淡地说。她几乎是尖刻地说出这些话的。这个家伙再也不走开了吗?梅利吃惊地看着她。船长做个手势,表示自责。
\par “亲爱的太太——我怎么能这样!你们得原谅我。请允许一个陌生人说句安慰的话,为自己的国家而死就是永生。”
\par 媚兰透过泪眼向他微笑着,思嘉却觉得有只盛怒且充满恨意的狐狸在撕咬着她的五脏六腑,可她对此却无能为力。他又说了一句漂亮话,也就是任何先生在这种情况下都会说的那种恭维话,但他根本不是在说真心话。他是在嘲笑她。他知道她根本不爱查理。梅利看不透他,真是天大的傻瓜。噢,上帝,别再让别的人看透他了,她心里惊恐地想着。他会不会把他知道的说出来呢?当然,他不是什么正人君子。如果人们不是正人君子的话,那谁也不知道他们会做出什么事情来。没有什么标准可以衡量他们的。她抬起头看着他,看到他即使在摇着扇子时嘴角也瘪着,一副嘲弄的同情样。他那副表情里有某些东西挑起了她的情绪,心里一阵厌恶之感使她重新聚起了力量。她突然从他手里夺过扇子。
\par “我没事,”她尖刻地说,“没有必要把我的头发扇得乱七八糟的。”
\par “思嘉,亲爱的!白船长,你得原谅她。她——她一听到有人提到可怜的查理的名字就会失态——也许,我们今晚根本就不该到这来。你知道的,我们还在服丧,这让她的头脑绷得太紧了——这种欢快场面和音乐,可怜的孩子。”
\par “我很能理解。”他说话的语气特别重,转身面对着媚兰,探询似的看了她一眼,直望到她那可爱、焦虑的眼睛深处去。这时,他的表情变了,黑色的面孔上换上了颇不情愿的尊重和温情。“我认为你真是个勇敢的年轻贵夫人,卫太太。”
\par “他对我就不说这些话!”思嘉气愤地想。梅利不解地笑了,回答道:
\par “我的天哪,白船长!医院护理会非要我们来照看这个货摊,因为最后时刻——要个枕头套?这有个挺漂亮的,上面绣有一面旗。”
\par 有三个骑兵出现在她的柜台前,她转身去应付他们了。有一刻,媚兰都在想,白船长真是太好了。然后,她又希望在她的裙子和放在货摊外面的痰盂之间能有比干酪包布更坚固的东西,因为那些满嘴琥珀色烟草汁的骑兵吐痰时可不像他们打长马枪时瞄得那么准。再下来,更多的客人挤到她的货摊前,她就把船长、思嘉和痰盂统统忘到脑后去了。
\par 思嘉一言不发地坐在凳子上扇着扇子,头也不敢抬,她心里真希望白船长回到他船上的甲板上去。
\par “你丈夫去世很久了吗?”
\par “哦,是的,很久了。差不多一年了。”
\par “那真的是千古了。”
\par 思嘉也说不准千古是什么意思,但他的声音里有诱惑的成分,这点是错不了的。于是,她不说话。
\par “他死时你结婚很久了吗?请原谅我问这些问题,我已经很久没有来这个地方了。”
\par “才两个月。”思嘉说。心里老大不情愿。
\par “简直是个悲剧。”他继续用轻松的语调说道。
\par “噢,去他妈的,”她心里狂怒地想着,“如果他是别的什么人,我就可以拉下脸来叫他滚开。但是,他知道有关希礼的事,知道我不爱查理。这样的话,我的手脚就被捆住了。”她还是不说话,低头看着扇子。
\par “这是你第一次在社交场合露面吗?”
\par “我知道这似乎很荒唐,”她赶快解释,“但要照看这个货摊的麦克卢尔家的姑娘们临时被叫走了,又没有其他人来顶替,所以媚兰和我——”
\par “为了事业,再大的牺牲也是值得的。”
\par 怎么,这不是埃尔辛太太说过的话吗。但她说的时候,不是用这种口气说的。思嘉生气的话刚想出口,但又强忍住了。毕竟她到这来不是为了事业,只是因为她在家里坐腻了。
\par “我一直在想,”他若有所思地说,“服丧这种制度,把妇女下半辈子的生活禁锢在黑绉纱里,禁止她们享有正常的乐趣,这和印度自焚一样野蛮。”
\par “自焚?”
\par 他笑了,她则因自己的无知涨红了脸。她恨那些使用她不懂的字眼说话的人们。
\par “在印度,一个男人死后实行火葬,而不是土葬,死者的妻子总是爬上火葬用的柴堆,跟尸体一块烧死。”
\par “那多可怕啊!他们干吗要这么做呢?警察对此也不管吗?”
\par “当然不会管。不把自己烧死的寡妇会成为社会的渣滓。所有那些受人尊敬的太太们都会因为她没有像个有教养的大家闺秀那样行事而对她说三道四——假如你今晚穿着红裙子、在舞会上领舞,坐在角落里的那些太太们也会这样对你评头论足的。我个人意见,随夫自焚也比我们南方这种活埋寡妇的可爱习俗仁慈多了。”
\par “你怎么敢说我被活埋了呢?”
\par “妇女们把捆束她们的锁链抓得多紧啊!你认为印度的习俗野蛮——但是,如果不是南部邦联需要你,今晚你敢在此露面吗?”
\par 这种关于性格特点的讨论总是令思嘉感到很困惑。而他的话就更是令她感到加倍不解了,因为她隐隐觉得,他的话里也有对的地方。但现在应该是把他驳倒的时候了。
\par “当然,我不会来的。要不就可能会——哦,对……不尊重——那就像是我不爱——”
\par 他的眼神在等着她说下去,含着玩世不恭的嘲弄意味。她不能说下去了。他知道她没爱过查理,他也不让她装出她应该表现出来的那种礼貌的情绪来。跟这么一个不是正人君子的人打交道是多么多么可怕的事啊。若是正人君子的话,他就总是会表现得完全相信一个淑女太太的话,就算他知道她明明在说谎也是如此。这就是南方人的骑士风度。一位绅士总是遵守一切规则,说适宜的话,想方设法使生活对一个淑女太太来说更容易一些。可这个人似乎根本就不在乎规则,而且,对没人谈过的事,他显然却津津乐道。
\par “我正屏住呼吸等着你说下去呢。”
\par “我觉得你太可恶了。”她说。无助地垂下了眼睛。
\par 他从柜台上倾过身来,直到他的嘴巴凑近了她的耳朵边嘶嘶发声。他模仿着雅典娜大厅里经常出现的舞台上那种反面人物的样子,模仿得像极了:“不用怕,好太太!你那有罪的秘密在我这非常安全!”
\par “噢,”她低声说道,情绪非常激动,“你怎么能说这种话!”
\par “我只是想放松一下你那紧张的神经。你要我说什么呢?‘做我的女人吧,漂亮的小姐,要不我就把一切都抖出来’?”
\par 她颇不情愿地迎视着他的目光,看到他的眼神就像个小男孩在戏弄人似的。她突然放声大笑起来。毕竟这种情势太可笑了。他也笑了,笑得很大声,以至角落里几个年长妇女都朝他们这边看。看到韩查理的寡妇和一个完全陌生的人相处得如此快乐,她们把头凑在一起,不以为然地议论开了。
\par  
\par 一阵鼓声响起,接着是一片“嘘”声。米德医生登上平台,挥着手让大家安静。
\par “我们大家都应该真诚地感谢这些迷人的太太小姐们,她们那坚持不懈的爱国之举不但使这次义卖会在捐款方面获得了极大的成功,”他开始说道,“而且把这个乱糟糟的大厅变成了怡人的居家之所,变成了一个在我周围到处可见迷人的玫瑰花蕾的美丽花园。”
\par 每个人都鼓掌表示赞同。
\par “太太小姐们都做出了最大努力。她们不但花了时间,而且用自己的双手付出了劳动。货摊上的漂亮物品更是加倍地漂亮,它们正是经由我们南方妇女的巧手制造出来的。”
\par 又有了更多的喊声表示赞同,白瑞德此时正毫不经意地斜靠在思嘉身边的柜台上,低声嘀咕着:“他是只浮华的山羊,对不对?”
\par 她吃了一惊,起先简直是惊呆了,这是对亚特兰大最受爱戴的公民的大不敬,她满眼责备地盯视着他。但医生那灰白下巴上的小胡子正晃动得厉害,看上去确实像只山羊,她拼命忍住才没笑出声来。
\par “但仅有这些是不够的。医院护理会的太太小姐们曾用她们的妙手抚平了许多因倍受折磨而皱起的眉头,还从死神嘴里挽回了我们在战斗中受伤的勇敢的官兵们的生命,而这些战斗是我们所有事业中最英勇的。她们是知道我们的需要的。我在此不一一举例了。我们需要更多的钱以购买从英国来的医疗器械和药品。今天晚上,和我们在一起的还有已成功地闯越封锁线达一年之久,而且,为了给我们带来我们需要的药品,还将继续这么做的英勇无畏的船长,白瑞德船长!”
\par 虽然因自己的名字被突然提到而措手不及,这个闯封锁线的人还是优雅地鞠了一躬——太优雅了,思嘉这么想着,试图对他的举动加以评价。几乎可以这么说,因为他对在场的每个人都如此蔑视,所以他似乎是礼貌得过头了。他这么鞠躬时,人群爆发出一阵掌声,角落里的太太们纷纷探头观望。这么说,和可怜的韩查理的寡妇厮混在一起的就是这个男人了!而查理死了还不到一年!
\par “我们需要更多的金子,我只好向你们要了,”医生继续说下去,“我要求你们作出牺牲,可这种牺牲跟我们穿着灰色制服的勇敢的战士作出的牺牲比起来,简直太微不足道了,小得似乎令人觉得很可笑。太太小姐们,我要你们的珠宝首饰。是我要你们的珠宝首饰吗?不,是南部邦联需要你们的珠宝首饰。南部邦联号召你们献出来,我也知道绝没有人会不愿意的。可爱的手腕上戴着个闪亮的珠宝镯子有多漂亮啊!我们爱国的太太小姐们胸前戴着发亮的金制胸针又有多漂亮啊!然而,比起印第安纳州所有的金子和珠宝来,牺牲来得还更漂亮!金子要被熔化,宝石被出售,所得的钱便用来购买药品和其他医疗器械。小姐太太们,有两个勇敢的伤员将提着篮子走过你们面前,而——”可他余下的话已经被暴风雨般的掌声、欢呼声和喧哗声盖掉了。
\par 思嘉的第一个念头就是感到庆幸,因为在服丧期间,她不能戴她那珍贵的耳环和那条挺重的金项链,那都曾是外祖母罗比亚尔的饰物。也不能戴那金黑两色的珐琅质手镯及石榴红胸针。她看见那个小个子义勇兵,那只没有受伤的胳膊上挎着一个橡木条编织的篮子,在她边上的大厅里的人群中转来转去,还看见妇女们,年长的也罢,年轻的也罢,嘻嘻哈哈却又迫不及待地卸下手镯,从穿了耳洞的耳朵上解下耳环,同时还假装痛得叫出声来。她们互相帮忙着解开项链的钩子,从胸口上解下胸针。不时的有金属碰撞金属的叮当声和叫喊声,喊着“等等——等等!我现在已经解下来了,喏!”梅贝尔·梅里韦瑟正把戴在胳膊肘上的一对可爱的手镯取下来。范妮·埃尔辛叫着:“妈妈,我可以吗?”也把别在鬈发上的小粒珍珠头饰取下来,这头饰在这家已经传了好几代人了。每有一件赠品放入篮子里,就响起一阵掌声和欢呼声。
\par 满脸是笑的小个子男人现在正朝她们的货摊走来,他胳膊上挎着沉重的篮子,走过白瑞德身边时,一个漂亮的金烟盒被随意地扔进了篮子。他走到思嘉面前时,把篮子放在柜台上稍事休息。她摇了摇头,双手摊开,示意她没什么好给他的。成了在场的人中唯一一个没东西可给的人,确实令人难堪。这时,她看到了大大的结婚戒指在闪着光。
\par 有一刻,她颇感困惑地试图回忆一下查理的脸——他把戒指戴在她手上时表情是怎么样的。但记忆模糊了,被一时的恼怒情绪弄模糊了,而对他的回忆总是给她带来这种恼怒的情绪。查理——正是他使她的生活就此结束、使她成为像老妇人般的女人。
\par 她猛地想卸下戒指,但被卡住了。义勇兵已向媚兰走去了。
\par “等一等!”思嘉叫道,“我有东西要给你!”戒指被卸下来了。正当她要把戒指扔进堆满手链、手表、戒指、胸针和手镯的篮子时,她注意到白瑞德的目光。他嘴角露出了一丝淡淡的微笑。她示威似的把戒指扔到那堆物件的顶部。
\par “噢,亲爱的!”梅利低声叫道,抓住了她的手臂,眼里闪耀着爱和自豪的光芒。“你真是很勇敢、很勇敢的姑娘!等一等——请等一下,皮卡德中尉!我也有东西要给你!”
\par 她在卸自己的结婚戒指。思嘉知道,自从希礼把它戴上去之后,它就从来没有离开过那个手指。其他人不知道,但思嘉知道,这戒指对她来说意味着什么。戒指好不容易被卸了下来,有一刻,它被紧紧地握在她小小的手心里。接着,它被小心地放在那堆首饰上面。两个姑娘站在那目送着义勇兵向角落里那群上了年纪的老妇人走去,思嘉满心对抗,媚兰的目光里满含同情,比眼泪所能表达的同情还更多。这两种表情都没有逃过站在她们身边的那个男人的眼睛。
\par “如果你没有勇气这么做,我也决不会有勇气这么做的,”梅利说着,把手环在思嘉的腰上,轻轻地按了按。思嘉突然想把她的手甩掉,尽力大喊一声“上帝保佑!”就像嘉乐被弄得烦躁不安时那样。但她看到白瑞德的目光,只好挤出一丝辛酸的微笑。梅利总是误解她行事的动机,这真令人不安——但若让她怀疑这是否是真的,那还不如让她误解好了。
\par “多美的姿态啊!”白瑞德轻声说道,“正是你们的这种牺牲精神在激励着我们那些穿灰色军服的小伙子们。”
\par 嘴里激烈的言辞欲脱口而出,她好不容易才把它们硬吞回去。他不管说什么都带着嘲讽的意味。她打心眼里不喜欢他,瞧他靠在货摊上那一副懒洋洋的样子。但他身上有一股激人向上的东西,这东西温馨、有活力、令人惊心动魄。她身上所有爱尔兰人的个性特点促使她起来向他那乌黑的眼睛挑战。她决定要把这人打下一两个台阶来。他知道她的秘密,这确实令人气恼,所以,为了改变这一点,她得让他处于某种不利的境地。她很想告诉他她对他的真实看法,但硬压下这股冲动。正像嬷嬷常说的,糖总是比醋更能吸引苍蝇,她决定要抓住这只苍蝇并使他屈服,这样,他就再也不能对她表示怜悯了。
\par “谢谢,”她柔声说道,故意曲解他的嘲讽,“从白船长这样的名人嘴里说出来的赞扬话确实值得感激。”
\par 他把头朝后一仰,放声笑了起来——简直是在狂吠,思嘉盛怒之下是这么想的,她的脸又一次涨得绯红。
\par “你干吗不把真实想法说出来?”他问道,放低了声音,使这话在嘈杂而激动的人群中只有她一个人能听清,“你干吗不说我是个该死的无赖、小人,我必须从这滚开,要不你就要叫这些穿灰色制服的勇敢的小伙子中的一个来把我赶出去?”
\par 她很想刻薄地加以反击,可话到嘴边又极力忍了回去,改口说道:“哦,白船长!你真是喋喋不休个没完!好像没人知道你有多出名、有多勇敢,是个——是个——”
\par “我对你太失望了。”他说。
\par “失望?”
\par “是的。在我们头一次重大会晤中,我还认为我终于碰到一个不只是漂亮而且还很勇敢的姑娘。可现在我才发现,你只是漂亮罢了。”
\par “你意思是说我是个胆小鬼?”她气得就像是只正在发怒的母鸡。
\par “一点也不错。你没有勇气说出你的真实想法。我初次见到你时,我就想:这姑娘真是个一百万个里难寻一个的姑娘。她不像其他这些傻里傻气的小傻瓜一样,相信她们的嬷嬷告诉她们的所有事,并且依样而行,却不管自己感觉如何。她们把所有的情感、欲望和微小却令人伤心的事用许多好听的话掩饰起来。我曾想:郝小姐这个姑娘有着令人罕见的活力。她知道她自己想要的是什么,根本不在乎把自己的想法说出来——或是摔花瓶。”
\par “噢,”她说,已经义愤填膺了,“那我现在就把我的真实想法说出来。哪怕你稍有一点教养的话,你就不该走到这来跟我说话!你该知道我再也不想见到你!可你不是个正人君子!你是个肮脏龌龊的杂种!你以为你那朽烂的小船能够逃脱北方佬的防线,你就有权利到这来嘲笑这些勇敢的男人和为事业作出一切牺牲的女人吗——”
\par “停下,停下——”他笑着制止她,“你的开场白说得好极了,而且说出了你的真实想法。但是,请不要跟我谈这事业,我对这些论调已经厌烦透顶了。我敢打赌,你也一样——”
\par “怎么,你怎么——”她又开口道,情绪很不稳定,接着她很快地控制了一下情绪,为自己陷入了他的圈套气得七窍生烟。
\par “你还没看见我,我就站在门口看着你了,”他说,“我也看了其他姑娘。她们的脸看上去全都像是从一个模子铸出来的。你的却不是。你的脸很容易让人家看透。你对你做的事并不用心,我敢打赌,你根本没有想着我们的事业和医院。你想跳舞,想玩个痛快,可你又不能这么做,这全都在你的脸上写着呢。你被看穿了,所以恼羞成怒。跟我说实话,我说的对不对?”
\par “我跟你没什么可说的了,白船长,”她尽力正经八百地说,努力把自己身上残余的自尊碎片拼凑起来,“就因为你是‘伟大的偷越封锁线的人’你就倍受欢迎,但这一点并没有赋予你侮辱妇女的权利。”
\par “伟大的偷越封锁线的人!真会开玩笑。请把你宝贵的时间再匀一点给我吧,要不你就让我冤死了。我不想让一个这么迷人的小爱国者对我对南部邦联的事业做出的贡献产生误解。”
\par “我并不在乎听你吹吹牛皮。”
\par “我是在做偷闯封锁线的生意,也确实在从中赚钱。一旦我不能从中赚钱的话,我就会停止不做的。你对此怎么看?”
\par “我觉得你是个唯利是图的无赖——就像北方佬一样。”
\par “说得太对了,”他咧嘴笑了,“北方佬也帮着我赚钱呢。嗯,上个月我把船直开到纽约港去,装了满满一船货物。”
\par “什么!”思嘉不禁饶有兴趣、激动万分地叫了起来。“他们没用炮把你轰成灰呀?”
\par “可怜的小天真!当然没有。北部联邦也有许多坚定的爱国者并不反对向南部邦联出售物品以从中赚钱。我把船开到纽约港,从北方佬的公司购买货物,当然是暗地里的交易,然后我便离开。要是有了一点危险,我就到拿骚去,还是这些北部联邦坚定的爱国者在那会给我弄到火药、炮弹和有群环的裙子。这比到英国去方便多了。有时候,闯到查尔斯顿或威尔明顿去有点困难——可是,你要是知道金子用处到底有多大,你一定会惊诧不已的。”
\par “噢,我知道北方佬很卑鄙,但我不知道——”
\par “干吗对北方佬出卖联邦、诚实地赚取一分钱吹毛求疵呢?一百年后就根本没关系了。结果还是一样的。他们知道,南部邦联最终是会被打败的,这样的话,他们为什么不从中赚取钱财呢?”
\par “打败——我们?”
\par “当然。”
\par “能不能请你离开我呢——或者说,有没有必要我去把马车叫来,回家去,好甩掉你?”
\par “好个恼怒的南方小叛兵。”他说,又突然笑了一下。他鞠了一躬,逍遥自在地走开了,把她留在那,胸部因白白地生气冒火而剧烈地起伏着。她心中填满了失望之感,自己却无法辨别,就像是一个孩子看到虚幻的东西消失之后有的那种失望之情一样。他怎么敢美化那些偷闯封锁线的人!他又怎么敢说南部邦联会被打败!他真该为此被枪毙——像个叛国者那样被枪毙。她环顾整个大厅,看着那些熟悉的面孔,他们对成功如此信心百倍,看上去如此勇敢、如此衷心。不知怎么的,她心里不禁掠过了一丝淡淡的寒意。被打败?这些人——哦,当然不会的!这个想法本身就是不可能的,不忠诚的。
\par “你们俩在嘀咕什么呀?”媚兰问道,转身面对着思嘉,因为她的客人都陆续走了。“我忍不住看了梅里韦瑟太太一下,注意到她始终都把眼睛盯在你身上。亲爱的,你知道她的嘴巴有多厉害。”
\par “噢,这个男人不可能——他是个没有教养的乡巴佬。”思嘉说道,“至于梅里韦瑟这个老太太,让她去嚼舌根好了。就为了她的缘故,我的行为举止就得像个傻瓜似的,对此我简直厌恶透了。”
\par “怎么啦,思嘉!”媚兰叫了起来,惊异极了。
\par “嘘——嘘,”思嘉说,“米德医生又有事情要宣布了。”
\par 医生提高了嗓门,人群又一次静了下来。医生先是对太太小姐们自愿献出自己的首饰表示谢意。
\par “现在,女士们,先生们,我要提一个令人吃惊的建议——这项改革可能会使你们中的一些人感到震惊,但我恳请你们记住,这一切都是为了医院及躺在医院里的伤病员。”
\par 大家都满心希望地慢慢往前挤,心里揣摩着这个严肃的医生会提出什么令人震惊的提议来。
\par “舞会马上就要开始,第一支舞曲当然是弗吉尼亚舞,紧接着是华尔兹。接下来是波尔卡舞、苏格兰舞和波兰舞,前面都由短短的弗吉尼亚舞开始。我知道得很清楚,领跳弗吉尼亚舞的人选还有点小小的竞争,所以——”医生擦了一把额上的汗水,嘲弄似的扫了墙角一眼,他的太太正跟其他上了年纪的女人一起坐在那呢。“先生们,如果你想和你中意的太太或小姐领跳弗吉尼亚舞,你得竞价才行。我来充任拍卖商,所得收入归医院。”
\par 许多正在扇着的扇子都突然停了下来,大厅里一片激动的低语声。老太太们所在的角落哗声大作,打心眼里不同意却又急于支持她丈夫的举动的米德太太也就处于极为不利的境地。埃尔辛太太、梅里韦瑟太太和怀廷太太气得满脸通红。可城卫队却突然发出了一片欢呼声,其他穿着制服的客人也高声附和着。年轻姑娘们拍手赞成,激动得欢呼雀跃的。
\par “你不觉得这是——这像是——有点像是黑奴拍卖会?”媚兰低声问道,心里没底地注视着跃跃欲试的医生。迄今为止,他在她心目中的形象一向是完美无缺的。
\par 思嘉什么也没说,但两眼发亮,可心里却因隐隐的痛楚在一阵阵抽紧。要是她不是寡妇就好了,要是她还是从前的郝思嘉,穿着苹果绿的裙子,胸前垂挂着深绿色的天鹅绒锻带,乌黑的头发上别着晚香玉,亭亭玉立地站在舞池里——她就能领跳弗吉尼亚舞了。是的,一定会那样的。肯定会有一打的男人争相为她竞价,把越来越高的钱付给医生。噢,可她现在却必须无奈地坐在这,违背自己的意愿,在舞会上做个受人冷落的小可怜,眼睁睁地看着范妮或梅贝尔作为亚特兰大的美女领跳弗吉尼亚舞!
\par 一片嘈杂的声音中传来了小个子义勇兵的声音,他的克里奥尔\footnote{克里奥尔人常指下面几种人:生于拉丁美洲的欧洲人后裔;美国墨西哥湾沿岸各州早期法国或西班牙殖民者的后裔;上述两种人与黑人或印第安人所生的混血儿等。}口音非常明显:“可以的话——我为梅贝尔·梅里韦瑟小姐出二十美元。”
\par 梅贝尔红着脸倚靠在范妮的肩上,两个姑娘把脸埋在对方的颈项里,咯咯直笑。这时,又有其他的声音叫着其他人的名字,出其他的价格。米德医生只得又面带微笑,对角落里传来的护理会的妇女们愤慨的嘀咕声完全置之不理。
\par 起先,梅里韦瑟太太态度冷淡,大声声明她的梅贝尔决不参加这种活动;可随着梅贝尔的名字被叫到的次数越来越多,价钱也渐渐升到七十五美元,她的抗议声便开始减弱了。思嘉双肘支在柜台上,对那些蜂拥在乐台周围、手里满是南部邦联发行的纸币、满心激动而欢笑的人群几乎可以说是怒目而视。
\par 现在他们全都可以跳舞了——只有她和那些老太太除外。每个人都可以玩得尽心,只有她不行。她看到白瑞德刚好站在医生的下方,她还来不及调整她脸上的表情,他便看到她了。他嘴角一撇,一边的眉毛扬了起来。她下巴一扬,把头扭开。突然,她听到有人在叫她的名字——那人的口音毫无疑问是查尔斯顿口音,声音盖过了其他人叫别人名字的声音。
\par “韩查理太太——一百五十块——金币。”
\par 人群中突然鸦雀无声,因为这个价钱,也因为这个名字。思嘉惊呆了,顿时僵在那里。她双手捧着下巴,原封不动地坐在那,眼睛因吃惊而睁得大大的。每个人都转头看着她。她看到医生从乐台上俯身对白瑞德低声说着什么。大概在告诉他她还在服丧,让她出现在舞池里是不可能的。她看到白瑞德懒洋洋地耸了耸肩。
\par “另找一个漂亮妞吧,可以吗?”医生问道。
\par “不行,”瑞德清晰地说,目光漫不经心地扫视着人群,“韩太太。”
\par “我告诉你,这是不可能的,”医生恼火地说,“韩太太不会愿意——”
\par 思嘉耳边响起了一个声音,起先,她还没意识到是自己的声音。
\par “不,我愿意!”
\par 她一跃而起,心怦怦跳得厉害极了,连她自己都担心会受不了。她的心之所以怦怦直跳,是因为自己又成了大家关注的中心,成了在场的所有姑娘中有人最想要的人,噢,最好的一点是,她又有可以跳舞的希望了。
\par “噢,我才不在乎呢!我根本不在乎他们会说什么!”她喃喃自语着,一阵甜蜜的狂热劲流遍了她的全身。她甩了甩头,快步走到货摊外边,像敲着响板似的用脚跟点着地,唰地打开黑色的丝绸扇子,大扇特扇起来。刹那间,她看到了媚兰满脸狐疑的面孔、上了年纪的妇人脸上的表情、使性子的姑娘及士兵们表示赞许的热情洋溢的神情。
\par 后来她便来到了舞池,白瑞德正穿过人群中的通道向她走来,脸上还挂着那丝令人讨厌的嘲讽似的微笑。但她不在乎——就算他是亚伯·林肯本人,她也不会在乎的!她又能跳舞了。她要领舞了。她拉开裙摆,向他微微行了一个屈膝礼,给了他一个粲然的微笑。他把一只手放在有褶边的衣服胸口上,鞠了一躬。利瓦伊先是吓了一跳,但马上掩饰了这一情形,高声叫道:“快找好舞伴,跳弗吉尼亚舞吧!”
\par 乐队便奏起了最好的弗吉尼亚舞曲《迪克西》\footnote{1895年美国南北战争时期一首歌颂南方的流行曲。}。
\par “你怎么敢让我这么引人注目,白船长?”
\par “可是,我亲爱的韩太太,你想引人注目的愿望是如此的明显!”
\par “你怎么能在这么多人面前叫我的名字?”
\par “你本可以拒绝的呀。”
\par “但是——我这是为了事业——我——你出这么多金币,我就不能想着自己了。别笑,大家都在看着我们呢。”
\par “不管怎样,他们都会看我们的。别想着向我推销事业这个无聊的话题。你想跳舞,我给了你机会。这是弗吉尼亚舞中最后的舞步,对吗?”
\par “不错——确实如此,我现在得停下来坐一会了。”
\par “为什么?我踩了你的脚了吗?”
\par “没有——可他们会议论我。”
\par “你真的很在乎吗——打心眼里在乎?”
\par “哦——”
\par “你并没犯什么罪,对不对?干吗不和我跳华尔兹?”
\par “可是,要是妈妈——”
\par “还绑在妈妈的围裙带上呢。”
\par “噢,你总用恶劣的话贬低美德,使它们听起来如此愚蠢。”
\par “可美德就是愚蠢的。如果人们议论你,你在乎吗?”
\par “不——可是——哦,我们还是别说这些吧。感谢上帝,华尔兹舞曲开始了。弗吉尼亚舞总是使我跳得喘不过气来。”
\par “别回避我的问题。别的女人说什么对你重要吗?”
\par “噢,如果你硬逼我回答的话——不重要!但人们会认为一个姑娘应该在乎的。不过今晚我不在乎。”
\par “妙极了!你现在开始为自己着想了,而不是让别人来为你着想。这是变聪明的开始。”
\par “噢,可是——”
\par “如果你也像我一样被别人大讲特讲的话,你就会意识到,这根本微不足道。想想看,查尔斯顿没有一家人会欢迎我。即使我对我们正义神圣的事业做出贡献,也没有对我开禁。”
\par “多可怕呀!”
\par “哦,一点也不。直到你失去了名声,你才会意识到,这是怎样的一个负担,或是什么才是真正的自由。”
\par “你真是在恶意毁谤!”
\par “是恶意毁谤,可却千真万确。假设你一直有足够的勇气——或是足够的钱财——那你没有名声也不打紧。”
\par “不是什么都能用钱来买的。”
\par “肯定是有人告诉过你这话。你自己绝对想不出这种陈词滥调的。钱不能买什么呢?”
\par “哦,这个,我不知道——怎么说,幸福和爱是买不来的。”
\par “一般说是可以的。买不来的时候,它也可以买一些最出色的替代品。”
\par “你是不是真有这么多的钱呢,白船长?”
\par “问这问题多没教养呀,韩太太!我太吃惊了。可是,我是有。对一个刚步入青年时期、被切断供给、身无分文的年轻人来说,我已经做得相当不错了。而且我相信,我可以从闯封锁线中赚够一百万。”
\par “噢,不可能!”
\par “哦,当然可能!大多数人似乎还没意识到这一点,从一种文明的废墟中所能赚的钱和从建立一种文明中所能赚的钱是可以画等号的。”
\par “这都是什么意思呀?”
\par “你的家庭,我的家庭以及今晚在这里的所有的人曾经从把荒野变成文明的过程中赚到了钱。那是在兴建帝国。兴建帝国时有很多钱。可是,毁灭帝国时有更多的钱。”
\par “你在讲什么帝国呢?”
\par “我们生活在其中的帝国——南方——南部邦联——棉花王国——它正在我们脚底下土崩瓦解。只是大多数傻瓜没有看到,不会利用这种倒塌而产生的有利形势。我正从这废墟上发财呢。”
\par “这么说,你真的认为我们会被打败?”
\par “是的,干吗要当鸵鸟呢?”
\par “噢,天哪,讲这些太让我厌烦了。你难道不会说些漂亮话吗,白船长?”
\par “如果我说你的眼睛是一对金鱼缸,盛满了最清澈的绿水,而每当鱼游到顶部时,就像现在这样,那你就迷人得像魔鬼一般。那你会高兴吗?”
\par “噢,我不喜欢那样……这音乐不是很美吗?哦,华尔兹我可以没完没了地跳下去!原来我还不知道自己这么想跳华尔兹呢!”
\par “你是和我跳过舞的舞伴中最漂亮的。”
\par “白船长,你不能把我搂得这么紧。大家都在瞧着呢。”
\par “如果没有人在看,你会在乎吗?”
\par “白船长,你真是忘乎所以了。”
\par “我一刻也没有。双手搂着你,我怎么会呢?……那是什么乐曲?不是支新的吗?”
\par “是的。这支挺神圣的,对不对?这是我们从北方佬那学来的。”
\par “这乐曲叫什么名字?”
\par “《这残酷的战争结束以后》。”
\par “歌词是什么?给我唱一下吧。”
\refdocument{
    \par “亲爱的,你记得我们
    \par 上次相见的时候吗?
    \par 你跪在我脚边,
    \par 告诉我你有多爱我,
    \par 噢,你穿着灰军服站在我面前,
    \par 显得有多骄傲。
    \par 你发誓决不
    \par 从我和我们的国家身边迷途他往。
    \par 伤心的哭泣、寂寞的哀鸣,
    \par 无谓的叹息和悲伤的眼泪,
    \par 一切的一切都无济于事!
    \par 这残酷的战争结束以后,
    \par 祈祷吧,让我们再次相会!”
}
\par “当然,原来的歌词是‘蓝军服’,可我们把它改成‘灰军服’了。噢,白船长,你华尔兹跳得好极了。你知道,大多数块头大的人都跳不好。想想看,到我能再跳舞以前,又不知过了多少年、多少年了。”
\par “只会是几分钟而已。我要再出价让你跳下一曲弗吉尼亚舞——还有下一曲,再下一曲。”
\par “噢,不行,我不能跳!你不该这么做!我的名声会被毁掉的。”
\par “它已经被裹在裹尸布里了,那再跳一曲又有什么关系呢?也许我跳了五六曲后会给别的小伙子一个机会,但我得跳最后一曲。”
\par “哦,好吧。我知道我是疯了,但我不在乎。我根本不在乎别人会说些什么。老是坐在家里,我简直腻透了。我要跳舞,跳舞……”
\par “不穿黑色孝服了?我讨厌黑绉纱孝服。”
\par “噢,我不能脱下丧服——白船长,你不该把我搂得这么紧。你再这样的话,我就生气了。”
\par “你生气时看上去美极了。我要再次搂紧你了——你瞧——就想看看你是不是真的会生气。那天在十二棵橡树时,你又生气,又扔东西,你根本不知道你当时有多迷人。”
\par “噢,请你别说了——你就不能把这忘了吗?”
\par “不能,这是我最珍贵的记忆之一——一个得到精心培养的南方美人,带有爱尔兰反——你很有爱尔兰人的个性,你知道。”
\par “噢,天哪,音乐结束了,白蝶姑妈正从后面的房间里走出来呢。我知道,梅里韦瑟太太肯定已经告诉她了。哦,看在上帝分上,我们还是走到窗户那边去看看窗外的景色吧。我不想让她现在就把我逮住。她的眼睛正瞪得像茶碟一样大呢。”

\subsubsection{第十章}



\par 第二天早晨吃蛋奶饼时,白蝶泪流满面,媚兰沉默不语,思嘉则心存对抗。
\par “他们真要说闲话的话,我也不在乎。我敢打赌,我比在那的哪个女孩为医院募到的钱都多——也比我们卖的所有那些乱七八糟的老旧玩意挣来的钱多。”
\par “噢,亲爱的,钱有什么关系呢?”白蝶呜咽着,十指绞在一起。“我只是无法相信我的眼睛,可怜的查理死了还不到一年……而那可恶的白船长把你弄得如此引人注目,他是个很可怕、很可怕的男人,思嘉。怀廷太太的表妹科尔曼太太的丈夫是查尔斯顿人,她告诉了我有关白瑞德的一些事。他是一家相当不错的家庭中的害群之马——噢,白家怎么会生出这种人来?他在查尔斯顿根本不受欢迎,有最放荡的坏名声,还有涉及一个姑娘的事——这件事太糟了,连科尔曼太太都不知道这是——”
\par “哦,我不相信他有这么坏,”梅利柔声说道,“他似乎是个十足的绅士,想想他一直在闯封锁线,那多勇敢——”
\par “他并不勇敢,”思嘉违背情理地说,吃蛋奶饼时倒了有半罐果汁。“他这么做只是为了钱。是他这么告诉我的。他对南部邦联的什么事都不关心,还说我们会被打败。可他的舞跳得好极了。”
\par 听的人都被惊得哑口无言。
\par “我待在家里待腻了,我再也不干了。如果他们就昨晚的事说我闲话,那我的名声已经被毁了,那他们再说什么也就无所谓了。”
\par 她根本没有意识到这个观点是白瑞德的。这想法来得很是时候,和她头脑里想的太符合了。
\par “噢,你妈妈听到这些时,她会怎么说呢?她会怎么看我呢?”
\par 埃伦要是知道她女儿的这种可耻行为,一定会惊恐万状的。思嘉想到这点,一股寒意袭上心头,感到负疚而不安。但一想到亚特兰大和塔拉之间隔了二十五英里,她不禁又振作起来。白蝶小姐不会告诉埃伦的。这会使她这个年长的陪伴者立于不利的境地。而只要白蝶不饶舌,她就会安然无事。
\par “我想——”白蝶说,“是的,我想我最好还是就这件事给亨利写封信——我太痛恨这么做了——可他是我们唯一的男性亲戚,让他去责骂白瑞德——噢,亲爱的,要是查理还在世就好了——你再也、再也不能和那个人说话了,思嘉。”
\par 媚兰一直默默地坐着,她双手放在膝上,让盘子里的蛋奶饼凉一些。她站起身,走到思嘉身后,双臂抱住思嘉的脖子。
\par “亲爱的,”她说,“别丧气。我能理解。你昨晚做的事是件勇敢之举,这一定对医院帮助很大。如果有人敢对你说三道四,我会去对付他们。……白蝶姑妈,别哭了。思嘉哪都不能去,这对她太苛刻了。她还是个孩子。”她手指把玩着思嘉乌黑的头发,“如果我们偶尔出去参加一些晚会,或许我们都会好受一些。也许我们都太自私了,只是悲伤地待在这里。战争时期不比平时。我一想到城里所有这些远离家园、晚上又没有朋友造访的战士们——还有那些已经能够走下病床却还不能重返部队的战士们——哦,我们太自私了。我们应该像其他人一样,此时应该有三个正在康复的病人在我们家,每星期天请几个士兵出来吃饭。好了,思嘉,别发愁了。人们一旦理解了,就不会说闲话的。我们都知道你爱查理。”
\par 思嘉其实根本就没发愁,媚兰轻柔的手拨弄着她的头发,使她感到很不痛快。她很想把头扭开,说:“噢,去你的!”因为昨晚城卫队和民兵的队员及医院出来的战士们争着和她跳舞的那种温馨感至今还记忆犹新。全世界的所有人中,她最不需要的就是梅利这个庇护人。她可以保护自己,谢谢,如果那些老猫们真想嚼舌根——哦,没有这些老猫,她也能活得好好的。世界上有这么多军官,她才没时间去在意那些老太婆会说些什么呢。
\par 在媚兰温柔的话语抚慰下,白蝶在轻轻地拭泪。这时,普里西手里拿着一个大信封走了进来。
\par “梅利小姐,这是给你的。一个黑人小孩送来的。”
\par “给我的?”梅利说着撕开信封,心里感到很纳闷。
\par 思嘉埋头吃着蛋奶饼,起先没注意到什么,后来她听到梅利叫出声来,而且看到她泪水夺眶而出。她抬起头,看到白蝶姑妈的手又要捂住胸口了。
\par “希礼牺牲了!”白蝶尖叫起来,头往后一仰,双臂便软了下来。
\par “噢,我的天哪!”思嘉也叫了起来,体内的血液似乎已经凝固成冰了。
\par “不!不!”媚兰叫道,“快!快把她的嗅盐拿来,思嘉!在那,在那,亲爱的,你好点了吗?深呼吸。不是的,不是希礼。真对不起,我吓着你了。我哭是因为我太高兴了,”她突然张开紧握的手掌,用力吻着手里的东西。“我太高兴了,”她又热泪盈眶了。
\par 思嘉飞快地看了一眼,见是一个大大的金戒指。
\par “你读读,”梅利说,指着地上的信,“噢,他真是太好、太善良了!”
\par 思嘉茫然地捡起那只有一页的信,看到上面黑色的字体刚劲、有力:“南部邦联也许需要热血男儿为之抛头颅、洒热血,但还没有要求妇女们献出自己的生命。亲爱的太太,请接受这个礼物作为我对你的勇敢行为的敬意吧。千万不要认为你的牺牲是徒劳无功的,因为这个戒指是用十倍于它的价值的价格赎回来的。白瑞德船长。”
\par 媚兰把戒指戴在手上,深情地注视着。
\par “我告诉过你他是个正人君子的,对不对?”她转身对着白蝶说,脸上虽然还泪珠点点,笑得却很粲然。“只有感情细腻、考虑周到的绅士才会想到当时我有多伤心——我会把金手链送去的。白蝶姑妈,你得写信给他,邀请他星期天晚上到我们家来吃饭,好让我谢谢他。”
\par 她们俩都很激动,似乎谁也没有想到,白船长没有把思嘉的戒指也赎回来。但她自己想到了,心里颇为不安。她知道,并不是白船长感情细腻才促使他作出如此有风度的举动,而是他在设法得到白蝶家的邀请,而且他也很明白该怎样达到目的。
\par  
\par “听到你最近的行为,我大为不安。”埃伦在信中写道。思嘉坐在桌边读着,眉头紧锁。坏消息当然传得更快。在查尔斯顿和萨凡纳,她经常听说亚特兰大人比南方其他地方的人都更会说闲言碎语,也比其他地方的人更爱管别人的闲事。现在她终于相信了。义卖会是星期一晚上举行的,今天才星期四。哪一只老猫居然自告奋勇写信给埃伦呢?有一刻,她曾怀疑是白蝶,但很快便否定了这种想法。可怜的白蝶穿着她那三号鞋,一直都在瑟瑟发抖呢,她害怕由于思嘉前面的行为而招致对自己的责备,所以决不会把她自己对思嘉疏于教导的事告知埃伦。很可能是梅里韦瑟太太。
\par “你居然这么不顾自己的教养而忘乎所以,这太令我难以相信了。你在服丧期间在公开场合露面,这种不合时宜的举动,我也就不追究了,因为我意识到,你是出于帮助医院的热望才这么做的。可你还跳了舞,而且是跟白船长这样的人!他的事我听得够多了(谁没听说过呢?),波琳姨妈刚刚在上星期还写信给我,说他是个名声不好的人,连他自己在查尔斯顿的家人都不欢迎他,当然他那伤心欲碎的妈妈除外。他是个彻头彻尾的坏蛋,利用你的年轻和无知,让你引人注目,在大庭广众之下辱没你和你的家庭。在这样的事情上,白蝶小姐怎么能如此失职呢?”
\par 思嘉看着桌子对过坐着的姑妈。老太太已经认出了埃伦的笔迹,胖嘟嘟的小嘴因害怕而噘着,就像一个害怕挨批评的小孩一样,希望能用眼泪来逃脱这一责罚。
\par “想到你这么快就忘了自己的出身和教养,我的心都碎了。我想过让你马上回家来,但那要由你父亲来定夺。他星期五会到亚特兰大去,去和白船长谈谈,再护送你回家来。虽然我从中调停,但我担心他对你会很严厉。我希望,但愿促成你过往行为的只是你的年轻和考虑不周。没有人能比我更希望为事业服务的了,我也希望我的女儿们能和我一样,但是辱没——”
\par 还有挺多大同小异的话,但思嘉没有继续把信读完。她第一次着着实实感到害怕了。现在,她再也不会感到可以不顾一切,可以有逆反心理了。她感到自己又年轻又负疚,就像十岁那年把一块沾了黄油的饼干扔到坐在桌边的苏埃伦身上时一样。想到她性情温和的妈妈这么严厉地谴责她,她爸爸也要到城里来和白船长谈话,她这才越来越意识到这件事的严重性。嘉乐也要对她严厉了。这次,她知道自己不能靠坐在他的膝上、表现出一副可爱、冒失的样子来逃避对自己的惩罚了。
\par “不是——不是坏消息吧?”白蝶颤着声问道。
\par “爸爸明天要来,来责罚我,就像鸭子猛啄绿花金龟一样。”思嘉郁郁不乐地说。
\par “普里西,把我的嗅盐找来,”白蝶饭刚吃了一半,她把椅子往后一推,颤着声音说,“我——我好像要晕倒了。”
\par “在你的裙子口袋里呐,”普里西说,她正在思嘉身后晃荡着,陶醉在这轰动一时的闹剧当中。发着脾气的嘉乐先生只要不是冲着她头发拳曲的脑壳发火,总是挺有趣的。白蝶在她的裙子里摸找着嗅盐,然后把这命根子凑到鼻子边。
\par “你们大家都得站在我这一边,不要让我单独和他在一起,一分钟也不行,”思嘉大叫着,“他很喜欢你们俩,只要你们和我在一起,他就不会对我大动肝火。”
\par “我不行,”白蝶站起身来,软弱无力地说,“我——我好像要病了。我要去躺一下。明天我一整天都得卧床。你得替我向他说抱歉。”
\par “胆小鬼!”思嘉心里想着,目光犀利地看着她。
\par 要面对性子火爆的郝先生,梅利虽然也吓得脸色苍白,但她还是振作起来卫护思嘉。“我会——我会帮你解释你是怎么为医院出力的。他一定会理解的。”
\par “不,他不会的,”思嘉说,“噢,如果像妈妈威胁的那样,我得含羞蒙辱地回塔拉去,那我一定会羞死的!”
\par “噢,你不能回家去,”白蝶大哭起来,“如果你走了,我就非得要——是的,要叫亨利来和我们住在一起,你知道的,我根本不能和亨利住在同一个屋檐下。晚上只有梅利在屋里,而城里又有这么多陌生的男人,我会很不安的。你这么勇敢,我就不在乎这里没有男人了!”
\par “噢,他不能把你带回塔拉去!”梅利说,看上去好像也要马上哭出来了,“这里现在是你的家了。没有你,我们怎么办呢?”
\par “如果你知道我对你是怎么看的,我不在你就会很高兴了。”思嘉愠怒地想,希望还有别人而不是媚兰来帮她避开嘉乐的怒火。被一个你如此不喜欢的人卫护,真让人恶心。
\par “也许我们得收回对白船长的邀请——”白蝶开口说道。
\par “哦,我们不能这么做!这样就不礼貌到极点了!”梅利苦恼地叫了起来。
\par “扶我到床上去。我要病倒了,”白蝶呻吟着,“噢,思嘉,你怎么能把这一切带到我的头上?”
\par 第二天下午嘉乐到的时候,白蝶已经病卧在床了。她从紧闭着门的卧室里传了许多表示抱歉的口信出来,让两个惊慌失措的姑娘招待客人吃晚饭。虽然嘉乐吻了思嘉,还赞许地在媚兰的脸上拧了一把,叫她“梅利表妹”,但他沉默不语,预示着不祥。思嘉倒宁愿他大喊大骂,对她加以责备。媚兰很守信用,像个窸窣作响的小影子一样紧跟在思嘉身边。嘉乐好歹还是个绅士,不便当着她的面申斥自己的女儿。思嘉不得不承认,媚兰应付得很好,就好像她根本不知道出了什么岔子似的。晚饭上了以后,她实际上一直成功地让嘉乐不停地说话。
\par “我很想了解县里发生的事,”她粲然地对他微笑着说,“英蒂和哈尼太懒怠写信了,我知道,你知道县里发生的所有事情。把乔·方丹的婚礼给我们说说吧。”
\par 嘉乐被奉承一番,心里顿时感到飘飘然的。他说婚礼是悄悄进行的,“不像你们的婚礼,”因为乔只有几天的休假。萨莉,芒罗家的那个毛头姑娘,看上去很漂亮。不,他记不起她穿什么衣服了,但他确实听说了,她没有第二天穿的衣服。
\par “她没有!”姑娘们叫了起来,吃了一惊。
\par “当然,因为她根本就没有过第二天。”嘉乐解释说,他还没想起兴许不该跟女人讲这些话,就已经高声大笑起来。思嘉的情绪因他的笑声而高涨起来,她不由得感激媚兰的机智。
\par “第二天乔就回弗吉尼亚去了,”嘉乐又很快补充道,“没有对邻里街坊、亲戚朋友的探访,也没有后来的舞会。塔尔顿家的孪生兄弟也回家来了。”
\par “我们听说了。他们康复了吗?”
\par “他们伤得并不重。斯图尔特膝部受了伤,一粒小弹丸则打穿了布伦特的肩膀。他们都因英勇作战在战地快讯上受了表彰,你们也听说了吗?”
\par “没有呢!跟我们说说看!”
\par “他们真是太莽撞了——他们俩都是。我相信他们有爱尔兰血统,”嘉乐自鸣得意地说,“我忘了他们立的是什么功,但布伦特现在是中尉了。”
\par 听到他们的英勇行为,思嘉感到很高兴,就像一个业主那样感到很高兴。一个男人若曾经是她的男朋友,她就总是相信他是属于她的,而他的所有好行为都将为她增光。
\par “我还有你们俩都感兴趣的消息呢,”嘉乐说,“他们说,斯图又到十二颗橡树求婚了。”
\par “哈尼还是英蒂?”梅利激动地问道,而思嘉则几乎是愤怒地盯视着她。
\par “噢,肯定是英蒂小姐。我这个包袱对他挤眉弄眼以前,她不是曾经把他牢牢地吸引住的吗?”
\par “噢。”梅利说着,嘉乐直率的话使她感到有点不好意思。
\par “还有呢,年轻的布伦特又开始在塔拉转悠了。就是现在!”
\par 思嘉连话也说不出来了。她的男朋友背信弃义,这简直就是对她的侮辱。特别是,她回想起她告诉他们说要和查理结婚时,孪生兄弟俩那暴跳如雷的样子。斯图尔特甚至威胁说要用枪打死查理,或是思嘉,或是他自己,或者干脆把他们三个都干掉。那真是最最令人激动的场面。
\par “是苏埃伦?”梅利问道,高兴得突然笑了起来,“但我认为肯尼迪先生——”
\par “噢,他呀?”嘉乐说,“弗兰克·肯尼迪还是没有表态,胆小得不得了。如果他还不开口说明他的意图的话,我很快会去问他的。不是他,是我的小宝贝。”
\par “卡丽恩?”
\par “她还是个孩子呢!”思嘉又能开口说话了,她尖刻地说。
\par “小姐,她只比你结婚时小一岁多罢了,”嘉乐反驳说,“你是不是在忌妒你原来的男朋友追你的妹妹呀?”
\par 梅利脸涨得通红,她不习惯这么坦率的话,便打手势要彼德把甜薯饼送上来。她狂乱地在头脑中搜寻着不会去谈论这些个人私事又能把郝先生此行的目的转移掉的话题。可她什么也想不出来。而嘉乐一旦打开了话闸子,便除了听众之外什么鼓励也不用了。他继续谈到军需部的营私舞弊行为,每个月的供给都在增加,还谈到杰弗逊·戴维斯不正直的傻冒行径,受丰厚酬金诱惑而参加了北方佬军队的爱尔兰人的卑劣举动等等。
\par 酒摆上来时,两位姑娘起身准备离开。嘉乐眉头紧锁,抬眼严厉地看了女儿一眼,要求她单独留下来几分钟。思嘉绝望地看了梅利一眼,梅利无奈地扭弄着手帕,走了出去,轻轻地把活动拉门拉上。
\par “好了,我的小姐,这是怎么回事!”嘉乐给自己倒了一杯葡萄酒,大声叫喊起来,“这举止可是太优雅了!你是不是在试图再找一个丈夫,而你当寡妇才当了多久?”
\par “别这么大声,爸爸,仆人们——”
\par “他们肯定全都知道了,大家都知道我们的面子全丢光了。你可怜的妈妈为此卧床不起,而我也没法抬起头来。真丢人。不行,小姑娘,你这次休想用眼泪来使我心软下来。”因为思嘉的眼睑已经开始眨巴眨巴的,嘴角也噘了起来,他赶紧这么说,声音显得有点慌乱。“我了解你。就在你丈夫的眼皮底下,你也一直在跟别人调情。不要哭。得了,今晚我也不多说了,因为我要去见这个大好人白船长,他居然这么不注重我女儿的声誉。但到了早晨——好了好了,别哭了。这对你也没有半点好处。我已下定决心,明天你得跟我回塔拉去,免得你又让我们丢脸。别哭了,小宝贝。看看我给你带什么来啦!这个礼物不是很漂亮吗?来,看看!你怎么能给我惹这么多麻烦,让我这么一个大忙人专程赶到这来?别哭了!”
\par 
\par 媚兰和白蝶几小时前就已经睡着了,思嘉在温暖的暗夜里却无法入眠。她的心情很沉重,心里感到很害怕。生活才刚刚开始,却要离开亚特兰大,回家去面对母亲!她宁愿去死也不愿去面对她妈妈。此时此刻,她巴不得自己死了才好,那样,每个人都会因自己如此可恶而感到难过的。她翻了个身,在闷热的枕头上辗转反侧,直到她听到从静寂的街道尽头传来一种声响。奇怪的是,虽然这声音有点含糊不清,听起来却很熟悉。她悄悄溜下床,走到窗边。在星空密布、光线暗淡的夜色中,上面覆盖着拱形树枝的街道柔情无限、漆黑一片。声音渐渐近了,有车轮声、马蹄声和马叫声。突然,她咧嘴笑了,因为她听到了爱尔兰土音很重、喝过威士忌后的声音在提高嗓门唱《低靠背车上的假腿人》,她很熟悉这个声音。这也许不是琼斯伯乐的听审日,但嘉乐此时的境况跟那时的是相同的。他正回家来呢。
\par 她看到一辆轻便马车黑乎乎的车身停在屋子前面,还有模糊不清的人影下了车。有人跟他在一起。两个人影在大门边停了一会,她便听到了门插响动的声音,嘉乐的声音清清楚楚地传了过来。
\par “现在我要给你唱《哀悼罗伯特·埃米特》了。你应该知道这支歌,我的小伙子。我来教你。”
\par “我很愿意学,”他的伙伴回答说,平平的慢吞吞的声音里强忍住笑,“但现在不行,郝先生。”
\par “噢,我的天哪,是那个可恨的姓白的家伙!”思嘉心里想着,起先还感到很不安。接着她便放下心来。至少他们没有朝对方开枪。他们在这个时辰这般模样一起回家来,那一定关系很好。
\par “我要唱的,你也要听,要不然我会把你这奥兰治党人枪毙掉。”
\par “不是奥兰治党人——是查尔斯顿人。”
\par “那也好不到哪里去。反而更糟糕。我在查尔斯顿有两个嫂嫂,我知道的。”
\par “他是不是要让全部街坊邻里都知道呀?”思嘉心想,不禁大为惊慌,伸手去拿晨衣。可她能做些什么呢?她总不能在这种三更半夜的时候下楼去把她父亲硬拉进屋来吧。
\par 倚在大门边的嘉乐没有再受到阻挠,头往后一仰,用男低音大声唱起了《哀悼》这支歌。思嘉双肘支在窗台上,极不情愿地笑了。如果她父亲不会唱变调,那一定是支很优美的歌。这也是她最喜欢的歌曲之一。有一会,她禁不住跟着那优美忧伤的歌词开始唱了起来:
\refdocument{
    \par “她离她年轻的英雄长眠的地方很远很远,
    \par 她周围的情人们围着她叹息。”
}
\par 歌声延续着,她听到了白蝶的屋里和梅利的屋里都有了声响。可怜的人哪,她们一定心情很沮丧。歌声停时,两个人影合二为一,走过人行小道,上了台阶。然后传来了一阵谨慎的敲门声。
\par “我想,我得下去看看,”思嘉寻思着,“他毕竟是我父亲,而可怜的白蝶宁愿死也不愿去的。”再说,她也不想让仆人们看到嘉乐现在这副模样。就算彼德试图把他弄上床去,他也会不守规矩的。只有波克知道怎么应付他。
\par 她把晨衣靠颈项边的别针别好,点燃了床边的蜡烛,匆匆走下黑漆漆的楼梯,来到前面的过道里。她先把蜡烛放到烛台上,开了锁,打开门。在闪烁的烛光中,她看到了白瑞德。他衣服的褶边纹丝不乱,正搀扶着个子矮小、体格却很结实的父亲。那支歌显然是嘉乐最后能发出的声音,就像天鹅临死时发出的美妙歌声一样,他正坦然地依靠在同伴的手臂上。他的帽子不见了,鬈曲的头发乱糟糟的,就像一头白色的鬃毛,领带歪到了耳朵边,胸前的衬衫还有点点酒迹。
\par “我说,这是你父亲吧?”白船长说,黝黑的脸上眼神很有趣。他看了一眼穿着睡衣的她,似乎能透过晨衣看到她的身体里面去。
\par “把他搀进来吧。”她简短地说,自己这副打扮使她感到很不好意思,同时也因嘉乐使她处于如此境地,让这个男人笑话她而感到很生气。
\par 瑞德向前推着嘉乐。“要不要我帮你把他弄上楼去?你无法应付他。他挺重的。”
\par 他大胆的建议使她惊得张大了嘴巴。如果白船长上了楼,光想想缩在床上发抖的白蝶和梅利会怎么想就够呛!
\par “我的圣母呀,不行!就在这,把他弄到客厅里的沙发上就得了。”
\par “你是说殉夫吗?”
\par “你脑袋里若能想着说话要有礼貌,我就会对你感激不尽的。就在这,现在让他躺下来。”
\par “要我把他的靴子脱下来吗?”
\par “不用了。他过去也曾穿着靴子睡过。”
\par 她为自己的失言真恨不得把舌头咬掉,因为他把嘉乐的腿放到另一只腿上时,轻声笑了起来。
\par “现在请你走吧。”
\par 他走出去,进了昏暗的过道,捡起掉在门槛边的帽子。
\par “我们星期天晚餐时再见。”他说着走了出去,随手悄无声息地关上门。
\par 思嘉五点半就起身了,后院的仆人们都还没起来准备早点。她悄悄走下楼梯,来到静静的楼下。嘉乐已经醒了,正坐在沙发上,双手紧抓着圆圆的脑壳,好像要把它捏碎在两个手掌之间似的。她走进来的时候,他偷偷瞧了她一眼。抬眼看她也使他痛得难以忍受,他不禁呻吟起来。
\par “唉哟哟!”
\par “你干的好事,爸爸,”她用气愤的低语开始数落他,“在那个时辰回来,还用歌声把街坊邻里都吵醒。”
\par “我唱歌了?”
\par “唱了!你唱了《哀悼》,声音还特大。”
\par “我不记得了!”
\par “邻居至死也会记得的,白蝶小姐和媚兰也会忘不了的。”
\par “我的老天哪,”嘉乐呻吟着,伸出舌苔厚厚的舌头舔着干燥的嘴唇。“牌局开始后我记得的就不多了。”
\par “牌局?”
\par “那个花花公子白瑞德吹牛说他是最棒的扑克玩家——”
\par “你输了多少钱?”
\par “啊,我赢了,这是自然的。喝一两杯就能帮我赢钱。”
\par “你看看你的钱包。”
\par 就好像每个动作都使他很痛苦一样,嘉乐从上衣口袋里取出钱包,打了开来。钱包里空无分文,他茫然无措、可怜巴巴地看着钱包。
\par “五百美元,”他说,“这是用来给郝太太从偷越封锁线的人那买东西的,现在连回塔拉的车费都没有了。”
\par 思嘉怒气冲天地看着空空如也的钱包时,头脑里便有了一个主意,随即迅速明了起来。
\par “我也没法在这城里抬起头来了,”她开口说道,“你把我们大家的脸面都丢尽了。”
\par “住嘴,小姑娘。你没看到我的头都要炸了吗?”
\par “喝得醉醺醺和白船长这样的人一起回家来,还扯嗓门唱歌,好让每个人都听见。不仅如此,还把钱也输光了。”
\par “这个人太精于玩牌了,根本就不是个绅士。他——”
\par “妈妈听说这件事会怎么说?”
\par 他痛苦万分、忧虑如焚地抬头看着她。
\par “你一个字也不会告诉你妈妈让她伤心的,对不对?”
\par 思嘉什么也没说,紧抿着嘴唇。
\par “想想看,这会使她多伤心,而她又是这么温柔。”
\par “你想想,爸爸,就在昨天晚上,你还说我把我们家的脸丢尽了!我,只不过是可怜兮兮地跳了会舞为那些士兵捐款罢了。噢,我真想哭。”
\par “哦,别这样,”嘉乐请求着,“我可怜的脑袋简直受不了了,无疑现在已经在崩裂了。”
\par “可你说我——”
\par “好了,小姑娘,好了好了,小姑娘,别为你可怜的父亲说过的话伤心了,我不是认真的,我不了解情况!没错,我敢肯定,你本意是好的。”
\par “你却要带我回家去丢人。”
\par “啊,亲爱的,我不会那么做的。那是跟你开玩笑。你不要和你妈妈提起钱的事吧?她已经被家里的开销搞得焦头烂额了。”
\par “不会,”思嘉坦率地说,“我不会的,只要你让我待在这儿,告诉妈妈根本没什么,是那些老猫在说三道四罢了。”
\par 嘉乐沮丧地看着自己的女儿。
\par “这和敲诈没什么两样。”
\par “昨天晚上和造谣也没什么两样。”
\par “我说,”他开始哄骗她,“我们会把这一切都给忘掉的。你觉得,像白蝶小姐这样漂亮的好好女士家里会有白兰地吗?再喝一口——”
\par 思嘉转过身,蹑手蹑脚地走过过道,来到餐厅,去取白兰地酒瓶。她和梅利背地里把这叫做“昏厥瓶”,因为白蝶跳动不规则的心脏使她晕倒——或是好像要晕倒时,她总是从这酒瓶里小抿一口。她的脸上现出胜利者的姿态,一点也没有对嘉乐不孝引起的羞愧感。如果再有爱管闲事的人写信给埃伦,谎话就可以抚慰她了。现在她又可以待在亚特兰大了。现在她几乎就可以随心所欲了,白蝶本来就是个无能的人。她开了酒柜门的锁,把酒瓶和杯子紧按在胸前站了好一会。
\par 她眼里浮现出在水花飞溅的桃树溪边举行的一连串野餐及石头山上的烧烤野餐,还有招待会和舞会,下午的舞会、乘轻便马车出去兜风以及星期天晚上的自助晚餐。她到时都能在场,置身于全部活动的正中间,成为男人们的中心。你若在医院为男人们做了哪怕是一丁点事,他们就很容易爱上你。她现在对医院不那么反感了。男人们生病的时候是很容易被挑逗得心旌摇荡的。他们会落入聪明的姑娘手里,就像在塔拉的桃树被轻轻摇动时,熟透的桃子就会掉下来一样。
\par 她拿着能恢复精力的酒回头向父亲走去,心里在感谢上帝,因为著名的郝家头脑也没有能在昨晚的较量中获胜。猛然间,她不禁纳闷,不知白瑞德和这件事有没有关系。












\subsubsection{第十一章}

\par 接下来这个星期,有一天下午,思嘉从医院回到家,一副疲惫不堪的样子,心里感到愤愤不平的。一整个早上,她一直站着,累得筋疲力尽,而梅里韦瑟太太却因为她给一个士兵的手臂缠绷带时坐在士兵的床上而严厉地批评了她,所以她心里很烦。白蝶姑妈和媚兰戴着最漂亮的帽子正跟韦德和普里西一起待在游廊上,已经准备好每周例行的访客活动。思嘉请求她们原谅,说自己不能陪她们了,然后便回到楼上自己的房间里。
\par 马车车轮的最后一丝声响渐渐远去,她知道自己已经很安全,全家人都看不到她了,她便悄悄溜到媚兰的房间门口,转动插在锁孔里的钥匙。这是个整齐、洁净的小房间,下午四点的阳光斜照进来,给了它一副宁静、温暖的神态。地板熠熠生辉,除了几块色彩明快的小地毯外,没铺别的地毯。洁白的墙壁毫无装饰,只有一个角落除外,那里是媚兰用来作临时圣坛的。
\par 在这个角落,上方挂着一面南部邦联的旗帜,旗帜下面挂着一柄金柄马刀。媚兰的父亲曾带着这把刀参加墨西哥战争,查理也曾佩着同一把刀投身战场。查理的饰带和手枪子弹带也挂在那,还有放在手枪皮套里的左轮手枪。马刀和手枪之间,是查理本人的一张达盖尔银版\footnote{1839年发明的、现已废弃不用的照相法。}照片,他穿着灰色的军服,显得非常挺拔、自豪,褐色的大眼睛亮闪闪的,光芒似乎溢出了镜框,嘴唇上则挂着羞涩的微笑。
\par 思嘉对照片看都不看一眼,而是径直走过房间,来到放在窄窄的床边桌子上的一个方形的青龙木信件盒前。她从里面拿出一捆用蓝色丝带绑在一起的信件,都是希礼亲笔写给媚兰的。最上面一封就是那天早晨刚到的,她打开的正是这封信。
\par 思嘉第一次偷读这些信件时,良心受到了强烈的谴责,又很担心被发现,手便哆嗦得厉害,以致连信封都几乎打不开来。可现在,由于屡次重犯,她对名誉问题已经麻木了,而她本来就没有对这问题考虑过多的。不仅如此,连担心被发现的恐惧感也渐渐消失了。偶尔想到这些时,她的心也会往下沉:“如果妈妈知道了,那会怎么样呢?”她知道,埃伦是宁愿看到她死也不愿知道她的这种不光彩的犯罪行为的。这起先也使思嘉很担心,因为她还是想在各个方面都能效仿她的母亲。但想读信的诱惑太大了,她只好把埃伦可能的想法抛至脑后。这些日子以来,她已习惯于把不痛快的事抛到脑后。她已经学会说:“我现在不去想烦人的这个那个事情。我明天再想吧。”一般说,明天到来时,要不就是她根本就没想到,要不就是因为时间的推延,烦恼程度已经得到了缓解,变得不那么沉重了。所以,偷看希礼的信件并没有给她造成太大的心理压力。
\par 媚兰对信件总是很慷慨,会把其中的一些部分读给白蝶姑妈和思嘉听。但是,使思嘉痛苦的正是她没有读出来的那部分,这也是促使她偷偷摸摸地读她小姑子的信件的原因。她必须知道,和媚兰结婚以后,希礼是不是已经爱上他的妻子了。她必须知道,他是不是在假装着爱她。他有没有给她写一些充满柔情的甜言蜜语呢?他表达的是怎样的情感,又用了怎样的温情呢?
\par 她小心地展开信纸。
\par 希礼纤细、均匀的笔迹跃入她的眼帘,她一读到“我亲爱的妻子”,便宽慰地松了口气。他还没有称她为“亲爱的”或是“宝贝”。
\par “我亲爱的妻子:你给我的来信中说到,你很吃惊,担心我会对你隐瞒我真正的想法,你问我这些日子里我头脑里想的是什么——”
\par “我的天哪!”思嘉想着,因负疚而感到一阵恐慌。“‘隐瞒他的真正想法’。梅利是不是看透了他的心思?或是我的心思?她是不是怀疑他和我——”
\par 她的手因害怕而颤抖起来,于是把信纸更靠近一些。但读到下一段时,她又放心了。
\par “亲爱的妻子,如果我对你隐瞒了什么事的话,那也是因为我不想在你的双肩上再增加一重负担,让你为我的身体安全和情绪而担心。但我无法对你隐瞒任何事,因为你太了解我了。别惊慌。我没有受伤,也没有生病。我不但吃得够多,偶尔还能有床铺睡觉。一个士兵也只能要求这些了。但是,媚兰,我心里背负着沉重的思想负担。我这就向你袒露心迹。
\par “就在这些夏日的夜晚,兵营里的人们早已入睡,我却辗转难眠。我抬头望着星空,一遍又一遍地问自己:‘你干嘛到这来,卫希礼?你在为什么而战呢?’
\par “当然不是为了名誉和荣誉。战争是件肮脏的勾当,而我不喜欢肮脏的东西。我不是职业军人,根本就不想去寻求那种泡沫名誉,即便是从大炮的嘴里寻求也不想。然而,我却来参战了——上帝的本意从来没有打算把我创造成别的什么人,只是一个勤学、热心的乡绅。媚兰,因为战斗的号角并没有使我热血沸腾,战鼓也没有促使我奋勇前进。我看得太清楚了,我们都被出卖了,被我们自己目空一切的南方人的自我出卖了。我们相信,我们一个人就能干掉一打北方佬,相信棉花大王可以统治整个世界。我们还被那些高高在上、那些我们尊重和崇敬的人嘴里说出来的话和引人注意的言辞以及偏见和仇视出卖了——什么‘棉花大王、蓄奴制、州权和去他的北方佬’等等。
\par “所以,当我躺在毯子上望着天上的星星,问自己‘你为什么而战’时,我想到了州权、棉花、黑奴和我们从小就被教育要去痛恨的北方佬。可我知道,这当中哪一个都不是我来打仗的原因。我反而好像看到了十二棵橡树,记起了月光是怎样斜照过白色的柱子的,还有在月光下怒放的木兰花那超凡脱俗的样子。攀援而上的玫瑰即使在最炎热的中午也把边上的游廊遮蔽得阴凉无比。我看到了妈妈在那做针线,还同我是个小男孩时一样。我还听到了黑人傍晚从田地里日落归来的声响。他们虽已筋疲力尽,却还唱着歌,准备吃晚饭。水桶被放到清凉的井里打水,轱辘的声响也回萦在耳际。还有通往河边的那条长路的沿路景观,一望无际的棉田,黎明时分从河边洼地腾腾升起的雾气。这就是为什么我人在此处却不爱牺牲、不爱受苦、不爱荣誉,也不痛恨任何人的原因。也许,热爱家园和乡间,这就是所谓的爱国主义吧。但是,媚兰,这个中含义比这深得多。因为,媚兰,我所说的这些东西只是我为之冒着生命危险去战斗的事情的象征,是我喜欢的那种生活的象征。因为我在为逝去的岁月而战,我太喜欢那逝去的岁月了,但是,我担心,不管死亡以何种方式关顾我,那种日子都已经一去不复返了。因为无论赢还是输,我们同样地都已经输了。
\par “如果我们赢了这场战争,拥有了我们梦想的棉花王国,我们也还是输了,因为我们将变成另一个民族,而往昔宁静的日子却已经一去不复返。整个世界将围在我们的门前叫嚷着要买棉花,我们也就可以控制价格。我担心,接下来我们就会变成像北方佬一样,埋头赚钱、追求财富、利润至上,也就是我们现在嘲笑他们的东西。可如果我们输了呢,媚兰,如果我们输了呢!
\par “我并不害怕危险、被捕,或是受伤。甚至连死亡我也不怕,如果死亡真的来临的话。但我害怕,一旦战争结束,我们便再也无法回到旧时光里去。而我是属于旧时光的人,我与现在这种厮杀的疯狂场面格格不入。我担心我无法适应未来的社会,就算我会努力也白搭。你也不会适应的,亲爱的,因为你我是一脉相承的。我不知道未来会给我们带来什么,但它肯定不会像过去那样美好而宜人。
\par “我躺在这,看着睡在身边的小伙子们,我不知道这孪生兄弟俩,或是亚历克斯和凯德会不会跟我有一样的想法。我不知道他们是否知道,他们为之而战的事业在打响第一枪的那一刻就已经输了,因为我们的事业其实就是我们的生活方式,而那已经不复存在了。但我认为,他们不会去想这些事情的,他们是幸运的。
\par “我向你求婚时没有为我们俩想到这一点。我只想到生活会一如既往、宁静安详、轻松适然、一成不变地在十二棵橡树延续下去。我们是一样的,媚兰,我们都喜欢同样宁静的东西。我只看到我们面前有数十年漫长的岁月,我们可以读书、听音乐、憧憬着美好的东西。但绝不是这个!从来就没想到这个!没想到此事会发生在我们大家头上,旧有的方式遭到了毁灭,还有这血腥屠杀和满腔的仇视!媚兰,没有什么值得我们去这么做——不管是州权、黑奴,还是棉花,都不值得我们去这么做。没有什么值得我们去承受正在发生或可能发生在我们头上的事,因为,如果北方佬打败了我们,那未来便会可怕得令人难以置信。而且,亲爱的,他们还是可能打败我们的。
\par “我不该写这些的。我甚至连想都不该去想。但你问我心里想的是什么,我心里就有担心被打败的恐惧。你还记得吗?在烧烤野餐会上,也就是宣布我们订婚的那一天,有个叫白先生的人,听口音是查尔斯顿人,他因为说了一些有关南方人无知的话而几乎引起了一场争斗。你记得吗?因为他说我们没什么铸造厂和工厂、制造厂和船只、兵工厂和金工车间,塔尔顿家的孪生兄弟俩要用枪结果他的性命。你记不记得他曾说过,北方佬的舰队可以把我们团团围困住,使我们的棉花运不出去?他是对的。我们是在用革命战争时期的滑膛枪和北方佬的新式步枪在打仗。封锁线很快就会严密得连医疗用品都进不来。我们应该对像白瑞德这样心里明白、玩世不恭的人加以注意,而不是对那些凭感觉——和空谈看待事情的人予以重视。他说,实际上,南方除了棉花和骄傲自大之外根本没有别的东西可能用来参加战争的。我们的棉花已一钱不值,而他所说的骄傲自大也便成了我们唯一剩下的东西了。但我把那骄傲自大叫做无可匹敌的勇敢。如果——”
\par 但思嘉还没看完就把信纸折了起来塞进信封,她觉得太无聊了,不想再往下读。再说,信里谈到被打败的蠢话,这种口吻使她感到隐隐的不快。她读媚兰的信毕竟不是要知道希礼困惑不解、毫无兴趣的想法的。过去他坐在塔拉的游廊上时,她已经听够了这些论调。
\par 她想知道的只是,他是不是给他的妻子写感情炽热的信件。到目前为止,他还没有。她读过信件盒里的每一封信,它们中没有一封不像是一个哥哥写给妹妹的信。每封信都充满深情、富含幽默感、话题漫无边际,但却不是情书。思嘉曾收到过太多的感情炽热的情书,读到这类信时不会辨认不出感情的真正口吻。可她感觉不到这种口吻。因此就像她每次偷看完信件后一样,一种沾沾自喜的满足感油然而生,围绕着她,因为她很肯定地觉得,希礼还爱着她。她总是轻蔑地想,媚兰为什么总是没有意识到希礼只是把她当成一个朋友在爱着呢。媚兰显然没有发现她丈夫的信里少了某些东西,媚兰从来就没有收到过其他人写给她的情书,没法把它们和希礼的信作一比较。
\par “他写这种污七八糟的信,”思嘉想,“如果我的丈夫给我写这种废话连篇的信,他肯定会受到我的斥责!哟,连查理的信写得都比他的好。”
\par 她捏着信纸的边沿,往回翻到第一页,看了看日期\footnote{英文书信的日期写在第一页。},并且把信的内容记住。信里不像达西·米德写给他父母的信或是达拉斯·麦克卢尔写给他的老处女姐姐费思小姐和霍普小姐的信那样,并没有一段段描写露营和进攻的文字。米德家和麦克卢尔家在全街区到处宣读这些信件。思嘉常常暗地里感到一种耻辱感,因为媚兰没有从希礼那收到这样的信,可以拿到针线组里大声朗读。
\par 希礼在给媚兰写信时似乎试图全然不顾战争,刻意要在他们周围画一个永恒的魔圈,把自从萨姆特堡成了当日要闻以来所发生的事都给圈到外面去。他几乎好像是试图去相信根本就没有战争发生。他写到他和媚兰都读过的书和一起唱过的歌,他们认识的老朋友,以及他环游欧洲时去过的地方。字里行间流露出一种向往着回到十二棵橡树的家的渴望,他整页整页地写到打猎,在有霜冻的秋夜,在星光下骑马走过宁静的森林小路的情景,还有烧烤野餐会、油煎食品野餐会、宁静的月夜及古老的房子那种安详的迷人的美。
\par 她琢磨着刚才读过的信里的话:“绝不是这个!从来就没想到会是这个!”它们似乎是一个倍受折磨的灵魂因要面对他无法面对却又不得不要去面对的事情时发出的呐喊。这使她颇为不解,因为他并不害怕受伤和死亡,那他害怕什么呢?不善分析的她不禁费尽心思去思考起这些复杂的思绪来。
\par “战争打扰了他,而他——他不喜欢会打扰他的事情……比如说我……他爱我,但他害怕跟我结婚,因为——担心我会搅乱他的思维和生活方式。不,这并不是他真正害怕的事。希礼不是胆小鬼。他不可能是的,战地快讯上有提到他的名字,斯隆上校也给梅利写信,告知希礼在带领部队冲锋时的英勇行为。一旦他对某事下定了决心,那就不会有别的人比他更勇敢,或是决心更大,可是——他生活在他自己的幻想世界里,而不是走出来活在这个人世间,他痛恨走到这个人世间来,而且——噢,我也不知道是什么!如果我几年前就能理解有关他的这件事,我知道,他就一定会和我结婚了。”
\par 她把信放在胸前,站在那热切地想着希礼,想了好一会。自从她爱上他的那一天起,她对他的感情就没有变过。十四岁那一年,那一天,她站在塔拉的游廊上,看到希礼满脸微笑地骑着马走过来,头发在早晨的阳光下银光闪闪的,那情景使她连话也说不出来。而她此时对他的感情还跟那时的感情一模一样。她的爱还是一个年轻姑娘对一个她不了解的男人的敬慕之情,这个男人有着她自己所没有的素质,但她却仰慕这些素质。他还是一个年轻姑娘梦想中完美的白马王子,而她的梦想无非就是让他承认爱她,希望能得到一个吻,此外别无所求。
\par 读过信后,她觉得他肯定还是爱她郝思嘉的,虽然他和媚兰结婚了也是一样,而这种确信便是她想要的一切了。她还是那个年轻、未被男人碰过的姑娘。就算查理笨手笨脚的举动和窘迫的亲近行为叩到了她体内深处那根富含激情的弦,她对希礼的梦想也不会以一个吻就结束的。何况和查理单独在一起的不多的几个月夜并没有激起她的感情,或说使她变成成熟的女人。到底什么才是激情,查理没有唤醒她的这个概念,也没有使她明白什么是柔情,或是什么是肉体和灵魂合二为一的真正亲近行为。
\par 对她来说,那种激情就意味着对说不清楚、女性无法分享的男性的疯狂苦役,是一种痛苦和令人尴尬的过程,而这不可避免地又会导致生孩子这一更加痛苦的过程。结婚就是像这样的,这她一点也不会感到惊奇。婚礼举行之前,埃伦就向她提到过,结婚是女人应该带着尊严和毅力忍受的事,而她守寡后,其他年长妇女的低声议论也证实了这一点。思嘉很高兴摆脱了激情和婚姻。
\par 她的婚姻结束了,但爱情并没有完结,因为她对希礼的爱是不一样的,这和激情及结婚没有任何联系,而是某种神圣、美得令人喘不过气来的东西。这种感情在她被迫保持沉默的漫长岁月中悄悄增长,在她经常回味的记忆和渴望当中汲取养分。
\par 她叹了口气,小心地用丝带绑好那捆信件,不下千次地感到纳闷,不知希礼身上的什么东西使她无法理解他。她试图把这件事情想出一个令人满意的结论来,但是,就像往常一样,她那简单的头脑不能帮她做出结论。她把信放回折叠式的写字台里,盖上盖子。接着她却皱起了眉头,因为她的思绪又回到了她读过的信的最后一部分,那里提到了白船长。希礼居然会对一年前那个无赖说过的话印象这么深,这有多奇怪呀!不能否认,白船长是个无赖,虽然他舞跳得很好。只有无赖才会说出像他上次在烧烤野餐会上说的有关南部邦联的那些话来。
\par 她走过房间来到镜子前,自我欣赏地轻轻拍着柔顺的头发。她的情绪又好起来了,每次一看到她白皙的皮肤和上斜的绿眼睛时,她总是如此,然后她便微笑着,好让酒窝现出来。接着她便把白船长忘到脑后去了,因为她记起了希礼有多喜欢她的酒窝。爱上别人的丈夫或偷看这个男人的妻子的信件,她在良心上并不感到痛苦,年纪轻轻、魅力十足的她并未受到搅扰,她又一次确证了希礼对自己的爱,这种心情也没有受到丝毫损坏。
\par 她开了门,心情轻松地走下昏暗而盘旋而下的楼梯。下了一半时,她竟然开始唱起《当这残酷的战争结束的时候》这首歌来。

\subsubsection{第十二章}

\par 战争在继续,大多数时候打的都是胜仗,但人们已经不再说“再打一次胜仗,战争就会结束”了,就像他们已经不说北方佬是懦夫一样。显然,现在大家都明白,北方佬远非胆小鬼,要战胜他们,光打一个胜仗是解决不了问题的。然而,在田纳西州,摩根将军和福里斯特将军带领的南部邦联的军队打了几次胜仗,布尔河第二次战役的胜利悬在人们的脑际,就像是看得见的北方佬的头皮一样,挂在那供人们心满意足地观赏。但为这些头皮付出的代价是惨重的。亚特兰大的医院和各个家庭里,伤病员人满为患,穿黑色丧服的妇女也越来越多了。奥克兰墓地里,阵亡者的一排排单调的坟墓每天都在向前延伸。
\par 南部邦联的货币令人惊恐地大幅度贬值,食品和衣物的价格相应大幅度上涨。军需部征收粮食的比例很重,以致亚特兰大的餐桌上也开始遭罪了。白面粉已经很少见,价格又贵,玉米粉面包已经代替饼干、面包卷和蛋奶饼,成了普通食品。肉店几乎没有牛肉出售,羊肉也很少,而且羊肉价格很贵,只有富人才买得起。但还是有很多猪肉,还有鸡肉和蔬菜。
\par 北方佬对南部邦联港口的封锁越来越严密,像茶叶、咖啡、丝绸、鲸骨紧身胸衣、古龙水、时装杂志和书籍等奢侈品非常稀少,而且价格昂贵,连最便宜的棉制品的价格也往上猛涨,太太小姐们只得遗憾地用旧衣服再对付一个季节。堆了好几年灰尘的织布机也从阁楼里拿了下来,几乎每个客厅里都出现了家纺的织物。每个人都开始穿家织衣服,包括士兵、贫民、妇女、儿童和黑人。南部邦联军服的颜色——灰色几乎已经绝迹,已经被家纺布的灰胡桃暗色所取代。
\par 医院已经为奎宁、甘汞、麻醉剂、氯仿和碘的匮乏而感到担忧。现在,亚麻布和棉制绷带太珍贵了,用过后不能扔掉。在医院护理的每位女士都把一篮篮血迹斑斑的绷带带回家来清洗,熨好之后再送回医院给其他受伤的人使用。
\par 但对刚刚从守寡的蝶蛹里冒出头来的思嘉来说,战争只是意味着快乐和激动的时光。即使衣物和食品极端匮乏也没有使她感到不安。重新融入这个世界,她感到太幸福了。
\par 当她想起过去的一年中日复一日、毫无二致的无聊日子时,生活的步伐似乎就加快到令人不可置信的地步。每一个早晨的到来都是一次激动人心的冒险。在这一天中,她可以见到过去不认识的男人,他们会要求拜访她,告诉她她有多漂亮,为她而战,或许是为她而死是多么特别的一种荣幸。她还是可以,而且确实是爱希礼的,直到她生命的最后一刻也还是如此,但这并不能阻止她诱骗其他男人向她求婚。
\par 战争一直存在着,只不过是在幕后进行着罢了,但这使人们在社会交往时采取了一种不拘礼节的令人愉快的方式。上了年纪的人用惊恐万分的态度看待这种不拘礼节的方式。妈妈们发现时有陌生男人来拜访她们的女儿,他们没有用介绍信就擅自上门了,而他们的祖先是谁,谁也不知道。令妈妈们感到震惊的是,自己的女儿居然和这些男人手拉手。等到举行完婚礼才吻过她丈夫的梅里韦瑟太太,看到梅贝尔亲吻小个子义勇兵勒内·皮卡德时,她简直不相信自己的眼睛。而当梅贝尔并未对此事感到害臊时,她更是惊愕不已。连勒内马上向她求婚这个事实也没有使事情好转起来。梅里韦瑟太太觉得,南方正在朝一个道德全线崩溃的时代迈进,而且还经常这么说。其他的母亲们从心底里有同感,把这一切的罪责全推到战争身上。
\par 要等上一年才能请求允许他们称呼姑娘的名字,当然前面得加上“小姐”两个字,男人们自然是等不及的,因为他们都指望能在一星期或一个月后就为国捐躯。他们也不愿去采用战前良好规矩要求的那种正规、冗长的求婚方式。他们很可能在三四个月后就求婚。姑娘们虽然很清楚,名门闺秀总是对头三次求婚表示拒绝的,现在却在头一次就迫不及待地冲上前去欣然接受了。
\par 这种不拘礼节使战争给了思嘉许多乐趣。除了护理工作中那些脏兮兮的事和烦人的卷绷带的活儿之外,她根本不在乎战争是否会永远打下去。事实上,她已经能够心平气和地容忍医院,因为这是个快乐的狩猎场。那些无助的伤病员轻而易举地就屈从于她的魅力之下。给他们换绷带、为他们洗脸、拍着他们的枕头以示抚慰,还有为他们扇扇子,他们便堕入爱河了。噢,经历了去年那难熬的岁月之后,她现在已经进入了天堂!
\par 思嘉又回到了她和查理结婚前的那个样子,就好像她从来没有和查理结过婚,从未因他的死受过打击,从来没有生过韦德似的。战争、结婚和生孩子全都已经过去了,从来没有拨动过她内心深处的心弦,她还是一点都没变。她是有个孩子,但他在红砖房里被别人照顾得很好,她几乎可以把他遗忘掉。在她的脑海里及内心深处,她又是郝思嘉了,又是县里的美女了。她的思想和活动和她在过去的日子里时没什么两样,只是活动的领域比过去宽得多罢了。白蝶姑妈的朋友们对她很不以为然,但她对此持漫不经心的态度,她的行为举止还跟她没结婚时一样——去参加晚会、去跳舞、和士兵们出去骑马兜风、打情骂俏,还是姑娘时做过的事,她现在都做,只是没有停止穿丧服而已。她知道,这将会是白蝶和媚兰绝对无法忍受的最后一击。她做寡妇时和她还是姑娘时同样有魅力,她因能够我行我素而感到异常愉快,只要顺她的心,她便满心快乐,更为自己的相貌和受欢迎的程度感到很自负。
\par 几个星期前,她还是挺痛苦的,可现在的她很幸福,因和她的男朋友在一块而感到幸福,也为他们一再肯定她的魅力而感到幸福,就好像是能跟已经与媚兰结过婚的希礼冒险地待在一起一样。但是,希礼远离此地时,不知怎的,希礼属于另一个女人这个想法便更容易忍受一些。有了亚特兰大和弗吉尼亚之间的数百英里路途,有时候他似乎也是属于她的,就像他属于媚兰一样。
\par 就这样,一八六二年秋天的几个月飞逝而去,护理、跳舞、兜风和卷绷带占据了所有的时间,只有到塔拉去作短期逗留的时间除外。可这些拜访却使她很失望,因为她没什么机会能和她妈妈一起安安静静地长谈,她在亚特兰大时可是对此颇为向往的。她没有时间在埃伦做针线时坐在她身边,闻着她裙声响处从她的马鞭草香囊里散发出来的淡淡的柠檬香味,感受她柔软的手在她的面颊上轻轻抚摸的感觉。
\par 埃伦现在越发消瘦了,成天心事重重的,从早晨到整个种植园入眠后很久,她还在忙个不歇。南部邦联军需部所要求的东西一月一月地在增加,她的任务就是使塔拉生产出物品来。连嘉乐也变忙了,这在许多年来还是第一次,因为他无法找到可以代替乔纳斯·威尔克森位置的监工,只得亲自骑马管理他的田地。埃伦太忙了,只能在晚上给思嘉一个吻,道声晚安。嘉乐又整天待在田里,为此思嘉感到很没劲。连她的妹妹们也都被自己的事占据了所有的时间。苏埃伦现在已经和弗兰克·肯尼迪达成了“共识”,用一种狡黠的意味唱《残酷的战争结束以后》,思嘉发现,这几乎使人无法忍受。卡丽恩成天把自己裹在对布伦特·塔尔顿的梦想当中,根本不是有趣的同伴。
\par 尽管思嘉每次总是带着愉快的心情回到塔拉的家中去,但白蝶和媚兰照老一套给她来信,请求她回去时,她从来都不会感到遗憾。埃伦这种时候总是唉声叹气的,好像突然才想到她的大女儿和唯一的外孙又要离开她了。
\par “但我不能太自私,要把你留在这。亚特兰大需要你去做护理工作,”她说,“只是——只是,亲爱的,在你走以前,我好像从来都没有时间和你谈谈,让我感觉一下你还是我的小姑娘。”
\par “我永远是你的小姑娘。”思嘉会这么说,把头埋在埃伦的胸前,负疚感陡然从心中升起,谴责着她。她没有告诉她妈妈,把她拉回亚特兰大去的不是为南部邦联服务,而是跳舞和交友。这些日子以来,她有很多事情都瞒着她妈妈。但最重要的是,她一直保守着白瑞德经常造访白蝶姑妈家这个秘密。
\par  
\par 在义卖会过后接下来的几个月中,瑞德一到城里便登门拜访,带思嘉去坐着他的马车兜风,陪她去参加下午的舞会和义卖会,在医院外面等她,好驱车送她回家。她不再担心他会出卖她,把她的秘密说出去,但她心灵深处还是潜伏着一丝使她不安的记忆,他看到过她脾气最坏的时候的样子,并且知道和希礼有关的真实情况。正是因为知道这一点,这才使她在他惹恼她的时候缄口不言。而他又经常惹她生气。
\par 他三十五岁左右,比她过去的所有男朋友年纪都大。要控制他、对付他,像她对付跟她差不多同龄的男朋友那样,她却感到像个孩子一样孤独无助。他看上去总是一副什么都不会使他感到吃惊,可又有很多事情使他感到很有趣的样子。而当他把她弄得哑口无言、怒气冲冲时,她又觉得自己比世界上任何东西都让他感到更有趣。她经常在他娴熟老练的引诱下公然火冒三丈,因为她不但有嘉乐的爱尔兰人的脾气,脸上又有从埃伦那遗传来的极富欺骗性的可爱神态。到目前为止,除了埃伦在场,她从来就不费心去控制自己的脾气。现在,因为害怕他那感到有趣的笑容而要吞回想说的话,真是太痛苦了。要是他也发发脾气,那她就不会觉得自己处于如此不利的境地了。
\par 和他的争论中,她很少获胜。争论过后,她就赌咒发誓,说他不可与之相交,他教养不好,不是正人君子,她将不再和他有任何来往。但他或迟或早的又回到亚特兰大,登门拜访,表面上好像是拜访白蝶姑妈,却又殷勤过头地送给思嘉一盒从拿骚带来的夹心糖。或是在音乐会上预先买得坐在她身边的权利,或者在舞会上声称要和她跳舞。她常常被他那冒失无礼却又无动于衷的神情弄得很开心,不禁哈哈大笑,便又原谅了他以往的不端行为,直到下一次为止。
\par 虽然他有这些使人气恼的特点,渐渐地,她却变得期待着他的来访了。他身上有些令人激动的东西,她对此无法进行分析。这种东西是和她认识的所有男人截然不同的。这种东西寓于他那高大身材的优雅举止中,令人透不过气来。这使得他一走进房间来,就像是房里突然被施加了物理学上的冲力似的。他乌黑的眼里那傲慢无礼却又无动于衷的嘲弄意味,挑起了她要征服他的欲望。
\par “这差不多就像是我爱上他了!”她茫然地想着,“但我没有,我真是弄不明白。”
\par 可那令人激动的感觉还在继续。他来访时,他十足的阳刚之气使白蝶姑妈那像淑女般有教养的房子显得窄小、苍白,且有点古板。思嘉并不是家里唯一一个对他的在场反应古怪、不自在的人,因为他也使白蝶姑妈处于不安和激动的情绪之中。
\par 白蝶知道,埃伦会反对他来拜访她的女儿,也知道查尔斯顿已经取缔了他在文明礼貌的社会里的地位,这是个不能轻易忽视的问题。但她无法抵御他刻意的奉承和亲吻她的手,就像苍蝇无法抵御蜜罐一样。再说,他还常常从拿骚带些小礼物给她,向她保证说他是特意为她买的,而且还冒着生命危险闯过封锁线带了进来——别在纸上的别针和缝衣针、扣子、银线轴和发卡。现在,要得到这些东西几乎是不可能的——太太小姐们戴的都是手削的木制发卡,用布包着橡树子代替扣子——白蝶缺乏道德毅力来拒绝它们。再说,她总是像孩子一样对出其不意地收到的礼包非常喜爱,抵御不住打开礼物的诱惑。而一旦打开后,她就觉得自己没法拒绝了。接受了礼物后,她也就鼓不起足够的勇气,对他说他的名声使他不合适来拜访三个没有男性保护的孤独的女人。白瑞德在屋里时,白蝶姑妈总是觉得她需要个男性保护人。
\par “我也不知道他是怎么回事,”她会无助地叹着气说,“可是——哦,我确确实实认为,如果我能够感觉到——哦,在他的内心深处,他是尊重妇女的话,那他倒是个很好、很吸引人的男人。”
\par 自从她的戒指被送回来后,媚兰发现瑞德是个举止优雅、心细得少有的绅士。察觉到这一点,她感到颇为吃惊。他一直对她很礼貌,但她跟他在一起总有点害羞,这大多是因为,跟不是自小就跟她认识的人在一起,她都会感到害羞。她暗地里为他感到难过,要是他知道她的这种感觉的话,一定会觉得很有趣的。她肯定地认为,他的生活中一定发生过与罗曼史有关的令人悲伤的事,这才使他变得难于对付、爱挖苦人。她觉得他需要的就是个好女人的爱。在她受到保护的一生中,她从没见识过什么是邪恶,几乎就不相信它的存在。有人在说瑞德和查尔斯顿那个姑娘的闲话时,她大为吃惊,根本不相信。所以,她不但没有站在他的对立面,反而使她对他羞羞答答地表示出更多的宽宏大量,因为她认为,这是对他的极大的冤枉,为此还颇为气愤。
\par 思嘉内心跟白蝶姑妈的看法一致。她也认为他对任何女人都不尊重,也许只有媚兰除外。每次他的眼睛在她身上上下逡巡时,她还是会觉得自己就像没穿衣服一样。这并不是说他曾说过什么。那样的话,她就可以用尖刻的言辞挖苦他了。可他黝黑的脸上那对眼睛带着令人不快的侮辱神情看人的方式,就好像所有的女人都是他心境好的时候供他享受的私有物品似的。只有和媚兰在一起时,他的这种神情才会无影无踪。他看媚兰的时候,从来就不会有那种冷漠的评判似的表情,眼里也没有嘲弄的意味。跟她说话时,他的声音里也有一种特别的语气,礼貌、尊重、急于表现自己。
\par “我真弄不明白,你为什么对她比对我要好得多。”一天下午,媚兰和白蝶都回房睡午觉了,只有思嘉和他单独待在一起,思嘉甩着性子说道。
\par 在过去的一小时中,媚兰在卷编织用的纱线,而瑞德一直为她举着,思嘉把这一切全看在眼里。她还注意到媚兰详细、自豪地谈论希礼的晋升时他脸上那种茫然而莫测高深的神情。思嘉知道,瑞德对希礼没有那种崇敬心理,根本不在乎他已经被提为少校这个事实。然而,他还是很礼貌地回答着,对希礼的英勇行为嘟嘟哝哝说着肯定的意见。
\par “如果我像这样老提希礼的名字,”她烦躁不安地想,“他就横眉竖眼的,发出那种恶劣的、知晓一切的微笑。”
\par “我比她漂亮多了,”她继续说道,“我不明白,你为什么对她比对我更好。”
\par “我敢不敢希望你这是在忌妒?”
\par “噢,别乱猜!”
\par “又一个希望破灭了。如果我对卫太太‘更好’,那是因为她值得我这么做。我认识的人中,善良、真诚、无私的人不多,而她是其中的一个。可是你也许没有注意到这些品德。再说,虽然她还很年轻,她是我有幸认识的不多的几个贵夫人中的一个。”
\par “你意思是不是说,你认为我不是个贵夫人?”
\par “我想,我们第一次见面时就已经达成共识,你根本就不是什么贵夫人。”
\par “噢,看你再敢满怀恶意、粗鲁透顶地再提那件事!你怎么能老孩子发脾气般的跟我作对?那是很久以前的事了,自那以后,我已经长大了。如果不是你老在唠叨、暗示这件事的话,我早把它忘得一干二净了。”
\par “我认为这不是什么孩子式的发脾气,我也不相信你已经变了。如果不能依自己的方式行事的话,你还是和那时一样会摔花瓶的。但通常情况下,你还是能够依自己的方式行事的,所以也就没有必要去摔小——摆——设了。”
\par “噢,你是——我真希望我是个男人!我要叫你滚出去,而且——”
\par “而且为了你的痛苦把我杀了。我可以在五十码开外把一个一角银币打穿。你还是使用你自己的武器更好——酒窝啦、花瓶啦这一类的东西。”
\par “你只不过是个无赖。”
\par “你指望我听到这会大发雷霆吗?很抱歉,我只好让你失望了。你用这些名副其实的骂名骂我,我不会生气的。我当然是个无赖,为什么不呢?这是个自由的国家,只要有人愿意,他就可以做个无赖。像你这样的人,我的夫人,只能是伪君子。就像是心里很黑却又千方百计遮掩的人一样。当被别人用名副其实的骂名称呼的时候,就会恼羞成怒。”
\par 在他平静的微笑和慢吞吞说话的态度面前,她感到孤独无助,因为她从来没碰到过像这样完完全全坚不可摧的人。她奚落、冷淡和谩骂的武器变钝了,因为她说什么都无法使他感到羞耻。经验告诉她,说谎者是最会为自己的诚实辩护的,同样,胆小鬼辩护的是他的勇气,没有教养的人为自己辩护的是自己的绅士风度,无赖为自己辩护的是自己的名誉。可瑞德不是这样的人。他承认一切,大笑着促使她继续说下去。
\par 这几个月来,他在此来来往往,来的时候没有事先通报一下,走的时候连声再见也不说。思嘉从来不知道,到底是什么生意促使他到亚特兰大来,因为很少有其他偷闯封锁线的人觉得有必要到离海岸这么远的地方来的。他们都是把货物卸在威尔明顿或查尔斯顿,那里便有来自南方各地的成群的商人和投机商等着他们,集中在那以拍卖的方式竞买走私物品。如果认为他到这来就是为了来看她,那倒会使她高兴,但连她那不非凡的虚荣心都使她不愿相信这一点。若是他曾经向她示过爱,哪怕是一次,或是对蜂拥在她周围的男人表示出忌妒,甚至想握握她的手,或是求她送给他一张照片或是手帕让他珍藏的话,她就会得意洋洋地认为他被她的魅力逮住了。但他还是那么令人讨厌,一点也不像情人的样子。最糟的是,他似乎能看穿她试图让他拜倒在石榴裙下的所有花招和伎俩。
\par 每次他一来,就会在女人群中引起一阵骚动不安。这不但是因为他身上有一种英勇无畏偷闯封锁线的人的浪漫色彩,而且有种不怀好意、必须禁止的素质。他的名声太坏了!每次,亚特兰大的老太太们都要聚在一起说三道四,他的名声便越来越坏,而这只会使他在年轻姑娘的心目中变得更加富有魅力。由于她们中大多数人都很天真无邪,听到的不会比“他对女人相当随便”这种言论更多的东西——至于男人到底是怎么样和女人“随便”的,她们就一无所知了。她们还听到人们低声议论,说没有姑娘跟他在一起是安全的。他虽有这样的坏名声,可自他第一次到亚特兰大来以后,却从未吻过哪个还没结婚的姑娘的手,这就令人颇为奇怪了。可这也只是使他更加神秘,越发令人激动而已。
\par 除了军中的英雄好汉,在亚特兰大,他是人们议论最多的人了。每个人都知道得很详细,他是怎样因为醉酒和“与女人有关的事”而被西点军校逐出校门的。有关查尔斯顿那个姑娘以及她被他杀害的哥哥那令人发指的传闻也是众所周知的事。和查尔斯顿的朋友通信又得到了更多的信息。他父亲,一位为人极好、有着钢铁般意志和骨气的老绅士,在他二十岁那一年起就一个子儿也不供应他了,甚至把他的名字从族谱中划掉。后来,在一八四九年的淘金热中,他闲荡到了加利福尼亚,然后又去了南美和古巴,对他在这些地方的活动,那报告也只会是趣味无穷的。与女人的艳史、几次持枪决斗、把枪支卖给中美洲的革命者。最糟的是,正如亚特兰大人所听说的,职业性赌博也包括其中。
\par 在佐治亚州,令人伤心的是,几乎每个家庭都至少有一个男性家庭成员或是亲戚会赌博,把钱、房子、土地和黑奴都给输掉了。但那是不一样的。一个男人可以因赌博而输掉一切,把自己变成个穷光蛋,可他还是个绅士。但专职赌博的人什么也不是,只能是个流浪汉。
\par 要不是战争造成的这种令人沮丧的局面和他对南部邦联政府的贡献,白瑞德是永远也不会被亚特兰大人接受的。可是现在,即使胸衣束得最紧的人也觉得,爱国主义要求他们要大度一些。比较多愁善感的人则倾向于这种观点:白家的这个害群之马已经为他邪恶的行为方式感到后悔,并且正在努力赎罪。所以,太太们觉得有责任破例作出让步,特别是对这个英勇无畏偷闯封锁线的英雄应该这样。现在大家都知道,南部邦联的命运要依靠偷闯封锁线的小船避开北方佬舰队的技术,就像它同样要依靠前线浴血奋战的士兵们一样。
\par 有传闻说,白船长是南方最好的舵手之一,他毫无畏惧,全不慌张,由于在查尔斯顿长大,他知道那个港口附近的卡罗来纳海岸的每一个水湾、每一条小溪、每一片沙洲和每一块岩石。在威尔明顿水域,他熟悉得就像在家里一样。他从来没有失去过一条船只,也从没被迫扔掉过货物。战争一开始,他就从默默无闻中一跃而起,用足够的钱买了一条小快艇。现在,当偷闯封锁线的每一船货物可以获得二十倍的利润时,他已经拥有四条船。他雇用好舵手,付给他们丰厚的报酬。他们在黑漆漆的晚上溜出查尔斯顿和威尔明顿,把棉花运到拿骚、英国和加拿大去。英国的棉纺厂正停工待产,工人们都快饿死了。每个偷闯封锁线的人只要能在和北方佬舰队的斗智中取胜,就可以在利物浦漫天要价。偏偏瑞德的船只很幸运,既能为南部邦联把棉花运出去,又能把南方急需的战争物资运进来。是的,夫人们都觉得,为了这么一个勇敢的人,她们可以原谅他,并且忘记有关他的好多事情。
\par 他是个英勇无畏的人物,并赢得了极高的回头率。他花起钱来很潇洒,骑着一匹黑色的种马,穿的衣服式样和裁剪总是上乘的。这后一条本身就足以吸引人们的注意,因为士兵们的制服现在已是褛褴破旧、毫无光泽,而普通百姓呢,即使穿着最好的衣服出现,上面也有补得颇为精巧的补丁和用织针编补的地方。思嘉心想,她从来没见过像他穿的那么漂亮的裤子,是浅黄褐色的格子布做的。至于说他的马甲,简直漂亮得难以形容,特别是那件绣着小朵小朵的粉色玫瑰花蕾的白色波纹绸马甲。他带着比这些服饰还更优雅的神态穿着这些衣服,就好像他自己根本不知道穿着它们有多荣耀似的。
\par 如果他要对谁刻意施展魅力的话,很少太太小姐能够加以抵御。最后,连梅里韦瑟太太也屈服了,邀请他星期天到她家去吃晚饭。
\par 在梅贝尔·梅里韦瑟的小个子义勇兵下一次休假时,她就要跟他结婚了。每次一想到这,她就伤心得放声大哭,因为她已打定主意,结婚时要穿着白色的缎子婚纱举行婚礼,可南部邦联没有白色的缎子。她也没法去借一条,因为过去几年中的缎子婚纱都已经拿去做战旗了。爱国的梅里韦瑟太太严厉斥责了她的女儿,并且指出,家纺布对一个南部邦联的新娘来说是最合适的新娘盛装,可这也没用。梅贝尔要缎子。为了事业,即使没有发夹、扣子、漂亮的鞋子、喜糖和茶,她也愿意去举行婚礼,甚至是带着自豪感去举行婚礼,但是,她想要缎子婚纱。
\par 瑞德从媚兰那听到这件事后,从英国带进来成码成码的白色缎子和花边面纱,并且把它们送给她作为结婚礼物。他送得非常巧妙,甚至让人觉得无法提付钱的事。梅贝尔高兴极了,差一点去亲吻他。梅里韦瑟太太知道,这么贵重的礼物——而且是用衣服作为礼物——是很不合适的,但是,瑞德用最华丽的言词告诉她,对一个我们最勇敢的英雄的新娘来说,用再好的服饰来打扮她也不过分。梅里韦瑟太太也想不出什么表示拒绝的方法来了。所以,梅里韦瑟太太邀请他吃晚饭,觉得这个让步比该付给他的礼物的价值还高出许多。
\par 他不但给梅贝尔送来了缎子,而且还就制作婚纱提出了非常好的建议。这个季节,巴黎的裙环比过去的宽,裙子则比过去的短。它们不再做褶边,而是集中在一块做成扇形的花采,底下露出镶边的衬裙。他还说,他在街上没看到人穿宽大的长裤,所以他想大概是已经“过时”了。后来,梅里韦瑟太太告诉埃尔辛太太说,如果她唆使他说下去的话,恐怕他就会具体地告诉她巴黎人在穿什么样的内裤了。
\par 如果不是他天生一副阳刚外形,那他对衣服、帽子和发型的不凡记忆便足以把他贬成个十足女性化的角色。太太小姐们围着他问有关流行式样的问题时,总是感到有点怪怪的,但她们还是一如既往地这么做。她们就像船只失事后的海员一样,已经和时髦的世界隔绝开来,因为很少有时尚书籍通过封锁线被运进来。尽管她们也知道,法国的太太小姐们可能已经把头发剪掉,戴浣熊皮帽子,然而瑞德对裙饰的记忆极好地代替了《戈戴伊式女性时尚》。他能够并且确实注意到了女性心理非常珍视的细节,每次从国外回来后,他都会被一群太太小姐围在中间,告诉她们今年的帽子更小了,但是帽顶很高,把头部的绝大部分都给盖住了;用来点缀的已经不是鲜花,而是羽毛;法兰西皇后的晚装已经不再梳成发髻,而是几乎把头发全部盘在头顶上,露出全部耳朵,还有晚礼服的领子又开得很低啦,低得令人吃惊啦什么的。
\par  
\par 尽管他过去名声不好,尽管隐隐约约有传闻说他不但在偷闯封锁线,而且在做食品投机生意,但是,他还是成了城里众人皆知的最受欢迎、最具浪漫色彩的人物,这种情况延续了好几个月之久。不喜欢他的人说,每次他来过亚特兰大之后,这里的物价就会猛涨五块钱。但是,即使暗地里有流言蜚语不时传出来,但只要他愿意,他照样能保证自己普受欢迎。可事实却相反,在试探过沉着稳重、颇为爱国的公民们的心理,而且赢得了他们的尊重和勉勉强强的喜欢之后,他身上某种邪恶任性的东西似乎又使他特意去冒犯他们,向他们显示他的行为只不过是一种伪装,而且他对此已不再感到有趣了。
\par 他好像对南方的每个人、每件事,特别是南部邦联,都有一种不受个人情绪影响的蔑视心理,而且根本不费心去加以掩饰。正是他对南部邦联发表的言论使得亚特兰大人先是茫然不解地看着他,接着便是冷淡,再下来就是义愤填膺了。即使在一八六二年,男人们就已经故意用冷漠的态度对他行礼,女人们则开始在晚会现场把女儿拉到自己身边。
\par 他不但在冒犯亚特兰大人那真诚而炽热的忠心中获得乐趣,而且还在最不适宜的情况下表现自己,并为此自得其乐。当善意的人们称赞他偷闯封锁线时的英勇行为时,他淡淡地回答说,处于危险境地时,他一直都很害怕,就像前线那些勇敢的士兵们一样害怕。每个人都知道,南部邦联的士兵没有一个是懦夫,显然他的这种说法使人特别恼火。他总是把士兵叫做“我们勇敢的小伙子”或是“我们穿灰色军服的英雄们”,同时却表示出对他们的极大侮辱。一些年轻太太希望他能和她们调情,称他是为她们而战的英雄之一,并向他致谢。他却会向她们行礼,而后宣称说这不是真的,因为只要能有相同的利润,他也会为北方佬妇女做同样的事的。
\par 自从义卖会那个晚上思嘉第一次见到他以来,他就一直以这种方式跟她说话。可是现在,他跟每个人说话都用一种稍加掩饰的嘲弄口吻。当别人表扬他对南部邦联作出贡献时,他总是回答说,偷闯封锁线不过是他手头的一笔生意。如果他能够从政府的合同中赚到同样多的钱,那他就肯定会放弃去冒偷闯封锁线的危险,而去卖假冒伪劣的衣服、掺沙的糖、变质的面粉和腐烂的皮革给南部邦联。他一边说,一边还用眼光扫视那些手里有政府合同的人。
\par 他的大多数话都是令人无法回答的,只会使他们感觉更糟。人们对那些手里有政府合同的人已经颇有微词。前线的人写信回来,一直在抱怨不到一星期就穿破的鞋子、点不燃的火药、一拉就断的马具、腐烂的肉类和长满象鼻虫的面粉。亚特兰大人尽量去相信,把这类东西卖给政府的人是从亚拉巴马、弗吉尼亚和田纳西州来的商人,而不是佐治亚州的。因为,难道佐治亚州的商人不是包括了那些最显赫的家族中的人吗?难道他们不是最先为医院的资金捐款,并且向士兵们的遗孤捐助的吗?难道他们不是最先为“迪克西”欢呼,并且最积极地要向北方佬讨还血债,至少在言辞上是如此的吗?反对从政府合同中牟取暴利的愤怒高潮还没有兴起,瑞德的话只是被当成缺乏教养的证明罢了。
\par 他不但通过暗讽高官的受贿行为和侮辱战士的英勇来冒犯城里人,而且诱骗尊贵体面的公民陷入尴尬境地,自己从中取乐。他总是忍不住去刺痛周围的人那自高自大、虚伪透顶和浮华虚夸的爱国热情,就像一个小男孩忍不住用针去刺气球一样。他巧妙地撕下浮夸自负的人的假面具,揭露那些无知、顽固的人,但他采用的方式极为巧妙,总是用似乎是极为礼貌的关心言辞把他的受害者引出来,使他们自己也不知道发生了什么事,直到他们傻呆呆地站在那,把自己的夸夸其谈、浮华自负和种种可笑之处暴露无遗。
\par 在城里人接受他的那几个月,思嘉对他不存半点幻想。她知道,他刻意的殷勤和华丽的言辞全都是假心假意的。她也知道,他扮演勇敢爱国的偷闯封锁线者的角色只是因为他觉得这很有趣。有时候,她好像觉得,他就像那些和她一起长大的同县的小伙子一样:热衷恶作剧、狂野不拘的塔尔顿家的孪生兄弟;有着邪恶灵感的方丹家的男孩,调皮淘气、爱戏弄人;可以熬通宵盘算耍弄别人的卡尔弗特家的小伙子们。但还是有区别的,因为在瑞德似乎轻松适然的外表下,在稍显温和的残忍之中有某种恶毒,几乎近于邪恶的东西。
\par 虽然她对他的虚伪知道得很清楚,但她还是喜欢他扮演带浪漫色彩的偷闯封锁线的人的角色。至少这会使她和他交往比先前容易得多。所以,当他撕下伪装,公然对亚特兰大人宣战,疏远他们的好意时,她感到极为不安。她不安是因为这似乎很愚蠢,而且某些针对他的严厉的评判还会落在她的头上。
\par 就在埃尔辛太太家为康复病人举办的银币捐助音乐晚会上,瑞德最终遭到了彻底的排斥。那天下午,埃尔辛家挤满了休假的士兵、医院的伤病员、城卫队队员和民兵成员,还有老太太、寡妇和年轻姑娘。每张凳子都坐满了人,连盘旋的长楼梯上都挤满了客人。门边站着埃尔辛的管家,他手里的雕玻璃碗承受不了银币的重负,已经被倒空两次了。这已足以可见音乐晚会的成功,因为现在值一美元的银币已经相当于六十块南部邦联的纸币。
\par 自以为有才华的每个姑娘都唱了歌,或弹了钢琴,以真人为背景的画也赢得了恭维的掌声。思嘉颇为洋洋自得,她不但和媚兰一起表演了一曲感人的二重唱《露珠出现在花瓣上的时候》,接着又唱了一首更为轻快的《噢,女士们,别去在意斯蒂芬!》而且,她还被选为在最后一幅画上作为背景人物,代表南部邦联的精神。
\par 她看上去迷人极了,穿着一件只是稍加装饰的白色粗布晨衣,希腊长袍,系着红蓝相间的腰带,一只手里拿着星星和彩带,另一只手里拿着曾经属于查理和他父亲的金柄马刀,正把它递给跪在面前的亚拉巴马州的凯里·阿什伯恩上尉。
\par 演完之后,她忍不住去搜寻瑞德的视线,看看他是不是欣赏她的美姿。可她却恼怒地看到,他正跟别人争论不休,很可能根本没注意到她。从他周围的人的脸上,思嘉可以看出,他们都被他的话给激怒了。
\par 她向他走去,这时出现了有时在聚会上会出现的那种令人奇怪的冷场。她听到全副武装的民兵队员威利·吉南直率地说:“我能不能这么理解,先生,你意思是说,我们这么多英雄已经为之捐躯的事业不是神圣的事业?”
\par “如果你被火车碾了,你的死并不会使铁路公司变得神圣起来,对不对?”瑞德问道,他的声音让人听起来就好像是在谦虚地征求意见似的。
\par “先生,”威利说,声音都发抖了,“如果我们不在这屋里——”
\par “想到会发生什么,我不禁全身发抖,”瑞德说,“因为,当然喽,你的勇猛是无人不知的。”
\par 威利脸涨得通红,所有的谈话都戛然而止。大家都很尴尬。威利身材健壮、身体健康,已到了参军年龄,可他并没有上前线。当然,他是他妈妈唯一的儿子。而且,毕竟要有人留在民兵队伍里保卫家园。但瑞德提到勇猛一词时,几个正在康复的傲慢的军官中,已经有人在窃笑了。
\par “噢,他干嘛不闭嘴呢!”思嘉气鼓鼓地想,“他这是在毁掉整个晚会!”
\par 米德医生的眉头紧锁,可怕极了。
\par “对你来说,没什么东西是神圣的,年轻人,”他用演讲时常用的声调说,“但对南方的爱国者和女士们来说,有很多东西都是神圣的。把我们的国土从侵略者手里解救出来就是其中之一,州权又是一个,还有——”
\par 瑞德看上去懒洋洋的,声音听起来有讨好的意味,但几乎是无聊乏味的。
\par “所有战争都是神圣的,”他说,“对那些只好去参战的人来说是这样。如果挑起战争的人不把它们弄得神圣起来,谁会那么愚蠢去打仗呢?但是,不管那些雄辩家如何煽动那些打仗的白痴们,也不管他们给战争冠之以如何高贵的目的,战争的原因从来就只会有一个,那就是钱。所有的战争实际上都是为了争钱。可没多少人意识到这一点。他们的耳朵里充斥着齐鸣的号角声,冲天的战鼓声以及待在家里的雄辩家的满口好话。有时候,煽动性的呼吁是‘不让异教徒涉足基督的坟墓!',有时又是‘打倒教皇制度!',有时是‘自由!',而有时又成了‘棉花,蓄奴制和州权!'”
\par “这教皇到底跟我们有什么关系呢?”思嘉想,“基督的坟墓对我们又有什么关系?”
\par 但当她朝怒气冲天的人群走去时,她看到瑞德潇洒地行了个礼,开始穿过人群朝门口走去。她也跟着他朝门口走,但是埃尔辛太太拉住她的裙子,挡住了她。
\par “让他走,”她说,清晰的声音传遍了安静得有些紧张的房间。“让他走。他是个叛国者,是个投机商!他是我们焐在胸口抚育出来的毒蛇!”
\par 瑞德站在过道里,一手托着帽子。他听到了他预料中会听到的话,于是转过身,打量了整个房间一会。他目光锐利地看了埃尔辛扁平的胸脯一眼,突然咧嘴笑了,而后才走了出去。
\par  
\par 梅里韦瑟太太坐着白蝶姑妈的马车回家,不等四个女士坐好,她就爆发了。
\par “这下好了,韩白蝶!我希望你这下该满意了!”
\par “满意什么?”白蝶忧心忡忡地叫道。
\par “满意那个你一直包庇的讨厌的姓白的家伙。”
\par 白蝶坐立不安,这种指责太让她感到难过了,她一时想不起来,其实梅里韦瑟太太也有好几次招待过白瑞德。思嘉和媚兰虽然想到了这一点,但自小就被教育要对年长者有礼貌的她们,对此事也不敢吱声,反而故意低头看着戴着连指手套的手。
\par “他侮辱了我们大家,也侮辱了南部邦联,”梅里韦瑟太太说道,结实的胸脯在华丽夺目的胸衣饰物下急剧地起伏着。“说我们是为钱而战!说我们的领袖骗了我们!他应该被扔进监狱去。是的,应该。我要和米德医生谈谈这件事。如果梅里韦瑟先生还活在人世的话,他一定会去收拾他的!好了,韩白蝶,你听着。你不能再让那个坏蛋进你的家门了!”
\par “噢。”白蝶无助地嘟哝着,一幅不如死了好的样子。她恳求似的看着两个眼睛朝下看的姑娘们,然后又满怀希望地朝彼德大叔笔直的后背看去。她知道他在用心地听着每一个字,希望他会转过身来插话,就像他经常做的那样。她希望他会说:“我说,多利小姐,你别烦白蝶小姐了。”但彼德连动也没动。他从心底里不喜欢白瑞德,可怜的白蝶也知道这一点。她叹了口气,说:“哦,多利,如果你认为——”
\par “我确实认为的,”梅里韦瑟太太坚定地说,“我真无法想像,原先是什么令你对他表示欢迎的。自今天下午以后,城里任何一个体面的家庭都不会再欢迎他了。请千万拿出点勇气来,禁止他再上你家的门。”
\par 她又目光锐利地扫了姑娘们一眼。“我希望你们俩记住我的话,”她接着说,“因为这其中也有你们的过错,你们都对他那么好。只要礼貌而坚决地告诉他,他的出现和不忠诚的言论在你们家显然不受欢迎就行了。”
\par 这时思嘉已经怒火中烧了,就像一匹马一样,只要有只陌生而粗暴的手一触到缰绳就会愤怒地用后腿直立起来。但她害怕开口。梅里韦瑟太太又会写信去给她妈妈的,她可不敢去冒险。
\par “你这头老水牛!”她心里想着,脸上因拼命忍住怒气而涨得通红。“要是能告诉你我对你和你那专横霸道的方式是怎么想的,那该多好啊!”
\par “我从来没想到,在我的有生之年,我还会听到对我们的事业如此不忠的话。”梅里韦瑟太太继续说着,现在的她已经陷入了由正义感激起的无比愤怒的激动当中。“谁要是认为我们的事业是非正义的、是不神圣的,那他就得被绞死!我不想再听说你们两个姑娘再跟他说话的事——我的天哪,梅利,是什么使你这么痛苦呢?”
\par 媚兰脸色苍白,双眼瞪得老大。
\par “我还是会再和他说话的,”她低声说道,“我不会对他无礼。我不会禁止他进我的家门。”
\par 梅里韦瑟太太吐出一口大气,用的力气如此之大,就好像她被人用力猛击过一样。白蝶姑妈两片肥嘟嘟的嘴唇也张开了。彼德大叔转过身来,目瞪口呆。
\par “哦,我为什么就没有勇气说这话呢?”思嘉想,又忌妒又羡慕。“那只小兔子怎么就有勇气跟梅里韦瑟这个老太太作对呢?”
\par 媚兰的手在发抖,但她赶紧接着说下去,好像担心一旦停下来就会失去勇气似的。
\par “我不会因为他说的话而对他无礼相待,因为——他把这话大声说出来,确实太不礼貌了——那是最愚蠢的行为——可这是——这正是希礼所想的。我不能禁止一个和我的丈夫有同样想法的人进我的家门。这太不公平了。”
\par 梅里韦瑟太太缓过气来后,对此加以指责。
\par “韩梅利,我这辈子还没听到过这样的谎话!卫家可从来没有出过胆小鬼——”
\par “我从没说过希礼是胆小鬼,”媚兰说,眼睛又开始发亮了,“我是说他想的和白船长想的一样,只是他用不同的话把它表达出来而已。他不会在音乐晚会上到处乱说,我希望如此。但他把这想法写信告诉我了。”
\par 思嘉的良心被刺痛了一下,她试图回忆起希礼到底写了些什么,会使媚兰说出这些话来。可她一看完信,信的大部分内容就被忘记了。她认为媚兰只是发疯了。
\par “希礼写信跟我说,我们不该去和北方佬打仗的。我们都是受了那些满嘴大话和持有偏见的政客和雄辩家的蒙骗而去打仗的,”梅利说得很快,“他说,这世界上什么也不值得我们去承受这战争会给我们带来的一切。他说荣誉根本就是什么也不是——只有痛苦和污秽。”
\par “噢!那封信,”思嘉想,“那是他所指的意思吗?”
\par “我不相信,”梅里韦瑟太太坚定地说,“你误解他的意思了。”
\par “我从来没有误解过希礼的意思,”媚兰虽然嘴唇在发抖,但还是平静地说,“我非常了解他。他的意思确确实实就是白船长所指的意思,只是他没有无礼地说出来而已。”
\par “你把卫希礼这样出色的人物和白船长这样的恶棍相比较,真该为你自己感到害臊才是!我想,你也认为这事业什么也不是吧!”
\par “我——我也不知道我是怎么想的,”媚兰拿不定主意,开口说道,她勃勃的生气没有了,因直言坦率而引起的恐慌抓住了她的心。“我——我愿意为事业而死,就像希礼那样。但是——我意思是说——我意思是说,我会让这些先生们去思考,因为他们精明多了。”
\par “我从没听说过这样的话,”梅里韦瑟太太从鼻子里哼了一声。“停车,彼德大叔,你已经驶过我家门口了!”
\par 彼德大叔一心在听着身后的谈话,不知不觉地,马车已超过了梅里韦瑟家的马车停车处。他把马车倒了回去。梅里韦瑟太太下了车,帽子上的丝带飘动着,就像暴风雨中的帆船一样。
\par “你会后悔的。”她说。
\par 彼德大叔挥了一下鞭子,马车跑了起来。
\par “你们这些年轻的小姐真该感到羞耻,你们让白蝶小姐过分紧张了。”他责备说。
\par “我没有过分紧张,”白蝶回答说,自己也感到吃惊,因为比这更不会紧张的情形都常常会使她昏厥过去。“梅利,亲爱的,我知道你这么做只是为我说话,真的,我很高兴看到有人杀杀多利的威风。她太飞扬跋扈了。你怎么会有那么大的勇气呢?可是,你真的认为你得对希礼说那样的话吗?”
\par “可这是真的,”媚兰回答说,开始轻声地哭了起来,“他那么想,我一点都不感到羞耻。他认为战争全错了,可他甘愿去打仗、去牺牲,而那比你认为是对的事情而战需要更大的勇气。”
\par “我的天,梅利小姐,别在桃树街上哭鼻子,”彼德大叔嘟哝着,加快了马车的步伐,“大家会说闲话的。等我们到家再哭吧。”
\par 思嘉什么也没说。媚兰把手伸到她的手心里寻求安慰,可她连握一下都没有。她读希礼的信只有一个目的——那就是让自己确信他还爱她。现在,媚兰给了信中写的内容一个全新的意思,这是她思嘉所没有看到的。她颇为吃惊地意识到,像希礼这样绝对完美的人,居然也会和白瑞德这样的恶棍有共同的想法。她寻思着:“他们俩都看到了战争的实质,可希礼愿意为之而死,瑞德却不愿意。我认为,那也说明了瑞德出色的理性。”她的思绪停顿了一下,为自己对希礼有这种想法感到惊恐极了。“他们俩都看到了令人不快的事实真相,可瑞德喜欢从表面上去看待它,用谈论它来激怒人们——而希礼几乎就无法去正视它。”
\par 这太令人茫然不解了。

\subsubsection{第十三章}

\par 在梅里韦瑟太太的唆使下,米德医生采取行动了。他给报社写了封信,信里的意思虽然很明显,但没有提到瑞德的名字。编辑感觉到这封信在社会上的戏剧性效果,把它登在了报纸的第二版。这本身已经是个创举了,因为这家报纸的头两版总是用来登有关黑奴、骡子、耕地、棺材、供出售或出租的房屋、性病的治疗方法、堕胎药及壮阳补品等等的广告的。
\par 整个南方已经开始响起了一片怨言,对投机商、牟取暴利之人及商人感到愤愤不平,而米德医生的信便是这一片怨言声中的第一声。自查尔斯顿港实际上被北方佬的炮舰封锁了以后,威尔明顿便成了主要走私大港,可这里的情况已经成了公开的丑闻。投机商们拥到威尔明顿,用现金买下整船整船的货物,然后囤积起来等候提价。价格总是会提高的,因为随着生活必需品越来越稀少,价格逐月在上升。普通百姓要不在一无所有的情况下克服着过日子,要不只好用投机商定的价格购买,而穷人和那些境况中等的人日子则越来越难过。随着价格上扬,南部邦联的钱币相应贬值,而这形成了一种疯狂购买奢侈品的狂热劲。偷闯封锁线的人受命运进生活必需品来,只允许他们顺便运些奢侈品进来。可现在,装满他们的船只的是价格更高的奢侈品,把南部邦联急需的东西都排斥在外了。人们因为担心明天价格会更高,钱会更没价值,于是用今天手里有的现钱狂热地购买奢侈豪华的物品。
\par 使事情更糟的是,从威尔明顿到里士满只有一条铁路线。当成千上万桶的面粉和整箱整箱的咸肉因等着运输而在路边的车站里腐烂变质的时候,有葡萄酒、塔夫绸和咖啡出售的投机商们却似乎总是能在货物到达威尔明顿的第三天就把它们运到里士满。
\par 原来在暗地里传来传去的流言蜚语现在已经公开,谈得沸沸扬扬的,说白瑞德不但掌管着他自己的四条船,以闻所未闻的价格出售物品,而且把别人船上的货物全部买断,囤积起来等候价格上扬。听说他还是一个资产达上百万美元的集团的头目,总部设在威尔明顿,目的就是在码头上购买走私物品。他们在该城市和里士满有几十个仓库,人们这么传说,仓库里堆满了囤积起来好卖更高价格的食品和衣物。士兵们和普通百姓都已经感到日子过得很紧巴,对他和他的同伙——投机商的行为已是怨声载道。
\par “南部邦联的海军中也有偷闯封锁线的一部,他们中不乏勇敢而爱国的人,”医生的信中最后写道,“他们都是些无私的人,不惜冒着生命危险,牺牲自己的所有财富,南部邦联也许就能因此而幸免于难。他们会被所有忠诚的南方人铭记在心,他们因所冒的危险而获取些微金钱上的回报,那是没有人会有怨言的。他们是无私的人,我们尊敬他们。这些人并不是我要说的人。
\par “但是,还有其他一些人,一些披着偷闯封锁线者的外衣却为自己谋私利的恶棍们。我恳请正在为最正义的事业而战斗的人严阵以待,对这些人类社会的秃鹫、贪得无厌的人予以公正的愤慨和报复。在我们的人因需要奎宁而死去的时候,他们运进来的却是缎子和花边,在我们的英雄因缺少吗啡而痛苦不堪的时候,他们的船却装满了茶和葡萄酒。这些人在吸吮追随罗伯特·李的人们的鲜血,我诅咒这些吸血鬼——这些在爱国将士们的眼皮底下把偷闯封锁线者这个名称变成臭水沟的人们。在我们的小伙子们光着双脚跋涉着去战斗的时候,我们如何能容忍这些食腐动物穿着锃亮的靴子,在我们中间走来走去?在我们的士兵们就着营火被冻得嗦嗦发抖、啃着发霉变质的咸肉的时候,我们又怎能忍受这些人喝着香槟酒、吃着斯特拉斯堡产的馅饼呢?我呼吁每个忠诚的南部邦联的公民行动起来,把他们驱逐出去。”
\par 亚特兰大人读了报纸,知道大智者已经说话了,于是,作为忠诚的南部邦联公民,他们都忙不迭地去把瑞德驱逐出去。
\par 在一八六二年秋天接待过他的所有家庭中,到一八六三年几乎只剩白蝶小姐家是他可以走进家门的了。而且,要不是媚兰,他很可能也不会在那受欢迎。每次他一到城里来,白蝶姑妈就特别紧张。她知道得很清楚,她若允许他登门拜访的话,她的朋友们都会怎么议论,但她还是没有勇气告诉他他在此不受欢迎。每次他到亚特兰大的时候,她就嘟着她那肥嘟嘟的嘴巴跟姑娘们说,她要到门口去见他,不让他进来。而每次他手里拿着一个小包、嘴里说着她既有魅力又漂亮的好话时,她又做不出来了。
\par “我只是不知道该怎么办,”她悲悲凄凄地说,“他就那么看着我,而我——如果我把话跟他明说了,他会做出什么事来呢,我真是怕得要死。他的名声这么坏。你认为他会不会揍我呀——或是——或是——噢,亲爱的,要是查理还活着就好了!思嘉,你必须告诉他,叫他不要再登门拜访了——用一种很礼貌的方式告诉他。噢,我!我确实认为是你在激励他,全城人都在说闲话呢。如果你妈妈知道了,她会对我说些什么呢?梅利,你不该对他那么好。对他冷淡些、疏远些,他就会明白了。噢,梅利,你觉得我是不是最好给亨利写张条子,叫他去和白船长说说?”
\par “不,我不这么认为,”媚兰说,“我也不会对他无礼的。我认为,在有关白船长的事情上,人们的行为就像那些没头没脑的小鸡一样。我相信,他不可能像米德医生和梅里韦瑟太太所说的那样做了那么多坏事。他不会不顾饿肚子的人而把粮食囤积起来的。对了,他甚至给了我一百美元捐给孤儿呢。我敢肯定,他跟我们任何人一样忠诚、爱国,他只是太傲慢了,不为自己辩护而已。你知道男人们发怒的时候有多固执。”
\par 白蝶姑妈对男人的事一无所知,不管他们发怒也罢,没发怒也罢,她都只会无可奈何地摇着她那胖胖的小手。至于思嘉,她早已顺从了媚兰那只看到每个人好的一面的习惯。媚兰是个傻瓜,但谁都对此无能为力。
\par 思嘉知道,瑞德一点也不爱国。虽然她宁死也不愿承认这一点,但她并不在乎。他从拿骚给她带来的小礼物,那些太太小姐们能够得体地接受的小饰物,对她才是最重要的东西。物价如此之高,如果她禁止他进这个家门,她到哪能弄到针啦、糖果啦、发夹啦什么的?不,毕竟白蝶姑妈是这屋里的家长,是长辈,是道德的仲裁人,把责任推到她身上,那可容易多了。思嘉知道,城里人都在谈论瑞德的来访,也在谈论她;但她同样知道,在亚特兰大人眼里,媚兰是不会做错事的,而如果媚兰都为瑞德辩护的话,那他的来访就还带着可敬的一面。
\par 然而,要是瑞德能够放弃他那异端邪说的话,生活就会更美好了。她和他一起在桃树街上走时,也就不用忍受看着他在公开场合遭人白眼的尴尬情形了。
\par “就算你是这么想的,可你干吗说出来呢?”她责备说,“你爱想什么就想什么好了,但不要说出来,那一切都会好得多。”
\par “那是你的方法,对不对,我绿眼睛的伪君子?思嘉,思嘉!我希望你能拿出些更有勇气的行动来。我原以为爱尔兰人怎么想就怎么说的,根本不会顾及后果。跟我说实话,有时候,你难道不是因要闭嘴不言而几乎要爆发了吗?”
\par “哦——是的,”思嘉颇不情愿地承认道,“他们谈起事业的时候,我确实感到很无聊。他们老是从早谈到晚,连中午也不例外。可是我的天,白瑞德,如果我承认的话,那谁也不会跟我说话了,小伙子们一个也不会跟我跳舞了!”
\par “啊,对了,一个人应该有人跟他跳舞,不惜一切代价。哦,我佩服你的自制力,可我发现,自己并不合适拥有它。不管这样会有多方便,我都不能让自己披着浪漫和爱国的外衣。已经有够多愚蠢的爱国者了,他们把每一分钱都拿到封锁线那去冒险。战争结束时,他们就要变成穷光蛋了。他们不需要我去凑数,不管是光耀爱国主义的记录,还是增加穷光蛋的花名册。让他们去戴那光环好了。他们应该受用的——就这一次我是真诚的——再说,再过一年半载,他们所有的一切也就只剩下光环了。”
\par “你知道得很清楚,英国和法国马上就会来支援我们,你说的这些话是很卑鄙的,而且——”
\par “怎么,思嘉!你一定在看报纸吧?你真让我感到吃惊。别再这么做了。这会使女人的头脑糊涂的。至于你的消息,我不到一个月前还在英国呢,我来告诉你是怎么回事吧。英国决不会给南部邦联提供帮助的。英国决不会把赌注压在处于劣势的一方,这就是英国之所以是英国的原因。再说,那个坐在王位上的胖胖的荷兰女人是个虔诚敬神的人,她不赞成蓄奴制。她宁肯让英国棉纺厂的工人们因为得不到我们的棉花而饿死,但决不会、决不会为拥护蓄奴制而战。至于法国,那个效仿拿破仑的意志薄弱者正在为在墨西哥建立法国殖民地而忙得一塌糊涂,根本没有时间来烦我们。事实上,他对战争表示欢迎,因为这样一来,我们就会忙得焦头烂额的,没有时间去把他的军队赶出墨西哥……不,思嘉,外来援助只是报纸杜撰出来以维护南方的信念的。南部邦联注定要失败。它现在就像骆驼一样,已经在以驼峰里的能量为生了,而即使是最大的驼峰也不是取之不尽、用之不绝的。我决定再做六个月偷闯封锁线的生意,然后就要洗手不干了。那以后再做就太冒险了。我要把船只卖给某个认为他能顺利过关的愚蠢的英国佬。但是这样也好,那样也罢,这都不会使我烦恼。我已经赚够了钱,全都以纯金存在英国的银行里。对我来说,那可不是毫无价值的纸币。”
\par 正像他往常的议论一样,这番话听起来似乎很有道理。其他人可能会把他的话称为叛国言论,可是对思嘉来说,他的话里总是带有某些常识和真理。而她也知道这是完全错误的,知道自己本该感到惊恐和气愤。实际上,她既不惊恐,也不气愤,但她可以装出来。这会使她觉得自己更受人尊重、更像个名门闺秀。
\par “我认为米德医生写的有关你的事是对的,白船长。唯一能拯救你的办法就是,把船卖掉,然后去参军。你是个西点军校的学生,而且——”
\par “你说话就像个在作巡回演讲的浸礼会牧师。要是我不想拯救自己呢?我干吗要去为维护一个要把我驱逐出去的体制而战呢?我倒是要看着它毁灭,从中取乐呢。”
\par “我从没听说过什么体制。”她生气地说。
\par “没有?可你还是其中的一分子,就像我过去一样。我敢打赌,你不会比我更喜欢这个体制的。哦,我为什么成了白家的害群之马?不为别的,就为这个——我没有和查尔斯顿人保持一致,我也做不到。而查尔斯顿就是整个南方,只不过成了缩影罢了。不知道你有没有意识到,这有多无聊?有许多事情,就因为人们总是这么做的,你也就必须这么做。又有许多毫无害处的事情,出于同样的原因,你就不能做。还有许多毫无意义的事情也使我颇为烦恼。没有跟你很可能已经听说过的那个年轻小姐结婚,这只是最后一记重击罢了。我为什么要和一个令人乏味的傻瓜结婚呢,就因为出了事,使我没法在天黑之前把她送回家?为什么我的枪法比她那暴怒的哥哥还准却要让他把我打死?如果我是个绅士,当然,我会让他把我杀了,那样就可以为我们白家洗去名誉上的污点了。但是——我喜欢活下去。于是我便活了下来,而且活得很好。……每当我想起我哥哥,想起他生活在查尔斯顿那群神圣的母牛当中,对她们特别尊敬,想起他庸俗乏味的妻子,他举办的圣塞西莉亚舞会,还有他一成不变的水稻田——那时我就明白与这体制决裂所能得到的补偿了。思嘉,我们的生活方式就跟中世纪的封建制度一样古老而过时。令人费解的是,它居然延续至今!它不得不要消亡,现在也正在消亡。而你却指望我会去听像米德医生那样的雄辩家的话,让他告诉我我们的事业是正义而神圣的?一听到战鼓响就抓起滑膛枪,冲到弗吉尼亚去为马尔斯·罗伯特抛头颅洒热血?你认为我是怎样的一个傻瓜呢?甘心受罚可不是我的特长。南方和我现在打成平手了。南方曾经挤兑我,要让我饿死。我没有饿死,反而从南方将死的痛苦中赚够了钱,补偿我已失去的生来就有的权利。”
\par “我认为你真是卑鄙无耻、唯利是图,”思嘉说道,但这话只是下意识地评价。他的大部分言论只在她的头脑里一掠而过,就像任何与她个人无关的谈话内容一样。但部分还是有道理的。在上等人的生活中,有这么多愚蠢透顶的事。得假装着她的心已经死了,进入坟墓了,而实际上却没有,这就是其中之一。她在义卖会上跳舞时,在场的每个人有多吃惊呀。每次,只要她说的和其他年轻妇女说的不一样,或是做的和其他年轻妇女做的不一样,那哪怕是有些微的不同,人们就会气愤得横眉竖眼的。但听到他抨击她最为厌倦的传统,还是引起了她的不快。她在这种人中生活得太久了,他们听到自己的想法被别人说出来时,还是会礼貌地装出一副没有受到打扰的样子来。
\par “唯利是图?不,我只是有远见罢了。虽然说那也许只是唯利是图的代名词。至少,不如我有远见的人会那么说。在一八六一年手里有一千块现金的人都可以做我做过的事,但极少人能够唯利是图到好好利用机会的地步!比如说,萨姆特堡一沦陷,但封锁线还没有设立以前,我用便宜得不能再便宜的价格买了几千桶棉花,把它们运到英国。它们现在还放在利物浦的仓库里。我一直没有卖掉。我要一直留着,等到英国棉纺厂不得不要买棉花的时候,那我开什么价,他们也就只好给我什么价了。如果我能卖一块美元一磅,我也一点也不会吃惊的。”
\par “除非大象能上树,你才能卖一美元一磅呢!”
\par “我相信我能做到的。棉花现在已经卖七十二美分一磅了。战争结束后,我就会成为有钱人了,思嘉,因为我有远见——对不起,是唯利是图。过去我曾经告诉过你,有两个时机是可以赚大钱的,一个是兴建国家的时候,另一个是在国家毁灭的时候。兴建时赚得慢,毁灭时赚得快。记住我的话,也许有一天会对你有用的。”
\par “我确实对你的建议很感激,”思嘉用极其挖苦的口气说道,“可我不需要你的建议。你认为我爸爸是个穷光蛋吗?我不管需要什么花费,他都有。再说,我还有查理的遗产呢。”
\par “我能想象,法国的贵族们在被送上处决死刑犯的囚车的那一刻也是这么想的。”
\par  
\par 瑞德常常指出,思嘉穿着黑色的丧服参加所有的社交活动很不协调。他喜欢明快的色彩,而思嘉的丧服和从帽子上垂挂到脚后跟的黑绉纱面纱使他感到很有趣,也使他感到很生气。但她还是固执地穿着黑衣服、戴着黑面纱。她知道,如果不再多等几年,而是现在就换成艳丽的衣服的话,城里人就更会说三道四了,说得肯定会比现在还厉害。再说,她又怎么向她的母亲交代呢?
\par 瑞德直率地说,黑绉纱面纱使她看上去像只乌鸦,而黑衣服则使她看上去老了十岁。这个有失风度的说法使她飞奔到镜子前,看看十八岁的自己是不是真的看上去像二十八岁的人。
\par “我想,你应该更有自尊,不会试图让自己看上去像梅里韦瑟太太那样,”他奚落她,“最好去尝尝那种痛苦,而不是戴着面纱去给这种痛苦做广告,何况,我敢肯定,你从来就不曾有过这种痛苦。我来和你打个赌。不出两个月,我要让你从头上取下那帽子和面纱,换上巴黎的新款式。”
\par “真的不行,我们别再讨论这个了。”思嘉说着,因他提到查理而颇为生气。瑞德正准备出发去威尔明顿,从那再到国外去,于是满脸带笑地告辞了。
\par 几星期后,一个阳光明媚的夏日的早晨,他又重新出现了,手里拿着一个装饰亮丽的帽盒。看见屋里只有思嘉一人,他把帽盒打开了。帽子被一层层的棉纸包着,那新颖的款式令她不由得大叫起来:“噢,太可爱了!”说着手就伸过去拿。因为极少看见新衣服,更不用说能亲手触摸了,于是,这帽子似乎就成了她所见过的帽子中最漂亮的。它是用绿色的塔夫绸制的,镶着淡绿色的波纹绸。系在下巴下的丝带和她的手一样宽,也是淡绿色的。这个时髦物的边檐拳曲着神气活现的绿色鸵鸟。
\par “把它戴上。”瑞德笑着说。
\par 她飞快地走过房间,来到镜子前,把帽子戴在头上,同时把头发往后扫,露出耳环,把帽带绑在下巴下。
\par “我看上去怎么样?”她大叫道,为了给他看,她转动着身子,摇着头好让羽毛颤动着。然而,还没看到他眼里肯定的神情,她就已经知道自己看上去很漂亮了。她看上去时髦得非常迷人,绿色的饰边使她的眼睛成了深黑色的祖母绿,而且闪闪发亮。
\par “噢,瑞德,这是谁的帽子呀?我要把它买下来。我会把每一分钱都给你的。”
\par “这是你的帽子,”他说,“还有谁可以戴这种绿色的帽子呢?难道你不觉得我会把你双眼的颜色记在脑海里么?”
\par “你真是为我配的吗?”
\par “是的,盒子上还有‘和平街’的法文字样,那对你有没有什么意义呢?”
\par 这对她根本没有意义。她正微笑着看着镜子里自己的形象。这一刻,什么对她都不重要,而唯一对她重要的只是,这是她两年来戴上的第一顶帽子,戴着它她看上去迷人极了。有了这顶帽子,她还有什么不能做呢!可慢慢的,她的笑容就消失了。
\par “你不喜欢吗?”
\par “噢,这真像做梦一样,可是——噢,要用黑绉纱把这可爱的绿色遮起来,把羽毛变成黑色,我真是恨死了。”
\par 他马上走到她身边,用灵巧的手指解下她下巴上的宽帽带。转瞬间,帽子已经放回盒子里了。
\par “你在干什么呀?你说过这是我的。”
\par “可不能把它变成服丧用的帽子。我会找到其他能够欣赏我的品位又有绿色眼睛的迷人小姐的。”
\par “噢,你不能这样!要是不能拥有这顶帽子,我会死的!噢,别这样,瑞德,别这么小气!把它给我吧。”
\par “而且要把它变成像你其他的帽子那样令人害怕的东西?不行。”
\par 她抓着盒子。要把这顶使她看上去又年轻又迷人的可爱的帽子给别的姑娘?噢,绝对不行!有一刻,她似乎看见了白蝶和媚兰惊恐的神情。她还想到埃伦和她会说些什么,不禁打了个寒噤。但虚荣心还是占了上风。
\par “我不会改变它的。我保证。好了,请你把它给我吧。”
\par 他略带讥讽地微笑着把盒子递给她,看着她重新戴上帽子,自顾自欣赏着。
\par “多少钱哪?”她突然问道,脸拉长了,“我只有五十美元,但下个月——”
\par “大约得两千块,南部邦联的钱币。”看着她愁眉苦脸的表情,他笑着说。
\par “噢,天哪——哦,要不现在我给你五十美元,然后,等我——”
\par “我不想要你的钱,”他说,“这是礼物。”
\par 思嘉的嘴都张大了。在男人送礼物这个问题上,那界限是得很准确很小心的。
\par “糖果和花,亲爱的,”埃伦一再说明,“也许一本诗集,或是相册,抑或是一小瓶香水,这些才是一个名门闺秀能从一个绅士手里接受的礼物。绝不能,绝不能接受任何贵重的礼物,即使从你的未婚夫那里也不行。绝对不能接受珠宝或是衣服之类的礼物,连手套或是手帕也不行。如果你接受了这样的礼物,男人们就会知道你不是什么名门闺秀,就会想放肆地占便宜了。”
\par “噢,天哪,”思嘉想着,先看了看镜子里的自己,再看看瑞德脸上令人不解的神情。“我真是无法告诉他我不能接受这个礼物。这太漂亮了。我——我几乎是宁愿他放肆地占点便宜,只要这只是个小便宜。”紧接着,她便为自己有这种想法感到很吃惊,脸一下涨得绯红。
\par “我会——我会给你五十美元——”
\par “你要是给我,我就把它扔到臭水沟里去。或者,更好的办法是,为你的灵魂买台弥撒。我相信,你的灵魂还是忍受得了弥撒的。”
\par 她勉强地笑了,绿色帽沿下自己含笑的身影使她迅速作出了决定。
\par “你到底想对我做些什么?”
\par “我在用上好的礼物引诱你,直到你那孩子气的理想消失殆尽,而你则任由我摆布为止,”他说,“只能从男人那里接受糖果和花,亲爱的。”他模仿着说,而她则不禁笑出声来。
\par “你真是个聪明、黑心肝、卑鄙无耻的人,白瑞德。你知道得很清楚,这顶帽子太漂亮了,我根本无法拒绝。”
\par 他的眼里带着嘲弄意味,同时也在欣赏着她的美丽。
\par “当然,你可以告诉白蝶小姐,说你给了我一顶由塔夫绸和绿色丝绸做的样品,画出了帽子的样子,而我从你这敲诈了五十美元。”
\par “不。我要说一百美元,她就会去告诉城里所有的人,而每个人都会忌妒我,对我的奢侈说三道四。可是,瑞德,你不能再给我带这么贵重的东西了。你真是太好了,可我真的不能接受别的东西了。”
\par “真的吗?哦,只要这会让我高兴,能让我看到这些东西能够使你更加迷人,我就会继续带礼物给你。我要给你带深绿色的波纹绸,做件上衣来配这顶帽子。我还要警告你,我并没那么好。我在用帽子和手镯来引诱你,把你领入一个深渊。你得一直把这记在脑子里:我从来不会毫无理由地做什么事,也从来不会给别人东西而不希望得到什么来作为回报。我总是要获取报酬的。”
\par 他乌黑的眼睛在她的脸上搜寻着,慢慢转向她的嘴唇。思嘉垂下了眼睑,心里一阵激动。现在,他要试图占点便宜了,就像埃伦所预料的那样。他要吻她,或说试图去吻她了,慌乱中她也无法确定会是哪一种情形。如果她拒绝,他就会从她头上扯下这顶帽子,送给别的姑娘。从另一方面来说,如果她第一次让他匆匆忙忙吻一下,那他就可能会给她带其他漂亮的礼物,希望能再次吻她。男人们对吻看得很重,只有天知道这都是为了什么。有很多时候,一个吻便能使他们全身心爱上一个姑娘,而如果这个姑娘很聪明,被吻了一次后便不再让他亲吻的话,就会闹出很多有趣的笑话来。让白瑞德爱上她,并且承认这一点,哀求她让他吻一下或是给他一个微笑,那是多令人激动的事啊。是的,她还是让他去吻她吧。
\par 但他并没有去吻她。她低垂眼睑从旁扫了他一眼,小声嘀咕着怂恿他。
\par “这么说,你总是要获取报酬的,是吗?那你希望我给你什么报酬呢?”
\par “那得等着瞧。”
\par “如果你认为我会用和你结婚来为这顶帽子付账,那我是不会这么做的。”她放胆说道,还摆摆头作出一副漂亮的挑逗模样,使上面的羽毛欢快地动来动去。
\par 他露出了小胡子下面洁白的牙齿。
\par “夫人,你真是自以为是。我并不想和你结婚,也不想和别的人结婚。我不是个适合结婚的人。”
\par “真的呀!”她叫了起来,吃了一惊,现在便能确定他是要占便宜了。“我也没打算吻你呢。”
\par “那你干吗噘着嘴,作出那一副可笑的样子来?”
\par “噢!”她瞟了镜子里的自己一眼,看到自己红红的嘴唇确实作出了待吻的样子,便叫了起来。“噢!”她又叫了一声,不禁怒从中来,脚也跺起来了。“你是我见过的最讨厌的人,就算我从此再也见不到你,我也根本不在乎!”
\par “如果你真的这么认为,你最好把帽子也踩了。哎呀,你现在是什么情绪呀,很可能你也知道,这挺合适。来吧,思嘉,把帽子踩了,让我看看你对我和我的礼物是怎么想的。”
\par “看你敢动这顶帽子。”她说着,抓着帽檐,往后退去。他跟着她,轻声笑了,把她的手握在手里。
\par “噢,思嘉,你这么年轻,真让我心痛,”他说,“我会吻你的,正如你期待的那样,”他随意地倾下身子,胡子擦着了她的面颊。“现在你是不是觉得,你该甩我一记耳光好维护你那礼仪?”
\par 她的嘴唇保持不了原有的姿势了。她抬起头看着他的眼睛,看到他乌黑深邃的眼里似乎趣味十足的,不禁哈哈大笑起来。他真是个爱戏弄人的人,这多令人气恼啊!如果他不想跟她结婚,甚至都不想吻她的话,那他干吗还这么经常来访、还给她带礼物?
\par “这样更好,”他说,“思嘉,对你来说,我是不良影响,若是你稍有理性,你就该让我收拾东西滚蛋——如果你做得到的话。我是很难摆脱的。可我对你来说太坏了。”
\par “是吗?”
\par “你看不出来吗?自从我在义卖会上遇见你,你的举止便变得骇人听闻,而大多数责任都在我。是谁鼓励你去跳舞的?又是谁迫使你承认你认为我们光荣的事业既不光荣也不神圣的?是谁唆使你承认,为那些堂而皇之的主义而献身的男人们都是傻瓜的?让你有那么多事让那些老太太们说三道四,又是谁的怂恿?是谁让你提早好几年、过快地摆脱了服丧的日子?最后,又是谁引诱你接受一件哪个名门闺秀也不会接受的礼物,同时又使你还保持着名门闺秀的身份?”
\par “你对自己自视过高了,白船长。我并没做什么会引起这么多闲话的事,而且,你提到的每件事,我都是在没有你帮忙的情况下完成的。”
\par “我对此表示怀疑,”他说,脸上突然现出宁静而忧郁的神情,“你到现在还会是韩查理伤心欲碎的寡妇,而且因你为受伤将士做的好事而名声在外。然而,最终——”
\par 但她并没有听他说,正在高兴地欣赏着镜子里的自己,心想下午就可以戴着这顶帽子到医院去给那些正在康复的军官送花。
\par 她根本没意识到,他最后那些话是很有道理的。她根本没有意识到,瑞德用尽办法撬开了她守寡这座监狱的大门,还了她自由之身,在她的少女时代本该早已消逝的时候,反而让她在未婚姑娘当中成为王后。她同样没有意识到,在他的影响下,她早已偏离了埃伦对她的教育。这个变化是逐渐的,她觉得藐视一种小小的习俗似乎和藐视另一种习俗毫无关系,而这一切似乎也都跟瑞德没有关系。她没有意识到,在他的怂恿下,她已经不顾她妈妈那许多有关礼仪的最严厉的禁令,忘记了端庄淑女的那些难学的课程。
\par 她只知道,这顶帽子是她有过的帽子中最合适的一顶,而它却没花她半分钱。而且瑞德一定在爱着她,不管他承认不承认。她肯定要找个办法让他承认这一点。
\par 第二天,思嘉站在镜子前,手里拿着一把梳子,嘴里咬着好几个发夹,正在试图梳出一种发型。梅贝尔刚到里士满去看过她的丈夫,说这种发型正在首都风行一时。它就叫做“猫、硕鼠和老鼠”,梳起来非常费劲。头发要从中间分开,在两边各梳成三卷等级不同的发卷,最大的一卷,也是最靠近中分线的一卷,叫做“猫”。“猫”和“硕鼠”都很容易梳,但是“老鼠”老是从她的发夹里滑出来,令她颇为恼怒。然而,她还是下决心要把发型梳好,因为瑞德要来吃晚饭,而他总是会注意到衣服或发型的新式样,并且总会对这些品头论足的。
\par 她费劲地梳着那浓密而又顽固的发卷,额头上已经汗珠点点。这时,她听到楼下过道里传来轻轻的跑步声,知道媚兰从医院回家来了。她听到她飞奔上楼,两级两级地上,不禁停下了手头的动作,发夹正举到半空。她意识到一定是出了什么差错,因为媚兰走路总是像个年长而有钱的贵妇人那样很有教养的。她走到门边,猛地打开门。媚兰跑了进来,脸涨得通红,一副惊恐的样子,看上去就像个自感内疚的孩子。
\par 她脸上挂着泪珠,帽带挂在脖子上,帽子则挂在背后,裙环摆动得很厉害。她手里紧紧抓着什么东西,一股浓重的廉价香水味随她一块飘进房间。
\par “噢,思嘉!”她哭叫着,关好门,一屁股坐在床上,“姑妈回家了没有?还没有?噢,感谢上帝!思嘉,我感到屈辱极了,宁愿去死!我几乎晕了过去,思嘉,彼德大叔威胁说,他要去告诉白蝶姑妈!”
\par “告诉什么?”
\par “说我和那个——和什么小姐——什么太太——说话来着。”媚兰用手帕使劲扇着闷热的脸,“那红头发的女人,叫贝尔·沃特琳的!”
\par “哦,梅利!”思嘉叫了起来,惊得大眼瞪小眼的。
\par 贝尔·沃特琳是她到亚特兰大来的第一天在街上见到的那个红头发女人。至今为止,毫无疑问,她是城里名声最臭的女人。许多妓女成群结队地来到亚特兰大,追着士兵们转,但由于贝尔火红色的头发和她那华丽俗气、过分时髦的衣服,她在妓女中还是鹤立鸡群。她很少出现在桃树街或是别的上等街区,但一旦她出现了,有身份的妇女都会忙不迭地横过马路,躲开她。可媚兰却和她说话。难怪彼德大叔会气愤不已了。
\par “要是白蝶姑妈知道了,我宁愿去死!你知道的,她会哭着告诉全城的人,那我的脸面也就丢光了,”媚兰抽泣着说,“这也不是我的错。我——我没法避开她。那样就太不礼貌了,思嘉,我——我很可怜她。你觉得我那样认为不好吗?”
\par 可思嘉并不关心其中的道德问题。像大多数单纯而有教养的年轻妇女一样,她对妓女有着极强的好奇心。
\par “她想干什么呢?她说了些什么呢?”
\par “噢,她的语法糟极了,但我看得出来,她是在尽力表现得讲究些,可怜的人哪。我从医院出来,可彼德大叔和马车没在门口等我,所以我就想走着回家。经过爱默森家的院子时,她就躲在篱笆后面!噢,谢天谢地,爱默森一家到梅肯去了!她说:‘求你了,卫太太,让我跟你说会话吧。’我也不知道她是怎么知道我的姓名的。我明白我得尽快跑开,可是——哦,思嘉,她看上去很忧伤,而且——哦,简直是在哀求。她穿着黑衣服,戴着黑帽子,也没有上妆。要不是那头红头发,看起来倒真是挺正派的。还没等我回答,她就接着说:‘我知道我不该和你说话,可我曾试着和那个雌孔雀——埃尔辛太太谈谈,她却把我从医院里赶了出来。'”
\par “她真的把她称为雌孔雀吗?”思嘉兴致勃勃地说,大笑起来。
\par “噢,别笑。这并不是什么好玩的事。似乎是——小姐,这个女人也想为医院做点事——这你想像得出来吗?她提出可以每天早晨到医院来护理,当然,埃尔辛太太是死也不会接受这点的,便把她赶出了医院。接着她又说:‘我也想做点事。难道我不是南部邦联的好公民吗,就像你一样?’思嘉,她也想帮忙,这打动了我。你知道,如果她也想为事业出力,她就不可能坏得一无是处。你觉得我这么想不对吗?”
\par “我的天,梅利,真要是不对,谁又在乎呢?她还说了些什么?”
\par “她说,她一直在观察着到医院去的太太们,认为我有——一张——一张善良的脸,所以便把我叫住。她有些钱,想让我拿到医院里去用,而且不要告诉任何人这钱是哪里来的。她说,如果埃尔辛太太知道这是什么钱的话,她肯定不会让这钱派用场的。什么样的钱!就是那时候,我认为我快晕过去了!我心情很沮丧,急于脱身,便说道:‘噢,好的,真的,你真是太好了,’或者是说了些傻话,她于是微笑着说:‘你才是名副其实的基督徒啊,’便把这脏兮兮的手帕硬塞在我手里。哦,你闻得到香水味了吗?”
\par 媚兰伸过一块男人用的手帕来,脏脏的,香水味特别浓,里面包着一些硬币。
\par “她正说着感谢我之类的话,说每星期她都会给我一些钱,就在这时,彼德大叔赶着车过来,看见了我!”梅利泪流满面,把头伏在枕头上,“当他看见我跟谁在一起时,他——思嘉,他对我大吼大叫的!他说:‘你马上给我上车!’当然,我只好从命。回家的路上,他一直在责备我,根本不让我解释,还扬言要告诉白蝶姑妈。思嘉,请你下楼去,请求他不要去告诉她。或许他会听你的。如果姑妈知道,哪怕是我正面瞧了那女人一眼,这也会要她的命的。行吗?”
\par “行,我会去的。我们还是先看看这里面有多少钱吧。感觉挺重的。”
\par 她解开打的结,一把金币滚到床上。
\par “思嘉,有五十块美元呢!而且是金币!”媚兰叫了起来,吓了一跳,手里数着明晃晃的金币。“告诉我,你认为用这个善良的——哦,钱行不行呢——哦,用——哦——把用这种方式赚的钱用在士兵们身上?你不认为也许上帝会理解她也想帮忙的一片苦心,即使钱不干净也不在乎吗?我一想到医院里需要那么多东西——”
\par 但是,思嘉已经没有注意听她说了。她正看着那块脏兮兮的手帕,心里充满了蒙受耻辱之情和满腔的怒火。手帕的一角有几个交织着的字母,是姓名的首字母“R. K. B. ”。她最顶层的抽屉里也有一块跟这一样的手帕,是昨天白瑞德刚借给她用来包他们采的野花花茎的。她已经打算好,今晚他来吃饭时就把它还给他。
\par 这么说,瑞德居然和可耻的沃特琳这个骚货混在一起,而且还给她钱。给医院捐的钱就是从这来的。偷闯封锁线得来的金币。想想瑞德和那个骚货鬼混以后,居然还有脸正视一个正派的女人!想想她居然还认为他在爱着她!这足以证明,他并没有爱上她。
\par 坏女人以及与她们有关的一切,对她来说是既神秘又令人作呕的事情。她知道,男人们光顾这些女人是因为太太淑女们无法启齿的原因——或者说,就算她提到,也只是低声耳语或是间接、委婉地提出来。她一直认为,只有平凡、粗俗的男人才会光顾这种女人。这以前,她从来没想到上等男人——也就是她在上等人的家里碰到的并且和她跳过舞的男人——居然也可能做这种事。这给她的思路开拓了一个全新的领域,而且是个可怕的领域。也许所有的男人都会做这种事!他们强迫自己的妻子做这种不光彩的行径,而实际上又去寻找下等女人,而且还付钱给她们!噢,男人都是这么卑鄙无耻的,而白瑞德是所有男人中最糟糕的一个!
\par 她要拿上这块手帕,当面摔到他脸上去,指着门让他滚蛋,并且永远永远不再跟他说话。可是,不行,她当然不能这么做。她应该永远永远都不让他知道,她居然知道有坏女人存在,更不知道他跟她们有瓜葛。名门闺秀是不会这么做的。
\par “噢,”她怒气冲冲地想,“要不是我是个名门闺秀的话,看我不把什么都告诉那个禽兽!”
\par 她把手帕揉成一团,然后下楼到厨房去找彼德大叔。经过火炉时,她把手帕扔进炉火中,看着它化成火焰,站在一旁徒劳地生着闷气。












\subsubsection{第十四章}

\par 一八六三年夏天到来的时候,希望在每个南方人的心中又膨胀起来。尽管生活必需品匮乏,生活艰苦,尽管有投机商和类似的给大家带来灾难的人,尽管死亡、疾病和痛苦几乎在每个家庭中都留下了印记,可南方人还是在说:“再打一次胜仗,战争就会结束。”说的口气比过去那些夏天还更愉快、更肯定。事实证明,北方佬确实很难对付,但他们最终还是开始瓦解了。
\par 对亚特兰大,对整个南方来说,一八六二年的圣诞节都是个愉快的节日。南部邦联在弗雷德里克斯堡的一次胜战,给了北方军队毁灭性的打击,北方佬的军队死伤数以千计。那个节日期间,到处笼罩着喜庆气氛,因为局势在转变,所以大家充满了喜悦和感激的情绪。穿灰胡桃色军服的部队,如今已是得到锻炼的生力军,他们的将军都已证明了他们的英勇气概。大家都知道,来年春天战事再开始时,北方佬就会被彻底击败的。
\par 春天来了,战事重新开始。五月,南部邦联在钱瑟勒斯维尔又取得了一次重大胜利。整个南方都兴高采烈的。
\par 在离家更近些的地方,一队北部联邦的骑兵冲进佐治亚,被南部邦联的军队打得一败涂地。人们还在大笑着,互相拍着对方的后背说:“对了,先生!老内森·贝德福德·福里斯特一跟上他们,他们就有得受啦!”四月底,斯特雷特上校和一千八百名北方骑兵来了次突袭,进入佐治亚地界,目的是要进攻亚特兰大北部六十英里的罗马。他们雄心勃勃,计划要切断亚特兰大和田纳西之间置关重要的铁路线,然后飞军南下,进入亚特兰大,摧毁南部邦联这个关键城市里的工厂和战争供给。
\par 要不是福里斯特,这次大胆的攻击一定会给南方造成惨重的损失。他虽只有对方三分之一的人马——可那是怎样的人马,怎样的骑兵呀!——他挥军迎战,不等他们到达罗马就截住他们,日夜苦战,最后活捉了全部人马!
\par 这一消息几乎和钱瑟勒斯维尔胜利的消息同时传到亚特兰大,全城好似成了喜悦和欢笑的海洋。钱瑟勒斯维尔的胜利可能更为重要,但俘虏了斯特雷特的骑兵无疑使北方佬显得滑稽可笑。
\par “不,先生,他们最好还是别跟老福里斯特胡来。”亚特兰大人喜笑颜开地说,这消息也一再地被重复来重复去。
\par 现在,局势对南部邦联的命运来说越来越好,人们顺势也被喜气洋洋地推向前去。诚然,自五月中旬以来,格兰特率领的北方军一直在围困维克斯堡。不错,当斯通沃尔·杰克逊在钱瑟勒斯维尔受了重伤时,南方遭受了令人厌恶的损失。不错,当T. R. R.科布将军在弗雷德里克斯堡被杀时,佐治亚失去了她最勇敢、最优秀的儿子之一。但是,北方佬再也承受不了像弗雷德里克斯堡和钱瑟勒斯维尔这样的惨败了。他们只好让步,然后这残酷的战争也就会结束了。
\par 七月到了,随之而来的是这么一则流言,说李正在向宾夕法尼亚进军。这消息后来在战报上得到了证实。李已经到了敌军的领地!李在逼他们战斗!这是这次战争最后的战役了!
\par 亚特兰大沸腾了,激动、兴奋,还有一种报复的迫切心情。现在北方佬该知道,战争在他们自己的国土上打意味着什么。现在他们该明白,肥沃的良田变成荒野、马匹和牛群被盗、房屋被烧毁、老人和小伙子们被拖去蹲监狱以及妇女和儿童被赶出来挨冻受饿是怎么回事了。
\par 每个人都知道,北方佬在密苏里、肯塔基、田纳西和弗吉尼亚等州都做了些什么。就连小孩都能满怀痛恨、一脸恐怖地详述北方佬在占领地的所作所为。亚特兰大已经挤满了从田纳西东部逃难过来的难民,全城人都从他们那里听到他们经受痛苦的第一手资料。在那个地区,南部邦联的支持者只占少数,而且战争的魔爪紧紧抓住了他们,就像在所有的边界各州一样,邻居告邻居的密,骨肉兄弟也自相残杀。这些难民大叫着要看到宾夕法尼亚变成一片固态的火海,连最慈善的老太太也一脸幸灾乐祸的冷酷神情。
\par 可是,消息一点一点地传来,说李发布了命令,不准动宾夕法尼亚州的所有私人财产,掠夺财物以死罪问斩,部队征用的每一物件都由部队付费——这样,这个将军若要保持其受人爱戴的地位,就得用他已经得到的所有威望来作为代价了。不要让官兵在那个繁荣的州中富足的仓库里变得松散懒散吗?李将军到底在想些什么呢?可我们的小伙子们正在挨饿,急需鞋子、衣服和马匹呢。
\par 一张达西·米德寄给医生的匆忙写就的便条在人们手中传来传去,这是七月初亚特兰大人得到的唯一的第一手材料,人们心里的愤怒感越来越强了。
\par “爸爸,你能不能设法给我弄一双靴子来?我已经光着脚两周了,而且我并不指望能再领到一双靴子。如果我的脚没有这么大的话,我也可以像其他人一样,从死去的北方佬脚上脱下一双来穿。可我至今未发现哪个北方佬的脚跟我差不多大的。如果你能给我找到一双,也别用邮寄的方式。路上会被人偷走的,这我也不怪他们。让菲尔坐上火车,带上鞋一块过来。我们会到哪里,我会很快写信给你的。现在我还不知道,只知道我们还要北上。我们现在在马里兰,大家都说我们要去宾夕法尼亚了……
\par “爸爸,我以为我们可以让北方佬也尝尝他们自己种下的苦果,可是将军说不行。就我本人来说,从烧毁北方佬的房屋中可以得到乐趣,就算因此而被枪毙,我也不会在乎的。爸爸,今天我们行军经过了你所见过的最大片的玉米地。我们家没有这样的玉米。哦,我得承认,我们在那片玉米地里暗地里抢了些玉米,因为我们都饿极了。何况,将军不知道的事也不会令他伤心。可那绿油油的玉米并未给我们带来半点好处。所有的小伙子都已得了痢疾,那玉米使得病情更加恶化。拖着一条伤腿走路也比患痢疾容易多了。爸爸,一定要设法给我弄双靴子来。我现在是上尉了,即使没有新军服或肩章,上尉也是应该有靴子的。”
\par 但是,部队已经进入宾夕法尼亚——那才是最重要的事。再打一次胜仗,战争就会结束,到时达西·米德想要多少靴子,就能有多少靴子,小伙子们可以开回家来,每个人又将既幸福又快乐。米德太太想像着她当兵的儿子最终回了家、待在家里时,连眼睛都湿润了。
\par 七月三日,连接北方的电报系统突然一片死寂,直到四日中午才有一些支离破碎、零零星星的消息慢慢传到亚特兰大的总部。在宾夕法尼亚一个叫做葛底斯堡的小镇附近,打了一场硬战,李所有的部队都参加了这场大战役。消息不太确定,来得也很慢,因为是在敌人的地盘上打战。消息首先是从马里兰传过来,再传到里士满,最后才到亚特兰大。
\par 忧虑与不安越来越强烈,恐惧心理占据了全城人的心。没有什么比不明白正在发生的事情更糟的了。有儿子在前线的家庭真挚地祈祷他们的儿子不在宾夕法尼亚,但那些知道自己的亲属是和达西·米德在同一团队的人则咬牙切齿地说,他们能参加这场能够一劳永逸地消灭北方佬的战斗,那是他们无上的光荣。
\par 在白蝶姑妈家,三个女人面面相觑,掩饰不了内心的恐惧。希礼就在达西所在的团队里。
\par 五日,传来了不好的消息,但不是从北部传来的,而是从西部传来的。维克斯堡沦陷了,在受到长期而艰苦的围攻之后沦陷了。实际上,从圣路易斯到新奥尔良的整个密西西比河流域都落到了北方佬的手里。南部邦联被一分为二。在其他任何时候,这个灾难性的消息都会给亚特兰大带来担心和悲伤。可现在,他们没有心情去管维克斯堡。他们在想着在宾夕法尼亚主动进攻的李。如果李在东部打了胜仗,那维克斯堡的损失根本就不算什么灾难。东部有费城、纽约和华盛顿,占领它们就会使北方陷入瘫痪,不但抵消了密西西比河流域的失败,而且得到的还要多。
\par 时间一小时一小时慢慢地过去了,灾难性的阴影笼罩着城市上空,连太阳也黯然失色。人们猛一抬头望向天空时,便会大吃一惊,好像对这本该乌云密布、飘忽而行的天空居然又晴朗又湛蓝感到不可置信似的。到处都有女士们三五成群地汇集在一起,屋前的游廊上、小径上、甚至大街的中央都站满了人群,互相谈论着说,没有消息就是好消息,试图安慰对方,显出一副勇敢的面孔。可是,还是有可怕的传闻,说李被杀害了,仗打输了,大量死伤人员的名单拥了进来,就像穿梭飞行的蝙蝠一样,在静静的大街上传来传去。虽然他们尽力不去相信这些传闻,可被恐慌抓住了心的全城人都冲到城中心、报社和总部,请求他们告知消息。什么消息都行,哪怕是坏消息也好。
\par 车站上集结了一群群人,希望从进站的火车那里听到一些消息,电报局、被人不断骚扰的总部前面、还有报社紧锁的门外都站满了人。这些人群安静得令人奇怪,而且还在悄悄地越聚越多。没有人说话。偶尔才有个老人颤抖着声音请求别人告知他消息,他们只听到一再重复的话:“除了还在战斗,电报上没有从北方来的消息。”这不但没有使人群相互耳语,反而使人群更是一片死寂。走路或坐着马车的妇女身上的流苏越现越多,拥挤的人群散发出的热气和烦躁不安的脚步扬起的灰尘使人感到窒息。女士们都没有说话,但她们苍白、紧绷着的脸上有一种无言的话语在恳求着,这比失声痛哭还更有力。
\par 几乎每个家庭都送了一个儿子、兄弟、父亲、情人或是丈夫去参战。他们全都在等着听到死亡已经降临他们家的消息。他们在等待着死亡的消息。他们并不是在等待被打败的消息,他们摒弃这“失败”的念头。即使现在他们的家人也许正在宾夕法尼亚山区被太阳烤干的草地上慢慢死去;即使现在南方的军队或许正在像冰雹侵袭时的稻子一样倒下去,但他们为之战斗的事业永远也不会倒。他们也许正在成千上万地牺牲,但是,就像相互争斗结成的果子一样,成千上万穿着灰色军服和灰胡桃色军服的新人,嘴里喊着复仇的口号,又会从地上冒出来去代替他们。这些人从哪儿来,谁也不知道。他们只知道,李是能创造奇迹的,弗吉尼亚的军队是战无不胜的。他们确信这一点,就像他们确信天上有个公正而忌妒的上帝一样。
\par  
\par 思嘉、媚兰和白蝶小姐坐在高背马车里,等在《每日观察》报社前面,打着阳伞遮着太阳。思嘉双手直发抖,头顶上的阳伞也晃来晃去的。白蝶很激动,圆脸上的鼻子一动一动,像个小兔子似的。可媚兰却坐在那像石雕一样,随着时间一分一秒地过去,她的眼睛也越睁越大。两个小时中,她只说过一句话,那是在她从她的网格拎包里拿出一小瓶嗅盐递给她姑妈的时候,这也是她一生中唯一一次带着最温柔的情感在跟她说话。
\par “拿着,姑妈,你若觉得要晕过去,那就用得上了。我得先告诉你,如果你晕过去了——你反正一定会晕过去的——再让彼德大叔送你回家,因为我不会离开这个地方,直到我听到——直到我听到消息为止。我也不想让思嘉离开我。”
\par 思嘉根本也不打算离开,不打算到她不能最早听到有关希礼消息的其他任何地方去。不,就算白蝶小姐死了,她也不会离开此地。希礼正在某个地方打仗,也许正在死去,而报社是她能知道确切消息的唯一地方。
\par 她环顾了一下人群,认出一些朋友和邻居。米德太太斜戴着帽子,手挽着十五岁的菲尔的手;麦克卢尔家的小姐们在尽力用颤抖的嘴唇盖住龅牙;埃尔辛太太坐得挺直,像个斯巴达式的妈妈一样,只有从她发髻旁垂挂下来的头发才流露出她内心的不安;范妮·埃尔辛脸色惨白,像个死鬼一样。(范妮当然不是在担忧她的兄弟休,她是不是真的像人们所相信的那样,在前线有个男朋友?)梅里韦瑟太太坐在马车里,轻轻拍着梅贝尔的手。梅贝尔看上去肚子已经很大了,即使她真的是小心地披着披巾,那她在大庭广众之下露面也是很不雅观的。她干吗要这么担心呢?没人听说过在路易斯安那的部队转到了宾夕法尼亚呀。这时候,她那粗鲁的小个子义勇兵在里士满安全着呢。
\par 人群边上有了骚动,白瑞德骑着马小心地穿过人群,朝白蝶姑妈的马车走来,站着的人们纷纷给他让路。思嘉想:“他真有勇气,这时候还到这儿来,因为他没有参军,这群暴民很可能会把他撕成碎片的。”他走近些时,她心想,自己很可能是第一个去撕扯他的人。希礼和其他小伙子们正在和北方佬浴血奋战,光着双脚、在炎热、饥饿中煎熬,腹部因疾病而发炎腐烂。这种时候,他怎么就敢坐在一匹好马上,穿着锃亮的靴子和白色的亚麻布套装,这么时髦阔气,保养得又这么好,还抽着上好的雪茄呢?
\par 他穿过人群慢慢走过来时,人们向他投去了怨恨的目光。老年人胡子盖着的嘴巴发出了嚎叫,天不怕地不怕的梅里韦瑟太太稍稍从马车里欠起身子,清晰地喊了一声“投机商!”那说话的语气把这个词变成了所有的称呼中最肮脏、最恶毒的词语。他根本不管别人,只对梅利和白蝶姑妈举了举帽子致意,骑马来到思嘉边上,倾下身子低声说道:“这个时候,你不认为米德医生应该像一只栖息在我们的旗帜上尖叫着的雄鹰一样,给我们作一场有关胜利的老掉牙的演讲吗?”
\par 因为忧虑不安,她的神经绷得紧紧的。她像只盛怒中的猫一样,飞快地转身面对着他。辛辣的言辞涌到了嘴边,但他摆摆手制止了她。
\par “我是来告诉你们这些女士们,”他大声说道,“我已经去过总部了,第一批伤亡名单已经到了。”
\par 听到这句话,那些近得能够听清他的话的人群中响起了一阵嗡嗡声,人群沸腾了,准备转身顺着白厅大街冲到总部去。
\par “别走,”他大叫道,在马鞍上坐直身子,把手举起来,“名单已经送到两家报社,正在印。就待在这好了!”
\par “噢,白船长,”梅利哭了起来,泪眼汪汪地转向他,“你来告诉我们真是太好了!他们什么时候会公布呀?”
\par “马上就会出来的,夫人。消息送到报社已经有半小时了。负责此事的少校不想在印好以前先泄露出来,担心想得到消息的人会把报社给拆掉。哦!看!”
\par 报社边上的一个窗户开了,一只手伸了出来,拿着一捆细长细长的长条校样,上面墨汁未干,密密麻麻地印着许多名字。人群奋力争夺着,把校样一撕两半,拿到的人试图从人群中退出来阅读,后面的人往前直挤,叫着:“让我过去!”
\par “抓住缰绳。”瑞德简短地说道,飞身跳到地上,把缰绳扔给彼德大叔。他们看到,他往前挤时,厚实的双肩在人群中清晰可见,不断野蛮地推着挤着。一会儿他就回来了,手里拿着六份。他扔了一份给媚兰,再把其他的分发给最近的几辆马车上坐着的几位小姐太太,有麦克卢尔家的小姐、米德太太、梅里韦瑟太太和埃尔辛太太。
\par “快点,梅利。”思嘉叫道,心都跳到了嗓子眼里。她看到梅利的手抖得厉害,根本拿不稳来读时,真是气恼极了。
\par “你拿去读吧。”梅利小声说道。思嘉从她手里一把抓了过来。姓氏W开头的。W开头的在哪里呢?噢,它们全在底下,都被弄脏了。“怀特,”她边读声音边颤抖着,“威尔金斯……温……泽布伦……哦,梅利,他不在名单上!他不在上面!噢,上帝,姑妈!梅利,把嗅盐拿来!把她扶起来,梅利。”
\par 梅利高兴得公然哭出声来,边安抚着白蝶小姐起伏不停的头,边把嗅盐放在她鼻子底下。思嘉在另一边撑着这位胖胖的老太太,她的心因快乐而在歌唱。希礼还活着。他连受伤都没有。上帝放了他一马,这有多好呀!这——
\par 她听到一声低声的呜咽,便转过身,看到范妮·埃尔辛把头埋在她妈妈的怀里,伤亡名单飘到了马车座底下,埃尔辛太太用双臂搂着女儿时,薄薄的嘴唇直发抖,悄悄对马车夫说:“回家,快点。”思嘉飞快地扫了一眼名单。休·埃尔辛不在名单上。范妮一定是有了个男朋友,而他现在已经死了。人群默默地、同情地为埃尔辛家的马车让道,跟在他们后面离开的是麦克卢尔姑娘们的柳条小马车。费思小姐在赶车,她紧绷着脸,像块石头一样,双唇第一次盖住了牙齿。霍普小姐一脸死灰,笔直地坐在她身边,紧紧抓着她姐姐的裙子。她们看上去像老太太一样。她们年轻的弟弟达拉斯是她们的至爱,也是这一对老处女在这世上唯一的亲人。达拉斯也走了。
\par “梅利!梅利!”梅贝尔在叫,声音里满是喜悦,“勒内没事!希礼也是!噢,感谢上帝!”披巾从她肩上滑落下来,她大腹便便的模样再明显不过了,可她和梅里韦瑟太太都破天荒第一次对此毫不在乎。“噢,米德太太!勒内——”她的声音马上变了。“梅利,快看!——米德太太,快告诉我!达西没有——?”
\par 米德太太低头看着大腿,听到有人叫她的名字也没有抬起头来。可坐在她身边的小菲尔的脸就像一本打开的书一样,大家都看得再明白不过了。
\par “哎,哎,妈妈。”他无能为力地说。米德太太抬起头来,跟媚兰的眼睛对视着。
\par “他现在不会需要那些靴子了,”她说。
\par “噢,亲爱的!”梅利叫了起来,又哭开了。她把白蝶小姐推开,让她靠到思嘉肩上,爬下马车,朝医生的夫人走去。
\par “妈妈,你还有我呢。”菲尔说道,无望地试图安慰他身边这个脸色惨白的妇人,“如果你能让我去,我就去杀掉所有的北方佬——”
\par 米德太太紧紧抓住他的手臂,好像永远不会放手似的,说道:“不!”闷声闷气的,好像被哽住了。
\par “菲尔·米德,你住嘴吧!”媚兰嘘声说道,爬上马车坐在米德太太身边,双臂抱住她。“你以为你也去被枪杀对你妈妈会有什么帮助吗?我从来没听过这么愚蠢的话。送我们回家,快点!”
\par 菲尔抓起缰绳。媚兰转身对思嘉说道。
\par “你一把姑妈送回家就到米德太太的家里来。白船长,你能不能捎个话给医生?他在医院里。”
\par 马车穿过四散的人群离开了。有些女人高兴得直哭,但大多数看上去都茫然失措的,似乎意识不过来落在她们身上的沉重打击。思嘉低头看着模糊不清的名单,快速浏览着,想看看有没有朋友们的名字。既然希礼安然无恙,她也可以想想别人了。噢,这名单有多长啊!亚特兰大的损失、整个佐治亚州的损失又有多惨重啊!
\par 天哪!“卡尔弗特——雷福德,中尉。”雷福!她突然记起了很久很久以前的那一天,他们一块离家出走,可黄昏时又回家来了,因为他们都饿了,而且害怕天黑。
\par “方丹——约瑟夫·K,列兵。”坏脾气的小个子乔!而萨莉的孕期还没过呢!
\par “芒罗——拉斐特,上尉。”拉斐特已经和凯思琳·卡尔弗特订婚了。可怜的凯思琳!她的损失是双重的,既失去了一个兄弟,又失去了心爱的人。可萨莉的损失更大——一个兄弟和一个丈夫。
\par 噢,这太可怕了。她几乎不敢再往下看。白蝶姑妈靠在她肩膀上,气喘吁吁、唉声叹气的。思嘉不客气地把她推到马车的一角,继续往下看。
\par 肯定,肯定——名单上不可能有三个姓“塔尔顿”的人。也许——也许印刷工匆忙间弄错了。可是没有。他们都在那。“塔尔顿——布伦特,中尉。”“塔尔顿——斯图尔特,下士。”“塔尔顿——托马斯,列兵。”而博伊德在战争开始那一年就死了,埋在弗吉尼亚的一个只有上帝才知道的地方。塔尔顿家所有的男孩都走了。汤姆,还有慵懒、双腿修长的双胞胎,以及他们热衷的闲聊、荒唐的恶作剧,还有优雅得像个舞蹈教练、说话像黄蜂般刻毒的博伊德。
\par 她再也读不下去了。她不知道是不是还有其他和她一起长大、一块跳过舞、互相调过情、和她接过吻的小伙子的名字也在名单上。她真希望自己能哭出来,能做些什么以减轻正在向她的喉咙深处抠挖的铁爪带来的痛苦。
\par “对不起,思嘉,”瑞德说。她抬头看着他。她已经忘了他还在那待着。“有很多你的朋友吗?”
\par 她点了点头,挣扎着说:“县里几乎每一家都有人——还有——塔尔顿家的三个男孩。”
\par 他一脸肃穆,几乎是一脸忧郁,眼里也没有了嘲弄的意味。
\par “这还没完呢。”他说,“这只是第一批名单,而且不全。明天的名单还会更长。”他放低声音,好让坐在附近的马车上的人听不见他说的话。“思嘉,李将军一定是打输了。我在总部听说,他已经撤到马里兰了。”
\par 她抬起头,一双惊恐的眼睛看着他,可她恐惧的心理并不是李将军的失败引起的。明天还会有更长的名单!明天。起先,希礼的名字不在名单上,她太高兴了,还没想到明天呢。明天。哦,此时此刻,他也许就已经死了,而她要等到明天才会知道,或许是从明天起一星期后才会知道。
\par “噢,瑞德,为什么要打仗呢?让北方佬出钱买黑奴不是好多了——或者我们干脆无偿地把黑奴送给他们,也比发生这一切好多了呀。”
\par “这不是黑奴的问题,思嘉。这只是借口而已。因为男人喜欢打仗,所以总是会有战争的。女人不喜欢,可男人喜欢——是的,比对女人的爱还更胜一筹。”
\par 他嘴角撇着,又挂上了他惯有的笑容,脸上严肃的表情不见了。他举了举他宽大的巴拿马草帽。
\par “再见了。我要去找米德医生了。我想,由我来告诉他他儿子的死讯,他一定感觉不到这其中的讽刺意味,但只是暂时的。以后,想到一个投机商给他捎去了一个英雄的死讯,他很可能会很痛恨的。”
\par  
\par 思嘉给白蝶小姐喝了些棕榈酒,让她躺到床上,叫普里西和厨娘照看她,自己下楼来到街上,到米德家去。米德太太和菲尔待在楼上,等着她丈夫回来。媚兰坐在客厅里,和一群充满同情心的邻居一起低声交谈着。她手里拿着针线和剪刀,正忙着改制一件埃尔辛太太借给米德太太的丧服。屋里已经充满了一种家制黑色染料味道,因为在厨房里,抽泣不止的厨娘正在大大的洗锅中搅着米德太太的所有衣服。
\par “她现在怎么样?”思嘉轻声问道。
\par “一滴眼泪也没有,”媚兰说,“女人要是哭不出来,那是很可怕的。我真不知道男人不哭出来是怎么承受一切打击的。我想,大概是因为他们比女人更坚强、更勇敢吧。她说她要亲自到宾夕法尼亚去把他的遗体运回来。医生是不能离开医院的。”
\par “这于她是太痛苦了!干吗不让菲尔去?”
\par “她担心,他一离开她的视线就会去参军。你知道,对他那个年龄的孩子来说,他个头挺大的,他们现在已经在招募十六岁的男孩了。”
\par 邻居们一个个悄悄地走了,不愿意在医生回家来的时候还在场。只有思嘉和媚兰还留在那,坐在厅里做着针线。媚兰看上去很伤心,但很平静,虽然眼泪还在不停地往下落,滴到她手里拿着的布料上。显然,她根本没有意识到,战争还在继续,而此时此刻,希礼也许已经牺牲了。思嘉心里一片慌乱,她不知道该不该告诉媚兰瑞德的话,让她也难过难过,以使自己得到安慰,还是自己知道就好了。最后,她决定还是不说为好。让媚兰认为她太担心希礼,那是绝对不行的。那天早晨,每个人,包括梅利和白蝶,都对自己的担忧太专注了,没有人注意到她的行为。她为此不禁对上帝大大感激一番。
\par 她们静静地缝了一会,听到外面有了声响。她们从窗帘里往外窥视着,看到米德医生正在下马。他双肩松垂,低着头,灰白的胡须像扇子一样散落在胸前。他慢慢走进屋来,放下帽子和包,默默地吻了吻两个姑娘,然后步履蹒跚地走上楼。一会儿,菲尔下来了,人又瘦又长的,一脸懊丧之情。两个姑娘用眼神表示出欢迎他加入她们的邀请,但他径直走到前面的游廊上,坐在最上面一级台阶上,把头埋在两个手掌之间。
\par 梅利叹了口气。
\par “他都要疯了,因为他们不让他去打北方佬。已经十五岁!噢,思嘉,有这么一个儿子真是太好了!”
\par “而且让他被杀死?”思嘉想的是达西,唐突地说。
\par “有了个儿子,即使他被杀了,也比从来没有儿子要好得多。”媚兰哽咽着说,“你不理解的,思嘉,因为你已经有了小韦德,可我——噢,思嘉,我太想要个孩子了!我知道,你一定会认为,我把这说出来真是太可怕了,可是这是真的,这也是每个女人想要的,你是知道这一点的。”
\par 思嘉硬忍住,不露出蔑视的神情来。
\par “如果上帝有意愿,希礼要被——被召唤走,我觉得我是可以承受得了的,虽然说如果他死了,我也宁愿去死。可上帝会给我力量承受这一点的。可若他死了,却没有——没有他留下的孩子来安慰我,那我就受不了了。噢,思嘉,你太幸运了!虽然你失去了查理,可你有他的儿子。可如果希礼走了,我就什么也没有了。思嘉,原谅我,可有时我确实很忌妒你——”
\par “忌妒——我?”思嘉叫了起来,心里愧疚不已。
\par “因为你有个儿子,而我没有。有时候,我甚至假装着韦德是我自己的儿子,因为没有孩子太可怕了。”
\par “胡——说——八——道!”思嘉松了口气。她瞟了一眼红着脸低头做针线的小个子女人。媚兰也许是想要孩子,可她肯定没有能怀孩子的身材。她只比一个十二岁的孩子高出一点点,臀部窄得像个孩子的一样,胸部也很扁平。媚兰有孩子,这个念头本身就使思嘉很反感。这勾起了太多她无法承受的思绪。如果媚兰有了希礼的孩子,这就像是从思嘉这里拿走了本该属于她的什么东西一样。
\par “请原谅我说了那些有关韦德的话。你知道,我太爱他了。你不生我的气吧,不会吧?”
\par “别傻了,”思嘉简短地说,“到游廊上去,帮帮菲尔。他在哭呢。”

\subsubsection{第十五章}

\par 被敌军逼回弗吉尼亚的部队驻扎在拉皮丹的冬季营房——自葛底斯堡被打败之后,这支军队已是筋疲力尽了——因为圣诞节要到了,希礼休假回到家中。思嘉已有两年多没见到他了,这一见面,不禁为自己强烈的感情吃了一惊。她站在十二棵橡树的游廊上看着他和媚兰结婚时,她认为自己再也不会像在那一刻那样带着一颗伤心欲碎的心爱着他了。可是现在,她意识到已经远去的那个夜晚,那种感情只不过是一个被宠坏的孩子得不到玩具时会有的感情罢了。现在,她的感情因长期的相思而急剧增强,况且,她还不得不保持沉默,这种压抑反而使她对他的爱意越来越深。
\par 卫希礼穿着已经褪色、打着补丁的军装,淡黄色的头发已被夏日的艳阳晒成了亚麻色,跟战前她曾经爱得死去活来的那个随和、眼神慵懒的小伙子相比,他整个儿跟换了个人似的。他更是比她激动一千倍。现在的他脸色黝黑、身材瘦弱,过去的他可是面色白皙、身材颀长的。现在,他嘴边垂挂着长长的金色胡须,修剪成骑兵的式样,十足一个完美士兵的形象。
\par 他穿着老旧的军服,极具军人风度地站得笔直,手枪套在破旧的枪套里,已磨损的刀鞘在他高帮的靴子上一碰一碰的,潇洒极了,已黯然失色的马刺闪着黯淡的微光——他已是南部邦联的卫希礼少校了。他现在已有了命令人的习惯,颇有自立和权威的安然神态,嘴角已经出现了岁月刻下的无情的皱纹。宽宽的肩膀和眼里冷酷明亮的光芒都有了某些陌生的新东西。过去懒洋洋、无精打采的他,现在就像正在四处觅食的猫一样警觉,那警觉程度就犹如神经一直绷得像小提琴的琴弦一样紧似的。他眼里有种疲倦、鬼魂般的神情,脸上的颧骨依然很好看,被太阳晒得黝黑的皮肤绷得紧紧的——依然是她那英俊的希礼,却又变得很不一样了。
\par 思嘉曾计划到塔拉去过圣诞节,但自收到希礼的电报后,这世上就再也没有什么力量可以把她从亚特兰大拉走了,即使是大失所望的埃伦直接命令她也不顶事了。如果希礼打算去十二棵橡树,她倒是会忙不迭地到塔拉去,好离他近些的;但他却写信叫他的家人到亚特兰大来和他团聚。卫先生、哈尼和英蒂已经来到城里了。回塔拉的家中去?分别了两年时间却要错过和他见面的机会?错过听他那使人的心跳都会加快的声音,错过从他的眼神里看出他还没有忘记她?绝对不行!不要说为了自己的妈妈,就算是为了世界上所有的妈妈也不行。
\par 希礼是圣诞节前四天回家来的,同行的还有同样在休假的一群同县的小伙子。自葛底斯堡战役后,这个群体的人数已经令人伤心地减少了。他们中有凯德·卡尔弗特,他既瘦削又憔悴,而且还不停地咳嗽;芒罗家的两个男孩,这是他们一八六一年以来的第一次休假,激动得话说个没完;还有亚历克斯·方丹和托尼·方丹,醉得够水平的,吵吵嚷嚷的,动不动就吵架。这群人转车得等两个小时,因为这群人中没喝醉的人总得费口舌使方丹家的这两个活宝不会互相打架,或是在车站和陌生人打架,希礼便把他们全都带到白蝶家来了。
\par “你们会认为他们在弗吉尼亚已经打够了,”凯德看着那两个活宝挖苦地说,他们正在为谁先吻焦急不安、受宠若惊的白蝶姑妈而像斗鸡一样争个不休。“可是没有。自我们到里士满后,他们就一直喝得烂醉、寻衅闹事。纠察队把他们逮住了,要不是希礼的花言巧语起了作用,他们就得到监狱里去过圣诞了。”
\par 可是,他说的话思嘉几乎一个字也没听进去。又和希礼待在同一个屋里,她简直是欣喜若狂了。这两年中,她怎么可能认为还有其他英俊、令人激动的好男人呢?希礼还在人世的时候,她怎么可能容忍得了和别人调情说爱呢?他又回家来了,隔开他俩的只是客厅里的小地毯。他坐在沙发上,一边坐着梅利,另一边是英蒂,哈尼则勾着他的肩膀。每次她一看到他坐在那,就得使尽全身的力气憋住,不让自己高兴得哭出来。要是她也有权利坐在他身边,手挽着他的手臂就好了!要是她可以每隔几分钟就能拍拍他的袖子,拉着他的手,用他的手帕擦去高兴的泪水,那就太美了。因为媚兰就在毫不害臊地做着这些事呢。她太幸福了,根本顾不上感到害羞或是应该含蓄一些。她挽着丈夫的胳膊,用眼神、微笑和泪水公然表示出无限柔情蜜意。思嘉也太高兴了,对此也并没有愤愤不平,她高兴得顾不上忌妒了。希礼终于回家来了!
\par 她不时用手摸摸他吻过的面颊,重新回味着他嘴唇印在上面时的激动心情,并且对他微笑着。当然,他第一个吻的不是她。梅利一下就扑入他的怀里,哭得语无伦次的,一直抱着他,好像再也不让他走似的。接着,英蒂和哈尼也拥抱了他,简直是把他从媚兰手里硬拉出来的。接着他又吻了他父亲,体面而极富爱意地拥抱了他,使他们之间那种强烈而无须言语表达的感情显露无遗。然后是白蝶姑妈,她一双发育不全的小脚正激动得上上下下跳个不停呢。最后,他才转向她,此时的她正被所有的小伙子包围着,都声称要吻她呢。他说:“噢,思嘉!你这无比漂亮、无比漂亮的小东西!”然后在她面颊上吻了一下。
\par 这一吻把她准备好要说的欢迎词都吻得飘到九霄云外去了。好几个小时以后,她才记起来他没有吻她的嘴唇。接着,她就头脑发热地想,要是他单独跟她见面的话,他就会吻她的嘴唇了,他肯定会弯下颀长的身躯,俯视着她,把她拉起来,让她踮着脚尖,久久地、久久地抱着她。就因为这么想使她很高兴,所以她就相信他是会那么做的。然而,还是有时间做所有的事情的,有一整个星期呢!她一定能够想办法让他单独和她待在一起,对他说:“你还记得我们俩过去经常沿着我们秘密的马道骑马的事吗?”“你还记得那天晚上我们坐在塔拉最高的台阶上,你朗诵那首诗歌时,月亮是什么样子的吗?”(我的天!那首诗歌的题目到底叫什么来着?)“你记得那天下午我扭伤了脚,你在黄昏时抱着我回家的情景吗?”
\par 噢,还有这么多事情她可以用“你记得吗?”来开头的。还有这么多珍贵的记忆可以把他带回到往昔那些美好的岁月。当时他们就像无忧无虑的孩子似的在县里闲逛,这么多事情都能使他回忆起韩媚兰插足以前的那些日子。而他们谈话的时候,或许她能从他的眼里看出越来越强烈的感情,暗示着在他对媚兰的那种丈夫对妻子的感情这道藩篱之后,他还在乎她,就像那天野餐会上他突然把真情说出来时那么动情地在乎她。她还没有想到去计划一下,如果希礼用明白无误的话语向她宣称对她的爱的话,他们又该怎么办。知道他确确实实在乎她,这就够了……是的,她能等,可以让媚兰先享用能抓着他的胳膊痛哭的幸福时刻。她的机会也会到来的。说穿了,像媚兰这样的姑娘怎么会知道什么才是爱情呢?
\par “亲爱的,你真像个叫花子,”媚兰说道,归家带来的第一阵激动已经过去了。“谁给你补的军服,他们干吗用蓝色的补丁呢?”
\par “我还以为我看上去潇洒得很呢,”希礼审视着自己的外表,这么说道,“你只要把我和那边那些乌合之众比一比,你就会对我更加欣赏了。是莫斯给我补的军服,考虑到他战前从未拿过缝衣针,我认为他补得真是好极了。至于蓝色的补丁嘛,如果要你作一选择,要么裤子上有洞,要么用一个被抓住的北方佬军服上的布片当补丁把洞补住——哦,那其实根本就无所谓选择了。至于说看上去像叫花子,你的丈夫没有光着脚回家来,你就应该谢天谢地了。上星期,我那双旧靴子完全破了,要不是我们运气好,打死了北方佬的两个侦察员,我们就只好把睡袋绑在脚上回家来了。他们中有一个的靴子我穿着倒是相当合适。”
\par 他伸出修长的腿让他们欣赏,高筒靴上满是划破的痕迹。
\par “另一个侦察兵的靴子我穿不合脚,”凯德说,“它们比我的小了两号,就这时候还使我痛得要死呢。但我还是要体面地回家去。”
\par “这只自私的猪不肯把它们给我们,”托尼说,“它们穿在我们小巧、贵族型的方丹家的人脚上一定非常合适。见他妈的鬼,我真没脸穿着这种粗劣的靴子去面对妈妈。战前,连我们家的黑奴穿这个她也不允许的。”
\par “别担心了,”亚历克斯说道,眼睛瞟着凯德的靴子。“我们坐火车回家时可以在火车上从他脚上脱下来。我倒不怕去面对妈妈,可我他妈——我是说,我可不打算让迪米蒂·芒罗看见我的脚趾都露出外面来了。”
\par “哟,它们是我的靴子了,我最先说我要的,”托尼说,开始对他的兄弟怒目而视;媚兰担心可能又会发生一次著名的方丹家族式的争吵,赶紧出来调停。
\par “我本来可以让你们姑娘们看看我的大胡子的,”希礼可怜兮兮地磨搓着自己的脸,上面还未痊愈的剃刀留下的疤痕还清晰可见。“那胡子可真够漂亮的,要我自己来说的话,不论是杰布·斯图尔特还是内森·贝德福德·福里斯特都没有比我更漂亮的胡子了。可我们到了里士满时,那两个无赖,”指的是方丹家的两个男孩,“认为,他们俩都把胡子剃掉了,我的也必须剃掉。他们把我按倒,给我剃掉了,我的头没有和胡子一起掉下来,那可真是奇迹啊。要不是埃文和凯德前来干预,连我的髭须也保不住了。”
\par “真是毒蛇!卫太太!你还得感谢我们哪。要不然你决不可能认出他,让他进屋来的,”亚历克斯说,“我们这么做是为了感谢他说服了纠察队,没把我们送进监狱去。如果你这么说话,我们现在就马上把你的髭须也剃掉。”
\par “噢,不,谢谢你们了!”媚兰赶紧这么说,紧紧抓住希礼,一副害怕的神情。因为这两个皮肤黝黑的小个子男人看上去什么暴行都做得出来。“我觉得这髭须漂亮极了。”
\par “这就是爱,”方丹兄弟俩说,互相郑重地点了点头。
\par 希礼走到寒风中送小伙子们,他们坐着白蝶姑妈的马车到车站去了。媚兰抓住思嘉的手臂。
\par “他那军服是不是太可怕了?我做的上衣是不是会给他一个惊喜?噢,要是我还有足够的布料做条裤子就好了!”
\par 对思嘉来说,给希礼做上衣是个令她痛苦的话题,因为她非常热切地希望,送这件圣诞礼物的是她自己,而不是媚兰。几乎可以毫不夸张地说,做军服的灰色呢绒现在可是比红宝石还更价值连城,希礼穿的已是大家熟识的家纺布。连灰胡桃色布现在也不多了,许多士兵都穿着从被俘的北方佬身上剥下来的衣服,只是用胡桃壳染料把它们染成一种深褐色而已。可是媚兰真是碰到了少有的运气,居然弄到足够做件上衣的绒面呢布料——上衣有点短,可好歹还是件上衣。她曾在医院护理过一位查尔斯顿的小伙子。他去世后,她从他头上剪下了一绺头发,寄给了他妈妈。一道寄去的还有他口袋里不多的几件物品以及一封安慰性的描述他度过一生最后几个小时的信,信中没有提到他死前所遭受的痛苦。于是,她们之间开始了通信来往。知道媚兰也有个丈夫在前线后,那位妈妈给她寄来了一段灰色的布料和铜纽扣,这本是她为她已经死去的儿子买的。这块布料很漂亮,又厚又暖和,还闪耀着微暗的光泽。毫无疑问,这是偷闯封锁线运进来的货物,无疑也是非常昂贵的东西。现在布料已经在裁缝手里了,媚兰正在催他,要他圣诞节早晨要做好。思嘉要能提供做军服所需要的其他东西,她一定是很乐意给的,只是所需要的材料在亚特兰大根本买不到。
\par 她也有件圣诞礼物要送给希礼,但在媚兰的灰色上衣的光彩映照下,她的礼物在意义上就逊色多了。这是个小小的“针线盒”,用法兰绒做的,里面装有一整包珍贵的缝衣针,是瑞德从拿骚买来送她的。还有三条亚麻布手帕,也是瑞德送她的,还有两团线以及一把小剪刀。但她想给他一些私人物品,一些一个妻子能够送给丈夫的东西,一件衬衫、一副长手套,或是一顶帽子什么的。噢,一定要一顶帽子。希礼戴的那顶平顶军便帽看上去可笑极了。思嘉一直就很讨厌这种帽子。如果石墙杰克逊没有戴着阔软边毡帽而戴着这种军便帽,那会是什么样子?那就会使他们一点尊贵的样子也没有。可在亚特兰大,能买到的帽子都是做得很粗劣的羊毛帽,而它们比那圆顶无边的军便帽还要俗气。
\par 她想到帽子的时候便想到了白瑞德。他的帽子太多了,夏天戴的宽边巴拿马帽、正式场合戴的海狸毛皮帽、打猎时戴的帽子、褐色、黑色和蓝色的阔软边呢帽。他有什么必要有这么多帽子呢?而她的希礼却要骑着马冒雨行进,雨水从帽子后面直滴到他的领口里。
\par “我要让瑞德把他那顶黑色的新毡帽给我,”她下了这个决心,“我要在边上缝一条灰色的缎带,缝上希礼的饰环,那看上去一定漂亮极了。”
\par 她的思绪稍停了停,心想如果不找个理由,可能很难得到那帽子。她当然不能让瑞德知道帽子是要给希礼的。哪怕是她只提到希礼的名字,他也会那样令人讨厌地耸起眉毛,他一贯如此,而且很可能会拒绝。哦,她得编造一个哀婉动人的故事,说是医院里有个士兵需要这顶帽子,而永远也不必让瑞德知道事实真相。
\par 那一整个下午,她想方设法和希礼单独待在一起,哪怕是几分钟也好。可是媚兰总是跟在他身边,还有英蒂和哈尼,她们那苍白、睫毛稀疏的眼里放着光,跟着他在屋里转来转去。看得出来,卫约翰为自己的儿子感到骄傲无比,但连他也没有机会和他静静地谈谈心。
\par 吃晚饭时也一样,他们全都缠着他,问他有关战争的问题。战争!谁在乎战争呢?思嘉认为,希礼对这一话题也并不是很在乎的。他详详细细地谈着,不时发出一阵大笑。他完全控制了整个谈话的局面,比以往任何时候都更像个主讲,可他似乎说得并不多。他告诉他们朋友们的一些笑话和有趣的故事,欢快地谈着那些临时凑合的代用品,把饥饿、冒雨长途行军看成是微不足道的事,还详细描述了在从葛底斯堡撤退时李将军骑马经过时的样子,他问道:“先生们,你们是佐治亚的军队吗?哦,没有佐治亚人,我们就没法打下去啦!”
\par 思嘉隐约感到,他谈兴很浓只是为了不让他们问一些他不想回答的问题。每当她看到他的目光里露出犹豫之色,并且在他父亲久久的、忧虑的目光注视下垂下眼睑时,她心里便有了一丝担心和茫然之感。希礼心里到底藏着什么呢?可这感觉一晃就过去了,因为她心里已经装不下别的东西,只有无尽的幸福感和想单独跟他在一起的热望。
\par 这种喜悦之感一直延续着,最后,围着一圈坐在未加盖的炉火前的每一个人都开始打哈欠了。卫先生和姑娘们告辞到旅馆去过夜。接着,希礼、媚兰、白蝶和思嘉在彼德大叔举灯照明下上了楼,这时思嘉才感到一丝寒意掠过心头。直到他们站在楼上的过道里的那一刻,希礼都还是她的,只是她一个人的,即使她整个下午都没有和他私下说过一句话,那也一样。可是现在,她跟他道了晚安,看见媚兰的脸上突然泛上一片红晕,浑身打颤,两眼望着地毯,虽然某种可怕的情感似乎攫住了她的心,但她还是露出羞答答的幸福样。希礼打开房间门时,媚兰连头都没抬起来,只是快步走了进去。希礼也匆匆忙忙道了声晚安,都没看上思嘉一眼。
\par 门在他们身后关上了,留下思嘉站在那目瞪口呆的,顿感孤独寂寞。希礼不再是她的。他是媚兰的了。只要媚兰还活着,她就可以走进房间,把门关上——把世上其余的一切都关在门外。
\par  
\par 现在,希礼马上要走了,要回到弗吉尼亚去,回到雨雪中去长途行军,回到雪地里的露营地去忍冻受饿,回到痛苦而艰难的军营中去。他那一头金黄色的头发漂亮而有光泽,颀长的身材令人骄傲。如今却要去冒险,兴许转瞬间就会失却生命,就像一只蚂蚁被粗心的脚后跟踩在脚下一样。过去的一周恍恍惚惚的,美妙得像梦境一般,充实的每一小时有多幸福啊,如今却都已经过去了。
\par 一个星期飞快地过去了,如同一场梦。梦里散发着松枝和圣诞树的芬芳,小巧的蜡烛和家制的金银丝织品闪闪发亮。这场梦里的每一分钟,过得就像心跳的频率那么快。在这令人激动得透不过气来的一周里,内心有某些东西促使思嘉痛苦而快乐地把每一分钟都浓缩起来,把发生的一切留在记忆深处,好等他走后好好回味回味。未来的几个月中,她可以在闲暇时细细品味这些发生的事——跳舞、唱歌、欢笑、去给希礼拿东西、猜测他想要的东西、他笑的时候跟着他笑、他说话的时候则侧耳静听、目光追随着他的身影,好让他挺直的身体的每一条线条、眉毛的每一耸动、嘴角的一撇一动都永久地印在你的脑子里——因为,一个星期过得是这么快,而战争却了无止期。
\par 她坐在客厅里的沙发上,腿上放着临走前要送给他的礼物,在等着他。他正在跟媚兰告别。思嘉祈祷着他下楼来时只有自己一个人,那上帝就是赐给她能单独和他待在一起的宝贵的几分钟了。她竖起耳朵,紧张地听着从楼上传来的声响。可是屋里静得出奇,静得连她自己的呼吸声听起来都很大声。白蝶姑妈已经在自己的房里埋在枕头里大哭特哭,因为希礼半小时前就已经跟她告别过了。媚兰卧室房门紧闭,既没有喃喃低语声,也没有哽咽的说话声。对思嘉来说,他已经在那房里待了好几个小时了。对他待在那里和媚兰告别的每一秒钟,她都反感到极点,因为一分一秒正在飞逝而去,而他的时间又是如此匆促。
\par 她想起了一星期中本打算对他说的所有的话。可一直都没有机会说,她也知道,或许她永远都不会有机会把这些话说出来的。
\par 这些傻乎乎的话,诸如:“希礼,你会很小心的,对不对?”“别湿了脚。你会很容易感冒的。”“别忘了在衬衫底下铺一张报纸在胸前。这挡风的效果挺好的。”可是还有别的话,她想要说的更重要的话,还有她想要听他说的话,那来得更加重要。即使他没有直接说出来,她也想从他眼神里意会到。
\par 有这么多话要说,而现在却没有时间了!如果媚兰跟着他到门口,到上马车的地方,那连剩下的不多的几分钟也会从她手里被夺走的。这过去的一星期中,她怎么没有找找机会呢?可是媚兰总是在他身边,两眼深情地望着他,屋里还总是有朋友、邻居和亲戚。从早到晚,希礼从来就没有独自一人待着的时候。到了晚上,卧室的门便关上了,只有他和媚兰独自待在一起。在过去的几天中,他一次也没有向思嘉传递过一个眼神,或是透露过一个字,只有一个哥哥对妹妹或是对朋友——终生的朋友显示的友爱。还不知道他是不是还爱着她,她是不能让他走的,而且也许是永远离开不再回来。那样的话,即使他死了,她也可以从他默默的爱中得到些暖人的安慰,直到她生命的最后一刻。
\par 似乎经过漫长的等待之后,她听到了楼上卧室里他的靴子的声音,还有开门和关门的声音。她听到他走下楼梯。独自一人!谢天谢地!媚兰一定是被离别的悲伤击倒了,没法离开房间。现在,在这宝贵的几分钟里,她可以单独和他待在一起了。
\par 他慢慢走下楼梯,踢马刺叮当作响,她还能听到他的马刀碰到高筒靴的隐隐约约的啪啪声。他来到客厅时,眼里现出忧郁之情。他想笑一笑,可他拉长着脸,脸色苍白,就像个体内有个伤口正在流血的人一样。他走进来时,她站起身来,带着她特有的那种傲慢之态,心想他是她见过的最英俊的军人了。在彼德大叔的精心擦拭下,他长长的手枪皮套和皮带闪闪发亮,银白色的踢马刺和刀鞘也熠熠生辉。新的上衣并不很合身,因为裁缝一直在赶活,有些针脚也太粗糙。灰色的新上衣明快的色彩和破旧、打着补丁的灰胡桃色裤子及刮痕累累的靴子极不协调,令人败兴。但在她看来,即使他没有银色的盔甲,他依旧是个神采奕奕的骑士。
\par “希礼,”她突然问道,“我能不能送你到火车站去?”
\par “请你别送了。我父亲和妹妹在那呢。不管怎么说,我宁愿记住你在这和我告别的情景,而不是在车站那令人心惧的地方。要留在记忆中的事情太多了。”
\par 她马上打消了这个念头。如果不喜欢她的英蒂和哈尼在离别的现场,她就不会有机会私下和他说话了。
\par “那我就不去了,”她说,“你瞧,希礼!我还有件礼物要给你。”
\par 到了把礼物给他的时候,她倒有点害羞了。她打开一个小包。这是条黄色的长饰带,是用中国丝绸做的,边上缘饰很多。几个月前,白瑞德从哈瓦那给她带来了一件黄色的披巾,上面华丽地绣着品红和蓝色的花鸟图案。这过去的一星期中,她耐心地拆下了所有的绣花,把方形的丝绸剪了下来,缝成了长条的饰带。
\par “思嘉,这太漂亮了!是你自己做的吗?那我会更加珍惜的。给我戴上,亲爱的。小伙子们看到我这么光彩的上衣和饰带,一定会眼红的。”
\par 她把色彩明快的饰带围在他细长的腰际,皮带的上方,在尾部打了个情人结。媚兰当然可以送给他新的上衣,但这条饰带是她的礼物,是她自己给他带到战场上去的秘密酬劳,这会使他每次一看到它便想起她。她退后一步,自豪地审视着他,心想,就连杰布·斯图尔特戴着他那眩目的饰带和羽饰,看起来也没有她的骑士那么英俊漂亮。
\par “这太漂亮了,”他再次说道,用手指摸着缘饰,“可我知道,你是用一件衣服或是披巾改制的。你不该这么做的,思嘉。现在漂亮的东西太难弄到手了。”
\par “噢,希礼,我——”
\par 她本想说:“如果你想要的话,我愿意把我的心剜出来让你带去的,”可她说出口的是:“我愿意为你做任何事!”
\par “真的吗?”他问道,脸上的忧郁之情少了一些,“那你确实可以为我做件事,思嘉,我不在的时候,这会使我更安心一些。”
\par “什么事?”她高兴地问道,准备什么奇事都答应他。
\par “思嘉,你能不能帮我照顾媚兰?”
\par “照顾梅利?”
\par 她的心往下一沉,一阵失望之感袭上心头,她痛苦极了。这么说,这就是他对她的最后要求了,而她却期盼他能对她允诺一些美好、惊人的事!接着,她便怒火中烧了。这一刻是她和希礼待在一起的时刻,是她独自和他待在一起的时刻。然而,虽然媚兰不在这,可她苍白的身影却还横在他们中间。他怎么能在他们告别的时刻提起她的名字呢?他怎么能要求她做这种事情呢?
\par 他并没有注意到她脸上的失望之情。他的眼光像过去一样从她身上穿过去,看到了她以外的别的东西,根本没在看她。
\par “是的,关照她一下,照顾照顾她。她太脆弱了,可她根本没意识到。她会让护理和缝制衣服这些事情把她自己累垮的。而她又是这么善良、胆怯。除了白蝶姑妈、亨利叔叔和你之外,她没有更亲近的亲戚,只有梅肯的伯尔家,可他们已是隔了两层的姑表亲。而白蝶姑妈——思嘉,你知道的,她就像个孩子。亨利叔叔又已是个风烛残年的人了。媚兰这么爱你,不仅仅是因为你是查理的妻子,而且是因为——哦,因为你就是你,她爱你就像爱一个妹妹一样。思嘉,如果我被杀了,她又没有人可以帮她,那她会发生什么事呢?一想到这点,我便一直做噩梦。你答应我吗?”
\par 她甚至连他最后的要求也没听见,那些预示凶兆的话“如果我被杀了”使她感到可怕极了。
\par 每天她都在读伤亡名单,心都提到嗓子眼里。她知道,一旦他出了什么事,那世界末日也就到了。可她总是、总是有一种内心的感觉在告诉她,就算南部邦联所有的部队都被歼灭了,希礼也会平安无事的。可现在他却说出了最可怕的话!她浑身都起了鸡皮疙瘩,恐惧之感袭上她的心头,这是她无法用理性与之抗衡的迷信式的恐惧。她身上的爱尔兰血统足以让她相信预见力,特别是预见死亡的时候。在他大大的灰色眼睛里,她看到了一种深深的忧伤,这她只能解释为一个感到冰冷的手指在他肩膀上触摸、已经听到彭西\footnote{彭西为苏格兰及爱尔兰一带传说中的一个报丧女妖。谁家听到她的哀哭,谁家就会死人。}的哀哭的男人才有的忧伤。
\par “你千万不能这么说!你想都不能这么想的。提起死运气会不好的!噢,赶快祈祷吧,快点!”
\par “你为我祈祷吧,再点燃些蜡烛。”他说,听到她声音里惊恐万分、迫不及待的口吻,他笑了。
\par 她已经不会回答了,脑海里已经出现一幕幕可怕的画面,把她给惊懵了。希礼躺在弗吉尼亚的雪地里死去,离她远远的。他在继续说着,他的声音里有一种语气、一种忧伤、一种无可奈何的口气,这更增加了她的恐惧,使她把刚才的愤怒和失望全都忘掉了。
\par “我是因为这个原因请求你的,思嘉。我也说不准我会发生什么事或是我们任何一个人会发生什么事。可是最终结束时,我会离此很远,就算我还活着,也会离此太远,无法关顾媚兰。”
\par “结——结束?”
\par “战争结束——也是世界的末日。”
\par “可是,希礼,你当然不会认为北方佬会打败我们的,对不?这一整个星期里,你都在讲李将军有多么强大——”
\par “这一整个星期我都在说谎,就像所有在休假的人一样。现在还没有必要让媚兰和白蝶姑妈担惊受怕,我干吗要让他们担惊受怕呢?是的,思嘉。我认为北方佬会打败我们。葛底斯堡是末日来临的开端。家里的人们不知道而已。他们无法意识到我们的境况现在是怎么样的,可是——思嘉,现在我手下的一些官兵已经是赤着双脚在作战,而弗吉尼亚的雪又下得很厚。每当我看到他们受冻的双脚包在破布和破旧的袜子里,看到他们留在雪地里的带血的脚印,而又明白自己却穿着一双靴子——哦,我总觉得我应该把自己的送掉,也光着脚才好。”
\par “噢,希礼,答应我,别把它们送掉!”
\par “我一看到那种情形,再看看北方佬的情况——我就看到了结果。哦,思嘉,北方佬用钱从欧洲几千几千地雇佣士兵!我们最近抓住的大多数俘虏甚至连英语都不会讲。他们都是德国人、波兰人和讲盖尔语的野蛮爱尔兰人。可我们一旦少了一个人,就没有人来代替他了。我们的鞋子穿破之后,就再也没有别的鞋子了。我们已经被逼入绝境了,思嘉。我们总不能跟整个世界打吧。”
\par 她的思绪很乱:“让整个南部邦联在尘土中灭亡吧。让世界末日来临吧,但你不能死!如果你死了,我也没法活了!”
\par “我希望你不会把我说的话告诉别人,思嘉。我不想让别人惊恐不安。哦,亲爱的,要不是我得向你解释我为什么要叫你照顾媚兰的话,我也不会说这些话让你担惊受怕的。她是这么脆弱,而你是如此坚强,思嘉。如果我出了什么事,只要知道你们俩在一起,那对我就是个安慰。你会答应的,对吗?”
\par “噢,是的!”她叫了起来,此时此刻,看到死亡的威胁近在咫尺,她什么都会答应的。“希礼,希礼!我不能让你走!我不够坚强,无法面对这一切!”
\par “你必须坚强,”他说,声音变得难以捉摸,有共鸣感,更加深沉,话说得很急,好像内心的急迫感促使他这么说似的。“你必须坚强。要不然我怎么受得了?”
\par 她的目光飞快地在他脸上搜寻着什么,同时感到很高兴,不知道他的意思是不是说,要离开她使他心都碎了,甚至就像使她心碎一样。他的脸照样拉长着,就像他和媚兰告别完下楼来的时候一样,可从他眼里,她什么也看不出来。他弯下身子,把她的手握在自己的手里,轻轻地在她额头上吻了一下。
\par “思嘉!思嘉!你这么善良、这么坚强、这么好。还这么漂亮,不单是你美丽的面孔,亲爱的,而是你的一切,你的身体、你的思想和你的心灵。”
\par “噢,希礼,”她幸福地喃喃低语,他的话和他触到她脸上的手使她激动不已。“只有你才——”
\par “我喜欢这么认为,也许我比大多数人都更了解你,我能够看见埋藏在你心灵深处的美,其他人都太粗心,或是匆匆忙忙的,没有注意到。”
\par 他停下不说了,手从她脸上垂了下来,但他的眼睛还在和她的眼睛对视着。她等了一会,屏住呼吸等他继续说下去,踮着脚等着听他说那三个有魔力的字眼。可她没有听到。她狂乱地巡视着他的脸,嘴唇颤抖着,因为她看出,他已经把话说完了。
\par 希望再次遭到挫败,这是她的心无法承受的。她不禁用孩子式的低语叫了声“噢!”然后颓然坐了下来,泪水浸湿了她的双眼,刺得她眼睛生疼。接着,她听到了车道上传来了不祥的声音,就在窗户外边,这声音更给她带来了希礼要离开的紧迫感。异教徒听到卡戎\footnote{卡戎是希腊神话中渡亡魂过冥河去的阴间的神。}的小船周围冥河水的流淌声时,也不可能有像现在这么凄凉寂寞的感觉。彼德大叔把自己裹在一床被子里,正在把马车赶出来,好送希礼到火车站去。
\par 希礼轻轻说了声“再见”,从桌上抓起她从瑞德那里花言巧语骗来的宽宽的毡帽,走进黑漆漆的前过道。他手已抓着门把,又转过身来久久地、绝望地凝视着她,好像要把她的脸和身体的每一个细微部分都装在脑海里带走似的。泪水模糊了她的视线,透过模糊的泪眼,她还是看到了他的脸,她喉咙里似被什么东西堵住似的,痛苦极了。她知道他就要走了,不能再得到她的关心,要离开这所房子这安全的避风港,远离她的生活,也许是永远地离她而去,可他却没有说出她如此渴望听到的话。时间正像推动水车的水流一样一分一秒地过去,现在已经太迟了。她跌跌撞撞地跑过客厅,跑进过道,抓住他饰带的末梢。
\par “吻我一下,”她喃喃而语,“给我来个吻别。”
\par 他双手温柔地抱住她,低下头凑近她的脸。他的嘴唇刚触到她的嘴唇,她便双臂紧紧勾住他的脖子,似乎都要窒息了。在飞逝而过、无法估量的转瞬间,他用力把她的身体靠在自己身上。接着,她便感到他全身的肌肉突然都紧张起来。他迅速把帽子扔到地上,伸手把她的双手从脖子上掰开。
\par “不行,思嘉,不行。”他低语,把她交叉着的双腕握在手里,直握得她发疼。
\par “我爱你,”她哽咽着说,“我一直在爱着你。我从来没有爱过别人。我和查理结婚只是为了——为了气你。噢,希礼,我太爱你了,我可以一路步行到弗吉尼亚去,只是为了能离你近一些!我可以给你做饭,给你擦鞋,为你饲养马——希礼,说你爱我!这可以让我下半辈子就靠这活下去!”
\par 他突然弯下腰拾起帽子,她扫视了一眼他的脸。这是她所见过的最最不快乐的脸了,那脸上所有的孤傲已经荡然无存。写在脸上的是他对她的爱和因她爱他而感到的喜悦,可是,与之抗争的却是屈辱和绝望。
\par “再见,”他哑着嗓子说道。
\par 门嘎吱一声开了,一阵冷风吹进屋子,把窗帘吹得飘动不已。思嘉看着他沿着人行小路朝马车跑去。马刀在冬日微弱的阳光下闪着微光,饰带的缘饰则逍遥自得地跳动不已。看到这里,她不禁浑身颤抖起来。

\subsubsection{第十六章}

\par 一八六四年,心情阴郁、意气消沉、满是凄风冷雨的一月和二月悄然逝去了。南部邦联不仅在葛底斯堡和维克斯堡战役中遭到惨败,而且整个南方的战线也已经崩溃。艰苦鏖战之后,几乎整个田纳西州都已经被北部联邦的军队占领了。然而,即使又遭受了这一惨重的损失,南方人的精神并没有崩溃。千真万确,坚强不屈的决心已经代替了心高气盛的希望,可人们仍然可以在乌云笼罩下找到云朵边缘的银光。这原因之一就是,北方佬在田纳西取得胜利之后,于九月份想乘胜追击,向佐治亚挺进,却遭到了沉重的挫败。
\par 在该州西北角的奇克莫加打了一场恶战,这是开战以来在佐治亚的土地上进行的第一次战役。北方佬占领了查塔努加后,接着就穿过山上的关口,向佐治亚前进。然而,他们却被赶了回去,而且损失很惨重。
\par 南方在奇克莫加获得的这一大胜利,亚特兰大和它的铁路线起了举足轻重的作用。在从弗吉尼亚通往亚特兰大而后再往北通往田纳西的铁路线上,朗斯特里特将军的部队火速开赴战场。全程几百英里的路上,铁轨被清扫得一干二净,为了这次行动,东南部所有可用的车辆全都征集在一块了。
\par 亚特兰大亲眼看见了一列又一列的火车从城里奔驰而过,一辆辆客车、棚车、平板车,满载着高声呼喊着的人们开赴前线。他们没带食物、没有睡觉就来了;没有马匹、救护车和供应物资的火车,等不及休息一会就来了;他们从火车上跳下来就加入了战斗的行列。北方佬便被赶出了佐治亚,退回田纳西。
\par 这是这场战争最伟大的业绩,一想到是自己的铁路线使这胜利成为可能,亚特兰大便自豪无比、洋洋得意。
\par 来自奇克莫加的这条好消息非常鼓舞人心。一整个冬天,南方都需要它来鼓舞人们的士气。现在,没有人否认北方佬都是勇敢善战的斗士,而他们终于也有了不错的将军。格兰特是个屠夫,根本不管取得一次战斗的胜利要杀戮多少人,可他就是能取胜。谢里登则是个会给南方人带来恐惧的名字。还有个叫舍曼的人,他的名字也越来越经常被人们提到。他是在田纳西和西部的战役中声名鹊起的,作为一名坚决而残忍的斗士,这一名声也越来越响。
\par 当然,他们中谁也无法和李将军相比。对将军和军队的信念还是很强。最终一定会胜利的信心从来就没有动摇过。可是战争拖得太久了。已经死了这么多人、伤了这么多人,还有这么多人或瘸或残的,这么多人成了寡妇,这么多人成了孤儿。而等在前面的依然是一场持久而艰苦的战斗,这又意味着更多的人会死去、更多的人会受伤,会有更多的寡妇和孤儿。
\par 使事情更糟的是,一种对那些身居高位的人隐隐约约的不信任渐渐在平民百姓中蔓延开来。许多报纸直言不讳地谴责戴维斯总统本人及他领导作战的方式。南部邦联内阁内部也存在分歧,戴维斯总统和他的将军们之间意见也不一致。货币迅速贬值。供给部队的鞋子和衣物非常紧缺,军械和药品供应就更少了。铁路也需要新的车厢以取代老旧的车厢,需要新的铁轨替换那些被北方佬的炮火炸坏的铁轨。战场上的将军们大嚷着要补充新的兵员,可新的兵员却越来越少。最糟的是,某些州的州长拒绝把自己州的民兵和武器送出自己州的州界,佐治亚的布朗州长也是其中之一。州里的队伍里有成千上万的健康男儿,前方军队想得都要疯了,可政府征用他们的请求却未能满足。
\par 随着货币再次贬值,价格又猛涨起来。牛排、猪肉和黄油都要三十五美元一磅,面粉一千四百美元一桶,苏打一百美元一磅,茶叶五百美元一磅。保暖的衣服即使有货,价格也贵得使人不敢问津。亚特兰大的太太小姐们已经在用破布做旧衣服的镶边,用报纸给衣服加厚用来挡风。一双鞋的价格从两百到八百美元不等,要看是用“纸板”做的还是用真皮做的。太太小姐们已经在穿用旧羊毛披巾和剪下来的毯子做的高帮松紧鞋,后跟是用木头做的。
\par 实际情况是,北方已经对南方进行了真正的围攻,虽然很多人还没有意识到这一点。北方佬的炮舰正在缩紧港口的封锁线,能偷偷闯过封锁线的船只已经很少很少了。
\par 南方一贯是靠出售棉花、购买它自己不能生产的东西过活的,可现在,它既没法卖也没法买。郝嘉乐在塔拉轧棉厂附近的小棚屋里已经存储了三年的棉花,可这对他没有半点好处。在利物浦,这可以带来十五万五千美元的收入,可根本没有希望把棉花弄到利物浦去。嘉乐已经从一个富有的人变成个还不知道怎样让全家及黑奴们过冬的人了。
\par 在整个南方,大多数棉花种植园主都陷入了同样的困境。随着封锁线越缩越紧,根本没有办法把南方专供出售的棉花运到英国市场,也没有办法像以往数十年中那样,把用出售棉花的钱购买的必需品运进来。以农业为主的南方和以工业为主的北方作战,现在需要的东西太多了,这些东西在和平时期是从来没有人想到要买的。
\par 这种形势下必然会出现投机商和牟取暴利的人,而且利用这种机会的人大有人在。由于食品和衣物越来越匮乏,价格又越涨越高,人们强烈反对投机商的呼声越来越高,恶意越来越盛。一八六四年初的那些日子里,一打开报纸就会看到痛斥投机商、称他们为掠夺成性的吸血鬼的社论,还号召政府要用强硬的手段镇压他们。政府作出了最大的努力,可结果却一无所获,因为困扰政府的事情太多了。
\par 人们对谁也没有像对白瑞德那样恨之入骨。偷闯封锁线渐渐变得太危险时,他卖掉了船只,现在公然做起食品投机买卖来了。从里士满和威尔明顿传回亚特兰大的有关他的事情,使那些在其他日子里接待过他的人羞悔得苦恼不安。
\par 尽管有这些痛苦和磨难,亚特兰大原有的一万人口在战争期间却翻了一番。连封锁线也使亚特兰大的威望提高了。自古以来,滨海城市在南方就一直占统治地位,商业上如此,其他方面也是如此。可是,现在港口都被封锁了,许多港口城市或被占领或被围攻,南方只能自己救自己了。如果南方最后会取得战争的胜利,重要的还是内陆地区,而亚特兰大现在成了万事的中心。和南部邦联其他地区一样,城里的人们正在遭受艰难困苦、物资匮乏、疾病和死亡带来的痛苦。可是,亚特兰大这个城市因为战争,得到的比失去的多。南部邦联的心脏——亚特兰大还在健全而有力地跳动着,铁路就是它的大动脉,运载着没完没了的人、弹药和供给。
\par  
\par 在其他日子里,思嘉对自己破烂的衣服和打着补丁的鞋子一定会感到很痛苦,可现在她却不在乎了,因为要紧的那个人不在这,看不到她。那两个月里,她很幸福,比以往任何时候都更幸福。当她双手环绕着希礼的脖颈时,她难道没有感觉到他心脏的跳动吗?难道没有看到他脸上那绝望的神情吗?这种神情比什么话都更能说明一切。他爱她。现在她敢肯定了,并且深信这一点,这使她非常快乐,甚至对媚兰更友好,她也能做到了。现在,她可得为媚兰感到难过了,媚兰既盲目又愚笨,思嘉不禁带着些微的鄙夷为她感到难过。
\par “在战争结束以后!”她想,“战争结束——然后……”
\par 有时候想着想着,有些恐惧感会刺痛着她:“那又怎么样呢?”但她把这种想法硬从脑海里赶走了。战争结束以后,不管怎样,一切都会安定下来的。如果希礼爱她,他当然不能继续和媚兰一起生活下去。
\par 可是,接下来呢,离婚是想都不能想的;埃伦和嘉乐都是虔诚的天主教徒,决不会让她和一个离过婚的男人结婚。这就意味着要离开教堂!思嘉认真思考过后决定,要在教堂和希礼之间作个选择的话,她会选择希礼。可是,噢,这样就会引起很多流言蜚语!离过婚的人不但会遭到教堂的摒弃,而且会遭到社会的摒弃。没有一个离过婚的人会受到欢迎的。然而,为了希礼,她愿意这么去冒险。为了希礼,她可以牺牲一切。
\par 不管怎样,战争结束时,一切都会好的。如果希礼这么爱她,他会找到解决办法的。她会想办法让他找到解决办法的。随着日子一天天过去,她心里越来越确信他对她的衷心,更加肯定北方佬最后被打败时,他一定会令人满意地安排好一切。当然,他说过北方佬会击败他们,但思嘉认为,那样想太愚蠢了。他这么说的时候,她不但不喜欢,而且很沮丧。但她几乎不在乎北方佬会赢还是会输。重要的是战争快点结束,希礼早点回家。
\par 三月的雨夹雪把每个人都阻在屋里时,最可怕的一击终于降临了。媚兰两眼兴奋得发亮,骄傲得不好意思地低着头,告诉思嘉她怀上孩子了。
\par “米德医生说,孩子八月底或九月份就会出世,”她说,“我已经想过——但我至今还不太肯定。噢,思嘉,这岂不是太好了?我一直忌妒你有韦德,也很想要个孩子。我曾经担心我不能有孩子,亲爱的,我真想要一打孩子!”
\par 媚兰这么说时,思嘉正在梳头准备睡觉。这时,她停了下来,梳子还举在半空。
\par “我的天!”她这么说,有一瞬间,她的意识是空白的。突然,媚兰紧闭的房门跃入她的脑海,一阵刀割般的痛苦传遍了她的全身,就像是希礼是她自己的丈夫却背叛了她所带来的那种痛苦一样。孩子。希礼的孩子。噢,他怎么能这样?他爱的是她而不是媚兰。
\par “我知道你一定会很吃惊的,”媚兰喋喋不休、上气不接下气地说下去,“这难道不是很好吗?噢,思嘉,我真不知道该怎么告诉希礼!我如果告诉他,也没什么不好意思的,或者——或者,哦,什么也不说,让他自己慢慢发现好了,你知道——”
\par “我的天哪!”思嘉说着,几乎哭了起来,头梳掉到地上去了。她手抓着大理石梳妆台的顶部,好不让自己摔倒。
\par “亲爱的,别这样!你知道的,有个孩子并不坏。你自己这么说的。你没必要为我担心的,虽然你看上去这么不开心是好心。当然,米德医生说我是——是,”媚兰脸红了,“太窄了,但是,也许我不会有什么麻烦的,而且——思嘉,你发现怀了韦德时,有没有写信告诉查理,还是说你妈妈或者也许是郝先生这么做了?噢,亲爱的,要是我也还有个妈妈这么做就好了!我只是不明白怎么——”
\par “别说了!”思嘉粗暴地说,“别说了!”
\par “噢,思嘉,我真是太笨了!对不起。我想,所有幸福的人都很自私。我忘了查理了,刚才——”
\par “别说了!”思嘉又重复了一遍,她在极力控制着自己的面部表情,使自己的心情平静下来。决不能让媚兰看出或是怀疑她是怎么想的。
\par 媚兰是最得体老练的女性了,自己的残忍行为使她眼里溢满了泪水。可怜的查理去世后几个月,韦德才出生,她怎么能勾起思嘉这些可怕的记忆呢?她怎么能这样没有头脑呢?
\par “我来帮你脱衣服吧,亲爱的,”她低声下气地说,“我来给你擦擦头。”
\par “你让我自己待着吧。”思嘉说着,脸绷得像块石头。媚兰因自责而放声大哭,逃离了房间,剩下思嘉自己一人面对床铺,一滴眼泪也流不出来,夹杂着受挫的傲气、幻想的破灭以及对伙伴的忌妒。
\par 她想,她再也无法和一个怀着希礼的孩子的女人住在同一个屋檐下了,她得回到塔拉去,回到那属于她的家中去。她不知道,自己如何才能再次面对媚兰,又不让她从她脸上看出她的秘密。第二天早上起床时,心里便有了要在早饭后马上收拾箱子回家的打算。她们坐在餐桌边,思嘉默默无语、心情郁郁,白蝶茫然无措,媚兰则可怜兮兮的,可是恰在这时,来了封电报。
\par 这是希礼的贴身卫士莫斯发给媚兰的。
\par “我到处都找过了,可还是找不到他。我要不要回家来?”
\par 没有人知道这指的是什么,可三个女人却面面相觑,惊恐得瞪大了眼睛,思嘉把回家的所有想法全忘到九霄云外去了。不等吃完早饭,她们就驾车到城里去给希礼的上校发电报,可就在她们走进电报局时,他的电报倒先来了。
\par “很抱歉地通知你,自三天前希礼少校去执行侦察任务以后,他便失踪了。我们会继续和你联系。”
\par 这一路回家真是一次可怕的旅程。白蝶姑妈脸埋在手帕里号啕大哭,媚兰直挺挺地坐着,脸色苍白,思嘉萎靡不振的,缩在马车的角落里不知所措。一回到家里,思嘉跌跌撞撞地走上楼梯,来到自己的卧室,从桌上一把抓起玫瑰经文,跪在地上想祈祷。可是祈祷词却说不出来。她只感到一种深不可测的恐惧感,知道上帝因为她罪孽深重已经不垂青于她了。她爱上了一个已经结婚的人,还想从他妻子那里把他夺过来,上帝就把他杀了,用以惩罚她。她很想祈祷,但她无法抬起头来让眼睛面对着上帝。她很想哭,但却欲哭无泪。眼泪似乎填满了她的胸腔,它们热得滚烫,在她的胸部燃烧着,可就是流不出来。
\par 房门开了,媚兰走了进来。她的脸就像是个从白纸上剪下来的心型似的,边沿嵌着黑色的头发,两眼瞪得大大的,就像个迷失在黑暗中的惊恐万状的孩子。
\par “思嘉,”她说着伸出了双手,“我昨天说了那些话,你得原谅我,因为你是——现在是我的一切了。噢,思嘉,我知道,我所爱的人已经死了!”
\par 不知怎的,她便扑在思嘉怀里了。她小小的乳房哭得一起一伏的,也不知怎么的,她们都躺到了床上,互相紧抱着。思嘉也在哭,哭得脸紧挨着媚兰的,双方的眼泪都润湿了对方的面颊。哭泣确实伤人伤得很厉害,但还是比哭不出来要好得多。“希礼死了——死了,”她这么想着,“我因爱他却害了他!”她再次悲从中来,从她的眼泪中得到安慰的媚兰则用双臂搂紧了她的脖子。
\par “至少,”她自言自语地说,“至少——我有了这个孩子。”
\par “我呢,”思嘉想着,由于受的打击太大,已无法顾及像忌妒这样的小事了,“我什么也没有了——什么也没有——什么也没有,只有他跟我说再见时脸上的那种神情。”
\par  
\par 最先的报告是“失踪——认为阵亡”,伤亡名单上也是这样说的。媚兰给斯隆上校发了一打电报,终于收到了一封信,信里充满了同情,解释说希礼和一个班的人骑马出去执行侦察任务,结果没有回来。有报告说,北方佬的阵线内有过小规模的作战。莫斯悲痛得都快要疯了,他冒着生命危险去搜寻希礼的尸体,可是什么也没找到。媚兰现在平静得出奇,把钱电汇给他,叫他回家来。
\par 当伤亡名单上出现“失踪——认为被捕”的字样时,伤心欲绝的家里又重新燃起了快乐和希望。媚兰天天都到电报局,几乎拉都拉不走。她去接每辆火车,希望会收到来信。她现在已经恶心想吐了,但她拒绝服从米德医生的命令,没有卧床休息。她的精力极度旺盛,不让自己平静下来;晚上,思嘉早已上床之后,还能听到隔壁房间里她走路的声音。一天下午,她从城里回家来,赶车送她回来的是惊恐万状的彼德大叔,扶着她的是白瑞德。她在电报局晕过去了。瑞德正好经过,看到那里一阵骚动,便护送她回家来。他把她抱上楼梯,送到卧室里。当惊恐万状的屋里人东跑西跑找热砖、毯子和威士忌时,他把她放到床上,让她躺在枕头上。
\par “卫太太,”他突然问道,“你怀孕了,对不对?”
\par 要是媚兰不是这么虚弱、这么难受、这么悲伤的话,听到他的问话,她一定会崩溃的。即使和女性朋友在一起,一有人提到她的状况,她也会窘迫不堪,而去米德医生那就诊也是痛苦的经历。而一个男人,特别是像白瑞德这样的男人,居然问这种问题,简直连想都没人敢想。可是,她虚弱而凄凉地躺在床上,于是只好点点头。她点过头之后,似乎就没这么可怕了,因为他看上去是这么善良,又这么关心她。
\par “那你就得更好地照顾自己了。你到处乱跑、担心忧虑,这对你没有半点好处,也许反而会伤了孩子。如果得到你的允许的话,卫太太,我将动用我在华盛顿所有的关系去打听卫先生的下落。如果他当了俘虏,他的名字就会在北部联邦的名单上,如果不在——哦,那没有什么比不能确知更糟的了。可你得先向我保证,好好照顾自己,否则的话,上帝在上,我一点也不愿插手的。”
\par “噢,你真是太好了,”媚兰哭了起来,“人们怎么能说你那么多可怕的事?”接着,她马上意识到自己的得体与老练,也害怕自己居然和一个男人谈论自己的情况,于是无力地哭开了。思嘉手里拿着一块用法兰绒包着的热砖飞奔上楼,看到瑞德正在拍她的手。
\par 他果真守信。她们绝不会知道他动用了哪些关系。她们也不敢问,知道一问可能会要使他承认和北方佬有过分密切的交往。得到消息已经是一个月以后的事了。她们一听到这消息,一下就升到了快乐的顶峰。可后来,心里却又被令人痛苦的担心忧虑占据了。
\par 希礼没有死!他受了伤,当了俘虏。记录表明,他在罗克艾兰,在伊利诺伊州的一个战俘营。他们最初大喜过望,只想到他还活着。可是开始平静下来以后,她们面面相觑,异口同声地说出“罗克艾兰”这个词,就好像本来是要说“在地狱里!”一样。就像安德森维尔在北方是个臭名远扬的地名一样,罗克艾兰也是个会给任何有亲属关押在那里的南方人带来恐怖的地方。
\par 林肯拒绝交换俘虏,认为这么做可以加重南部邦联的负担,因为他们得给北部联邦的俘虏吃饭,还得看管他们,这样便可以促使战争早日结束。佐治亚的安德森维尔已经有成千上万穿蓝色军服的人。南部邦联的人缺乏配给,实际上连自己的伤病员都没有药和绷带用。他们就没什么东西可以和俘虏们一起共享了。他们给俘虏们吃的是前线的士兵们吃的东西,肥猪肉和干豌豆,这种食谱使得北方佬像苍蝇一样纷纷死去,有时一天就死一百人。北方军被这类报告激得火冒三丈,也越发苛刻地对待南部邦联的俘虏。而条件最糟的就是罗克艾兰了。食物匮乏,三个人合用一条毯子,天花、肺炎和伤寒大肆流行,给这个地方赢得了传染病院的称号。有四分之三的人是活着进去却再也没有出来的。
\par 希礼就在那个鬼地方!希礼是还活着,可他受伤了,被送到罗克艾兰。他被押送到那儿时,伊利诺伊州的雪一定也已经下得很厚了。白瑞德打听到他的消息以后,他是不是因为伤痛而死去了呢?他是不是也成了天花的牺牲品?他是不是也得了肺炎却连盖的毯子都没有呢?
\par “噢,白船长,是不是有什么办法——你不能动用你的关系让他和别人交换回来吗?”媚兰哭着说。
\par “为比克斯比太太的五个男孩,那个宽厚仁慈、行为公正的林肯先生可以哭得悲痛欲绝,但对在安德森维尔正在死去的成千上万的北方佬,他却一滴眼泪也不洒,”瑞德说着,嘴角又翘了起来。“即使他们全死了,他也根本不在乎。命令已经发出去了,不能交换战俘。我……我过去没告诉你,卫太太,你丈夫本来有个机会出来的,可他拒绝了。”
\par “噢,不!”媚兰不相信地叫了起来。
\par “是的,这是真的。北方佬在征兵去打印第安人,从南部邦联的俘虏中征兵。每个宣誓要忠诚的俘虏可以入伍两年去打印第安人,然后就会被释放,送到西部去,卫先生拒绝了。”
\par “噢,他怎么能拒绝呢?”思嘉叫了起来,“他干吗不宣誓,然后一离开监狱就逃回家来。”
\par 媚兰像个复仇小女神似的转身面对着她。
\par “亏你想得出来,他会做这种事?先宣那卑鄙的誓,背叛南部邦联,然后再背叛对北方佬的诺言!我宁愿听到他死在罗克艾兰,而不愿听到他宣那种誓。他若死在监狱里,我倒觉得很自豪。可是,他要做那种事,我就再也不见他了,永远不见!他当然会拒绝的。”
\par 思嘉送瑞德到门口时,愤愤不平地问道:“要是你的话,你难道不加入北方佬的部队,然后再逃走,免得死在那个地方吗?”
\par “当然会的。”瑞德说。露出了髭须下面的牙齿。
\par “那希礼干吗不这么做呢?”
\par “他是个绅士。”瑞德说。思嘉茫然不解的,这个高尚的词怎么可能传递出玩世不恭和鄙夷的意味呢?

\subsection{第三部}


\subsubsection{第十七章}

\par 一八六四年五月到了——这个五月炎热、干燥,鲜花刚结出花蕾,就已经枯萎了——舍曼将军率领下的北方军再次进军佐治亚,开到了离亚特兰大西北部一百英里远的多尔顿。有传言说,佐治亚和田纳西的州界附近会发生恶战。北方军正集结部队,要对西部和亚特兰大的铁路线发起攻势。这条铁路线是连接亚特兰大和田纳西及西部的干线。去年秋天,就是这条干线运载南方部队开到前线,取得了奇克莫加战役的胜利。
\par 然而,总的说来,虽然多尔顿附近要开战,但亚特兰大并没有受到干扰。北方佬就集中在奇克莫加战役的战场东南部几英里远处。他们过去曾经试图穿过那一地区的山口关隘,但被赶了回去。他们还会被赶跑的。
\par 亚特兰大——及至整个佐治亚州——明白,这个州对南部邦联来说太重要了,乔·约翰斯顿将军不能让北方军在州界内待太长时间。乔老将军和他的部队不会让北方佬到多尔顿以南的地方,一个也不会放他们过来。因为太多事情都要靠佐治亚来运作,而现在它还未受到太多的干扰。这个未遭蹂躏的州是南部邦联的大粮仓、军工车间和仓库。部队所需的大部分火药和武器以及大多数棉制品和毛制品都是这里生产的。亚特兰大和多尔顿之间是罗马,一个有炮厂和其他产业的城市,埃托瓦和阿拉图纳则有里士满以南最大的铁制品基地。亚特兰大不但有制造手枪和马鞍、帐篷和弹药的工厂,而且有南方规模最大的轧钢厂、主要铁路站及大医院。亚特兰大还是南部邦联赖以生存的四条铁路干线的交汇处。
\par 所以,并没有人为此特别担忧。多尔顿毕竟离此很远,在接近田纳西州界的地方。田纳西已经打了三年仗,人们已经习惯,总认为那个州是个遥远的战场,几乎和弗吉尼亚和密西西比河一样远。再说,乔老将军和他的部队又挡在北方佬和亚特兰大之间,而大家都知道,除李将军外,再也没有哪位将军比约翰斯顿将军更棒的了,因为石墙杰克逊已经离开人世。
\par 五月一个暖意袭人的晚上,在白蝶姑妈家的游廊上,米德医生把普通民众对这一问题的看法作了个总结。他说,亚特兰大根本没什么好怕的,因为约翰斯顿将军正像铜墙铁壁一样坚守在山上呢。大家听着他说话内心感觉却各不相同,在越来越浓的暮色中,人们静静地躺在摇椅里摇动着,看着本季节第一批萤火虫在黄昏中飞来飞去,觉得颇为不可思议。大家都心事重重的。米德太太把手放在菲尔的手臂上,希望医生说的话会是真的。她知道,如果战事更紧的话,菲尔就非得去参战了。他已年满十六,参加了城卫队。自葛底斯堡战役后,范妮·埃尔辛一直都是脸色苍白、两眼凹陷的,几个月以来,那幅折磨人的画面已经在她那业已疲惫不堪的心里刻上了深深的印痕——部队撤往马里兰时,在一次艰难的冒雨长途跋涉中,达拉斯·麦克卢尔死在一辆颠簸不停的牛车上。现在,她正试图摆脱这一痛苦的画面。
\par 凯里·阿什伯恩上尉伤残的手臂又发痛了,何况一想到他追求思嘉的举动毫无进展,他就更是万分沮丧。自从知道卫希礼被捕的消息后,他就陷入了这种境地,虽然他还不知道这两件事之间有什么联系。思嘉和媚兰都在思念希礼,像往常一样,除非有紧急任务或者一直跟她们谈话才会使她们分心,要不她们每时每刻都在思念他。思嘉想他想得很苦,也很伤心:“他一定是已经死了,要不我们早就应该有他的消息了。”媚兰则时刻都在反复地克服着恐惧心理,不断对自己说:“他不可能死的。要不我会知道的——要是他死了,我会感觉到的。”夜幕中,白瑞德半倚半靠地站着,他穿着做工精致的靴子,双腿随意交叉着,黑黝黝的脸上一副茫然的神色,谁也看不出来那是什么样的表情。韦德在他怀里熟睡着,一副心满意足的样子,手里抓着一个很干净的如愿骨\footnote{鸟胸的叉骨。西方迷信说两人同扯此骨时,扯到长的一段的人可以有求必应。}。瑞德来访时,思嘉总是让韦德很迟才去睡觉,因为这个腼腆的孩子很喜欢他。奇怪的是,瑞德似乎也喜欢韦德。通常,孩子在场时,思嘉总感到很烦,但韦德在瑞德面前表现却非常出色。至于白蝶姑妈呢,她心神不定的试图止住打嗝,因为他们晚餐时吃的是只咬不动的老公鸡。
\par 那天早晨,白蝶姑妈作了个决定,可这决定后来却让她后悔不迭。那就是,她最好在这老鸡王老死以前把它杀了,免得它思念它那些老早以前就已经被吃掉的鸡眷们。它一连好几天垂头丧气地徘徊在空空如也的鸡窝旁,萎靡不振的,也不打鸣了。彼德大叔扭住它的脖子后,一想到只有自己一家人独自享用,白蝶姑妈便良心不安了。由于她的许多朋友都有好几个星期没尝过鸡肉的滋味了,所以,她建议伙伴们到她家来吃晚饭。媚兰怀孕已有五个月,也有好几个星期没有在公共场合露面或是接待客人了。听到这种想法,把她给吓坏了。可是这一次,白蝶姑妈非常坚决。自己一家人独自享用这只公鸡,那太自私了。如果媚兰把裙环往上移一些,那谁也不会注意到什么,不管怎么说,她的胸部不是也很平嘛。
\par “噢,姑妈,可我不想见人,希礼他——”
\par “希礼他并没有——走,”白蝶姑妈说着,声音都发抖了,因为在她心里,她已经很确定希礼已经死了。“他跟你一样还活蹦乱跳的活在世上,有人做伴对你也有好处。我还要请范妮·埃尔辛。埃尔辛太太曾经求我想办法调动她的情绪,让她见见别人——”
\par “噢,姑妈,可是这么逼她太残忍了,可怜的达拉斯死了才——”
\par “好了,梅利,你要是和我争辩的话,我会苦恼得哭出来的。我想,我是你姑妈,我知道分寸的。我要开一次宴会。”
\par 这样,白蝶姑妈便开了个宴会。马上要开宴时,却来了个她意想不到、也不愿见到的客人。就在满屋子飘荡着烤鸡的香味时,刚刚结束一次神秘旅行的白瑞德敲响了门。他腋下夹着一大盒用纸花边包着的夹心糖,已准备好满口语义相关的奉承话要对她讲。虽然白蝶姑妈知道医生和米德太太对此人的看法,而范妮对没参军的人又会多么反感,但也没有别的法子,只好邀他入席。在街上米德夫妇和埃尔辛一家是不会和他说话的,但在朋友家里,他们当然也会对他以礼相待。再说,柔弱的媚兰对他的保护比以往任何时候都更坚定,自从他帮她打听希礼的消息后,她已公开宣称,只要他还活在人世,不管别人怎么说他,她的家门都永远对他敞开。
\par 看到瑞德举止颇为得体,白蝶姑妈心里的石头才落了地。他一直和范妮说话,既同情她又尊重她,甚至令她对他露出了笑脸,晚宴则非常顺利。这真是一次王公盛宴。凯里·阿什伯恩带来了一点茶叶,这是他在去安德森维尔的路上从一个被捕的北方军的烟袋里找到的。于是,每个人都喝了一杯散发着淡淡的烟草味的茶,还分了一点很难咬烂的老鸡肉,用玉米粉加洋葱做的足量的调味品,一碗干豌豆,不少米饭和肉卤。肉卤有点湿糊糊的,因为没有面粉,无法把它拌得稠一些。甜点有甜苹果派,再就是瑞德带来的夹心糖。先生们喝黑莓酒时,瑞德拿出正宗的哈瓦那雪茄跟他们共享。这时,所有人都承认这确是一次盛宴。
\par 女士们都在屋前的游廊上,先生们过来加入了她们的行列,话题又回到战争上来。现在,话题总会转到战争上来,所有话题都从战争谈开,或者最后回到战争——有时令人伤感,大多数时候却令人高兴,但总是离不开战争。战争罗曼史、战争期间的婚礼、医院或是阵亡、露营、作战和行军中发生的事、勇猛、懦弱、幽默、伤心、损失和希望。希望总是有的,总是有的。虽然入夏以前有过许多失利,但对胜利的信心还是坚定而不可动摇。
\par 阿什伯恩上尉宣布说,他已经申请从亚特兰大转到多尔顿的部队去,并且已经获得批准。女士们用眼神默默地亲吻着他那僵硬的胳膊,极力掩饰着骄傲之情,声称他不能去。如果他走了,那谁来跟她们相处呢?
\par 听到这些话,年轻的凯里一脸困惑,但也非常高兴。说这些话的人中,有米德太太、媚兰、白蝶姑妈和范妮这样的已婚妇女和老处女,他倒很希望思嘉说的是心里话。
\par “哦,他很快就会回来的,”医生说着,一只手搂着凯里的肩膀。“只要打一场小仗,北方佬就会连滚带爬地逃回田纳西去。到那以后,福里斯特将军会关照他们的。诸位女士们没有必要因为北方佬靠近了我们而惊慌失措,约翰斯顿将军和他的部队正像铜墙铁壁一样立在山上等着他们呢。是的,铜墙铁壁。”他又重复了一次以示强调,“舍曼绝对无法过来。他永远也无法把乔老将军赶走。”
\par 女士们笑着表示同意,因为他话说得如此轻松,当是不容置疑的真理。男人毕竟比女人更懂这些事,如果他说约翰斯顿将军是铜墙铁壁,那他就一定是铜墙铁壁。现在只有瑞德一个人在说话了。吃完晚饭到现在,他一直没有吭声,嘴角撇着坐在夜幕中,听着有关战争的谈话,手里抱着熟睡的孩子,让他靠在自己肩上。
\par “我相信,有传闻说舍曼的援军已经到来了,他现在有十万大军?”
\par 医生的回答很简短。一踏进这个家门,医生就发现餐桌上有一个是他打心里不喜欢的人。自那以后,他就一直很紧张。碍于他对白蝶小姐的尊敬,自己又是她家的客人,他这才克制住自己,不让自己真正的感觉表露得太明显。
\par “你说什么,先生?”医生大声反问道。
\par “我相信,阿什伯恩上尉刚刚说过,约翰斯顿将军大约只有四万军队,连受上次胜利的鼓舞又回到连队去的逃兵也算在内了。”
\par “先生,”米德太太气愤地说,“南部邦联的军队里是没有逃兵的。”
\par “请原谅,”瑞德带着嘲弄意味地说,“我指的是那几千在休假却忘了回到连队中去的人,还有那些养了六个月的伤却还留在家里,像过去一样做事或者正在春耕的人。”
\par 他的两眼炯炯有神,米德太太却怒气冲冲地咬着嘴唇。她遭到这样的反驳,思嘉真想笑出声来,显然瑞德触到了她的痛处。成千上万人逃避职责,躲在沼泽地里和山里,纠察队也无法把他们拖回部队去。他们声称,这是“富人的战争,穷人的战役”,他们已经受够了。但是,数量更多的还是这样的一些人。他们的姓名还留在开小差的名册上,但又不想永远离开部队。他们白白等了三年想得到休假,收到的却是家里的坏消息:“我们在挨饿。”“今年不会有好收成了——这里没有人种庄稼。我们在挨饿。”“军需部连小猪都拿走了。我们已经好几个月没有收到你们的钱。我们正在靠干豌豆过活。”
\par 合唱曲的声音总是越来越嘹亮:“我们在挨饿,你的妻子、孩子和父母。什么时候才有个尽头呢?你什么时候回家?我们在挨饿,在挨饿。”由于部队人数在迅速减少,他们的休假申请被否决了,于是,这些士兵没有得到允许便擅自回家,去耕种田地、种植庄稼、修补房屋、修建围栏。部队军官很清楚形势,看到不久便有一场恶战,他们写信给这些士兵,要求他们归队,什么问题都好办。通常情况下,士兵们若看到家里还能支持几个月才会挨饿,便会返回连队。“耕种假”不会被看成是面对敌人时的临阵脱逃,但同样削弱了部队的战斗力。
\par 这一瞬间着实令人难堪,米德医生赶紧开口说话。他的声音很冷淡:“白船长,我们的部队和北方佬的军队之间那些数也数不清的差异并不重要。一个南部邦联的士兵比得上一打北方军。”
\par 女士们都点点头。大家都明白这一点。
\par “战争刚开始时是这样,”瑞德说,“如果南部邦联的战士们有子弹上枪膛、脚上有鞋穿、胃里有食物的话,现在兴许也还是这样。哎,阿什伯恩上尉,你说对不对呀?”
\par 他的声音还是很轻柔,有种特别谦卑的意味。凯里·阿什伯恩看上去很不高兴,因为他也特别讨厌瑞德。如果可能,他本很乐于站在医生这一边,可他不能撒谎。他虽然手臂残废了,但还申请转到前线去,那是他意识到形势严峻,这是普通百姓还没有意识到的。还有很多人,有的拄着木制拐杖跌跌撞撞地走路,有的只剩下一只眼睛,有的手指已不见踪影,有的已经失去一只胳膊,但他们都在悄悄地从军需部、医院、邮政和铁路系统转回原来的战斗团队去。他们知道,乔老将军需要每一个人手。
\par 他没有说话。米德医生却大发雷霆了,他大吼道:“过去,我们的士兵没有鞋穿、没有饭吃,却打了胜仗。他们还将继续参战,并且获得胜利!我告诉你,约翰斯顿将军是动不了的!自古以来,山上的要塞从来就是被侵略民族的避难所和坚固的堡垒。想想——想想瑟莫比利!”\footnote{古希腊战场之一。公元前480年,波斯人在此击败了斯巴达人的一支军队。}
\par 思嘉绞尽脑汁思索着,但瑟莫比利对她来说什么意义也没有。
\par “他们在瑟莫比利一直打到最后一个人也战死为止,对不,医生?”瑞德问道。他嘴角抽动着,强忍住笑。
\par “你是不是故意在侮辱人,年轻人?”
\par “医生!请别这样!你误解我了!我只是在问一些情况。古代历史我记不太清了。”
\par “如果需要的话,我们的部队也会战死到最后一个人,不然北方佬别想向佐治亚内陆挺进,”医生严厉地说,“但情况绝不会是这样。只要一场小规模的仗,他们就会把北方佬赶出佐治亚。”
\par 白蝶姑妈赶紧站起来,叫思嘉弹琴唱歌给他们听。她看得出来,这么谈下去,马上就会出现大吵特吵的场面。她也很清楚,只要她邀请瑞德吃晚饭,就一定会有麻烦。但他在场的时候总少不了有麻烦。但他到底是怎么惹的麻烦,她从来就弄不清楚。天哪!天哪!思嘉到底看中了他什么呢?亲爱的梅利又怎么能这么护着他?
\par 思嘉顺从地走进客厅,游廊上顿时鸦雀无声,这沉寂中充斥着对瑞德的怨恨。居然有人不完全地相信约翰斯顿将军和他的部队会战无不胜,这怎么可能呢?相信也是神圣的职责,那些不忠之人即便不相信的话,至少也得闭嘴不言吧。
\par 思嘉弹了几小节和弦,她的歌声从客厅里直飘到他们的耳际。歌声甜美、忧伤,唱的是一首流行歌曲:
\refdocument{
    \par “在刷得雪白的病室里
    \par 躺着已经死去和行将死去的人——
    \par 刺刀、弹片和子弹伤——
    \par 有个人的心上人全都遇上。
    \par  
    \par 有个人的心上人如此年轻,如此勇敢!
    \par 他苍白、可爱的脸上还残留着——
    \par 孩提时代优雅举止的踪迹——
    \par 可很快又要被墓穴的尘土掩去。”
}
\par “金色的鬈发潮湿地缠结在一起。”思嘉的女高音并非无可挑剔,她继续哀唱着。范妮欠了欠身,用微弱、哽咽的声音说:“唱点别的吧!”
\par 思嘉又惊又窘,琴声戛然而止。接着,她又手忙脚乱地弹起了《灰色的上衣》的开头几小节,却又很不自然地停了下来。她想起来了,那首歌同样使人肝肠欲断。琴声再次停了下来,因为她真的是不知所措了,所有的歌都涉及死亡、分离和悲伤。
\par 瑞德迅速站起身来,把韦德放在范妮的腿上,走进了客厅。
\par “弹《我的老家肯塔基》。”他平静地建议道,思嘉很感激他,马上弹了起来。瑞德出色的男低音和她一块唱了起来。他们唱到第二段时,游廊上的人们才松了口气,可只有老天才知道,这根本不是什么欢快的歌曲。
\refdocument{
    \par “再背几天这沉重的包袱!
    \par 尽管这包袱永远不会变轻!
    \par 只要再过几天,我们便可以在路上蹒跚前行!
    \par 那时,我的肯塔基老家,再对你道声晚安!”
}
\par 米德医生的预测没有错——至少到目前为止是这样。约翰斯顿确实像座铜墙铁壁一样屹立在一百英里外多尔顿的山峦间。他屹立在那里,稳如泰山,和想穿过山谷、进军亚特兰大的舍曼进行着艰苦卓绝的斗争。最后,北方军只好退了回去,另做打算。由于无法通过正面进攻突破南部邦联的防线,所以,他们只能在夜幕笼罩下采取迂回战术,绕过山上的关隘,希望能进攻约翰斯顿的后部,在离多尔顿十五英里远的里萨卡切断他背后的铁路线。
\par 那两条宝贵的铁路犹如两个孪生兄弟,既然它们已身临险境,南方军便离开了他们死守的散兵壕,在星光映照下,抄近路急行军到里萨卡。当北方军从山峦间蜂拥而出,与南方军遭遇时,南方军已经严阵以待。他们掩藏在胸墙后面,准备好炮火,上好的刺刀熠熠生辉,准备工作做得很充分,足以和在多尔顿时相比。
\par 多尔顿的伤员断断续续带来消息,说乔老将军已经撤往里萨卡。听到这消息,亚特兰大人全都震惊了,心里隐隐地感到不安,就像是西北部的天空中飘着一小片乌云似的,这是夏天雷雨到来时最先出现的云朵。将军到底是怎么想的,居然让北方佬往佐治亚腹地又前进了十八英里?山峰是天然屏障,连米德医生也这么说过。乔老将军为什么不把北方佬阻在那儿呢?
\par 约翰斯顿在里萨卡拼死奋战,再次击退了北方军。但是舍曼采用了同样的侧面进攻战术,指挥大军从另一侧渡过乌斯塔诺拉河,再次进攻南方军后部的铁路线。南方军又从红色的壕沟里被火速召回来保护铁路。他们困乏不堪,行军和打仗使他们精疲力竭,折磨他们的还有一直填不饱的肚子,但他们还是又来了一次急行军,朝山谷进发。他们抵达离里萨卡六英里远的小镇卡尔洪,赶在北方佬前边建好了掩体,再次严阵以待,等北方军一到,就给他们来个迎头痛击。进攻开始了,打了些小规模的硬仗,北方军再次被击退。疲惫不堪的南方军躺倒在武器上,祈祷着能暂时缓一缓,休息休息。可是,这根本不可能。舍曼不屈不挠,一步步前进,指挥大军又来了个更大的迂回包抄,南方军不得不要再次撤到后方去保护铁路。
\par 南方军在半睡半醒的状态中前进,绝大多数都累得脑子也没法转了。然而,只要他们还有思想,他们就信任乔老将军。他们知道自己是在撤退,但同样也明白他们还没有被打败。他们只是没有足够的兵力坚守防御工事,也无法击败舍曼的侧面进攻。只要北方佬停下来和他们交战,他们就可以打败北方佬,而且也一定能打败北方佬。这次撤退的结果会怎么样,他们也不知道。但是,乔老将军知道自己在做什么,这对他们来说已经足够了。他指挥撤退的方式真是漂亮极了,因为他们没损失什么兵力,而北方军战死和被捕的数量却不计其数。他们没有损失一辆火车,只损失了四门大炮。后方的铁路也安然无恙。舍曼所有的正面进攻、骑兵突袭和侧翼进攻都没有伤到铁路一根毫毛。
\par 铁路,那条迂回穿行在阳光灿烂的山谷间、通往亚特兰大的细长的铁路还是他们的。士兵们在看得见铁轨的地方躺下睡觉,铁路则在星光下闪着微光。而士兵们躺倒魂归西天时,他们茫然的眼睛最后看到的也是在无情的烈炎下闪闪发亮的铁轨,光亮中还散发出热量。
\par 他们退回山谷时,一大批难民比他们还先来一步。种植园主和穷苦白人,富人和穷人,黑人和白人,妇女和儿童,年老的、瘸腿的、受伤的、早已有孕在身的,挤满了通往亚特兰大的通道。坐火车的、步行的、骑马的、坐在箱子和家庭用品堆得高高的马车上的,比比皆是。撤退的大军前面五英里远处便是这些难民。他们在里萨卡、卡尔洪和金斯顿都稍作停留,在每个地方都希望能听到北方佬已被打退的消息,他们好回家去。可那艳阳高照的路上就是没有返回的人流。南方军经过之处尽是空荡荡的房子、废弃的农场、门户半启的孤零零的小屋。各处可见一些独自留守家园的妇女和惊恐万状的黑奴。他们来到路边,为战士们欢呼,拎着一桶桶井水给焦渴的士兵们解渴,为伤兵们包扎伤口,还把死去的士兵埋在自家的墓地里。但大体上说,整个阳光灿烂的山谷已经被弃置不用、一片荒凉,只有没人伺弄的庄稼孑然挺立在焦干的田地中。
\par 约翰斯顿在卡尔洪又受到迂回攻击,只好回到阿代尔斯维尔。这里发生了激烈的遭遇战,然后又到卡斯维尔,再到卡特斯维尔。而敌人此时已经从多尔顿又前进了五十五英里。再往后十五英里的纽霍普教堂,南方军在此挖壕固守,决心站稳脚跟。北方军的战线开了过来,一点也不放松,就像一条大蛇盘绕着身子,恶狠狠地进攻着。虽然它受了伤会往后退,但总是会发起新的攻势。双方在纽霍普教堂决一死战,连续打了十一天,北方佬的每次进攻都被南方军以鲜血为代价打退了。南军再次受到迂回攻击的约翰斯顿,只得把战斗力越来越弱的战线又往后退了几英里。
\par 在纽霍普教堂,南方军的死伤不计其数。伤员整火车整火车地拥入亚特兰大,整个城市都惊呆了。即使奇克莫加战役之后,这个城市也从没见过这么多的伤员。医院人满为患,伤员只好躺在空闲商店的地上及仓库里的棉花包上。每家旅馆、寄宿处和私宅都挤满了不幸的伤员。白蝶姑妈也分到了几个,她曾提出抗议,说媚兰目前的情况比较难办,房子里有陌生男人,那是极不合适的。看到可怕的场面,她可能会早产。可她的抗议等于白搭。媚兰把她的裙环往上提了一点点,掩饰一下她越来越大的肚子,伤员便进驻这所砖房了。没完没了的烧煮、搀扶、帮助翻身、给伤员扇扇子,连续不断的清洗、卷绷带和捡棉绒。多少个温暖的夜晚,隔壁房间传来喋喋不休的胡话,使人整个晚上彻夜难眠。最后,整个城市被塞得满满的,再也无法照顾更多的伤员了,过剩的伤员只好被送到梅肯和奥古斯塔的医院去。
\par 这股伤员的大回流带来了互相矛盾的消息,惊恐万分的难民又越来越多地拥入已经很拥挤的亚特兰大,整个城市一片嘈杂。天边那一小片云朵迅速变成一大片阴沉的暴风云,从中好像还隐隐约约刮出了一股凉风。
\par 谁也没有对部队的战无不胜失去信心,但每个人,至少是普通老百姓,都已经对将军失去信心了。纽霍普教堂离亚特兰大只有三十五英里!仅仅三个星期,将军就已经让北方佬把他往后推了六十五英里!他干嘛不阻住北方佬,却一而再、再而三地撤退呢?他真是个傻瓜,而且比傻瓜还更傻。城卫队身在亚特兰大,一点危险也没有。队里的白胡子老人和州里的民兵队员们坚持说,连他们也可以把这一仗打得更漂亮,还在白色的台布上画出地图,证明自己的论点。由于战线越发疏松,将军又被迫一直后撤,他便拼命向布朗州长要求,要这些队员也去参战,可是,这些州属部队总是觉得他们很安全,这么想其实也非常合乎情理。杰夫·戴维斯也曾要求过要调用这些人马,可州长毕竟还是拒绝了。他现在干嘛要答应约翰斯顿将军呢?
\par 战斗然后撤退!再打,再撤退!在过去的二十五天里,在已经退出的七十英里土地上,南方军几乎每天都在战斗。纽霍普教堂现在已经被穿灰色军服的南方军远远抛在后面,对那里的记忆掺杂着一系列模模糊糊的记忆,闷热难挡、尘土飞扬、饥饿难忍、疲惫不堪、在车辙道道的红土路上跋涉、在红色的泥泞中践踏、撤退、挖沟、作战——撤退、挖沟、作战。纽霍普是一场梦魔,那是属于另外一种生活的,比格尚蒂也是这样,他们在此掉转方向,像守护神似的和北方佬开战。可是,虽然战场上海蓝蓝的一片,全是战死的北方军,但北方佬却源源不断,全是新入伍的士兵。东南方那条蓝色战线正向南方军后部、向铁路——向亚特兰大包抄,那条邪恶不幸的弧线总是还存在!
\par 从比格尚蒂,筋疲力尽、缺乏睡眠的南方军继续沿着通往肯纳索山的道路撤退。这里离小镇玛丽埃塔很近,他们在这里布下了一条十英里长的弧形战线。在山两边陡峭的山麓上,他们挖好了散兵壕,在高高的山峰上也布好了炮兵。因为骡子无法爬坡,士兵们一边谩骂不停、一边挥汗如雨,把重型武器沿着险峻的山坡拖上山去,到亚特兰大的信使和伤员一再给惊恐万分的城里人带来消息,要他们放心。肯纳索的山峰是坚不可摧的。附近的派恩山和洛斯特山也一样,也都修筑了防御工事。北方佬动不了乔老将军的人马,他们现在也很难再采取迂回战术了,因为山顶上的大炮控制了方圆几英里的路口。亚特兰大的呼吸轻松了,但是——
\par 但是,肯纳索山仅仅在二十二英里以外!
\par 肯纳索山来的第一个伤员到的那天,清早七点钟,梅里韦瑟太太的马车就停在了白蝶姑妈的门外,这在过去是从来没有听说过的。黑人利瓦伊大叔捎来口信,思嘉必须马上穿好衣服到医院去。范妮·埃尔辛和邦内尔家的姑娘们坐在车后座上,她们也是一大早就从酣睡中被叫醒的,此时正连连打着哈欠。埃尔辛家的嬷嬷气鼓鼓地坐在驾驶座上,腿上放着一篮刚洗过的绷带。思嘉心里老大不乐意,因为前一天城卫队举办了一场晚会,她跳舞跳了个通宵,双脚一点力气也没有。她暗暗诅咒效率很高、不知疲倦的梅里韦瑟太太、诅咒伤员和整个南部邦联,此时,她穿上那件最旧的印花上衣,普里西正在给她扣扣子。这件衣服是她专门穿去医院做护理工作的。她三口两口吞下代替咖啡的用烤玉米和地瓜干做的苦饮料,出去加入了姑娘们的行列。
\par 这些护理工作真是让她烦透了。就在这一天,她告诉梅里韦瑟太太,埃伦已经给她来信,让她回家小住几天。可这下她就有好果子吃啦,那个令人尊敬的老太太袖子卷得老高,粗壮的身体围着一块大围裙,严厉地看了她一眼,说:“别让我再听到这种蠢话了,思嘉。我今天就给你妈妈写信,告诉她我们有多需要你,我相信她能理解,会让你留下来的。好了,系上围裙,跑到米德医生那里去。他需要人帮他给伤员敷药。”
\par “噢,上帝,”思嘉闷闷不乐地想,“麻烦就在这。妈妈会要我留下来的,我要是非得再闻这些恶臭味,那我就只有死路一条了!我真希望自己也是个老太太,这样我就可以欺负年轻姑娘,而不是被人欺负了——而且还能叫像梅里韦瑟这样的老猫见鬼去!”
\par 是的,她对医院简直是讨厌极了,恶臭的气味、虱子、伤痛、没洗澡的身体。若说护理工作有什么新鲜感和浪漫情调的话,一年前也已经消失殆尽了。再说,撤退中受伤的这些人并不像过去的伤员那么吸引人。他们对她根本没有兴趣,也没什么话说,只会说:“仗打得怎么样了?乔老将军现在在做些什么?这个乔老将军真是了不起的聪明人。”她觉得乔老将军根本不是什么了不起的聪明人。他就只会让北方佬开进佐治亚腹地八十八英里远处。不,他们一点魅力也没有。再说,他们中许多人离死神已经很近,死起来很快,悄没声息的,没剩下什么力气和血中毒、坏疽、伤寒和肺炎作斗争,而这些病在他们抵达亚特兰大、找到医生前早就已经患上了。
\par 这天天气很热,成群结队的苍蝇从敞开的窗户飞进来。疼痛没有摧毁这些士兵们的意志,这些又肥又懒的苍蝇却做到了。一股股臭味和一阵阵痛苦在她周围此起彼伏。她端着一个脸盆跟着米德医生走来走去,汗水湿透了她刚刚浆硬的衣服。
\par 噢,站在医生旁边就有那种恶心感,他锋利的手术刀割开生坏疽的肌肉时,那感觉是想吐又不敢吐出来!噢,听到手术室里进行截肢手术时传来的尖叫声,那又有多恐怖!看到等着医生来医治的战士们那一张张紧张、惨白的脸,心里便会产生懊丧、可怕却无可奈何的同情心,还有的战士们耳边充斥的是尖叫声,有的则等着听这些恐怖的话:“对不起,我的孩子,可那只手只得切除了。是的,是的,我知道;可是你瞧,看到那些红色的条纹了吗?只得切除了。”
\par 现在氯仿很紧缺,只有最厉害的截肢手术才能用。鸦片也珍贵得不得了。它只被用来为弥留之际的人减轻痛苦,让他离开这个世界,尚有一口气的人是不能用的。奎宁和碘根本就没有。是的,思嘉对这一切都厌烦透了。那天早晨,她真希望自己像媚兰一样能有怀孕这样的借口。那大概是现在既不参加护理、又能为公众所接受的唯一借口了。
\par 中午,她脱下围裙。梅里韦瑟太太正忙着给一个瘦长难看又不识字的山里人写信,她偷偷溜了出来。思嘉觉得自己再也无法忍受了。这是强加给她的职责,她还知道,中午到站的列车再送来伤员时,那就会够她忙到晚上的——而且很可能要饿肚子。
\par 她快步穿过两个距离不长的街区,向桃树街走去,大口呼吸着新鲜空气,束得很紧的紧身胸衣能让她吸多大口,她就吸多大口。她站在街角,犹豫着不知下一步该怎么办,既不好意思回白蝶姑妈家去,又下定决心不回医院去。这时,白瑞德正好驾车从此经过。
\par “你看上去就像个捡破烂的小孩。”他说,两眼打量她打着补丁的淡紫色印花上衣。衣服已被汗水印得东一道西一道,脸盆里溅出来的水更是把它弄得污迹斑斑。思嘉又窘又气,不禁怒火中烧。他干嘛老是要注意女人的服饰呢,还居然敢如此无礼地对她现在毫不整洁的穿着妄加评论?
\par “我一句话也不想听你说。你下来,扶我上车,带我到没人看得见的地方去。就算他们绞死我,我也不回医院了!我的天,我可没有发动这场战争,我根本不明白,我为什么就得做到死,而且——”
\par “真是我们光荣事业的叛徒!”
\par “真是责人严而利己宽。你扶我上车去。我不在乎你到哪儿。你现在得载我兜兜风。”
\par 他敏捷地跳下车来。她突发奇想,看到个身心健全的男人真是太好了。他不缺眼睛,不缺胳膊断腿,也没有因痛苦而脸色苍白,或是因疟疾而脸色发黄。他看上去营养丰富、身体健康。他的穿戴也很体面,上衣和裤子都是同一种面料做的,穿在身上非常合适,既不会太宽松,也不会紧得几乎动不了。它们还是簇新的,不会破洞百出,露出脏兮兮、光秃秃的肌肉及毛茸茸的大腿。他看起来就像是在这世界上了无牵挂似的,而在这种世道,这一点本身就已经够令人吃惊的了。因为其他人全都在担心忧虑,心事重重,一脸严肃的神情。他褐色的脸上无动于衷,红润的嘴巴线条分明,像女人的一样,显然很性感。他把她抱上车去,爽朗地大笑起来。
\par 他身材高大,肌肉擦着他裁剪很好的衣服,发出窸窸窣窣的声音。他上了车,坐在她身边。像以往一样,她感觉到他强健的体魄,心里被触动了,就像是被人猛击了一记似的。她看着他衣服下隆起的有力的双肩,心里涌起一股迷恋之情。这使她颇为不安,还有点害怕。他的身体似乎很健康、很强壮,健康、强壮得就像他敏锐的思维一样。他的力量是一种轻松适然、优雅得体的力量,慵懒得就像一头在阳光下伸展四肢的美洲狮,而这美洲狮却又警觉得很,随时都准备好扑上前去展开进攻。
\par “你这个小骗子,”他边唤着马,边这么说,“你和士兵们跳舞跳了个通宵,还送给他们玫瑰花和丝带,告诉他们说你是多么希望为事业献出自己的生命。可一要你去包扎几个伤口,抓几个虱子,你就连滚带爬、逃之夭夭了。”
\par “你就不能说点别的,把车赶快一点?只要梅里韦瑟老爷爷不要碰巧从他商店里出来看见我,并且告诉那个老太婆,那就是我的万幸了——我是指梅里韦瑟太太。”
\par 他用鞭子碰了碰骡子,骡子脚步轻快地跑过五角场,穿过把这城市一分为二的铁轨。载着伤员的列车已经进站了,在炎热的阳光下,抬担架的人正快手快脚地忙活着,把伤员移到救护车和有篷的军用货车上。看着他们,思嘉心里没有一丝不安的感觉,只是为自己成功地逃避了差事而感到莫大的安慰。
\par “我对那破医院厌烦极了,讨厌极了,”她说,用手抚平下摆宽大的裙子,还把下巴上帽带绑成的蝴蝶结绑牢些。“每天都有越来越多的伤员进来。这都是约翰斯顿将军的错。如果他在多尔顿勇敢地阻击北方佬的话,他们就已经——”
\par “可他确实勇敢地阻击过,你这啥事也不懂的小姑娘。但如果他一直艰守在那的话,舍曼就可以从侧面包抄他,把他卡死在两翼的部队之间。而且,他很可能就把铁路丢了,而约翰斯顿正是为铁路而战。”
\par “噢,那,”思嘉支吾着,她对军事战略一无所知。“不管怎么说,还是他的错。他应该对此采取行动才是,我认为他应该被免职。他为什么不站稳脚跟好好打仗而要撤退呢?”
\par “你跟其他人一样,大肆叫嚷着要‘把他头砍了’,就因为他无法做到不可能办到的事。在多尔顿,他是耶稣救世主,而现在在肯纳索山,他却成了叛徒犹大,而这一切就发生在六个星期内。可是,若让他把北方佬赶回二十英里去,他就又变成耶稣了。我的孩子,舍曼的兵力是约翰斯顿的两倍,他可以用两个士兵的生命来换一个我们勇敢的小伙子。而约翰斯顿连一个士兵都丢不起。他急需新的兵员,可他得到的是什么呢?‘乔·布朗的宠物们’,他们会帮什么忙呀!”
\par “民兵是不是真的要被叫去参战啊?还有城卫队?我还没听说呢。你是怎么知道的?”
\par “有传言在说,也就知道了。传言是今天早晨从米利奇维尔来的火车上传出来的。民兵和城卫队都要被派去补充约翰斯顿将军的部队。是的,布朗州长心爱的队员们最后也很可能要去闻闻火药味了,我想,大多数人都会大吃一惊的。他们肯定从来都没想到会要开拔。实际上,州长等于曾向他们许诺过,他们是不要开拔的。哦,简直是跟他们开了个天大的玩笑。他们以为他们已经有了防弹衣,因为州长甚至跟杰夫·戴维斯对着干,拒绝派他们去弗吉尼亚,说是需要他们保卫这个州。谁又曾想到,战争真的打到他们自家的后院来了,而他们也真的非得保卫自己的州不可了。”
\par “噢,你怎么还笑得出来,你这冷酷无情的家伙!想想城卫队里那些老先生们和小男孩吧!哦,小菲尔·米德非去不可了,还有梅里韦瑟老爷爷及韩亨利叔叔。”
\par “我不是说那些小男孩和墨西哥战争的老兵们。我是在说像威利·吉南那样勇敢的年轻人。他们喜欢穿着漂亮的军服,舞刀弄剑的——”
\par “还有你自己呢!”
\par “亲爱的,这一点也不会使我难堪!我不穿军服,不舞刀弄剑,南部邦联的命运与我毫无关系。再说,就此而言,我不会死在城卫队或是任何部队里。我在西点军校受过足够的训练,能让我享用终身……哦,我希望乔老将军交好运。李将军无法给他提供任何帮助,因为在弗吉尼亚,北方佬已经够他忙的了。所以,佐治亚州的部队是约翰斯顿能得到的唯一补充了。他本该得到更好的兵力,因为他是个伟大的战略家。他总是能想办法在北方佬到达之前抵达某个地方。可他如果想保护铁路,他就不得不要往后撤;你听着,他们要是把他从山里赶了出来,到较平的地段时,他会被碎尸万段的。”
\par “到这里的时候?”思嘉叫了起来。“你知道得很清楚,北方佬绝不会进到这么远的地方来!”
\par “肯纳索离这只有二十二英里,我敢打赌,你——”
\par “瑞德,你看,路那头!那群人!他们不是士兵。到底是什么……哦,他们是黑奴!”
\par 街那头扬起一大片红色的尘土,由远而近,尘土中传来一片脚步声及上百个或是更多的黑人的声音,喉音很重,在随意地唱着一首曲子。瑞德把马车赶到街边停下,思嘉好奇地看着满身大汗的黑人。他们肩上扛着凿镐和铁锹,由一个军官和一小队戴着工兵徽章的人带领着。
\par “到底是什么……?”她又开口了。
\par 接着,她的目光便落到了走在前排的一个正唱着歌的大个子黑人身上。他大约有六英尺半高,身材高大,皮肤漆黑,迈着剽悍的动物般轻巧自如的步伐,领着整帮人唱着“下去,摩西”,洁白的牙齿一露一露的。当然,在这世界上,再也没有比大个子萨姆(塔拉的工头)个头更高、声音更大的人了。可大个子萨姆在离家这么远的地方干什么呢?特别是现在,种植园里没有监工,而他就是嘉乐的左膀右臂呢。
\par 她从座位上欠起身,想看仔细些,这时大个子看到她,认出她来了,漆黑的脸上绽开了欢快的笑容。他停下脚步,放下铁锹,朝她走来,一边还对他近旁的黑人大叫道:“见鬼!这是思嘉小姐!你们,伊莱贾、阿波斯特尔、普罗菲特!那是思嘉小姐!”
\par 队伍中一阵忙乱。人群犹犹豫豫地停了下来,咧嘴笑着。大个子萨姆身后跟着另外三个大块头黑人,他们穿过马路朝马车跑来,紧跟在后面的是困惑不解、大喊大叫的军官。
\par “回到队伍中去,你们这些家伙!回去,我在叫你们哪,要不我就——哦,是韩太太。早晨好,夫人,早晨好,先生。你们要到哪儿去,要煽动兵变和不服管束?天知道,今天早晨,这些小伙子已经给我添够多麻烦了。”
\par “噢,兰德尔上尉,别怪他们!他们是我们家的人。这是萨姆,我们的工头,还有塔拉庄园的伊莱贾、阿波斯特尔和普罗菲特。当然,他们得跟我说说话。你们好吗,小伙子们?”
\par 她跟他们一一握手,雪白的小手都被他们宽大的黑色手掌全给盖住了。这次见面使这四个人高兴得欢呼雀跃的,这下可以向同伴炫耀一下自己家有个多么漂亮的年轻小姐了,他们脸上一脸得意的神色。
\par “你们到离塔拉这么远的地方来干什么?我敢肯定你们一定是逃出来的。难道你们不知道巡逻队是一定能抓住你们的?”
\par 他们被这玩笑逗乐了,高兴得哇哇大叫。
\par “逃出来?”大个子萨姆回答说,“不,我们没有逃出来。他们派人来叫我们来的,因为我们比塔拉其他人个子更大,身体更壮。”他白色的牙齿得意得老露出来,“他们特别指名要俺,因为俺歌唱得好。是的,是弗兰克·肯尼迪先生,他经过的时候把我们带走的。”
\par “可为什么呢,大个子萨姆?”
\par “我的天,思嘉小姐!你难道没听说?我们要去给白人先生挖沟,好让他们在北方佬来的时候藏起来。”
\par 这种对散兵壕的天真解释使兰德尔上尉和马车上坐的人都忍不住笑了起来。
\par “当然,他们要俺走时,嘉乐先生自然很不高兴。他说,没有俺,他没法弄好塔拉。可是埃伦小姐说:‘把他带走吧,肯尼迪先生。南部邦联比我们更需要大个子萨姆。’她给了俺一块钱,要俺照白人先生吩咐的去做。这样,我们就到这里来了。”
\par “这都是怎么回事,兰德尔上尉?”
\par “噢,这很简单。我们得加固亚特兰大的防御工事,要多挖几英里长的散兵壕,将军没有办法从前线的兵员中抽调兵力去做这件事。所以我们就强行征用乡下强壮的黑人来干了。”
\par “可是——”
\par 一丝恐惧掠过思嘉的心头,令她不寒而栗。挖更多的散兵壕!他们为什么需要更多的散兵壕呢?过去的一年中,亚特兰大周围已经建了一系列大型的土筑多面堡,里面还安了大炮,从城中心起方圆一英里都有。这些大型的土木工事都和散兵壕相连,一英里又一英里,直到把整个城市环绕住。现在却还要更多的散兵壕!
\par “可是——我们已经修筑好防御工事,为什么还要修筑更多的工事呢?我们连现有的都用不上了。将军肯定不会让——”
\par “我们只有环城一英里的地方才有防御工事,”兰德尔上尉唐突地打断她,“想舒舒服服——或说安然无恙,这防御工事离城就太近了。这些新的工事会延伸得远一些。你知道,再撤退一次,我们的队伍就退到亚特兰大了。”
\par 他马上对自己最后这句话感到后悔了,因为她的眼睛因为恐惧而瞪得大大的。
\par “当然,不会再撤退的,”他赶忙补充说,“肯纳索山上的防线是坚不可摧的。大炮都布在山麓两侧,可以控制所有的道路,北方佬不可能通过的。”
\par 但是思嘉注意到,在瑞德懒散而锐利的目光注视下,他垂下了眼睛。她害怕了。她想起了瑞德的话:“如果北方佬把他赶下山来,到较平的地段时,他就会被碎尸万段的。”
\par “噢,上尉,你认为——”
\par “哦,当然不会!你连一秒钟也没必要担心的。乔老将军比较相信预先防御。这是我们挖更多战壕的唯一理由……可我得走了。和你说话,真是令人愉快……和你们的主人告别吧,小伙子们,我们得走了。”
\par “再见了,小伙子们。哎,如果你们病了、受伤了或是遇到麻烦了,就告诉我。我就住在桃树街,就在那,差不多是城尽头的最后一所房子。等等——”她在包里摸找着。“噢,天哪,我一个子儿也没有。瑞德,给我一点钱,诺,大个子萨姆,给自己和小伙子们买些烟抽。好好干,照兰德尔上尉吩咐的去做。”
\par 乱糟糟的队伍重新排好队,路上又扬起了一片红色的尘土。他们走了,大个子萨姆又领头唱起歌来。
\refdocument{
    \par “走吧,摩西!到遥远的埃及去!
    \par 去告诉法老
    \par 把我们的人放掉!”
}
\par “瑞德,兰德尔上尉在对我撒谎,其他所有的男人也一样——他们不想让我们女人知道事实真相,怕我们会晕倒。还说他没撒谎?噢,瑞德,若是没有危险,他们干吗要挖这些新的胸墙?部队真的这么缺人手,居然到要用这些黑人的地步了吗?”
\par 瑞德唤着骡子。
\par “部队太缺人手了。要不然城卫队为什么要被调出来呢?至于挖壕沟,嗯,万一城被围了,防御工事就被认为是很有用的。将军准备在此决一死战。”
\par “围城!哦,掉转马头。我要回家去,回到塔拉的家里去,马上就走。”
\par “是什么使你这么苦恼呀?”
\par “围城!我的上帝,围城!我听说过围城!爸爸曾经经历过,或者是他的爸爸曾经经历过。爸爸告诉我……”
\par “什么时候的围城?”
\par “德罗达赫的围城,克伦威尔占领爱尔兰的时候。他们连吃的都没有。爸爸说,他们全都饿死在街上,最后他们就吃猫、老鼠,甚至吃蟑螂这样的东西。他还说他们投降之前,有过人吃人的现象,我从来就不知道该不该相信这一点。克伦威尔占领了该城之后,所有的妇女都——围城!圣母马利亚呀!”
\par “你是我见过的最最无知的年轻人了。德罗达赫大概是在十六世纪发生的事,那时郝先生根本就还没出生。再说,舍曼也不是克伦威尔。”
\par “当然不是,但比他还更糟糕!他们说——”
\par “至于说那些爱尔兰人围城时吃的奇怪的食物——就我个人来说,我也会欣然吃下一只美味可口的多汁老鼠,就像吃下旅馆里最近提供的一些食物一样。我想,我只得回里士满去啦。那里总是有美味佳肴等着你,只要你有钱付账就行。”看到她脸上那副惊恐万分的神情,他眼里露出了嘲弄意味。
\par 她为自己露出了慌乱之情感到很不好受,便大叫道:“我真不明白你在这待这么长时间干什么!你想的只不过就是过得舒服,吃得痛快以及——以及那一类事情。”
\par “我真不知道还有什么比吃呀——哦——那一类事情更令人愉快的过日子的方式了,”他说,“至于说我为什么待在这——哦,我读过很多有关围城、被围攻的城市以及类似的书,可我从来没有亲眼见识过。所以,我想待在这亲眼看见一下。我不会受到伤害的,因为我是个平民百姓,不是战斗人员,再说,我需要这种经历。千万别错过新的经历,思嘉。它们会使你的大脑更发达。”
\par “我的大脑已经够发达了。”
\par “也许这一点你是知道得最清楚的,可我要说——那样就太没风度了。或许,我待在这是为了围城真的开始时能救救你。我还从来没有救过危难中的小姐呢。那也是一种新的经历。”
\par 她知道,他又在取笑她了,可她还是从他的话里感觉出某种认真的意味。她摇了摇头。
\par “我不需要你来救我。我会照顾好自己的,谢谢。”
\par “别这么说,思嘉!如果你愿意,想想就行了,但千万别对男人说这种话。北方姑娘们的麻烦就出在这。如果她们不是老跟你说她们会照顾好自己、谢谢你这些话,她们就会是最迷人的了。一般说,她们说的也是实话,上帝保佑她们。所以,男人们便让她们自己照顾自己去了。”
\par “瞧你,说起来没完没了的,”她冷冷地说,自己被说成像个北方姑娘,这种侮辱比什么都厉害。“我相信,关于围城的事是你在撒谎。你知道的,北方佬绝不能到达亚特兰大。”
\par “我跟你打赌,一个月内他们就会抵达这里。我跟你赌一盒夹心糖,赢的话——”他乌黑的眼睛移到了她的嘴唇上,“你让我吻一下。”
\par 有一瞬间,害怕北方佬侵入的恐惧感紧紧抓住了她的心,但一听到“吻”这个字,恐惧感便烟消云散了。这可是她熟门熟路的,比军事行动有趣多了。她好不容易才控制住自己,不让自己因高兴而笑出声来。自他送给她那顶帽子那天起,瑞德便再也没有什么进一步的举动了,也就是说,不管在什么方面都可以被认为是情人之举的举动。他从来就不会上当受骗,去谈论一些私下里的话题,就算她一直努力也白搭。可是现在,她丝毫没有施展什么诡计,他却在谈“接吻”了。
\par “我才不在乎这类私下里的话题呢,”她冷冰冰地说,设法挤出了一个皱眉的动作,“再说,对一头猪,我也同样会送上一个香吻的。”
\par “人各有所好,我经常听说,爱尔兰人对猪有偏爱——实际上是把猪养在床铺底下。可是,思嘉,你太需要接吻了。这就是你不对劲的地方。你的所有男朋友都太尊重你了,天知道这是为什么,或者说他们太怕你,在你身边就老是出错。结果,你变得傲慢得很令人难以忍受。你应该被人吻,而且这个人应该知道如何接吻。”
\par 谈话并没有像她希望的那样进行。她和他在一起的时候,从来就没有像她希望的那样。历来都是这样,这是场决斗,而她被击败了。
\par “我想,你自以为是最合适的人吧?”她挖苦地说,拼命控制自己,不让自己发脾气。
\par “噢,是的,如果我刻意去找这个麻烦的话,”他漫不经心地说。“别人都说我接吻吻得很好。”
\par “噢,”她爆发了,自己的魅力受到蔑视,她为此愤愤不平,“哦,你……”可她的眼睛突然却又一片茫然。他在微笑,但在他乌黑的眼睛深处,有一丝微弱的亮光闪了一下,就像是一抹不太旺的火焰。
\par “当然,你很可能会纳闷,我为什么没有在高雅地碰了你一下后乘胜追击,就是我送给你那顶帽子的那天——”
\par “我从来没有——”
\par “那你就不是个好姑娘了,思嘉,很遗憾听到这话。男人不吻她们的时候,所有真正的好姑娘都会想想为什么的。她们知道,不应该要求他们这么做,如果他们真这么做了,她们就得表现出受到侮辱的样子来,可是还是一样,她们都希望男人会……哦,亲爱的,振作起来。总有一天,我会吻你的,你也会喜欢的。但不是现在,所以,我请求你不要太不耐烦了。”
\par 她知道他在取笑她,可是,和以往一样,他的取笑总是使她很恼火。他说的话总是有很多是真的。哦,也就是这点毁了他。他若是如此没有教养,想对她很放肆的话,她就会给他点颜色看看。
\par “能不能请你掉个头,白船长?我想回医院去了。”
\par “你真的这么想吗,我的护理天使?那么,虱子和污秽还是比我的谈话更可取了?我绝不会阻止一双情愿为我们光荣的事业劳作的手。”他掉转马头,回头朝五角场驶去。
\par “说到我为什么没有采取进一步的行动,”他无动于衷地说,就好像她并没有表明谈话已经结束似的,“我在等你再长大一些。你知道,我现在吻你不怎么好玩。对我自己的乐趣,我是很自私的。我从来没想过要去吻个孩子。”
\par 他忍住笑,因为从眼角的余光中,他看到她的胸部起伏不停,虽然默默无言,可显然非常愤怒。
\par “还有,”他继续轻声说着,“我在等你对那值得尊敬的卫希礼的记忆慢慢淡去。”
\par 一提到希礼的名字,她的心里突然涌起一阵痛楚,眼角也一阵刺痛,忽然很想痛哭一场。淡去?对希礼的记忆永远也不会淡去,即使他死了一千年,也绝不会淡去。她想到希礼受了伤,被关在遥远的北方佬的监狱里。他已在弥留之际,身上没有毯子盖,没有一个爱他的人在握着他的手。想到这里,她心里顿时痛恨起身边这个保养得极好的人来,他那慢吞吞的声音总是在嘲笑人。
\par 她愤恨交加,一句话也说不出来。他们默默地向前走了一段路。
\par “实际上我对你和希礼的什么事都很了解,”瑞德重拾起话题,“一开始就看到了十二棵橡树你那不雅的一幕。自那以后,我两眼睁着就看到了许多事情。什么事情呢?噢,你对他还保留着一个女学生式的浪漫情怀,他也在他那尊贵的个性允许的范围内给你些回报。卫太太却对此一无所知,而在你们之间,你对她耍了漂亮的一招。我实际上了解所有的一切,只有一件事不太明白,而这激起了我的好奇心。那个高尚的希礼有没有吻过你,给自己不朽的灵魂抹黑呢?”
\par 他得到的回答是面无表情的沉默,她还把头扭了过去。
\par “啊,哦,这么说,他真的吻过你了。我猜是他在这里休假的时候吧。可现在,他很可能已经死了,你则把这永远珍藏在心里。但我敢肯定,你会慢慢淡忘的。当你把他的吻忘掉后,我会——”
\par 她气势汹汹地转过身来。
\par “你——见鬼去吧,”她绷着脸说,绿色的眼睛愤怒得眯成了一条缝,“让我下车,要不我就要跳下去了。我再也不想和你说话了。”
\par 他把马车停了下来。还没等他下车扶她,她已经跳下车。她的裙环被车轮钩住了,那一瞬间,五角场的人流都能瞥见她的衬裙和长裤。接着,瑞德俯下身,很快地松开了钩住的地方。她一言不发地掉头离去,连回头看一眼都没有,而他却轻声笑了,嘴里还呼唤着马匹。

\subsubsection{第十八章}

\par 自开战以来,亚特兰大第一次听到了战争的声音。一大清早,城市的喧嚣还没有开始,肯纳索山上的炮声便从远处飘然而至,声音沉闷、隐隐约约的,但极有可能转化成夏天的雷电声。有时候,甚至在交通嘈杂的大中午也能听到。人们装着不去听它,尽量和往日一样交谈、欢笑、做着自己的事,仿佛离他们只有二十二英里远的地方并没有北方佬。可是,耳朵总是会不由自主地竖起来去听那里传来的声音。整个城市罩上了一层忧心忡忡的面纱,不管人们的手里在忙活什么,耳朵却都在聆听着,聆听着,心跳便会突然加快,这种情况一天会有一百次之多。声音是不是变大了?还是说是他们自己认为变大了呢?约翰斯顿将军这次能不能阻住他们呢?他做得到吗?
\par 只要表层的薄纸一捅破,底下的恐慌就露出来了。撤退把大家的神经绷的一天比一天紧,已经接近崩溃的边缘。谁也不说害怕这个字眼。这个话题是个禁忌,可是紧张的神经却通过大肆攻击将军表现了出来。公众的感觉就像在发高烧一样。舍曼已经到了亚特兰大的大门口。再撤退一次,南方军就要退进城里来了。
\par 给我们一个不再往后撤的将军!给我们一个能站住脚跟、拼死奋战的人!
\par 耳边回响着远处传来的隆隆炮声,州里的民兵,即“乔·布朗的宠物们”,还有城卫队,离开了亚特兰大,去保卫约翰斯顿背后查特胡奇河上的桥梁和渡口。这是个灰蒙蒙、阴沉沉的日子,他们穿过五角场,拐上了到玛丽埃塔的路。天下起了濛濛细雨。全城人都出来欢送他们。桃树街上,他们在商店前面的木制遮篷下,一个挨着一个站着。尽力表现出高兴的样子来。
\par 思嘉和梅贝尔·梅里韦瑟·皮卡德获准离开医院去送那些人出征,因为亨利叔叔和梅里韦瑟爷爷都在城卫队。她们和米德太太站在一起,挤在人群中,踮起脚尖,好看得清楚些。虽然思嘉心里也装满了南方人共同的心愿,相信战争进程中只有最令人振奋、最使人放心的事,可看着这一排排走过去的杂牌军,心里不禁打了个寒噤。如果这群本该蹲防空洞的乌合之众,这些老人和童子军都被召出来的话,局势一定是到了孤注一掷的地步!当然,队伍中也有身强力壮的年轻人,他们穿着靓丽的城卫队制服,头上的羽毛摇来摆去,饰带还在跳动不停。那制服式样是经过共同筛选后才定下来的。可是,队伍中有这么多的老人和少年,看到他们,她的心一阵紧似一阵,既怜惜他们,又感到很害怕。在淅淅沥沥的小雨中,一些比她父亲年纪还大的老翁尽量跟着横笛和鼓乐队的节奏,洋洋得意地向前走着。梅里韦瑟老爷爷肩上披着梅里韦瑟太太最好的格子披巾挡着雨,站在第一排,他用一个粲然的微笑向姑娘们致意。她们挥着手帕,对他喊着祝福的道别话;可梅贝尔却紧紧抓住思嘉的胳膊,喃喃低语:“噢,可怜的老爷爷!一场真正的暴风雨差不多就能要了他的命!他的腰部风湿痛——”
\par 亨利叔叔走在梅里韦瑟老爷爷后面一排。他黑色的长大衣领口直竖到耳际,皮带上别着两把墨西哥战争时用的手枪,手里还拎着一个毛毡制的手提包。旁边走着的是和亨利叔叔年龄几乎相仿的黑仆,他举着一把雨伞为自己和主人遮着雨。和长辈们肩并肩走着的是少年童子军。他们没有一个看上去超过十六岁。许多人都是从学校跑出来参军的,不时还可看到穿着制服的军校学员,已经被雨淋湿的灰色军帽紧紧扣在头上,帽子上插着黑色的公鸡羽毛,干净的白色帆布皮带交叉着系在湿漉漉的胸部。菲尔·米德就在他们中间。他佩戴着他死去的哥哥的马刀和马枪,看上去非常骄傲。他的帽子还用针别在一边,一副勇敢无畏的样子。米德太太尽力挤出一丝微笑,不停挥着手,直到他从面前走了过去,接着便把头靠在思嘉的背上。那一刻,她全身的力气似乎突然消失了,人变的瘫软无力。
\par 这些人中,许多人完全没有配备武器,因为南部邦联既没法给他们发枪支,也没法给他们发弹药。他们希望能从被杀死或是被捕的北方佬手里缴获武器武装自己。许多人靴子上插着长猎刀,手里拿的是又长又粗的棒子,顶部有尖尖的铁尖头,人们把它们称作“乔·布朗长矛”。幸运些的,肩上便扛着老旧的燧发式步枪,皮带上别着火药筒。
\par 撤退中,约翰斯顿已经损失了大约一万兵力。他需要一万多新的兵员。“可这,”思嘉害怕地想,“就是他所能得到的!”
\par 大炮隆隆驶过,溅起的泥浆洒到了欢送的人身上。这时,一个骑着骡子走在大炮旁边的黑人吸引了她的视线。这是个一脸严肃、脸色像马鞍的颜色一样的年轻黑人。看到他,思嘉叫了起来:“是莫斯!希礼的莫斯!他到底在这干什么呀?”她挤过人群,来到街沿石边上,大叫道:“莫斯!停一下!”
\par 看到她后,小伙子勒住缰绳,漾开了粲然的笑容,飞身下了骡子。骑马走在他身后的一个浑身湿漉漉的中士叫了起来:“别下骡子,小伙子,要不我就开枪啦!我们得准时赶到山上。”
\par 莫斯不知所措地看看中士,又看看思嘉。她踏着泥泞的泥浆,走近经过的车旁边,拉住了莫斯马镫的皮带。
\par “噢,一会工夫就行,中士!你别下来,莫斯。你到底到这来干什么?”
\par “俺又要去打仗了,思嘉小姐。这次不是和希礼先生,而是和约翰老先生一起去。”
\par “卫先生!”思嘉不禁目瞪口呆。卫先生已经快七十岁了。“他在哪?”
\par “在大炮后面,思嘉小姐。在那后面!”
\par “对不起,夫人。走吧,小伙子!”
\par 思嘉在那站了好一会,大炮过处,泥浆没到了她的脚踝。“噢,不!”她心想。“不可能的。他年纪太大了。他也不喜欢打仗,就像希礼一样!”她退后几步,退到街沿石边,扫视着经过的每一张面孔。最后一门大炮也开过去了,拉着弹药箱的前车吱吱呀呀地驶了过来,溅起了一片泥浆。这时,她看到了他,高高瘦瘦的身材,身榜挺直,银白的长发披散在脖颈周围。他骑着一匹草莓色的小母马,这匹马在泥浆飞溅、坑坑洼洼的路上择路而行,姿态极为轻巧,仿佛是个穿着缎子裙子的夫人。哦——那是内利!塔尔顿太太的内利!比阿特丽斯·塔尔顿最宝贝的母马!
\par 看到她站在泥泞中,卫先生高兴地笑了。他勒住马缰,下了马,朝她走来。
\par “我一直希望会看到你,思嘉。你家里人要我传递的口信太多了。但时间来不及了。我们今天早晨才到这,他们就催着我们马上出城,你都看到了。”
\par “噢,卫先生,”她抓着他的手,拼命叫着。“别走!你为什么也要去呢?”
\par “呵,这么说,你也认为我太老啦!”他笑了,这简直就是希礼的微笑,只不过出现在一张更苍老的脸上而已。“也许我年纪大了,行军虽然不行,但骑马射击还算可以。塔尔顿太太真是太好了,她把内利借给了我,所以我的马是挺不错的。我希望内利不会出什么事。如果她出事的话,我就无颜面对塔尔顿太太了。内利是她剩下的最后一匹马。”现在的他在放声大笑,把她的恐惧也给赶跑了。“你妈妈、爸爸和妹妹都很好,他们叫我转达他们对你的爱。你爸爸今天差一点就和我们一块来了!”
\par “哦,爸爸不行!”思嘉害怕极了,叫了起来,“爸爸不行!他不会去打仗吧,对不对?”
\par “不,但他原来是想去的。当然,他膝盖不能弯曲,走不了远路,但他要骑马跟我们一起走。你妈妈提出,他要是能跳过牧场的围栏,她就同意,她说部队里会有很多难骑的路段。你爸爸认为这太容易了,可是——你相信吗?他的马跑到围栏跟前时,却死死地停下不跳了,你爸爸就从它头顶上摔了下来!他没扭断脖子就是奇迹了!你知道他有多固执的。他爬了起来,又试了一次。哦,思嘉,他总共摔了三次,最后郝太太和波克把他抬到床上去了。此事弄得他心烦意乱的赌咒,发誓说肯定是你妈妈‘在那畜生的耳边嘀咕了什么话’。他还不能起来自行走动,思嘉。你没必要因此感到很丢脸。毕竟得有人留在家里为部队种庄稼。”
\par 思嘉根本不会觉得丢脸,只是感到非常庆幸。
\par “我把英蒂和哈尼送到梅肯,去和伯尔一家住在一起,让郝先生看管塔拉的同时,照管一下十二棵橡树……我得走了,亲爱的。让我吻吻你漂亮的脸蛋吧。”
\par 思嘉下巴扬了起来,喉咙里一阵堵塞,心里感到很痛苦。她太喜欢卫先生了。很久很久以前,她还指望过能做他的儿媳妇呢。
\par “你还得把这一吻带给白蝶和媚兰,”他说着,又轻轻吻了她两次,“媚兰怎么样?”
\par “她很好。”
\par “啊!”他注视着她,却又看穿了她,像希礼曾经做过的那样,越过她,灰色的眼睛里目光飘忽不定,直看到另外一个世界去。“我本来是很想看看我第一个孙子的。再见了,亲爱的。”
\par 他翻身骑上内利,慢吞吞地走了,手里还拿着帽子,银发暴露在细雨中。思嘉重新加入梅贝尔和米德太太的行列,这时才猛然意识到他最后说的话的含义。迷信使她心里一阵恐惧。她画着十字,试图祈祷一番。他提到了死亡,就像希礼过去那样,而现在希礼他——谁也不能提到死的!这会引诱上帝也提及死亡。三个女人在雨中默默地往医院走去,思嘉心里祈祷着:“也不能是他,上帝。不能是他和希礼!”
\par 从多尔顿撤到肯纳索山是五月初到六月中旬发生的事。六月炎热而多雨,日子一天天过去,舍曼又无法使南方军从那些陡峭、滑溜的山坡上撤走,希望便再次抬起头来。大家又变得兴高采烈的,对约翰斯顿将军的言辞也更为友善了。潮湿的六月渐渐转入了越发潮湿的七月,南方军在那壕沟纵横的制高点拼死奋战,牵制着舍曼,亚特兰大因此而欢欣鼓舞。希望就像香槟酒一样涌入了大家的头脑。好哇!好哇!我们阻住他们了!晚会和舞会盛行一时。一有士兵从前线来城里过夜,有人就会为他们开宴会,之后便是舞会。姑娘们和先生们的比例已经是十比一了,所以舞会上大多数人都是女的,她们都争着和士兵们跳舞。
\par 亚特兰大挤满了人,有来访者、难民、医院里伤兵的家属们、在山上打仗的士兵们的妻子和母亲,她们都希望他们受伤时能离他们近些。除此以外,由于乡下只剩下十六岁以下的孩子和六十岁以上的老头,一群群姑娘也都来到城里。白蝶姑妈最不赞成这最后一种人了,她认为她们到亚特兰大来没有别的原因,只是为了找个丈夫。她们这么厚颜无耻,这使她感到很纳闷,不知这个世界到底会变成什么样子。思嘉对她们也持不赞成态度。这些十六岁的姑娘们有着红红的脸蛋,粲然的笑容,能使人忘记她们改过两次的上衣和打着补丁的鞋子。但她倒不在乎她们带来的激烈竞争。她自己的衣服比她们大多数人的都更漂亮、更新,这还得感谢白瑞德最后偷闯封锁线的那艘船给她带来的面料。但是,她毕竟已经十九岁,而且年纪还会越来越大,而男人总是习惯追求傻乎乎的年轻姑娘。
\par 和这些漂亮的年轻姑娘比起来,一个拖着一个孩子的寡妇毕竟处于劣势,她心想。但在这些令人激动的日子里,守寡和当了妈妈这两件事比以往任何时候给她造成的压力都来得轻。白天在医院护理,晚上又参加晚会,其间的间隙,她几乎连韦德的面都见不上。有时候,在相当长的一段时间内,她真的忘了自己还有个孩子。
\par 温暖、潮湿的夏夜,亚特兰大的家门大开,欢迎保卫城市的战士们。从华盛顿街到桃树街,所有大房子灯火点点,招待着从散兵壕归来的泥迹斑斑的勇士们。班卓琴声、小提琴声和跳舞的脚步声、轻松愉快的欢笑声,在夜色中传得很远。一群群人围在钢琴边,用欢快的声音唱着忧伤的歌曲“你的信来了却来晚了”,衣衫褴褛的勇士们深情地望着躲在火鸡羽毛做的扇子后面咯咯直笑的姑娘,请求她们别再等下去了,要不然会来不及的。可只要她们做得到,没有一个姑娘会干等的。歇斯底里式的快乐和激动的狂潮淹没了整个城市,他们闪电式地结婚了。约翰斯顿把敌人挡在肯纳索山的那个月中,结婚的人不计其数。新娘红着脸、一脸幸福地出现在人们面前,华丽的服饰是从多至一打的朋友那匆匆忙忙借来的。新郎则佩戴着马刀,马刀碰撞着打着补丁的膝盖。这么令人激动,这么多的晚会,这么多激动人心的事!好哇!约翰斯顿正把北方佬挡在二十二英里以外的地方呢!
\par  
\par 不错,肯纳索山的防线是坚不可摧的。打了二十五天后,连舍曼将军也相信了这一点,因为他的损失太大了。他不再采取正面进攻的方式,又挥师进行了一次大范围的包抄,想占领南方军和亚特兰大之间的地段。这个战略又一次奏效了。为了保护后方,约翰斯顿只得放弃了他防守很好的山峰。在那次战斗中,他已经失去了三分之一的兵力,余下的部队步履艰难地在雨中跋涉,穿过乡间地带,朝查特胡奇河开去。南方军已经不能指望有更多新的兵员了,而从田纳西以南到战场之间这一线铁路,现在却掌握在北方佬手里,它每天都在源源不断地给舍曼送来新的兵员和装备。就这样,穿灰军服的部队的战线便穿过泥泞的田地向后退去,向后朝亚特兰大退去。
\par 被认为不可攻克的阵地最终失守了,新的恐惧席卷了整个城市。在二十五天狂欢、幸福的日子里,每个人都向别人保证,这是不可能发生的。可现在却真的发生了!然而,将军肯定能把北方佬阻在河对岸。虽然说只有天才知道到底办得到办不到,因为河流离这里太近了,只有七英里远!
\par 但是舍曼又从侧翼采取行动,在他们的上游渡河。疲惫不堪的南方军被迫急急忙忙蹚过浑浊的河水,驻守在侵略者和亚特兰大之间。他们在城北面的桃树河河谷里匆匆挖掘浅浅的掩体,亚特兰大则陷入痛苦和恐慌之中。
\par 战斗,然后撤退!战斗,然后撤退!每次撤退都使北方佬离城更近一些。桃树河离城只有五英里了!将军到底是怎么想的?
\par “给我们一个能站住脚浴血奋战的人!”这一呼声甚至传到了里士满。里士满也知道,一旦亚特兰大失守,战争也就完结了。部队渡过查特胡奇河后,约翰斯顿将军被免职了。他属下的一个指挥官——胡德将军接管了部队,整个城市的呼吸才轻松了一些。胡德不会撤退的。那个胡须飘动、双眼炯炯有神的高个子肯塔基人是不会撤退的!他还以“大炮”的绰号而闻名呢。他会把北方佬从河边赶回去的,是的,赶回多尔顿去。可是,部队却呼喊着“把乔老将军还给我们!”因为他们从多尔顿一路跟随乔老将军转战至此,疲惫不堪地走过了不知多少路途。他们也知道他们面临的局势,而普通百姓是不会知道的。
\par 不等胡德准备好,舍曼就发起了进攻。在指挥权变更后的那一天,北方军的将军以迅雷不及掩耳之势,袭击了离亚特兰大六英里远的小镇迪凯特,并且占领了该镇,在那里切断了铁路。这条铁路连接着亚特兰大和奥古斯塔、查尔斯顿、威尔明顿和弗吉尼亚。舍曼给了南部邦联极为致命的一击。已经到了采取行动的时候了!亚特兰大强烈要求采取行动!
\par 接下来的七月,一个热得冒汗的下午,亚特兰大的愿望终于实现了。胡德将军所做的远远不只是站住脚浴血奋战。他在桃树河对北方军发起了猛烈的攻势,命令他的属下从散兵壕里冲出来,向比他们多一倍的穿蓝色军服的北方军战线发起进攻。
\par 每个人心里都很害怕,都在祈祷着胡德的进攻会把北方佬赶走。他们听着沉闷的炮声及成千上万步枪的射击声。虽然离城中心有五英里远,但声音很大,听起来几乎就像是从隔壁街区传过来的。他们听得见大炮的隆隆声,看得见滚滚的硝烟在树林的上空翻卷着,就像挂在低空的云朵。可是,一连好几个小时都没人知道战况如何。
\par 快到傍晚的时候,传来了第一条消息,可是消息很不确定,互相矛盾,令人感到很害怕。消息是由战斗刚开始那几个小时中受伤的伤员带来的。这些人开始鱼贯而来,有的独自一人,有的成群结队,伤势较轻的搀扶着那些一瘸一拐、步履蹒跚的。他们很快就形成了一条稳定的人流,痛苦万状地朝医院走去。他们的脸被硝烟炮火熏得黑乎乎的,就像黑人一样,浑身上下都是尘土和汗水,伤口没有包扎,血已经凝固,成群的苍蝇围着他们直转。
\par 从城北面挣扎着走过来的伤兵们最早到达的房子之一便是白蝶姑妈的房子。他们一个接一个迈着蹒跚的脚步来到门口,一屁股坐在绿油油的草地上,哇哇乱叫:
\par “水!”
\par 一整个艳阳似火的下午,白蝶姑妈和她的家人,黑人也罢,白人也罢,全都站在烈日下,提着水桶,拿着绷带,舀水给伤兵们喝,为他们包扎伤口,直到绷带全部用完,连撕开的床单和毛巾也全部用尽为止。白蝶姑妈是个看见血就会晕过去的人,现在却把这点全给忘了,跑上跑下忙活着,那双小脚穿的鞋又太小,走得小脚都肿了起来,再也支撑不了她。连挺着大肚子的媚兰也忘了羞怯,兴奋地和普里西、厨娘及思嘉一起忙活着,脸上的紧张神情不亚于任何一个伤员。最后,她晕倒了,可连个让她躺的地方都没有,只好让她躺在厨房的桌子上,因为屋里的每张床铺、每张椅子和每张沙发都挤满了伤员。
\par 在这一片忙乱中,小韦德完全被遗忘了。他蹲在前面游廊的栏杆后面,像个关在笼子里的惊恐万状的小兔子一样,眼睛因恐惧而瞪得大大的,边吮着大拇指,边打着嗝。有一次,思嘉看到他,便厉声对他说:“到后院玩去,韦德!”但眼前这副惨景令他又害怕又着迷,他便没有照母亲说的话去做。
\par 草地上满是疲乏沮丧的人,他们太累了,无法再往前走。由于受伤,他们都已经太虚弱,根本无法动弹。彼德大叔便把这些人弄上马车,载到医院去,一趟一趟地载,直到老马都累得大汗淋漓。米德太太和梅里韦瑟太太也把她们的马车派来了。马车上路时,连弹簧都被伤员的体重压弯了。
\par 后来,在漫长、炎热的夏日黄昏,从战场上开过来的救护车隆隆驶过,还有盖着帆布篷的军需货车。接着就是农场货车、牛车,甚至还有私人马车,这些都是被医疗队征用的车辆。它们从白蝶姑妈的屋子前面经过,在崎岖不平的路上颠簸着,里面载满了伤员和濒临死亡的士兵,一滴滴鲜血滴落到红色的尘土中。看到提着水桶、拿着勺子的女人,车辆都停了下来,响起了一片低语声:
\par “水!”
\par 思嘉托着那些摇晃不已的头,让那些焦渴的嘴唇能喝到水;把成桶的水泼在那些满身尘土、正在发烧的士兵身上,还把水泼在开裂的伤口上,好让那些人的疼痛能得到暂时的缓解。她踮着脚尖,把勺子递给救护车司机,心都跳到了喉咙口,对每个司机发问:“有什么消息没有?有什么消息没有?”
\par 大家都这么回答她:“具体情况还不知道,夫人。现在说还为时尚早。”
\par 夜幕降临了,气候闷热难挡。天空中一丝风也没有,黑人举着燃烧的松节,使天气更加闷热。尘土塞满了思嘉的鼻孔,嘴唇也直发干。那天早晨刚刚洗得干干净净、浆得硬硬的淡紫色印花裙子已被血水、尘土和汗水弄得斑迹点点的。希礼写信时说过,战争不是什么光荣的事,而是污秽和痛苦。这么说,这就是希礼所指的意思了。
\par 劳累给这整个画面蒙上了一层不真实的梦幻般的色彩。这不可能是真的——如果这是真的,那这世界就乱套了。如果不是真的,那她为什么要站在白蝶姑妈这宁静安详的前院里,站在这闪闪烁烁的火光中,把水泼向这些即将死去的朋友们呢?有这么多人都曾经是她的朋友。他们看到她时,都尽力挤出一丝微笑。这么多她很熟悉的人沿着这黑乎乎、尘土飞扬的道路颠簸着,这么多人死在她的眼皮底下,成群的蚊子和小昆虫伏在他们流着鲜血的脸上。她曾经和这些人一起跳舞、一起欢笑,她曾为他们弹过琴、唱过歌,她曾取笑过他们、安慰过他们、爱过他们——一点点。
\par 她在一辆牛车最下面一层的伤员中看到了凯里·阿什伯恩。他头部中弹,已经剩下最后一口气了。可她要想弄他出来,就得烦扰另外六个伤员,所以她就只好让他上医院去了。后来,她听说,不等医生来看他,他就死了,后来被埋在某个地方,谁也不知道具体在哪里。那一个月里安葬的人太多了,都是被埋在奥克兰墓地里匆匆掘出的浅浅墓穴里。媚兰感到很伤心,因为无法拿到凯里的一绺头发,好送给他在亚拉巴马的妈妈。
\par 闷热的夜晚在慢慢地过去,她们累得腰酸背痛,连膝盖也直不起来了。思嘉和白蝶向一个又一个人发问:“有什么消息没有?有什么消息没有?”
\par 随着漫漫长夜一分一秒地过去,她们终于听到了回答,可这回答却令她们脸色惨白,面面相觑。
\par “我们在撤退。”“我们只好撤退了。”“他们的人数比我们多出好几千人。”“惠勒带领的骑兵在迪凯特被切断了,北方佬袭击了他们。我们得增援他们。”“我们的部队马上会全部撤到城里来。”
\par 思嘉和白蝶紧紧抓住对方的手臂,不让自己摔倒。
\par “北方佬——北方佬真的要来了吗?”
\par “是的,夫人,他们是会来,但他们不会来这么远的地方。”“别发愁,小姐,他们无法占领亚特兰大的。”“不,夫人,我们在城四周有上百英里的防御工事呢。”“我亲耳听到乔老将军说过:‘我可以永远守住亚特兰大。'”“可我们现在没有乔老将军了。我们有——”“住嘴,你这个白痴!你想吓唬太太小姐们吗?”“北方佬永远无法占领这个地方的,夫人。”“你们这些太太小姐们为什么不到梅肯或是别的更安全的地方去呢?你们在那没有亲戚吗?”“北方佬不会占领亚特兰大的,但他们既然想占领它,这对太太小姐们便不太好。”“会有一次很猛烈的炮轰。”
\par 第二天,下了一场温暖的透雨,天空中雾气蒙蒙的。成千上万吃了败仗的部队拥进亚特兰大,要从这里经过。他们疲惫不堪、又饿又累,连续七十五天的战斗和撤退,搞得他们筋疲力尽,他们的战马饿得只剩皮包骨,大炮和弹药箱上绑着残缺不全的绳索和牛皮条。但他们走来的时候并不像毫无次序的乱民和乌合之众。他们有条不紊地走着,穿着褴褛的衣衫却还洋洋自得,破损的红色战旗在雨中高高飘扬。在乔老将军的领导下,他们学会了该怎样撤退,乔老将军可是把撤退也当成同进军一样的战略的。一排排胡子拉碴、衣衫褴褛的士兵和着《马里兰!我的马里兰!》的音乐,沿着桃树街前进,全城人都出来为他们欢呼。不管是打了胜仗还是吃了败仗,他们同样都是他们的战士。
\par 不久以前还穿着华丽簇新的军服出征的州里的民兵,现在走在受过战火洗礼的队伍中,已经很难认出来了。他们太肮脏、太邋遢,但眼里有了一种新的神采。他们道歉了三年,一再解释为什么没有上前线,现在这些已经是几辈子以前的事了。他们已经放弃了后方的安全,换来了战斗的艰辛。许多人都已经用悠闲自在的生活换来了痛苦不堪的死亡。他们现在是老兵了,只服了很短时间兵役的老兵,但还是算老兵,而且他们表现得很出色。他们在人群中搜寻着朋友的面孔,骄傲、挑战似的注视着他们。他们现在可以昂首挺胸了。
\par 城卫队的老人和男孩走了过来。白发苍苍的老人累得连脚都几乎抬不起来了,男孩的脸上是一副过早面对大人的问题而感到疲倦的表情。思嘉看到了菲尔·米德,几乎认不出他来了。炮灰和尘垢把他的脸弄得漆黑,严峻的考验和过度的疲乏使他神经极为紧张。亨利叔叔一瘸一拐地走了过去,在雨中,他没有戴帽子,身上披着一块老旧的油布,中间穿了一个洞,头从洞里伸了出来。梅里韦瑟老爷爷坐在一个炮架上,光着的脚裹着被子的破布片。虽然她尽力寻找着,可连卫约翰的影子也没见到。
\par 然而,约翰斯顿的老部下们还是迈着毫不疲倦、无忧无虑的步伐走了过来。三年以来,他们一直都是这样。他们还有余力对漂亮的姑娘们咧嘴而笑,向没有参军的男人开着粗鲁的玩笑。他们正在开赴环绕全城的战壕——不是匆匆忙忙挖成的浅浅的战壕,而是齐胸深的、用沙袋加固过的土木工事,顶部还插着削尖的木棒。这些战壕环绕了全城,一英里又一英里,红色的沟壑上面盖上了红色的土堆,等着人来填满它。
\par 人群向队伍欢呼着,就像他们是凯旋归来的勇士一样。每个人心里都怀有恐惧,可是,既然现在已经知道了事实真相,既然最糟的事情已经发生,既然战争已经打到家门口,全城人反倒变了。现在不再恐慌,不再歇斯底里了。心里所想并不会在脸上表现出来。每个人看上去都很快乐,虽然说这种快乐伴随着紧张感。每个人都尽力向部队展示一副勇敢、自信的面孔。每个人都在重复着乔老将军在被免职前不久说过的话:“我可以永远守住亚特兰大。”
\par 现在,胡德也不得不撤退了,很多人便和战士们一样,希望乔老将军能够官复原职,可是他们强忍着不说出来,只从乔老将军的话中获得勇气。
\par “我可以永远守住亚特兰大!”
\par  
\par 约翰斯顿将军采用的是谨慎的战术,这可不是胡德的作风。他一会从东面袭击北方军,一会又从西面袭击北方军。舍曼把全城团团围住,就像个摔跤运动员,试图从对手的身上找到一个抓手的地方。胡德没有待在散兵壕里等着北方军来向他们进攻。他大胆地出去迎击他们,向他们猛扑过去。仅仅几天工夫,亚特兰大战役和埃泽拉教堂战役都打完了,这两个地方都是规模较大的交战,这反倒使桃树街的交战变成是小打小闹了。
\par 可是,北方佬总是会回来再次开战。他们的损失惨重,但他们输得起。他们的炮兵一直在猛轰亚特兰大,待在家里命也不保,屋顶被掀翻了,街上被炸出一个个大弹坑。城里人在地下室、在坑道里、在铁路沟渠里挖出的浅浅的隧道中尽可能地躲避着炮火。亚特兰大被包围了。
\par 在胡德将军接管指挥权后的十一天内,他损失的兵力几乎和约翰斯顿七十四天中打仗和撤退时损失的兵力一样多。亚特兰大已经三面受敌。
\par 亚特兰大到田纳西的铁路全线现在都落到了舍曼的手里。他的军队穿过铁路到了东部,切断了往西南方向通往亚拉巴马的铁路。只有往南的一条铁路,就是通往梅肯和萨凡纳的还在通行。城里挤满了士兵、伤员和难民,这唯一的一条铁路线已经满足不了这个水深火热的城市迫切的需要了。但只要这条铁路还掌握在手中,亚特兰大就能够坚持下去。
\par 这条铁路太重要了,思嘉意识到这一点时,她害怕极了。为了控制这条铁路,舍曼一定会奋勇作战,而胡德也会拼死保住它。因为这条铁路贯穿全县,而且经过琼斯伯勒。而塔拉离琼斯伯勒只有五英里!比起亚特兰大这个尖叫声不断的地狱来,塔拉倒像是个避难所。可是,塔拉离琼斯伯勒只有五英里!
\par  
\par 亚特兰大战役打响那天,思嘉和其他太太小姐们坐在商店的平屋顶上,打着小巧的阳伞遮着太阳,坐在那观战。但是,第一发炮弹落到街上时,她们便赶紧逃到地下室去了。就在那天晚上,妇女、儿童和老人组成的撤退大军开始从城里出发了。他们的目的地是梅肯,那天晚上乘火车走的人当中,有许多人在约翰斯顿从多尔顿撤退时就已经逃难过五六次了。和到亚特兰大时的旅程相比,他们现在的旅程可是轻松多了。许多人只带着一个毛毡袋和包在印花大手帕里的简陋的午餐。到处可见一脸惊恐的仆从们拿着银水罐、刀叉及在第一次开仗时抢救出来的家庭画像。
\par 梅里韦瑟太太和埃尔辛太太都不肯走。医院需要她们。她们还自豪地说,她们一点都不害怕,哪个北方佬也无法把她们从自己的家里赶走。但梅贝尔和她的孩子及范妮·埃尔辛都去了梅肯。自结婚以来,米德太太第一次采取了反叛的行动,医生命令她坐火车到安全的地方去,可她断然拒绝了。她说,医生需要她。再说,菲尔就在战壕里,她想离他近些,万一……
\par 但怀廷太太和思嘉圈子里的许多太太小姐都走了。白蝶姑妈是最早对乔老将军的撤退策略提出非难的人,现在也是最早收拾行李的人之一。她说,她的神经很脆弱,受不了噪声。她担心一有爆炸就会晕过去,连想走到地下室去也办不到。不,她可不是害怕。她孩子般的小嘴很想装出一副英勇的神情来,但是办不到。她要到梅肯去,和她的表妹伯尔老太太住在一起,姑娘们得跟她一块去。
\par 思嘉可不愿去梅肯。她虽然也被炮弹吓坏了,但她宁愿待在亚特兰大而不愿到梅肯去,因为她打心里讨厌伯尔老太太。多年以前,在卫家举行的一次晚会上,思嘉和伯尔太太的儿子威利接吻时,被她当场逮住,伯尔太太便说她很“放荡”。“不,”她对白蝶姑妈说,“我要回塔拉去,让梅利跟你一起去梅肯吧。”
\par 一听到这话,媚兰便伤心、害怕地哭了起来。白蝶姑妈飞奔去找米德医生时,媚兰抓住思嘉的手恳求道:
\par “亲爱的,别到塔拉去,别离开我!没有你我就太孤单了。噢,思嘉,孩子出生时,没有你在我身边,我会死的!是的——是的,我知道我有白蝶姑妈在身边,她是很好。但她毕竟从来没生过孩子,有时候,她还会使我神经很紧张,紧张得想尖叫出来。别抛弃我,亲爱的。对我来说,你一直像是我的妹妹一样,再说,”她惨淡地笑了笑,“你答应过希礼要照顾我的。他对我说过,他会恳求你这么做的。”
\par 思嘉不解地盯着她。她这么讨厌这个女人,几乎都无法掩饰这一点,梅利怎么可能还如此爱她呢?梅利怎么会这么傻,猜不出她在默默地爱着希礼这个秘密呢?这几个月中,她在痛苦的煎熬中等着有关他的消息,已经不下百次地泄露了自己的秘密。可媚兰什么也没有看到,媚兰什么也看不到,只看到她所爱的人身上的优点……不错,她是答应过希礼她会照顾媚兰。“噢,希礼!希礼!过了这么多个月,你一定已经死了!可是现在,你这诺言反倒伸出手来把我抓住了!”
\par “哦,”她唐突地说,“我确实答应过他,我也不会毁约。可我不想去梅肯和伯尔那只老猫待在一起。只要五分钟我就会把她的眼珠子都抓出来的。我要回塔拉的家中去,你可以和我一块去。你去了,妈妈一定会很高兴的。”
\par “噢,我也赞成这个主意!你妈妈人也很好。可是你知道的,孩子出生时,姑妈要是没跟我在一起,她非死不可。我知道她不会到塔拉去。那里离打仗的地方太近,姑妈想要安全些。”
\par 米德医生上气不接下气地赶来了。白蝶惊恐万状地去叫他,他还以为媚兰至少是要早产了。他非常生气,发了一大通牢骚。知道她不舒服的原因后,他开口说话了。他的话便把事情定了下来,一点商量的余地也没有。
\par “你去梅肯是不可能的,梅利小姐。你要离开此地,我可不能答应。火车又挤又没个准,如果需要火车运送伤员或是部队和装备,乘客随时有可能在森林里被叫下车去。像你这种情况——”
\par “如果我和思嘉一起到塔拉去——”
\par “我跟你说吧,我不会让你走的。去塔拉的火车也就是去梅肯的火车,情况是一样的。再说,现在谁也不知道北方军在哪里,可他们无处不在。你坐的火车甚至有可能被拦截。就算你安全抵达琼斯伯勒,到塔拉也还有五英里,路很不好走。行动不方便的妇女是走不了那种路的。再说,方丹老医生参军以后,县里一个医生也没有了。”
\par “可还有接生婆呢——”
\par “我说的是医生!”他粗暴地说,目光无意识地落到她小巧的身架上,“我不会让你走的。这很危险。你不会想让孩子在火车上或是马车上出生吧,对不对?”
\par 从医学角度如此坦率的话使夫人们脸窘得通红,不再吱声了。
\par “你只能待在这里,这样我就能关照你。你还必须卧床休息。不能在楼梯上走上走下。到地下室去。那是绝对不行的,即使炮弹从窗口飞进来也不行。这里的危险毕竟也没那么大。我们很快就能把北方佬打回去的……好了,白蝶小姐,你马上到梅肯去,让年轻姑娘们待在这里。”
\par “没有年长的人陪伴?”她大叫起来,一脸愕然。
\par “她们都已经结过婚了,”医生烦躁地说,“米德太太家离这只隔了两座房子。梅利小姐这种情况,她们不会再在家里接待男性客人了。我的天,白蝶小姐!这是在战时。我们现在没法顾及礼节了。我们得为梅利小姐着想。”
\par 他步履沉重地走出房间,等在前面的游廊上,直到思嘉走了过去。
\par “我得把实话告诉你,思嘉小姐,”他开口说道,用手捋着胡子,“你似乎是个懂得一些常识的年轻姑娘,所以你也不必脸红了。我不想再听到诸如梅利小姐要走的话。我很怀疑她能否经受得了这种旅途。往最好处想,她生的时候也会非常困难——你知道,她的臀部太窄,生的时候很可能需要用产钳,所以,我不想让任何无知的黑人接生婆给她瞎弄。像她那样的女人是不该生孩子的,可是——不管怎样,你把白蝶小姐的箱子收拾好,送她去梅肯。她老是吓得半死,只会使梅利小姐心里难受,对她半点好处也没有。好了,小姐,”他目光锐利地瞥了她一眼,直看到她的心里去,“我也不想听到你要回家的话。你跟梅利小姐待在一起,等她把孩子生下来。你不会害怕吧,对不对?”
\par “噢,不会!”思嘉在撒谎,但很坚定。
\par “那才是个勇敢的姑娘。你们需要的话,米德太太会来陪伴你们的。如果白蝶小姐把她的仆人带走了,我会叫老贝齐来给你们做饭。不会要很久的。再过五个星期,孩子就会出世。可是头胎孩子都很难说,再加上这隆隆的炮声。孩子随时都可能会出生。”
\par 这样,白蝶姑妈泪流满面地去了梅肯,把彼德大叔和厨娘也带走了。她出于爱国热情,一时冲动把马车捐给了医院,可马上就后悔了,这又使她流了更多的眼泪。思嘉、梅利和韦德及普里西留了下来。虽然炮轰还在继续,这屋子已经安静了许多。

\subsubsection{第十九章}

\par 围城的头几天,北方军不时突破守城的防线,一会这里撕了个口子,一会那里打开个洞。炮弹到处开花,思嘉被吓坏了,只能孤独无助地打着哆嗦,双手捂住耳朵,随时准备着被炸成灰烬。一听到象征炮弹来临的呼啸声,她就冲到媚兰的卧室去,颓然倒在她身边。于是两个人紧紧拥抱着,把头埋在枕头里,“噢!噢!”地尖叫着。普里西和韦德则急急忙忙跑到地下室去,蹲在布满蛛网的黑暗中。普里西尖声高叫着,韦德则低声饮泣,还打着嗝。
\par 头顶上呼啸而过的是死亡的威胁,埋在羽毛枕里又几乎透不过气来,思嘉暗暗诅咒媚兰,正是为了她,她才无法跑到楼梯底下的地下室去。可医生不许媚兰走动,思嘉只得跟她待在一起。除了担心会被炸得粉身碎骨外,媚兰的孩子随时都可能出世,这也同样使她感到害怕。每次一想到这点,思嘉浑身都会冒出黏糊糊的冷汗。如果孩子要出生了,她该怎么办?她知道,现在的炮弹就像四月里下雨一样满街乱落,要她在这种时候出去找医生,那她宁愿让媚兰死。她也知道,普里西是宁愿被打死也不会去冒这个险的。如果孩子要出生了,她该怎么办?
\par 一天晚上,她和普里西正在为媚兰准备晚餐,她们低声讨论了这些事情。令思嘉大为吃惊的是,普里西居然消除了她的恐惧。
\par “思嘉小姐,俺想,梅利小姐要生孩子的时候,如果我们找不到医生,你也不用担心。俺能对付。生孩子的事,俺什么都知道。难道俺妈不是接生婆吗?她不是把俺也教成接生婆了吗?把这事交给俺就行了。”
\par 知道有经验的帮手在身边,思嘉心里的石头落了地,呼吸也更轻松了,但她还是渴望着这一痛苦早点结束,快快过去。她急于远离爆炸的炸弹,渴望回到塔拉家中那宁静的氛围中去,于是,每天晚上,她都在祈祷着孩子第二天就能出世,这样,她就可以从诺言中解脱出来,可以离开亚特兰大。塔拉远离所有的痛苦,好像很安全。
\par 思嘉很想家,很想她的妈妈,她这辈子还从来没有这么热切地想念过别的什么呢。如果她在埃伦身边,那不管发生什么,她都不会害怕的。在尖声呼啸、震耳欲聋的炮声中又过了一天之后,每天晚上上床睡觉时,她都下定决心要告诉媚兰,她在亚特兰大一天也无法再忍受下去了,她要回家去,媚兰就只好到米德太太家里去。可是,她头一碰到枕头,脑海里便浮现出她和希礼最后一次见面时他脸上的神情。他因内心的痛苦而拉长着脸,嘴角却挂着一丝笑容:“你会照顾媚兰的,是不是?你这么坚强……答应我。”她也就答应了。希礼已经不知在什么地方在地下长眠了。但不管在哪里,他都在注视着她,要她守约。不管自己还活在人世或长眠地下,她都不能有负于他,不管要付出多大的代价。就这样,她又一天天地留下来了。
\par 埃伦来信极力要求她回家去,她在回信中把围城的危险缩小到最低的程度,解释了媚兰危险的处境,答应孩子一出生就回家去。埃伦对亲戚关系非常敏感,血缘关系也罢,姻亲关系也罢。她又回了信,勉强同意她待在那,但要求说必须马上先送韦德和普里西回家。普里西举双手赞成这个建议。现在的普里西一听到什么异常的声响,就会变成一个吓得牙齿直打颤的白痴。她很多时间都蹲在地下室里。要不是米德太太的呆头呆脑的老贝齐,姑娘们过得可就惨了。
\par 思嘉和她妈妈一样,急于把韦德从亚特兰大送走。这不但是为了孩子的安全,而且是因为他老是害怕,那样子使她心烦。韦德已经被炮弹吓得不敢说话了。即使轰炸暂停的时候,他也老是拉着思嘉的裙子,吓得连哭都哭不出来。晚上他不敢去睡觉,怕黑,怕睡着了北方佬会来把他抓走。晚上,他紧张不安、抽抽答答的哭声刺得她的神经都受不了。她心里其实也和他一样害怕,可他紧张、拉长的脸每时每刻都在提醒她这一点。为此,她非常生气。是的,塔拉才是适合韦德待的地方。普里西得把他带到那去,然后再马上回来。孩子出生时,她得在场。
\par 然而,思嘉还没来得及送他们两人踏上回家的旅程,就传来了这样的消息,说北方军开到了南面,在亚特兰大和琼斯伯勒之间的铁路沿线到处骚扰,小打小闹。假如北方佬拦截了韦德和普里西坐的火车呢——想到这点,思嘉和媚兰脸都白了。大家都知道,北方军对孤独无助的孩子所施的暴行比对妇女的还更恐怖。所以她又害怕送他回家了,他也就留在了亚特兰大,像个惊恐万状、默默无言的小鬼魂,拼命跟着他妈妈,手里一时半刻没有抓住他妈妈的裙子,他就会感到害怕。
\par 七月的天气非常炎热,围城在继续。夜晚阴沉、宁静,伴有不祥之感。夜晚过去了,炮声隆隆的白天又开始了。可这个城市开始调整自己了。事情好像是这样的,既然最糟的事情都已经发生,他们便再也没有什么可害怕的了。他们曾经害怕围城,而现在围城已经发生,而且毕竟还不算太糟。日子照旧可以过下去,而且也确实和往常几乎没什么两样。他们知道,他们正坐在一座火山上。可火山若要爆发,他们也无能为力。那为什么现在就要担心呢?何况火山很可能根本就不会爆发。就看看胡德将军是怎样把北方军挡在城外的就行了!看看骑兵部队是怎样把到梅肯的铁路控制在手里的!舍曼永远也不会得到它!
\par 尽管面对落下的炮弹和越来越不足的配给,但他们表面上显得很不在乎;尽管北方军离他们只有半英里远,但他们却只当没看见;尽管对散兵壕里穿着褴褛的灰色军服的部队有无限的信心,可是,亚特兰大这个城市的表皮下面,流动着一股狂野的情绪,不知道第二天会发生什么。悬而未决、担心忧虑、痛苦、饥饿和一会充满希望,一会又伤心失望带来的痛苦,正在使这表皮一天薄过一天。
\par 渐渐的,思嘉从朋友们一张张勇敢的脸上获得了勇气。无法治愈的就必须忍受,大自然也在宽厚仁慈地调整着自己。思嘉也从其中获得了力量……当然,听到爆炸声她还是会跳起来,但她不再尖叫着跑去把头埋在媚兰的枕头底下了。现在,她也能够一边大口吃着东西,一边无力地说:“那颗炮弹挺近的,对不对?”
\par 她现在不怎么害怕了,这还因为生活已经蒙上了一层梦幻般的色彩,这是场可怕的噩梦,可怕得一点真实感也没有。她,郝思嘉,不可能处于这么危险的境地当中,每时每刻都受到死亡的威胁。生活那种安宁的进程,不可能在这么短的时间内就变得面目全非。
\par 这太不真实了,不真实到了荒唐的地步。天亮时还是色彩柔和的蓝蓝的天空,后来就被炮火的硝烟玷污了,烟雾就像挂在低空的雷云一样笼罩着整个城市;温暖的中午曾经到处飘荡着一簇簇忍冬属植物和爬藤玫瑰的恬淡的幽香,现在却变得如此可怕。炮弹呼啸着落在街上炸裂开来,仿佛世界末日来临。弹片飞到了几百码开外,人和动物被炸得粉身碎骨。
\par 下午的午睡已经不再安静、慵懒,战争的喧闹时不时或有停息,可桃树街却每时每刻都生气勃勃、忙乱热闹——大炮和救护车隆隆驶过;伤员从散兵壕里蹒跚而来;部队匆匆忙忙跑步而过,被指挥官从城这边的壕沟里调到城那边的工事去,因为那里敌人的攻势很强;传令兵们沿街冲向司令部,好像南部邦联的命运全都掌握在他们的手里。
\par 炎热的夜晚带来了些许安宁,可这安宁却伴随着一种不祥之感。静谧的夜晚降临时,那是太过安静了——似乎雨蛙、昆虫和打着瞌睡的模仿鸟也都害怕过头了,它们在夏夜的常规合唱中,好像连音调都不敢提高。最后一道防线中的旧式步枪不时发出尖利的噼啪声,打破了这种宁静。
\par 在夜深人静、灯火尽熄的时候,媚兰也已酣然入睡,整个城市一片死静。思嘉在辗转难眠之时,经常会听到前门门插的咔嗒声和轻柔、急迫的敲门声。
\par 总是有士兵站在前面的游廊上,黑暗中看不清他们的脸,但许多不同的声音异口同声地从黑暗中传来,跟她说话。有时候是阴影中某个斯文的声音:“夫人,非常抱歉打扰了你,你能不能给我和我的战马一些水喝?”有时候是山地人那种生硬的喉音,有时候又是最南端长满狗牙草的平坦的乡间那怪怪的鼻音。偶尔,沿海地带那慢吞吞的声音也会在她心里打个激灵,这使她想起了埃伦的声音。
\par “小姐,我有个伙伴本是要到医院去的,可他好像到不了那里了。你能不能把他收下来?”
\par “夫人,我喝点水就行了,如果有的话,我也想要块玉米饼。”
\par “夫人,请原谅打搅了你——我能不能在你的游廊上过夜?我看到那有玫瑰,还闻到了忍冬青的香味,这里太像家里了,所以我斗胆——”
\par 不,这些夜晚不是真的!它们都只是一场梦魇,这些人全都是梦魇的一部分。她看不见他们的身体和脸庞,温煦的暗夜里只传来他们跟她说话的疲惫不堪的声音。提水、招待饭菜、在房子前面的游廊上放好枕头、包扎伤口及抱着生命垂危的士兵那脏兮兮的头。不,这些事不可能发生在她身上!
\par 七月底的一天晚上,来敲门的却是亨利叔叔。亨利叔叔身上少了雨伞和毛毡旅行袋,连肥胖的大肚皮也不见了。他粉红色的胖脸上皮肤松弛,一褶一皱的,就像是叭喇狗颈部下垂的皮肉,长长的白发脏得简直无法形容。他几乎可以说是光着双脚,脚上还爬有虱子。他饥饿交加,可他那暴躁的脾气却丝毫也没有改变。
\par 他说:“这真是场愚蠢透顶的战争,连我这样的老头也得去端枪打仗。”虽然他这么说,可给姑娘们的印象却是,亨利叔叔在自鸣得意呢。他也像年轻人一样派上用场了,他正在做着和年轻人一样的事。再说,他还能赶得上年轻人,比梅里韦瑟老爷爷强多了,他跟她们说起这些时,显得很高兴。老爷爷的腰部风湿病又犯了,而且很厉害,上尉想免去他的兵役。可老人不愿回家。他坦率地说,他宁愿听上尉的咒骂和凌辱,也不愿回去忍受儿媳妇的悉心照料。她总是不停地要求他不要嚼食烟草,还要他每天洗胡子。
\par 亨利叔叔只待了一会儿,因为他只请了四小时的假,可有一半的时间得花在从防御工事到家里的路上。
\par “姑娘们,我得有一阵子不能来看你们了,”他正坐在媚兰的卧室里。思嘉提来一桶凉水放在他面前,他起疱的双脚正在水里舒舒服服地蠕动着。“我们的连队早上就要开拔了。”
\par “到哪去?”媚兰害怕得抓住了他的手臂问道。
\par “别把手放在我身上,”亨利叔叔烦躁地说,“我身上有虱子在爬呢。要不是有虱子和痢疾,战争就会像野餐一样有趣了。我要到哪去?咳,没有人告诉我,可我倒有个相当不错的预感。我们早晨就要朝南往琼斯伯勒去,除非我错得太离谱才不是这样。”
\par “噢,为什么要往琼斯伯勒去呢?”
\par “因为那里要打一场大战,小姑娘。如果可能的话,北方佬正想把那里的铁路夺过去呢。如果他们成功了,那就得跟亚特兰大说再见了!”
\par “噢,亨利叔叔,你觉得他们会成功吗?”
\par “哪会这样,姑娘们!不会的!有我在那,他们怎么可能成功呢?”亨利叔叔望着她们一脸害怕的样子,咧嘴笑了。可紧接着又一脸严肃:“那会是场硬战,姑娘们。我们得打赢。你们当然知道,北方佬已经占领了所有的铁路线,只有到梅肯的那条除外,但他们占领的远不只这些。也许你们姑娘们还不知道,他们也占领了每一条公路、马车道和马道。只剩下麦克多诺路了。亚特兰大就像被装进了袋子,而拉紧袋口的绳子就在琼斯伯勒。如果北方佬占领了那里的铁路,他们就可以拉紧绳子,把我们闷在里面,就像在小袋中的负鼠一样。所以,我们的目标就是不让他们占领那条铁路……我可能要离开一阵子了,姑娘们。我就是来向你们大家告别的,同时证实一下思嘉还跟你在一起,梅利。”
\par “她当然还跟我在一起,”媚兰嗔爱地说,“别为我们担心,亨利叔叔,千万要保重。”
\par 亨利叔叔在破地毯上擦干湿漉漉的脚,再穿上破烂不堪的鞋,嘴里嘟哝着。
\par “我得走了,”他说,“我还要走五英里路呢。思嘉,你给我装些午饭让我带走。什么都行。”
\par 他吻别了媚兰,下楼来到厨房。思嘉正把一块玉米饼和几个苹果包在餐巾里。
\par “亨利叔叔——真的——真的这么严重吗?”
\par “严重?见鬼,是的!别傻了。我们已经退到最后一道壕沟了。”
\par “你认为他们会到塔拉吗?”
\par “哦——”亨利叔叔开口说道。大事当前,女人还只会考虑自己的事,这使他很恼怒。可是看到她一副担惊受怕、愁眉苦脸的样子,他又心软了。
\par “当然不会。塔拉离铁路线还有五英里,那条铁路才是北方佬想要的。你真的还不如绿花金龟有头脑,小姑娘。”他突然停下不说了。“我今晚走了这么多路,不单是为了来跟你们告别的。我是来告诉梅利一些不好的消息的,可我刚想开口,却又不忍心对她说了。所以,我想让你来告诉她。”
\par “希礼没有——你没听说什么吧——他——死啦?”
\par “得啦,我一直站在散兵壕里,烂泥没到了屁股上,我怎么可能听到希礼的消息呢?”老先生烦躁地反问,“不。是他父亲的事。卫约翰死了。”
\par 思嘉颓然坐了下去,手里还抓着包了一半的午饭。
\par “我是来告诉梅利的——可我开不了口。这得由你来办了。再把这些东西交给她。”
\par 他从口袋里掏出一块挺沉的金表,表带还在晃悠着,还有久已辞世的卫太太的一副袖珍画像和一对袖口的大扣子。思嘉曾经无数次看到卫约翰手上戴着这块手表,现在猛一看到它,这才着着实实明白过来,希礼的父亲真的死了。她惊愕极了,既哭不出来,也说不出话来。亨利叔叔坐立不安的在一边咳嗽,不敢看她,怕看到她流眼泪,那会使他自己也感到很难过。
\par “他很勇敢,思嘉。把这告诉梅利。叫她写信跟他家的姑娘们说说。就他的年龄来说,他不愧是个好战士。一发炮弹打中了他。正巧落在他和他的马身上。把马都炸伤了——我亲自开枪把马打死的,可怜的东西。它真是匹出色的小母马。你最好也给塔尔顿太太写封信,告知她这一点。她非常珍视这匹马。把我的午饭包起来吧,孩子。我得走了。好了,亲爱的,别太往心里去。对一个老人来说,在年轻人的事业中死去,没有什么比这更好的方式了吧?”
\par “哦,他不该死的!他不该去打仗。他本该好好活着,看着他的孙子长大,平静地死在床上。噢,他干嘛要去呢?他不赞成脱盟,他也痛恨战争——”
\par “我们很多人都这么想,可又有什么用呢?”亨利叔叔烦躁地吸着鼻子。“我都一把年纪了,你以为我会乐意让北方佬的步枪手把我当靶子吗?可现在,作为一个绅士,已经没有别的选择了。和我吻别吧孩子,别为我担心。我会平安无事地度过这战争年月的。”
\par 思嘉吻了吻他,听着他的脚步声下了台阶,消失在黑暗中。她还听到前面大门门插打开的声音。她站在那里端详着手里的纪念品,看了好一会。然后才上楼去告诉媚兰。
\par  
\par 七月底传来了不受欢迎的消息,正如亨利叔叔所预料的,北方军再次挥师琼斯伯勒。他们在离城四英里处切断了铁路线,可却被南部邦联的骑兵部队打败了;工兵部队头顶烈日,挥汗如雨,修复了铁路线。
\par 思嘉都快急疯了。她等了整整三天,心里越等越害怕。后来嘉乐来了一封信,这才使她放下心来。敌人没有到塔拉。他们能听到战争的声音,但北方军的影也没见着。
\par 嘉乐的信里说到北方佬是怎样从铁路线上被赶跑的,信里大话连篇,牛皮吹得震天响,不知道的人还以为,这是他独自一人亲自创下的丰功伟绩呢。有关部队的勇敢行径,他写了满满三大页,在信末才简单地提到卡丽恩生病了,郝太太说是伤寒。她的病不太重,思嘉不用为她担心。可思嘉现在是无论如何也回不了家了,即使铁路很安全也白搭。围城开始时,思嘉和韦德没有回家,这倒使郝太太很高兴。郝太太说,思嘉必须上教堂去念些玫瑰经,好让卡丽恩早日恢复。
\par 最后这件事倒是使思嘉良心不安,因为她已经有好几个月没有上教堂去了。她也曾经也认为这一疏忽是极大的罪过,但是,不知怎的,现在没去教堂似乎并不像过去那样觉得罪孽深重了。但她还是听她妈妈的话,到房间去跪在地上急匆匆地咕噜了一段玫瑰经。她站起身,感到祈祷后并不像过去那样能得到心理安慰。有一段时间,她甚至还觉得,虽然每天有几百万几千万人向上帝祈祷,但上帝并没有垂顾她、南部邦联或是整个南方。
\par 那天晚上,她坐在屋前的游廊上,把嘉乐的信放在胸前,这样,她就可以不时地摸一摸,感觉塔拉和埃伦离她近一些。客厅里的灯光透过窗户,在被葡萄藤覆盖着的黑暗的游廊上投下金色的影子,缠结在一起的黄色爬藤玫瑰和忍冬青在她周围形成了一堵香味纷杂的围墙。夜宁静极了。从太阳落山到现在,连声枪响也没有,整个世界似乎离她很远。思嘉躺在躺椅上,前后摇动着。自从听到塔拉来的消息后,她一直感到很寂寞、很难受,希望能有人跟她在一起,谁都可以,连梅里韦瑟太太也行。可是,梅里韦瑟太太在医院值夜班,米德太太则在家里给从前线回家来的菲尔准备晚宴,媚兰又睡着了。甚至连碰巧有客人来访的希望也没有。过去这个星期中,一个客人都没有,因为每个能走的人都在散兵壕里,要不就在琼斯伯勒附近的乡间追击北方军。
\par 像现在这样独自一人待着,这对她来说并不是很经常的事,她不喜欢这样。独自一人时就得想事情,而这些日子里,所想的东西都令人不快。像其他人一样,她也养成了一个习惯,老是想起过去,想起死去的人。
\par 今晚,亚特兰大的夜如此宁静,她可以闭上眼睛,想像着自己回到了塔拉乡间那安详的岁月,生活没有变化,也不会变化。可是她知道,县里的生活永远也不会像过去一样了。她想起塔尔顿家的四个男孩,红头发的双胞胎和汤姆及博伊德,一股伤心之情涌到了喉咙口。咳,斯图尔特和布伦特本来哪一个都可能成为她的丈夫的。现在,等战争结束,她倒是可以回到塔拉去住,但她再也无法听到他们从雪松车道上冲过来时粗野的“喂”“嗨”的叫喊声了。还有舞跳得绝棒的雷福德·卡尔弗特,他再也不会选她做舞伴了。还有芒罗家的男孩,小乔·方丹及——
\par “噢,希礼!”她啜泣着,把头埋在手里,“我永远也不会习惯你的离去!”
\par 她听到前门咔哒响了一声,赶紧抬起头来,飞快地用手擦着泪眼。她站起身来,看到白瑞德从小径上走了过来,手里拿着他那顶巴拿马大帽子。自那天在五角场匆匆忙忙地从他的马车上下来以后,她至今也没有见过他。那一次,她可是明说了不想再见到他的。可现在如果有人跟她说说话,把她的思绪从希礼身上转移开,她也会很高兴。她马上便把这些思绪从脑海中赶走了。他显然已经忘了那次不快,或者说假装已经忘了。他坐在她脚边最高的一级台阶上,提都不提他们上次的分歧。
\par “这么说你没有逃难到梅肯去!我听说白蝶小姐已经撤退了,我自然也认为你也走了。所以,我看到你这有灯光时,便到这来看一下。你干嘛留在这里呢?”
\par “留下来陪媚兰。你知道,她——哦,她现在不能逃难。”
\par “呀!”他说道,灯光中,她看到他皱紧了眉头,“你不是要告诉我卫太太还在这吧?我还从来没听过有这么蠢的事。她那种情况太危险了。”
\par 思嘉默默无言,窘得不行,因为媚兰的情况不是她可以和一个男人讨论的话题。瑞德居然知道这对媚兰很危险,这也使她很难堪。一个单身汉知道这点,说明这人很坏。
\par “你就不想想我也可能会受伤的,你太没有风度了。”她尖刻地说。
\par 他双眼发亮,觉得很有趣。
\par “我敢打赌,你随时都能跟北方佬斗争的。”
\par “我还不敢肯定这是不是恭维话呢。”她说,心里拿不准他说这话是什么意思。
\par “不是,”他回答说,“你什么时候才会停止从男人最随意的话里寻找恭维话呢?”
\par “等我死到临头的时候。”她这么回答着,心里却在想,即使瑞德从来不恭维她,也总是有男人恭维她的,她不禁笑了。
\par “虚荣,虚荣,”他说,“至少,你对这点还是很坦率的。”
\par 他打开烟盒,抽出一根黑色的雪茄,凑到鼻子下闻了一会。他划燃火柴,往后靠在一根柱子上,双手握着放在膝盖附近,默默地抽了一会烟。思嘉重新摇动躺椅,温煦的夜晚无形的黑暗包围着他们。在玫瑰和忍冬青丛中做窝的模仿鸟从酣睡中醒来,发出了怯生生的柔和的叫声。接着,好像又慎重考虑了一下,又不吭声了。
\par 游廊上,瑞德的影子突然爆发出一阵笑声,笑声很轻,声音不大。
\par “这么说,你和卫太太待在一起!这是我遇到过的最最奇怪的怪事了!”
\par “我倒觉得一点也不奇怪。”她不安地回答说,马上警觉起来。
\par “不奇怪?若这样你就没有个性了。一段时间以来,我有这样的印象,你几乎容忍不了卫太太。你认为她又傻又笨,你对她的爱国热情也感到很厌烦。只要能够用言语诋毁她,你是极少会放弃这种机会的。所以,你居然会在这种炮轰时期做出这么无私的选择,跟她待在一起,我自然就会觉得奇怪啰。好了,跟我说说,你为什么要这么做?”
\par “因为她是查理的姐姐——对我也像个姐姐一样。”思嘉这么回答他,尽量维护着自己的尊严,虽然双颊已经在微微发红了。
\par “你意思是说,是因为她是卫希礼的寡妇?”
\par 思嘉猛地站起身来,尽力克制着自己的愤怒。
\par “我本来差点就要原谅你原来的粗鲁行为了,可现在我做不到了。如果我不是心情不好的话,我本来是不会让你站在这游廊上的,而且——”
\par “坐下坐下,把你那弄皱的皮衣弄平整一些,”他说,声音变了。他伸手,拉住她的手,把她拉回椅子上坐下。“你干嘛这么闷闷不乐呢?”
\par “噢,我今天收到塔拉来的一封信。北方佬离我家已经很近了,我小妹又患了伤寒,而且——而且——所以,现在的情况是,即使我能回家,说真的,我也真想回家,但我妈妈也不会让我回了,她担心我也会患上伤寒。噢,天哪,我真的是太想回家了!”
\par “好了,别为这哭了,”他说,可声音友善多了,“即使北方佬真的来了,你在亚特兰大也比在塔拉安全得多。北方佬不会伤害你,而伤寒却会伤害你。”
\par “北方佬不会伤害我!你怎么能说这种谎话呢?”
\par “我亲爱的姑娘,北方佬不是魔鬼。他们不像你认为的那样头上长角身上长刺。他们和南方人也很相像——只是言谈举止更差一些而已。当然,口音也很可怕。”
\par “可是,北方佬会——”
\par “强奸你?我认为不会。当然,他们也想这么做。”
\par “你再说这么难听的话,我就要进屋去了。”她叫了起来,暗影掩饰了她发红的双颊,她为此颇为欣慰。
\par “坦率一点。那难道不是你刚才在想的吗?”
\par “噢,当然不是!”
\par “噢,可是偏偏就是!我看透你的心思,你却对我生气,那没用的。那正是我们这些娇生惯养、心灵纯洁的南方夫人小姐们所想的。她们头脑里一直有这种念头。我敢打赌,连梅里韦瑟太太这样的寡妇……”
\par 思嘉无声地张大了嘴巴。她记得,在这非常时期,只要两三个年长妇女聚在一起,她们就在嘀咕这种事,总是在弗吉尼亚、田纳西或是路易斯安那,离家近的地方倒从来没听说过。北方佬强奸妇女、用刺刀挑开孩子的肚子、在老人的头顶上放火烧房子——虽然她们没有在街头巷尾大喊大叫,但每个人都知道这些事是真的。如果瑞德还算个正派人,他就应该意识到这些事都是真的,而且不该谈这些。这根本就不是什么好笑的事。
\par 她听见他在轻声发笑。有时候他真是可恶。其实,大多数时候他都很可恶。一个男人知道女人真正在想什么、说什么,那是太可怕了。这会使一个姑娘觉得自己好像被剥光衣服、赤身裸体一样。从好女人那里,男人是绝对不会知道这些东西的。他看穿了她的心思,她很生气。她喜欢认为自己对男人来说是个谜。可她知道,瑞德却认为她就像玻璃一样透明。
\par “说到这些事,”他继续说下去,“你屋里有没有保护人或是年长妇女什么的?令人钦佩的梅里韦瑟太太或是米德太太?她们老是看着我,好像我到这来就是没安好心似的。”
\par “米德太太晚上经常过来,”思嘉回答着,话题改变了,她感到很高兴。“可她今晚来不了了。她的儿子菲尔回家了。”
\par “这太幸运了,”他轻声说道,“只有你一个人在这!”
\par 他声音里有某种东西使她的心跳都加快了,为此也感到很兴奋。她觉得自己脸红了。她经常听到男人声音里的这种口吻,知道这就意味着要宣布对她的爱了。噢,这多有趣啊!只要他说出他爱她,那就等着瞧,看她怎么收拾他。这过去的三年中,他对她说过那么多讽刺挖苦的话,她现在可以和他算算总账了。她要诱使他对她展开攻势,却让他徒劳无功,陷入困境。那天,他看到她甩了希礼一耳光,她甚至要为那一幕雪耻。然后,她再柔情地告诉他,她只能做他的妹妹,再用战争这些冠冕堂皇的理由作为借口,抽身而出。她不安地笑了,心里却在欢唱,在期待。
\par “别笑,”说着,他拉过她的手,把它翻过来,双唇便紧紧地吻在手掌上。他温暖的嘴唇一吻上她的手,某种充满活力、电流般的感觉便从他体内传到她身上,她周身都被这种令人战栗的感觉环抱住了。他的嘴唇移到了她的手腕上,她知道,随着她心跳的加快,他一定感觉到了她脉搏的跳动。她试图抽出自己的手,但她没有成功——这股危险而温馨的感觉使她真想用手捋着他的头发,感觉一下他的嘴唇吻在自己嘴上的感觉。
\par 她并没有爱上他,她慌乱地对自己说。她爱的是希礼。可又如何解释这种使她双手发抖,肚脐发凉的感觉呢?
\par 他轻声笑了。
\par “别把手抽出去!我不会伤害你的!”
\par “伤害我?我可不怕你,白瑞德,也不怕任何穿皮鞋的男人!”她叫了起来,声音发颤、双手发抖,她为此感到很恼怒。
\par “多令人钦佩的观点呀,可还是请你小点声。卫太太会听见的。请你稳定一下自己的情绪。”听上去他对她的慌乱感到很高兴。
\par “思嘉,你是喜欢我的,对不对?”
\par 这倒更像是她所期待的。
\par “哦,有时候是,”她小心翼翼地回答着,“在你的举止不像流氓的时候。”
\par 他又笑了,把她的手心放在自己硬邦邦的面颊上。
\par “我认为,你喜欢我。正是因为我是个流氓。在你备受呵护的生活中,你认识的十足的流氓太少了,所以这点差别对你有着神奇的魅力。”
\par 这可不是她所期待的话,她又试图抽出自己的手,却没有成功。
\par “那不是真的!我喜欢好人——你可以指望他永远是绅士的男人。”
\par “你指的是可以永远让你欺负的男人。这只是定义不同。但不是什么问题。”
\par 他又吻了吻她的手心,她脖子后背的皮肤又激动地战栗了。
\par “可你确实喜欢我。你能不能爱我呢,思嘉?”
\par “啊!”她得意洋洋地想,“现在我可逮住他了!”她考虑了一下,冷淡地回答说:“真的不行。就是说——除非你彻底改变一下你的行为举止,否则不行。”
\par “可我不打算改。这么说你就不能爱我啰?那正是我所希望的。因为,虽然我非常喜欢你,可我不爱你。若让你两次为这种没有回报的爱受罪,那确实也太可悲了,对不对,亲爱的?我能叫你‘亲爱的’吗,韩太太?不管你喜欢不喜欢,我都要叫你‘亲爱的’,所以这不是什么问题,可还是要得体一点。”
\par “你不爱我吗?”
\par “不,确实不爱。你希望我爱你吗?”
\par “别这么自以为是!”
\par “你希望过!哎呀,让你的希望破灭了!我应该爱你的,因为你很迷人,在很多毫无用处的方面又很有才能。可许多女人也都有魅力,也创下了很多伟业,可还是跟你一样没用。不,我不爱你。可我确实非常非常喜欢你——因为你的良心有弹性,因为你很少费心去掩饰你的私心,还因为你身上那种精明的实用主义,恐怕这点你是从某个辞世不久的爱尔兰农民祖先那继承下来的。”
\par 农民!好呀,他在侮辱她!她气急败坏地张嘴要说什么,却一个字也说不出来。
\par “别打断我,”他请求道,捏了捏她的手,“我喜欢你,是因为我自己身上也同样有那些特点,相同的特点就导致了喜欢。我发现你对神圣的、有着木鱼脑袋的卫先生仍然保留着美好的回忆,而他这六个月中很可能已经躺在坟墓里了。可你的心里也应该有我的位置。思嘉,别再动来动去了!我向你声明。自第一次在十二棵橡树看到你,我就一直想要你。那时你正在对可怜的韩查理施展魅力呢。我想要你的欲望比想要任何女人的欲望都更强——而我等你的时间也比等任何女人的时间都更长。”
\par 他最后说的话使她大吃一惊,连气也喘不过来了。虽然他一再侮辱她,可他确实很爱她,然而他却采取了截然相反的举动,不想坦率地用言语表达出来,因为担心她会笑话他。得,她得教教他,马上报复他一下。
\par “你是在求我跟你结婚吗?”
\par 他放开她的手,大笑起来,搞得坐在椅子上的她不禁往后缩了缩身子。
\par “我的上帝,绝不是!我不是告诉过你,我不是个适合结婚的人吗?”
\par “可是——可是——什么——”
\par 他站起身来,手放在胸口,滑稽地鞠了一躬。
\par “亲爱的,”他平静地说,“我一次也没有引诱过你,在这种情况下,我要你做我的情妇,借此赞美一下你的聪明才智。”
\par 情妇!
\par 她从心里喊出了这个字眼,呐喊着自己被卑鄙地侮辱了。但她虽然万分惊讶,但在最初一刹那,她并没有感到受了侮辱。他居然认为她是个傻瓜,她只觉得愤怒得不得了。如果他给她提供的是这样的位置,而不是她所期待的求婚,那他一定认为她是个傻瓜。愤怒、被挫败的虚荣心和破灭的希望使她的头脑一片混乱。还没想好用哪些合乎道德的理由来申斥他,她便脱口而出:
\par “情妇!那除了变成那群贱货,我还能变成什么?”
\par 接着,她意识到自己说了什么,不禁惊愕地拉长了脸。他笑得都噎住了,偷偷窥视着坐在暗影中的她。她已经惊得哑口无言,用手帕盖住了嘴巴。
\par “这就是我为什么喜欢你的缘故!你是我认识的女人中唯一一个坦率的人,唯一一个用实用的眼光看问题而不会用有关有罪和道德这些大话来遮盖问题实质的女人。其他任何一个女人都会晕过去,然后让我走人。”
\par 思嘉跳了起来,羞得满脸通红。她怎么能说出这种话来呢!她,埃伦的女儿,是有教养的人,她怎么能坐在那听着这种贬低人的话,接着又做出如此不知廉耻的回答呢?她应该尖叫出来。她应该昏厥过去。她应该默默地、冷淡地转过身,迅速从游廊上跑掉。可现在来不及了!
\par “我会让你走人的,”她叫了起来,也顾不上媚兰或是街那头的米德一家是否能听到她的叫声了。“滚出去!你怎么敢对我说这种话!我做了什么怂恿你这么做了吗——让你以为——滚出去,别再到这来了。这次我是认真的。别再拿着你那些没用的饰针和丝带到这来,以为我会原谅你。我要——我要告诉我父亲,他会宰了你!”
\par 他抓起帽子,行了个礼。在灯光中,她看到他在笑,髭须下的牙齿也露了出来。他一点也不会不好意思,她说的话只让他觉得很有趣,他正兴趣盎然地看着她呢。
\par 噢,他简直太可恶了!她猛地转过身,朝屋里走去。她抓住门把,很想砰的一声把门关上,可让门固定开着的门钩太沉了,她拉不动。她用力拉着,弄得气喘吁吁的。
\par “要我帮忙吗?”他问道。
\par 她觉得,自己要是再在此地多待一分钟,她的某根血管就会破裂。她匆匆忙忙冲上楼去。到了楼上,她还听到他礼貌地为她关上了门。

\subsubsection{第二十章}

\par 八月炎热、喧嚣的日子已进入尾声,炮击也突然停止了。降临在城市上空的这种宁静真是令人大吃一惊。邻居们在街上碰面时面面相觑,心里都极为不安,不敢肯定会发生什么事。在喧哗吵闹的日子过后,这种宁静并没有使紧张的神经松弛下来,反而是一有可能就使神经变得更加紧张。谁也不知道为什么北方佬的炮火停息下来了;也没有部队的消息,只知道他们大批撤出城周围的防御工事,开到南部去保护铁路。谁也不知道仗在什么地方打,也不知道是不是真的在打仗,如果有战争,那战争又是怎么打的。
\par 现如今,只有口头传来传去的消息。由于纸张、墨水和人手都很缺,自围城开始以后,报纸已经暂时停止发行了。而那些传得最快的小道消息也不知是从哪里冒出来的,迅速传遍了全城。现在,在这令人焦虑的宁静当中,人群聚集在胡德将军的司令部前,要求知道消息,还有大量的人集中在电报局和车站,希望得到消息,得到好的消息,因为每个人都希望,舍曼的大炮沉默下来意味着北方军已经全线撤退,南方军正在把他们沿路赶回多尔顿去。可是,什么消息也没有。电报线静悄悄的,唯一残存的铁路是通往南部的,可那铁路线上也没有火车来,邮电服务已经中断了。
\par 尘土飞扬、热得令人透不过气来的秋天悄悄来临了,突然安静下来的城市像要窒息了一样。天气干燥得令人气喘吁吁,这种负担又压在了人们疲惫不堪、焦虑万千的心灵上。思嘉很希望能收到塔拉的来信,想得都快要疯了,却还要撑着一副勇敢的面孔。自围城开始以来她一直生活在隆隆的炮声中,这似乎已经永恒不变,直到这不祥的宁静降临为止。然而,离围城开始的日子仅仅才三十天。围城围了三十天!城周围挖了一圈圈的红土散兵壕,大炮那单调的隆隆声从不停息,救护车和牛车排成长龙,朝医院开去,鲜血一滴滴滴落在尘土飞扬的街上。掩埋队的工作已经超负荷,不等死去的士兵尸体凉透,他们就把尸体拖出来,像扔木头一样把他们扔进一排排望不到尽头的浅浅的沟里去。仅仅才过了三十天!
\par 从北方军从多尔顿往南进军开始算,仅仅才四个月!仅仅四个月!思嘉回想着那遥远的日子,心想那是发生在另一种生活中的事。噢,不!当然不止四个月。简直像过了一辈子。
\par 四个月前!哦,四个月前,多尔顿、里萨卡和肯纳索山对她来说都还只是铁路沿线的地名。可现在都是战役名了——是约翰斯顿往亚特兰大撤退途中拼死作战却徒劳无功的战役。现在,桃树溪、迪凯特、埃泽拉教堂及尤托伊溪也不再是令人赏心悦目的地方,不再是令人愉悦的地名。它们曾经是宁静的小山村,那里挤满了热情好客的朋友们。她曾经和英俊的军官们在溪水缓缓而流的溪岸上野餐,那里土质松软,绿树成荫。可是现在,她再也不会把它们当成美好之处,这些地名也都成了战役名,她曾经坐过的松软碧绿的草地已被大炮轮子碾得粉碎,被短兵相接、刺刀相见的士兵们拼死作战时踩得稀巴烂,也被枪弹打得痛苦不堪的尸体压扁了……现在,慵懒的河水更红了,佐治亚的红土曾经使它成了红色的河流,但现在比以往任何时候都更红。人们都说,自从北方佬渡过桃树溪后,溪水便变成猩红色的了。桃树溪、迪凯特、埃泽拉教堂、尤托伊溪,它们不再是地名了,而是埋着友人的坟墓,在乱丛林和浓密的树荫下,未被掩埋的尸体在那里腐烂发臭,它们也成了亚特兰大城的四条边线。舍曼曾试图强攻进来,但胡德的部队顽强地把他们击退了。
\par 终于,南部传来消息,传到这紧张兮兮的城里来,可这消息却使人惊恐万分,对思嘉来说更是这样。舍曼将军又在试着进攻该城的第四条边线了,正在攻打琼斯伯勒的铁路线。现在,北方军大量集结在该城的第四条边线上,不再是小打小闹的部队或是特遣骑兵部队,而是大规模的北方部队。成千上万的南方军只得从城附近的防线撤走,准备迎头抵抗。这就是为什么炮火突然停息的原因。
\par “为什么是琼斯伯勒呢?”一想到塔拉离琼斯伯勒那么近,恐怖便抓住了思嘉的心。“他们为什么总要攻打琼斯伯勒呢?他们为什么不找个别的地方去攻打铁路线呢?”
\par 她已经有一个星期没有收到塔拉的来信了,而嘉乐上次捎给她的简短字条更是增添了她的恐惧。卡丽恩的病情已经恶化,现在已是病入膏肓了。可现在等邮件来还得好几天,要过好几天,她才能知道卡丽恩到底还活着还是已经离开人世。噢,要是她在围城一开始时就回家就好了,管他有没有媚兰!
\par 琼斯伯勒在打仗——这是大多数亚特兰大人都知道的,可仗打得怎么样,那就没有一个人能说得出来了。最最没有根据的传闻噬咬着全城人的心。终于,一个从琼斯伯勒来的传令兵带来了令人放心的消息,说是北方军被击退了。但是琼斯伯勒被打开了一个缺口。他们撤退前焚烧了车站,切断了电报线,破坏了三英里长的铁轨。工兵部队疯也似的忙着修复铁路,但这得花好一段时间,因为北方军拆了枕木,用它们堆营火,把扭曲的铁轨横在火上烧,烧得通红滚烫的,再把它们缠在电线杆上,最后,它们看上去就像是一个个巨型的开塞钻。现在,要重铺铁轨是非常困难的,其实,修复任何铁制品都很困难。
\par 不,北方军还没有到塔拉。给胡德将军送快讯的是同一个传令兵,他对思嘉肯定了这一点。战打完后,他在琼斯伯勒遇见嘉乐,那时他正要启程到亚特兰大来。嘉乐请他带一封信给她。
\par 可爸爸在琼斯伯勒干什么呢?年轻的传令兵在回答时显得颇为不安。嘉乐想找个部队军医和他一块到塔拉去。
\par 思嘉站在屋前的游廊上,沐浴在阳光下,一边向年轻人道谢,说让他费心了,一边便觉得双膝软了下去。如果埃伦的医术治不好卡丽恩,那她一定是快要死了,嘉乐才要去找医生!传令兵走了,扬起了一小片红色的尘土。思嘉颤抖着双手撕开信封,打开嘉乐的信。现在,南部邦联的纸张太短缺了,嘉乐的信是写在她上次给他的信的夹缝里的,读起来颇为费劲。
\par “亲爱的女儿,你妈妈和两个姑娘都得了伤寒。她们病得都很重,可我们还是要抱最大的希望。你妈妈躺倒在床上时叫我写信给你,叫你无论如何也不能回家来,以免你和韦德染上这种病。她向你转达她对你的爱意,叫你为她祈祷。”
\par “为她祈祷!”思嘉飞奔上楼,来到自己的房间,跪在床边祈祷着,比以往任何时候都更虔诚。此刻的她不念正规的玫瑰经了,只是一直重复这些话:“圣母啊,别让她死!如果你让她活下去,我一定做个好人!求你了,别让她死!”
\par 接下来整整一个星期,思嘉像只被奴役的动物一样在屋里走来走去,等着消息,一听到马蹄声便惊跳起来,晚上有士兵来敲门时便冲下暗黑的楼梯,可没有任何从塔拉来的消息。横在她和家里的似乎不是区区二十五英里尘土路,而是整块大陆。
\par 邮电系统还是被破坏了,没有人知道南方军在哪里,也没有人知道北方军想干什么。什么也不知道,只知道成千上万的部队,穿灰色军服的也罢,穿蓝色军服的也罢,正在亚特兰大和琼斯伯勒之间的某个地方。整整一个星期,从塔拉没有传来一个字的音信。
\par 思嘉在亚特兰大的医院里见过很多伤寒病人,知道要是得了这种可怕的病,那一个星期意味着什么。埃伦病了,也许正在死去,而思嘉却在亚特兰大孤独无助地守着一个孕妇,在她和自己的家之间还横着两支军队。埃伦病了——也许正在死去。可埃伦不可能生病的!她从来没有生过病。单单生病这个想法就是令人不可置信的,这已威胁到思嘉安稳生活的根基。每个人都会生病,但埃伦从来不生病。埃伦照看别的病人,使他们重新康复。她不可能生病的。思嘉太想回家了。她想回塔拉,那种极度渴望的心情,就像是一个惊恐万分的孩子,疯也似的想到她所知道的唯一一个避难所去。
\par 家!那座不规则地朝四周扩建的白色房子,窗口飘动着白色的窗帘,草坪上长着浓密的苜蓿草,蜜蜂飞来飞去忙活着。屋前台阶上,黑人小孩“嘘嘘”地把鸭子和火鸡从花圃里赶走。宁静的红土地及在阳光下泛着白光的一英里又一英里的棉花!家!
\par “噢,去他妈的媚兰!”她不下千次地想,“她干嘛不跟白蝶姑妈一起去梅肯呢?那才是她该去的地方,去和她的亲戚在一起,而不是和我在一起。我跟她没有血缘关系。她干嘛这么死拖着我?要是她能去梅肯,我早就回家去和妈妈在一起了。即使现在——即使现在,要不是这孩子的话,尽管有北方佬,我还是会找机会回家去的。也许胡德将军会派卫队护送我去。他是个好人,胡德将军,我知道我能让他派卫队护送我去,还会给我一面停战旗,让我通过防线。可我得等这个孩子出生!……噢,妈妈!妈妈!你别死!……为什么这个孩子还不出世呢?我今天得去找找米德医生,问问他有没有什么催生的办法,这样我就可以回家了——如果我能有卫队护送就好了。米德医生说她会难产。亲爱的上帝!假如她死了!媚兰死了。媚兰死了。而希礼——不,我不能这么想,这样不好。但是希礼——不,我不能这么想,因为不管怎么说,他很可能也已经死了。可他要我答应会照顾她。可是——如果我不照顾她,她死了,而希礼却还活着——不,我不能这么想。这是有罪的。我已经向上帝许诺,如果他不让妈妈死,我要做个好人。噢,要是孩子马上出生就好了。要是我能离开此地就好了——回家——到任何地方去,就是不要待在这里。”
\par 这个城市宁静得令人感觉有不祥之兆,思嘉现在恨透了它,可她曾一度喜欢过它。亚特兰大不再是她喜欢过的欢快且欢快得要命的地方。它就像被瘟疫袭击过一样可怕,如此宁静,在围城的喧嚣声过后,变得宁静得很恐怖。噪声当中有兴奋,炮轰当中有危险。可在接下来的宁静中却只有恐怖。整个城市似乎已经魔鬼附身,害怕、忐忑不安及回忆纠缠着它。人们的面孔看上去全都消瘦了,人们能看见的士兵本来就没几个,而思嘉看到的也全都是一脸疲倦,就像是赛跑运动员在已经毫无希望获胜的情况下还在坚持跑完最后一圈似的。
\par 转眼到了八月的最后一天,随之而来的是令人信服的传闻,说是自亚特兰大战役以来最猛烈的战役打响了。是在南边的什么地方。亚特兰大在等着战役的转机,嬉闹、玩笑都停止了。保护亚特兰大的已经只剩下最后一道壕沟。士兵们早在两个星期前就已经知道这个消息,但城里的每个人到现在才知道。如果梅肯的铁路沦陷,亚特兰大也将沦陷。
\par  
\par 九月初的一天清晨,思嘉一醒来便被一种恐惧感包围住了,这使她几乎透不过气来,这种恐惧在她前一天晚上上床睡觉时就已经有了。睡觉睡得她都有点迟钝了:“昨天晚上我上床睡觉时担心的是什么呢?噢,对了,是战争。某个地方在开战,昨天!噢,谁赢了呢?”她迅速翻身而起,揉着眼睛,忧虑的心里重新背上了昨天的负荷。
\par 即使在大清早,空气也很闷热,灼热的天气预示着有个赤日炎炎的中午和明晃晃的蓝天,还有古铜色的太阳无情地当空而照。外边的路上静悄悄的,没有马车经过,也没有部队沉重的脚步走过时扬起的红尘。邻居家的厨房里没有了黑人懒洋洋的声音,也没有了早餐准备好的欢快叫声,因为除了米德太太和梅里韦瑟太太,所有的邻居都逃到梅肯去了。米德太太和梅里韦瑟太太的家里也没有传来任何声响。沿街下去,一度繁忙的商业区悄无声息,许多商店和办公场所都锁了门,关上了门板。它们的主人都在乡下的什么地方,手里还端着步枪呢。
\par 过去的一星期中,每个早晨都宁静得出奇,可今天早晨迎接她的这种宁静,似乎比过去一星期中任何一个早晨都更不吉利。她赶紧起身,不再像往日那样还要先翻来翻去,伸伸懒腰什么的。她来到窗边,希望看到一张邻居的面孔,看到能够鼓舞人心的场面。可路上空荡荡的。她注意到,树上的叶子虽然还是墨绿色的,但很干燥,蒙上了一层厚厚的红色尘土。前院没人伺弄的花草也都枯萎了,一副令人伤心的样子。
\par 她正站在窗前向窗外望去,远处一种声响传到了她的耳朵里,声音很微弱、很沉闷,就像是即将到来的雷雨从远处发出的第一声声响。
\par “雨,”这是她心里闪过的第一个念头,在乡间长大的她接着就想:“我们当然很需要下雨。”可转瞬间,又想:“雨?不!不是雨!是大炮!”
\par 她的心跳加快了,身子探出窗户,竖起耳朵听着远方的隆隆声,试图辨清是从哪个方向传来的。可传来微弱炮声的地方离这太远了,有一会她都没法辨别。“让它从玛丽埃塔传来吧,上帝!”她祈祷着。“或是迪凯特,桃树溪也行。但不要从南面传来!不要从南面传来!”她更紧地抓住窗台,竖起耳朵凝神听着,远方的声响似乎更大声了。是从南面传来的。
\par 南面在开炮!而南面有琼斯伯勒和塔拉——还有埃伦。
\par 此时此刻,北方佬也许已经在塔拉了,就现在!她又侧耳听了听,可耳朵里的血管怦怦直跳,只感觉得到远处的炮火声。不,他们不可能在琼斯伯勒。如果他们到了那么远的地方,声音应该更微弱、更模糊才对。但他们至少应该在离琼斯伯勒十英里远的路上。很可能在拉夫雷迪这些小村落附近。可琼斯伯勒就在拉夫雷迪再过去一点,仅仅十英里多一点。
\par 南面在开炮,炮声很可能就敲响了亚特兰大的丧钟。可对为妈妈的安危忧虑万千的思嘉来说,南面开战就意味着在塔拉附近开战。她在地上走来走去,双手绞在一起,头脑里第一次闪过穿灰色军服的部队可能战败的念头,还有与此相关的一些念头。她之所以会有这个念头,是因为想到舍曼的千军万马离塔拉这么近,这把对战争的所有恐惧都带到她眼前。虽然围城的枪炮震得窗玻璃噼啪作响,虽然缺乏吃的和穿的,但她从来没有过这种恐惧,即使是那一排排没有尽头的垂死的士兵也没有使她如此恐惧过。舍曼的部队离塔拉只有几英里远!即使北方军被打败了,他们也可能沿着到塔拉的路上撤退。而嘉乐手头放着三个生病的女人,他是绝不可能逃走的。
\par 噢,要是她现在在那里就好了,管他有没有北方军。她光着脚在地上走着,睡袍裹着双腿。她走得越久,就越发觉得那是凶兆。她很想回家。她想待在埃伦身边。
\par 她听到从下面的厨房里传来嘎嘎的瓷器声,是普里西在准备早餐,可是没有米德太太的黑奴贝齐的声音。普里西尖声唱着忧郁的小调:“只要再过几天,背着那令人疲惫的包袱……”歌声激怒了思嘉,歌词悲伤的含义使她感到很害怕。她套上一件轻便晨衣,啪嗒啪嗒走进过道,来到后门楼梯口,大喊道:“别再唱了,普里西!”
\par “是的,夫人。”普里西闷闷不乐的声音传到她耳边。她深吸了口气,突然觉得很不好意思。
\par “贝齐到哪去啦?”
\par “俺不知道。她没来。”
\par 思嘉走到媚兰的门边,开了一条缝,向里窥视着。房间里阳光灿烂。媚兰穿着睡衣躺在床上,眼睛闭着,眼圈发黑,心形的脸蛋浮肿,细长的身体可怕地扭曲着。思嘉不怀好意地希望希礼能看到她现在这个样子。她看上去比她所见过的任何孕妇的情况都更糟。她正看着时,媚兰睁开了眼睛,脸上绽开轻柔、温馨的微笑。
\par “进来吧,”她邀请道,笨拙地侧过身来,“太阳一出来我就醒了,一直在想事情。思嘉,我有些事要问你。”
\par 她走进房间,刺目的阳光把床铺照得透亮,她在床边坐了下来。
\par 媚兰伸出手,拉起思嘉的手,温柔地、信任地握着。
\par “亲爱的,”她说,“又开炮了,我很难过。是琼斯伯勒的方向,对不对?”
\par 思嘉“嗯”了一声,那种想法重新出现在脑海里,她的心跳又加快了。
\par “我知道你有多担心。我知道,要不是我的话,上星期你听说你妈妈生病时就会回家了。对不对?”
\par “是的。”思嘉不礼貌地回答说。
\par “思嘉,亲爱的。你对我太好了。没有一个姐妹能像你这么善良,这么勇敢。为此,我爱你。真对不起,我给你添麻烦了。”
\par 思嘉目瞪口呆。爱她,真的吗?这个傻瓜!
\par “思嘉,我一直躺在这想事情,我要请你帮个大忙。”她的手握得更紧了,“如果我死了,你能不能收养我的孩子?”
\par 媚兰的眼睛瞪大了,她声音轻柔,热切地恳求着,两眼炯炯有神。
\par “你会吗?”
\par 思嘉用力抽回自己的手,恐惧袭遍了她的全身,她说话的声音都变粗了。
\par “噢,别傻了,梅利。你不会死的。每个女人在生头胎的时候都以为自己会死。我知道我那时也一样。”
\par “不,你不会的。你从来没害怕过什么事。你这么说只是为了让我振作起来罢了。我倒不怕死,我只是害怕丢下孩子,如果希礼——思嘉,答应我,如果我死了,你会收养我的孩子。那我就不害怕了。白蝶姑妈年纪太大了,不能抚养孩子成人。哈尼和英蒂虽然很好,可是——我想要你抚养我的孩子。答应我,思嘉。如果是个男孩,把他抚养成希礼那样的人,而如果是女孩——亲爱的,我希望她会像你一样。”
\par “真见鬼!”思嘉大叫一声,从床上跳了起来,“事情已经够糟的了,你还在说死?”
\par “对不起,亲爱的。可是,答应我。我想就在今天了。我敢肯定会是今天。请你答应我。”
\par “噢,好的,我答应你。”思嘉说着,茫然不解地低头看着她。
\par 媚兰真的这么傻,真的不知道她有多在乎希礼?还是说她什么都知道,觉得就因为这份爱,思嘉就会好好照顾希礼的孩子?思嘉心里一冲动,真想把这些疑问喊出来,可话到嘴边又忍住了。媚兰这时拉起她的手,在自己的面颊上放了一会。她眼里又恢复了平静的神情。
\par “你为什么认为会是今天呢,梅利?”
\par “从黎明开始,我就有阵痛了——但不是很厉害。”
\par “有阵痛了?哦,你干嘛不叫我?我叫普里西去找米德医生。”
\par “不要,还没有必要那么做,思嘉。你知道他有多忙的,他们大家都很忙。只要给他捎个口信,就说我们今天说不定什么时候会需要他就行了。叫人去叫米德太太,告诉她,叫她到这来,坐在我身边。她会知道什么时候才真的要去找他。”
\par “噢,别再这么大公无私了。你知道你和医院里任何人一样需要医生。我马上叫人去找他。”
\par “不,请别这样。有时候要一整天才能生下来,我只是不想让医生在这瞎坐几小时,而所有那些可怜的小伙子又那么需要他。去叫米德太太就行了。她知道的。”
\par “噢,那好吧。”思嘉说。

\subsubsection{第二十一章}



\subsubsection{第二十二章}



\subsubsection{第二十三章}



\subsubsection{第二十四章}



\subsubsection{第二十五章}



\subsubsection{第二十六章}



\subsubsection{第二十七章}



\subsubsection{第二十八章}



\subsubsection{第二十九章}



\subsubsection{第三十章}



\subsection{第四部}



\subsubsection{第三十一章}



\subsubsection{第三十二章}



\subsubsection{第三十三章}



\subsubsection{第三十四章}



\subsubsection{第三十五章}



\subsubsection{第三十六章}



\subsubsection{第三十七章}



\subsubsection{第三十八章}



\subsubsection{第三十九章}



\subsubsection{第四十章}



\subsubsection{第四十一章}



\subsubsection{第四十二章}



\subsubsection{第四十三章}



\subsubsection{第四十四章}



\subsubsection{第四十五章}



\subsubsection{第四十六章}



\subsubsection{第四十七章}










\subsection{第五部}

\subsubsection{第四十八章}



\subsubsection{第四十九章}




\subsubsection{第五十章}



\subsubsection{第五十一章}



\subsubsection{第五十二章}



\subsubsection{第五十三章}



\subsubsection{第五十四章}



\subsubsection{第五十五章}



\subsubsection{第五十六章}



\subsubsection{第五十七章}



\subsubsection{第五十八章}



\subsubsection{第五十九章}



\subsubsection{第六十章}



\subsubsection{第六十一章}



\subsubsection{第六十二章}



\subsubsection{第六十三章}



\subsection{作者大事略}








