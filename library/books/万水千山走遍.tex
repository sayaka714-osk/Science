
\section{万水千山走遍}

\par 作者:三毛
\par 出版社:北京出版集团北京十月文艺出版社
\par 出版时间:2017-03
\par ISBN:9787530214756

\subsection{大蜥蜴之夜\\\small{墨西哥纪行之一}}

\par 当飞机降落在墨西哥首都的机场时,我的体力已经透支得几乎无法举步。长长的旅程,别人睡觉,我一直在看书。
\par 眼看全机的人都慢慢地走了,还让自己绑在安全带上。窗外的机场灯火通明,是夜间了。
\par 助理米夏已经背着他的东西在通道边等着了。经过他,没有气力说话,点了一点头,然后领先出去了。
\par 我的朋友约根,在关口里面迎接,向我高举着手臂。我走近他,先把厚外套递过去,然后双臂环向他拥抱了一下。他说:“欢迎来墨西哥!”我说:“久等了,谢谢你!”
\par 这是今年第四次见到他,未免太多了些。
\par 米夏随后来了,做了个介绍的手势,两人同时喊出了彼此的名字,友爱地握握手,他们尚在寒暄,我已先走了。
\par 出关没有排队也没有查行李。并不想做特殊分子,可是约根又怎么舍得不使用他的外交特别派司?这一点,是太清楚他的为人了。
\par 毕竟认识也有十四年了,他没有改过。
\par “旅馆订了没有?”我问。
\par “先上车再说吧!”含含糊糊的回答。
\par 这么说,就知道没有什么旅馆,台北两次长途电话算是白打了。
\par 在那辆全新豪华的深色轿车面前,他抱歉地说:“司机下班了,可是管家是全天在的,你来这儿不会不方便。”
\par “住你家吗?谁答应的?”改用米夏听不懂的语言,口气便是不太好了。
\par “要搬明天再说好吗?米夏也有他的房间和浴室。你是自由的,再说,我那一区高级又安静。”
\par 我不再说什么,跨进了车子。
\par “喂!他很真诚啊!你做什么一下飞机就给人家脸色看?”米夏在后座用中文说。
\par 我不理他,望着窗外这一千七百万人的大城出神,心里不知怎么重沉沉的。
\par “我们这个语文?”约根一边开车一边问。
\par “英文好啰!说米夏的话。”
\par 说是那么说,看见旁边停了一辆车,车里的小胡子微笑着张望我,我仍是忍不住大喊出了第一句西班牙文——“晚安啊!我的朋友——”
\par 这种令约根痛恨的行径偏偏是我最爱做的,他脸上一阵不自在,我的疲倦却因此一扫而空了。
\par 车子停在一条林荫大道边,门房殷勤地上来接车,我们不必自己倒车入库,提着简单的行李向豪华的黄铜柱子的电梯走去。
\par 约根的公寓,他在墨西哥才安置了半年的家,竟然美丽雅致高贵得有若一座博物馆,森林也似的盆栽,在古典气氛的大厅里,散发着说不出的宁静与华美。
\par 米夏分配到的睡房,本是约根的乐器收藏室,里面从纸卷带的手摇古老钢琴、音乐匣、风琴,到全世界各地大大小小的各种古古怪怪可以发声音的东西,都挂在墙上。
\par 我被引着往里面走,穿过一道中国镶玉大屏风,经过主卧室的门外,一转弯,一个客房藏着,四周全是壁柜,那儿,一张床,床上一大块什么动物的软毛皮做成的床罩静静地等着我。
\par “为什么把我安置在这里?我要米夏那间!”
\par 我将东西一丢,喊了起来。
\par “别吵!嘘——好吗?”约根哀求似的说。
\par 心里一阵厌烦涌上来,本想好好对待他的,没有想到见了面仍是连礼貌都不周全,也恨死自己了。世上敢向他大喊的,大概也只有我这种不买账的人。
\par “去小客厅休息一下吗?”约根问。
\par 我脱了靴子,穿着白袜子往外走,在小客厅里,碰到了穿着粉红色制服、围洁白围裙的墨西哥管家。
\par “啊!您就是苏珊娜,电话里早已认识了呀!”
\par 我上去握住她的手,友爱地说着。
\par 她相当拘谨,微屈了一下右脚,说:“请您吩咐——”
\par 约根看见我对待管家不够矜持,显然又是紧张,赶快将苏珊娜支开了。
\par 我坐下来,接了一杯威士忌,米夏突然举杯说:“为这艺术而舒适的豪华之家——”
\par 对于这幢公寓的格调和气派,米夏毫不掩饰他人全然的沉醉、迷惑、欣赏与崇拜。其实这并没有什么不对,公平地说,这房子毕竟是少见的有风格和脱俗。而米夏的惊叹却使我在约根的面前有些气短和不乐。
\par “阿平,请你听我一次话,他这样有水准,你——”米夏忍不住用中文讲起话来。
\par 我假装没有听见,沉默着。正是大梦初醒的人,难道还不明白什么叫做盖世英雄难免无常,荣华富贵犹如春梦吗?
\par 古老木雕的大茶几上放着我的几本书,约根忙着放《橄榄树》给我们听。这些东西不知他哪里搞来的,也算做是今夜的布景之一吧,不知我最厌看的就是它们。
\par 波斯地毯,阿拉伯长刀,中国锦绣,印度佛像,十八世纪的老画,现代雕塑,中古时代的盔甲,锡做的烛台、银盘、铜壶——没有一样不是精心挑选收集。
\par “收藏已经不得了啦!”我说,衷心地叹了口气。
\par “还差一样——你猜是什么?”他笑看着我,眼光中那份收藏家的贪心也掩饰不住了。
\par 刚刚开始对他微笑的脸,又刷一下变了样子。
\par 我叹了口气,坐在地毯上反手揉着自己的背,右肩酸痛难当,心里一直在对自己说:“我试了!试了又试!再没有什么不好交代的,住两日便搬出去吧!”
\par 约根走去打电话,听见他又叫朋友们过来。每一次相聚,他总是迫不及待地拿我显炫给朋友们看,好似一件物品似的展览着。
\par 米夏紧张地用中文小声说:“喂!他很好,你不要又泄气,再试一次嘛!”
\par 我走开去,将那条苍苍茫茫的《橄榄树》啪一下关掉,只是不语。
\par 旅程的第一站还没有进入情况,难缠的事情就在墨西哥等着。这样的事,几天内一定要解决掉。同情心用在此地是没有价值的。
\par 门铃响了,来了约根的同胞,他们非常有文化,手中捧着整整齐齐的十几本书和打字资料,仔细而又友爱地交给我——全是墨西哥的历史和地理,还有艺术。
\par 我们一同谈了快三小时,其实这些上古和马雅文化,在当年上马德里大学时,早已考过了,并没有完全忘记。为了礼貌,我一直忍耐着听了又听——那些僵死的东西啊!
\par 他们不讲有生命的活人,不谈墨西哥的衣食住行,不说街头巷尾,只有书籍上诉说的史料和文化。而我的距离和他们是那么的遥远,这些东西,不是我此行的目的——我是来活一场的。
\par “实在对不起,米夏是我的助理,这些书籍请他慢慢看。经过二十多小时的飞行,我想休息了!”
\par 与大家握握手,道了晚安,便走了。
\par 米夏,正是见山不是山、见水不是水的年龄,新的环境与全然不同的人仍然使他新鲜而兴奋。留下他继续做听众,我,无法再支持下去。
\par 寂静的午夜,我从黑暗中惊醒,月光直直地由大玻璃窗外照进来。床对面的书架上,一排排各国元首的签名照片静静地排列着,每张照片旁边,插着代表元首那国的小旗子。
\par 我怔怔地与那些伟大人物的照片对峙着,想到自己行李里带来的那个小相框,心里无由地觉着没有人能解的苍凉和孤单。
\par 墨西哥的第一个夜晚,便是如此张大着眼睛什么都想又什么都不想地度过了。
\par   
\par 早晨七点钟,我用大毛巾包着湿头发,与约根坐在插着鲜花、阳光普照的餐厅里。
\par 苏珊娜开出了丰丰富富而又规规矩矩的早餐,电影似的不真实——布景太美了。
\par “不必等米夏,吃了好上班。”我给约根咖啡,又给了他一粒维他命。
\par “是这样的,此地计程车可以坐,公共汽车对你太挤。一般的水不可以喝,街上剥好的水果绝对不要买,低于消费额五十美金的餐馆吃了可能坏肚子,路上不要随便跟男人讲话。低级的地区不要去,照相机藏在皮包里最好,当心人家抢劫——”
\par “城太大了,我想坐地下车。”我说。
\par “不行——”约根叫了起来,“他们强暴女性,就在车厢里。”
\par “白天?一千七百万人的大城里?”
\par “报上说的。”
\par “好,你说说,我来墨西哥是做什么的?”
\par “可以去看看博物馆呀!今天早晨给自己去买双高跟鞋,这星期陪我参加宴会,六张请帖在桌上,有你的名字——”
\par 我忍住脾气,慢慢涂一块吐司面包,不说一句伤人的话。那份虫噬的空茫,又一次细细碎碎地爬上了心头。
\par 约根上班之前先借了我几千披索,昨日下机没来得及去换钱。这种地方他是周到细心的。
\par 推开米夏的房间张望,他还睡得像一块木头,没有心事的大孩子,这一路能分担什么?
\par 为什么那么不快乐?右肩的剧痛,也是自己不肯放松而弄出来的吧!
\par 苏珊娜守礼而本分,她默默地收桌子,微笑着,不问她话,她不主动地说。
\par “来,苏珊娜,这里是三千披索,虽说先生管你伙食费,我们也只在这儿早餐,可是总是麻烦您,请先拿下了,走的时候另外再送你,谢谢了!”
\par 对于这些事情,总觉得是丰丰富富地先做君子比较好办事,虽说先给是不礼貌的,可是,这世界上,给钱总不是坏事。
\par 苏珊娜非常欢喜地收下了。这样大家快乐。
\par  
\par “那我们怎么办?照他那么讲,这不能做,那又不能做?”
\par 米夏起床吃早餐时我们谈起约根口中所说的墨西哥。
\par “低于五十美金一顿的饭不能吃?他土包子,我们真听他的?”我笑了。
\par “你不听他的话?他很聪明的。”米夏天真地说。
\par “认识十四年了,也算是个特殊的朋友,有关我半生的决定,他都有过建议,而我,全没照他的去做过——”我慢慢地说。
\par “结果怎么样?”米夏问。
\par “结果相反的好。”我笑了起来。
\par “昨天晚上,你去睡了,约根说,他想拿假期,跟我们在中美洲走五个星期,我没敢讲什么,一切决定在你,你说呢?”米夏问。
\par 我沉吟了一下,叹了口气:“我想还是一个人走的好,不必他了,真的——”
\par “一个人走?我们两人工作,你却说是一个人,我问你,我算谁?”
\par “不知道,你拍你的照片吧!真的不知道!”
\par 我离开了餐厅去浴室吹头发,热热的人造风一阵又一阵闷闷地吹过来。
\par 米夏,你跟着自然好,如果半途走了,也没有什么不好。毕竟要承当的是自己的前程和心情,又有谁能够真正地分担呢?
\par 
\par 住在这个华丽的公寓里已经五天了。
\par 白天,米夏与我在博物馆、街上、人群里消磨,下午三点以后,约根下班了,我也回去。他要伴了同游是不答应的,那会扫兴。
\par 为着台北一份译稿尚未做完,虽然开始了旅程,下午仍是专心地在做带来的功课。
\par 半生旅行飘泊,对于新的环境已经学会了安静地去适应和观察,并不急切于新鲜和灿烂,更不刻意去寻找写作的材料。
\par 这对我来说,已是自然,对于米夏,便是不同了。
\par “快闷死了,每天下午你都在看译稿,然后晚上跟约根去应酬,留下我一个人在此地做什么?”米夏苦恼地说。
\par “不要急躁,孩子,旅行才开始呢,先念念西班牙文,不然自己出去玩嘛!”我慢慢地看稿,头也不抬。
\par “我在笼子里,每天下午就在笼子里关着。”
\par “明天,译稿弄完了,寄出去,就整天出去看新鲜事情了。带你去水道坐花船,坐公车去南部小村落,太阳神庙、月神庙都去跑跑,好吗?”
\par “你也不只是为了我,你不去,写得出东西来吗?”米夏火起来了。
\par 我笑看着这个名为助理的人,这长长的旅程,他耐得住几天?人生又有多少场华丽在等着?不多的,不多的,即使旅行,也大半平凡岁月罢了。米夏,我能教给你什么?如果期待得太多,那就不好了啊!
\par 认真考虑搬出约根的家到旅馆去住,被他那么紧迫钉人并不算太难应付,只是自己可能得到的经验被拘束在这安适的环境里,就未免是个人的损失了。
\par 决定搬出去了,可是没有告诉米夏,怕他嘴不紧。约根那一关只有对不起他,再伤一次感情了。
\par 才五天,不要急,匆匆忙忙地活着又看得到感得了什么呢!
\par 
\par 不是为了这一夜,那么前面的日子都不能引诱我写什么的,让我写下这一场有趣的夜晚,才去说说墨西哥的花船和街头巷尾的所闻所见吧!
\par 不带米夏去参加任何晚上的应酬并没有使我心里不安。他必须明白自己的职责和身份,过分地宠他只有使他沿途一无所获。
\par 再说,有时候公私分明是有必要的,尤其是国籍不一样的同事,行事为人便与对待自己的同胞有些出入了。
\par 
\par 那一夜,苏珊娜做了一天的菜,约根在家请客,要来十几个客人,这些人大半是驻在墨西哥的外交官们,而本地人,是不被邀请的。
\par 约根没有柔软而弹性的胸怀。在阶级上,他是可恨而令人瞧不起的迂腐。奇怪的是,那么多年来,他爱的一直是一个与他性格全然不同的东方女孩子。这件事上怎么又不矛盾,反而处处以此为他最大的骄傲呢?
\par 再大的宴会,我的打扮也可能只是一袭白衣,这样的装扮谁也习惯了,好似没有人觉得这份朴素是不当的行为。我自己,心思早已不在这些事上争长短,倒也自然了。
\par 当我在那个夜晚走进客厅时,已有四五位客人站着坐着喝酒了。他们不算陌生,几个晚上的酒会,碰来碰去也不过是这几张面孔罢了。
\par 男客中只有米夏穿着一件淡蓝的衬衫,在那群深色西装的中年人里面,他显得那么的天真、迷茫、兴奋而又紧张。冷眼看着这个大孩子,心里不知怎的有些抱歉,好似欺负了人一样。虽然他自己蛮欢喜这场宴会的样子,我还是有些可怜他。
\par 人来得很多,当莎宾娜走进来时,谈话还是突然停顿了一会儿。
\par 这个女人在五天内已见过三次了,她的身旁是那个斯文凝重给我印象极好的丈夫——文化参事。
\par 她自己,一身银灰的打扮,孔雀似的张开了全部的光华,内聚力极强的人,只是我怕看这个中年女人喝酒,每一次的宴会,酒后的莎宾娜总是疯狂,今夜她的猎物又会是谁呢?
\par 我们文雅地吃东西、喝酒、谈话、听音乐、讲笑话,说说各国见闻。不能深入,因为没有交情。为了对米夏的礼貌,大家尽可能用英文了。
\par 这种聚会实在是无聊而枯燥的,一般时候的我,在一小时后一定离去。往往约根先送我回家,他再转回去,然后午夜几时回来便不知道了,我走了以后那种宴会如何收场也没有问过。
\par 那日因为是在约根自己家中,我无法离去。
\par 其中一个我喜欢的朋友,突然讲了一个吸血鬼在纽约吸不到人血的电影:那个城里的人没有血,鬼太饿了,只好去吃了一个汉堡。这使我又稍稍高兴了一点,觉得这种谈话还算活泼,也忍受了下去。
\par 莎宾娜远远地埋在一组椅垫里,她的头半枕在别人先生的肩上,那位先生的太太拼命在吃东西。
\par 一小群人在争辩政治,我在小客厅里讲话,约根坐在我对面,神情严肃地对着我,好似要将我吃掉一样地又恨又爱地凝视着。
\par 夜浓了,酒更烈了,室内烟雾一片,男女的笑声暧昧而释放了,外衣脱去了,音乐更响了。而我,疲倦无聊得只想去睡觉。
\par 那边莎宾娜突然高叫起来,喝得差不多了:“我恨我的孩子,他们拿走了我的享受,我的青春,我的自由,还有我的身材,你看,你看——”
\par 她身边的那位男士刷一抽身站起来走开了。
\par “来嘛!来嘛!谁跟我来跳舞——”她大嚷着,张开了双臂站在大厅里,嘴唇半张着,眼睛迷迷濛濛,说不出是什么欲望,那么强烈地狂奔而出。
\par 唉!我突然觉得,她是一只饥饿的兽,在这墨西哥神秘的夜里开始行猎了。
\par 我心里喜欢的几对夫妇在这当儿很快而有礼地告辞了。分手时大家亲颊道晚安,讲吸血鬼故事给我听的那个小胡子悄悄拍拍我的脸,说:“好孩子,快乐些啊!不过是一场宴会罢了!”
\par 送走了客人,我走回客厅去,在那个阴暗的大盆景边,莎宾娜的双臂紧紧缠住了一个浅蓝衬衫的身影,他们背着人群,没有声息。
\par 我慢慢经过他们,坐下来,拿起一支烟,正要找火,莎宾娜的先生啪一下给我凑过来点上了,我们在火光中交换了一个眼神,没有说一句话。
\par 灯光扭暗了,音乐停止了,没有人再去顾它。梳妹妹头发、看似小女孩般的另一个女人抱住约根的头,半哭半笑地说:“我的婚姻空虚,我失去了自己,好人,你安慰我吗——”
\par 那边又有喃喃的声音,在对男人说:“什么叫快乐,你说,你说,什么叫快乐——”
\par 客厅的人突然少了,卧室的门一间一间关上了。
\par 阳台不能去,什么人在那儿纠缠拥抱,阴影里,花丛下,什么事情在进行,什么欲望在奔流?
\par 我们剩下三个人坐在沙发上。
\par 一个可亲的博士,他的太太跟别人消失了,莎宾娜的先生,神情冷静地在抽烟斗,另外还有我。
\par 我们谈着墨西哥印地安人部落的文化和习俗,紧张而吃力,四周正在发生的情况无法使任何人集中心神,而我的表情,大概也是悲伤而疲倦了。
\par 我再抽了一支烟,莎宾娜的先生又来给我点火,轻轻说了一句:“抽太多了!”
\par 我不再费力地去掩饰对于这个夜晚的厌恶,哗一下靠在椅垫上,什么也不理也不说了。
\par “要不要我去找米夏?”这位先生问我,他的太太加给他的苦痛竟没有使他流露出一丝难堪,反而想到身边的我。而我对米夏又有什么责任?
\par “不!不许,拜托你。”我拉住他的衣袖。
\par 在这儿,人人是自由的,选择自己的生命和道路吧!米夏,你也不例外。
\par 莎宾娜跌跌撞撞地走进来,撞了一下大摇椅,又扑到一棵大盆景上去。
\par 她的衣冠不整,头发半披在脸上,鞋子不见了,眼睛闭着。
\par 米夏没有跟着出现。
\par 我们都不说话,大家窒息了似的熬着。
\par 其实,这种气氛仍是邪气而美丽的,它像是一只大爬虫,墨西哥特有的大蜥蜴,咄咄地向我们吹吐着腥浓的喘息。
\par 过了不知多久,博士的太太疯疯癫癫地从乐器室里吹吹打打地走出来,她不懂音乐,惊人的噪音,冲裂了已经凝固的夜。一场宴会终是如此结束了。
\par 唉唉!这样豪华而狂乱的迷人之夜,是波兰斯基导演的一场电影吧!
\par 那只想象中的大蜥蜴,在月光下,仍然张大着四肢,半眯着眼睛,重重地压在公寓的平台上,满意地将我们吞噬下去。
\par 
\par 还有两个客人醉倒在洗手间里。
\par 约根扑在他卧室的地毯上睡了。
\par 我小心地绕过这些身体,给自己刷了牙,洗了脸,然后将全公寓的大落地窗都给它们打开来吹风。
\par 拿了头发刷子,一间间去找米夏。
\par 米夏坐在书房的一块兽皮上,手里在玩照相机,无意识地按快门,咔嚓一下,咔嚓又一下,脸上空空茫茫的。
\par 我一面刷头发,一面喊了一声:“徒儿——”
\par “没做什么,真的——”米夏淡淡地说。
\par “这没什么要紧,小事情。”我说。
\par “可是我没有做——”他叫了起来。
\par “如果今夜我不在呢?”我叹了口气。
\par 米夏不响,不答话。
\par “莎宾娜可怜——”他说。
\par “不可怜——”
\par “阿平——你无情——”
\par 我慢慢地梳头发,没有解释。
\par “今夜够受了——”米夏喘了一口大气。
\par “有挣扎?”我笑了。
\par 米夏没有笑,怔怔地点了点头。
\par “没有见识的孩子,要是真的事情来时你又怎么办?”我站起来走开了。
\par “阿平——”
\par “明早搬出去,旅馆已经打电话订了,这一种墨西哥生涯到此为止了,好吗?”我说。
\par 
\par \rightline{一九八一年十一月十五日在墨西哥}

\subsection{街头巷尾\\\small{墨西哥纪行之二}}


\par 这一趟旅行虽说会发生些什么样的事情全然是未知,可是行万里路、读万卷书的时代已经过去了。仍然算是有备而来的。
\par 我的习惯是先看资料,再来体验印证个人的旅行。
\par 这一回有关中南美的书籍一共带了四册,要找一家便宜而位置适中的旅馆也并不是难事,书上统统都列出来了。
\par 来到墨西哥首都第六天,一份叫做“EL HERALDO DE MEXICO”的报纸刊出了我的照片。与写作无关的事情。
\par 那么大的照片刊出来的当日,也是我再梳回麻花辫子,穿上牛仔裤,留下条子,告别生活方式极端不同的朋友家,悄悄搬进一家中级旅馆去的时候了。
\par 旅馆就在市中心林荫大道上,老式的西班牙殖民式建筑,白墙黑窗,朴素而不豪华,清洁实惠,收费亦十分合理,每一个只有冲浴的房间,是七百披索,大约是合二十七元美金一日,不包括早餐。
\par 书上列出来的还有十元美金一日的小旅馆,看看市区地图,那些地段离城中心太远,治安也不可能太好,便也不再去节省了。
\par 助理米夏在语言上不能办事与生活,这一点再再地督促他加紧西班牙文。鼓励他独自上街活动,不可以完全依靠我了。
\par 墨西哥城是一个方圆两百多平方公里,坐落在海拔二千二百四十公尺高地的一个大城市。
\par 初来的时候,可能是高度的不能习惯,右耳剧痛,鼻腔流血,非常容易疲倦,这种现象在一周以后便慢慢好转了。
\par 有生以来没有在一个一千七百万人的大城市内住过,每天夜晚躺在黑暗里,总听见警车或救护车激昂而快速的哀鸣划破寂静的长夜。这种不间断的声音,带给人只有一个大都会才有的巨大的压迫感,正是我所喜欢的。

\subsubsection*{这一张张美丽的脸}
\par 除了第一日搬去旅舍时坐的是计程车之外,所用的交通工具起初还是公共汽车,后来试了四通八达的地下车之后,便再也舍不得放弃了。
\par 大部分我所见的墨西哥人,便如上帝捏出来的粗泥娃娃,没有用刀子再细雕,也没有上釉,做好了,只等太阳晒干,便放到世上来了——当然,那是地下车中最最平民的样子。
\par 这儿的人类学博物馆中有些故事,述说古时在这片土地上的居民,他们喜欢将小孩子的前额和后脑夹起好几年,然后放开,那些小孩子的头变成扁平的,脸孔当然也显得宽大些,在他们的审美眼光中,那便是美丽。
\par 而今的墨西哥人,仍然有着那样的脸谱,扁脸、浓眉、大眼、宽鼻、厚唇,不算太清洁,衣着鲜艳如彩虹,表情木然而本分。而他们身体中除了墨西哥本地的血液之外,当然掺杂了西班牙人的成分,可是看上去他们仍是不近欧洲而更近印地安人的。
\par 常常,在地下车中挤着去某个地方,只因时间充分,也因舍不得那一张张已到了艺术极致的脸谱,情愿坐过了站再回头。
\par 人,有时候是残酷的,在地下车中,看见的大半是贫穷的人,而我,却叫这种不同的亦不算太文明装扮的男女老幼为“艺术的美”,想起来是多么大的讽刺。
\par 墨西哥城内每天大约有五百到两千个乡下人,涌进这个大都市来找生活。失业的人茫茫然地坐在公园和街头,他们的表情在一个旁观者看来,张张深刻,而这些对于饥饿的肚子,又有什么关联?



\subsubsection*{自杀神}
\par 虽说对于参观大教堂和博物馆已经非常腻了,可是据说墨西哥的“国家人类学博物馆”仍然可能是世界上最周全的一座,为了对得起自己的良知,还是勉强去了。
\par 第一次去,是跟着馆内西语导游的。他不给人时间看,只强迫人在馆内快速地走,流水账似的将人类历史尤其墨西哥部分泼了一大场,进去时还算清楚,出来时满头雾水。
\par 结果,又去了第二次,在里面整整一日。虽说墨西哥不是第一流的国家,可是看过了他们那样大气势的博物馆,心中对它依然产生了某种程度的尊敬。
\par 要说墨西哥的日神庙、月神庙的年代,不过是两千多年以前。他们的马雅文化固然辉煌,可是比较起中国来,便不觉得太古老了。
\par 只因那个博物馆陈列得太好,介绍得详尽,分类细腻,便是一张壁画吧,也是丰富。馆内的说明一律西班牙文,不放其他的文字,这当然是事先设想后才做的决定。我仍是不懂,因为参观的大部分是外国人。
\par 古代的神祇在墨西哥是很多的,可说是一个想象力丰富的多神民族。日神、月神、风神、雨神之外,当然还有许许多多不同的神。
\par 也可能是地理环境和天灾繁生,当时的人自然接受了万物有灵的观念,事实上,此种信仰是因为对大自然的敬畏而产生的。
\par 其中我个人最喜欢的是两个神——玉米神和自杀神。
\par 玉米是我爱吃的食物之一,可说是最爱的。有这么一位神,当然非常亲近他。
\par 当我第一次听见导游用棒子点着一张壁画,一个个神数过去,其中他滑过一个小名字——自杀神时,仍是大吃了一惊。
\par 跟着导游小跑,一直请问他古时的自杀神到底司什么职位,是给人特许去自杀,还是接纳自杀的人,还是叫人去自杀?
\par 导游也答不出来,只笑着回了我一句:“你好像对自杀蛮感兴趣的,怎么不问问那些影响力更深、更有神话意义的大神呢?”
\par 后来第二次我自己慢慢地又去看了一次博物馆,专门研究自杀神,发觉他自己在图画里就是吊在一棵树上。
\par 世上无论哪一种宗教都不允许人自杀,只有在墨西哥发现了这么一个书上都不提起的小神。我倒觉得这种宗教给了人类最大的尊重和意志自由,居然还创出一个如此的神,是非常有趣而别具意义的。
\par 墨西哥大神每一个石刻的脸,看痴了都像魔鬼。
\par 这么说实在很对不起诸神,可是他们给人的感应是邪气而又强大的。没有祥和永恒的安宁及盼望。他们是惩罚人的灵,而不是慈祥的神。说实在,看了心中并不太舒服,对于他们只有惧怕。
\par 是否当时的人类在这片土地上挣扎得太艰苦,才产生了如此粗暴面孔的神祇和神话呢!

\subsubsection*{金字塔}
\par 当然,我们不可避免地去了西班牙文中仍叫它“金字塔”的日神庙及月神庙。
\par 据考证那是公元前两百年到公元九百年时陶特克斯人时期的文明。在今天,留下了人类在美洲壮观的废墟和历史。
\par 那是一座古城,所谓的日神月神庙是后人给它们加上去的名称。外在的形式,像极了埃及的金字塔,只是没有里面的通道,亦没有帝王的陵墓。
\par 为了这些不同年代的人类文明和古代城市的建筑,我看了几个夜晚的资料,预备在未去之前对它们做一个深切的纸面上的了解。
\par 然后米夏与我在转车又转车之后,到了那个叫做“阿那乌阿克之谷”(VALLE DE ANAHUAC)的底奥帝乌刚诺的金字塔。
\par 烈日下的所谓金字塔,已被小贩、游览车,大声播放的流行音乐和大呼小叫的各国游客完全污染光了。
\par 日神庙六十四公尺高的石阶上,有若电影院散场般的人群,并肩在登高。手中提着他们的小型录音机,放着美国音乐。
\par 我没有去爬,只是远远地坐着观望。米夏的红衬衫,在高高石阶的人群里依旧鲜明。
\par 那日的参观没有什么心得。好似游客涌去的地方在全世界都是差不多的样子。
\par 当米夏努力在登日神庙顶时,我借了一辆小贩的脚踏车,向着古代不知为何称为“死亡大道”的宽大街道的废墟上慢慢地骑去。
\par 本想在夜间再去一趟神庙废墟的,终因交通的问题,结果没有再回去。
\par 我还是不羞耻地觉得城镇内的人脸比神庙更引人。
\par 至于马雅文化和废墟,计划中是留到洪都拉斯的“哥庞”才去看一看了。

\subsubsection*{吃抹布}
\par 第一次在街头看见路边的小摊子上在烘手掌大的玉米饼时,我非常喜欢,知道那是墨西哥人的主食“搭哥”(taco),急于尝尝它们。
\par 卖东西的妇人在我张开的掌心中啪一下给了一张饼,然后在饼上放了些什么东西混着的一摊馅,我将它们半卷起来,吃掉了,有酱汁滴滴答答地从手腕边流下来。
\par “搭哥”的种类很多,外面那个饼等于是一张小型的春卷皮,淡土黄色的,它们永远不变。
\par 里面的馅放在一只只大锅里,煮来煮去,有的是肉,有的是香肠,有的看不清楚,有的猜不出来。要换口味,便换里面的东西。
\par 在城内,除非是游客区,那儿可以吃中国菜、意大利面食,还有丹麦甜点蛋饼之外,也可以吃“搭哥”。
\par 可是当我们坐车离城去小村落时,除了“搭哥”之外,实在没有别的东西可吃。
\par 在城外几百里的小镇上,当我吃了今生第几十个“搭哥”之后,那个味道和形式,实在已像是一块抹布——土黄色的抹布,抹过了残余食物的饭桌,然后半卷起来,汤汤水水地用手抓着,将它们吞下去。
\par 一个“搭哥”大约合几角到一元五美金,看地区和内容,当然吃一个胃口是倒了,而肚子是不可能饱的。这已是不错了,比较起城内高级饭店的食物,大约是十倍到十五倍价格的差距。虽然我们的经费充足,仍是坚持入境问俗,一路“搭哥”到底。这对助手米夏便是叫苦连天,每吃必嚷:“又是一块小抹布!”
\par 在墨西哥的最后一日,我怕米夏太泄气,同意一起去吃一顿中国饭,不肯去豪华的中国饭店,挑了一家冷清街角的。先点了两只春卷——结果上来的那个所谓“春天的卷子”的东西,竟然怎么看,怎么咬,都只是两只炸过了的“搭哥”。
\par 吃在一般的墨西哥是贫乏而没有文化的。
\par 它的好处是不必筷子与刀叉,用手便可解决一顿生计,倒也方便简单。至于卫不卫生就不能多去想它了。


\subsubsection*{货物大同}
\par 在城内的游客区里,看见美丽而价格并不便宜的墨西哥人的“大氅”,那种西班牙文叫做“蹦裘”(poncho)的衣物。
\par 事实上它们只是一块厚料子,中间开一个洞套进颈子里,便是御寒的好东西了。
\par 我过去有过两三个“蹦裘”,都因朋友喜欢而送掉了。这次虽然看见了市场上有极美丽的,总因在游客出没的地区,不甘心付高价去买它。
\par 下决心坐长途车去城外的一个小镇,在理由上对米夏说的是请他下乡去拍照。事实上我有自己的秘密,此行的目的对我,根本是去乡下找漂亮、便宜,而又绝对乡土的“蹦裘”来穿。
\par 坐公路车颠几百里去买衣服也只有最笨的人——而且是女人,会做的事情,不巧我就有这份决心和明白。
\par 到了一个地图上也快找不到的城镇,看到了又是所谓景色如画的贫穷和脏乱。我转来转去找市场——资料书中所说的当地人的市集,找到了,怪大的一个广场。
\par 他们在卖什么?在卖热水瓶、镜子、假皮的皮夹、搪瓷的锅、碗、盆、杯、完全尼龙的衣服、塑胶拖鞋、原子笔、口红、指甲油、耳环、手镯、项链——
\par 我到处问人家:“你们不卖poncho?怎么不卖poncho?”
\par 得到的答复千篇一律,举起他们手中彩色的尼龙衣服向我叫喊:“这个时髦!这个漂亮!怎么,不要吗?”

\subsubsection*{水上花园}
\par 那是过去的一大片沼泽,而今部分已成了城镇,另外一小部分弯弯曲曲的水道,仍然保存着,成了水上的花园。
\par 本来也是要自己去划船的。星期天的旧货市场出来后计划去搭长途公车。我的朋友约根算准我必然会在星期日早晨的市集里与当地人厮混。他去了,也果然找到了我与米夏。
\par 于是,我们没有转来转去在公车上颠,坐了一辆大轿车,不太开心地去履行一场游客必做的节目。
\par 一条条彩色缤纷的木船内放着一排排椅子,比碧潭的大船又要大了些。墨西哥人真是太阳的儿女,他们用色的浓艳,连水中的倒影都要凝固了。
\par 参考书上说是二十五块美金租一条船,划完两小时的水道。船家看见是大轿车来的外国人,偏说是五十美金,我因不肯接受约根的任何招待,坚持报社付钱,就因如此,自己跑去与人争价格,已经降到四十块美金了,当然可以再减。讲价也是一种艺术,可惜我高尚的朋友十分窘迫,不愿再磨,浪费了报社的钱,上了一条花船。
\par 三个人坐在船中木头似的沉默无聊,我忍不住跑去船尾跟船家说话,这一搭上交情,他手中撑的那支好长的篙跑到我手上来了。
\par 用尽了气力撑长篙,花船在窄窄的水道里跟别的船乱撞,这时我的心情也好转了,一路认真撑下去。
\par 本来没有什么特别的水道,只因也有音乐船,卖鲜花、毯子和食物的小船一路挤着,它也活泼起来。
\par 虽是游客的节目,只因长篙在自己的手中,身份转变成了船家,那份生涯之感便是很不同了。
\par 那一天,我的朋友约根没有法子吃他昂贵的餐馆,被迫用手抓着碎肉和生菜往玉米饼里卷着做“搭哥”吃。买了一大堆船边的小食。当然,船夫也是请了一同分食的。
\par 水上花园的节目,一直到我们回码头,我将粗绳索丢上岸,给船在铁环上扎好一个漂亮利落的水手结,才叫结束。
\par 自己动手参与的事情,即便是处理一条小船吧,也是快乐得很的。奇怪的是同去的两位男士连试撑的兴趣都没有。


\subsubsection*{你们求什么}
\par 又是一个星期天,也是墨西哥的最后一日了。
\par 我跟米夏说,今天是主日,我要去教堂。
\par 来了墨西哥不去“爪达路沛大教堂”是很可惜的事情。据说一五三一年的时候,圣母在那个地方显现三次,而今它已是一个一共建有新旧七座天主教堂的地方了。
\par “爪达路沛的圣母”是天主教友必然知道的一位。我因心中挂念着所爱的亲友,很喜欢去那儿静坐祷告一会儿,求神保佑我离远了的家人平安。
\par 我们坐地下车往城东北的方向去,出了车站,便跟着人群走了。汹汹滔滔的人群啊,全都走向圣母。
\par 新建大教堂是一座现代的巨大建筑,里面因为太宽,神父用扩音机在做弥撒。
\par 外面的广场又是大得如同可以踢足球。广场外,一群男人戴着长羽毛,光着上身,在跳他们古代祭大神的舞蹈。鼓声重沉沉地混着天主教扩音机的念经声,十分奇异的一种文化的交杂。
\par 外籍游客没有了,本地籍的人,不只是城内的,坐着不同形状的大巴士也来此地祈求他们的天主。
\par 在广场及几个教堂内走了一圈,只因周遭太吵太乱,静不下心坐下来祷告。那场祭什么玉米神的舞蹈,鼓得人心神不宁,而人群,花花绿绿的人群,挤满了每一个角落。
\par 我走进神父用扩音机在讲话的新教堂里去。
\par 看见一对乡下夫妇,两人的身边放着一个土土的网篮,想必是远路来的,因为篮内卷着衣服。
\par 这两个人木像一般地跪在几乎已经挤不进门的教堂外面,背着我,面向着里面的圣母,直直地安静地跪着,动也不动,十几分钟过去了,我绕了一大圈又回来,他们的姿势一如当初。
\par 米夏偷偷上去拍这两人的背影,我看得突然眼泪凝眶。
\par 那做丈夫的手,一直搭在他太太的肩上。做太太的那个,另一只手绕着先生的腰,两个人,在圣母面前亦是永恒的夫妻。
\par 一低头,擦掉了眼泪。
\par 但愿圣母你还我失去的那一半,叫我们终生跪在你的面前,直到化成一双石像,也是幸福的吧!
\par 我独自走开去了,想去广场透透气,走不离人群,而眼睛一再地模糊起来。
\par 那边石阶上,在许多行路的人里面,一个中年男人用膝盖爬行着慢慢移过来,他的两只手高拉着裤管,每爬几步,脸上抽筋似的扭动着,我再低头去看他,他的膝盖哪里有一片完整的皮肤——那儿是两只血球,他自己爬破的一摊生肉,牛肉碎饼似的两团。
\par 虽然明知这是祈求圣母的一种方式,我还是吓了一大跳,哽住了,想跑开去,可是完全不能动弹,只是定定地看住那个男人。
\par 在那男人身后十几步的地方,爬着看上去是他的家人,全家人的膝盖都已磨烂了。
\par 一个白发的老娘在爬,一个二十岁左右的青年人在爬,十几岁的妹妹在爬,一个更小的妹妹已经忍痛不堪了,吊在哥哥的手臂里,可是她不站起来。
\par 这一家人里面显然少了一个人,少了那个男子的妻子,老婆婆的女儿,一群孩子的母亲——
\par 她在哪里?是不是躺在医院里奄奄一息?是不是正在死去?而她的家人,在没有另一条路可以救她的时候,用这种方法来祈求上天的奇迹?
\par 看着这一个小队伍,看着这一群衣衫褴褛向圣母爬去的可怜人,看着他们的血迹沾过的石头广场,我的眼泪迸了出来,终于跑了几步,用袖子压住了眼睛。
\par 受到了极大的惊骇,坐在一个石阶上,哽不成声。
\par 那些人扭曲的脸,血肉模糊的膝盖,受苦的心灵,祈求的方式,在在地使我愤怒。
\par 愚蠢的人啊!你们在求什么?
\par 苍天!圣母马利亚,下来啊!看看这些可怜的人吧!他们在向你献活祭,向你要求一个奇迹,而这奇迹,对于肉做的心,并不过分,可是你,你在哪里?圣母啊,你看见了什么?
\par 黄昏了,教堂的大钟一起大声地敲打起来,广场上,那一小撮人,还在慢慢地爬着。
\par 我,仰望着彩霞满天的苍穹,而苍天不语。
\par 这是一九八一年的墨西哥一个星期天的下午。

\subsection{青鸟不到的地方\\\small{洪都拉斯纪行}}
\par 由墨西哥飞到洪都拉斯的航程不过短短两小时,我们已在洪国首都“得古西加尔巴”(Tegucigalpa)的机场降落了。
\par 下飞机便看见掮枪的军人,虽说不是生平第一次经验,可是仍然改不掉害怕制服的毛病。对我,制服象征一种隐藏的权力,是个人所无能为力的。
\par 排队查验护照时,一个军人与我默默地对峙着,凝神地瞪着彼此,结果我先笑了,他这也笑了起来,踱上来谈了几句话,心情便放松了。
\par 那是一个寂寞的海关,稀稀落落的旅客等着检查。
\par 碰到一个美国人,是由此去边境,为萨尔瓦多涌进来的难民去工作的。
\par 当这人问起我此行的目的时,我说只是来做一次旅行,写些所闻所见而已。在这样的人面前,总觉得自己活得有些自私。
\par 我们是被锁在一扇玻璃门内的,查完一个,守门的军人查过验关条,就开门放人。
\par 当米夏与我被放出来时,蜂拥上来讨生意的人包围了我们。
\par 有的要换美金,有的来抢箱子提,有的叫我们上计程车,更有人抱住脚要擦鞋。
\par 生活的艰难和挣扎,初入洪国的国门便看了个清楚。
\par 我请米夏与行李在一起坐着,自己跑去换钱,同时找“旅客服务中心”,请他们替我打电话给一家已在书上参考到的旅馆。
\par 洪都拉斯的首府只有四五家世界连锁性的大旅馆,那儿设备自然豪华而周全。可是本地人的客栈也是可以住的,当然,如果付的价格只是十元美金一个房间的话,也不能期待有私人浴室和热水了。
\par 此地的钱币叫做“连比拉”(Lempira)。这本是过去一个印地安人的大酋长,十六世纪时在一场赴西班牙人的和谈中被杀。而今他的名字天天被洪都拉斯人提起无数次——成了钱币。
\par 两个连比拉是一块美金。
\par 计程车向我要了十二个连比拉由机场进城,我去找小巴士,可是那种车掌吊在门外的巴士只能坐十二个人,已经客满了。于是我又回去跟计程车司机讲价,讲到六个大酋长,我们便上车了。
\par 公元一五〇三年,当哥伦布在洪都拉斯北部海岸登陆时,发现那儿水深,因此给这片土地叫做“洪都拉斯”,在西班牙语中,便是“深”的意思。
\par 并不喜欢用落后或者先进这些字句来形容每一个不同的国家,毕竟各样的民族有他们自己的生活形态与先天不平等的立国条件。
\par 虽然那么说,一路坐车,六公里的行程,所见的洪都拉斯仍是寂寞而哀愁的。
\par 便是这座在印地安语中称为“银立”的三十万人的首都,看上去也是贫穷。
\par 这是中美洲第二大面积的国家,十一万两千八十八平方公里的土地,百分之四十五被群山所吞噬,人口一直到如今还只三百万左右。
\par 洪都拉斯出产蔗糖、咖啡、香蕉、棉花和一点金矿、锡矿,据说牛肉也开始出口了。
\par 我到了旅馆除了一张床之外,完全没有其他的家具。走道上放着一张方桌子,我将它搬了进房,作为日后写字的地方。
\par 米夏说他床上有跳蚤,我去看了一看,毯子的确不够清洁,可是没有看见什么虫,大半是他心理作用。当然,旅馆初看上去是有些骇人。
\par 街上的餐馆昂贵得不合理,想到此地国民收入的比率,这样的价格又怎么生活下去?
\par 走在路上,沿途都是讨钱的人。
\par 初来洪都拉斯的第一夜,喝了浴室中的自来水,大概吃下了大肠菌。这便昏天黑地地吐泻起来,等到能够再下床走路,已是两天之后了。
\par 在旅舍内病得死去活来时,米夏向“马雅商店”的中国同胞去讨了热水,如果不是那壶热水和人参茶救命,大概还得躺两天才站得起来。
\par 三十万人的首都没有什么特别可看的东西,十六世纪初叶它本是一个矿区小镇,到了现在,西班牙殖民式的教堂和建筑仍是存在的,有些街道也仍是石块砌成的。
\par 城内好几家中国饭馆和杂货店,看见自己的同胞无孔不入地在世界各地找生活,即便在洪都拉斯这样贫穷而幽暗的地方,也住了下来,心中总是一阵又一阵说不出的黯然。
\par 这儿纯血的印地安人——马雅的后裔,可说找不到,百分之九十是混血、棕色皮肤的人,只有少数北部海岸来的黑人,在城内和谐地生活着。
\par 虽说整个的山城是杂乱而没有秩序的,可是一般的建筑在灰尘下细看仍是美丽,窄窄的石砌老街,漆得红黄蓝绿有若儿童图画的房子,怎么看仍有它艺术的美。
\par 生活在城市中,却又总觉得它是悲伤而气闷的,也许是一切房舍的颜色太浓而街道太脏,总使人喘不过气来似的不舒服,那和大都市中的灯火辉煌又是两回事了。
\par 洪都拉斯首都的夜,是浓得化不开的一个梦境,梦里幽幽暗暗,走不出花花绿绿却又不鲜明的窄巷,伸手向人讨钱苦孩子的脸和脚步,哀哀不放。
\par 这儿,一种漆成纯白色加红杠的大巴士,满街地跑着。街上不同颜色和形式的公车,川流不息地在载人,他们的交通出人意料的方便快捷。
\par 特别喜欢那种最美的大巴士,只因它取了一个童话故事中的名字——青鸟。
\par 青鸟在这多少年来,已成了一种幸福的象征,那遥不可及而人人向往的梦啊,却在洪都拉斯的街道上穿梭。
\par 我坐在城内广场一条木椅上看地图,那个夜晚,有选举的车辆,插着代表他们党派的旗子大声播放着音乐来来回回地跑,有小摊贩巴巴地期待着顾客,有流落街头的人在我脚旁沉睡,有讨钱的老女人在街角叫唤,更有一群群看来没有生意的擦鞋童,一路追着人,想再赚几个铜板。当然,对面那座大教堂的石阶上,偶尔有些衣着整齐的幸福家庭,正望了弥撒走出来——
\par 就在这样一个看似失落园的大图画里,那一辆辆叫做“青鸟”的公车,慢慢地驶过,而幸福,总是在开着,在流过去,广场上的芸芸众生,包括我,是上不了这街车。
\par “不,你要去的是青鸟不到的地方!”长途总车站的人缓缓地回答我。
\par 计划在洪都拉斯境内跑一千四百公里,工具当然是他们的长途汽车,其实也知道青鸟是不会跑那儿的,因为要去的小城和村落除了当地的居民之外,已经没有人注意它们了。
\par 那是“各马亦阿爪”城中唯一的客栈。
\par 四合院的房子里面一个天井,里面种着花、养着鸡、晒着老板一家人的衣服。小孩在走廊上追逐,女人在扫地煮饭,四个男人戴着他们两边向上卷的帽子围着打纸牌。而我,静静地坐在大杂院中看一本中文书。因为肠炎方愈,第一日只走了不到一百公里,便停住了。
\par 平房天花板的木块已经烂了,小粉虫在房间里不断地落下来。床上没有毯子,白床单上一片的虫,挡也挡不住。
\par “我的床不能睡。”米夏走出房间来说。
\par “可以,晚上睡在床单下面。”我头也没抬地回了一句。
\par 天气仍是怪凉的,这家小客栈坚持没有毯子,收费却是每个房间二十个连比拉,还是落虫如雨的地方,只因他们是这城内唯一的一家,也只有将就了。
\par 问问旅舍里的人第二天计划要去的山谷,一个七八小时车程距离,叫做“马加拉”的印地安人村落,好似没有人知道。他们一直在收听足球赛的转播,舍不得讲话。
\par 小城本是洪都拉斯的旧都,只因当年目前的京城“得古西加尔巴”发现了银矿,人口才往那儿迁移了。
\par 一条长长的大街,几十家小店铺,一座少不了的西班牙大教堂,零零落落的几家饭店,就是城内唯一的风景了。当然,为了应应景,一小间房间,陈列着马雅文物,叫做“博物馆”。
\par 小城一家杂货店的后院给我们找到了。极阴暗的一个食堂。没有选菜的,老妇给了煮烂的红豆,两块硬硬的肉,外加一杯当地土产的黑咖啡,便收六块连比拉,那合三块美金,同吃的还有一位警察,也付一样价格。
\par 虽然报社给的经费足足有余,可是无论是客栈和食堂,以那样的水准来说,仍是太贵了。
\par 照相胶卷在这儿贵得令人气馁,米夏只剩一卷墨西哥带过来的,而我们有三架照相机。
\par 黄昏时我们在小城内慢慢逛着没事做时,看见大教堂里走出一个拿着大串钥匙的老年人,我快步向他跑过去。
\par “来吧!米夏,开心点,我们上塔顶去!”我大喊起来。
\par 老人引着我们爬钟楼,六个大铜钟是西班牙菲利普二世时代送过来的礼物,到如今它仍是小城的灵魂。那个老人一生的工作便是在守望钟楼里度过了。
\par 我由塔边小窗跨出去,上了大教堂高高的屋顶,在上面来来回回地奔跑。
\par 半生以来,大教堂不知进了多少座,在它屋顶上跑着却是第一次。不知这是不是冒犯了天主,可是我猜如果他看见我因此那样的快乐,是不会舍得生气的。毕竟小城内可做的事情也实在不多。
\par 坐小型巴士旅行,初开始时确是新鲜而有趣的事情。十七八岁的男孩算做车掌吊在门外,公路上若是有人招手,车尚没有停稳他就跳了下去,理所当然地帮忙乘客搬货物和行李,态度是那样的热心而自然,拼命找空隙来填人和货,车内的人挤成沙丁鱼,货里面当然另有活着的东西:瘦瘦的猪,两只花鸡。因为不舒服的缘故,那只猪沿途一直号叫。
\par 一对路边的夫妇带了一台炉子也在等车,当然炉子也挤进来了,夫妇两人那么幸福地靠在炉子边,那是天下唯一的珍贵了。
\par 泥沙飞扬的路上,一个女人拿着小包袱在一座泥巴和木片糊成的小屋前下车,里面飞奔出来几个衣衫褴褛的孩子,做母亲的迫不及待地将手中几片薄饼干散了出去。那幅名画,看了叫人心里不知是什么滋味。
\par 这儿是青鸟不到的地方,人们从没有听过它的名字,便也没有梦了。
\par 米夏与我一个村一个镇地走。太贫苦的地方,小泥房间里千篇一律只有一张吊床。窗是一个空洞框框,没有木板更没有玻璃窗挡风。女人和一堆孩子,还有壮年的男人呆呆地坐在门口看车过,神色茫然。他们的屋旁,大半是坡地,长着一棵橘子树,一些玉米秆,不然什么也不长的小泥屋也那么土气又本分地站着,不抱怨什么。
\par 看见下雨了,一直担心那些泥巴做成的土房子要冲化掉,一路怔怔地想雨停。
\par 洪都拉斯的确是景色如画,松林、河流、大山、深蓝的天空、成群的绿草牛羊,在在是一幅幅大气魄的风景。
\par 只是我的心,忘不了沿途那些贫苦居民的脸孔和眼神,无法在他们善良害羞而无助的微笑里释放出来。一路上,我亦是怔怔。
\par 旅行了十天之后,方抵达洪都拉斯与危地马拉的边境。马雅人著名的“哥庞废墟”便在丛林里了。
\par 这一路如果由首都直着转车来,是不必那么多时间的,只因每一个村落都有停留,日子才在山区里不知不觉地流去了。
\par 有生以来第一次,全身被跳蚤咬得尽是红斑,头发里也在狂痒。那么荒凉的村落,能找到地方过夜已是不易,不能再有什么抱怨了。
\par 还是喜欢这样的旅行,那比坐在咖啡馆清谈又是充实多了。
\par 到了镇名便叫“哥庞废墟”的地方,总算有了水和电,也有两家不坏的旅舍,冷冷清清。
\par 我迫不及待地问旅舍的人供不供热水,得到的答复是令人失望的。
\par 山区的气候依旧湿冷,决定不洗澡,等到去了中北部的工业城“圣彼得稣拉”再找家旅馆全身大扫除吧!
\par 这片马雅人的废墟是一八三九年被发现的,当时它们在密密的雨林中已被泥土和树木掩盖了近九个世纪。
\par 据考证,那是公元后八百年左右马雅人的一个城镇。直到一九三〇年,在发现了它快一百年之后,才有英国人和美国人组队来此挖掘、重建、整理。可惜最最完整的石雕,而今并不在洪都拉斯的原地,而是在大英博物馆和波士顿了。
\par 虽然这么说,那一大片丛林中所遗留下来的神庙,无数石刻的脸谱、人柱,仍是壮观的。
\par 在那微雨寒冷的清晨,我坐在废墟最高的石阶顶端,托着下巴,静静地看着脚下古时称为“球场”、而今已被一片绿茵铺满的旷野,幻想一群高大身躯的马雅人正在打美式橄榄球,口中狂啸着满场飞奔。
\par 千古不灭的灵魂,在我专注的呼唤里复活再生。神秘安静布满青苔的雨林里,一时鬼影幢幢。
\par 我捡了一枝树枝,一面打草一面由废墟进入丛林,惊见满地青苔掩盖的散石,竟都是刻好的人脸,枕头般大的一块又一块。艳绿色的脸啊!
\par 一直走到“哥庞河”才停了脚步,河水千年不停地流着,看去亦是寂寞。
\par 米夏没有进入树林,在石阶上坐着,说林里有蛇。竟不知还有其他或许更令他惊怕的东西根本就绕着他,只是他看不见而已。
\par 当我们由“哥庞”到了工业城“圣彼得稣拉”时,我的耐力几乎已快丧失殆尽了。
\par 路面是平滑而大部分铺了柏油的,问题是小巴士车垫的弹簧一只只破垫而出,坐在它们上面,两个位子挤了三个人,我的身上又抱了一个五六岁的女孩子,脚下一只花鸡扭来扭去,怕它软软的身体,拼命缩着腿。这一路,两百四十多公里结结实实的体力考验。
\par 下车路人指了一家近处的旅馆,没有再选就进去了——又是没有热水的,收费十几块美金。
\par 米夏捉了一只跳蚤来,说是他房间的。
\par 本想叫他快走开,他手一松,跳蚤一蹦,到我身上来了,再找不到它。
\par 自从初来洪都拉斯那日得了一场肠炎之后,每日午后都有微烧,上唇也因发烧而溃烂化脓了,十多日来一直不肯收口结疤。
\par 为了怕冷水冲凉又得一场高烧,便又忍住不洗澡,想等到次日去了北部加勒比海边的小城“得拉”再洗。
\par 仔细把脸洗干净,牙也刷了,又将头发梳梳好,辫子结得光光的,这样别人看不出我的秘密。虽然如此,怎么比都觉自己仍是街上最清洁的人。
\par 那一晚,放纵了自己一趟,没有要当地人的食物,去了一家中国饭店,好好吃了一顿。
\par 也是那一晚,做了一个梦,梦中,大巴士——那种叫做青鸟的干净巴士,载了我去了一个棕榈满布的热带海滩,清洁无比的我,在沙上用枯枝画一个人的名字。画着画着,那人从海里升出来了,我狂叫着向海内跑去,他握住了我的双手,真的感到还是湿湿的,不像在梦中。
\par 由“圣彼得稣拉”又转了两趟车,是大型的巴士,也是两个人的座位三个人挤了坐,也是载了货。它不是梦中的“青鸟”。
\par “得拉”到了,下车看不到海。车站的人群和小贩也不同于山区小村的居民,他们高瘦而轻佻,不戴大帽子,不骑马,肤色不再是美丽的棕色,大半黑人。房子不再有瓦和泥,一幢幢英国殖民地似的大木头房子占满了城。
\par 过去洪都拉斯的北部是英国人、荷兰人,甚而十九世纪末期美国水果公司移来的黑人和文化。西班牙人去了内陆,另外的人只是沿海扩张。
\par 一个同样的小国家,那么不同的文化、人种和风景。甚而宗教吧,此地基督教徒也多于天主教了。
\par 那片海滩极窄,海边一家家暗到有如电影院似的餐馆就只放红绿色的小灯,狂叫的美国流行歌曲污染了大自然的宁静,海浪凶恶而来,天下着微雨。
\par 城里一片垃圾,脏不忍睹,可惜了那么多幢美丽的建筑。十几家大规模的弹子房比赛似的放着震耳欲聋的噪音。唉,我快神经衰弱了。
\par 菜单那么贵,食物是粗糙的。旅馆的人当然说没有热水。这都不成问题了,只求整个的城镇不要那么拼命吵闹,便是一切满足了。
\par 夜间的海滩上,我捡了一只垃圾堆里的椰子壳,将它放到海里去。海浪冲了几次,椰子壳总是去了又漂回来。
\par 酒吧里放着那首“I Love You More Than I Can Say”,中文改成《爱你在心口难开》的老歌。海潮里,星空下,恰是往事如烟——
\par 我在海边走了长长的路,心里一直在想墨西哥那位小神,想到没有释放自己的其他办法,跑进旅馆冰冷的水龙头下,将自己冲了透湿透湿。
\par 这个哀愁的国家啊!才进入你十多天,你的忧伤怎么重重地感染到了我?
\par 回到首都“得古西加尔巴”来的车程上,一直对自己说,如果去住观光大饭店,付它一次昂贵的价格,交换一两日浴缸和热水的享受,该不是羞耻的事情吧!
\par 可是这不过是行程中的第二个国家,一开始便如此娇弱,那么以后的长程又如何对自己交代呢?毕竟这种平民旅行的生涯,仍是有收获而值得的。
\par 经过路旁边的水果摊,葡萄要三块五毛连比拉一磅,气起来也不肯买。看中一幅好油画,画的就是山区的小泥房和居民,要价四千美金。我对着那个价钱一直笑一直笑,穷人的生活真是那么景色如画吗?
\par 米夏看我又回到原先那家没有热水的旅舍去住,他抗议了,理由是我太自苦。
\par 我没理他,哗哗地打开了公用浴室的冷水,狠狠地冲洗起这一千四百多公里的尘埃和疲倦来。
\par 旅舍内关了三整日,写不出一个字。房间换了一间靠里面的,没有窗,再也找不到桌子,坐在地上,稿纸铺在床上写,撕了七八千字,一直怔怔地在回想那一座座鬼域似凄凉的村庄。家徒四壁的泥屋,门上挂着一块牌子,写着“神就是爱”,想起来令人只是文字形容不出的辛酸。
\par 可是不敢积功课,不能积功课。写作环境太差,亮度也不够。不肯搬去大旅馆住,也实在太固执。
\par 这儿三日观光饭店连三餐的消费,可能便是山区一贫如洗的居民一年的收入了。
\par 虽说一路分给孩子们的小钱有限,报社经费也丰丰足足,可是一想到那些哀愁的脸,仍是不忍在这儿做如此的浪费。窗外的孩子饿着肚子,我又何忍隔着他们坐在大玻璃内吃牛排?当然,这是妇人之仁,可是我是一个妇人啊!
\par 最后一日要离去洪都拉斯的那个黄昏,我坐在乞儿满街的广场上轻轻地吹口琴。那把小口琴,是在一个赶集的印地安人的山谷里买的,捷克制的,算做此行的纪念吧!
\par 便在那时候,一辆青鸟巴士缓缓地由上街开了过来。
\par 米夏喊着:“快看!一只从来没有搭上的青鸟,奔上去给你拍一张照片吧!”
\par 我苦笑了一下,仍然吹着我的歌。
\par 什么青鸟?这是个青鸟不到的地方!
\par 没有看见什么青鸟呢!

\subsubsection*{后记}
\par 洪都拉斯是一个景色壮丽、人民有礼、安静而有希望的国家。他们也有水准极高的工业、城镇和住宅区。
\par 这篇文字,只是个人旅行的记录,只因所去的地方都是穷乡僻地,所处的亦是我所爱好最基层的大众。因此这只代表了部分的洪都拉斯所闻所见,并不能一概而论,特此声明。

\subsection{中美洲的花园\\\small{哥斯达黎加纪行}}

\par 这一路来,常常想起西班牙大文豪塞万提斯笔下的唐·吉诃德和他的跟随者桑却的故事。
\par 吉诃德在书本中是一位充满幻想,富正义感,好打抱不平,不向恶势力低头的高贵骑士。他游走四方,凭着一己的意志力,天天与幻想出来的敌人打斗——所谓梦幻骑士也。
\par 桑却没有马骑,坐在一匹驴子上,饿一顿饱一餐地紧紧跟从着他的主人。他照顾主人的一切生活起居,当主人面对妖魔时,也不逃跑,甚至参加战斗,永远不背叛他衷心崇拜的唐·吉诃德。
\par 当然,以上的所谓骑士精神与桑却的忠心护主,都是客气的说法而已。
\par 从另一个角度去看这两个人,一个是疯子,另一个是痴人。
\par 此次的旅行小组的成员也只有两个人——米夏与我,因此难免对上面的故事人物产生了联想。
\par 起初将自己派来演吉诃德,将米夏分去扮桑却,就这样上路了。
\par 一个半月的旅程过去了,赫然惊觉,故事人物身份移位,原来做桑却的竟是自己。
\par 米夏语文不通,做桑却的我必须助他处理,不能使主人挨饿受冻,三次酒吧中有什么纠缠,尚得想法赶人走开——小事不可惊动主人。
\par 在这场戏剧中,米夏才是主人吉诃德——只是他不打斗,性情和顺。
\par 只要一想到自己的身份,沿途便是笑个不休。
\par 当我深夜里在哥斯达黎加的机场向人要钱打公用电话时,米夏坐在行李旁边悠然看杂志。
\par 生平第一次伸手向人乞讨,只因飞机抵达时夜已深了,兑换钱币的地方已经关门,身上只有旅行支票和大额的美金现钞。不得已开口讨零钱,意外地得到一枚铜板,心中非常快乐。
\par 洪都拉斯已经过去了,住在哥国首都圣荷西有热水的旅舍里,反觉恍如梦中。
\par 在洪国时奔波太烈,走断一双凉鞋,走出脚上的水泡和紫血,而心中压着的那份属于洪都拉斯的叹息,却不因为换了国家而消失。
\par 写稿吧!练练笔吧!如果懒散休息,那么旅行终了时,功课积成山高,便是后悔不及了。
\par 一个月来,第一次跟米夏做了工作上的检讨,请他由现在开始,无论是找旅馆、机票、签证或买胶卷、换钱、搭车、看书、游览……都当慢慢接手分担,不可全由我来安排,他的日常西语,也当要加紧念书了。
\par 说完这些话,强迫米夏独自进城办事,自己安静下来,对着稿纸,专心写起沿途的生活记录来。
\par 这一闭关,除了吃饭出去外,摒除万念,什么地方都不去,工作告一段落时,已是在哥斯达黎加整整一周了。
\par 七日中,语文不通的米夏如何在生活,全不干我的事情。
\par 据说圣荷西的女孩子,是世界上最美的,米夏却没有什么友谊上的收获。只有一次,被个女疯子穷追不舍,逃回旅馆来求救,被我骂了一顿——不去追美女,反被疯子吓,吓了不知开脱,又给疯子知道了住的地方,不是太老实了吗?
\subsubsection*{中美洲的花园}
\par 哥斯达黎加号称中美洲的瑞士,首都圣荷西的城中心虽然不能算太繁华,可是市场物资丰富,街道比起洪国来另是一番水准,便是街上走的人吧,气质便又不同了。
\par 这个西邻尼加拉瓜、东接巴拿马、面积五万一千一百平方公里的和平小国,至今的人口方才两百万人左右。
\par 这儿的教师多于军队,是个有趣的比例。一九四八年时,哥斯达黎加宣布中立,除了一种所谓“国家民防队”的组织维持国内秩序之外,他们没有军防。
\par 据说,当西班牙人在十六世纪进占这片土地的时候,当地的印地安人因为欧洲带过来的传染病,绝大多数都已死亡,因此混血不多,是一个白人成分极高的国家。
\par 东部加勒比海边的里蒙海港地区,因为十九世纪末期“美国联合水果公司”引进了大批牙买加的黑人来种植香蕉,因此留下了黑人劳工的后裔,占数却是不多。
\par 哥斯达黎加在一八〇五年由古巴引进了咖啡,政府免费供地,鼓励咖啡的种植。四十年后,它的咖啡已经供应海外市场。又四十年以后,国内铁路贯穿了加勒比海与太平洋的两个海港,咖啡的外销,至今成了世上几个大量出口国之一。
\par 在建筑哥国的铁路时,来自中国的苦力,因为黄热病、极坏的待遇和辛苦的工作,死掉了四千人。那是一八九〇年。
\par 那条由圣荷西通到里蒙港的铁路,我至今没有想去一试。
\par 一节一节铁轨被压过的是我们中国人付出的血泪和生命。当年的中国劳工,好似永远是苦难的象征,想起他们,心里总是充满了流泪的冲动。
\par 哥斯达黎加实是一个美丽的国家,在这儿,因为不曾计划深入全国去旅行,因此便算它是一个休息站,没有跑远。
\par 去了两个距首都圣荷西不远的小城和一座火山。沿途一幢幢美丽清洁的独院小平房在碧绿的山坡上怡然安静地林立着,看上去如同卡通片里那些不很实在的乐园,美得如梦。
\par 这儿不是洪都拉斯,打造的大巴士车厢一样叫“青鸟”,而我,很容易就上了一辆。
\par 中美洲躲着的幸福之鸟,原来在这儿。
\subsubsection*{中国的农夫}
\par 在哥国,好友的妹妹陈碧瑶和她的先生徐寞已经来了好几年了。
\par 离开台北时,女友细心,将妹夫公司的地址及家中的电话全都写给了我,临行再三叮咛,到了哥国一定要去找这一家亲戚。
\par 只因我的性情很怕见生人,同时又担心加重别人的负担,又为了自己拼命写稿,到了圣荷西一周之后,徐寞夫妇家的电话仍是没有挂过去。
\par 其实自己心里也相当矛盾,徐寞是中兴大学学农的,进过农技队。而今不但是此地一家美国农技公司的大豆推广专家,同时也与好友合作经营自己的农场。他当是一个与自己本性十分相近的人才是。
\par 碧瑶是好友的亲妹妹,十几年前她尚是个小娃娃时便见过的,当然应该拜望。
\par 眼看再过三日便要离此去巴拿马了,偏是情怯,不太肯去麻烦别人,只怕人家殷勤招待,那便令我不安了。
\par 电话终于打了,讷讷地自我介绍,那边徐寞就叫起我三毛来,说是姐姐早来信了,接着碧瑶也在喊,要我过去吃晚饭。巧是他们农场大麦丰收,当天请了许多朋友,中国人、外国人都有,定要一同去吃饭。
\par 晚上徐寞开车亲自来接,连米夏都强邀了一起去,这份情谊,叫人怎么拒绝?
\par 徐寞及碧瑶的家,如果在台北,是千万富翁才住得起的花园小平房,他们却说是哥国最普通的住宅。
\par 我仍有一些失望,只因徐家不住在农场里。其实孩子上学的家庭,住在偏远的农场上是不方便的,徐家两个可爱的孩子,五岁的小文是双声带,家中讲中文,学校讲西文。可是她的儿童画中的人脸,都是哥斯达黎加味道的。
\par 那个夜晚,遇见了在此定居的中国同胞,其中当然有徐寞农场的合伙好友们。
\par 这些农夫谈吐迷人,修辞深刻切合,一个个有理想、有抱负,对自己的那块土地充满着热爱和希望。
\par 他们称自己的农场是“小农场”,我听听那面积,大约自己走不完那片地就要力竭。
\par 如果不是为了社交礼貌,可能一个晚上的时间都会在追问农场经营的话题上打转。毕竟对人生的追求,在历尽了沧桑之后,还有一份拿不去的情感——那份对于土地的狂爱。我梦中的相思农场啊!
\par 谁喜欢做一个永远飘泊的旅人呢?如果手里有一天捏着属于自己的泥土,看见青禾在晴空下微风里缓缓生长,算计着一年的收获,那份踏实的心情,对我,便是余生最好的答案了。
\par 徐寞和碧瑶怪我太晚通知,来不及去看他们的农场和乡下。最后徐寞又问我,能不能多留几日,与米夏一同下乡去。
\par 我不敢改变行程,只怕这一下乡,终生的命运又要做一次更大的变动。而现实和理想必然是有距离的。更怕自己孤注一掷,硬是从头学起,认真辛苦地去认识土地,将自己交付给它,从此做一个农妇——
\par 徐寞在送米夏和我回旅舍时,谈起他的孩子,他说:“希望将来她也学农!”听了这话,心里深受感动,他个人对土地、对农夫生活的挚爱,在这一句平凡的话里面表露得清清楚楚。
\par 我们这一代的移民是不同的了!
\par 哥国地广人稀,局势安定,气候温和,人民友善真诚。学农的中国青年,在台湾,可能因为土地有限而昂贵,难以发展。在这儿,如果不怕前十年经营的艰苦,实是可以一试的地方。带着刻苦耐劳不怕吃苦的中国人性格,哥斯达黎加会是一片乐土。
\par 上面这番话,包括了作者十分主观的情感和性向。事实上移民的辛酸和价值,见仁见智,每一个人的机遇又当然是不同了。
\par 光是选择了自己的道路和前程,能否成功,操在自己手中的那份决心,事实上只有一半的承诺和希望,毕竟大自然也有它的定律在左右着人的命运呢!
\subsubsection*{另一种移民}
\par 圣荷西是一个不满三十万人口的首都,满街中国餐厅,几步便是一个。去了几家,营业都不算太兴旺,价格却是不公平的低廉。想来此地餐馆竞争仍烈,价高了便更不能赚钱。
\par 去了一家中国饭店认识了翁先生。都是宁波人,谈起来分外亲切。那晚没有照菜单上的菜吃,翁先生特别要了“清蒸鱼”给我尝。
\par 这份同胞的情感,没有法子回报。也只有中国人对中国人,不会肯在食物上委屈对方,毕竟我们是一个美食文化的民族。
\par 翁先生来了哥斯达黎加五年,娶了此地的女子为妻。白手成家,年纪却比米夏大不了两三岁。能干的青年,中文程度在谈吐中便见端倪,在见识上亦是广博,分析侨情十分中肯,爱家爱国,没有忘记自己的来处,在异乡又创出一番天地。想想他的年纪,这实是不容易。
\par 所以我又说,这一代的移民,我们华人移民,在哥斯达黎加,是表现杰出的。
\subsubsection*{我想再来}
\par 与徐寞和碧瑶相见恨晚,他们可爱的大孩子小文,赚去了我的心,另一个因为太小,比较无法沟通。
\par 碧瑶说得一口西班牙文,初来哥国时住在没有水电的农场上,那种苦日子一样承受了下来。而今相夫教子,过得怡然本分,说起农场和将来,亦是深爱她自己选择的人生,这一点,便是敬她。
\par 三日相聚,倒有两日是碧瑶煮菜包饺子给米夏与我吃。
\par 徐家的朋友们,个个友爱,更可贵的是彼此谈得来,性向相近,都是淡泊的人。
\par 本是没有什么离情的异乡,因为每一个人的友谊,使我一再想回哥斯达黎加。
\subsubsection*{异乡人}
\par 在我的旅程中,哥国是来休息的一站,便真的放松了自己。有时就坐在公园内看人。
\par 一个卖爆米花的潦倒中年人,掮了一个大袋子,就在公园里一个人一个人地去兜。默默地看他跑了三四圈,竟没有一笔小生意成交。
\par 最后他坐到我身边的长椅子上来,头低垂着,也不去卖了。
\par “你怎么不卖给我呢?”我笑着问他。
\par 他吃了一惊,抬起头来,马上打开了袋子,拿出纸口袋来,问我要几块钱一包的。
\par 我不忙接米花,问他今日卖了多少。他突然眼睛湿湿的,说生意不好做。
\par 原来是古巴出来的难民,太太孩子都留在那儿,只等他在异乡有了发展去接他们。
\par “卖了几个月的爆米花,自己都三餐不济,只想等到签证去美国,可是美国没有一个人可以担保入境,有些早来的古巴人在这里已经等了三年了,而我——”
\par 我静静地听着他,看他擦泪又擦泪,那流不干的眼泪里包含了多少无奈、辛酸和乡愁——
\par “这包米花送给您,在这个异乡从来没有人跟我讲讲心里的话,说出来也好过些了,请您收下吧!”
\par 他交给我一个小包包,站起来慢慢地走开去了。
\par 我摸摸口袋里的钱,还有剩的一沓,忍不住去追他,塞在他的衣服口袋里,不说一句话就跑。后面那个人一直追喊,叫着:“太太!太太!请您回来——”
\par 自己做的事情使我羞耻,因为数目不多,同情别人也要当当心心去做才不伤人。可是金钱还是最现实的东西。第一日抵达哥国时,别人也舍给我过一枚铜板,那么便回报在同样的一个异乡人身上吧!
\par 我是见不得男人流泪的,他们的泪与女人不同。
\subsubsection*{离去}
\par 只因圣荷西是一个在十八世纪末叶方才建造的城市,它确是一个居住的好地方,但是在建筑和情调上便缺少了只有时间才能刻画出来的那份古意盎然。
\par 这儿没有印地安人,亦是不能吸引我的理由之一。哥国太文明了。
\par 走断了一双鞋,在此又买了一双新的,预备走更长的路。
\par 离去时,坐在徐寞的吉普车上,看着晴空如洗的蓝天和绿色的原野,一路想着农场的心事——我会为着另一个理由再回这儿来吗?
\par 上机之前要米夏给徐寞拍照。这一些中国好青年在海外的成就和光荣,是不应该忘记的。



\subsection{美妮表妹\\\small{巴拿马纪行}}

\par 又是陌生的一站了。
\par 机场大旅馆的价格令人看了心惊肉跳,想来小旅馆也不可能便宜。
\par 这儿是巴拿马,美国水准,美式风格,用的钞票也干脆是美金,他们自己只有铜板,纸钞是没有的,倒也干脆。
\par 旅途中经费充足,除了洪都拉斯超出预算之外,其他国家都能应付有余。可是住进巴拿马一家中级旅舍时,却使人因为它的昂贵而忧心了。
\par 抵达的那个夜晚,安置好行李,便与米夏拿了地图去老城中心乱走,只想换一家经济些的安身。
\par 找到一家二十多块美金一间的,地区脏乱不堪,恶形恶状的男女出出进进,它偏叫做“理想旅舍”。
\par 门口的醉汉们也罢了,起码躺在地上不动。那些不醉的就不太好了,即使米夏在我身旁,还是不防被人抓了一把。我停住了步子,骂了那群人一句粗话,其实他们也实在没有什么认真的恶意,却将米夏吓得先跑了几步才回头。
\par 那样的地区是住不得的了。
\par 二姨的女儿在此已有多年了,虽然想念,却又是担心惊动他们一家,住了一夜,迟迟疑疑,不知是不是走的那日再打电话见见面,这样他们便无法招待了。
\par 虽说如此,才有四日停留,巴拿马不预备写什么,而亲情总是缠心,忍不住拨了电话。再说,这个妹夫我也是喜欢的。
\par 只说了一声:“美妮!”那边电话里的表妹就发狂地喊了起来——“平平姐姐——”
\par 那声惨叫也许是她平日的语气,可是还是害我突然哽住了。表妹十年远嫁,她的娘家亲人还算我是第一个来巴拿马。
\par 过了一会儿,表妹夫也打电话来了,惊天动地地责我不叫人接机,又怪不预先通知,再问我身体好不好,又说马上下班,与表妹一同来接了家去。
\par 这份亲情,因为他们如此亲密的认同,使我方才发觉,原来自己一路孤单。
\par 虽然不喜欢劳师动众,可是眼见表妹全家因为我的抵达而当一回大事,也只有心存感激地接受了他们的安排和招待。
\par 在旅馆楼下等着表妹与妹夫来接时,我仍是紧张。米夏说好是不叫去的,他坐在一边陪我。
\par 妹夫外表没有什么改变,只是比以前成熟了。
\par 表妹相逢几乎不识,十年茫茫,那个留着长发、文静不语的女孩,成了一个短发微胖戴眼镜的妇人。
\par 表妹拉着我的手腕便往外走。当然米夏也被强拉上车了。
\par “不要米夏去,我们自己人有话讲,他在不方便!”我抗议着。
\par 表妹倒是实际:“有什么话要讲?吃饭要紧,先给你们好好吃一顿再做道理!”
\par 十年前,表妹二十岁,妹夫也不过二十四五,两个不通西班牙文的大孩子,远奔巴拿马,在此经商,做起钟表批发买卖,而今也是一番天地了。
\par 表妹与我仍说上海话,偶尔夹着宁波土话,一点不变。变了的是她已经羼杂了拉丁美洲文化的性情:开放、坦率,西班牙文流利之外,还夹着泼辣辣的语调。是十年异乡艰苦的环境,造就了一个坚强的妇人,她不再文弱,甚而有些强悍。
\par 用餐的时候,我无意间讲起表妹祖母在上海过世的消息,本以为她早就知道的,没想台北阿姨瞒着她。这一说,她啪一下打了丈夫一掌,惊叫起来:“德昆!德昆!我祖母死啦!死掉啦!”说着说着便要哭出来了!
\par 眼看要大哭了,一转念,她自说自话,找了一番安抚的理由,偏又是好了起来。
\par 初次见面,在餐厅里居然给了表妹这么一个消息,我自己内疚了好几日,谁晓得她不知道呢?
\par “你前两年伤心死了吧?”表妹问我,给我夹了一堆菜。
\par “我吗?”我苦笑着,心里一片空空茫茫。
\par “要是表姐夫还活着,我们家起码有我跟他讲讲西班牙文——”表妹又说。
\par 我突然非常欣赏这个全新的表妹,她说话待人全是直着来的,绝不转弯抹角,也不客套,也不特别安慰人,那份真诚,使她的个性突出、美丽,而且实在。
\par 只有四日停留,不肯搬去表妹家,只为着每日去会合米夏又得增加妹夫的麻烦。虽然那么样,表妹夫仍然停了上班。自由区的公司也不去了,带着米夏与我四处观光。
\par 换钱,弄下一站的机票,吃饭的和一切的一切都被他们包办了。在巴拿马,我们没有机会坐公共汽车。
\par 名为表姐,在生活起居上却被表妹全家,甚而他们的朋友们,照顾得周周密密。
\par 在这儿,同胞的情感又如哥斯达黎加一般地使人感动。
\par 农技团苏团长一家人过来表妹处探望我,一再恳请去他们家用餐。妹夫不好意思,我也坚持不肯麻烦苏妈妈。结果第二日,“使馆”的陈武官夫妇,中国银行的向家,苏家,彭先生,宋先生加上表妹自己,合起来做了满满一席的酒菜,理由是——请远道来的表姐。
\par 苏家的女孩子们离开中国已经好多年了,家教极好,仍看中文书,是我的读者。武官太太陈妈妈也是喜欢看书的。看见别人如此喜爱三毛,心里十分茫然,为什么自己却不看重她呢?难道三毛不是部分的自己吗?
\par 巴拿马本是哥伦比亚的一部分,当年它的独立当然与美国的支持有着很大的关系。
\par 运河与自由贸易区繁荣了这个国家,世界各地的银行都来此地吸取资金。市区像极了美国的大城,街上的汽车也是美国制造的占大多数,英文是小学生就开始必读的语言。
\par 虽然美国已将运河交还给巴拿马政府了,可是美军在此驻扎的仍有三万人。
\par 妹夫与表妹各人开的都是美国大车,度假便去迈阿密。免不了的美国文化,可是在家中,他们仍是实实在在的中国人。生意上各国顾客都有,而平日呼朋引伴地度周末,仍旧只与中国朋友亲密。
\par 在表妹可以看见海景的高楼里,妹夫对我干干脆脆地说:“什么外国,在家里讲中国话,吃中国菜,周末早晨交给孩子们,带去公园玩玩,下午打打小牌,听听音乐,外面的世界根本不要去看它,不是跟在中国一样?”
\par 我听了笑起来,喜欢他那份率真和不做作,他根本明白讲出来他不认外国人,只赚他们的钱而已。这是他的自由,我没有什么话说。
\par 这又是另一种中国移民的形态了!
\par 要是有一日,巴拿马的经济不再繁荣,大约也难不倒表妹夫。太太孩子一带,再去个国家打市场,又是一番新天新地。
\par 中国人是一个奇怪而强韧的民族,这一点是在在不同于其他人种的,随便他们何处去,中国的根,是不容易放弃的。
\par 表妹来巴拿马时根本是个不解事的孩子,当年住在“哥隆”市,接近公司设置的自由区。在那治安极坏的地区,一住五年,等到经济环境安定了才搬到巴拿马市区来。
\par 回忆起“哥隆”的日子,她笑说那是“苦笼”。两度街上被暴徒抢皮夹,她都又硬夺了回来。
\par 被抢当时表现得勇敢,回家方才吓得大哭不休。这个中国女孩子,经过长长的十年之后,而今是成熟了。
\par 我看着表妹的三个伶俐可爱的孩子和她相依为命的丈夫,还有她的一群好中国朋友,心中非常感动,毕竟这十年的海外生活,是一份生活的教育,也是他们自己努力的成果。
\par 表妹与表妹夫深深地迷惑了米夏,他一再地说,这两个人的“个性美”。虽然表妹夫的西班牙文不肯文绉绉,粗话偶尔也滑出来,可是听了只觉那是一种语调,他自己的真性情更在里面发挥得淋漓。奇怪的是,这些在家中只讲中文的人,西班牙文却是出奇的流利。
\par 在巴拿马的最后一日,曾“大使”夫妇与“中央社”的刘先生夫妇也来了表妹夫家中。
\par “大使”夫妇是十多年前在西班牙做学生时便认识的,只因自己最怕麻烦他人,不敢贸然拜望,结果却在表妹家碰到。
\par 聆听“大使”亲切的一番谈话,使我对巴拿马又多了一份了解。只因这一站是家族团聚,巴拿马的历史和地理也便略过了。
\par 三天的时间飞快地度过,表妹和他们朋友对待我的亲切殷勤,使我又一次欠下了同胞的深情。
\par 临去的那一个下午,表妹仍然赶着包馄饨,一定要吃饱了才给上路。她的那份诚心,一再在实际的生活饮食里,交付给了我。
\par 行李中,表妹硬塞了中国的点心,说是怕我深夜到了哥伦比亚没有东西吃。
\par 妹夫再三叮咛米夏,请他好好做我的保镖。
\par 朋友们一趟又一趟地赶来表妹夫家中与我见面,可说没有一日不碰到的。
\par 机场排队的人多,妹夫反应极快,办事利落,他又一切都包办了。
\par 表妹抱着小婴儿,拖着另外两个较大的孩子,加上向家夫妇和他们的小女儿、彭先生、应先生……一大群人在等着与我们惜别。
\par 进了检查室,我挥完了手,这才一昂头将眼泪倒咽回去。
\par 下一站没有中国人了,载不动的同胞爱,留在我心深处,永远归还不了。
\par 巴拿马因为这些中国人,使我临行流泪。这沉重的脚踪,竟都是爱的负荷。


\subsection{一个不按牌理出牌的地方\\\small{哥伦比亚纪行之一}}

\par 这一路来,随行的地图、资料和书籍越来越重,杂物多,索绊也累人。
\par 巴拿马那一站终于做了一次清理,部分衣物寄存表妹,纸张那些东西,既然已经印在脑子里,干脆就丢掉了。
\par 随身带着的四本参考书,澳洲及英国出版的写得周全,另外两本美国出版的观点偏见傲慢,而且书中指引的总是——“参加当地旅行团”便算了事。于是将它们也留在垃圾桶中了。
\par 说起哥伦比亚这个国家时,参考书中除了详尽的历史地理和风土人情介绍之外,竟然直截了当地唤它“强盗国家”。
\par 立论如此客观而公平的书籍,胆敢如此严厉地称呼这个占地一百多万平方公里的国家,总使人有些惊异他们突然的粗暴。
\par 书中在在地警告旅行者,这是一个每日都有抢劫、暴行和危险的地方,无论白昼夜间,城内城外,都不能掉以轻心,更不可以将这种情况当做只是书中编者的夸张。
\par 巴拿马台湾农技团的苏团长,来此访问时,也遭到被抢的事情。
\par 可怕的是,抢劫完苏团长的暴徒,是昂然扬长而去,并不是狂奔逃走的。
\par 米夏在听了书中的警告和苏团长的经历之后,一再地问我是不是放弃这一站。而我觉得,虽然冒着被抢的危险,仍是要来的,只是地区太差的旅舍便不住了。
\par 离开台湾时,随身挂着的链条和刻着我名字的一只戒指,都交给了母亲。
\par 自己手上一只简单的婚戒,脱脱戴戴,总也舍不得留下来。几番周折,还是戴着走了那么多路。
\par 飞机抵达博各答的时候,脱下了八年零三个月没有离开手指的那一个小圈,将它藏在贴胸的口袋里。手指空了,那份不惯,在心理上便也惶惶然地哀伤起来。
\par 夜深了,不该在机场坐计程车,可是因为首都博各答地势太高,海拔两千六百四十公尺的高度,使我的心脏立即不适,针尖般的刺痛在领行李时便开始了。没敢再累,讲好价格上的车,指明一家中级旅馆,只因它们有保险箱可以寄存旅行支票和护照。
\par 到了旅馆,司机硬是多要七元美金,他说我西班牙话不灵光,听错了价格。
\par 没有跟他理论,因为身体不舒服。
\par 这是哥伦比亚给我的第一印象。
\par 住了两日旅舍,第三日布告栏上写着小小的通告,说是房价上涨,一涨便是二十七元美金,于是一人一日的住宿费便要六十七元美金了。
\par 客气地请问柜台,这是全国性的调整还是怎么了,他们回答我是私自涨的。
\par 他们可以涨,我也可以离开。
\par 搬旅馆的时候天寒地冻,下着微雨,不得已又坐了极短路的计程车,因为冬衣都留在巴拿马了。
\par 司机没有将马表扳下,到了目的地才发现。他要的价格绝对不合理,我因初到高原,身体一直不适,争吵不动,米夏的西班牙文只够道早安和微笑,于是又被迫做了一次妥协。
\par 别的国家没有那么欺生的。
\par 新搬的那家旅馆,上个月曾被暴徒抢劫,打死了一个房间内的太太,至今没有破案。这件事情发生之后,倒是门禁森严了。
\par 初来首都博各答的前几日,看见街上每一个人紧紧抱着他们皮包的样子,真是惊骇。生活在这么巨大的,随时被抢的压力下,长久下去总是要精神衰弱的。
\par 米夏一来此地,先是自己吓自己,睡觉房间锁了不说尚用椅子抵着门。每次唤他,总是问了又问才开。
\par 便因如此,偏是不与他一起行动,他需要的是个人的经历和心得,不能老是只跟在我身边拿东西,听我解释每一种建筑的形式和年代。便是吃饭吧,也常常请他自己去吃了。
\par 个人是喜欢吃小摊子的,看中了一个小白饼和一条香肠,炭炉上现烤的。卖食物的中年人叫我先给他二十五披索,我说一手交钱一手交饼,他说我拿了饼会逃走,一定要先付。
\par 给了三十披索,站着等饼和找钱,收好钱的人不再理我,开始他的叫喊:“饼啊!饼啊!谁来买饼啊!”
\par 我问他:“怎么还不给我呢?香肠要焦了!”
\par 他说:“给什么?你又没有付钱呀!”
\par 这时旁边的另一群摊贩开始拼命地笑,望望我,又看着别的方向笑得发颤。这时方知又被人欺负了。
\par 起初尚与这个小贩争了几句,眼看没有法子赢他,便也不争了,只对他说:“您收了钱没有,自己是晓得的。上帝保佑您了!”
\par 说完这话我走开,回头对那人笑了一笑,这时他眼睛看也不敢看我,假装东张西望的。
\par 要是照着过去的性情,无论置身在谁的地盘里,也不管是不是夜间九点多钟自己单身一个,必然将那个小摊子打烂。
\par 那份自不量力,而今是不会了。
\par 深秋高原的气候,长年如此。微凉中夹着一份风吹过的怅然和诗意。只因这个首都位置太高,心脏较弱的人便比较不舒服了。
\par 拿开博各答一些小小的不诚实的例子不说,它仍是一路旅行过来最堂皇而气派的都市。殖民时代的大建筑辉煌着几个世纪的光荣。
\par 虽说这已是一生中第一百多个参观过的博物馆,也是此行中南美洲的第十二个博物馆了。可是只因它自己说是世上“唯一”的,忍不住又去了。
\par 哥伦比亚的“黄金博物馆”中收藏了将近一万几千多件纯金的艺术品。制造它们的工具在那个时代却是最最简陋的石块和木条。金饰的精美和细腻在灯光和深色绒布的衬托下,发出的光芒近乎神秘。
\par 特别注意的是一群群金子打造的小人,有若鼻烟壶那么样的尺寸。他们的模样,在我的眼中看来,每一个都像外太空来的假想的“人”。
\par 这些金人,肩上绕着电线,身后背着好似翅膀的东西,两耳边胖胖的,有若用着耳机,有些头顶上干脆顶了一支天线般的针尖,完全科学造型。
\par 看见这些造型,一直在细想,是不是当年这片土地上的居民,的确看过这样长相和装备的人,才仿着做出他们的形象来呢?这样的联想使我立即又想到朋友沈君山教授,如果他在身边,一定又是一场有趣的话题了。
\par 博物馆最高的一层楼等于是一个大保险箱,警卫在里面,警卫在外面,参观的人群被关进比手肘还厚的大铁门内去。
\par 在那个大铁柜的房间里,极轻极微号角般的音乐,低沉、缓慢又悠长地传过来。
\par 全室没有灯光,只有专照着一座堆积如黄金小山的聚光灯,静静地向你交代一份无言的真理——黄金是唯一的光荣,美丽和幸福。
\par 放出那层严密保护着金器的房间,再见天日时,刚刚的一幕宝藏之梦与窗外的人群再也连不上关系。
\par 下楼时一位美国太太不断叹息着问我:“难道你不想拥有它们吗?哪怕是一部分也好!天啊,唉!天啊!”
\par 其实它们是谁的又有什么不同?生命消逝,黄金永存。这些身外之物,能够有幸欣赏,就是福气了。真的拥有了它那才叫麻烦呢!
\par 在中南美洲旅行,好似永远也逃不掉大教堂、美国烤鸡、意大利馅饼和中国饭店这几样东西。
\par 对于大小教堂,虽说可以不看,完全意志自由,可是真的不进去,心中又有些觉得自己太过麻木与懒散,总是免不了去绕一圈,印证一下自己念过的建筑史,算做复习大学功课。
\par 至于另外三种食的文化,在博各答这一站时,已经完全拒绝了。尤其是无孔不入的烤鸡、汉堡和麦克唐纳那个国家的食物和文化,是很难接受的。至于中国饭店,他们做的不能算中国菜。
\par 在这儿,常常在看完了华丽的大教堂之后,站在它的墙外小摊边吃炸香蕉、芭蕉叶包着有如中国粽子的米饭和一支支烤玉米。
\par 这些食物只能使人发胖而没有营养。
\par 博各答虽是一个在高原上的城市,它的附近仍有山峰围绕。有的山顶竖了个大十字架,有的立了一个耶稣的圣像,更有一座小山顶上,立着一座修道院,山下看去,是纯白色的。
\par 只想上那个白色修道院的山顶去。它叫“蒙色拉”,无论在哪一本参考书,甚至哥伦比亚自己印的旅游手册上,都一再地告诫旅客——如果想上“蒙色拉”去,千万乘坐吊缆车或小铁路的火车,不要爬上去,那附近是必抢的地区。
\par 城里问路时,别人也说:“坐计程车到吊缆车的入口才下车吧!不要走路经过那一区呀!”
\par 我还是走去了,因为身上没有给人抢的东西。
\par 到了山顶,已是海拔三千公尺以上了,不能好好地呼吸,更找不到修道院。山下看见的那座白色的建筑,是一个教堂。
\par 那座教堂正在修建,神坛上吊着一个金色的十字架;神坛后面两边有楼梯走上去,在暗暗的烛光里,一个玻璃柜中放着有若人身一般大的耶稣雕像——一个背着十字架、流着血汗、跪倒在地上的耶稣,表情非常逼真。
\par 在跌倒耶稣的面前,点着一地长长短短的红蜡烛,他的柜子边,放着许许多多蜡做的小人儿。有些刻着人的名字,扎着红丝带和一撮人发。
\par 总觉得南美洲将天主教和他们早期的巫术混在一起了,看见那些代表各人身体的小蜡像,心中非常害怕。
\par 再一抬头,就在自己上来的石阶两边的墙上,挂满了木制的拐杖,满满的、满满的拐杖,全是来此祈求,得了神迹疗治,从此放掉拐杖而能行走的病人拿来挂着做见证的。
\par 幽暗的烛光下,那些挂着的拐杖非常可怖,墙上贴满了牌子,有名有姓有年代的人,感恩神迹,在此留牌纪念。
\par 对于神迹,甚而巫术,在我的观念里,都是可以接受的,毕竟信心是最大的力量。
\par 就在那么狭小的圣像前,跪着一地的人,其中一位中年人也是撑着拐杖来的,他燃了一支红烛,虔诚地仰望着跌倒在地的耶稣像,眼角渗出泪来。
\par 那是个感应极强的地方,敏感的我,觉得明显的灵息就在空气里充满着。
\par 我被四周的气氛压迫得喘不过气来,自己一无所求,而心中却好似有着莫大的委屈似的想在耶稣面前恸哭。
\par 出了教堂,整个博各答城市便在脚下,景色辽阔而安静,我的喉咙却因想到朋友张拓芜和杏林子而哽住了——他们行走都不方便。
\par 又回教堂里面去坐着,专心地仰望着圣像,没有向他说一句话,他当知道我心中切切祈求的几个名字。
\par 也代求了欧阳子,不知圣灵在此,除了治疗不能行走的人之外,是不是也治眼睛。
\par 走出圣堂的时候,我自己的右腿不知为何突然抽起筋来,疼痛不能行走。拖了几步,实在剧痛,便坐了下来。在使人行走的神迹教堂里,我却没有理由地跛了。那时我向神一直在心里抗议,问他又问他:“你怎么反而扭了我的腿呢?如果这能使我的朋友们得到治疗,那么就换好了!”他不回答我,而腿好了。
\par 代求了五个小十字架给朋友,不知带回台湾时,诚心求来的象征,朋友们肯不肯挂呢?
\par 虽说身上没有任何东西可抢,可是走在博各答的街上,那份随时被抢的压迫感却是不能否认地存在着。
\par 每天看见街上的警察就在路人里挑,将挑出来的人面对着墙,叫他们双手举着,搜查人的身体,有些就被关上警车了。
\par 在这儿,我又觉得警察抓人时太粗暴了。
\par 米夏在博各答一直没有用相机,偶尔一次带了相机出去,我便有些担心了。
\par 那一日我坐在城市广场里晒太阳,同时在缝一件脱了线的衣服。米夏单独去旧区走走,说好四小时后回公园来会合。
\par 一直等到夜间我已回旅馆去了,米夏仍未回来。我想定是被抢了相机。
\par 那个下午,米夏两度被警察抓去搜身,关上警车,送去局内。
\par 第一回莫名其妙地放了,才走了几条街,不同的警察又在搜人,米夏只带了护照影印本,不承认是证件,便又请入局一趟。
\par 再放回来时已是夜间了。这种经历对米夏也没有什么不好,他回来时英雄似的得意。
\par 这个城市不按牌理出牌,以后看见警察我亦躲得老远。
\par 离开博各答的前两日,坐公车去附近的小城参观了另一个盐矿中挖出来的洞穴教堂,只因心脏一直不大舒服,洞中空气不洁,坐了一会儿便出来了,没有什么心得。
\par 哥伦比亚的出境机场税,是三十块美金一个人,没有别的国家可以与它相比。
\par 记录博各答生活点滴的现在,我已在厄瓜多尔一个安地斯山区中的小城住了下来。
\par 飞机场领出哥伦比亚来的行李时,每一只包包都已打开,衣物乱翻,锁着的皮箱被刀割开大口,零碎东西失踪,都是博各答机场的工作人员留给我的临别纪念。
\par 那是哥伦比亚,一个非常特殊的国家。



\subsection{药师的孙女——前世\\\small{厄瓜多尔纪行之一}}
\par 那时候,心湖的故事在这安地斯山脉的高原上,已经很少被传说了。
\par 每天清晨,当我赤足穿过云雾走向那片如镜般平静的大湖去汲水的时候,还是会想起那段骇人的往事。
\par 许多许多年前,这片土地并不属于印加帝国的一部分。自古以来便是自称加那基族的我们,因为拒绝向印加政府付税,他们强大的军队开来征服这儿,引起了一场战争。
\par 那一场战役,死了三万个族人,包括我的曾祖父在内,全都被杀了。
\par 死去的人,在印加祭司的吩咐下,给挖出了心脏,三万颗心,就那么丢弃在故乡的大湖里。
\par 原先被称为银湖的那片美丽之水,从此改了名字,我们叫它“哈娃哥恰”,就是心湖的意思。
\par 那次的战役之后,加那基族便归属于印加帝国了,因为我们的山区偏向于城市基托,于是被划分到阿达华伯国王的领地里去。
\par 那时候,印加帝国的沙巴老王已经过世了,这庞大的帝国被他的两个儿子所瓜分。
\par 在秘鲁古斯各城的,是另一个王,叫做华斯加。
\par 岁月一样地在这片湖水边流过去。
\par 战争的寡妇们慢慢地也死了,新的一代被迫将收获的三分之二缴给帝国的军队和祭司,日子也因此更艰难起来。
\par 再新的一代,例如我的父母亲,已经离开了故乡,被送去替印加帝王筑石头的大路,那条由古斯各通到基托的长路,筑死了许多人。而我的父母也从此没有了消息。
\par 母亲离开的时候,我已经是个懂事而伶俐的孩子,知道汲水、喂羊,也懂得将晒干的骆马粪收积起来做燃料。
\par 她将我留给外祖父,严厉地告诫我要做一个能干的妇人,照顾外祖父老年后的生活,然后她解下了长长一串彩色的珠子,围在我的脖子上,就转身随着父亲去了。
\par 当时我哭着追了几步,因为母亲背走了亲爱的小弟弟。
\par 那一年,我六岁。一个六岁的加那基的小女孩。
\par 村子里的家庭,大半的人都走了,留下的老人和小孩,虽然很多,那片原先就是寂静的山区,仍然变得零落了。
\par 外祖父是一个聪明而慈爱的人,长得不算高大,他带着我住在山坡上,对着大雪山和湖水,我们不住在村落里。
\par 虽然只是两个人的家庭,日子还是忙碌的。我们种植玉米、豆子、马铃薯,放牧骆马和绵羊。
\par 收获来的田产,自己只得三分之一,其他便要缴给公共仓库去了。
\par 琼麻在我们的地上是野生的,高原的气候寒冷,麻织的东西不够御寒,总是动物的毛纺出来的料子比较暖和。
\par 母亲离开之后,搓麻和纺纱的工作就轮到我来做了。
\par 虽然我们辛勤工作,日子还是艰难,穿的衣服也只有那几件,长长的袍子一直拖到脚踝。
\par 只因我觉得已是大人了,后来不像村中另外一些小女孩般地披头散发。
\par 每天早晨,我汲完了水,在大石块上洗好了衣服,一定在湖边将自己的长发用骨头梳子理好,编成一条光洁的辫子才回来。
\par 我们洗净的衣服,总是平铺在清洁的草地上,黄昏时收回去,必有太阳和青草的气味附在上面,那使我非常快乐,忍不住将整个的脸埋在衣服里。
\par 在我们平静的日子里,偶尔有村里的人上来,要求外祖父快去,他去的时候,总是背着他大大的药袋。那时候,必是有人病了。
\par 小时候不知外祖父是什么人,直到我一再地被人唤成药师的孙女,才知治疗病人的人叫做药师。
\par 那和印加的大祭司又是不同,因为外祖父不会宗教似的作法医病,可是我们也是信神的。
\par 外祖父是一个沉默的人,他不特别教导我有关草药的事情,有时候他去很远的地方找药,几日也不回来。家,便是我一个人照管了。
\par 等我稍大一些时,自己也去高山中游荡了,我也懂得采些普通的香叶子回来,外祖父从来没有阻止过我。
\par 小时候我没有玩伴,可是在祖父的身边也是快活的。
\par 那些草药,在我们的观念里是不能种植在家里田地上的。
\par 我问过外祖父,这些药为什么除了在野地生长之外,不能种植它们呢?
\par 外祖父说这是一份上天秘密的礼物,采到了这种药,是病家的机缘,采不到,便只有顺其自然了。
\par 十二岁的我,在当时已经非常著名了,如果祖父不在家,而村里的小羊泻了肚子,我便抱了草药去给喂。至于病的如果是人,就只有轮到外祖父去了。
\par 也许我是一个没有母亲在身边长大的女孩,村中年长的妇女总是特别疼爱我,她们一样喊我药师的孙女,常常给我一些花头绳和零碎的珠子。
\par 而我,在采药回来的时候,也会送给女人们香的尤加利叶子和野蜂蜜。
\par 我们的族人是一种和平而安静的民族,世世代代散居在这片湖水的周围。
\par 在这儿,青草丰盛,天空长蓝,空气永远稀薄而寒冷,平原的传染病上不了高地,虽然农作物在这儿长得辛苦而贫乏,可是骆马和绵羊在这儿是欢喜的。
\par 印加帝国的政府,在收税和祭典的时候,会有他们的信差,拿着不同颜色和打着各样绳结的棍子,来传递我们当做的事和当缴的税,我们也总是顺服。
\par 每当印加人来的时候,心湖的故事才会被老的一辈族人再说一遍。那时,去湖边汲水的村中女孩,总是要怕上好一阵。
\par 外祖父和我,很少在夜间点灯,我们喜欢坐在小屋门口的石阶上,看湖水和雪山在寂静平和的黄昏里隐去,我们不说什么多余的话。
\par 印加帝国敬畏太阳,族人也崇拜它,寒冷的高原上,太阳是一切大自然的象征和希望。
\par 当然,雨季也是必需的。一年中,我们的雨水长过母羊怀孕的时间。
\par 小羊及小骆马出生的时候,草原正好再绿,而湖水,也更阔了。
\par 我一日一日地长大,像村中每一个妇女似的磨着玉米,烘出香甜的饼来供养外祖父。在故乡,我是快乐而安静的,也更喜欢接近那些草药了。
\par 有一日,我从田上回来,发觉屋里的外祖父在嚼古柯叶子,这使我吃了一惊。
\par 村子里的一些男人和女人常常嚼这种东西,有些人一生都在吃,使得他们嘴巴里面都凹了一块下去。这种叶子,吃了能够使人活泼而兴奋,是不好的草药。
\par 外祖父见到了我,没有什么不好意思的表情,他淡淡地说:“外祖父老了!只有这种叶子,帮助我的血液流畅——”
\par 那时候,我才突然发觉,外祖父是越来越弱了。
\par 没有等到再一个雨季的来临,外祖父在睡眠中静静地死了。
\par 在他过世之前,常常去一座远远的小屋,与族人中一个年轻的猎人长坐。那个猎人的父母也是去给印加人筑路,就没有消息了。
\par 回来的时候,外祖父总是已经非常累了,没有法子与我一同坐看黄昏和夜的来临,他摸一下我的头发,低低地喊一声:“哈娃!”就去睡了。
\par 在我的时代里,没有人喊我的名字,他们一向叫我药师的孙女。
\par 而外祖父,是直到快死了,才轻轻地喊起我来。他叫我哈娃,也就是“心”的意思。
\par 母亲也叫这个名字,她是外祖父唯一的女儿。
\par 外祖父才叫了我几次,便放下我,将我变成了孤儿。
\par 外祖父死了,我一个人住在小屋里。
\par 我们的族人相信永远的生命,也深信转世和轮回,对于自然的死亡,我们安静地接受它。
\par 虽然一个人过的日子,黄昏更寒冷了,而我依然坐在门前不变地看着我的故乡,那使我感到快乐。
\par 那一年,那个叫做哈娃的女孩子,已经十五岁了。
\par 外祖父死了没有多久,那个打猎的青年上到我的山坡来,他对我说:“哈娃,你外祖父要你住到我家去。”
\par 我站在玉米田里直直地望着这个英俊的青年,他也像外祖父似的,伸手摸了一下我的头发,那时候,他的眼睛,在阳光下湖水也似的温柔起来。
\par 我没有说一句话,进屋收拾了一包清洁的衣物,掮起了外祖父的药袋,拿了一串挂在墙上的绳索交给这个猎人。
\par 于是我关上了小屋的门,两人拖着一群骆马和绵羊还有外祖父的一只老狗,向他的家走去。
\par 我的丈夫,其实小时候就见过了,我们的狗几年前在山里打过架。
\par 当时他在打猎,我一个人在找草药,回家时因为狗被咬伤了,还向外祖父告过状。
\par 外祖父听到是那个年轻人,只是慈爱而深意地看了我一眼,微笑着,不说什么。
\par 没晓得在那时候,他已经悄悄安排了我的婚姻。
\par 有了新的家之后,我成了更勤劳的女人,丈夫回来的时候,必有烤熟的玉米饼和煮熟的野味等着他。那幢朴素的小屋里,清清洁洁,不时还拿尤加利的树叶将房间熏得清香。
\par 我们的族人大半是沉默而害羞的,并不说什么爱情。
\par 黄昏来临时,我们一样坐在屋前,沉静地看月亮上升。而我知道,丈夫是极疼爱我的。
\par 那时候,村里的药师已经由我来替代了。
\par 如同外祖父一个作风,治疗病家是不能收任何报酬的,因为这份天赋来自上天,我们只是替神在做事而已。
\par 虽是已婚的妇人了,丈夫仍然给我充分的自由,让我带了狗单独上山去摘草药。
\par 只因我的心有了惦记,总是采不够药就想回家,万一看见家中已有丈夫的身影在张望,那么就是管不住脚步地向他飞奔而去。
\par 那时印加帝国已经到了末期,两边的国王起了内战,村里的人一直担心战争会蔓延到这山区来。
\par 虽然我们已成了印加人收服的一个村落,对于他们的祭司和军队,除了畏惧之外,并没有其他的认同,只希望付了税捐之后,不要再失去我们的男人。
\par 战争在北面的沙拉萨各打了起来,那儿的人大半战死了。北部基托的阿达华伯国王赢了这场战役,华斯加王被杀死了。
\par 也在内战结束不多久,丈夫抱了一只奇怪的动物回来,他说这叫做猪,是低原的人从白人手中买下来的。
\par 我们用马铃薯来喂这只猪。当时并不知猪有什么用处。
\par 三只骆马换回了这样的一只动物是划不来的。
\par 村里偶尔也传进来了一些我们没有看过的种子。
\par 我渴切地等待着青禾的生长,不知种出来的会是什么样的农作物。
\par 有关白人的事情便如一阵风也似的飘过去了,他们没有来,只是动物和麦子来了。
\par 平静的日子一样地过着,我由一个小女孩长成了一个妇人。我的外祖父、父亲、母亲都消失了,而我,正在等待着另一个生命的出世。
\par 作为一个药师的孙女,当然知道生产的危险,村中许多妇人便是因此而死去的。
\par 黄昏的时候,丈夫常常握住我的手,对我说:“哈娃!不要怕,小孩子来的时候,我一定在你身边的。”
\par 我们辛勤地收集着羊毛,日日纺织着新料子,只希望婴儿来的时候,有更多柔软而暖和的东西包裹他。
\par 那时候,我的产期近了,丈夫不再出门,一步不离地守住我。
\par 他不再打猎,我们每餐只有玉米饼吃了。
\par 那只猪,因为费了昂贵的代价换来的,舍不得杀它,再说我们对它也有了感情。
\par 一天清晨,我醒来的时候,发觉门前的大镬里煮着几条新鲜的鱼。这使我大吃一惊,叫喊起丈夫来。
\par 心湖里满是跳跃的银鱼,可是百年来,没有一个人敢去捉它们,毕竟那儿沉着我们祖先的身体啊!
\par 丈夫从田上匆匆地跑回来,我痛责他捕鱼的事情,他说:“哈娃!你自己是药师的孙女,怀着孩子的妇人只吃玉米饼是不够的,从今以后吃鱼吧!”
\par 丈夫每夜偷偷去湖里捉鱼的事情,慢慢地被族人发现了。他们说我们会遭到报应,可是我们不理会那些闲话。
\par 只因跟着丈夫相依为命,生产的事情,约好了绝对不去请求村中的老妇人来帮忙。她们能做的不多,万一老妇人们来了,丈夫是必定被赶出去的,没有丈夫在身边,那是不好过的。
\par 在一个寒冷的夜里,我开始疼痛。
\par 悄悄起床煎好了草药才喊醒沉睡的丈夫。
\par 起初两个人都有些惊慌,后来我叫丈夫扶着,包着毯子到门外的石阶上去坐了一会儿,这便心静了下来。
\par 那是我最后一次看见月光下的雪山、湖水和故乡茫茫的草原。
\par 挣扎了三个日出与日落,那个叫做哈娃的女人与她未出世的孩子一同死了。
\par 在一汪油灯的旁边,跪着爱她如命的丈夫。他抱着哈娃的身体,直到已成冰冷,还不肯放下来。
\par 那是后人的日历十六世纪初叶,一个被现今世界统称为南美印地安人的女子平凡的一生。
\par 哈娃离世时十九岁。















\subsection{悲欢交织录\\\small{三毛故乡归}}

\par 中国这片海棠叶子,实在太——大了。
\par 而我,从来不喜欢在我的人生里,走马看花、行色匆匆。面对它,我犹豫了,不知道要在哪一点,着陆。
\par 终于,选择,我最不该碰触的,最柔弱的那一茎叶脉——我的故乡,我的根,去面对。
\par 从小,我们一直向往着那“杏花烟雨江南”,到底是怎样一个地方,竟然能让乾隆皇帝六下江南。于是,放弃了大气磅礴的北方,决定走江南。在春天,去看那无际的油菜花。
\par 就这么决定了,要先对祖先和传统回归,对乡愁做一个交代,然后,才能将自己的心情变成一个游客。
\par 因此,在南方的第一大城——上海,降落。它,是我父母出生的地方。
\par 在上海,有个家,就是三毛爸爸——漫画家张乐平的家。
\par 在现今的三毛还没有出生以前,张乐平已经创造了一个叫做三毛的孤儿——这个孩子和父母总是无缘的。所以,这个叫三毛的女子,也就和那个叫三毛的小人儿一样,注定和父母无缘。即使是回家吧,也不过只得三天好日子而已。
\par 张府方才三日天伦,又必匆匆别离,挥泪回首,脚步依依,而,返乡之行开始了。
\par 那时候,三毛回大陆的消息已经见报,三毛不能是她自己了,三毛是三毛。于是,搬进了上海同济大学招待所,没有去住旅馆。招待所有警卫。为着身体的健康,自己的心有余而力不足,三毛对广大的中国知识青年保持着一段距离,免得在情感上过分的冲击与体力上过分的消耗,使自己不胜负荷。
\par 那个张爱玲笔下魂牵梦萦,响着电车叮叮\UncommonChar{𪠽𪠽},烤着面包香,华洋夹杂的大上海,果然气派不同。
\par 但是,跑不完哪。
\par 七天之后,还是离开上海,到了苏州。
\par 姑苏,苏州,林黛玉的故乡,而那位林妹妹是《红楼梦》里非常被人疼惜的一个角色。
\par 那天到了苏州已是黄昏。为着已经付了昂贵的车资,把行李往表哥家一丢,就道:“我们利用车子赶快走吧!”随行关爱三毛的亲戚都问:“要去什么地方那么急迫?”答:“寒山寺。”
\par 四点多钟的下午,游客已经散尽。
\par 天气微凉,初春雨滴在风里斜斜地打在绿绿发芽的枫叶上。轻轻地走进寒山寺,四周鸦雀无声。绿荫小道上,一个黑衣高僧大步走来,这时蹲了下去,对着背影咔嚓一声,一张照片,并没有惊动任何人。
\par 走到禅房,看到一个大和尚静悄悄地在写字,两个小和尚在一旁拉纸。站在门槛外,头伸进门里,微微一笑。
\par 小和尚认出来者是他的精神好友,叫了一声“哎唷”。于是被请进禅房,又是微微一笑。就在大和尚还没有了悟过来来者是谁的时候,双林小和尚立即道:“这是台湾来的,鼎鼎大名的作家三毛小姐。”
\par 三毛此时已知了一分,三毛在中国的所有名声,并不是个脚踏实地之人,只是个“鼎鼎大名的三毛”而已,此时,内心一阵黯然。
\par 了然了,是一个虚的。
\par 于是,大和尚给写了一幅字,于是也还出一幅字出来。拿起笔来一挥,自称郑板桥式。写好之后,大和尚极有分寸地合掌,道了再见。
\par 小和尚依依不舍,送了出来,跟到一栋小楼,就在三毛措手不及的时候引上楼梯。一个转弯,哎呀,三毛叫了一声,寒山寺那口大钟就在眼前。
\par 钟在眼前,心中说了一句:“这是假的。那个真的钟已经到日本去了。”
\par 但是钟就是钟,也就不再分真分假。
\par 小和尚把三毛引到钟锤垂吊之处,道:“你敲。”
\par 当时本想谦虚,一看,钟上塑着八卦,那个钟锤正对着乾卦“≡”字。自己的名字就在上面,大好机会如何不敲,须知机会稍纵即逝。手一扬,扶住钟锤,开始用尽全身的气和志——冲撞,横着冲的。
\par ㄅㄤ(bɑnɡ)——余音几乎要断了,
\par ㄅㄤ——余音要断,
\par ㄅㄤ——
\par 撞毕三下。一边旁听的亲友都说:“这一生再要听钟,必在某年某月的某一天黄昏,静坐在寒山寺外,等待,感受今天这种措手不及之下的寒山寺的钟声。”
\par 下得楼来,靠在墙上问自己:这莫非是梦吧?!双脚几乎无法走路。
\par 踯躅走到香火的地方,见到明明一座禅寺,禅的境界何需香火?此时开口笑道:“上香不必了。”
\par 正待举步,小僧来报:“性空法师请入禅房。”原来那收入相机的黑衣高僧就是方丈性空。
\par 方丈来了,留下一幅字,小和尚立即上前卷好。以三毛之名留下一件东西之后,离去。
\par 回到家里,嫂嫂开饭了。
\par 从此,苏州五日,成了一个林黛玉,哭哭笑笑,风、花、雪、月。
\par 走进苏州小院,笑道:“这个院子跟照片里的,不同。照片里的中国名园,看了也不怎么样,深入其境的时候,嗳——”不说话了。
\par 旁边的人问:“跟照片有什么不同呢?”
\par 又道:“少了,一阵风——吧!”
\par 这时,微风吹来,满天杏花缓缓飘落地上。众人正要穿越花雨,三毛伸手将人挡住,叫道:“别动,且等,等林妹妹来把花给葬了,再踩过去。林妹妹正在假山后面哭着呢,你们可都没听到吗?”
\par 如此五日。
\par 五日之后,经过一条国人所不太知悉的水道,开始了河上之行。
\par 跟着堂堂哥哥行在一条船上,做妹妹的就想:“这不是林妹妹跟着琏二哥哥走水道回家去吗?”这时哥哥累极,躺下就开始打呼,妹妹看到哥哥累了,轻轻打开船舱门。
\par 哥哥警觉性高,扬声说道:“妹妹不要动,哪里去?”妹妹用吴侬软语说:“外面月亮白白的,我去看看。”哥哥实在力竭,便说:“妹妹,那么自己当心,不要掉到水里去。”
\par 这一夜,沿着隋炀帝的运河,一路地走,妹妹开始有泪如倾。
\par 水道进入浙江省的时候,哥哥醒来,已是清早。哥哥问了一句话,妹妹没听清楚,突然用宁波话问道:“梭西?”这一路,从上海话改苏州话,又从苏州话改成宁波话。妹妹心中故国山河随行随变,都在语言里。
\par 杭州两日,躲开一切记者。记者正在大宾馆里找不到三毛的时候,已然悄悄躲进铺位,开始挤十六路公共汽车。
\par 那时三毛不再是三毛,三毛只是中国十一亿人里的一株小草,被人——尽情践踏。
\par 两天的经历,十分可贵。
\par 只因血压太低,高血压七十,低血压四十,六度昏了过去。妹妹终于道:“哥哥,不好了,让我们回故乡吧。”
\par 当车子进入宁波城,故乡人已经从舟山群岛专来远迎。此去四小时之路,只要车子行过的地方,全部绿灯。
\par 到了码头,船长和海军来接,要渡海进入舟山群岛。来接的乡亲方才问说:“刚才一路顺畅,知道为什么吗?”答道:“没有注意,一直在看两岸风景。”问话的人又说:“绿灯一条龙,全是为你,妹妹。”妹妹脸色不大好看,回答:“也太低估我了,我可不是这等之人。”一时场面颇窘。
\par 船进舟山群岛鸭蛋山码头,船长说:“妹妹,远道而来,码头上这么多人等着你,这一声入港的汽笛——你拉。”妹妹堂而皇之地过去。
\par 尖叫呀,那汽笛声,充满着复杂的狂喜,好似在喊:“回来啦——”
\par 船靠岸,岸上黑压压的一大群人。自忖并无近亲在故乡,哥哥说:“他们都是——记者。”妹妹不知道要把这一颗心交给故乡的谁?便又开始洒泪。
\par 上岸,在人群里高唤:“竹青叔叔,竹青叔叔,你——在——哪——里?”眼睛穿过人群拼命搜索——陈家当年的老家人——倪竹青。
\par 人群挤了上来,很多人开始认亲,管他是谁,一把抓来,抱住就哭。乡愁眼泪,借着一个亲情的名词,洒在那些人的身上。
\par 抱过一个又一个,泪珠慌慌地掉。等到竹青叔叔出现,妹妹方才靠在青叔肩上放声大哭。“竹青叔,当年我三岁零六个月,你抱过我。现在我们两人白发、夕阳、残生再相见,让我抱住你吧。”说罢,又是洒泪痛哭。
\par 然后,这一路走,妹妹恍恍惚惚,一切如在梦中。将自己那双意大利短靴重重地踩在故乡的泥土上,跟自己说:“可不是——在做梦吧?”
\par 这时候,所有听到的声音都说着一样的话:“不要哭,不要哭。回来了,回来了,回来了。好了好了,好了好了。休息了,休息了,休息了。好了好了好了好了好了好了……”
\par 妹妹的泪流不止歇。
\par 当时一路车队要送妹妹直奔华侨宾馆,妹妹突然问:“阿龙伯母在哪里?她是我们在故乡仅存的长辈,要去拜访。”于是,车子再掉头驶近一幢老屋。
\par 人未到,妹妹声先夺人:“阿龙伯母——平平回来啦——”老太太没来得及察觉,一把将她抓来往椅子上一推,不等摄影记者来得及拍照,电视台录影的人还没冲进来,妹妹马上跪了下来,磕三个头,一阵风似的,又走了。上华侨宾馆。
\par 好,父母官来了。记者招待会来了。
\par 三天后,回到定海市郊外——小沙乡,陈家村。祖父出生的老宅去了。
\par 那一天,人山人海,叫说:“小沙女回来了。”
\par 三毛有了一个新的名字——小沙女。
\par 乡亲指着一个柴房说:“你的祖父就是在这个房间里出生的。”妹妹扑到门上去,门上一把锁。从木窗里张望,里面堆着柴,这时候妹妹再度洒泪。
\par 进入一个堂堂堂伯母的房子,有人捧上来一盆洗脸水,一条全新的毛巾,妹妹手上拿起,心下正想脸上还有化妆,又一转念,这毛巾来得意义不同,便坦坦然洗掉——四十年的风尘。用的是——故乡的水。
\par 水是暖的。妹妹却再度昏倒过去。
\par 十五分钟之后,妹妹醒来,说道:“好,祭祖。”
\par 走到已经关了四十年的陈家祠堂,妹妹做了一个姿势,道:“开祠堂。”
\par 乡人早已预备了祭祖之礼,而不知如何拜天祭祖,四十年变迁,将这一切,都遗失了。点了香一看,没有香炉,找了个铁罐头也一样好。妹妹一看,要了数根香,排开人群,叫了一声:“请——让开。”
\par 转过头来,对着天空,妹妹大声道:“先谢天,再谢地,围观的乡亲请一定让开,你们——当不起。”
\par 回过身来,看到一条红毯,妹妹跌跪下去。将香插进那破破小罐头里。此时妹妹不哭,开始在心中向列位祖先说话:“平儿身是女子,向来不可列入家谱。今日海外归来的一族替各位列祖衣锦还乡,来的可是个,你陈家不许进入家谱之人。”
\par 拜祖先,点蜡烛,对着牌位,平儿恭恭敬敬地三跪九叩首——用的是闽南风俗。因为又是个台湾人,从关帝庙里看来的。
\par 拜完,平儿又昏过去,过了十五分钟后,醒来,道:“好,上坟。”数百人跟着往山上去了。
\par 几乎是被人拖着上山,好似腾云驾雾。
\par 来到祖父坟前。天刚下过雨,地上被踩得一片泥泞。妹妹先看风水,不错。再看地基稳不稳固,水土保持牢不牢靠,行。再看祖父名字对不对,为他立碑人是谁,再看两边雕的是松,是柏,是村花,点头道:“很好。”这才上香。
\par 坟前,妹妹放声高唤:“阿$\text{ㄧ}\grave{\text{ㄚ}}$(yà)、阿$\text{ㄧ}\grave{\text{ㄚ}}$——魂——魄——归——来,平平来看你了。”此时放怀痛哭。像一个承欢膝下的孙儿,将这一路心的劳累、身的劳累,都化做放心泪水交给亲爱亲爱的祖父。
\par 正当泪如雨下之时,一群七八岁的小孩穿着红衣在一旁围观,大笑。心里想起贺知章的句子:乡音不改鬓毛——儿童——笑问客从何处来。他们只道来了一个外地人,坐着轿车来的,对着一个土馒头在那里哭。他们又哪里懂得。
\par 儿童拍手欢笑,但是在场四十岁以上的人眼眶里全含着一泡泪,有的落了下来,有的忍着。
\par 一切祭祖的形式已完。父亲的老书记竹青叔走到毛毯前,扑通跪了下去,眼睛微微发红,开始磕头。三毛立即跪下,在泥泞地里,还礼。
\par 亲友们,乡人们,陆续上来。外姓长辈的,平儿在泥地里还礼;平辈的,不还礼。乡人一面流泪,一面哭坟:“叔公啊,当年我是一个家贫子弟,不是你开了振民小学给村庄里所有孩子免费来读书,今天我还做不成一个小学的老师,可能只是一个文盲。”少数几个都来拜啦,都来哭啦。这时陈姓人站着,嗳——可暂时平了,那过去四十年——善霸之耻。
\par 还完礼,祖父魂魄并未归来。平儿略略吃惊。
\par 扑到新修墓碑上,拍打墓碑叫唤:“阿$\text{ㄧ}\grave{\text{ㄚ}}$,阿$\text{ㄧ}\grave{\text{ㄚ}}$,你还不来。时光匆匆,不来,我们来不及了。”
\par 来了,阿$\text{ㄧ}\grave{\text{ㄚ}}$来了。留下几句话。
\par 平儿听了祖父的话,收起眼泪擦干。抓起祖父坟头一把土,放进一个塑胶袋。平儿道:“好,我们走了,下山吧。”
\par 下山路滑,跟随记者有的滑倒,有的滚下山坡,只小沙女脚步稳稳地,一步一踏。只见她突然蹲下,众人以为又要昏倒,又看她站起来,手里多了一朵白色小野花。红色霹雳袋一打开,花朵轻轻摆进去。不够。再走十步之后,又蹲一次,一片落叶,再蹲一次,一片落叶,再蹲一次——三片落叶。
\par 好了。起身道:“故乡那口井,可没忘,我们往它走去。”
\par 祖父老宅的水井仍在。
\par 亲戚疼爱小沙女,都以为台湾小姐娇滴滴的,立即用铅桶打了一桶水上来要给。妹妹道:“别打,让我自己来。”乡人问:“你也会打水吗?”小沙女道:“你们可别低估了人。”
\par 于是,把水倒空,将桶再放进井里去,把自己影子倒映在水里,哐[插图]一声,绳子一拉,满满一桶水。
\par 水倒进一个瓶子里。不放心沿途还有很多波折,深恐故乡的水失落。拿起一个玻璃杯,把没有过滤的、混混的井水装了,不顾哥哥一旁阻拦:“妹妹不可以,都是脏的——”一口喝下。
\par 东张西望,看到屋顶上有个铁钩挂着,一指:“那个破破旧旧的提篮,可还用吗?”堂堂伯母说:“提篮里不过是些菜干,妹妹可要菜干吗?”妹妹答:“菜干不必,提篮倒是送给我也好。”
\par 堂伯母把提篮擦擦,果然给了平儿。
\par 喝了井水,拿了提篮,回到旅馆,还是不放心。拿出那罐土,倒来那瓶井水,掺了一杯,悄悄喝下。心里告诉自己:“从此不会生病了,走到哪里都不再水土不服。”
\par 两天后,三毛离开了故乡。
\par 天,开始下起了绵绵细雨,送别它的小沙女。正是——风雨送春归。
\par 妹妹洒泪上车,仍然频频回首道:“我的提篮可给提好啊!”里面菜干换了,搁着一只陈家当年盛饭的老粗碗。
\par 上船了,对着宾馆外面那片美丽的鸦片花,跟自己说:“是时候了。”拿着一块白色哭绢头,再抱紧一次竹青叔,好,放手。上船。
\par 此时,汽笛响了,顾不到旁的什么,哭倒在栏杆上,自语:“死也瞑目。”
\par 此生——
\par 无——憾。
\par 是了,风雨送春归,在春楼主走也。是《红楼梦》里,“元迎探惜”之外多了的一个姊妹——在春。
\par 走了走了。
\par 好了好了。不再胡闹了。


\subsection{但有旧欢新愁\\\small{金陵记}}

\par 每当有亲友返回大陆旅游或归乡,总会在临行之前顺口问一句:“可要带些什么东西给你?”我的回答千篇一律:“口袋里放些雨花台的小石头回来,那就千恩万谢了。”人说:“你还是爱石头。”我笑说:“是呀——嗳——”
\par 也有些年轻人不明白——这么不晓事的,会紧接着追问:“雨花石在什么地方?难道要亲自去为你捡石头吗?”我又嗳了一声,说:“南京呀——”
\par 说起南京,那是中国六大名都之一。
\par 每当我看历史,看到唐、虞、夏、商、晋、战国、梁武帝、三国、太平天国……这些时代,总也脱不开那份跟南京这个地名的联想。包括有一回,坐计程车,驾驶先生姓澎名建业,我就去问他是否江苏南京人。那人说不是,反问我为什么如此猜测。我说:“三国时代吴国建都南京,时称——建业。”那位老先生笑说:“小姐倒是反应灵敏,却是不相干的。倒想请教小姐——我有个弟弟唤做建康,你当他是何地人联想?”我说:“那您令弟必然又是个南京人——晋朝时代,五胡乱华,晋室被迫南渡——就在南京建都,当时南京叫做建康。以后宋、齐、梁、陈还有南唐,都在那个地方建都。”那位驾驶先生笑道:“小姐倒是真会拉扯,我的儿子叫做子文,总不能也给扯到南京去了吧?”我说:“那就得请求您大伯伯让我在车子里开窗——抽烟,就再说了——”他连忙递上自己的长寿烟来,说:“你请、你请。”我取一支烟,说:“好——东南之奥区,群山之总汇——金陵也。而其统领,实为钟山。钟山是大茅山脉的支阜——这不扯了。钟山又名蒋山——东汉时代有个秣陵尉——蒋子文,就死在那里——而得名。”我向窗外忽——吹了一口烟,笑说:“大伯伯,您老家河北,这我也听得出来。不过请看,您老——公子的大名,不也在南京给扯出来了?”那位驾驶伯伯喜得一直回头看我,都不顾那交通了。又说:“小姐好生有趣,倒是再来考考你。我们河北小地方,在《水浒传》里可出了个大名鼎鼎的县——至今还在——是我老家。你倒猜猜?”我当当心心地笑说:“大伯伯莫非是清河县人氏?”那位荣民伯伯突然一踩刹车,快没把那紧贴在我们后面开车的一位小姐给惊死——她按喇叭骂我们。我说:“大伯伯是武大郎的好同乡。”那伯伯喊道:“你小姐怎不说我是好汉武松的小同乡,倒来笑话人。”此时的我,在这种对话中,正是明月如霜,好风如水,我又笑说:“倒是您老哦——可别笑话了自家人。请问伯伯——武大武二是一母所生两个,还是几母所生?”那计程车一拍手,喊声“对呀——”的同时,我的地方到了,我付钱,下车。那追赶的声音急迫:“小姐可是南京人——”我开始冲向绿灯跑斑马线,在人群里大喊一声——中国人——吓倒了另一位低头迎面而来的——小姐。
\par 我并不是南京人。
\par 倒是祖父陈宗绪先生,在当年南京一个叫做下关的地方,经营着北方袁世凯家族事业的一部分。祖父将“启新洋灰”由天津水运过来——一桶一桶装的,做了江南五省的代理。什么叫做洋灰呢?说白些,就是现代的水泥。当时,祖父的事业并不只是如此,同时另做木材五金冰厂以及美孚煤油这些买卖。为着运输上的方便,祖父在南京下关就靠着长江的地方,设立了仓库和码头。又买下了大片的土地,盖起了这么五六十幢二层楼的房子成为一条一条的街巷。我父亲,就在南京度过了他的童年。
\par 有关父亲的童年,据父亲说,并不是只在南京一个地方生活的。我问他这是为了什么,他说:“打仗呀!兵来了我们就得逃难。一天到晚都在逃的吔!”我又问父亲:“你是民国初年的人了,你们逃什么难呢?”他说:“军阀嘛,南方北方打来打去,中国人自己打个不停。”我说:“对了,阿爷自传里也说:自从孙文革命以来,业务一落千丈。”父亲说:“这是祖父文件里的事,你又知道了。”我说:“这种信函、地契、自传、族谱……你们是不看的了,家族中也只有我。当年你们晒衣箱,我就晒在大太阳底下拼命地看。”父亲说:“后来,南京那一整条街,都给日本人炸光了。”我说:“我们的产业也就完了。”父亲说:“是。我们近代中国人的命运和战争有着不可分割的关系。”我心里笑说:“唉——还不是一样吃饭睡觉生孩子——伟大。”
\par 我就是在那一睁开眼睛就是兵荒马乱的——中国,出生的。我看见在几道河流的三角地带,停着一架铁鸟,我在很高的坡地上看得清楚,我们正往那架大鸟的方向接近。我又看见,姐姐和我进入了一种空间,很窄的空间,我们被安放在一个饼干盒子上坐好,四周尖锐的声音将我的耳膜弄得很痛。我看见我的头顶有网篮一般的吊袋,我看见母亲的脸色有些紧张。我感觉到坐着的铁盒子开始倾斜,我绝对没有可能听见母亲向我说话,因为巨大的声响盖过了一切。可是,我知道,我正在经历一桩危险奇特的事情。当这种时刻来临时,我感到害怕,于是,我用手紧紧地扳住那坐着的铁盒子,对自己悄悄说:“哦——耶稣基督。”
\par “那是什么地方,有水围绕?”四十年后我问父亲。
\par “那是重庆珊瑚坝飞机场嘛!”父亲说。
\par “我为什么不是坐在椅子上?”我说。
\par “你倒记得了。你的确坐在铁盒子上飞的。”
\par 抗战胜利了,我的出生卡住了一个和平的时代。就那样,我们全家由重庆飞回了南京。那个祖父居留过的城市。
\par 祖父并不在南京,我没有见过阿爷。那时候祖父回到舟山群岛老家定海城去了。
\par 我的家庭意识并不成长在南京。如果有家人肯听我拉扯,好好的大晴天都能够给我扯到落大雨——我的记忆来自我出生的那一天——有人在说——嗳,又来了个妹妹,也好也好。我听见父亲骑着大马飞奔而来,马蹄的声音方才歇了,他本人的脚步静静踏入房间。我又听见有人对父亲说:“是个女孩。”——我心虚,不敢啼哭。我知道——这是父亲来世上跟我照面第一回。
\par 我又听见马蹄的声音哒哒响过青石板的路面,而我只看见好大的格子布蒙在我身上。我在一辆马车里——深夜,向什么地方走去。那马蹄的声音催人放心入梦,空气中充满着树林般的清香。母亲就在旁边。然后,那个平平坦坦的大宅第,为我的童年,拉开了序幕。
\par 我的家,在南京、鼓楼、头条巷、四号。
\par 那是不必有人教的,因为我没有单独出门的权力,所以也不必记住地址以便迷路了好回来。那是因为哥哥姐姐们讲电话给同学听,讲自己住在什么地方、什么门牌,就给我听进心里去了一辈子。我们是一个大家庭。父亲的长兄、长嫂、我的四个堂兄一个堂姐加上我的嫡亲姐姐、父亲母亲和另外一个小男孩子——马蹄子和他的妈妈兰瑛以及兰瑛的亲戚——门房老婆婆,还有那永不消失的江妈、大师傅、吴妈、小赵和那温文儒雅、默默无言的竹青叔叔,全在这个大宅第中一起度着八年抗战之后再度重建家园的岁月。
\par “一共二十个人。那我们可是依靠祖上余荫在过日子,对不对?”我后来请问父亲,已在好多年后了。
\par 父亲说:“没有。”父亲说,祖父当年告老还乡去了。南京的大房子是租下来的。这一大家族没有分过家。是大伯父汉清先生和父亲做事情来维持的。
\par 在南京,父亲和大伯父没有另设办公室,那幢三层楼西式洋房的楼下书房,做了他们兄弟两人的法律事务所。至于楼上的几个房间给了伯父全家人共住。楼下除了客厅书房饭厅之外,另有小房间,那是竹青叔叔居住的地方。
\par 竹青叔姓倪,是我们同乡——祖父至爱的乡侄,练得一手好字,当年一切文书全以毛笔字抄写的时代,青叔是伯父以及父亲必须的依靠。青叔自家人,名义上是法律事务所的书记。父亲长竹青叔七岁。
\par 我们是两房姐妹兄弟大排行。我行第七。
\par 就因为我的弟弟也来到了这个世界,虽然家中人口众多,江妈被分到我们二房来看视,但是我还是意识到了自己的孤单。那时候我两岁多。不知要说哪一种语言。我们家中,上海话、宁波话、四川话和南京话混着讲,我也就没有了特定的母语。
\par 当年,不上学的孩子只我和小婴儿弟弟,其他的手足白天不常在家。没有人讲话也是好的,小时候,就因为不必讲话,反而学会了一样终生的兴趣——观察。
\par 那个房子是独幢的,成为一个回字形。有围墙,不算太高,如果我爬上假山,站在假山顶上就可以看见外面的街道。如果我不爬假山只站在院子里,我能看见鼓楼那幢建筑以及在空中飘扬的英国旗子和苏联的国旗。英国人和俄国人是我们的邻居。在那幢大房子里面,有正门,两面对开的。正门旁边有着一扇小门,于是门房老婆婆的房间就在那里了。
\par 每当有客来的时候,先在门房处按铃,如果有名片的来人,会把名片交给门房,于是名片被先送了进去,伯父或者父亲就站在楼下迎客人。当客人要离开的时候,必然由主人亲自送到大门外,方才告别。
\par 也不止是客人才来的,那时,有一种推销员,他们不是白俄就是由苏联流亡过来的犹太人,在身上披挂着好重的地毯,也会来按铃。有一次我听见一位地毯人跟父亲说:“OK——You get it.”然后彼此握了握手。我们家就多了一条地毯。那是我今生第一次听见英文。
\par 当然,墙外的岁月与我是没有太大关系的,可是每当那——“马头牌冰棒!马头牌冰棒!”的吆喝声开始传进墙来的时候,我们家里的后院水井中,就开始被泡下了西瓜。要吃的黄昏,就像打水一般,用个桶下去,哐[插图]一声——冰西瓜就上来了。
\par 不,我们是有自来水的,井水用来洗车子。
\par 刚刚讲的是前门,在回字形左边的地方,另有好大的边门,伯父和父亲的三轮车、吉普车就放在后院边门的地方。于是,前院种了梧桐树、桑树和花草。那分隔前后院的篱笆成了一面花的墙——爬墙玫瑰。一切客人来时,视线中望去,并没有生活的痕迹,只能看见大树、草地这种东西。而我的游踪,却是满屋子转着。我酷爱后院那鲜明活泼的生活——大师傅炒菜、江妈纳鞋底、吴妈烫衣服、小赵洗车子、兰瑛打她的孩子、门房老婆婆打兰瑛。一到了夏天,堂哥们兴趣大,弄来了个“手摇机器”开始自制冰淇淋。那时候我总听见他们说:“再加些盐,不够。”很多年以后,我还是肯定冰淇淋是盐做出来的。那时候我不问为什么,那是小七时代,问了也得一句:“小孩子走开!”没有回答。
\par 其实,家中住着二十个人,常常来的人就不止这么些。伯母的弟弟们老往我们家中跑,那三舅舅和五舅舅的样子,我至今记得。那时他们是一种有着救国思想的热血青年,一天到晚跟堂哥堂姐讲政治,国民党也是那时候知道的,还有委员长蒋介石也曾听过。每当,舅舅们讲到他们的理想,声音就低起来了,中国共产党这种名词,总是在对我先来凶一句“你小孩子走开呀——”之后,在我背后轻轻传来。我觉得有一种气氛不对劲,可是哪里说得上来。他们在冬天特别讲得多。都靠着壁炉悄悄讲。
\par 夏天了,马蹄子总是要长脓疮,而且长在头顶上,母亲把他的头发给剃了,满头涂上白白的粉,我和马蹄子常常头靠头的——顶住,不是玩,抢秋千。母亲看见就要喊:“你们不要头靠头呀!看传染了——妹妹你也没有头发。”我哪里明白那么多。只知道,如果想抢赢,只要叫声:“兰瑛——”
\par 兰瑛是门房老太太给引介进来帮忙家事的,拖着个孩子,并不知男人有没有。门房老婆婆是病着了,病着病着不能起床了,兰瑛每天拿了饭菜得去喂她。每当老婆婆坐在床沿,而兰瑛拿个汤匙叫“你吃呀——你吃呀——”的时候,我就静静地观察老太太的白骨——她极瘦。那两只小腿在夏天里给露了出来,一种令人惊异的细枯。也在那一个夏天,家中有人说:“不成了,要走了,最好给她准备准备。”我知道必是讲门房。我常常一个人去偷看她,倒看她怎么走。那一天,我亲眼看见一串白色的飞蛾由老婆婆的口里飞出去,我很惊奇,跑了开去,又没有人好去告诉,因为不太会形容这种现象。那天晚上,老婆婆死了。
\par 她的棺材被抬上了一辆大卡车,伯父、父亲,还有很多人都坐上了车,我自然只是旁观者。兰瑛哭得怪大声的。那是我第一次意识到死亡。当时我边看死人边采栀子花苞,一共四朵。
\par 母亲告诉我:“妹妹,我们要相信耶稣,信耶稣可以得永生。”我很认真地又一次点头。在我学讲话的时代中,爸爸妈妈伯伯嬷嬷——我的大伯母,是共同存在的大人物,还有一位就是耶稣。我实在不知道他是谁,怎么每天晚上睡觉以前,母亲总是带了头要我们小孩子闭上眼睛,然后母亲就开始——“求你——求你——求你——”了呢?于是,我了然了,耶稣是一种看不见的东西,正如只有我——看见过门房老婆婆口中飞出去的白蛾一样,别人是看不见的。所以耶稣是一种比飞蛾更奇妙的东西,因为连我也看不见,一次也没有见过。有一次我为了讨好母亲,想,她最爱的名字好像就是耶稣。于是我说:“我要上天堂去看耶稣了。”母亲立即骂我:“不要乱说话。”我实在不明白大人的心理。爱他,怎么又怕真的碰见他呢?
\par 而那幢大房子之外的世界,也并不是永恒地将我被高墙所阻隔。每到星期天,母亲会拉了姐姐和我,走路去一个有着许多排长条椅子的地方,在那边唱歌——他们叫那种歌——赞美诗。我一周一次的出门,在三岁半的时候,实在是托了我主耶稣基督之福,好让我出去逛逛。虽然那教堂不是夫子庙,总也聊胜于无。
\par 在南京,我们住着西洋式的房子,过着西式的耶诞节。每到雪花快要飘落的冬季,那家中大客厅的壁炉上面,自有哥哥姐姐给铺上了白棉花造成雪景,也会跑出晶晶闪闪的小碎片被什么人给撒在白雪上。当,耶诞树顶上那一大颗银色的伯利恒之星被悬挂了上去的时候,自会有人向我叫喊:“快,把袜子拿出来——挂在壁炉边边上,今天晚上圣诞老公公要来送礼物啰——”我从来不在这件事情上费过心,那种大家庭团聚在一起的时光,是我最不自由的证明——每一个人,每一个人都比我大,他们对我说话都是命令式的,包括——“乖——过来。跳一个舞给大家看。好。一二三——跳。”
\par 在冬天下雪不能够去院子里玩的时候,我最爱最爱跑到楼下的书房中去。那是家中的办公室,也是竹青叔写公文的地方。而我们小孩子,一再被严重警告——不许进去玩的禁地。在那安静极了的地方,我看见了至今仍然酷爱把玩的文房四宝。它们,就像那竹青叔叔,永远一袭长袍,不说什么话,而散发出一份文人雅士的清幽之气——谓之风华。这我自小时候就喜欢上了的家中一角,却是很少进去。大人很欢喜我去看耶诞树并且赞叹它,而我的爱物,却是一只书房中的中国小瓷花缸。
\par 瓷花缸比一只汤碗还小,里面斜斜搁着一支比珍珠耳环还要细小的水勺,父亲用这水勺淘水,放在大砚台中磨出墨来写字。每当无人的午后,只要江妈不注意我,我就往书房中跑。进去了,先上椅子,再上桌子,趴在小瓷缸边,一小匙一小匙的水往砚台凹处当心地倒下去,再拿起墨来,把自己弄成全身上下黑漆漆的时候,大概已经被捉了出来。
\par 冬天的孩子被母亲捉住,一定用棉袍把我们变成圆球,行动很不方便。两只手臂总是成为八字形,小脚也肿了起来——穿元宝鞋了。在这种包裹的季节里,院子开始积雪,哥哥姐姐们打雪仗啦!他们放寒假,不必上学。在大雪纷飞的开始,家中大的孩子们——十七八九岁了,会等、等、等,等到他们说“好啦”的时候,积雪一定够厚,厚到可以堆雪人了。哥哥们做出来的雪人老是咬着一支烟斗,那眼睛——是一圈圈葡萄干给塞出来的。总是雪人一做好就开始打雪仗,平日不太理会我的那上面六个兄姐,在这种时候特别注意到跑不快的我,那种雪弹——啪——往我飞来的时候,只有给自己炸掉,哭都不好哭,不然就不给参加了。打中了还得合作倒地——叫做——死啦!
\par 有一次,院子里还在呼啸开战的当儿,我悄悄跑进了书房。那会子撞到了父亲,他对我说:“不许碰东西。”父亲离开了,我哪里忍得住不爬上椅子去碰文具。还是那只放水的小花瓷缸,水面上露出小勺子来。我只轻轻一拿小勺子,那小水缸啪一下子碎开了,而水不流出来——它们结成了冰。我意识到自己闯了大祸,立即开溜,心跳得好快。不久之后,父亲在楼梯间中将我找到了,把我带进书房,轻轻问我:“这小水缸是不是你弄破的?”我拼命摇头。
\par 那是我今生第一次不开口也说了谎,动机出于害怕。
\par 那一次,没有人打骂我,我被单独留在书房中罚站。那竹青叔叔——不过二十多岁但是绝对不参加哥哥们游戏、谈话的他,悄悄走了进来,抱起了三岁零六个月的我,交给我一支毛笔。
\par 可以想见,四十年后,当竹青叔叔和我再度在故乡舟山群岛的码头上相见时,我狂喊着“竹青叔叔——”同时扑进他怀里去时,那——[插图]如三鼓,铿然一叶,黯黯梦魂惊断——在他和我的泪眼中,数十年光阴重叠镜头般,哗哗流转成时空倒置的浮生幻境。
\par 四十年前的楼下书房之外,在南京那幢大房子里的二楼,还有一间图书室。大人的书,给放在架子上面,儿童书籍被排在接近地板的地方。我常常躲在书架跟墙缝的角落里看小人书——我没有不识字的记忆而我还没有上幼稚园。就在那个快乐天堂里,我发现唯一的堂姐明珠,坐在床沿,生气般地垂着头,而三舅的一位男同学,正在向她下跪。那个图书室大概也是明珠姐姐的睡房,不然哪来的一张单人床呢?
\par 当时,是我先进去看书的,挤在凹进去的墙缝里,他们两个也进来了,而没有发现我的存在。于是,男的向姐姐求爱。姐姐一看到那呆住了的我,一推跪着的人,自己就冲了出去,接着那个三舅的男同学也冲了出去。我的心,啪一下炸掉了,炸成好多好多鸡心由空中再向自己的身上慢慢、慢慢飘落下来。那好几天,我魂不能守舍,一直脸上发热。我亲眼看见了一件比耶稣基督、飞蛾更神秘的东西——爱情。就在南京的图书室里那个下跪男人的反光眼镜里。
\par 也是在那一场好戏里,我手中正拿着一本漫画书——《三毛从军记》。
\par 四十年之后的初春,我下了中国民航,在大上海的夜里,上了汽车往一个人直奔而去。我奔向归乡第一站中的第一个人——他——八十二岁——他——站在寂静的巷堂中被儿女搀扶着迎接我——我——紧张得跑了起来,我们同时张开了手臂,我这天涯倦客,轻轻拥抱住了——三毛的创作者——张乐平大师。一时里我哭了。方才知道,浮生如梦,只要还是眼底有泪,又何曾舍得梦觉。
\par 南京故居的那个爱看书的小孩子,再一度不知今夕何年。
\par 当时,我又何曾明白,徐蚌会战,山河易色——是什么时代的转换带动了包括我们家族的变迁。只听见,伯母告诉她的弟弟们:“你们这种样子的言论,国民党要来捉人了。”家中呈现了一份不寻常的紧张气氛。不,那不是因为祖父阿爷的过世,那也是紧张的。全家大小突然在我身边消失了好久好久,连姐姐弟弟都不见了。我跟着江妈,唱“春天里呀百花香”。家人再出现时,母亲逼我穿上一种白色的布鞋,我不肯穿上,母亲指着墙角幽暗的地方对我说:“你不听话,看,阿爷的鬼魂从那个地方冒出来捉你。”我怕得不得了,就穿上了那双白布鞋。那是我第一次又意识到,除了飞蛾——我可是看到的、耶稣基督、爱情之外,生命中还可以有另外一种看不见的东西,而这种东西叫做——鬼魂。可是哥哥的泪并不因为他怕鬼。
\par 我从来没有看见三堂哥哭过,他十多岁了,喜欢养蚕,而我很不喜欢这种凉凉软软的东西,它们灰白的颜色也令人感到恶心。就有那么一天,我爬到窗沿边去玩,窗沿下放着一盘盘竹子编成的好大扁盘,盘子里面数千条哥哥养的蚕。我一不当心,仰面跌倒下去,跌在那些蚕的身上,我一时爬不起来,那些未死的蚕开始爬到我身上来,在我尖声狂叫的同时,哥哥赶来——发觉我压死了他的数百条心肝宝贝——他哭了。
\par 而上三段我正在说起并不知道南京家中紧张气氛的来临是为了什么的时候,父亲突然交给我好大一沓钞票——真的金圆券——国民政府的钞票,对我淡淡地说:“拿去玩吧——没有用了。”那是一种比鬼魅更要令人不安的东西,看得见的,在我小小的手中,一大沓——钞票,父亲叫我拿去玩。在那同时,三堂哥把他视为第二生命的蚕宝宝,整盘整盘地给抖落到院子中的桑树上去——他站在假山上把蚕往树顶上倒,口里说:“你们自己活命去吧,我不能再养你们了。”
\par 我听见明珠姐姐对大伯父说:“要走你们走,我要留下来念大学。”我听见母亲跟父亲深夜里商量——先带妹妹走,还是先带宝宝——宝宝是我的大弟。我看见箱子,大箱子由阁楼上被拖了下来。我看见地毯被卷了起来,我看见小赵、江妈、吴妈、兰瑛日渐严肃的面容——他们忙。我看见哥哥们理书包、丢书。我看见家中人来人往,我听见姐姐的同学们向她说再见。我发觉母亲不许我跟马蹄子抢一只玩具熊,她对我说:“你不许抢,留下来给他,统统给他。”在这些不合一般生活秩序中最使我惧怕的却是一种“分离的意识”:明珠姐姐要跟父亲分开。舅舅们可能被一种力量捉去。母亲在选择弟弟和我。姐姐的小朋友不再一起上学。代表行动的箱子一口一口出现。哥哥宝爱的蚕要被倒在树上。明明是纸钞,父亲给了我又说它没有用。我们的书都不能再翻——叫我们放下。生活中每天一样的日子不能够再度出现。
\par 我当时并不能明白,中国人的命运和那永不停止的战争,和小小的我有着什么关系。而我所甚感知足的日子,为什么要以离开,成为我长大的记忆。我以为,南京鼓楼的一切,就是我的全部;而我不是刚刚被送进鼓楼幼稚园通过了一场考试——在老师们面前唱歌跳舞,而被允许去做幼儿生了吗?
\par 没有人向我解释这一场变化。
\par 我生命中第二次的迁移发生在南京火车站。当我被举着放进车厢里去时,我看见家中不可分离的江妈、小赵、大师傅、兰瑛他们,拼命向车内的我们递塞吃的东西,连平日不常吃到的香蕉都成串地往我们丢上来。他们紧紧拉住母亲的手、姐姐的手、我的手。火车长鸣一声——汽笛拉起了尖锐的声音,车子慢慢开动了,双方的手链不肯放开。人群中,车外送行的老家人,叫喊起来:“小妹——妹妹——快快回来——太太——三五个月——就快回来——我们当心看住房子——快去快回呀——不过又是一场逃难——”他们哭了,车速渐渐加快,我们被拉得快断掉的手,啪一下松了。母亲哗一下扑倒在卧铺上。我不敢出声,看见母亲那个样子,我吓得不能动弹。
\par 我们就这样离开了南京。
\par 那是公元一九四九年底的冬天。
\par 总有人来问我:“三毛我们要去大陆了,你要什么东西?”
\par 我说:“请你心里为我带些雨花台的小石头来,就很感谢了。”
\par 听的人说:“上半年你只身去了大陆,光是江苏浙江就走了三十七天,难道没有去南京吗?”
\par 我笑了笑,摇摇头。
\par 父亲说:“对呀——你这次回大陆怎么没有去南京看看呢?”我说:“肯定碰到明珠姐姐,如果我去头条巷。”母亲骇了一跳,说:“明珠不是死在‘文化大革命’了吗?”“没有。”我说:“假如我这次走进南京的老房子——我当然先向书房走去,人还正在花园里呢,背后会有笑声说——妹妹这一觉睡得好长,都黄昏了才起来。看——姐姐手里什么好东西,过来拿呀——”我说:“明珠姐姐就站在我背后假山上,手里面捧着那同治年间粉彩小花水缸,笑着向我招手哪——”父亲说:“你又来吓人了。”我说:“我可是被吓了一跳,问说——明珠姐姐你不是死在沈阳的吗?怎么倒来吓我?姐姐笑着说——妹妹可真是睡蒙过去了,尽说胡说——看,四岁多的娃娃了还不知道梳头洗脸,不看江妈又要来数落你了。”母亲说:“好了,快吃饭,不要再做白日梦了。”“对啦!明珠姐姐也说——妹妹不过做了一场梦。什么台湾欧洲非洲美洲的,不看哥哥姐姐都还在大学中学,妹妹到底怎么环游世界去了。都是墙外边那面英国旗子飘啊飘地把妹妹梦里飘零四十年——”
\par 母亲看了我一眼,把个电视遥控器轻轻一压——民进党正在演讲,桌子拍得好大声呀——那声音淹没了明珠姐姐的讲话,我笑了起来。
\par 我看见了,就在三百八十度电视机画面中间的我,我正用自己的脚踪,再度走向南京的故居,在那夕阳将尽的黄昏。我轻轻按铃,站在门外等待。夜茫茫。让我进去可不可以?我是以前这幢房子的住客。重尊无处——嗳——一切的东西都缩小了尺寸。觉来小园行遍。让我上去图书室好吗——明珠姐姐——异时对——明珠姐姐你在家吗——燕子楼空——那怎么连江妈也看不到了呢——好——当它是——来呀——三毛——古今如梦——我们在这梧桐树下合拍一张照片好不好——对——用镁光灯——笑呀——不要叹气嘛——一二三我们笑呀——看,这黄楼夜景多么美丽——还有这秋天的月亮当头照着——好了。快。拍好了就快走吧。车子在等。后天我们飞回台北就快快去冲底片了——
\par 如果,我青石板的街道——哒哒的马蹄——是个过客——不是不是归人——我——不带走一片云彩——我——挥一挥手——我——走了——如果这也要参破成空——望断成空故国成空心眼成空——那一个失去了梦的人,活得活不下去又活得活不下去——小姐可是南京人——大伯伯你我可是个中国人。


\subsection{夜半逾城\\\small{敦煌记}}


\refdocument{
    \par 印度悉达多太子十九岁时,有感人世生老病死各种痛苦,为了寻求解脱诸苦方法,决定舍弃王族生活,于一日夜间乘马逾越迦毗罗卫城到深山修道。悉达多骑马上,驭者车匿持扇随行马后。天人托着马足飞奔腾空而去。空中飞天一迎面散花,一追逐前进。
    \par \rightline{——敦煌莫高窟\ 三七五窟\ 西壁龛南侧壁画故事}
}
\par “那么你是后天早晨离开吗?”父亲说。
\par 我说:“是。”
\par “好,祝你旅途愉快了。”父亲又说。
\par 我谢了父母,回到自己温暖的小楼来坐了一夜。天亮了,再静坐到黄昏,然后慢慢走路去了父母家。
\par “咦,我们以为你不再来了。等等[插图],我们看完这个电视剧。”父母说。
\par 我等了十数分钟。坐了一会儿。
\par “那么我走了。”我说。
\par “好,祝你旅途愉快哦!”
\par “谢谢。”我轻轻说,再深深地看了父母一眼。
\par 回家之后,将房子上上下下的尘埃全部清除,摸摸架上书籍、拍松所有彩色靠垫、全部音乐卡带归盒、屋顶花园施上肥料浇足水、瓦斯总门确定关好、写了几封信贴足邮资,这才打开衣柜,将少数衣裳卷卷紧,放进大背包里去。拿了一本书想带着行路——《金刚经》,想想又不带了。
\par 离开家的清晨,是一个晴天,我关上房门之前,再看了一眼这缤纷的小屋,轻轻对它说:“再见了。我爱你。谢谢。”
\par 
\par 当我亲眼看见那成排的兵马俑就立在我面前时,我的心跳得好快,梦境一般的恍惚感,再度成为漩涡,将我慢慢、慢慢,卷进一种奇异的昏眩里去。
\par 去年在江南的时候,也是这个样子。
\par 这是我第二次归去。
\par
\par 当国际旅行社的海涛在嘉峪关机场接到我的时候,我笑着跟他握手,彼此道了辛苦。
\par 一路上舟车的确紧张,行色匆匆,总也不感觉人和天有着什么关系,直到进入河西走廊,那壮阔的大西北方才展现了大地的气势。
\par 车子到了嘉峪关的城关口,海涛说下来拍照,然后再上车开进去。
\par 我没有再上车,将东西全部丢在座位上,开始向那寸草不生的荒原奔去。
\par 在那接近零度的空气里,生命又开始了它的悸动,灵魂苏醒的滋味,接近喜极而泣,又想尖叫起来。
\par 很多年了,自从离开了撒哈拉沙漠之后,不再感觉自己是一个大地的孩子、苍天的子民。很多人对我说:“心嘛,住在挤挤的台北市,心宽就好了呀。”我说:“没有这种功力,对不起。”
\par 海涛见我大步走向城墙,一不当心又跑了起来,跑过他身边的时候,海涛说:“是太冷了吗?”我说:“不是。很快乐。跑跑就会平静下来的。”
\par 站在万里长城的城墙上。别人都在看墙,我仰头望天。天地宽宽大大、厚厚实实地将我接纳,风吹过来,吹掉了心中所有的捆绑。
\par 我跑到无人的一个角落去,[插图]——长啸了一下,却吓到了躲在转弯墙边的一对情侣。我们三个人对视了几秒钟,我咯咯笑着往大巴士狂奔而去,没有道歉。
\par 趴在窗口等开车的时候,远处那驻守的解放军三三两两地正在追逐嬉耍——他们也在跑着玩。我笑了起来。
\par 
\par 离开了嘉峪关,我的下一站是敦煌。
\par 海涛说,休息吧,接着而来的七八小时车程全是戈壁——戈壁就是荒原的意思。
\par 荒原的变化是不多的,它的确枯燥——如果你不爱它。车上的人全都安静了,我睁大着眼睛,不舍得放过那流逝在窗外的每一寸风景,脑海中那如同一块狗啃骨头形状的地图——中国甘肃省,又在意识里浮现出来。
\par 而我这一回,将这辆行走的巴士和我自己也放进想象的地图中间去,一时里,那种明显的漩涡再度开始旋转,我又不能控制地被卷进了某种不真实的梦境里去。它,这一回掺杂进了那条《大黄河》的音乐曲调作为背景,鬼魅一般占住了我全部的思绪。
\par 虽然外边起了大风暴,我还是悄悄推开了那么一公分的窗框。为着担心坐在我身后的人不喜欢,我回了一下头。
\par 我回过身来,将窗子砰一下关了起来,心里惊骇到不能动弹:“怎么会是他?”
\par 我不敢再回头,呆呆地对着窗外,我听见有声音在说:“原来你在这儿。”
\par 这原是两个人的位子,却是我一个人给坐了,当然是我自己在对自己说话。又有声音说:“去年在姑苏的时候,林妹妹先用一块雪白的丝手帕托人在一场宴会里悄悄送上,等到我上了那条运河从水道去杭州的时候,她左手戴了一只空花的白手套出现在岸边哭得死去活来地送别——”
\par 我疑疑惑惑地再度回头,又看见了那光头的青年。我接触到他那双眼睛,我再度回过身来看着窗外那连绵到天边的电线杆,又听见自己在说同样的话:“宝玉,原来你在这儿。”
\par 这时,昏眩的感觉加重了,我对自己说:“不好了,今生被这本书迷得太厉害,这不是发疯了吧?为什么一到中国,看见的人全是它的联想,包括大西北也扯上了宝玉和出家。”
\par 我不敢再回头,拿出喷水小壶来,往脸上喷了一些凉水。
\par 一时里我发觉我已经站在那个青年人的座位前。我们含笑望着彼此。我说:“你从哪里来?”他说:“兰州。”我说:“你到哪里去?”他说:“敦煌。”我说:“你去敦煌做什么?”他说:“我住在莫高窟。”我说:“你在莫高窟做什么?”他说:“我临摹壁画。”
\par “你怎么会临摹?”
\par “我不知道。”
\par “学的?”我说。
\par “小时候就会了。”
\par 我说:“我认识你。”他说:“我也认识你。”
\par 我笑说:“我是谁?”他说:“你是三毛。”
\par 我觉得疲倦如同潮水般地淹住了我,又有声音在我心里响起:“我以为,你会说,你认识我——因为我是你的三姐姐探春,不然、不然,好歹我当年也是你们大观园里的哪一个人——”
\par 我又对他笑笑,我们就是微微地笑着。后来,我坐回了自己的位子,两三小时,不再讲话。
\par 再回首的时候,那个青年拿手掌撑着面颊,斜躺在座位上。
\par 一霎间,宝玉消失了。他不是。
\par 
\par “小兄弟,看你是一座涅槃像。”我笑说。
\par 车里的人听我这么说,都开始看他。他抿抿嘴,恬散的笑容,如同一朵莲花缓缓开放。
\par “你叫什么名字?”我说。
\par “伟文。”
\par “一九六七年出生的。”我肯定地说,不是问句。
\par “对了。”
\par 旁边的一位乘客插进来说:“那请你也看看我是哪一年生的。”我说:“没有感应不行的。”笑指着伟文,又说:“他的生肖是——”我心里想的超出了十二生肖,我心里说:“他是蟾蜍。”
\par “我是青蛙。”伟文突然说。
\par 我深深地看了伟文一眼,一笑,走了。
\par 那个傍晚,我们抵达了敦煌市。
\par 我将简单的行李往旅馆房间里一丢,跑下楼去吃了一顿魂不守舍的晚饭,这就往街上走去。
\par 海涛说:“今晚起大风,可惜没得夜市了。三毛加件衣服,认好路回来。”
\par 我说:“没事。”这句没事在大陆非常好用。
\par 无星无月的夜晚,凛冽的风,吹刮着一排排没有叶子的白杨树,街上空空荡荡,偶尔几辆脚踏车静悄悄滑过身边、行人匆匆赶路、商店敞开着、没有顾客,广场中心一座“飞天”雕像好似正要破空而去。
\par 我大步在街道上行走,走到后来忍不住跑到街道中间去试走了一段——没有来车,整条长长的路,属于我一个人。我觉得很不习惯,又自动回到人行道上来。另一个旅者,背着他的背包,戴着口罩与我擦肩而过。这时我看见有旅舍外边写着:“住宿三元。”
\par 一时里,我的思绪又把正在走路的自己,给夹进了那几本放在台北家中书架上的“敦煌宗教艺术”的书籍里去混成一团。天是那么的寒冷,我被冻在一种冷冷的清醒里面。
\par 这时候经过一家大商场,想起来这一路过来都是用手指梳头的,进去买一把梳子倒也很好。我一个一个的柜台看过去,对于那些乡土气息的大花搪瓷杯碗起了爱恋之心,可是没有碰触它们。付完了梳子钱,我说:“同志,你没有找我钱。”那位同志叫喊起来:“我明明找给你了。”我打开腰包再看,零钱就在里面。那时候,隔壁一个柜台在放录音带,他们把扩音机放得震天价响,我听见罗大佑的《恋曲一九九〇》在大西北之夜里惆惆怅怅地唱着——或许明日太阳西下倦鸟已归时,你将已经踏上旧时的归途——一幅巨大的标语在路灯下高悬——“效法雷锋精神”。
\par 我进入了另一种时空混乱的恍惚和不能明白,梦,又开始哗哗地慢慢旋转起来。
\par 就在那个邮箱的旁边,我又看到了他。
\par “伟文。”我说,“今天是一九八九年几月几日?”
\par 伟文看着我手中拿着的小录音机,轻轻摇头说:“三毛。你怎么了?”
\par 我哦了一声,没有做什么解释,笑起来了。
\par 伟文和我完全沉默地开始大街小巷地走着。风,在这个无声的城市里流浪,夜是如此的荒凉,我好似正被刀片轻轻割着,一刀一刀带些微疼地划过心头,我知道这开始了另一种爱情——对于大西北的土地,这片没有花朵的荒原。
\refdocument{
    \par 亲爱的朋友,我走了。
    \par 当我在敦煌莫高窟面对“飞天”的时候,会想念你。谢谢多年来真挚的友情。再见的时候,我将不再是以前的我了。
    \par \rightline{爱你的朋友三毛}
}
\par 离开台湾之前,我把三五封这样的信件,投进了邮箱,又附上一九九〇年四月四日拍摄的照片,清楚注明日期,然后走进了候机室。
\par 一路上,其实不很在意经过了什么地方又什么地方,只有在兰州飘雪的深夜里看到黄河的时候,心里喊了她一声母亲。那一夜我没有阖过眼。
\par 敦煌的夜晚,在旅馆客厅里跟海涛、伟文,一些又加进来的国内朋友坐了一会儿。我变得沉静,海涛几次目示我,悄悄对我说:“三毛,去睡。”我歉然地站起来道了晚安,伟文叫住我,举起了我遗落在沙发上的小背包,我笑着摇摇头说:“不行,太累了。”
\par 其实我正在紧张。潜意识里相当的紧张。
\par 明天,就是面对莫高窟那些千年洞穴和壁画的日子。
\par 我的生命,走到这里,已经接近尽头。不知道日后还有什么权利要求更多。
\par 那一夜,我独自在房间里,对着一件全新的毛线衣——石绿色,那种壁画上的绿,静静地发愣。天,就这么亮了。
\par 
\par 我又看见了海涛和伟文,在升起朝阳的清晨。
\par “早上好。”我笑着打招呼。
\par “你完全变了一个人。”伟文说。
\par 我笑着掠了一下梳洗清洁的头发,竖竖外套的领子,说:“过了今天,还会再有更大的变化。”
\par 那时候广场上有人陆陆续续上来请求一起拍照。我把海涛一拉,说:“来,我们来拍照。”
\par 我将他拉开了人群,小声说:“海涛,我要跟你打商量,今天,是我的大日子,一会儿这一车的人到了莫高窟,你负责他们参观的事情,我会一下子就不见了。你不要找我也不要担心我不回市区里来吃中饭。到了黄昏,我自会找到你的车子回来。放心。”
\par 这时候三五个人过来问我:“三毛,兵马俑和莫高窟比起来你怎么想呢?”
\par 我说:“古迹属于主观的喜爱,不必比的。严格说来,我认为,那是帝王的兵马俑,这是民间的莫高窟。前者是个人欲望和野心的完成,后者满含着人类对于苍天谦卑的祈福、许愿和感恩。敦煌莫高窟连绵兴建了接近一千年,自从前秦苻坚建元二年,也就是公元三六六年开始——”
\par 我突然发觉在听我讲话的全是甘肃本地人,这一下红了脸,停住了。
\par 其实,讲的都是历史和道理。那真正的神秘感应,不在莫高窟,自己本身灵魂深处的密码,才是开启它的钥匙。
\par 在我们往敦煌市东南方鸣沙山东面断崖上的莫高窟开去时,我悄悄对伟文说:“你得帮我了,伟文,你是敦煌研究所里的人。待会儿,我要一个人进洞子,我要安安静静地留在洞子里,并不敢指定要哪几个窟。我只求你把我跟参观的人隔开,我没有功力混在人群里面对壁画和彩塑,还没有完全走到这一步。求求你了——”
\par “今天,对我是一个很重要的日子。”我又说。
\par 当那莫高窟连绵的洞穴出现在车窗玻璃上时,一阵眼热,哭了。
\par 海涛宣布停车照相的时候,我站在结冰的河岸边、白杨树林的枯枝下,举起相机拍了的——不是那些洞穴。
\par 
\par 当那西北姑娘,研究所里工作的小马——马育红,为我把第一扇洞穴的门轻轻打开时,我迟疑了几秒钟。“要我为你讲解吗?”小马亲切地问。“我持续看过很多年有关莫高窟的书,还有图片。”我说。伟文拉了她一下。我慢慢走进去,把门和阳光都阖在外面了。
\par 我静静站在黑暗中。我深呼吸,再呼吸、再呼吸——
\par 我打开了手电棒,昏黄的光圈下,出现了环绕七佛的飞天、舞乐、天龙八部、胁侍眷属。我看到了画中灯火辉煌、歌舞蹁跹、繁华升平、管弦丝竹、宝池荡漾——壁画开始流转起来,视线里出现了另一组好比幻灯片打在墙上的交叠画面——一个穿着绿色学生制服的女孩正坐在床沿自杀,她左腕和睡袍上的鲜血叠到壁画上的人身上去——那个少女一直长大一直长大并没有死。她的一生电影一般在墙上流过,紧紧交缠在画中那个繁花似锦的世界中,最后它们流到我身上来,满布了我白色的外套。
\par 我吓得熄了光。
\par “我没有病。”我对自己说,“心理学的书上讲过:人,碰到极大冲击的时候,很自然地会把自己的一生,从头算起——在这世界上,当我面对这巨大而神秘——属于我的生命密码时,这种强烈反应是自然的。”
\par 
\par 我仆伏在弥勒菩萨巨大的塑像前,对菩萨说:“敦煌百姓在古老的传说和信仰里,认为,只有住在兜率天宫里的你——‘下生人间’,天下才能太平。是不是?”
\par 我仰望菩萨的面容,用不着手电筒了,菩萨脸上大放光明灿烂,眼神无比慈爱,我感应到菩萨将左手移到我的头上来轻轻抚过。
\par 菩萨微笑,问:“你哭什么?”
\par 我说:“苦海无边。”
\par 菩萨又说:“你悟了吗?”
\par 我不能回答,一时间热泪狂流出来。
\par 我在弥勒菩萨的脚下哀哀痛哭不肯起身。
\par 又听见说:“不肯走,就来吧。”
\par 我说:“好。”
\par 这时候,心里的尘埃被冲洗得干干净净,我跪在光光亮亮的洞里,再没有了激动的情绪。多久的时间过去了,我不知道。
\par “请菩萨安排,感动研究所,让我留下来做一个扫洞子的人。”我说。
\par 菩萨叹了口气:“不在这里。你去人群里再过过,不要拒绝他们。放心放心,再有你回来的时候。”
\par 我又趺坐了一会儿。
\par 菩萨说:“来了就好。现在去吧。”
\par 
\par 我和小马、伟文站在栏杆边边上说着闲话,三个人,透着一片亲爱祥和。
\par “伟文,为什么我看过的这些洞子里,只有那尊弥勒菩萨的洞顶开了天窗,这样不是风化得更快了吗?菩萨的脸又为什么只有这一尊是白瓷烧的呢?”
\par 伟文说:“没有天窗。不是瓷的。”
\par “可是我明明没有举手电棒,那时候根本是小马在外边替我拿着电棒的。有明显的强光直射下来,看得清清楚楚。”我说。
\par 伟文看着我,说:“我不知道。”
\par 我一掉头,开始去追其他的参观者,我拦住一个,问他弥勒菩萨是什么样子,我听了不相信,又拦住了两个人追问。他们一致说:“太高了,里边暗暗的,看不清楚什么。”
\par 我腿软,坐了下来,不能够讲一句话。
\par 一群人等在栏杆外大树下,叫喊:“三毛,下来让我们合照呀——”
\par 伟文说:“可以绕这边走,再躲一下。累不累?”
\par 我说:“不累。不让人久等,我们过去吧。”
\par 当我在空无一人的柏油路面上踩着白杨树影行走的时候,海涛带队的大巴士由后面开过来,突然刹车了。台湾同胞蔡健夫从车上双手递下来两本由他助印的经书,说:“三毛,前人藏经在莫高窟,我们要把这份工作再延续下去,接好了,你一会儿代表交给敦煌研究所了。”
\par 我在阳光下打开了一本大红封面的经书,赫然发现那一段正在发愿,愿在来生一愿如何、二愿如何、三愿如何……我看到书中第八愿的时候,伟文匆匆跑上来,我将经书一合。
\par 伟文说:“走了,去我们所里吃中饭。”
\par 我笑说:“嗳。”
\par 那一路,我对自己说,这又是一次再生的灵魂了,不必等待那肉身的消亡。那第九个愿其实我已看到了半段,伟文恰好上来将我阻住,那么就在今生自自然然去实践前面的几个发愿心也是好的。
\par 跟伟文在食堂里吃过了中饭,研究所里的女孩子们请我去她们宿舍里去坐坐,我满含感激地答应了。
\par 往宿舍去的小路上,一个工作人员跑上来拦住了我,好大声地说:“三毛,我得谢谢你,当初我媳妇儿嫌我收入不高又在这么远离人烟的地方工作,不肯答应我的求婚,后来她看了你的书,受到了感动,就嫁给我了。现在呀,胖儿子都有了,谢谢你大媒。”
\par 我握住这个人的双手,眼里充满了笑意。
\par “远离人烟吗?真的。就我们所里这一百多人住在这里。一星期嘛,有车进一次城。冬天游客不来了,更是安静。”一个会讲德语的女孩子说,她是接待员。
\par “想离开吗?”我靠在床上问她们。
\par “想过。真走到外边儿去,又想回来。这是魔鬼窟哦——爱它又恨它,就是离不开它。”
\par “有没有讲西班牙文的接待员?”我问。
\par “什么文都有,就少西班牙文人才。”
\par 我心跳得好快。把手去揉胸口。
\par “累了?三毛睡一下。”
\par 我摇摇头,说:“明天要去吐鲁番了。舍不得。”
\par 女孩子们说:“那就留下来了。”
\par 我把衣袖蒙住了眼睛,说:“来了就好。现在得去。没有办法。”
\par 
\par 黄昏了,我们在莫高窟外面大泉河畔那成千的白杨树林里慢慢地走,伟文不说什么话,包括下午我们再进了一个洞,爬架子,爬到高台上去看他的临摹,他都不大讲话。我们实在不必说什么,感应就好了。
\par “那边一个山坡,我们爬上去。”伟文说。
\par 我其实累了,可是想:伟文不可能不明白我身体的状况,他想带我去的地方,必然是有着含意的。
\par 我们一步一步往那黄土高地上走去,夕阳照着坡上坐着的三个蓝衣老婆婆,她们口中吟唱着反复而平常的调子:“南无阿弥陀佛——南无阿弥陀佛——南无阿弥陀佛——”一面唱着一面用手拍打着膝盖,那梵音,在风中陪着我一步一步上升。经过老太太们时,伟文说:“距离这里四十公里的地方,有一座佛寺,老太太们背着面粉口袋,走路去,要好几天才回得来,她们在寺里自己和面吃。”我听着听着,就听见好像是老太太在说:“好了、好了。来了、来了。”
\par 山坡的顶上,三座荒坟。那望下去啊——沙漠瀚海终于如诗如画如泣如诉一般地在我脚下展开,直到天的终极。
\par 我说:“哦——回家了。就是这里了。”
\par 伟文指指三座沙堆做成的坟,只用土砖平压着四周的坟,说:“这是贡献了一生给莫高窟的老先生们,他们生,在研究所里,死了也不回原籍,在这里睡下了。”又说:“清明节刚过,我们来给他们上坟呢!”
\par 一个被风弄破了的纸花圈,在凉凉的大气里啪啪地吹打着。明亮的大红、橘黄、雪白是这片沙地上特别寂寞的颜色。
\par “伟文,你也留在这里一辈子?”我说。
\par “嗳。”
\par “临摹下来的壁画怎么保存呢?”
\par “库存起来。有一天,洞子被风化了,还有我们的记录。”
\par “喜欢这个工作吗?”
\par “爱。”
\par “上洞子多少年了?”
\par “五年。”
\par “将来你也睡在这儿。”
\par “是。”
\par 夕阳染红了这一大片无边无际的沙漠,我对伟文说:“要是有那么一天,我活着不能回来,灰也是要回来的。伟文,记住了,这也是我埋骨的地方,到时候你得帮帮忙。”
\par “不管你怎么回来,我都一样等你。”
\par “好,是时候了。”我站起来,再看了一眼那片我心的归宿,说,“你陪我搭车回敦煌市去了。”
\par “小马,再见了。莫高窟的一扇扇门,是你亲手为我打开。我会永远记得感激你。”我紧紧地拥抱着小马。一撒手,大步走去,不敢回头。
\par 海涛问我这莫高窟的一日过得如何,我点点头,一笑,上车了。
\par 伟文一路跟车子送到敦煌市,他手里一个袋子也没有的,卷着一团布,也不知做什么。
\par 我跑回敦煌市的旅馆里,快快脱下了那件V字领的毛线衣,放在一个小包包里面。
\par “伟文,快,今晚有夜市,我们去坐露天茶馆吃小摊子。”我接近欢悦地叫喊起来。
\par “吃摊子吗?”
\par “不然呢?吃饭店多么辜负了地方风味。”
\par 我半躺在露天茶座上,用厚外套盖住自己。今天没有风暴,满街的人们,不挤的一种活泼,将这敦煌衬得另是一番流丽风情。
\par 夜来了,我得回旅馆。而我实在舍不得。
\par “你是从壁画上下来接我的对不对?”我又问一遍伟文。
\par 他开玩笑地说:“是。”
\par “不过,你不是佛,你是一种——嗯——弟子。这是我的感觉。”
\par 伟文指指乍一下亮起来的霓虹灯,说:“看灯。”
\par “哦,很好看。”我赞叹着人间灯火,受到了很真切的感觉。而那广场中间白色的塑雕“飞天”依旧舞出了她那飞上天去的姿势。
\par “这不过是塑像罢了,真的她,早就飞来飞去了。”
\par 我指指广场中心,向伟文笑笑。
\par 这时,台湾来的同胞向我叫过来——他们也在街上,“三毛,我们去跳舞,来嘛来嘛——我们去跳DISCO,吔吔吔——”一个宝贝蹲在我座位旁边扭来扭去。
\par 我笑着把他们挥挥走,亲爱地啪一下轻打了那个台湾青年的头。整条街上又饱满了这样在唱着的歌——轻飘飘的旧时光就这么溜走,转头回去看看时已匆匆数年,苍茫茫的天涯路是你的飘泊……
\par 我仍旧在想为什么那个弥勒大佛在我眼中变成白瓷面孔?又在想那照明给我看的光束为何别人都没有看见的问题。侧过去看看伟文,他手里卷着的那包布料轻描淡写地递了过来。我突然发觉伟文像极了他正在临摹壁画的洞子——那位站在南无本师释迦牟尼佛身边的大弟子——阿难。
\par “这是我很爱的一件衣服,还有一本有关敦煌的书、几套敦煌壁画的明信片,你带去了做个纪念。”伟文说。
\par 我慢慢打开了那块灰色的布料——一件小和尚的僧衣,对襟开的,在我手里展开。
\par “我喜欢。谢谢你。”
\par 我的手抚过柔软棉布的质地,抬眼看了一下穹苍,天边几颗小星星疏疏落落地挂了上来。
\par “明天我要走了。”我轻轻说。
\par “嗳。”
\par “以后的路,一时也不能说。”我说,“我们留地址吗?”
\par “都一样。”伟文说。
\par “我也是这么想。”我又说,“我看一本书上说,我们甘肃省有一种世界上唯一的特产,叫做‘苦水玫瑰’,它的抗逆性特别强韧,香气也饱含馥郁,你回去,告诉所里的女孩子,她们就是。”
\par “知道了。”
\par “年纪轻轻的,天天在洞子里边面壁,伟文,我叫做——这是你的事业,不是企业,我们知道做事情和赚钱有时候是两回事的,对不对?”我说。
\par “我也是这么看法。”
\par “谢谢你们为敦煌所做的事情。也谢谢你给我这两天的日子。”
\par “没事。”
\par “我给你讲个故事,就散了。”我开始说,“很久以前,一个法国飞机师驾着飞机,因为故障,迫降在撒哈拉沙漠里去。头一天晚上,飞机师比一个飘流在大海木筏上面的遇难者还要孤单。当天刚破晓的时候,他被一种奇异的小孩声音叫醒,那声音说——请你……给我画一只绵羊……”
\par 伟文很专心很专心地听起《小王子》的故事来。
\par “很多年以后,如果你偶尔想起了消失的我,我也偶然想起了你,伟文,我们去看星星。你会发现满天的星星都在向你笑,好像铃铛一样。”
\par “嗳。”
\par “记住我选的地方了?那个瞭望沙漠的小坡?”
\par “记得。”
\par 我们在一个十字路口站住了,旅馆在我的背后。我拿出放着绿色毛衣的口袋来,紧了一紧伟文送给我的衣服。
\par “伟文,恰好我要给你的纪念,也是一件衣服。现在我把我的颜色,亲手交给你了。”
\par “好,我收下。”
\par 天是那么的黑,因为没有月亮。
\par 我看见伟文的一双眼睛,寒星一样看住了憔悴的我。
\par 我知道,我们再也不会也不必联络了。
\par 我再看了伟文最后一眼,他的身后,那DISCO的霓虹灯和“飞天”同时存在着,一前一后。
\par “那么我走了。”我说。
\par “嗳。”伟文抿了抿嘴唇,重重地点了一下头。
\par 我转身,慢慢、慢慢往天边的几颗星星走上去,口袋里那把旅馆钥匙,被我轻轻握在掌心中。


\subsection{附录}


\subsubsection{一封给邓念慈神父的信}

\par \leftline{敬爱的邓神父:}
\par 收到您的来信的现在,我正在巴西旅行。这封信经过《联合报》转到台北我父母的家中,因为是限时信,很抱歉地由我父亲先代为拆阅了,然后转到巴西给我。
\par 拜读了您的英文信之后,我的心里非常的难过与不安,在我的文字中,无意间伤害到了您的情感和国家,虽然并不是故意的,可是这件事情的确是我个人在处置上的粗心和大意。
\par 身为一位哥伦比亚的公民,在看到了我对于他自己国家的报道上有所偏差时,必然是会觉得痛心的。您写信向我抗议是应当的行为。这一点,如果我与您换了身份与国籍,也一定会向作者写出同样的信来,在这儿,我要特别向您以及您的国家道歉。
\par 因为我这次旅行,在哥伦比亚恰巧碰到了一些不诚实的事情,首都博各答的治安也因事先阅读书籍的报道而影响了我的心理,因此便写了出来。事实上,世界上任何国家,每一个城市,每日都有大小不同的暴行在发生,这不只是哥伦比亚,是全球人类的悲哀和事实,不巧我的文字中记录下来的只有一国,这当然是不公平的。尤其使我歉疚的是——我深深地伤害到了一位为着我们中国人而付出了爱与关心的神父,这是我万万不愿意的。
\par 在我旅行结束回到台湾去时,请您千万给我一个补过的机会,请您答应见我,接受我个人的道歉。希望这件事情能有一个挽回的机会,不但是向您私人致歉,我也有义务将这封信发表,算做对哥伦比亚这个国家的歉意。
\par 我们都是有信仰的人,对于这个美丽的世界和生命,除了感恩之外,必然将天主的爱也分布到人间。您,早已做到了这一点,而我,却在这份功课上慢慢学习。爱,是没有国籍也没有肤色之分的,这份能力来自上天,失了它,我们活着又有什么其他的意义呢!
\par 看完您的来信已经一天了,可是我心中的愧疚不能使我安睡,请您了解我的真诚,但愿因为这一篇文字,而使我们因此做了朋友。回到台北时,我要来“耕莘文教院”拜望您,如果您肯接见我,当是我最大的欢喜,因为可以当面向您解释和交谈,也但愿您对我的粗心大意能够有所教导,都是我当向您学习的地方。
\par 许多的话,说出来并不能减轻我内心的负担,可是这封信是一定要写的,请您原谅,宽容,实在是十分对不起。急着回来见您!
\par 敬祝
\par \leftline{安康}
\par \rightline{晚}
\par \rightline{三毛敬上}

\subsubsection{飞越纳斯加之线}

\rightline{米夏}

\par 小型飞机终于从崎岖不平的碎石跑道上起飞了,飞进沙漠的天空,早晨的空气清凉又干爽。我心里在想:“又要飞了。”
\par 又飞了,不过,这一趟空中之旅就是不一样。自从三毛和我去年离开台湾,我们曾经飞过千山万水,飞越过成千上万各有悲欢离合的芸芸众生。
\par 每一次在飞机降落之后,我们刚刚才看清楚一片新土地,也才揭开这片土地的一点点秘密,不过,只有一点点。一个人穷毕生之力也不足以完全了解一个地方,包括我们自己的家乡在内。时间过得太快,我们还没准备妥当,就又要上飞机了。
\par 我坐在驾驶员的旁边,小飞机起飞的时候,他在胸前画十字,我心里就在想:“这一趟一定跟以前不一样。”他的举动给我一种很奇怪的感觉。由于这趟旅程的终点充满了神秘色彩,驾驶员的举动倒很适合这种气氛。
\par “纳斯加之线嘛!”三毛说。
\par “什么线?”我回问三毛。在我们前往秘鲁途中,三毛问我知不知道这个有名的古迹。
\par “我们马上就要到秘鲁了,难道你对南美洲最令人不解的谜竟然一无所知吗?”
\par “我当然知道,每个人都知道,玛丘毕丘,印加帝国失落的古城,对不对?”
\par “不对啦,那是一个废墟,是印加人过去居住的地方,唯一令人不解的是,他们为什么放弃了那个城市。我现在说的是一个直到今天都没有人能解开的谜。”
\par “什么谜?”
\par “你没有看过登尼肯(Von Daniken)的书,还是根本没听说过他的书?”
\par “谁的书?”我问。她每提一个问题,我就愈发觉得自己没知识。三毛看过不少杂书,她看西班牙文、德文书,当然还有中文书,虽然她自谦英文不行,但无损于她阅读英文作品。三毛不仅看书,而且过目不忘。
\par 她不仅看书过目不忘,她对看到的东西,吃过的东西,在哪里吃,跟谁一起吃的,以及价钱多少,都有很好的记性。
\par 有一天,她真令我大吃一惊,她能记得十一年前住在芝加哥时香肠卖多少钱,并且拿来跟利马市华埠香肠的价钱相比。
\par 在这次旅行中,我不只一次觉得自己像个笨瓜,这个中国女孩子总会问出一些我从未念过或记不得的事情。
\par 三毛像老师教笨学生一样,很有耐心地向我解释:“登尼肯是一个作家,他写了一本书,谈到我们这个世界上有些未解开的谜,他认为这些奥秘与地球以外的生命有关。”
\par “我不是从他的书里第一次听说纳斯加之线,但是,我看了他的书以后,就很想到秘鲁观光,亲自看一看。”又说。
\par 飞机把我带到了纳斯加这个绿洲小城的上空,“亲自看一看”这句话还在我的脑际回响。纳斯加坐落在秘鲁南方的大沙漠中。
\par 从空中看,这个小城像一个绿色的岛,大片的荒漠一直伸展到地平线上的山脉,只有这一小片绿色。
\par 在我们的脚下,一天的作息刚刚开始。一个女人在井边洗她一头乌黑的长发,一座泥屋升起了袅袅炊烟。一对父子已经带着工具骑自行车上工了,母亲和儿媳妇留在家里。
\par 一屋又一屋,一街又一街,到处都有日常的活动。在我这趟飞行中,至少有一小段时间没有把我跟我熟悉的日常生活完全隔离。
\par 飞机飞过城中心的时候,我往下看那家旅馆,三毛想必还在床上休息。
\par “实在是不太对。”我觉得,“她才应该在飞机上,去看沙漠中的那些神秘的巨大图案,不该由我去。”
\par 我心里很难过,因为三毛竟不能去看这些神秘的古迹,她一直认为这些东西真是南美洲比较重要、比较有趣的一景。
\par 说实在的,她已无法上飞机。在前往纳斯加途中,三毛开始晕车,因为长途公车在秘鲁崎岖的道路上行驶,颠得厉害。
\par 公车愈往前行,她晕得愈厉害。几个小时她都默默不语,一手按在头上,一手按着肚子,后来,她喘着气说:
\par “我晕得好像要死了!”
\par “我们下一站一定要下车!”
\par “不行!”
\par “但是,你病得很重,不能再走。”
\par “没关系,我们一定要到纳斯加。”三毛很坚决地说。
\par 这是她典型的个性。一旦她下定决心,什么事也阻止不了她达到目标。
\par 经过大约五百公里的折磨,深夜里我们终于到了纳斯加。感谢上天,公车站附近有一家旅馆,我们住进去的时候,三毛已经十分虚弱了。
\par “米夏,我告诉你,我真的病了。”我扶她进房间的时候,她很痛苦地说。
\par “吃一点药,好好休息。”
\par “明天我不能飞了。”三毛有气无力地说。
\par “什么?”我简直不能相信刚才听到的话,我知道她累疯了,身上有病痛,但是,我认识中的三毛不会就此罢手。
\par “你不知道自己在说些什么。今晚好好休息,我们明天再谈。”
\par “我不行。”
\par “可是,你盼望了那么久,跑了那么远的路。”我表示不平。
\par “别傻了,你今天已经看到我在公车上是什么样子。如果我坐那架小型飞机飞上天,我会晕死。”
\par “我们能不能买些什么药来?”
\par “以前试过所有这一类的药,没有一种管用。即使到兰屿,只坐很短时间的飞机,下飞机的时候我也快要死了。”
\par “那你为什么要到纳斯加来,你明知纳斯加之线只有从空中才能看到?”
\par “我以为我可以勉强自己,可是,经过今天在公车上的情形以后,我知道我在空中支持不到五分钟。”三毛深深叹口气,“你走吧,让我休息!”
\par 飞机飞过旅馆上空,我希望她好好休养。我还是不相信她竟会放弃这个机会,不过,我知道,她一定达到了体力的极限,才会忍痛这样决定的。
\par 仰望万里无云的碧蓝天空,我不禁要问,上天何其不公,为什么世间一个意志最强的女子,身子却经不起风霜。
\par 没有多久,我们已经离开纳斯加很远。我们还要在荒凉的沙漠上空再飞二十二公里,才能看到一个已经消失的文明所留下的巨大创作。
\par “你是哪里人?”有人用西班牙话问我。一上飞机,我就专心在想缺席的三毛,还没留意到飞机上其他的人。
\par 我朝说话的人望去,看到驾驶员笑着跟我招呼。
\par “美国人,”我用非常蹩脚的西班牙语回答,“你呢?”
\par “我是秘鲁人,不过,我母亲是意大利人,我父亲是法国人。”
\par 我很想多问一些关于他的事情,无奈我的西班牙语已经技穷,只好笑笑,大家都没再说话。
\par 其他的座位上只有两个年轻人,他们用德语交谈。虽然我是第三代的德裔美国人,可是,我对德语一窍不通。
\par 我觉得我跟他们有很大的距离,就像我与地面上的人相隔甚远。既然没有交谈的对象,我就设想,如果是三毛,而不是我在飞机上,情况会有什么不同。
\par 她的西班牙语和德语都说得很好,她的聪明活泼会透过语言发散出来,让人如沐春风。任何人如果跟三毛聊过五分钟,一定会念念不忘。她讲话就像玫瑰在吐露芬芳。
\par 在这趟单独飞行之前,我体会不出如果没有我的老板娘,这趟南美之行就不够圆满。
\par 沙漠很快就越过了,在破晓的阳光中,展现出一片到处都是石头的不毛之地,有一种寂静的美。
\par “我们马上就要到了。”我们的驾驶员说。
\par 他指向第一道线,我赶紧把照相机准备好。
\par 在我们底下,有一块绵延好几公里,至少有半公里宽的广大地区,看起来像飞机跑道。
\par 这条地带的边缘很平、很直,好像是用建筑师的直尺画出来似的。在界线以内的表面地区,没有任何石头,而且很平滑,与周围崎岖及多石块的沙漠恰成对比。
\par 一个甚至没有文字的农业文化,怎么会有这种技术造出这么庞大、这么精确的地界标呢?
\par 这是登尼肯在他的书中提到的一个问题。他对这个问题提出一个理论作为答案,他认为,纳斯加文化(在西元八百年达到巅峰,大约比印加帝国的兴起早四百年)不可能有足够的技术造出这样的地界标。登尼肯推论的结果是,这些纳斯加之线是地球以外生物的杰作,他们把这片沙漠当做降落的场所。
\par 这只是一个理论而已,而且很难证明是否正确。无论是谁铺的,这些线铺了许多。有些铺成长方形,有些是三角形,有些线成直角交叉。
\par 我们飞越的是一个布满了几何图形的沙漠,而且不知这些图形是怎么来的,可是,这还只是纳斯加之谜的一半。
\par 驾驶员指向地面,用英文说:“Monkey(猴子)。”然后把机身急转,让我们仔细看清刻在沙漠中的巨大图形。图形很简单,看起来像是出自儿童之手。
\par 沙漠中这一块地盘变成了动物园。我们飞越过鸟、鱼、蛇、鲸鱼、蜘蛛、狗,甚至还有一棵树的图案。
\par 这些图形最令人惊讶之处是体形庞大。其中有一只鸟,翅膀超过一百公尺。没有空中鸟瞰之助如何能刻出这些图形?为什么要刻这些图案?这是迄今仍未解开的谜。
\par 我们飞过一个小小的观测塔,此塔是由德国女子玛丽亚·雷奇所建,她花了将近三十年的时间研究这些奥秘。
\par 不过,花了那么多的时间,她只做了一个结论——这些庞大的线和动物图形可能是巨大天文历的一部分。她并且认为,这些线大约是在西元前一千年左右所建,远在纳斯加文化出现之前。
\par 直到今天,雷奇和登尼肯都不能证明他们的理论是对的,因此,纳斯加之线之谜仍然无人能解。
\par 我们的飞机在这个谜团的上空再盘旋几圈,让我们看这些动物图形和跑道最后一眼,然后飞回纳斯加。
\par 我们默默地离开这片沙漠,奥秘仍未揭开,只有山边一个由不知来历的古人所刻的巨大人形,在那里永久守望着迄今仍未解开的纳斯加之线之谜。
\\
\par \leftline{三毛注:}
\par 米夏并未在文中说明,其实在赴纳斯加之线以前,已在秘鲁全境做了近六十小时的长途公车之旅,因此体力不继,未能到空中去。不是晕车五百公里,是晕车近六十小时不退。



