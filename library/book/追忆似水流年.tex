



\section{追忆似水年华}


\par 书名:追忆似水年华(全七卷)
\par 作者:[法]马塞尔·普鲁斯特
\par 译者:许渊冲,周克希,徐和瑾,李恒基等
\par 出版社:译林出版社
\par 出版时间:2012-06
\par ISBN:9787544716239


\subsection{第一卷\ 在斯万家那边}


\subsubsection*{再版说明}

\par 译林出版社于1989年出版《追忆似水年华》中译本,是中国出版史上一个填补空白的重要事件。这部划时代的意识流小说进入中国读者的视线之后,二十多年来受到广大读者的喜爱,多次再版,并获得“全国优秀畅销书”奖、“全国优秀外国文学图书奖一等奖”等奖项,入选“60年中国最具影响力的600本书”。
\par 尽管译者都具有扎实的翻译功底,且翻译态度极为严谨,编辑在组织翻译、审读及编校等工作上亦颇费苦工,但因这部作品长达二百万字,文体复杂、经纬绵密,由十五位译者共担翻译重任,且首版出书时间紧迫,难免留下一些遗憾。鉴于此,我们决定再版《追忆似水年华》七卷本。
\par 本次修订参考专家及读者意见,重点主要在以下三个方面:一是按照规范,统一了七卷中前后不一致的人名、地名及专有名词;二是增删和调整部分注释;三是参照原文,对译本进行了谨慎的梳理和修订。
\par 因水平所限,本次修订工作难免会有不足乃至失误之处,恳请读者包涵,并能一如既往地提出宝贵意见。
\par \rightline{译林出版社}
\par \rightline{2012年5月}

\subsubsection*{编者的话}

\par 马塞尔·普鲁斯特是十九世纪末、二十世纪初法国伟大的作家。在法国乃至世界文学史上,他同巴尔扎克一样,都占据着极其重要的地位。特别是一九八七年以来,法国好几家有影响的出版社,竞相重新出版普鲁斯特的名作《追忆似水年华》;评论和研究普鲁斯特创作成就的各种学术活动,也在法国及欧美许多国家广泛地开展起来。这股热潮的重新出现,充分显示出普鲁斯特这部巨著的价值及其影响。
\par 《追忆似水年华》以独特的艺术形式,表现出文学创作上的新观念和新技巧。小说以追忆的手段,借助超越时空概念的潜在意识,不时交叉地重现已逝去的岁月,从中抒发对故人、往事的无限怀念和难以排遣的惆怅。普鲁斯特的这种写作技巧,不仅对当时小说写作的传统模式是一种突破,而且对日后形形色色新小说流派的出现,也产生了深远的影响。
\par 对于这样一位伟大的作家,对于这位作家具有传世意义的这部巨著,至今竟还没有中译本,这种现象,无论从哪个角度来看,显然都是不正常的。正是出于对普鲁斯特重大文学成就的崇敬,并且为了进一步发展中法文化交流,尽快填补我国外国文学翻译出版领域中一个巨大的空白,我们决定组织翻译出版《追忆似水年华》这部巨著。
\par 外国文学研究者都知道,普鲁斯特的这部巨著,其含义之深奥,用词之奇特,往往使人难以理解,叹为观止,因此翻译难度之大可想而知。为了忠实、完美地向我国读者介绍这样重要的作品,把好译文质量关是至关重要的。为此,在选择译者的过程中,我们做了很多的努力。现在落实下来的各卷的译者,都是经过反复协商后才选定的,至于各卷的译文如何,自然有待翻译家和读者们读后评说,但我们可以欣慰地告诉读者,其中每一位译者翻译此书的态度都是十分严谨、认真的,可以说,都尽了最大的努力,对此,我们表示衷心的感谢。为了尽可能保持全书译文风格和体例的统一,在开译前,我们制定了“校译工作的几点要求”,印发了各卷的内容提要、人名地名译名表及各卷的注释;开译后又多次组织译者交流经验,相互传阅和评点部分译文。这些措施,对提高译文质量显然是有益的。
\par 关于此书的译名,我们曾组织译者专题讨论,也广泛征求过意见,基本上可归纳为两种译法:一种直译为《寻求失去的时间》;另一种意译为《追忆似水年华》。鉴于后一种译名已较多地在报刊上采用,按照“约定俗成”的原则,我们暂且采用这种译法。我们期待着广大读者的批评与指正。
\par \rightline{韩沪麟}
\par \rightline{1989年1月}


\subsubsection*{序}

\begin{center}
    安德烈·莫罗亚作\ 施康强译
\end{center}

\par 对于1900年到1950年这一历史时期而言,没有比《追忆似水年华》更值得纪念的长篇小说杰作了。这不仅仅因为普鲁斯特的作品像巴尔扎克的著作一样规模宏大。别的人写过十五部或二十部小说,有时还颇具才气,但是总不能给人以得到一种启示,读到一个总结的印象。这些作者满足于开发众所周知的“矿脉”;马塞尔·普鲁斯特却发现新的“矿藏”。《人间喜剧》把外部世界作为自己的领地;它囊括金融界、编辑部、法官、公证人、医生、商人、农民;巴尔扎克旨在描绘,他也确实描绘了整整一个社会。相反,普鲁斯特的一个独到之处是他对材料的选择并不在意。他更感兴趣的不是观察行动本身,而是某种观察任何行动的方式。从而他像同时代的几位哲学家一样,实现了一场“逆向的哥白尼式革命”。人的精神重又被安置在天地的中心;小说的目标变成描写为精神反映和歪曲的世界。
\par 用普鲁斯特书里的事件和人物来说明这位作家的特点,其荒谬程度将不亚于把雷诺阿说成是一个画过妇女、儿童、花卉的人。雷诺阿之所以成为雷诺阿,并非因为他画了这些模特儿,而是因为他把任何模特儿都摆在某种虹彩一般绚丽的光线之中。普鲁斯特本人在写到贝戈特的时候曾经指出,作品的取材与天才的形成无关。天才能使任何材料增辉生色。贝戈特成长的家庭环境从表面上看是索然寡味的,但是贝戈特却用这个素材写出一部杰作。这是因为,借助他的大脑这部小机器,他能高翥远翔,从而像飞越沙漠的飞行员隐约看到在地面上看不出来的、埋在沙子底下的城廓一样,看到事物蕴藏的秘密。因此在谈论《追忆似水年华》之前,先要说明普鲁斯特为什么比任何人更善于“飞离”这个他似乎十分眷恋的世界。
\paragraph*{一}
\par 他熟悉的天地由哪些成分组成?首先是博斯地区的一所小城——伊利耶,他童年时代每年都随家人在那里度假;是他的祖父母、父亲、母亲、兄长、叔父、舅父、婶母、姨母;是他在乡下的邻居。其次是一个巴黎社交圈子;他在孔多赛中学的同学、他父亲的朋友以及几个女人:洛尔·海曼、爱弥尔·斯特劳斯夫人、塞维尼伯爵夫人;还有阿芒·德·卡雅韦夫人、博兰古夫人、格莱福尔勃伯爵夫人的沙龙,后来又通过罗贝·孟德斯鸠的引荐,逐渐结识整个上流社会;通过他的韦尔舅舅们和他的外婆家,进入犹太人的圈子;通过卡布尔和比诺大街的网球场,与几位妙龄少女订交;至于平民百姓,他只见过几个仆人、几个开电梯的和当茶房的,服兵役时的几个伙伴和伊利耶城的几个店主;说到作家和艺术家,他只通过阿纳托尔·法朗士、雷纳尔多·阿恩、马德莱纳·勒梅尔和埃勒,对他们的生活略有所知。总之他的见闻所及仅系法国社会一个很薄的剖面。不过这又有什么关系呢?普鲁斯特将不是从广度,而是从深度上开掘他的“矿脉”。
\par 好几项特征注定他日后要从事写作。他的气质是神经质的,敏感到病态的程度。他有一个令人钦佩的母亲,对他无比宠爱,因此他遇到最细微的不和谐也如同受到伤害,最淡薄的敌意或者最不经意的可笑行径都会在他心头留下痛苦的记录。换了一个躯壳较厚的人,有些场景不会产生持久的印象,他碰上却会终生难忘,在他的思想里像地狱里受尽煎熬而找不到出路的灵魂一般骚动。(例如:某天晚上她母亲拒绝在他入睡前吻他,过后禁不住他的恳求又让步了。后来,为寻找意中人他曾深夜在巴黎街头奔走。还有他在社交场合遭受的一些屈辱,先是在《让·桑德伊》,后来在《追忆似水年华》里都有痕迹可寻。)“作家受到命运不公正的待遇之后,总要尽力寻求补偿。”我们这位作家尤其迫切地需要补偿、解释和安慰。
\par 由于他患有慢性哮喘,虽说不是废人,却年纪轻轻就成为病人,每年有一定时间必须闭门谢客。这种隐居有助于把生活转化为艺术。“唯一真实的乐园是人们失去的乐园。”普鲁斯特以一千种方式重复这一想法。“幸福的岁月是失去的岁月,人们期待着痛苦以便工作。”他被逐出童年时代的伊甸园,失去了幸福,于是就企图重新创造幸福。
\par 他的精神患病甚于肉体。早在少年时代,他已发现唯一吸引自己的爱情在人们眼里是反常的。他不比纪德\footnote{安德烈·纪德(1869—1951),著名小说家,在法国文学史上占有重要地位。},敢向家里人挑战。“家庭啊,我恨你们”\footnote{见纪德的《地粮》。}这类表白完全违背他的本性。我们可以想象他怎样在内心经历长时间的、痛苦的斗争,终归战败;他怎样努力克制自己的欲望;怎样旧病重犯,最终确信自己无可救药。如果把普鲁斯特看做不道德的人,那就大错特错了。他诚然背离道德规范,但是他因此而痛苦。出于这层原因,他也有忏悔和分析自己的需要,而这有利于写小说。
\par 最后,这个怀有如此强烈的写作冲动的年轻人,正好具备从事写作的条件。他不仅秉有神经质人敏锐的悟性,从而获得宝贵的材料,而且掌握渊博的知识,从而知道怎样利用这些材料。他母亲嗜爱法国和英国的古典大作家,让他也寝馈其中。我们时代很少有人比他更熟悉圣西门、塞维尼夫人、圣勃夫、福楼拜、波特莱尔;他的拟作证明他与这些作家灵犀相通。他研究过他们的思想方式、创作手法和风格。他若不是我们时代最伟大的小说家,本可以当最伟大的批评家。对英国作家的了解使他有可能进行知识上的杂交,这对强化一个人的思想如同生理上的杂交能增强一个种族的体质一样有效。他曾指出自己从托马斯·哈代、乔治·艾略特、狄更斯,尤其从拉斯金得到一些教益。我们时代没有任何作家比他更有学问,更加懂行。
\par 然而事情的奇妙正在于,他具备如此出色的条件本可以当一个威严的、多少有点学究气的传统作家,但他偏偏拒绝走这条现成的路子。在这里,他那位趣味高雅的母亲给他的教诲又起作用了。“对于应该怎样烹调某些菜肴、演奏贝多芬的奏鸣曲和殷勤待客,她自信能掌握最合适的分寸……况且对这三件事情来说,最合适的分寸几乎是相同的:手法简洁、朴实无华、饶有韵致。”普鲁斯特对于风格的看法并无二致。作为技巧出众的演奏家,他有时禁不住拖长一段曲子(电话接线小姐——山楂树——盖尔芒特王妃的浴缸)。最优秀的普鲁斯特,本色的普鲁斯特,却在风格上刻意求工的同时不失自然。没有人比他更精确地记录下口语的音乐性和每个阶层的人特有的语调。
\par 他有那么多的东西要表达,不说出来简直会憋死。他长期寻找一个题材以便表达所有这一切,却一直没有找到。童年时代,他在维福纳河两岸漫步,曾经隐约感到在一幢房子的屋瓦底下或者一棵长条拂地的柳树下面隐藏着某些真相,有待于他去揭穿;二十五岁或三十岁时,他反复搜索记忆的宝库,还是没有找到他需要的东西。1896年,他发表一部短篇小说和诗歌合集《欢乐和时日》。这本书染上世纪末的颓风,使人想起《白色杂志》、让·德·蒂南和奥斯卡·王尔德。没有一个读者猜到作者有一天将成为我们最伟大的文学革新家。然后,从1899年到1904年,他悄悄地写满许多练习本:那是一部自传性长篇小说《让·桑德伊》。一气呵成以后,作者从未修改。
\par 他没有发表这部作品,甚至想毁掉它:作品有许多页已被撕掉。今天我们在这部作品里发现了《追忆似水年华》中大部分为我们喜爱的优点。若干使普鲁斯特魂牵梦萦的场面,日后将以完善的形式记录下来,在这里已经初露端倪。透澈的分析、诗意的描写、对滑稽可笑言行地道的狄更斯式的描绘:这一切都非高手莫属。然而他当初不发表这部草稿是对的。他若那样做了,后来就不会以无比高超的技巧重写同一个题材。他写这部草稿的时候,他的双亲犹在,而且还可能是他最初的读者,所以他不能在作品里坦率处理他认为是最主要的东西。对于我们这些普鲁斯特迷来说,《让·桑德伊》是一部引人入胜的书,但是书中的人物和事件与原型相比变化不大,还不足以成为完美的艺术品。
\par 《让·桑德伊》里的观察者已是一位大师。不过普鲁斯特不满足于观察。他认为美犹如童话里的公主,被某个可怕的魔法师关在一座城堡的塔楼里。为了搭救这位公主,我们打破一千扇门还是徒劳,而大部分人忙于享受生的乐趣,不久就放弃寻找。但是像普鲁斯特这样的人宁可放弃其他一切,也要找到被囚禁的公主。总有一天,他受到启示,福至心灵,确信自己已有把握。他将得到秘密的、令人目眩的报偿。他说:“人们敲遍所有的门,一无所获。唯一那扇通向目标的门,人们找了一百年也没有找到,却在不经意中碰上了,于是它就自动开启……”
\paragraph*{二}
\par 这扇“唯一的”门通向什么呢?当它突然自动开启时,他隐约看到的那部“与《一千零一夜》和圣西门的《回忆录》篇幅相等”的作品究竟是什么样子呢?他有什么重要的话要说,不惜为之牺牲其他一切呢?普鲁斯特浩瀚的交响乐里将出现什么主题呢?
\par 第一主题,是时间。他的书以这个主题开端、告终。“假如假以天年,允许我完成自己的作品,我必定给它打上时间的印记:时间这个概念今天以不可抗拒的力量强迫我接受它。我要在作品里描写人们在时间中占有的地位比他们在空间中占有的微不足道的位置重要得多,即便这样做会使他们显得类似怪物……”我们周围的一切都处于永恒的流逝、销蚀过程之中,普鲁斯特无日不为这个想法困扰。“就像空间有几何学一样,时间有心理学。”人类毕生都在与时间抗争。他们本想执著地眷恋一个爱人、一位友人、某些信念;遗忘从冥冥之中慢慢升起,淹没他们最美丽、最宝贵的记忆。
\par 古典哲学假定“有一种不变的信仰犹如精神的雕像形成我们的人格”,这座雕像在外部世界的冲击下坚定不动如磐石。但是普鲁斯特知道自我在时间的流程中逐渐解体。为期不远,总有一天那个原来爱过、痛苦过、参与过一场革命的人什么也不会留下。我们将在小说里看到斯万、奥黛特、希尔贝特、布洛克、拉谢尔、圣卢怎样逐一在感情和年龄的聚光灯下通过,呈现不同的颜色,就像舞女的白色衣裙在灯光下依次变成黄色、绿色或蓝色一样。沉溺在爱河中的自我不能想象,几年以后,同一个自我一旦从爱情中解脱出来,又会是什么样子。而且可叹的是“房屋、街衢、道路和岁月一样转瞬即逝”。我们徒然回到我们曾经喜爱的地方;我们决不可能重睹它们,因为它们不是位于空间中,而是处在时间里,因为重游旧地的人不再是那个曾以自己的热情装点那个地方的儿童或少年。
\par 然而我们的历任自我并不完全消失,因为它们能在我们的睡梦中,甚而在清醒状态下重现。普鲁斯特在他的交响乐的第一乐章即陈述睡醒的主题,这并非事出偶然,而是有意为之。每天早晨,在片刻迷糊之后,我们重新拥有我们自身;这说明我们从未完全失去它。马塞尔在他生命的最后几年能在自己身上某处听到“小铃铛清脆的铁质铃声不时响起、无休无止、吵吵嚷嚷”,在他童年时代每次铃响总是宣告斯万来访。那必定是这个铃铛从未停止在他身上丁冬作响。因此时间看起来好像完全消逝,其实不然,它正与我们自身融为一体。由此产生了作为普鲁斯特作品的根源的想法,即追寻似乎已经失去,其实仍在那里,随时准备再生的时间。
\par 这个追寻只能在人们视为“真实”的那个世界里进行。其实这个世界是不真实的,至少是不可认识的,因为我们看到的世界永远受到我们自身的情欲的歪曲。世界不是一个,而是成千上万;“每天清晨有多少双眼睛睁开,有多少人的意识苏醒过来”,便有多少个世界。因此,要紧的不是生活在这些幻觉之中并且为这些幻觉而生活,而是在我们的记忆中寻找失去的乐园,那唯一真实的乐园。“过去”便是我们每个人身上都存在的某种永恒的东西。我们在生命中某些有利时刻重新把握“过去”,便会“油然感到自己本是绝对存在的”。所以,除了第一个主题:摧毁一切的时间而外,另有与之呼应的补充主题:起保存作用的回忆。不过我们这里指的不是随便哪一种回忆;普鲁斯特的主要贡献在于他教给人们某种回忆过去的方式。
\par 难道有好几种回忆过去的方式吗?至少有两种。人可以试图借助智力,通过推理、文件和佐证去重建过去。这一自主的回忆决不可能使我们感到过去突然在现在之中显露,而正是这种突然显露才使我们意识到自我的长存。必须发动不由自主的回忆,才能找回失去的时间。那么不由自主的回忆怎样发动呢?得通过当前的一种感觉与一项记忆之间的偶合。我们的过去继续存活在滋味、气息之中。普鲁斯特写道:“不要忘记,我生命中有个反复出现的动机……比对阿尔贝蒂娜的恋情还要重要的动机,即重温旧事,这也是献身艺术者的上好材料……一杯茶、散步场上的树木、钟楼等等。”玛德莱娜点心便是出色的例子。
\par 叙述者一旦辨认出这种形似海贝的蛋糕的味道,整个贡布雷便带着当年他曾在那里感受的全部情绪,从一杯椴花茶中浮现出来;亲身的经历使这座小城在他眼里倍觉动人。当前的感觉与重新涌现的记忆组成一对。这个组合与时间的关系,犹如立体镜与空间的关系。它使人们产生时间也有立体感的错觉。在这一瞬间,时间被找回来了,同时它也被战胜了,因为属于过去的整整一块时间已变成属于现在的了。因此艺术家在这种时刻感到自己征服了永恒。任何东西只有在其永恒面貌,即艺术面貌下才能被真正领略、保存:这就是《追忆似水年华》的根本、深刻和创新的主题所在。别的作家(夏多布里盎、钱拉·德·奈伐尔)曾经窥见这个主题,但是他们没有在自己的直觉的指引下走到底,没有敞开通向神奇境界的大门。唯有普鲁斯特发现,在第一个回忆的诱发下,人们以为已经永远遗忘的世界好像附丽在这个最初的回忆上面,会从一杯茶中整个涌现出来。
\par 概括说,他的小说是一个聪明绝顶、敏感到痛苦地步的人的经历。这个人从小就出发寻找绝对的幸福,他在家庭里、爱情中、世界上都没有找到绝对幸福,最后像宗教神秘主义者一样到时间之外去寻找一种绝对存在。他在艺术中找到这个绝对物,因此小说与小说家的生平融为一体,而小说结尾时说叙述者找回了失去的时间,可以开始写他的书了。就这样,这部书像一条长蛇首尾相衔,绕成一个巨大的圆圈。
\paragraph*{三}
\par 不由自主的回忆以其魔法唤醒过去之后,叙述者看到什么东西呢?居中一座乡下房子,是他们外祖母、他的父母、他的姑姑莱奥妮(与亲朋相处时富有喜剧性的人物)、女仆弗朗索瓦丝(妙不可言的肖像)以及几名配角。挨着贡布雷的住所涌现一所外省花园,夏天晚上一位邻居,斯万先生,没有斯万太太陪同,常来看望叙述者的父母。贡布雷周围伸展着一片既熟悉又神秘的地带。对于童年时代的叙述者来说,这片地带分成两“边”:斯万那一边,即斯万家的产业当松维尔,和盖尔芒特那一边,即盖尔芒特家的城堡所在地。盖尔芒特家系出贵族名门,马塞尔有时瞥见他们望完弥撒后步出教堂,视他们为高不可攀的天人;人家告诉他这一家人是热纳维也夫·德·布拉邦特\footnote{中世纪传说,热纳维也夫是布拉邦特公爵的女儿,齐格菲伯爵的妻子。伯爵出征,但不知妻子已怀孕。总管戈洛引诱热纳维也夫未成,遂诬告她与人私通。伯爵下令将她处死,仆人们没有执行命令,放她一条生路。后来夫妻相见,真相大白。}的后裔,他们过着神仙般的日子。就这样,生命以名字阶段开始。盖尔芒特家、斯万夫人、她的女儿希尔贝特·斯万:叙述者对所有这些人所知甚微,对于他来说他们只是些名字。
\par 一个接着一个,这些名字将变成有血有肉的人。后来叙述者介入盖尔芒特家的生活圈子,这家人对他仍有吸引力,但是不复有英雄的威望。盖尔芒特公爵夫人酷似教堂里彩画玻璃上的女圣徒,后来成为马塞尔的朋友。马塞尔将发现,她虽然才思敏捷,但是思想浮浅,还有自私、冷酷的一面。盖尔芒特家别的成员,夏吕斯男爵和迷人的罗贝·德·圣卢,原先处于半明半暗的光线下得到美化,后来将依次在前台的强光灯下暴露原形。叙述者逐渐发现,这些人物曾如幻灯映出一般,组成了一个神奇世界,这些男人和女人的名字底下隐藏着时而残酷、时而平庸的现实。小说的材料不在现实世界之内,而是在现实世界和想象世界的差距之中。
\par 在爱情领域,也有一个词语阶段。在这个阶段,人惑于古典或浪漫作品中对这一感情的描绘,追求不可能实现的心心相通。但是“爱情本身与我们对爱情的看法之间的差别判若天壤”。普鲁斯特试图以比传统小说家更多的真实性去描绘相遇相悦、离怀别苦,以及最终的冷淡。夏娃本是从亚当体内抽出来的:这个象征十分正确。我们入睡后一条腿的位置没有放对,便有心爱的女人翩然入梦。我们在邂逅相逢时用我们自身的想象做材料塑造的那个恋人,与日后作为我们的终生伴侣的那个真实的人毫无关系。斯万娶了从他梦中走出来的奥黛特为妻,结果面对的奥黛特却是一个他不爱的人,“与他根本合不来”。叙述者马塞尔起先认为阿尔贝蒂娜俗不可耐,其貌不扬,但是因为她“不可捉摸”,周身笼罩着神秘的光晕,便对她产生依恋之情,最终爱上她了。
\par 爱情的对象被占有之后,只要怀疑依然存在,爱情可以保持不衰。我们发现自己曾经如此重视的东西原来纯属虚妄之后,如果嫉妒占据了我们心灵的荒漠,这一发现还不足以使我们痊愈。幸亏“回忆有时混乱,接着感情出现间歇”。最后,经过长期睽别,遗忘来临,驱除了爱情的种种幻觉。至于在《索多姆和戈莫尔》中致力描写的变态爱情,它与正常的爱情遵循同一条变化曲线。爱情的实际对象是马车夫,缝制背心的裁缝,还是妓女或公爵夫人,这都无关紧要,因为按照普鲁斯特的说法,爱情的本质在于爱的对象本非实物,它仅存在于情人的想象之中。
\par 同样地,马塞尔童年时代的两条“边”:斯万那边和盖尔芒特那边,对于他曾是陌生、迷人、秘密的世界,后来他得以实地勘察这两个世界时,却在其中找不到任何东西能引起他强烈、持久的兴趣。追逐时尚与追逐爱情一样令人失望。斯万渴望加入维尔迪兰的小圈子,马塞尔则想厕身盖尔芒特家的沙龙。一旦他们如愿以偿,认识并征服了小圈子和沙龙,两者便一钱不值了。唯一有吸引力的世界是我们尚未进入的世界。一切都比儿童的眼睛看到的要简单、平淡。从贡布雷看出去,两条“边”之间好像隔着一道鸿沟。不料它们竟在作品的顶上组成巨大的圆拱,最终汇合在一起:斯万的女儿希尔贝特嫁给盖尔芒特家的圣卢。两条边的对立原来也是假的。现实在显露真相的同时烟消云散。
\par 我是故意用圆拱这个词的。普鲁斯特的作品刚发表的时候,批评家们未能立即理解它的结构,不知道它在结构上与大教堂一样简单、稳重。作者自己是意识到这一点的:“当你对我谈到大教堂的时候,你的妙语不由得使我大为感动。你直觉到我从未跟人说过的第一次形诸笔墨的事情:我曾经想过为我的书的每一部分别选用如下标题:大门、后殿彩画玻璃窗,等等。我将为你证明,这些作品唯一的优点在于它们全体,包括每个细微的组成部分都十分结实,而批评家们偏偏责备我缺乏总体构思。我若采用类似的标题,便能事先回答这种愚蠢的批评……”
\par 确实如此,在完工的作品里有那么多精心安排的对称结构,那么多的细部在两翼相互呼应,那么多的石块在开工伊始就砌置整齐,准备承担日后的尖拱,以致读者不能不佩服普鲁斯特把这座巨大的建筑当做一个整体来设计的杰出才智。就像序曲部分草草奏出的主题后来越演越宏伟,最终将以勇猛的小号声压倒陪衬音响一样,某一《在斯万家那边》仅仅露了脸的人物将变成书中的主角之一。(例如:在外叔祖父家里见过一面的那位穿一身粉红的夫人,后来变成奥黛特·德·克雷西,又变成斯万夫人,最后成为福什维尔夫人;画家比施原是维尔迪兰的“小核心”的成员,后来成为伟大的埃尔斯蒂尔;在妓院里与叙述者春风一度的那个女子,日后重逢时改名拉谢尔,已是圣卢钟爱的情妇。)
\par 就像一个巨大的桥拱跨越岁月,最终把斯万那一边和盖尔芒特那一边联接起来一样,翻过几千页书以后,将有别的感受—回忆组合与马德莱娜小蛋糕的主题相呼应(叙述者在到威尼斯的旅途上见到的大小不等的铺路石块;他在盖尔芒特王妃的图书馆里见到上了浆、烫得挺括的毛巾时,巴尔贝克海滨顿时在他眼前重现)。整个建筑的拱顶石无疑是罗贝和希尔贝特的女儿圣卢小姐。这只是一件小石雕,从底下仰望勉强可见,但是在这件石雕上“无形无色、不可捕捉”的时间确确实实凝固为物质。圆拱从而连接起来,大教堂于是竣工。到这个时候,作者作为艺术家和作为人同时得救。从那么多的相对世界里涌现出一个绝对世界了。
\par 因此普鲁斯特的小说是一种肯定,一种解脱。就像凡德伊的七重奏一样,其中两个主题——毁坏一切的时间和拯救一切的记忆对峙着:“最后,欢乐的主题取得胜利;这已不再是从空荡荡的天空背后发出的几乎带着不安的召唤;这是一种不可名状的快乐,好像来自天堂,这种快乐与奏鸣曲里的快乐差别之大,犹如贝里尼画中温和、庄重、演奏双颈诗琴的天使与米开朗琪罗笔下某一穿紫袍、吹大号角的大天使的差别。我知道我永远不会忘记快乐呈现的这个新的色彩,这个引导我们寻求一种超尘世的快乐的召唤……”
\par 克洛德·莫里亚克写过一本关于普鲁斯特的出色的小书,他在书里强调普鲁斯特独特的欢乐概念很有见地:“因为和普鲁斯特在一起,我们除了知道感情有间歇,更知道幸福也是时而袭来,时而消失的。这一阵阵欢乐的清风来自什么地方呢?”来自艺术。大艺术家“为我们掀开丑恶与无聊的帷幕的一角,我们由于隔着这道帷幕才对世界失去好奇心”。像凡·高用一把草垫椅子,德加或马奈用一个丑女人做题材,画出杰作一样,普鲁斯特的题材可以是一个老厨娘一,股霉味一,间外省的寝室或者一丛山楂树。他对我们说:“好好看:世界的全部秘密都藏在这些简单的形式下面了。”
\paragraph*{四}
\par 人生中有些出神入化的时刻,当前偶然获得的感觉使过去重现,于是我们快乐地感到自身存在的持久性;不过一个人一生中罕遇这种时刻。那么怎样才能在每一页书上都把被囚禁的美释放出来呢?这里用得着风格:“在一项描写中,人们可以无穷尽地罗列位于被描写地点的各种物体;但是真相仅在作家择定两件不同的物体、指出它们的相互关系的那个瞬间开始披露。艺术世界中这一相互关系类似科学世界中唯一的因果关系。作家还需要用美丽的风格形式的圆环把这两件物体关闭在内,他甚至围住了生命,当他举出两种感觉的共同特点,用一项隐喻把两件物体结合起来,从而显示它们的本质,使它们摆脱时间的影响,并用词的组合形式的不可描述的锁链把它们拴在一起……”
\par 通过揭示某一陌生事物或某一难以描写的感情与一些熟悉事物的相似之处,隐喻可以帮助作者和读者想象这一陌生事物或这一感情。当然普鲁斯特不是第一个使用形象的作家。对于原始人,形象也是一种自然的表达手段。但是普鲁斯特比同时代任何作家更加理解形象的“至上”重要性;他知道形象怎样借助类比使读者窥见某一法则的雏形,从而得到一种强烈的智力快感;他也知道怎样使形象常葆新鲜。
\par 既然比喻的目的是用熟知的事物解释未知的事物,那么比喻的第二项,即那个好像是透明的、可以透过现实被看到的东西就与我们熟悉的感觉之间有了联系。荷马有理由吟唱:“勇猛如怒狮……”因为他的听众曾经与狮子搏斗过。普鲁斯特指出现代的隐喻应该在事物后面唤起味觉、嗅觉、触觉这一类永远真实的基本感觉,或者展示作为任何艺术的首要成分的动植物形象(夏吕斯变成大黄蜂,絮比安化成兰花,盖尔芒特家的人变作禽鸟)。最后,它也可以从当代各学科中借用现实生活的形象。所以在普鲁斯特的文章里不时出现科学、心理学、政治学的形象。
\par 我们任意打开几页书,便能采撷到一束新鲜的形象花束,如叙述者的母亲对弗朗索瓦丝说:“诺布瓦先生把她说成是‘第一流的统帅’,就像是国防部长在阅兵式结束后向将军转达一位路过的外国君主的祝贺……”马塞尔这个时候正爱上希尔贝特·斯万,他把与斯万家有关的一切都视做神圣;当他听父亲说到斯万家住的套房普普通通时,一种亵渎之感使他全身血液沸腾:“我本能地感到我的精神应该向斯万家的威望,以及我自己的幸福奉献必要的牺牲,于是不管我刚才听到什么,我内心作主,像笃信者摒弃勒南的《耶稣传》一样,永远不去想他们居住的套房平常得很,连我们也可以住进去的……”叙述者的母亲把斯万夫人为扩大她在社交界的联系而四出拜访比作一场殖民地战争:“现在特龙贝家已经就范,邻近的部落不久也要投降……”她在街上遇见斯万夫人,回家时说:“我看到斯万夫人进入战争状态;她想必准备出征马萨诸塞人、锡兰人或者特龙贝尔人,预期大获全胜……”最后一例:斯万夫人邀请一位好心肠但令人讨厌、喜欢串门的太太上门做客,因为她知道“这只活跃的‘工蜂’一旦戴上装饰羽毛的帽子,带着名片盒,能在一个下午光顾多少资产者家庭的花萼……”
\par 普鲁斯特另一个爱用的手法是借助艺术品说明实在的事物。在他生活的那个“想象博物馆”的时代,凡是有教养的人都能理解美术作品提供的参考依据。为了让读者领会奥黛特的美色,普鲁斯特提到波堤切利;为了描绘布洛克的古怪,他抬出贝里尼的《穆罕默德二世》。他把弗朗索瓦丝的谈话比做巴赫的赋格曲,把夏吕斯先生投向絮比安的眼色比做贝多芬戛然而止的乐句。大画家和大音乐家把我们领进位于词语之外的世界,没有他们我们不可能进入这个世界。普鲁斯特经由美学达到玄学。这条路选得不坏。
\par 所以隐喻在这部作品里占据的地位相当于宗教仪式里的圣器。普鲁斯特眷恋的现实都是精神性的,但是因为人既是灵魂,又是肉体,他需要物质性的象征帮助他在自身和不能表达的东西之间建立联系。普鲁斯特最先懂得,任何有用的思想的根子都在日常生活里,而隐喻的作用在于强迫精神与它的大地母亲重新接触,从而把属于精神的力量归还给它。雨果出于本能也懂得这个道理,但是普鲁斯特通过智力和使用方法达到同一个目的了。
\paragraph*{五}
\par 阿兰曾经指出,小说在本质上应是从诗到散文,从表象到一种实用的、仿佛是手工产品的现实的过渡。普鲁斯特是纯粹的小说家。没有人比他更善于帮助我们在自己身上把握生命从童年到壮年,然后到老年的过程。所以他的书一旦问世,便成为人类的圣经之一。他简单的、个别的和地区性的叙述引起全世界的热情,这既是人间最美的事情,也是最公平的现象。就像伟大的哲学家用一个思想概括全部思想一样,伟大的小说家通过一个人的一生和一些最普通的事物,使所有人的一生涌现在他笔下。
\\ 
\par \textbf{普鲁斯特年谱}
\begin{center}
    \par 徐继曾\ 编译
\end{center}
\par 1871年7月10日 马塞尔·普鲁斯特生于巴黎位于布洛尼林园与塞纳河之间的奥德伊市拉封丹街96号其外叔祖父路易·韦伊家。马塞尔为其父母的长子。其父阿德里安·普鲁斯特通过学衔考试,任医学院教授,其母让娜·韦伊,较教授年轻十五岁。马塞尔的父母住在巴黎罗瓦街8号。
\par 1873年5月24日 马塞尔的弟弟罗贝·普鲁斯特出生。
\par 8月1日 普鲁斯特教授一家自罗瓦街迁至马尔泽尔布路9号。
\par 自1878年起马塞尔每年随其父母前往厄尔卢瓦尔省他父亲的出生地伊利耶度复活节假。他们住在教授的姐姐儒勒·阿米纳夫人家。伊利耶为普鲁斯特作品中贡布雷的原型,自1971年起改名为伊利耶贡布雷。约1881年马塞尔首次患哮喘。
\par 1882年10月2日马塞尔入丰塔纳中学五年级(中学最低年级),四个月后,该校恢复孔多塞中学名称。由于健康关系,马塞尔缺课颇多。
\par 约1887年马塞尔在香榭丽舍大街与政治家、后于1895年任第三共和国总统的费利克斯·富尔及玛丽·贝纳达基两夫妇的女儿们相识。
\par 1887—1888年在修辞班受业于马克西姆·戈谢。按当时法国中学生在读完五四三二一、、、、年级后,按文理科分班,文科再读修辞班一年,哲学班一年。
\par 1888—1889年在哲学班受业于阿尔封期·达尔吕。得“法文作文”(哲学论述)比赛第一名。
\par 1889年6月通过中学毕业会考,获文学业士学位。马塞尔在孔多塞中学与雅克·比才、日后成为剧作家并当选法兰西学院院士的罗贝·德·弗莱、后来成为史学家的达尼埃尔·阿莱维结识,并为校内刊物(《绿色评论》、《丁香评论》)撰稿。他开始出入于马德莱娜·勒梅尔、阿芒·德·加亚维夫人、斯特劳斯夫人的沙龙。加亚维夫人将他介绍给大作家阿纳托尔·法朗士;斯特劳斯夫人娘家姓阿莱维,为著名作曲家乔治·比才的遗孀,马塞尔在她家中结识的花花公子夏尔·阿斯,后来成为其作品中夏尔·斯万的原型。1889年11月15日普鲁斯特自愿在奥尔良步兵第76团入伍,与罗贝·德·利里结识。
\par 1890年11月15日作为二等兵退伍。在法学院及政治科学自由学院注册入学。
\par 1891年9月在芒什省冈市附近的卡堡度假。此处有海滨浴场,即普鲁斯特作品中的巴尔贝克。
\par 1892年3月《宴会》杂志创刊,普鲁斯特为之撰稿。该刊于1893年3月停刊。
\par 1893年为《白色评论》撰稿。开始与诗人、艺术评论家、审美家、花花公子罗贝·德·蒙代斯吉乌交往。
\par 1894年准备文学士学位考试。在卡尔瓦多斯省特鲁维尔度暑假。
\par 1895年3月取得文学士学位。
\par 6月经考试被马扎然图书馆录用为馆员。
\par 7月暂调国民教育部。12月获准长假,普鲁斯特从此不再担任公务员。
\par 9月与其友作曲家雷纳尔多·阿恩同游布列塔尼。自1895年9月至1900年初普鲁斯特撰写其第一部长篇小说,终未完成,直至1952年始以《让·桑德伊》之名发表。
\par 1896年6月12日普鲁斯特的第一部作品《欢乐与时日》在加尔曼雷维出版社出版,由法朗士作序,马德莱娜·勒梅尔作水彩插图,雷纳尔多·阿恩作与音乐有关的评注。这部作品的许多片段在此之前已在《白色杂志》、《每周评论》及《高卢人报》上发表。
\par 1894年2月与让·洛兰决斗。
\par 1898年在德雷福斯案件中,普鲁斯特力主重审。
\par 1900年1月20日英国艺术评论家兼社会学家约翰·拉斯金逝世。普鲁斯特在《艺术与珍品专栏》(1月27日)中撰文悼念。不久在《费加罗报》发表题为《拉斯金在法国的巡礼》的文章,4月又在《法兰西信使》上发表论文《拉斯金在亚眠圣母院》(该论文后来又重刊于拉斯金所著《亚眠的圣经》的法译本序中)。普鲁斯特在其母及雷纳尔多的英籍表姐玛丽·诺林格的帮助下从事拉斯金作品的法译工作。
\par 5月与母同游意大利。在威尼斯与玛丽·诺林格相逢。
\par 10月普鲁斯特全家迁居古塞尔街45号。
\par 1903年11月26日父亲去世。
\par 1904年在《法兰西信使》中刊载拉斯金所著《亚眠的圣经》的法译本。
\par 1905年9月26日母亲去世。
\par 12月普鲁斯特神经深受刺激,不得不在塞纳河上的布洛尼住院六周。
\par 1906年在凡尔赛小住一段时间后,普鲁斯特迁居奥斯曼路102号。失眠日益严重,为隔绝一切噪音,普鲁斯特于1910年请人将他卧室的墙壁全部加上软木贴面。
\par 在《法兰西信使》中发表拉斯金另一部著作《芝麻与百合》的法译本,并冠以1905年6月15日已在《拉丁文艺复兴》杂志上发表过的一篇长序;此序日后稍加修改,在《什锦与杂记》中以《读书日》之名重新发表。
\par 1907年在卡堡度暑假;普鲁斯特以后每年都来此间,直至1914年。同年,乘汽车游览,由阿戈斯蒂耐里为其开车,参观了诺曼第诸教堂。
\par 1908—1909年在《费加罗报》发表一系列杂文,其题材为当时被揭露的冒险家勒穆瓦纳的种种骗局。
\par 1909年6月普鲁斯特草拟论文一篇,反对圣勃夫所用批评方法。他久已有意通过这一途径来阐述他个人的美学原则。这篇论文终未完成,因为他多年间念念不忘重操小说旧业,他写的那篇《让·桑德伊》不过是这部巨著的一个梗概。
\par 1912年阿戈斯蒂耐里当上他的秘书。
\par 1913年写毕《追忆似水年华》中的三部,即《在斯万家那边》、《盖尔芒特家那边》、《重现的时光》,但无出版商愿意接受。贝尔纳·格拉塞后来同意出版,但应由作者出资;且不顾普鲁斯特的愿望,仅同意先出第一部,《盖尔芒特家那边》须在1914年,《重现的时光》则须在1915年始能问世。
\par 11月8日《在斯万家那边》出版。
\par 1914年5月30日阿戈斯蒂耐里在此之前已与普鲁斯特分手,学习驾驶飞机,是日驾一架单翼机在海滨阿尔卑斯省的昂蒂布海岸上空遇难身亡。
\par 6月1日《新法兰西评论》发表《追忆似水年华》第二卷摘录,该卷即将在贝尔纳·格拉塞出版社出版。这些摘录属于《在少女们身旁》。
\par 7月1日《新法兰西评论》再次发表《追忆似水年华》的摘录,系《盖尔芒特家那边》的第一卷中的梗概。
\par 8月贝尔纳·格拉塞应征入伍,《追忆似水年华》的出版工作中断。自1915年起,普鲁斯特改写小说的第二及第三部分,作了大量增补。1916年与格拉塞断绝交往,自此其作品即由新法兰西评论社出版。
\par 1918年11月30日《在少女们身旁》在新法兰西评论社印毕。
\par 1919年3月28日《什锦与杂记》在新法兰西评论社印毕。
\par 6月因原住所由银行收买,被迫迁出奥斯曼路,在洛朗毕夏街8号甲女演员莱雅纳拥有的一所房子中觅得一暂时栖息之所。
\par 10月迁入阿姆兰街44号,在此直住至逝世。
\par 12月10日《在少女们的身旁》以6∶4票通过获龚古尔奖。罗朗·多热莱斯的《木十字架》落选。阿尔封斯·都德之子、新闻记者与作家莱翁·都德在票选中起了重大作用。
\par 1920年8月7日《盖尔芒特家那边》的第一卷在新法兰西评论社印毕。
\par 11月在《巴黎评论》上发表《致友人(论风格)》。这是普鲁斯特为保尔·莫朗的中篇小说《细弱的储备》所作的序。
\par 1921年1月在《新法兰西评论》中发表《谈福楼拜的风格》。4月30日《盖尔芒特家那边》第二卷及《索多姆和戈摩尔》的第一卷在新法兰西评论社印毕。
\par 5月在网球场博物馆参观荷兰画展时,普鲁斯特突感不适。
\par 6月在《新法兰西评论》上发表《谈波特莱尔》一文。1922年4月3日《索多姆和戈摩尔》的第二卷在新法兰西评论社印毕。
\par 11月18日马塞尔·普鲁斯特与世长辞。
\par 1923年《女囚》在新法兰西评论社出版。
\par 1925年《女逃亡者》以《阿尔贝蒂娜不知去向》为名在新法兰西评论社出版。
\par 1927年《重现的时光》在新法兰西评论社出版。
\par 自1950年起《马塞尔·普鲁斯特与贡布雷之友协会通讯》出版。
\par 1952年《让·桑德伊》在新法兰西评论社出版。
\par 1954年《驳圣勃夫》,附《新杂记》,由新法兰西评论社出版。《追忆似水年华》评注本三卷,由伽里玛出版社在《七星丛书》中出版。
\par 1970年普鲁斯特《通信集》注释本第一卷在普隆出版社出版,由菲力普·戈尔勃评介。
\par 1971年普鲁斯特其他作品的评注本在《七星丛书》中出版,其中包括《让·桑德伊》一卷(附《欢乐与时日》)、《驳圣勃夫》一卷(附《什锦与杂记》及《随笔和文章》)。
\par 赠迦斯东·卡尔梅特先生:
\par 谨致深深的、衷心的感激。
\par \rightline{马塞尔·普鲁斯特}


\subsubsection*{第一部\ 贡布雷}


\paragraph*{一}
\par 在很长一段时期里,我都是早早就躺下了。有时候,蜡烛才灭,我的眼皮随即合上,都来不及咕哝一句:“我要睡着了。”半小时之后,我才想到应该睡觉;这一想,我反倒清醒过来。我打算把自以为还捏在手里的书放好,吹灭灯火。睡着的那会儿,我一直在思考刚才读的那本书,只是思路有点特别;我总觉得书里说的事,什么教堂呀,四重奏呀,弗朗索瓦一世和查理五世争强斗胜呀,全都同我直接有关。这种念头直到我醒来之后还延续了好几秒钟;它倒与我的理性不很相悖,只是像眼罩似的蒙住我的眼睛,使我一时觉察不到烛火早已熄灭。后来,它开始变得令人费解,好像是上一辈子的思想,经过还魂转世来到我的面前,于是书里的内容同我脱节,愿不愿意再挂上钩,全凭我自己决定;这一来,我的视力得到恢复,我惊讶地发现周围原来漆黑一片,这黑暗固然使我的眼睛十分受用,但也许更使我的心情感到亲切而安详;它简直像是没有来由、莫名其妙的东西,名副其实地让人摸不到头脑。我不知道那时几点钟了;我听到火车鸣笛的声音,忽远忽近,就像林中鸟儿的啭鸣,标明距离的远近。汽笛声中,我仿佛看到一片空旷的田野,匆匆的旅人赶往附近的车站;他走过的小路将在他的心头留下难以磨灭的回忆,因为陌生的环境,不寻常的行止,不久前的交谈,以及在这静谧之夜仍萦绕在他耳畔的异乡灯下的话别,还有回家后即将享受到的温暖,这一切使他心绪激荡。
\par 我情意绵绵地把腮帮贴在枕头的鼓溜溜的面颊上,它像我们童年的脸庞,那么饱满、娇嫩、清新。我划亮一根火柴看了看表。时近子夜。这正是病羁异乡的游子独宿在陌生的客舍,被一阵疼痛惊醒的时刻。看到门下透进一丝光芒,他感到宽慰。谢天谢地,总算天亮了!旅馆的听差就要起床了;呆一会儿,他只要拉铃,就有人会来支应。偏偏这时他还仿佛听到了脚步声,自远而近,旋而又渐渐远去。门下的那一线光亮也随之又消失。正是午夜时分。来人把煤气灯捻灭了;最后值班的听差都走了。他只得独自煎熬整整一宿,别无他法。
\par 我又睡着了,有时偶尔醒来片刻,听到木器家具的纤维格格地开裂,睁眼凝望黑暗中光影的变幻,凭着一闪而过的意识的微光,我消受着笼罩在家具、卧室乃至于一切之上的朦胧睡意,我只是这一切之中的小小的一部分,很快又重新同这一切融合在一起,同它们一样变得昏昏无觉。还有的时候,我在梦中毫不费力地又回到了我生命之初的往昔,重新体验到我幼时的恐惧,例如我最怕我的姨公拽我的鬈曲的头发。有一天,我的头发全都给剃掉了,那一天简直成了我的新纪元。可是梦里的我居然忘记了这样一件大事。直到为了躲开姨公的手,我一偏脑袋,醒了过来,才又想起这件往事。不过,为谨慎起见,我用枕头严严实实地捂住了自己的脑袋,然后才安心地返回梦乡。
\par 有几次,就像从亚当的肋叉里生出夏娃似的,有一个女人趁我熟睡之际从我摆错了位置的大腿里钻了出来。其实,她是我即将品尝到的快感的产物,但是,我偏偏想象是她给我送来了快感。我在她的怀抱中感到自己的体温,我正打算同她肌肤相亲,正巧这时我醒了。同我刚才分手的那位女子相比,普天之下无论是谁都似乎不及她更可亲,我的脸上还感到她的热吻的余温,我的身子还感到她的肢体的重量。假如有时候也确有这种情况,梦里的女子赶巧同我在生活中认识的哪位女士相貌一样,那么我必全力以赴地达到目的:非同她梦里再聚不可,就像有些人那样,走遍天下也要亲眼见见他们心目里的洞天仙府,总以为现实生活中能消受到梦境里的迷人景象。她的音容笑貌在我的记忆中逐渐淡漠;我已忘却梦中人的倩影。
\par 一个人睡着时,周围萦绕着时间的游丝,岁岁年年,日月星辰,有序地排列在他的身边。醒来时他本能地从中寻问,须臾间便能得知他在地球上占据了什么地点,醒来前流逝过多长的时间;但是时空的序列也可能发生混乱,甚至断裂,例如他失眠之后天亮前忽然睡意袭来,偏偏那时他正在看书,身体的姿势同平日的睡态大相径庭,他一抬手便能让太阳停止运行,甚至后退,那么,待他再醒时,他就会不知道什么钟点,只以为自己刚躺下不久。倘若他打瞌睡,例如饭后靠在扶手椅上打盹儿,那姿势同睡眠时的姿势相去更远,日月星辰的序列便完全乱了套,那把椅子就成了魔椅,带他在时空中飞速地遨游,待他睁开眼睛,会以为自己躺在别处,躺在他几个月前去过的地方。但是,我只要躺在自己的床上,又睡得很踏实,精神处于完全松弛的状态,我就会忘记自己身在何处,等我半夜梦回,我不仅忘记是在哪里睡着的,甚至在乍醒过来的那一瞬间,连自己是谁都弄不清了;当时只有最原始的一种存在感,可能一切生灵在冥冥中都萌动着这种感觉;我比穴居时代的人类更无牵挂。可是,随后,记忆像从天而降的救星,把我从虚空中解救出来:起先我倒还没有想起自己身在何处,只忆及我以前住过的地方,或是我可能在什么地方;如没有记忆助我一臂之力,我独自万万不能从冥冥中脱身;在一秒钟之间,我飞越过人类文明的十几个世纪,首先是煤油灯的模糊形象,然后是翻领衬衫的隐约的轮廓,它们逐渐一点一画地重新勾绘出我的五官特征。
\par 也许,我们周围事物的静止状态,是我们的信念强加给它们的,因为我们相信这些事物就是甲乙丙丁这几样东西,而不是别的玩意儿;也许,由于我们的思想面对着事物,本身静止不动,才强行把事物也看做静止不动。然而,当我醒来的时候,我的思想拼命地活动,徒劳地企图弄清楚我睡在什么地方,那时沉沉的黑暗中,岁月、地域,以及一切、一切,都会在我的周围旋转起来。我的身子麻木得无法动弹,只能根据疲劳的情状来确定四肢的位置,从而推算出墙的方位,家具的地点,进一步了解房屋的结构,说出这皮囊安息处的名称。躯壳的记忆,两肋、膝盖和肩膀的记忆,走马灯似的在我的眼前呈现出一连串我曾经居住过的房间。肉眼看不见的四壁,随着想象中不同房间的形状,在我的周围变换着位置,像漩涡一样在黑暗中转动不止。我的思想往往在时间和形式的门槛前犹豫,还没有来得及根据各种情况核实某间房的特征,我的身体却抢先回忆起每间房里的床是什么式样的,门是在哪个方向,窗户的采光情况如何,门外有没有楼道,以及我入睡时和醒来时都在想些什么。我的压麻了的半边身子,想知道自己面对什么方向,譬如说,想象自己躺在有顶的一张大床上,面向墙壁侧卧。这时我马上就会想道:“唷!我总算睡着了,尽管妈妈并没有来同我道晚安。”我是睡在已经死去多年的外祖父的乡间住宅里;我的身躯,以及我赖以侧卧的那半边身子,忠实地保存了我的思想所不应忘怀的那一段往事,并让我重又回想起那盏用链子悬在天花板下的照明灯——一盏用波希米亚出产的玻璃制成的瓮形吊灯,以及那座用西埃纳的大理石砌成的壁炉。那是在贡布雷,在我外祖父母的家里,我居住过的那个房间;离现在已经很久很久了,如今我却犹如身临其境,虽然我的睡意朦胧,不能把故物的情境想得清清楚楚;待我完全清醒之后,我能回忆得更细致些。
\par 后来,新的姿势又产生新的回忆;墙壁迅速地滑到另一边去:我睡在德·圣卢夫人家的乡间住宅里。天哪!至少十点钟了吧。他们一定都吃过晚饭了!我这个盹儿打得也太久了。每天晚上,更衣用餐前,我总要陪德·圣卢夫人外出散步,回来后先上楼打个盹儿。自从离开贡布雷,好多年过去了。住在贡布雷的日子,每当我们散步回来得比较晚,我总能在我住的那间房间的窗户玻璃上,看到落日的艳红的反照。如今在当松维尔,在德·圣卢夫人的家里,过的却是另一种生活。而且我只在晚间出去,沿着我从前在阳光下玩耍过的小路,踏着婆娑的月影散步,我感受到另一种愉快。归来时,远望我住的那个房间,只见里面灯火明亮,简直像黑夜中独有的一座灯塔。回去后我并不急于更衣用餐,而是先睡上一觉。
\par 这些旋转不已、模糊一片的回忆,向来都转瞬即逝;不知身在何处的短促的回忆,掠过种种不同的假设,而往往又分辨不清假设与假设之间的界限,正等于我们在电影镜\footnote{电影镜:美国发明家爱迪生和他的助手狄克逊于1891年发明的一种放映影片的设备,状如柜,供一人观看。}中看到一匹奔驰的马,我们无法把奔马的连续动作一个个单独分开。但是我毕竟时而看到这一间、时而又看到另一间我生平住过的房间,而且待我清醒之后,在联翩的遐想中,我终于把每一个房间全都想遍:
\par 我想起了冬天的房间。睡觉时人缩成一团,脑袋埋进由一堆毫不相干的东西编搭成的安乐窝里:枕头的一角,被窝的口子,半截披肩,一边床沿,外加一期《玫瑰花坛》杂志,统统成了建窝的材料,凭人以参照飞禽筑窝学来的技巧,把它们拼凑到一块,供人将就着栖宿进这样的窝里。遇到冰霜凛冽的大寒天气,最惬意不过的是感到与外界隔绝(等于海燕索居在得到地温保暖的深土层里)。况且那时节壁炉里整夜燃着熊熊的火,像一件热气腾腾的大衣,裹住了睡眠中的人;没有燃尽的木柴毕毕剥剥,才灭又旺,摇曳的火光忽闪忽闪地扫遍全屋,形成一个无形的暖阁,又像在房间中央挖出了一个热烘烘的窑洞;热气所到之处构成一条范围时有变动的温暖地带。从房间的旯旯旮旮,从窗户附近,换句话说,从离壁炉稍远、早已变得冷嗖嗖的地方,吹来一股股沁人心脾的凉风,调节室内的空气。
\par 我想起了夏天的房间。那时人们喜欢同凉爽的夜打成一片。半开的百叶窗上的明媚的月亮,把一道道梯架般的窈窕的投影,抛到床前。人就像曙色初开时在轻风中摇摆的山雀,几乎同睡在露天一样。
\par 有时候,我想起了那间路易十六时代风格的房间。它的格调那样明快,我甚至头一回睡在里面都没有感到不适应。细巧的柱子支撑住天花板,彼此间的距离相隔得楚楚有致,显然给床留出了地盘;有时候正相反,我想到了那间天花板又高又小的房间。它简直像是从两层楼的高处挖出来的一座金字塔,一部分墙面覆盖着坚硬的红木护墙板,我一进去就被一股从未闻到过的香根草的气味熏得昏头胀脑,而且我认定紫红色的窗帘充满敌意,大声喧哗的座钟厚颜无耻,居然不把我放在眼里。一面怪模怪样、架势不善的穿衣镜,由四角形的镜腿架着,斜置在房间的一角。那地方,据我惯常所见,应该让人感到亲切、丰硕;空洞的镜子偏偏挖走了地盘。我一连几小时竭力想把自己的思想岔开,让它伸展到高处,精确地测出房间的外形,直达倒挂漏斗状的房顶,结果我白白煎熬了好几个夜晚,只是直挺挺地躺在床上,忧心忡忡地竖起耳朵谛听周围的动静,鼻翼发僵,心头乱跳,直到习惯改变了窗帘的颜色,遏止了座钟的絮叨,教会了斜置着的那面残忍的镜子学得忠厚些。固然,香根草的气味尚未完全消散,但毕竟有所收敛,尤其要紧的是天花板的表面高度被降低了。习惯呀!你真称得上是一位改造能手,只是行动迟缓,害得我们不免要在临时的格局中让精神忍受几个星期的委屈。不管怎么说吧,总算从困境中得救了,值得额手称庆,因为倘若没有习惯助这一臂之力,单靠我们自己,恐怕是束手无策的,岂能把房子改造得可以住人?
\par 当然,我现在很清醒,刚才还又翻了一回身,信念的天使已经遏止住我周围一切的转动,让我安心地躺进被窝,安睡在自己的房内,而且使得我的柜子、书桌、壁炉、临街的窗户和两边的房门,大致不差地在黑暗中各就其位。半夜梦回,在片刻的朦胧中我虽不能说已纤毫不爽地看到了昔日住过的房间,但至少当时认为眼前所见可能就是这一间或那一间。如今我固然总算弄清我并没有处身其间,我的回忆却经受了一场震动。通常我并不急于入睡;一夜之中大部分时间我都用来追忆往昔生活,追忆我们在贡布雷的外祖父母家、在巴尔贝克、在巴黎、在东锡埃尔、在威尼斯以及在其他地方度过的岁月,追忆我所到过的地方,我所认识的人,以及我所见所闻的有关他们的一些往事。
\par 在贡布雷,每当白日已尽黄昏将临,我就愁从中来,我的卧室那时成为我百结愁肠的一个固定的痛点,虽然还不到该我上楼睡觉的钟点,离我同妈妈和外祖母分手、即使不睡也得回房去独自待着的时间还差一大截。家里的人发觉我一到晚上就愁眉苦脸,便挖空心思设法让我开心。他们居然别出心裁地给我弄来一盏幻灯,趁着我们等待开晚饭的当口,把幻灯在我的房内的吊灯上套好,这东西跟哥特时代初期的建筑师和彩画玻璃匠那样,也是用捉摸不定的色光变幻和瑰丽多彩的神奇形象来取代不透光的四壁。绘上了传奇故事的灯片,就等于一面面彩画玻璃窗,只是它们光影不定,忽隐忽现。可是我的悲愁却有增无减。因为我对房内的一切早已习惯,一旦照明发生变化,习惯也就受到破坏。过去除了睡觉使我苦不堪言之外,其他一切倒还过得去,因为我已经习惯。如今房内被照得面目全非,我一进去,就像刚下火车第一次走进山区“客栈”或者异乡旅馆的房间一样,感到忐忑不安。
\par 心怀叵测的戈洛\footnote{戈洛和热纳维耶夫是中世纪欧洲传说中的人物。戈洛是传奇英雄齐戈弗里特的宫廷总管,热纳维耶夫是齐戈弗里特的妻子。齐戈弗里特听信谣传,冤枉其妻与戈洛通奸,戈洛便乘机诱使热纳维耶夫充当他实现野心的工具。但热纳维耶夫忠于齐戈弗里特;可惜冤情大白时她因悲痛过度而死。}从覆盖着小山坡的绿荫团团的三角形的森林中,一蹦一跳地骑马走来,又朝着苦命的热纳维耶夫·德·希拉特\footnote{戈洛和热纳维耶夫是中世纪欧洲传说中的人物。戈洛是传奇英雄齐戈弗里特的宫廷总管,热纳维耶夫是齐戈弗里特的妻子。齐戈弗里特听信谣传,冤枉其妻与戈洛通奸,戈洛便乘机诱使热纳维耶夫充当他实现野心的工具。但热纳维耶夫忠于齐戈弗里特;可惜冤情大白时她因悲痛过度而死。}居住的宫堡,一蹿一跃地走去。椭圆形的灯片镶嵌在框架中,幻灯四角有细槽供灯片不时地插换。弧形的边线把灯片上的宫堡的其余部分切出画外,只留下宫堡的一角;楼前是一片荒野,热纳维耶夫站着发愣。她系着蓝色的腰带,宫堡和荒野则是黄澄澄的。我不看便知它们必定是黄颜色,因为幻灯尚未打出之前,单凭布拉邦特这一字字铿锵的大名,就已经预示了这种颜色。戈洛驻马片刻,愁眉苦脸地谛听我的姨祖母夸张其辞地大声解说。他看来都听懂了,他的举止神情完全符合姨祖母的指点:既恭顺又不失庄重。听罢,他又蹦跳着继续赶路,没有任何东西能阻挡他不慌不忙地策马前行。即使幻灯晃动,我照样能在窗帘上分辨出戈洛继续赶路的情状:在褶凸处,戈洛的坐骑鼓圆了身体;遇到褶缝,它又收紧肚子。戈洛的身体也像他的坐骑一样,具有神奇的魔力,能对付一切物质的障碍,遇到阻挡,他都能用来作为赖以附体的依凭,即使遇到门上的把手,他的那身大红袍,甚至他的那副苍白的尊容,便立刻俯就,而且堂而皇之地飘然而过;他的神情总是那么高贵,那么忧伤,但是对于这类拦腰切断的境遇,他却面无难色,临危不乱。
\par 当然,我从这些光彩熠熠的幻灯画面中,感受到迷人的魅力,它们像是从遥远的中世纪反射过来的昔日景象,让一幕幕如此古老的历史场面,在我的周围转悠着重现。但是,这种神秘、这种美,闯进了我的卧室,究竟引起我什么样的不安,我却说不清楚。我已经慢慢地把自我充实了这间卧室,以至于对房间本身早已置诸脑后,我总先想到自我,然后才会念及房间。如今习惯的麻醉作用既然停止生效,我于是动起脑筋来,开始有所感触,真要命!我的房门的把手,同天下其他房门把手不同之处,仿佛就在于它看来不需要我去转动便能自行开启,因为对我说来,把手的运行已经成为无意识的举动,它现在不是在权充戈洛的身体吗?晚饭的铃声一响,我赶紧跑进饭厅;饭厅里的大吊灯既不知有戈洛其人,也从未结识过蓝胡子\footnote{蓝胡子:民间传说中的人物。他杀死了六位妻子,第七位妻子在他尚未下手前发现了他前面六位妻子的尸体,骇极;后来幸亏她的两位兄弟及时赶到,杀死蓝胡子,救了她的性命。},它只认得我的父母和列位长辈,以及桌上的罐焖牛肉;它每天晚上大放光芒,把光芒投入我妈妈的怀抱。热纳维耶夫·德·布拉邦特的不幸遭遇,更使我感到妈妈怀抱的温暖;而戈洛造下的种种罪孽,则触动我更诚惶诚恐地检查自己的意识。
\par 用罢晚饭,唉!我得马上同妈妈分手了;她要留下陪大家聊天。遇到好天气,他们在花园里闲谈;若天公不作美,大家也只好呆在小客厅里了。我说的大家,其实不包括外祖母。她认为,“人在乡下,居然闭门不出,简直是罪过。”每逢大雨滂沱的日子,她都要同我的父亲争论,因为父亲不让我出门,偏要把我关在屋里读书。“你这种做法,”她说,“没法让他长得身体结实,精力充沛;而这小家伙尤其需要增强体力和锻炼意志。”我的父亲耸耸肩膀,聚精会神地审视晴雨表,因为他爱研究气象。而我的母亲呢,这时尽量蹑手蹑脚地少出声响,唯恐打扰了我的父亲。她温柔而恭敬地看着他,但并不盯住看,并不想看破他自鸣清高的秘密。我的外祖母却不然,无论什么天气,她都爱去室外,即使风雨大作,即使弗朗索瓦丝生怕名贵的柳条椅被淋湿,匆忙地把它们往屋里搬,外祖母也会独自在花园里,听凭风吹雨淋,而且还撩起额前凌乱的灰白头发,好让头部更加领受到风雨的保健功用。她说:“总算痛痛快快透一口气!”她还沿着花园里的小路,兴致勃勃地踩着小步,连蹦带跳地跑起来。那些小路新近由一位才来不久的园丁按照自己的设想拾掇得过分规整对称,足见他毫无自然感;我的父亲今天居然一早就请教此人,问会不会变天。外祖母的跑步动作,轻重缓急自有调节,这得看暴风雨癫狂的程度、养生学保健的威力、我所受的教育的愚昧性以及花园内对称的布局等因素在她心中所激起的各不相同的反应来决定。她倒根本不在乎身上那条紫酱色的长裙会不会溅上泥水,她从来没有这样的顾虑,结果她身上泥点的高度,总让她的贴身女仆感到绝望,不知如何才好。
\par 倘若我外祖母的这类园内跑步发生在晚饭之后,那么只有一件事能让她像飞蛾扑火一样立刻回来。小客厅里亮灯的时候,准是牌桌上已经有饮料侍候,这时姑祖母大叫一声:“巴蒂尔德!快来,别让你的丈夫喝白兰地!”在园内转圈儿跑步的外祖母就会争分夺秒地赶回来。为了故意逗她着急(外祖母把一种完全不同的精神带进了我们的家庭中来,所以大伙儿都跟她逗乐,存心作弄她),我的姑祖母还当真让我的外祖父喝了几口他不该喝的酒。可怜的外祖母走进小客厅,苦口婆心地求他放下酒杯;外祖父一赌气,索性仰脖喝了个涓滴不剩。外祖母碰了一鼻子灰,伤心地走开了,不过她脸上依然带着微笑,因为她待人向来宽厚,从不计较面子得失,这种对人对己的胸怀在她的目光中化为微笑,同我们在别人脸上见到的微笑绝然相反,它除了自我解嘲之外毫无嘲讽的意味。这一笑对我们大家来说,等于是用目光代替亲吻;她的那双眼睛,见到她所疼爱的亲人,从来都只以目光传递她怀中热切的爱怜。姑祖母狠心作弄她,她苦口婆心劝说外祖父不要贪杯,偏偏她又心肠仁慈,落得自讨没趣。这种场面我后来是习以为常了,甚至还当做笑柄,嘻嘻哈哈地、毫不犹豫地同作弄她的人沆瀣一气笑话她,还硬让自己相信这不算作弄。可是,当初我是气得要命的,恨不能去打姑祖母。然而那时我已经学得像个小大人,跟懦怯的大人一样,听到“巴蒂尔德,快来,别让你的丈夫喝白兰地”这样的叫声,我采取了我们长大成人后的惯常态度,也就是见到苦难和不平,扭过脸去以求得眼不见为净。我爬上书房隔壁紧挨着屋顶的那个小房间,躲在那里抽抽搭搭地哭起来。房间里有一股菖蒲花的香味,窗外还传来墙根下那株野生的醋栗树的芳香,有一枝开满鲜花的树梢居然伸进了半开半掩的窗户。凭窗远望,能一直望到鲁森维尔宫堡的塔楼;这间小屋原来派的用场更特殊也更平常,可是那些年里长期成为我的避难所,大概是因为它地处偏僻,我又可以把自己反锁在里面,所以一旦需要孤身独处、不容他人打扰的事要做时,我就躲到这里来,有时读书,有时胡思乱想,有时偷偷哭泣,有时自寻欢乐。唉!我当时哪里知道,我的外祖父在忌口方面往往不拘小节地出点差错,我又偏偏缺乏意志,身体娇弱,以至于一家人对于我的前途都感到渺茫,这些事儿着实让我的外祖母操了多少心。她在下午或者晚上没完没了地跑个不停,我们只见她跑来跑去,偏着脑袋仰望苍天,她那清秀的脸庞,鬓角下肤色焦黄,皱纹密布,年复一年地变得像秋后翻耕过的土地泛出紫色。她出门时,半遮的面纱挡住了她的腮帮,上面总挂着几滴由于寒风或忧思的刺激而不自觉地流下的眼泪,又渐渐让风吹干。
\par 我上楼去睡,唯一的安慰是等我上床之后妈妈会来吻我。可是她来说声晚安的时间过于短促,很快就返身走了,所以当我听到她上楼来的脚步声,当我听到她的那身挂着几条草编装饰带的蓝色细麻布的裙子窸窸窣窣走过有两道门的走廊,朝我的房间走来的时候,我只感到阵阵的痛苦。这一时刻预告着下一个时刻妈妈就会离开我,返身下楼,其结果弄得我竟然盼望我满心喜欢的那声晚安来得越晚越好,但愿妈妈即将上来而还没有上来的那段空白的时间越长越好。有几次,妈妈吻过我之后,开门要走,我居然想叫她回来,对她说:“再吻我一次吧。”可是,我知道,这样一来她马上会一脸不高兴,因为她上楼来亲我,给我平静的一吻,是对我的忧伤、我的不安所作出的让步,已经惹得我的父亲不高兴了。父亲认为这类道晚安的仪式纯属荒唐。妈妈也恨不能让我早日放弃这种需要,这种习惯。她决不会让我滋生新的毛病,也不会允许我等她走到门口之后再请她回来亲亲我,况且,只要见到她面有愠色,她在片刻前给我带来的宁静也就受到彻底破坏。她刚才像在领圣体仪式上递给我圣饼似的,把她的温馨的脸庞俯向我的床前。我的嘴唇感受到她的存在,并且吸取了安然入睡的力量。总的说来,比起客人太多,妈妈不能上来同我说声晚安的那些晚上,她能在我房内待上一会儿,哪怕时间很短,也总算不错了。所谓客人,平时只限于斯万先生。除了几位顺路来访的外地客人之外,他几乎是贡布雷屈趾舍间的唯一的客人。有时候,他以邻居的身份与我们同进晚餐(自从他同门户不相当的女子结婚之后,他很难得来了,因为我的长辈们不愿意接待他的妻子),有时候,他在晚饭之后不请自来。晚上,我们在房前那棵高大的板栗树下,围坐在铁桌的四周纳凉,忽听得花园的那一头传来声响,倒不是不打铃就进门的自家人弄响的那门铃声,丁丁当当地闹个不休,像劈头倒下的一盆雪水,弄得你晕头转向;这回我们听到的是专为来客设置的那种椭圆形的镀金的门铃声,它怯怯地丁冬两响。于是大家面面相觑:“有客人?会是谁呀?”其实大家心里明白,除了斯万先生,没有别人;我的姑祖母以身作则地大声数落开了,她力求说得自然:她教诲我们不该窃窃私语;让来人以为我们在议论他不该听到的事,是最不礼貌的行为。接着,我们看到,最爱找茬儿到花园里去走走的外祖母,已经走上前去侦察。她总乘机悄悄地把沿路的玫瑰花树的支架拔掉,让枝头的花朵显得更自然些,就像当妈妈的用手拨弄拨弄孩子的头发,把被理发师梳理得过于服帖的头发弄得蓬松自然些。
\par 我们全都屏息静气,等待外祖母回来报告侦察到的“敌情”,好似我们身陷敌众我寡的包围,一时进退不定,难下对策。接着外祖父开口说话了:“我听得出,是斯万的声音。”确实,只有他的声音最好辨认,他那张脸却难以看清;因为怕招蚊子,我们在花园纳凉时尽量少点灯。斯万长着鹰钩鼻,绿眼珠,脑门儿很高,头发黄得发红,剪成勃莱桑那样的发式\footnote{勃莱桑发式:一种把头发剪成刷子一样长短的发式,类似我国的“小平头”,因著名演员勃莱桑留这种发型而得名。}。这时,我正要不动声色地吩咐仆人拿果子露来;我的外祖母认为用果子露招待客人最相宜,因为它不显得那么特殊,才更显得得体。斯万先生虽说比我的外祖父年轻得多,却同他关系密切。我的外祖父是他的父亲的好朋友;他的父亲为人善良,就是古怪,据说,有时候一点儿小事就能使他的感情的冲动中断,思路改变。我在饭桌上每年都要听我外祖父提到好几次有关他的轶事,而且每次都一样,都是说斯万爷爷对他的妻子的死所采取的态度。他妻子病重时,他曾日夜在病榻前侍候。那时,我的外祖父已经好久没有同他见面了;听到斯万夫人的死讯他连忙赶到斯万家在贡布雷附近的庄园。为了不让他见到妻子入殓的场面,我的外祖父好不容易才把哭成泪人儿的他从灵房劝走。他们俩在阳光惨淡的花园里走了几步。斯万先生忽然拉住我的外祖父的胳膊,大声说道:“啊!老兄,这样好的天气,咱俩一块儿散步,有多好呀!你不觉得美吗?这些树,这些山楂花,还有你从来也没有对我夸过的那片池塘。你干吗愁眉苦脸?你没有感到这微风吹得人多舒服?啊!我说归说,总还是活着有意思呀,我亲爱的朋友阿梅代!”突然间,他又想起了死去的妻子。他怎么能在这种时候听任愉快的心情涌现出来?其中的原因若加以深究或许过于费事,所以他只拍拍自己的脑门儿,揉揉眼睛,擦擦夹鼻眼镜的镜片。每当遇到挠头的难题,他经常以此打发。然而,他并不能忘怀丧偶的痛苦,他在妻子死后又活了两年,他常对我的外祖父说:“也真怪,我常常想起可怜的妻子,只是不能一次想许多。”于是,“像可怜的斯万老爹那样细水长流”,成了我的外祖父爱说的一句口头禅,即使提到毫不相干的事儿,他也总把这句话挂在嘴边。我的外祖父是我心目中最公道的法官,他的判决对我来说等于量刑的准则,有些过错我本来倾向于严加谴责的,后来根据他的意见改为从宽发落。倘若外祖父不接着说,“怎么?他心眼儿好!”那我简直要把斯万爷爷看成混世魔王了。
\par 他的儿子小斯万先生一连好几年——尤其在结婚以前——常来贡布雷看望我的姑祖母和外祖父、外祖母。他们根本没有想到小斯万已经不再同父辈的故旧世交们来往了,而且我们并不觉得斯万这个姓有多显赫,所以我的长辈们接待他简直像接待微服察访的贵人,完全不知道这位客人的真实地位,等于老实正派的旅店老板,无意中留宿了大名鼎鼎的江洋大盗,应该说不知者不罪。我的长辈们哪里想得到他们接待的这位斯万先生其实是跑马总会里数一数二的阔绰的会员,巴黎伯爵和高卢公爵所宠信的密友,圣日耳曼区上流社会中的一位大红人呢?
\par 我们对斯万在交际场中的豪华生涯一无所知,显然部分原因是他本人守口如瓶、性格矜持,但还有部分原因是由于当时的布尔乔亚对整个社会抱有一种印度种姓式的观念,总以为社会是由封闭的种姓阶层组成的,一个人自呱呱坠地那天起,就永远属于他父母所在的阶层,除掉某些偶然情况外——譬如在某个行业中出人头地,或者同门第不相当的家庭联姻,此外再没有别的途径能跻身到高一等的阶层中去。斯万老先生是证券经纪人,小斯万注定一辈子属于那个贫富由收入决定的阶层,钉是钉铆是铆,就跟划分纳税等级一样分明。只要知道他父亲跟什么人交往,就可判断他同什么人交往,以及跟什么人交往才算地位相当。倘若他自己另结新交,那只能算做少不更事,他们家的老世交们,例如我的外祖父、外祖母,对此都能宽宏地视而不见,尤其是他在父亲死后,仍忠心耿耿地来看望我们,我们更应不予计较。但是,有充分理由肯定,他若在大街上遇到那些我们不认识的人,他决不会当着我们的面同他们打招呼的。如果有人硬要给他一个同他的个人情况相符的社会商数,那么,在地位同他父亲相当的其他经纪人的子弟当中,他的这个商数肯定是偏低的,因为他不讲排场,而且对古董和油画“着迷”之极。他如今住在一幢老房子里,家里堆满他收藏的宝贝。我的外祖母总想去参观参观,不过那座房子位于奥尔良滨河街,我的姑祖母认为住在那个地段有失身份。“您是行家吗?我这么问是为您好,因为您有可能弄到些商人转手的次货。”姑祖母曾这么对他说过;她也确实认为斯万是个草包,没有什么高明之处,甚至在智力方面也平平庸庸,这种人在交谈中往往对正经的话题避而不谈,却在琐细的小枝小节上精确到令人乏味的程度,不仅提到菜谱时他不厌其详,而且同我外祖母的两位妹妹议论艺术问题时,他也同样不知趣。她们要他谈谈见解,讲讲他认为某一幅画好在哪里,他居然闭口不谈,简直不顾礼节。要么——如果可能的话——他就提供一大堆具体细节,诸如这幅画由哪家博物馆收藏的,作于哪一年,等等。通常,他只是每次不重复地说段故事,来给我们解闷;不外乎他最近又跟谁遇到了什么事儿,他倒是总选择我们认识的有关人物,比如,贡布雷的药房老板,我们家的厨娘或车夫。不用说,那些故事逗得我的姑祖母笑出声来,但是,她弄不清是什么引她发笑的,是因为斯万总在那些故事中当尴尬角色呢,还是他的故事讲得俏皮:“您真算得上一位典型人物了,斯万先生!”我们家唯独姑祖母有点俗气,所以每当有人提到斯万,她都不惮费神地要提醒不谙内情的人,说斯万本来可以在奥斯曼大街或者歌剧院大街弄到一套住宅的,他是斯万老先生的儿子,父亲起码给他留下四五百万的家当,可是他偏偏乖张任性。我的姑祖母认为,一个人乖张任性,在别人眼里一定显得非常滑稽,所以有一回——那是正月初一,在巴黎,斯万先生送她一包冰糖栗子,当时不少人在场,姑祖母不失时机地问斯万道:“哎!斯万先生,您还住在酒库附近吗?您就是为了一旦去里昂不至于误了火车钟点吗?”说着,她从夹鼻眼镜的上面,用眼角扫了一眼在场的其他客人。
\par 但是,倘若有人把下面的实情告诉我的姑祖母,她会更感到出奇的:这位斯万先生,作为斯万老先生的儿子,完全“有资格”受到“上层资产阶级的淑女名媛们”的款待(这类特权斯万似乎有意让女士们作主),巴黎最德高望重的公证人或法律事务代理人都可以出具担保,但是他却悄悄地过着另外的生活。在巴黎的时候,他说是要回家睡觉去,但一旦离开了我们的家,出门之后才走几步,便折到另外的方向,上别的经纪人或者合股人所不能光顾的沙龙里去玩。这种事情,我的姑祖母倘若知道,准会觉得非同小可,异乎寻常的程度相当于一位学识渊博的妇女同阿里斯泰\footnote{阿里斯泰:希腊神话中的人物;是教会人们养蜂的神仙。}交情颇深,后来听说这位阿里斯泰同她促膝谈心之后,接着就钻进了忒提斯\footnote{忒提斯:希腊神话中的人物;海神。}管辖的汪洋王国,深入到凡人的肉眼所无法看透的海中洞府,而且据维吉尔\footnote{维吉尔(公元前70年—19年):拉丁诗人。有关阿里斯泰的描述,见于他的诗作《农事诗》。}描述,他在那里受到了热烈的欢迎;或者,简单点说,像一幅异乎寻常的画,这倒更容易使我的姑祖母产生联想,因为,在贡布雷,我们的点心盘子上就有那样的画,阿里巴巴出现在我们的餐桌上,当阿里巴巴一旦发觉周围已无人在场时,他会钻进珠宝辉映的山洞里去,谁也想不到洞里竟有那么多耀眼的宝贝。
\par 有一天——那时我们住在巴黎——他在晚饭后来看我们,他为自己穿了一身夜礼服而连连致歉。他走了之后,弗朗索瓦丝说,据车夫透露,他方才是同一位王妃“共进晚餐”的。“对,”我的姑祖母继续织着毛线,连眼皮都没有抬,只是耸耸肩膀,不动声色地挖苦说,“同一位身份不明的王妃。”
\par 所以,我的姑祖母对他相当不客气。她认为,我们请他来做客,是给他面子;夏天,他每回来我们家,总提着一筐自己园子里出产的桃子和覆盆子,而且他每次从意大利旅行回来,总要送给我好几张美术名作的照片;这些,我的姑祖母认为都是理所当然的。
\par 遇到要大摆筵席的日子,偏偏手头又没有制作风味酱汁或凤梨色拉的配方,我的姑祖母就托他想办法弄,但又不请他来赴宴;她居然不觉得这么做有什么不妥,反而认为他还不够体面,不宜请他在招待首次光临的贵客的席面上作陪。如果谈话的内容涉及到法兰西王室的几位亲王,我的姑祖母就对斯万说:“这几位大贵人,您跟我一样,咱们都永远高攀不上,还是不谈算了,您说是不是?”她哪里知道,也许当时斯万的口袋里偏巧正装着一封从特威克汉姆\footnote{特威克汉姆:伦敦西南郊的一个住宅区,法国资产阶级大革命后,不少流亡英国的法王室贵族侨居在那里。}寄来的信呢。赶上哪天晚上,我外祖母的妹妹表演唱歌,我的姑祖母就吩咐斯万推钢琴、翻琴谱,把这么一位斯斯文文的人支使得团团转,她那种不知深浅的粗放做法,就像是不识货的孩子,拿着古董当不值钱的东西玩,根本不知道爱惜。当时在俱乐部会员中那样赫赫有名的斯万,同我的姑祖母心目中所创造出来的斯万,说不定有天壤之别。晚上,在贡布雷的小花园中,铃铛怯怯地响过丁冬两声之后,我的姑祖母便用她所知道的有关斯万家的一切陈年掌故,来充实她所创造的那个默默无闻、毫无主见的人物,并使他生动起来,于是他在黑暗的背影中清晰地显现,我的外祖母则紧跟在他的后面。他只要一开口,我们就认出他是谁。但是,即使从我们日常生活中最微不足道的小事来看,我们谁都不能构成在人人眼中都一样的物质的整体,总是仁者见仁,智者见智;我们的社会人格,其实是别人的思想创造出来的。甚至例如被我们称之为“看望熟人”那样简单的行为,就部分而言,也具有智力的性质。我们用我们所掌握的有关他的一切概念,来充实我们所见到的这个人的音容笑貌。我们的心目中有关他的全貌,不用说大部分包含了上述的概念。最终,那些概念使他的面颊丰满起来,而且贴切地勾画出他鼻梁的轮廓,进而把音量区分得那样纤毫不差,好似音量只是一层透明的外罩,我们每次看到这张脸庞,听到这种声音,我们就又遇上那些概念,并听从那些概念。也许,我的姑祖母、外祖父、外祖母们在勾画斯万的形象时,由于无知而删略了他在社交场中所具备的许多特点,而在别人看来,他的眉宇间充满了一股风流倜傥的英俊气息,只是这股潇洒之气,遇到他的鹰钩鼻,就像遇到了天然屏障那样驻足流连;但是,他们也能在斯万那张失去了魅力的脸盘上,在那片空荡荡的、开阔的眉宇间,在那双已经贬值的眼睛的深处,堆积起半是记忆半是遗忘、模糊而亲切的残迹,那是我们在乡居期间与芳邻每周一次共进晚餐之后,在牌桌边或花园里一起度过的闲暇时光所留下的残迹。我们的朋友的体态外貌,于是像有关他的父母的记忆一样,变得十分充实,当年的斯万成了一位完整的、生动的人。今天,当我回忆由我后来认识得相当准确的斯万,进而联想到早年的斯万,我简直好像是离开了一个人,去接近另一个完全不同的人。在那早年的斯万的身上,我发现了我少年时代的可爱的错误,而且早年的斯万同后来的斯万相似之处很少,倒是更像我当年所认识的其他人,似乎人的一生无非同博物馆一样,其中同一个时代的肖像都具有一种家庭特征,一种相同的色调——早年的斯万,整日悠闲,散发出大栗树、覆盆果和蒿草叶的芳香。
\par 然而,有一天我的外祖母有事去求一位她以前在圣心教堂认识的太太帮忙(由于我们的门第观念,我的外祖母后来不愿意再同她来往了,尽管她们彼此都觉得很相投),出名的望族布永伯爵家的女儿维尔巴里西斯侯爵夫人对我的外祖母说:“我想您同斯万先生很熟吧?他是我的侄儿洛姆亲王家的好朋友。”
\par 那天我的外祖母回家时心情很兴奋。她对维尔巴里西斯侯爵夫人劝她租一套房间住住的那幢门前有悦目园景的大楼赞不绝口,对在大楼院子里开铺子揽活儿的织补匠父女俩尤其满意。她有一条裙子在楼梯上挂破了,求织补匠修补。她说织补匠的女儿简直像颗珍珠,而那位父亲则是她生平所见到的最高雅、最无可挑剔的人,在我的外祖母的心目中,高雅同社会地位绝对无关。她最赏识织补匠的答话,她跟我的妈妈说:“塞维尼\footnote{塞维尼(1626—1696):法国女作家,有《书简集》传世,文笔清丽,感情细腻,措辞委婉典雅。}都说不到那样高雅得体!”相反,当她说到她在维尔巴里西斯夫人家遇到的那位侯爵夫人的侄子时,她的评语却是:“啊,我的孩子,那人太平庸了!”
\par 至于侯爵夫人关于斯万的那席话,其效果非但不能抬高斯万在我的外祖母的心目中的身价,反倒使侯爵夫人降低了身份。我们根据外祖母的信仰,在给予维尔巴里西斯夫人的评价中,为她定下一项义务:她不得做出违背身份的事情;而她居然认识斯万其人,甚至允许自己的侄子同他交往,这是有失体统的行为。“什么!她认识斯万?你不是说她同麦克——马洪元帅还沾点亲吗,她怎么能这样?”我的长辈们对于斯万的社交活动抱有的这种看法,后来更因他同声名狼藉的社交圈内的一位女子结婚而得到进一步的确定。那女子差不多是交际花一类的人物,斯万倒从没有打算把她介绍给我们认识。结婚之后他依然单独来我们家做客,只是来得不那么勤了。我的长辈们认为,仅就那位女子的地位而论,便足以推想斯万通常在什么圈子里鬼混;他们对那个圈子的内情并不知晓,但估计斯万是在那里遇到她的,后来又同她结婚。
\par 但是,有一次我的外祖父从报上得知斯万先生是某某公爵家星期午餐席上忠实的常客。那位公爵的父亲和叔叔都是路易菲利浦当政时显赫的国务要员。外祖父一向对小道消息很有兴趣,因为那些细枝末节能使他的思想潜入莫莱、巴斯基埃公爵和布洛伊公爵等人的私生活中去。他得知斯万同那些国务要员的熟人经常来往,不免喜出望外。我的姑祖母却相反,她对那条新闻的解释于斯万极为不利;凡是在自己出身的“种姓”之外,在自己的社会“阶层”之外另行选择交往对象的人,在她的心目中都等于乱了尊卑的名分,是很讨厌的。她认为,这是贸然放弃长辈们辛苦建立的实惠;有远见的家长们总为自己的儿孙体面地奠定下亲朋关系的基石,让他们日后坐享同牢靠的人亲密交往的成果,岂可轻率地掷置不顾(我的姑祖母甚至不再接见我们家的一位公证人朋友的儿子,因为他同一位亲王家的小姐结了婚,我的姑祖母认为,等于就此由受人尊敬的公证人儿子的身份,下降到据说有时会受到后妃们青睐的冒险家、贴身侍从或马夫之流的卑贱地位)。我的外祖父本打算在第二天晚上乘斯万来吃晚饭的时候,向他打听那几位要人的情况,因为我们新近发现原来他们都是他的朋友。姑祖母狠狠地批评了他的这种打算。另外,外祖母的两位妹妹——这是两位虽具备外祖母的高尚品性却不具备她那份聪明才智的老小姐——也毫不含糊地宣称,姐夫居然有兴致涉及这类无聊的话题,她们万万不能苟同。她们都是洁身自好的人,而且正因为如此,所以决不能对飞短流长的闲话感兴趣;即使具有历史意义的传闻,她们也从不过问;一般地说,凡是同审美与操行无直接关系的话题,她们从不答腔。对于直接或间接涉及到世俗生活的一切谈论,她们打心眼儿里不感兴趣。只要饭桌上出现轻薄的谈吐,或者仅仅是实惠的话题,而两位老小姐又无法把话题引回到她们所热衷的内容上来,她们就干脆暂停听觉器官的接受功能,让它处于开始衰竭的境地。那时,如果我的外祖父必须引起两位小姨的注意,就得求助精神病医生对付精神分散的患者所采用的物理刺激法:用刀刃连击玻璃杯的同时,大喝一声并狠狠瞪上一眼。精神病大夫往往在日常交往中也使用这类粗暴的方法来对付身心完全健康的人,也许是由于职业养成的习惯,也许他们把人们都看做有点疯病。
\par 老太太们也有兴高采烈的时候,譬如说,斯万来我们家吃晚饭的前一天,亲自给她们送来一箱阿斯蒂出产的葡萄酒。我的姑祖母拿着一份登有“柯罗画展”消息的《费加罗报》,在一件展品名字的旁边,注上了“夏尔·斯万先生所藏”这几个字样。姑祖母说:“你们看到没有?斯万居然露脸,名字登在《费加罗报》上!”
\par “我早就跟你说过,他是很有鉴赏力的。”外祖母说。
\par “你当然了,”姑祖母接过话来说,“你的看法总跟我们不一样。”她知道我的外祖母的看法从来跟她不一致,至于我们会不会赞成她,她并没有十分把握,所以她有意硬拉上我们一起来反对外祖母。她竭力想用自己的见解把我们统统纳入反对外祖母的阵营。但是我们偏偏谁都不接话,我的外祖母的两位妹妹表示要跟斯万提到《费加罗报》上刊登的那句小注,姑祖母劝她们千万免开尊口。每当她发现别人身上有个她所缺少的长处,哪怕微不足道,她也要坚决否定,认为不是长处,而是一个缺点;她不仅不会羡慕人家,反而觉得人家可怜。
\par “我认为你们这样做并不会使他高兴;我很清楚,我要是看到自己的名字这样显眼地登在报上,会觉得很扫兴的,倘若有人跟我提到这种事,我决不会沾沾自喜。”
\par 不过她倒没有硬要说服我的两位姨祖母,因为她们俩最怕俗气,所以她们在影射到谁的时候,总能把话说得婉转曲折,达到不露痕迹的地步,甚至连当事人都察觉不到。至于我的母亲,她力求我的父亲答应不跟斯万提到他的妻子,而只跟他提到他所钟爱的女儿,因为据说斯万是为了女儿才同他的妻子结婚的。
\par “你可以只问一句‘她好不好’就行了,他的生活一定过得很不痛快。”
\par 可是我的父亲不乐意:“我才不呢!你尽胡思乱想。这么说不招人笑话吗?”
\par 我们当中只有一个人把斯万的来访当做痛苦的心事,那就是我。因为每当有外人来访,或者只有斯万一人做客,晚上妈妈就不到楼上我的卧室里来同我道晚安了。我总比别人先吃晚饭,然后坐在桌子旁边;一到八点钟,我就该上楼了。我只能把妈妈通常在我入睡时到我床前来给我的那既可贵又纤弱的一吻,从餐厅一直带进卧室;我脱衣裳的时候,还得格外小心,免得破坏那一吻的柔情,免得它稍纵即逝的功效轻易消散化为乌有。所以,越是遇到那样的晚上,我受妈妈一吻时就越有必要小心翼翼。但是,我又得当着众人的面,匆匆忙忙地接过那一吻,抢走那一吻,甚至没有足够的时间和必要的空闲对我的举止给以专心致志的关注:好比头脑不健全的人在关门的时候尽量不去想别的事情,以便疑惑袭来时用关门时留下的回忆来战胜它。
\par 门铃怯怯地响起丁冬两声,那时我们都在花园里休息。我们知道是斯万来访;但是人人都带着疑问的表情面面相觑,并派遣我的外祖母前去侦察。
\par “别忘了,用明确的话感谢他送了酒来。你们也都知道,酒味很醇正,而且有一大箱。”外祖父叮嘱两位姨祖母说。
\par “你们又说悄悄话了,”姑祖母训斥道,“要是上谁家去,听到人家在窃窃私语,多不自在!”
\par “啊!敢情是斯万先生吧!咱们呆会儿问问他,明天是不是大晴天。”我的父亲说。
\par 我的母亲认为,她若一开口就会把我们全家自从斯万结婚以来可能在态度上使他感到的难堪统统消除。她找了一个空当,乘机把斯万领到一边。但是我跟在她后面,我舍不得离开她一步,心里想,呆会儿我要把她留在饭厅里了,我上楼去睡觉不能像每天晚上那样得到她亲一亲的慰藉了。
\par “哎,斯万先生,”母亲说,“您女儿好吗?我相信她一定像她爸爸那样,已经能鉴赏出色的艺术作品了。”
\par 这时我的外祖父走过来,说:“快来呀,同我们一起坐到游廊里来。”
\par 母亲只得把话打住,但是她从无可奈何中又萌生一个微妙的念头,好比优秀的诗人让蛮横的韵律逼出最美的诗句,“呆会儿咱们俩单独说说您女儿的近况吧,”我的母亲悄声对斯万说,“只有当母亲的才体会得到您的苦心。我相信她妈妈也一定会同意我的看法的。”
\par 我们全都围坐在铁桌的四周。我真不愿意想到今天晚上我将无法入睡,独自熬过苦闷的长夜;我尽量说服自己,那些失眠的时刻没有什么了不起,因为明天一早我就会忘记得干干净净;我尽量让自己想到未来,这样,我就能像踏上桥梁似的越过令人心寒的深渊。但是我的思想跟集中了焦点的目光那样被心事绷得很紧,我全神贯注在母亲的身上,容不得半点无关的印象钻进我的心房。各种思想确实都能闯进我的脑海,但是,一切有可能扣动我心扉的美,或者干脆只是可能转移我的注意力的怪念头,统统都被我排斥在我的心扉之外,就像上了麻药的病人,医生给他动手术时他心里一清二楚,只是不感到疼;我也照样能背诵我喜爱的诗,照样能观察到我的外祖父为了诱导斯万谈及奥迪弗雷—巴斯基埃公爵而作出的种种努力,但是背诵的诗句并不能激起我的感情,观察外祖父的举止也不能使我开心。外祖父的努力终于毫无成效。他刚向斯万提到一个与他有关的问题,我的一位姨祖母马上觉得提得不合时宜,等于造成冷场,而她认为只有打破冷场的尴尬局面才是符合礼貌的行为,于是就对另一位姨祖母说:
\par “你倒是想想看,弗洛拉\footnote{此处原文为“赛莉纳”,似有误,应为“弗洛拉”,故从企鹅版的英译本改为“弗洛拉”。},我认识一位瑞典女教师,她把有关斯堪的纳维亚国家合作社的最最有趣的细节,向我作了详细的介绍。咱们应该请她哪天来吃顿晚饭。”
\par “对了!”她的姐姐弗洛拉回答说,“不过我也没有白浪费时间。我在凡德伊先生家遇到了一位德高望重的学者,他跟莫邦很熟,莫邦向他详谈了创造角色的过程。这多有意思。他是凡德伊先生的邻居,我本来不知道!他非常彬彬有礼。”
\par “并非只有凡德伊先生才有彬彬有礼的芳邻。”我的姨祖母赛莉纳高声接口道。由于她胆小怕羞,所以声音特别尖;更由于她深思熟虑,语气显得很不自然。她一面说,一面——用她自己的话说——有意朝斯万那边望了一眼,与此同时,我的姨祖母弗洛拉听出赛莉纳的弦外之音是对斯万送来阿斯蒂葡萄酒表示感谢,所以也望了斯万一眼,那神情既有感谢之意,又带点挖苦,也许她不过是想强调她的妹妹的措辞巧妙,也许她嫉妒斯万居然使她的妹妹如此开窍,善于辞令,更也许她情不自禁地要挖苦斯万几句,因为在她看来斯万已穷于对答了。
\par “我看,咱们可以请那位先生屈趾光临,来用晚餐的,”弗洛拉接下去说,“只要一提到莫邦或者马特纳夫人,他准能一气儿连谈几个钟头。”
\par “那才动人哪。”我的外祖父叹了一口气说;他心想,大自然已经不幸地、彻底地排除了人们对瑞典合作社或者莫邦创造角色之类的问题产生浓厚兴趣的可能性,因为它忘了为我的两位姨祖母的才情增添一点佐料;若要把莫莱或者巴黎伯爵的私生活讲得有滋有味,就得添油加醋。
\par “既然说到这里,”斯万对我的外祖父说,“我下面要说的倒跟您问我的问题很有关系,虽然表面上看并不相干,但从某些方面看,其实并无太大的不同。今天上午,我重读了圣西门\footnote{圣西门(1675—1755):法国作家,公爵,政治活动家,所著《回忆录》是路易十四当政后期以及摄政王时期的重要的历史见证。}的著作,其中有几句话您或许会觉得有点意思。那是有关他出使西班牙的那一卷;在他的全集中,那一卷写得并不出色,只是一本日记罢了,但作为日记,至少写得非常生动;仅就这一点而论,就同我们认为每天非看不可的乏味的报纸有所区别。”
\par “我不同意您的看法,有时候我觉得看报令人非常高兴。”我的姨祖母弗洛拉打断了斯万的话,以此来表示她已经在《费加罗报》上看到了那句注解,说明柯罗的哪幅油画是由斯万所收藏的。
\par 姨祖母赛莉纳连忙补充道:“就是说,当报纸上提到我们所关心的人和事的时候。”
\par “倒也是,”斯万不免感到意外,答道,“我之所以说报纸不好,是因为报上天天让咱们去注意那些无聊的小事,而咱们一生中难得三四回读到含英咀华的好书,既然咱们天天早晨要急于看报,那么他们就应当把报纸办得好一些,增加一些内容,我不知道怎么说才好……比如说,来一点帕斯卡尔\footnote{帕斯卡尔(1623—1662):法国数学家、物理学家、哲学家和作家,对现代实证主义、直觉主义哲学很有影响。}《思想集》之类的文章!(他故意调侃似的把《思想集》三字说得夸张其辞,以免显得学究气)那种切口烫金的精装书,咱们每隔十年不过翻上一回,”他补充一句,像有些社交界人士装得愤世嫉俗,对富丽堂皇的东西不屑一顾似的,“书里咱们又读到些什么?无非是希腊王后幸驾戛纳,莱昂公主举办化装舞会,好像只有这样才合乎规矩。”说到这里,他又后悔失言,把正经事说得过于轻佻。他解嘲似的接着说道:“咱们的话题太高雅了,我不明白为什么咱们要谈论这样‘高深的尖端’。”这时,他转身对我的外祖父说:“还是说圣西门吧。书里说莫莱夫里埃居然有胆量向他的儿子们伸手。您知道,关于这位莫莱夫里埃,圣西门是这么说的:‘他简直像只厚壁酒瓶,里面只有起码的水分,粗俗而愚蠢’。”
\par 弗洛拉赶紧插话道:“酒瓶有薄有厚,我倒是知道有些瓶子里装着完全不同的东西。”她想乘机谢谢斯万,因为那箱阿斯蒂葡萄酒,斯万是送给她们姐妹俩的。
\par 斯万一时十分尴尬,硬着头皮往下说:“圣西门是这样写的:‘我不知道他是无知呢还是存心犯傻,他居然想伸过手去,同我的孩子们握手,我幸亏及时发觉,没有让他得逞。’”
\par 我的外祖父对于“无知呢还是存心犯傻”这种说法佩服得五体投地,可是赛莉纳小姐,由于圣西门这么一位文学家的大名没有让她的听觉功能完全沉入麻痹状态,听到这话顿时义愤填膺:
\par “什么?您居然钦佩这样的描写?好!不过,这能说明什么问题?难道同样是人,这个人就不如那个人吗?人只要聪明、勇敢、善良,公爵也罢,马夫也罢,有什么关系?您的圣西门倒好,居然教他的儿子们不理睬正派人的友好表示,这也算教子有方?简直恶心!您居然敢引为经典!”
\par 我的外祖父眼看谈话遇到这么多的障碍,非常扫兴,感到已不可能诱导斯万讲点他爱听的故事了,于是悄声对我的妈妈说:
\par “上次你告诉我的那句诗是怎么说来着?碰到眼前这种情况,倒可以让我舒一口气。你提个头吧,啊,想起来了:‘主啊,有多少美德您教我们憎恨!’\footnote{原诗应为:“天哪,有多少美德您教我们憎恨。”引自高乃依的悲剧《庞贝之死》。}唉,说得真好啊!”
\par 我两眼盯住了妈妈,我知道,只要一开晚饭,他们就不会让我呆到晚饭结束,为了不使我的父亲扫兴,妈妈不会让我当着大家的面像我在卧室里那样地亲她好几遍的。所以,在餐厅里,在就要开晚饭的时候,在我感到那时间即将来临的当口,我就先为那短促而悄然的一吻,从我力所能及的方面,作好一切准备:我用眼睛选定妈妈脸上的某一个部位,作为我的吻的落点;由于我在精神上已经有了吻的开端,所以我作好思想准备,以便在妈妈把脸凑过来的刹那间,我能充分地感受到我嘴唇贴着的她那部分的肌肤的温存;我好比一个画家要画幅肖像,但是描绘对象只能短暂地出现几次,画家在准备调色板之前,早已根据自己所作的笔记作好细致的回忆,即使描绘对象不在场,他也能画得惟妙惟肖。然而,晚饭的铃声还没有打响,我的外祖父却残忍地说(虽然他并没有意识到自己的残忍):“这孩子看样子很累,该上楼睡觉去了,再说,咱们今天晚饭吃得晚。”我的父亲本来就不如我的母亲和外祖母那样一丝不苟地信守协议,这时说道:“是啊,快,睡觉去。”我想过去亲亲妈妈,就在这一刹那,晚饭的铃声响了。
\par “不必了,别麻烦你的妈妈了。这也就等于道过晚安了,这种表示本来就多余可笑。快点,上楼去!”
\par 我等于连盘缠费都没有领到就得上路;我必须像俗话所说“戗着心眼儿”登上一级一级的楼梯,我的心只想回转到母亲身边去,因为母亲还没有吻我,还没有以此来给我的心灵发放许可证,让她的吻陪我回房。但是,我不得不违心上楼。这可恨的楼梯呀,每当我踏上梯级,总不免凄然若失,那股油漆味可以说已经吸收了、凝聚了我天天晚上都要感到的那种特殊的悲哀,也许正因为如此,一闻到它我才更感到痛心;我的智慧在这种嗅觉的形式下变得木然而丧失了功能。当我们沉入梦乡时,我们不会感到牙疼,只觉得仿佛有一位姑娘掉进水里,我们拼命把她从水里打捞起来,捞起又掉下,掉下又捞起,一连二百次;或者,好比有那么一句莫里哀的诗,我们不停地背诵。处于这种情况,我们只有醒来才能舒口气,我们的智慧才能使牙疼摆脱掉见义勇为的伪装和吟诵诗句的假相。当登楼时的悲哀以迅雷般的速度侵入我内心时,我所感到的却是舒心的反面。这种侵入几乎是顿时发生的,悲哀通过我嗅到的楼梯的特殊的油漆味,突然不知不觉地钻进我的心扉,这比通过精神的渗透更具有毒害心灵的功效。我一进卧室,就得把一切出入口全部堵死,把百叶窗合上,抖开被窝,为我自己挖好墓坑,然后像裹尸一样换上睡衣。那时正当夏令,由于我睡在罩着厚布床幔的大床上太热,他们就为我在房内另外放了一张铁床。我在尚未葬身铁床之前忽然萌生了反抗的念头,我要施个囚犯惯施的诡计,我给母亲写了一封信,说有要紧事要当面禀告,信上不便说,只求她上楼来见我。我只怕弗朗索瓦丝不肯为我送信。她是我的姨祖母家的厨娘,我住在贡布雷的时候,起居由她负责照料。我想,家里有客时要她给我的母亲递信,其难度之大正等于求剧院门房给正在台上演出的女演员送便条,几乎是办不到的。不过,能办不能办,弗朗索瓦丝自有一部严峻专横、条目繁多、档次细密、不得通融的法典,其间的区别一般人分辨不清,也就是琐细至极(所以她那套法典大有古代法律的风貌,那些古代法律残忍处可下令大批杀戮嗷嗷待哺的婴儿,可是有些条文却慈悲得连山羊羔的肉都禁止用母山羊的奶来炖,还禁止啃食动物大腿上的筋)。
\par 有时候,弗朗索瓦丝顽固地拒绝为我们干托她办的事;由此而论,似乎她的“法典”对于上流社会的复杂规矩和交际场合的种种讲究都有所估计,而这些,单凭她这样一个农村女仆的所见所闻,是得不到任何暗示的。我们只能说,她身上有一种非常古老、高尚,但又不为人们所理解的法兰西传统陈迹,好比我们在那些手工业城市中所见到的那样,陈旧的华屋证明往昔曾是王公幸驾之地,化工厂的工人们从事劳动的场地周围,有古老的雕塑珍品,主题有泰奥菲尔遇到圣母显灵,或者埃蒙四兄弟乘坐神马逞威。\footnote{泰奥菲尔和埃蒙四兄弟均为传说中的人物。相传公元六世纪时僧侣泰奥菲尔曾把灵魂卖给了魔鬼,后追悔莫及,遂祈求圣母救助,终以诚心感动圣母,显灵勾销了卖魂契。十三世纪时游吟诗人吕特贝夫曾把这一传说编成诗体说唱,广为流传,后来壁画和浮雕等美术形式也采用这一主题。埃蒙四兄弟的故事见诸十二世纪法国英雄史诗《勒诺埃德·蒙多邦》。相传埃美公爵有四子:勒诺、阿拉尔、吉夏尔和里查,统称“埃蒙四子”(“埃蒙”为“埃美”的昵称或贱称),他们在同查理大帝作战时,勇武异常,有坐骑名巴雅尔,一跃千尺。}
\par 至于我当时的那个特殊情况该如何发落,弗朗索瓦丝的“法典”自有毫不含糊的规定:尊长敬客。所以除非发生火灾,她多半不可能为我这区区小儿去惊扰正陪着斯万先生说话的母亲大人。弗朗索瓦丝经常教训说:不仅对父母长辈要孝敬,对亡人、僧侣和王上要恭敬,还应该尊敬受到款待的宾客;这一套敬人之言倘若出自某部著作,我或许会深受感动,偏偏出自她的口中,我听了不免又气又恼,尤其是因为她说得那么一本正经,细声细气;尤其是今天晚上,她把请客吃晚饭看成神圣的礼仪,结果她必定拒绝惊扰宴会的礼仪。不过我还是要试试运气,于是我毫不迟疑地撒谎说,这封信并非我自己要写,我上楼时妈妈吩咐过,看看有没有她要找的东西,务必给她一个答复;要是不给妈妈捎句话去,她会生气的。我明明知道弗朗索瓦丝根本不信,她跟原始人一样,感觉比咱们灵敏得多,能从一般人觉察不到的征兆中一眼看透咱们企图掩饰的真相。她把信封足足端详了五分钟,好似单凭审察纸质和笔迹便可知道信封里的内容,换句话说,便可确定应按她那部“法典”中的哪一项“条款”来处置。随后,她无可奈何地走出房间,那表情等于说:“唉!有那样一个孩子,做父母的也真算倒霉!”转眼间她又回来了,说现在席上正在用冰冻甜食,大师傅无法当着众人的面把信递给我妈妈,得等到上漱口盅的当口才有法子送去。我的焦虑顿时得到冰释,顷刻间乾坤扭转,方才我离开母亲还意味着得等到明天才能重聚,可是呆会儿我的便条至少会把无影无踪的我,喜孜孜地带进妈妈所在的那间厅堂,而且会在我妈妈的耳畔悄悄地谈论我;虽然母亲看到便条肯定会不高兴(而且由于我的拙劣手段将使我在斯万的眼中显得十分可笑,她更会加倍地生气)。一秒钟之前,我还觉得餐桌上的冰冻甜食——“坚果冰淇淋”以及漱口盅之类的享受无聊透顶,邋遢可憎,因为我的妈妈是在我不在场时独自享受的。可现在,那间原来对我极不友好,禁止入内的餐厅,忽然向我敞开大门,就像一只熟得裂开了表皮的水果,马上就要让妈妈读到我便条时所给予我的亲切关注,像蜜汁一般从那里流出来,滋润我陶醉的心房。我与母亲已经不再相隔异处;屏障倒塌了,柔情的丝丝缕缕重又把我和她系到一起。而且,还不止如此,妈妈还一定会上来看我!
\par 我方才苦恼地想:斯万如果看到我给母亲的信,并且猜出我的用心,一定会瞧不起我;然而我后来才知道,他一生之中对类似的苦恼有过长期的体会,谁也比不上他更了解我。自己所爱的人在自己不在场或不能去的地方消受快乐,对他来说,是一件烦恼苦闷的事,是爱情教他尝到的滋味。那样的烦恼苦闷,从某种意义上说,本来就注定属于爱情,而且一旦落入爱情之手它就变得具有专门的含义;但是它钻进像我这样生活中还没有出现过爱情的人的心中,它实际上是对爱情的期待;它漫无目的、自由自在地游动着,并无一定的钟情对象,只为某一天出现的某种感情效劳,这种感情有时是对父母的依恋,有时是对同伴的友谊。
\par 弗朗索瓦丝回来告诉我说,我的信即将交给母亲。那时我感到无比的喜悦。我在感情见习期所领受到的这种喜悦,斯万也早就体会过:这其实不过是哪位好心的朋友,或者我们心爱的女子的哪位亲戚,让我们空欢喜一场罢了。比如说,我们来到哪家公馆或者哪家剧院,知道我们的心上人也来这里参加舞会或者观看首场演出,这时有位朋友先是发现我们在门外踯躅,几近绝望地等待着同心上人接近的机会。那位朋友认出我们是谁,热心地过来招呼,问我们来这里有何贵干。我们就胡乱编套谎话,声称有要紧事必须告诉他的某位女亲戚或者某位女朋友。他连忙请我们放心,说这事再好办不过;他把我们领进门厅,答应五分钟之内一定送她下楼。我们多感激他呀——正等于这时我多感激弗朗索瓦丝!这样与人为善的中间人,仅凭一句话就改变了我们的心境:刚才我们还认为里面的灯红酒绿一定乌七八糟到不堪设想的地步,而且其中必有几股同我们作对的、邪恶的、蛊惑人心的旋风把我们的心上人裹胁而去,让她嘲笑我们;可是顷刻之间,我们觉得这样的晚会还过得去,有人情味,甚至大有好处!若以那位向我们打招呼的朋友的态度来看(因为他也是晚会中的一员),我们可以推断其他宾客不至于会有多坏。原先我们不知道她在里面会享受到什么样的乐趣,那漫长的时辰可望而不可即,残酷地折磨人的感情,如今却出现了一个供我们潜入其间的缺口;在构成那些时间的序列中有那样一个时刻,同其他时刻一样真实,却又更为重要,因为它同我们的心上人关系更为密切,它活灵活现地出现在我们的眼前,我们占有它,参与其间,它几乎是我们自己创造出来的,这就是有人要去告诉她,我们就在楼下的那个时刻。也许,晚会的其他时刻同那个时刻并无本质的差别,并不更令人心醉而使我们痛苦万分,因为好心的朋友已经明白告诉我们:“她肯定会非常高兴下来的!跟您谈谈总比在楼上百无聊赖要好得多。”唉!斯万有过这方面的经验:感到她所不爱的人处处跟踪,甚至一直盯到晚会的门口,她岂能不生气?而第三者的好心并不能打消她的气恼,结果经常是只有那位好心的朋友一人下楼。
\par 我的母亲没有来,甚至连一点面子(也就是不拆穿我编的那套找东西的瞎话)都不肯给,反倒让弗朗索瓦丝对我说:“不理!”后来我经常听到大旅社的门房或者游乐场的听差对可怜巴巴的姑娘说同样的话。那姑娘惊讶地反问道:“什么?他不理?怎么可能呢?您确实把我的信交到他手里了吗?那好!我再等等。”而且,这样的姑娘无一例外,都不需要门房给她另点一盏小煤气灯;她只在黑角落里静候,偶尔能听到门卫同跑堂嘀咕几句天气好坏之类的话,接着门卫就发觉时间不早,打发跑堂赶紧把某位顾客吩咐的酒拿去冰镇。——我当时谢绝了弗朗索瓦丝的好意(她自告奋勇要给我泡杯药茶),我也不要她留下陪我,只让她回配膳室去。
\par 我钻进被窝,合上眼睛,尽量不去听他们在花园里喝咖啡时的聊天声。这样过了几秒钟,我感到其实早在我给妈妈写信的那会儿,早在我不顾她会生气向她靠拢甚至以为马上就要同她聚首的那会儿,我已经把见不到妈妈我照常睡觉的路子给堵塞了。我的心突突乱跳,阵阵发痛,本指望以逆来顺受求得安宁,结果反而增添心中的骚乱。突然间,我的烦恼烟消云散,像服了一剂强烈的镇静药,到这时才开始见药效;痛苦消释,周身舒坦:因为我下了决心,不再勉强自己在见到妈妈前就入睡,我要等妈妈上楼睡觉时,不顾一切地去同她亲一亲,虽然这事肯定会惹得她接连几天同我生气。烦恼既消,平静使我感到异常的喜悦,那种异样的感觉,不亚于期待、饥渴和如临深渊的恐惧。我轻轻推开窗户,坐到床前,几乎一动不动,生怕楼下的人听到我的动静。窗外万籁也仿佛凝固在静寂的期待中,唯恐扰乱明净的月色;月亮把自己反射的光辉,延伸到面前的万物之上,勾画出它们的轮廓,又使它们显得格外悠远;风景像一幅一直卷着的画轴被徐徐展开,既细致入微,又恢宏壮观。需要颤动的东西,如栗树枝头的叶片,在轻轻颤动。但它颤动得小心翼翼、不折不扣,动作那样细密而有致,却并不涉及其他部分,同其他部分判然有别;它独行其是。远处的嗡嗡声扩散在不吸音的寂静之中,听来像是从市区那一边的花园中传来的,那么微弱又那么清晰,好比是轻声的演奏,像音乐学院的乐队十分高明地演奏轻音的乐段,每一个音符都像是从离音乐厅很远的地方传来的,但又都清晰可辨。音乐会上的常客侧耳倾听——倘若斯万请客,我的两位姨祖母也能有幸在座——他们似乎在一支军队还没有拐进特雷维斯街之前就已经能听到远处前进的脚步声了。
\par 我心中有数,我当时把自己置于最不利的境地,最终会从我的长辈们那里得到最为严厉的处罚,其严厉的程度,外人实际上是估计不到的。他们或许以为,充其量是犯了真正丢脸的过错所造成的那种后果吧。但是,在我所受到的教育中,错误的轻重次序,同其他孩子所受的教育很不一样。大人们早已使我习惯于把一些错误看得比另一些错误严重(否则我或许没有必要受到那样细心的管教了)。我现在才明白,凡属严重错误都有一个共同的性质:那就是没有克制感情的冲动。不过当时谁都没有这么说罢了。谁都没有指出错误的根源,因为倘若说穿,我或许会认为自己情有可原,或者甚至认为自己本来就没有能力克制。不过对于错误的来龙去脉我并不陌生:在犯错误前,我必定先感到极其苦恼;犯错误后,我又必定受到严厉的处罚。我知道,我刚才的错误,与我过去因而受到重罚的错误属于同一性质,虽然程度上这次要严重得多。倘若等我母亲上楼睡觉时,我迎上前去,她见我为了同她说声晚安居然等候在过道里而一直没有睡觉,那么,她就会再不让我住在家里了。等天一亮,她会把我送去住校,这是一定的。唉!难道五分钟之后我只有跳楼吗?我倒宁可跳楼的。现在我的全部愿望是见到妈妈,同她说声晚安。为了实现这一愿望,我已经走得太远,再想回头已不可能。
\par 我听到大人们送斯万出门的声音;门铃告诉我斯万已经走远。我伏到窗前,听妈妈问父亲:龙虾的滋味是否可口?斯万先生是否又添了一次咖啡腰果冰淇淋?妈妈还说:“我觉得龙虾味道一般,下次我要用别的香料来做。”
\par “我都不知道怎么说才好,总觉得斯万的模样变多了,”我的姑祖母说,“他都成老头儿了!”
\par 姑祖母一向惯于把斯万看做一成不变的小伙子,一旦发觉斯万比她想象中的年纪要显老些,她就大惊小怪。而其他人则开始议论说斯万的这种老相不正常,太过分,有失面子,只有单身汉才这么老气横秋呢;对于那些单身汉来说,不是觉得大白天得过且过,没什么盼头,就是觉得大白天长得要命,因为他们心目中白天是空洞的永昼,没完没了的钟点自天亮之后就开始增多,他们却没有子女来共同分享这些时间。
\par “我相信,他那位爱卖俏的妻子够他操心的。在贡布雷谁不知道她跟一位夏吕斯先生同居呀?传得满城风雨。”
\par 我的母亲倒发觉斯万先生近来脸色开朗多了:“他一不顺心,就跟他父亲当年一样,揉眼睛、摸脑袋。不过他近来这种动作少多了。照我看,他其实已经不爱他的妻子了。”
\par “那是自然的,他已经不爱她了,”外祖父说,“我收到过他的一封信,这是很久以前的事了,信上说到这件事。我尽量不把它当真,不过他在信里倒把自己的感情表白得很清楚,至少说明他对妻子的爱情已经淡漠下来,哎!你们俩呀你们俩!怎么不谢谢他送来的阿斯蒂麝香葡萄酒呢?”外祖父转身问他的两位小姨子。
\par “怎么?我没有道谢吗?说句良心话,我还以为自己转着圈儿已经对他委婉地表达了谢意呢。”姨祖母弗洛拉回答说。
\par “不错,你转弯抹角地说得很得体,我真钦佩你。”姨祖母赛莉纳说。
\par “你也一样,说得很有分寸。”
\par “是的,我提到芳邻的那段话,连我自己都深感得意。”
\par “什么?你们这也算感谢人家!”外祖父失声叫道,“这些话我倒都听到了,不过我怎么也想不到你们是说给斯万听的。你们不必怀疑,我认为他根本没有听出你们的弦外之音。”
\par “看你说的,斯万可不是笨人,我肯定他领会到了。我总不能跟他提到几瓶酒、多少钱吧?”
\par 我的父亲和母亲在花园里单独地坐了一会儿,后来父亲说:“咱们上楼睡去吧,好吗?”
\par “你愿意上楼咱们就上楼吧,亲爱的,虽然我现在一点都不困;倒不是冰淇淋里的那点儿咖啡弄得我这样精神,我发觉用人的房间里灯还没灭,可怜弗朗索瓦丝一直在等我呢。我要去请她帮我解开紧身上衣后面的搭扣,你先更衣去吧。”
\par 母亲打开了安着铁花条的门,走进正对着楼梯的门厅。我很快就听到她上楼关窗的声音。我蹑手蹑脚走进过道,心怦怦乱跳,激动得几乎寸步难移,不过这至少不是难过得心跳,而是提心吊胆,是过分兴奋。我看到楼梯井下烛光摇曳,那是我母亲秉烛上楼,接着我看到了妈妈,我扑上前去。她先是一愣,不知道是怎么一回事。随后她显出怒容,一声不吭,事实上过去为了更微不足道的过错她都能一连几天不理我。如果那时妈妈对我说一句话,这虽然意味着她不会不理我,但对我来说也许是更可怕的征兆,因为比起严厉的惩罚来,不理我、生气毕竟只能算不足挂齿的小事。她若开口,那就像辞退用人似的,虽说得平心静气,但是下了决心的;送儿子出门的母亲,给儿子一吻是为了告别;而只想跟儿子生几天气就了事的母亲是不肯吻儿子的。然而这时妈妈听到已经换好衣裳的父亲走出更衣室上楼来了,为了避免父亲训我一顿,她急得呼哧呼哧对我说道:“快跑,快跑,别让你爸爸看到你像个疯子似的等在这儿!”
\par 可是我还是反复地说:“来跟我说声晚安!”我一面说,一面提心吊胆地看着父亲的烛光已经照到楼梯边的大墙上。不过父亲越来越近倒正好可以被我用来作为一种讹诈的手段,我希望妈妈为了避免父亲见到我,对我说:“先回到房里去,我呆会儿来看你。”
\par 来不及了,父亲这时已经出现在我们的跟前,我不觉念念有词地说了句谁也没有听到的话:“完了!”
\par 然而我并没有遭殃。父亲向来不像妈妈和外祖母那样对我宽容,允许我这样那样;凡她们允许的,父亲总不允许。他根本不顾什么“原则”,也谈不上什么“人权”。譬如例行的散步,别人是不会不让我去的,即使不让,起码也得给我许个愿。父亲却随口说个理由,或者干脆毫无理由,就在将要出发之前突然取消我去的权利。要么就像今天晚上那样,明明离开晚饭的时间还早,偏打发我快走:“上楼睡觉去,不必多说!”但是,也正由于他如外祖母所说没有原则,也就无所谓坚持了。
\par 他绷着脸奇怪地看我一眼。后来妈妈尴尬地解释几句。他说:“那你去陪陪他吧。你不是说还没有睡意吗?你就呆在他房里好了,反正我不需要你照应。”
\par “可是,亲爱的,”母亲不好意思,回答说,“这跟有无睡意无关,总不能惯孩子……”
\par “谈不上惯,”父亲耸耸肩膀,“事情明摆着,这孩子心里不痛快,脸色那么难看,做父母的总不能存心折磨他吧!等他真弄出病来,你更要迁就他了。他的房里不是有两张床吗?吩咐弗朗索瓦丝为你收拾一下大床,你今晚就陪他睡吧。好,晚安,我不像你们那么好激动,我可要睡了。”
\par 我还不能够感谢父亲;他凡是听到他称之为感情用事的话,只会恼怒。我不敢有所表示;他还没有走开,已经在我们跟前显得那么高大,他穿着一身白色睡袍,头上缠着淡紫和粉红两色的印度开士米头巾;自从得了头痛病之后,他睡觉总以此缠头。他的动作就像斯万先生送给我的那幅版画中的亚伯拉罕\footnote{亚伯拉罕:圣经中的人物,据说是希伯莱人的祖先。上帝为了考验他,要他献出自己的儿子以撒祭神,他同意了。撒拉是他的妻子。},那幅版画是根据伯诺索·戈索里\footnote{伯诺索·戈索里(1420—1497):意大利画家。上面说到的那幅画系他所作的二十三幅“旧约故事”中的一幅,作于1468—1484年,原存比萨“康波·圣托”教堂,第二次世界大战时毁于兵燹。}的原作复制的,画中亚伯拉罕要萨拉狠心舍弃伊萨克。这已经是多年前的事了。当年烛光渐升的那面楼梯旁的大墙早已荡然无存。有许多当年我以为能在心中长存不衰的东西也都残破不堪,而新的事物继而兴起,衍生出我当年意料不到的新的悲欢;同样,旧的事物都变得难以理解了。我的父亲也早已不会再对我的母亲说:“陪他去吧。”出现这种时刻的可能性对于我来说已一去不复返。但是,不久前,每当我侧耳倾听,我居然还能听到我当年的哭泣声。当着父亲的面我总竭力忍着,等到与母亲单独在一起时我才忍不住地哭出声来。事实上这种哭泣始终没有停止过;只因为现在我周围的生活比较沉寂,才使我又听到了它,好比修道院的钟声白天被市井的嘈杂所掩盖,人们误以为钟声已停,直到晚上万籁俱寂时才又遐迩可闻。
\par 那天晚上我的母亲就在我的卧室里过夜;我犯了这样严重的错误,准备受到让我离家住校的惩罚,不料父母却对我恩宠倍加,过去我做了好事都从来没有得到这样的奖赏。我的父亲即使对我恩宠倍加,他的举止言谈仍具有专制武断、奖罚不当的成分,这已成为他行为的特征;在一般情况下,他办事多凭兴之所至,难得深思熟虑。他打发我睡觉去的时候,那种态度我称之为严厉恐怕太过分,其实赶不上妈妈和外祖母严厉。他的天性在许多方面虽说同我很不一样,但同妈妈和外祖母就更有天壤之别。他八成直到现在都没有猜到我每天晚上有多伤心一,而这点妈妈和外祖母却了如指掌,只是她们太疼我了,不忍心让我尝到痛苦的滋味,她们要我自己学会克服痛苦,以此来减轻我多愁善感的毛病和磨炼我的意志。至于父亲对我的疼爱,那是另一种类型的,我不知道他有没有她们那样的勇气:他只要一发现我心里不痛快,就对我的母亲说:“去安慰安慰他。”
\par 妈妈那天晚上就呆在我的房里了。弗朗索瓦丝看到妈妈坐在我的身边,握住了我的手,任我哭个不停也不训斥我,她看出必定发生了什么非同小可的事,便问妈妈:“夫人,少爷怎么啦,哭成那样?”我本来是有权盼望妈妈来同我道晚安的,可是眼下的情况那样不同,妈妈看来不想以任何懊恼之情来损害这不同寻常的时刻,便这样回答说:“他自己也弄不明白,弗朗索瓦丝,他神经太紧张;快给我铺好大床,然后上楼睡去吧。”就这样,破天荒头一回,我的忧伤没有被看做应该受罚的过错,而是一种身不由己的病症。方才妈妈正式承认了,这是一种精神状态,我是没有责任的;我松了一口气,我不必在苦涩的眼泪中掺进什么顾忌了,我可以痛哭而不至于犯下过失。在弗朗索瓦丝面前,我深为这种人情的复归而自豪。一小时前,妈妈拒绝上楼到我的房间里来,还不屑一答地吩咐我快睡;如今她那番通情达理的话,把我抬到了大人的高度,使我的痛苦一下子脱离了幼稚的境界,达到成熟,我的眼泪由此获得解放。我应该感到高兴,然而我不高兴。我觉得母亲刚才对我作出的第一次让步,她一定很为之痛心,她第一次在她为我所设想的理想面前退缩;她那么勇敢的人,第一次承认失败。我觉得,我取得胜利是跟她作对;我使她的意志松懈、理性屈服,不过是因为她怜恤我有病,怕我伤心过度,顾念我年幼。我觉得那天晚上开始了一个新纪元,而且将成为一个不光彩的日子留传下来。倘若当时我有勇气开口,我就会对妈妈说:“不,我不要,你别睡我这儿。”但是,我深知妈妈有审时度势之明,用现在的说法,就是很现实主义。这种明哲的态度,使她的理想主义天性有所收敛,不像外祖母那样热得像团火。我心里有数,现在既然毛病发作,妈妈宁可让我起码得到些慰藉,免得惊动父亲。当然,在妈妈那样温柔地握着我的手,想方设法止住我眼泪的那天晚上,她的俊俏的脸庞还闪耀着青春的光彩;但是,我偏偏认为不该这样。她若怒容满面,我或许还好受些;我童年时代从来没有见到过她这样温情脉脉,这反倒使我感到悲哀。我仿佛觉得自己忤逆不孝,偷偷地在她的灵魂中画下第一道皱纹,让她的心灵长出第一根白发。想到这里,我就哭得更凶了。这时候,我看到了从来没有依我亲昵撒娇的妈妈,突然受到我情绪的感染,在竭力忍住自己的眼泪。她感到我看出她想哭,便笑着对我说:“瞧,我的小宝贝,我的小傻瓜,再这么下去,弄得妈妈也要像你一样犯傻劲儿了。好了好了,既然你不想睡,妈妈也不困,咱们别这么哭哭啼啼地待着,倒不如干些有意思的事,拿出一本书看看吧。”可是偏偏房间里没有书。“要是我把你外祖母准备在你生日那天送给你的书先拿给你,你不会不高兴吧?想好了,等到后天你什么礼物也没有,你不会失望吧?”
\par 正相反,我高兴极了。妈妈去拿了一包书来,从包装纸看,那些书又短又宽,仅凭这初步印象,(虽然是笼统的,而且还隔着一层纸)它们的吸引力就已经大大超过新年颜料盒和去年的蚕宝宝了。那几本书是《魔沼》、《弃儿弗朗沙》、《小法岱特》和《笛师》。后来我才知道,外祖母起先挑选的是缪塞的诗、卢梭的一本著作,还有《印第安娜》\footnote{《印第安娜》也是乔治·桑所著的小说。};因为,外祖母固然认为无聊的书同糖果点心一样对健康有害,但她却并不否认天才的恢宏气魄甚至对一个孩子的思想都能产生影响,这种影响不见得比旷野的空气和海面吹来的风更不利于健康,更缺乏振作活力的功效。但是当我的父亲得知她送我那几本书时,几乎把她看成疯子,因而她只好再次亲自出马,光顾舒子爵市的书店,免得我不能及时拿到礼物(那天的天气热得灼人,外祖母回家时难受极了,医生警告我母亲说:以后切不可再让她累成那样)。外祖母一下就选中了乔治·桑的这四本田园小说,“我的女儿,”她对我妈妈说,“我总不能存心给孩子买几本文字拙劣的书看呀。”
\par 确实,我的外祖母从不凑合买那些智力方面得不到补益的东西,她尤其看重能教我们在物质享受和虚荣满足之外寻求愉快的优美的作品。即使她有必要送人一件实用的礼物,譬如一把交椅,一套餐具,一根拐杖,她也要去找“古色古香的”,似乎式样既然过时,实用性也就随之消失,它们的功用也就与其说供我们生活所需,倒不如说在向我们讲解古人的生活。她希望我的卧室里挂几张古建筑的照片,或者很美的风景图片。可是当她去选购时,虽然照片上的内容不乏审美价值,她总觉得照相这种机械复制方式,让平庸和实用过于迅速地得其所哉了。她要想办法做点手脚,虽说无法完全排除商业性的俗气,但至少要削弱它,在大的方面仍用艺术来取代它,给它引进一些艺术的“厚度”:譬如说,不要实景照片。她问斯万:有哪位大画家画过沙特尔大教堂、圣克鲁大喷泉和维苏威火山?她宁可送我油画照片:柯罗的《沙特尔大教堂》,于贝尔·罗贝\footnote{于贝尔·罗贝(1733—1808):法国版画家、油画家。}的《圣克鲁大喷泉》和透纳\footnote{透纳(1775—1851):英国画家,是印象派的先驱者之一。}的《维苏威火山》;虽说仍是照片,艺术档次毕竟高了一级。但是,倘若摄影师不拍古建筑,不拍自然风景,这些都由大艺术家去描绘,摄影师只拍艺术家画下来的景物,那么,他倒算做得更名正言顺了。一触及流传甚广的作品,我的外祖母就千方百计稽古溯源,她请教斯万,某某作品有没有版画复制品?倘若有,她倒更看重一些旧版画,因为在版画本身之外另有一种价值,例如那些临摹杰作原貌的版画,而杰作原貌今天我们已经无幸拜识了(就像莫冈在达·芬奇的《最后的晚餐》原作变样以前临摹刻制的那幅版画)。
\par 应该说,用送礼物来理解艺术,这种方法并不总能收到辉煌的功效。提香有一幅画,画的是威尼斯,据说背景是环礁湖,我从那幅画上所得到的威尼斯印象,肯定不如照片所能给予我的印象准确。我的姑祖母倘若存心跟外祖母作对,开一份清单,一一列举她送了多少把交椅给新婚夫妻或老夫老妻,那些椅子的最初受礼者是想日常使用的,可是椅子经不起坐者的体重,立刻散架垮掉,那么这笔账无人能算得清。然而我的外祖母认为太在乎家具结实的程度未免鼠目寸光,木器上明明还留有昔日的一点风采,一丝笑容,一种美的想象,怎能视而不见?那些木器虽说从我们已经不习惯的某个方面还符合某种需要,但就连这一点也能像一些老掉牙的成语那样使她欣赏备至,我们却只能从中看到一种在我们现代语言中已经被习惯磨损得影迹莫辨的隐喻。外祖母作为生日礼物送给我的那几本乔治·桑的田园小说,恰恰就像一件旧家具那样,里面充满了过时的短语,早已变成了形象化的说法,除了农村,别处已经听不到还有人这么说了。我的外祖母在一大堆书中偏偏选购这几本,正等于她更乐于赞美一所有哥特式阁楼之类老式点缀的住宅,这些东西能使她心头萌生一种自得其乐的情绪,使她生发思古的幽情,可以领她到往昔的岁月中去作一番不可能实现的漫游。
\par 妈妈坐在我的床边;她拿了一本《弃儿弗朗沙》。发红的封面和莫名其妙的书名,在我的心目中,给弗朗沙平添一种明显的个性和神秘的魅力,我还从未读过名副其实的小说。过去听说乔治·桑是典型的小说家,仅凭这一点,就足以使我想象《弃儿弗朗沙》中一定有某种难以界定的、引人入胜的内容。用来煽起好奇之心或恻隐之情的叙述手段,某些令人不安和催人惆怅的表达方法,有点知识的读者一眼就看出这些同别的许多小说一样;可是在我眼里,它们却是感人肺腑的一种外观,流露出《弃儿弗朗沙》所特有的本质。我并不把一本书看成一件有许多同类的事物,而把它当做与众不同的人,其存在的理由只在于它自身。在书中那些日常事件中,司空见惯的情节里,短而又短的字里行间,我感到一种奇特的语调,别具一格的抑扬顿挫。故事在展开,我却觉得晦涩费解,更何况我往往一连读上几页,心里都在想别的事。这样分心的结果造成连贯情节的中间出现一段段接不上茬的空隙,再加上妈妈朗读时凡描写爱情的地方都略去不念,空隙更有增无已,所以磨坊姑娘与那小伙子之间各自的态度发生令人费解的变化,在我看来就好像打上了非常神秘的印记;其实,他们之间萌生的爱情得到了发展,足可解释那些变化,我却一厢情愿地设想神秘的根源出自“弃儿”这个名称。我不知道这个名称的含义,只觉得听来受用;我不明白那个小伙子为什么叫“弃儿”,这称号给他披上了一层鲜艳、绚丽和迷人的色彩。
\par 我的母亲朗读时固然常常不忠实于原文,可是她朗诵起来也着实令人钦佩。凡读到感情真挚处,她不仅尊重原意,而且语气朴实,声音优雅而甜润。甚至在日常生活中,倘若有人(且不说什么艺术品)引起她类似的爱怜或钦佩,她也能从自己的声音、举止和言谈中,落落大方地避免某些东西,做到恭谦待人:为了不使曾经遭受丧子之痛的母亲勾起往日的旧恨,她避开活泼的词锋;为了不使老人联想到自己已届风烛残年,她不提节日和生日;为了不使年壮气盛的学者感到兴味索然,她不涉及婆婆妈妈的话题。她如此恭谦大度,实在令人感动。同样,我的母亲读乔治·桑的散文,还能读出字里行间所要求的种种自然而然的温情和豁达亲切的意蕴。乔治·桑笔下充满善良和高雅的情操,外祖母的教诲早已使妈妈学会把这两种情操看做生活中的高尚品格(直到后来我才让妈妈明白它们在文学作品中未必是高尚的品格),所以她朗读时细心地从声音中排除掉一切狭隘情绪和矫揉造作的腔调,以免妨碍感情的洪流涌进字里行间。乔治·桑的字字句句好像是专为妈妈的声音而写的,甚至可以说完全同妈妈心心相印。为了恰如其分,妈妈找到了一种由衷的、先于文字而存在的语气;由它带出行文,而句子本身并不能带出语气;多亏这种语调,她在朗读中才使得动词时态的生硬得到减弱,使得未完成过去时和简单过去时在善中有柔、柔中含忧,并引导结束的上一句向开始的下一句过渡;这种过渡,有时急急匆匆,有时却放慢节律,使数量不等的音节服从统一的节奏,给平淡无奇的行文注入持续连贯、情真意切的生气。
\par 悲哀一俟平息,我便沉溺在妈妈伴我过夜的温情之中。我知道如此夜晚不可再得,我最大的心愿莫过于在夜间如此凄凉的时刻有妈妈在房中相伴;这种心愿同生活的需要和大家的期望太对立了,简直是南辕北辙,所以那天夜间我暂得的满足不过是勉强的例外。明天我的苦恼照常还会出现,而妈妈却不会再留在这里。但是只要我的焦虑一时得到平息,我就不知焦虑为何物了;况且明晚毕竟还远,我心中盘算:到时候再想办法,时间并不会给我带来更大的神通,因为事情毕竟不由我的愿望决定;只是现在事情还没有落到我的头上,这就更使我觉得侥幸避免是可能的。
\par 就这样,在很长一段时期内,每当我半夜梦中回忆及贡布雷的时候,就只看到这么一块光明,孤零零地显现在茫茫黑暗之中,像腾空而起的焰火,像照亮建筑物一角的电光,其余部分都沉没在黑夜里。这块光明上尖下宽:下面是小客厅、餐厅、花园中幽暗小径的开头一截(无意中造成我哀愁的祸首斯万先生要从那面走来)和门厅(我要由此而踏上楼梯的第一级),而攀登起来令我心碎的楼梯则构成这个不规则棱锥体的非常狭窄的锥干;顶部是我的卧室、卧室外的过道、过道口的玻璃门,我的母亲就是从那里进来的。总之,老在晚上那个钟点见到、同周围事物完全隔绝、在黑暗中孤零零地显现的,就是这么一幕简而又简的布景(等于一般老式剧本的开头为供外省演出参考而作的布景提示),为了重演我更衣上床的那出戏,这些道具是少得不能再少了;似乎贡布雷只有楼上楼下,由一部小小的楼梯连接上下,似乎只有晚上七点钟这一个时辰。说实话,倘若有人盘问我,我或许会说贡布雷还有别的东西,别的时辰。但,那将是我有意追忆,动脑筋才想到的一鳞半爪;而有意追忆所得到的印象并不能保存历历在目的往事,反正我决不会自愿地去回想贡布雷的其他往事。它们在我的心目中其实早已死了。
\par 永远消亡了?可能吧。
\par 这方面偶然的因素很多,而次要的偶然,例如我们偶然死去,往往不允许我们久久期待首要的偶然带来的好处。
\par 我觉得凯尔特人\footnote{凯尔特人:公元前2000年在中欧形成的一个印欧语系的种族。他们自青铜时代起,从莱茵河及多瑙河之间的地区向西扩展,进入高卢中部。公元前六世纪至前二世纪,是他们扩张的极盛时期;公元前一世纪左右为罗马人所征服。}的信仰很合情理。他们相信,我们的亲人死去之后,灵魂会被拘禁在一些下等物种的躯壳内,例如一头野兽,一株草木,或者一件无生物,将成为他们灵魂的归宿,我们确实以为他们已死,直到有一天——不少人碰不到这一天——我们赶巧经过某一棵树,而树里偏偏拘禁着他们的灵魂。于是灵魂颤动起来,呼唤我们,我们倘若听出他们的叫唤,禁术也就随之破解。他们的灵魂得以解脱,他们战胜了死亡,又回来同我们一起生活。
\par 往事也一样。我们想方设法追忆,总是枉费心机,绞尽脑汁都无济于事。它藏在脑海之外,非智力所能及;它隐蔽在某件我们意想不到的物体之中(藏匿在那件物体所给予我们的感觉之中),而那件东西我们在死亡之前能否遇到,则全凭偶然,说不定我们到死都碰不到。
\par 这已经是很多很多年前的事了,除了同我上床睡觉有关的一些情节和环境外,贡布雷的其他往事对我来说早已化为乌有。可是有一年冬天,我回到家里,母亲见我冷成那样,便劝我喝点茶暖暖身子。而我平时是不喝茶的,所以我先说不喝,后来不知怎么又改变了主意。母亲着人拿来一块点心,是那种又矮又胖名叫“小玛德莱娜”的点心,看来像是用扇贝壳那样的点心模子做的。那天天色阴沉,而且第二天也不见得会晴朗,我的心情很压抑,无意中舀了一勺茶送到嘴边。起先我已掰了一块“小玛德莱娜”放进茶水准备泡软后食用。带着点心渣的那一勺茶碰到我的上腭,顿时使我混身一震,我注意到我身上发生了非同小可的变化。一种舒坦的快感传遍全身,我感到超尘脱俗,却不知出自何因。我只觉得人生一世,荣辱得失都清淡如水,背时遭劫亦无甚大碍,所谓人生短促,不过是一时幻觉;那情形好比恋爱发生的作用,它以一种可贵的精神充实了我。也许,这感觉并非来自外界,它本来就是我自己。我不再感到平庸、猥琐、凡俗。这股强烈的快感是从哪里涌出来的?我感到它同茶水和点心的滋味有关,但它又远远超出滋味,肯定同味觉的性质不一样。那么,它从何而来?又意味着什么?哪里才能领受到它?我喝第二口时感觉比第一口要淡薄,第三口比第二口更微乎其微。该到此为止了,饮茶的功效看来每况愈下。显然我所追求的真实并不在于茶水之中,而在于我的内心。茶味唤醒了我心中的真实,但并不认识它,所以只能泛泛地重复几次,而且其力道一次比一次减弱。我无法说清这种感觉究竟证明什么,但是我只求能够让它再次出现,原封不动地供我受用,使我最终彻悟。我放下茶杯,转向我的内心。只有我的心才能发现事实真相。可是如何寻找?我毫无把握,总觉得心力不逮;这颗心既是探索者,又是它应该探索的场地,而它使尽全身解数都将无济于事。探索吗?又不仅仅是探索,还得创造。这颗心灵面临着某些还不存在的东西,只有它才能使这些东西成为现实,并把它们引进光明中来。
\par 我又回过头来苦思冥想:那种陌生的情境究竟是什么?它那样令人心醉,又那样实实在在,然而却没有任何合乎逻辑的证据,只有明白无误的感受,其他感受同它相比都失去了明显的迹象。我要设法让它再现风姿,我通过思索又追忆喝第一口茶时的感觉。我又体会到同样的感觉,但没有进一步领悟它的真相。我要思想再作努力,召回逝去的感受。为了不让要捕捉的感受在折返时受到破坏,我排除了一切障碍,一切与此无关的杂念。我闭目塞听,不让自己的感官受附近声音的影响而分散注意。可是我的思想却枉费力气,毫无收获。我于是强迫它暂作我本来不许它作的松弛,逼它想点别的事情,让它在作最后一次拼搏前休养生息。尔后,我先给它腾出场地,再把第一口茶的滋味送到它的跟前。这时我感到内心深处有什么东西在颤抖,而且有所活动,像是要浮上来,好似有人从深深的海底打捞起什么东西,我不知道那是什么,只觉得它在慢慢升起;我感到它遇到阻力,我听到它浮升时一路发出汩汩的声响。
\par 不用说,在我的内心深处搏动着的,一定是形象,一定是视觉的回忆,它同味觉联系在一起,试图随味觉而来到我的面前。只是它太遥远、太模糊,我勉强才看到一点不阴不阳的反光,其中混杂着一股杂色斑驳、捉摸不定的漩涡;但是我无法分辨它的形状,我无法像询问唯一能作出解释的知情人那样,求它阐明它的同龄伙伴、亲密朋友——味觉——所表示的含义,我无法请它告诉我这一感觉同哪种特殊场合有关,与从前的哪一个时期相连。
\par 这渺茫的回忆,这由同样的瞬间的吸引力从遥遥远方来到我的内心深处,触动、震撼和撩拨起来的往昔的瞬间,最终能不能浮升到我清醒的意识的表面?我不知道。现在我什么感觉都没有了,它不再往上升,也许又沉下去了;谁知道它还会不会再从混沌的黑暗中飘浮起来?我得十次、八次地再作努力,我得俯身寻问。懦怯总是让我们知难而退,避开丰功伟业的建树,如今它又劝我半途而废,劝我喝茶时干脆只想想今天的烦恼,只想想不难消受的明天的期望。
\par 然而,回忆却突然出现了:那点心的滋味就是我在贡布雷时某一个星期天早晨吃到过的“小玛德莱娜”的滋味(因为那天我在做弥撒前没有出门),我到莱奥妮姨妈的房内去请安,她把一块“小玛德莱娜”放到不知是茶叶泡的还是椴花泡的茶水中去浸过之后送给我吃。见到那种点心,我还想不起这件往事,等我尝到味道,往事才浮上心头;也许因为那种点心我常在点心盘中见过,并没有拿来尝尝,它们的形象早已与贡布雷的日日夜夜脱离,倒是与眼下的日子更关系密切;也许因为贡布雷的往事被抛却在记忆之外太久,已经陈迹依稀,影消形散;凡形状,一旦消褪或者一旦黯然,便失去足以与意识会合的扩张能力,连扇贝形的小点心也不例外,虽然它的模样丰满肥腴、令人垂涎,虽然点心的四周还有那么规整、那么一丝不苟的绉褶。但是气味和滋味却会在形销之后长期存在,即使人亡物毁,久远的往事了无陈迹,唯独气味和滋味虽说更脆弱却更有生命力;虽说更虚幻却更经久不散,更忠贞不贰,它们仍然对依稀往事寄托着回忆、期待和希望,它们以几乎无从辨认的蛛丝马迹,坚强不屈地支撑起整座回忆的巨厦。
\par 虽然我当时并不知道——得等到以后才发现——为什么那件往事竟使我那么高兴,但是我一旦品出那点心的滋味同我的姨妈给我吃过的点心的滋味一样,她住过的那幢面临大街的灰楼便像舞台布景一样呈现在我的眼前,而且同另一幢面对花园的小楼贴在一起,那小楼是专为我的父母盖的,位于灰楼的后面(在这以前,我历历在目的只有父母的小楼);随着灰楼而来的是城里的景象,从早到晚每时每刻的情状,午饭前他们让我去玩的那个广场,我奔走过的街巷以及晴天我们散步经过的地方。就像日本人爱玩的那种游戏一样:他们抓一把起先没有明显区别的碎纸片,扔进一只盛满清水的大碗里,碎纸片着水之后便伸展开来,出现不同的轮廓,泛起不同的颜色,千姿百态,变成花,变成楼阁,变成人物,而且人物都五官可辨,须眉毕现;同样,那时我们家花园里的各色鲜花,还有斯万先生家花园里的姹紫嫣红,还有维福纳河塘里飘浮的睡莲,还有善良的村民和他们的小屋,还有教堂,还有贡布雷的一切和市镇周围的景物,全都显出形迹,并且逼真而实在,大街小巷和花园都从我的茶杯中脱颖而出。



\paragraph*{二}
\par 贡布雷,从十里开外远远望去(当我们在复活节前的最后一个星期乘火车来到这里,从铁路那头望去),所见只有教堂一座。这教堂概括了市镇的风貌,代表了市镇,并向远方的人们宣告,这里有座市镇,它在为市镇说话。然而,当你走近贡布雷,市镇看上去就像一位身披深色大氅的牧羊女迎风站立在田野中间,市镇上鳞次栉比的房屋,等于是挤挤攘攘贴在牧羊女大氅周围、拱起灰溜溜背脊的羊群。中世纪遗留下来的城墙,有些地方已经倾圮,但当年完美的弧形残迹犹存,一截截围住了城区的房舍,同古画中的城池一样。就居家而论,贡布雷不免有些凄凉,街面上的房屋都取材于当地出产的青石,门前有台阶,房上是尖尖的山墙,给门前投下一片阴影,弄得街上相当昏暗,以至太阳刚下山,家家户户的“大厅”就得拉帘掌灯。好些街道是以圣人的姓氏命名的(其中不少同贡布雷早年的几位领主的历史有关):圣伊莱尔街,圣雅克街——我姨妈的房子就在那条街上,铁栅外是圣伊尔德迦尔特街,花园的旁门开出去是圣灵街;贡布雷的这些街道在我的记忆的角落里依然存在,而且蒙上了五光十色,同我今天心目中的人间的色调大不相同,所以我实际上觉得它们色色俱全,还有那座高踞于市镇中心广场的教堂,我觉得比幻灯机的投影更虚幻,有时候我甚至认为,倘若有幸能再穿过圣伊莱尔街,到鸟儿街古风盎然的“鸟儿客栈”去租间客房,那简直比同戈洛结识、同热纳维耶夫·德·布拉邦特交谈更神妙虚幻,像是同隔世的天外来往一样。从“鸟儿客栈”的地下室的气窗里飘散出来的厨房的气味,至今我还时有所闻,依然是那样热乎乎的,一阵一阵地飘到我的鼻前。
\par 那时我们住在我外祖父的表妹——我的姑祖母——的家里,她是莱奥妮姨妈的母亲。自从奥克达夫姨夫去世之后,莱奥妮姨妈从此不肯离开贡布雷,不肯离开贡布雷的那幢房屋,不肯离开她的房间,她的床。她不肯“下来”了,总那么躺着,那么凄凄切切,有气无力,病病恹恹,老想不开。她那个套间的窗外是圣雅克街,这条街到头是“大草坪”(同市中心三条街交叉的街心绿化地带“小草坪”遥遥相对)。街面灰溜溜的,单调划一,几乎家家门口都有砂岩砌成的三级高台阶,整条街像是由哥特石刻匠人在原块石头上凿出来的一道深沟,本来打算在上面刻耶稣降生的马槽或者耶稣受难的坟场的,我的姨妈实际上只占用两间相通的房间,她每天下午呆在其中的一间,好让用人给另一间通风。那是乡绅家常见的那种房间。世界上有些地方,大气中或海面上游动着亿万种肉眼看不到的原生动物,它们在闪光、在散发出芳香。那两间房内也一样,也有千百种气味令人心醉,那是从品德、智慧和习惯中散发出来的芳香,氤氲中悬凝着一个人内心深处隐而不露、丰富至极的全部精神生活;当然,也还有例如从附近田野里传来的那些自然气息和时令色彩,但是它们一到这里便失去了野趣,变得人情味十足,而且凝滞闭塞,跟用当年从果园里摘下之后便藏进柜子的水果制成的果汁冻那样香甜而透明;它们固然也随季节的更迭而变换,毕竟具有了柜藏的风味和家用的格局,新鲜面包的温馨消融了白色冰霜的凛冽,就像村里报时的大钟,悠闲而准时,散淡而有序,既漫不经心又高瞻远瞩。洁净的床单,清新的晨意,虔诚的气氛,和谐地融合在一片宁静之中,不过这种宁静,只给人增添愁绪罢了,倒为并非身临其境、仅是匆匆过客的人提供了汲取无尽诗意的宝库。这里的空气如此幽闭,好似一朵纤细娇美的花,沉寂中饱含营养,而且香甜诱人,使我一踏进门槛便油然而起馋涎欲滴的感觉,尤其是在复活节那个星期的开头几天,那时早晨还寒意料峭,当时我刚来贡布雷不久。我去姨妈那边请安,她们先让我在外间稍候。乍暖还寒时节的阳光,扑到炉火前来取暖,两砖之间的柴禾已经蹿起耀眼的火苗,给整间屋子抹上一股油烟的气味,弄得像农舍大火炉前的一面火墙,又像宫堡华屋的壁炉上的大炉罩。呆在那样暖和的地方,但愿外面雨雪交加、洪水横溢才好,这样也可给深居的舒适更增添冬蛰的诗情。我在供桌和交椅之间走动着。那些交椅蒙着毡绒面子,靠背上方总安着方括弧形的头靠,熊熊的炉火,像发酵的面团,散发出令人垂涎的芳香,空气也随之布满气泡;清晨湿润而明媚的朝气早已催发出这一层层的芳香,而且把它们一片片翻动,把它们烤黄,给它们打上绉褶,使它们松软膨胀,从而做成一大块虽无形迹却香甜可感的乡村糕点,简直像一大张“脆皮夹心饼”。这里的壁橱、柜子,还有画着枝叶图案的壁纸,发出比点心更香脆、更细腻、更有名、更干燥的异香,我回到房里,总不免怀着难以启齿的艳羡,沉溺在花布床罩中间那股甜腻腻的、乏味的、难以消受的、烂水果一般的气味之中。
\par 我听到姨妈在里面房内低声地自言自语。她说起话来总是轻声细语,因为她认为自己头脑里有什么东西已经破碎,在里面飘浮着,她若大声说话,那东西就会移动,但是她又忍不住长久的沉默,即使身边没有人在场她也得自言自语,因为她相信这对肺部有益,能防止血液停滞,对于她常犯的胸闷气憋也有缓解的功效。她整天有气无力地苟延残喘,每一点小小的感觉都看得非同小可,她使这些感觉具有活动不定的机能,所以更难以憋在心里。由于没有知己可以对之倾诉,她只好自言自语,于是滔滔不绝的独白成为她唯一的活动方式。不幸,想什么就说什么的习惯一旦形成,她也就顾不得隔墙有耳了,所以我常听她自言自语说:“我准是没有记错,又是一夜没睡。”(因为她的大言不惭莫过于自称日夜不睡,我们全家上下言谈中也都始终尊重她的这种说法,不露半点马脚。例如,早晨弗朗索瓦丝不是去“叫醒她”,而是到她的“屋里去”;当我的姨妈想在白天打个瞌睡,我们就说她要“思考思考”,或者说她想“闭目养神”;她一旦自己说漏嘴,忘乎所以地说“什么什么把我惊醒了”或者“我梦见什么什么”之类,话一出口她自己先就羞红了脸,接着便很快恢复常态。)
\par 我在外间稍候片刻之后,进去向她请安;弗朗索瓦丝正给她沏茶。倘若我的姨妈那时感到心绪不宁,她就吩咐以药代茶。遇到这种情况,总由我负责从药袋里把一定量的椴花茶倒进一只小碟,然后倾入开水。干燥的花梗变得弯弯曲曲,梗梗相勾地组成荒诞不经的图案,其中绽出一朵朵苍白的小花,像是由哪位画家按照最完美的装饰意图有心点缀上去的。失去了本色或者改变了原貌的叶片变成了一堆七零八落的碎片,有的像飞虫透明的翅翼,有的像一枚标签的白色的反面,有的像一瓣玫瑰,跟鸟儿叼来筑巢的材料一样,聚集到一起,编织成片。无数琐碎的细枝末节,倘若马虎应付,本来都可能忽略掉的,只是药剂师不惮麻烦才作了这样精细的炮制,但这些细枝末节却给我喜出望外的愉快,等于在一本书中惊喜地发现某位熟人的大名,我从这些细枝末节中认出它们原本是地地道道的椴花叶梗,与我在车站大街的椴树枝上所见略同;外表有所不同,恰恰是因为它们不是赝品,而是地道的真货,只是它们已经老化。每一种新的品格都只是老品格的变态,所以我在一团团小小的灰色泡沫中辨认出枝头初绽的绿芽;尤其是那片圆月形的嫣红宜人的反光,把细梗丛中的小花一朵朵衬托得好似挂在枝头的金色的玫瑰,等于投射在墙面上的一丝微光,让人约摸看出哪个部位曾经有过一幅壁画;这反光也成为一种标记,标明椴树上哪个部位曾经“彩色斑斓”,哪个部位本来就没有色泽,同时它还向我证明,这些花瓣在点缀药袋以前曾经为春日的黄昏散布过醉人的芳香。这嫣红的烛光仍留有它们昔日的颜色,只是已经半明半灭,在残烛上昏昏摇曳,好比花儿欲谢,时近黄昏。片刻之后,姨妈会在她品尝残花枯叶香味的那杯热茶中,泡一块“小玛德莱娜”,待点心泡软以后,就送我尝一口。
\par 她的床这一面有一个柠檬木的黄色立柜和一张既当药案又当供桌的桌子,上面是一尊圣母像和一瓶维希圣泉水,下面放了几本祷文和一些药方,祈祷和服药所需的一切都齐全了,不至于耽误早上服药和黄昏祈祷。床的那一面贴近窗户,街景尽收眼底。她从早到晚就像波斯王公披阅史册那样地研读贡布雷街头的日常要事,说它日常,其实风味之古老胜似远古史册;尔后,她同弗朗索瓦丝一起对见闻进行评述。
\par 我到姨妈那里不出五分钟就被她打发走了,她怕我太耗费她的精神。她把苍白淡漠的前额凑到我的唇边。在早晨那个时候,她额前的假发还没有梳理,脊骨像荆冠上的芒刺鼓出睡衣,又像一串诵经用的念珠。她对我说:“可怜的孩子,你走吧,快去准备做弥撒;你要是在楼下遇到弗朗索瓦丝,就叫她别在下面光贪玩,早点上楼来看看我有什么需要她照料的。”
\par 照料她多年的弗朗索瓦丝那时已经想到自己早晚有一天要专门侍候我们,所以我们住在那里的几个月当中,她确实对我姨妈不甚尽心。我小时候在来到贡布雷前,莱奥妮姨妈还年年到巴黎她母亲家过冬,那时我跟弗朗索瓦丝很生疏;有一年正月初一,母亲领我去姑祖母家拜年,进门前妈妈给我一张五法郎的钞票,嘱咐说:“千万别给错了,你听我说过‘你好,弗朗索瓦丝’之后,再把钱给她;到时候我会轻轻捅你一下胳膊的。”我们一走进姨妈家的过厅,便影影绰绰瞅见一顶白得耀眼、挺括纤薄得像糖丝织成的便帽下面堆着一副预表感激的笑容。
\par 那就是弗朗索瓦丝;只见她像神龛里的圣徒塑像似的,一动不动地站在门框里。待我们适应了门厅的幽暗之后,才分辨出她的表情中含有与人为善的无私的爱,以及发自肺腑的对上等人的尊敬,而能得到新年礼物的希望更在她内心最美好的部位激发出这样的敬爱之情。妈妈使劲地拧了一下我的手臂,大声说道:“你好,弗朗索瓦丝。”听到这一信号,我赶紧松开手指,让钞票落到虽说半推半就却已经伸了过来的那只手的掌心。但是,自从我们住到贡布雷之后,弗朗索瓦丝成了我最熟悉的人。她最乐于侍候我们,至少在开头那几年,她侍候我们像侍候我姨妈那样地尽心尽力,实际上她对我们更加巴结,因为我们除了同她的主人是一家人之外,还具备另一种魅力:她尊重无形中连结家庭成员的血缘关系,尊重的程度不亚于古希腊的悲剧诗人,况且我们不是她惯常侍候的主人。我们到达贡布雷的那天,她迎接我们时有多高兴!我们是复活节之前到达的。她埋怨天气还不转暖,害得我们一路挨冻;那时节倒确实寒风砭骨。我的妈妈问她的女儿可好?侄儿外甥们是否安康?还问到她的外孙乖不乖?她打算把他培养成什么人?小外孙长得像不像外祖母?
\par 等大伙儿走开之后,妈妈还同她谈起她的父母,打听他们在世时的生活细节,因为妈妈知道弗朗索瓦丝在父母去世之后,好多年中都还伤心落泪。
\par 妈妈早就看出来了:弗朗索瓦丝不喜欢女婿,因为他破坏了她同女儿相依为命的乐趣,只要女婿在场,她就无法同女儿畅叙家常。所以,每当弗朗索瓦丝到距离贡布雷几里以外的地方去看望女儿,妈妈总要笑呵呵地对她说:“弗朗索瓦丝,今天倘若赶上朱利安有事出门,你就只好同玛格丽特单独过这一整天了,不用说你会感到遗憾的,不过你总能将就,是不是?”听到这话,弗朗索瓦丝就哈哈笑道:“夫人,您什么事都看得一清二楚;您的眼光比给奥克达夫夫人查病的爱克斯光还要厉害(爱克斯光这几个字,她故意说得佶屈聱牙,而且莞尔一笑,像是自我解嘲,笑自己无知至此,居然也搬弄科学名词儿),人家肚皮里有什么东西,您一看就透。”说罢,她就躲开了,仿佛对人家的关心感到过意不去,也可能是为了躲到一边去免得人家看到她抹眼泪。在妈妈之前,还从没有人使她产生过这样暖人心怀的激动,她头一回感到自己的生活,自己的幸福,自己的痛苦,除她自己这样一个苦老太婆之外,还能有别人关心,还能成为另一位妇女悲喜的缘由。
\par 我们住在贡布雷的那些日子里,我的姨妈也只好牺牲掉一些同弗朗索瓦丝做伴的时间,因为她知道我的母亲对这位聪明勤快的女用人有多器重。打从清早五点起,弗朗索瓦丝就拾掇得干净利索地下厨干活了,她那顶软帽上的褶裥,一条条挺括漂亮,像刚出炉的瓷胎;她打扮得跟去教堂做大弥撒似的。她干什么都在行,像马一样吃苦耐劳,无论身体好坏,总是闷头干活,而且轻手轻脚,跟没有干活一样。倘若妈妈要杯热水或者要点咖啡,在姨妈的女用人当中只有她才会端来滚烫的开水或者热咖啡。她是那样一类的用人,既让生客一见就讨厌(也许因为他们心中有数,知道他们对眼前的客人一无所求,主人宁可客人不上门也不会把他们辞退,所以他们犯不着巴结客人,对客人不免怠慢),又得到主人分外的宠信,因为主人考验过他们的实际能力,表面的讨好和低眉顺眼的絮叨固然能给客人留下良好的印象,却往往掩盖无法调教的低能,故而主人反倒并不在乎。
\par 弗朗索瓦丝先把我的外祖父母和父母侍候安顿好,然后才上楼侍候我的姨妈服用蛋白酶,同时问她午饭要吃什么。她一到楼上,就不易避开某些问题,得发表见解或作出解释了。
\par “弗朗索瓦丝,你倒想想看,古比尔夫人居然比平时晚了一刻钟来找她的姐姐;她要是在路上再多磨蹭一会儿,恐怕要在弥撒开始之后才能赶到教堂了。”
\par “咳,敢情!”弗朗索瓦丝答道。
\par “弗朗索瓦丝,你要是早来五分钟,你就能看到安贝夫人了,她手里的那捆芦笋比加洛大娘菜摊上的要粗上两倍。你想法子向她的女用人打听打听,她是从哪儿弄来的?今年你做什么配菜都少不了放芦笋,你很可以为咱们家的那几位旅行家也弄点这么粗的芦笋来嘛。”
\par “没有什么奇怪的,那是从神甫先生的园子里弄来的。”弗朗索瓦丝说。
\par “哈!你真能哄人,可怜的弗朗索瓦丝,”我的姨妈耸耸肩膀接口道,“从神甫先生的园子里弄来的!你明明知道他那儿的芦笋长得又小又坏。告诉你吧,她手里的芦笋,足足有胳膊那么粗呢。当然,不是你的胳膊,而是像我的这条今年又瘦了许多的胳膊。弗朗索瓦丝,你没有听到这嗡嗡的钟声吗?闹得我脑袋都要炸了!”
\par “没有,奥克达夫夫人。”
\par “啊!可怜的孩子,足见你的脑袋真结实,这是托上帝的福。刚才拉马格洛娜找比普罗大夫来了。大夫紧跟着就同她一起走了,他们是在鸟儿街那边拐弯的,准是哪家孩子病了。”
\par “哎哟!我的上帝,”弗朗索瓦丝叹息道。她听不得有谁遭难,即使在天涯海角有一位她压根儿不认识的人遇到不幸的消息传到她的耳里,她也总要连连叹息。
\par “弗朗索瓦丝,这丧钟究竟是为谁在敲呀?啊,我的上帝,该是为卢梭夫人敲丧钟了。瞧我,怎么居然忘了:她在那天夜里就过世了。啊!我也快了,善良的上帝该把我召回去了,自从我可怜的奥克达夫归天之后,我这脑袋就不知道是怎么弄的,害得你白白为我耗费许多光阴,我的孩子!”
\par “不,奥克达夫夫人,我的光阴没有那么金贵。时间本是上帝白给的,又没有要咱们破费。我现在得去看看火灭了没有。”
\par 弗朗索瓦丝和我的姨妈就这样对当天发生的第一批事件,在上午联合评述了一场。但是有时候发生的事件具有相当神秘、相当严重的性质,我的姨妈感到不能坐等弗朗索瓦丝上楼之后再论短长,于是整幢房子里响起四下震耳的铃声。
\par “可是,奥克达夫夫人,现在还不到服用蛋白酶的钟点呀,”赶上楼来的弗朗索瓦丝说道,“莫不是您感到有些乏力,顶不住吗?”
\par “不是的,弗朗索瓦丝,”姨妈说,“要说乏力,你是知道的,如今我已难得有什么时候不感到衰竭的了;我早晚有那么一天跟卢梭夫人一样,自己还没有明白过来就咽气了。我倒不是为了这个才打铃叫你的。你没有料到吧?我刚才看得一清二楚,就跟现在看到你一样,我看到古比尔夫人领着一个女孩子走过去,那个女孩子我居然压根儿不认识!你赶紧到加米杂货铺去买两个苏\footnote{法国货币单位,二十苏相当于一法郎。}的盐,戴奥多尔不至于不告诉你她是谁家的孩子。”
\par “准是比班先生的女儿,”弗朗索瓦丝更愿意当场作出解释,因为她今天上午已经到加米杂货铺去过两次了。
\par “比班先生的女儿!哦!你真能哄人,可怜的弗朗索瓦丝!照你说,我还能认不出她来吗?”
\par “我没说是他的大女儿,奥克达夫夫人,我说是他的小女儿,那个在儒伊寄读的小丫头。我好像早晨就见到过她。”
\par “啊!除非像你说的,”姨妈说,“那她准是来过节的。没错!不用再打听了,她准是来过节的,这么说来,咱们呆会儿准能见到萨士拉夫人来敲她妹妹家的门,吃午饭嘛!没错!我刚才看到加洛班点心铺的小伙计提了一盒果馅大饼走过。你瞧着吧,这饼准是送到古比尔夫人家去的。”
\par “古比尔夫人家只要一来客人,奥克达夫夫人,您就等着瞧吧,她的那一帮人不久都会赶来吃午饭的,现在已经不早了。”弗朗索瓦丝说罢急于下楼张罗午饭,心安理得地抛下我的姨妈独自观景消遣。
\par “哪里!中午以前不会来。”我的姨妈无可奈何地接口道,说着,她担心地看一眼座钟,但只是偷偷的一瞥,免得让人发现万事不管的她,居然对古比尔夫人要请谁来吃饭,有如此高的雅兴打听,可恨的是这种兴致可能还得有劳她干等个把钟头。“偏偏又要赶到我吃午饭的时候才来!”她自言自语地咕哝道。吃午饭对于她来说是种相当称心的消遣,她不希望有别的事情打扰,“你千万别忘了:把我的奶油鸡蛋放在一只平底盘里。”只有平底盘上才画有人物,我的姨妈每顿饭都要看着解闷。她戴上眼镜,辨认当天盘子上的人物故事:阿里巴巴和四十大盗,阿拉丁和神灯。她一面看,一面微笑着说:“很好,很好。”
\par “我倒可以上加米杂货铺去一趟,探探消息……”弗朗索瓦丝看出我的姨妈不再打发她去杂货铺,便这样说道。
\par “不,不必了,那准是比班小姐。我的可怜的弗朗索瓦丝,很对不起,为了这么一件小事我让你上来一趟。”
\par 然而我的姨妈心里很明白:她打铃让弗朗索瓦丝上楼,决不是为一桩小事,因为在贡布雷,一个不为人知的人简直跟神话里的神仙一样不可思议。事实上,过去每当圣灵街或者中心广场骇人听闻地出现这类人物,总会有人进行细致的调查,结果没有一次不把这类神奇人物最终纳入“熟人”之列,或者把他的为人摸得一清二楚,或者把他的身份弄清个大概,总跟贡布雷的什么人沾点亲吧。这位是索东太太的儿子,服兵役期满之后复员归来;那位是贝德罗神甫的侄女,是从修道院里出来的;还有本堂神甫的兄弟,在夏多丹当税务官,新近才退休,来这里过节。起先有人见到他们,以为贡布雷竟然出现大家不认识的人,不免心里惶惶不安,原来无非是没有一下认出来,或者没有一下弄清他们的身份罢了。其实索东太太也好,本堂神甫也好,都早就有言在先,说他们正盼望出远门的亲人回来呢。晚上,我散步回家,上楼去跟我的姨妈说说散步时的见闻,倘若我不慎说起我们在老桥附近遇到了一位外祖父不认识的人,姨妈必定失声叫道:“居然连你外祖父都不认识!啊!我才不信哪!”话虽这么说,她毕竟有点按捺不住,非要弄个水落石出不可,于是盘问外祖父:“姨父,你们在老桥附近究竟碰到谁了?连您都不认识?”——“怎么不认识,”我外祖父回答说,“那是普罗斯贝,就是布耶伯夫人家园丁的弟弟。”——“噢,他呀!”姨妈总算放心了,脸还有点红;她耸了耸肩膀,苦笑一声,补充说道:“因为他方才说你们遇到了一位您不认识的人!”所以家里的人叮嘱我以后说话千万谨慎,切不可不假思索地乱讲,惹得姨妈那样激动。贡布雷无论家畜还是居民,彼此都认识,所以倘若姨妈偶尔发现有一条她不认识的狗走过,她就必定不住地搜索枯肠,把她的推理才能和悠闲的时间全都消耗在这件难以理解的事情上去。
\par “那准是萨士拉夫人的狗。”弗朗索瓦丝说道,其实她并没有十分把握,目的只在于使姨妈安心,免得她“耗费精神”。
\par “好像我连萨士拉夫人的狗都不认得了!”姨妈接口道,她的批判精神轻易不接受靠不住的说法。
\par “啊,是了,准是加洛班先生新近从里瑟欧带回来的那条狗。”
\par “啊!除非是那条狗。”
\par “据说,它可乖巧了,”弗朗索瓦丝补充说,这情报她是从戴奥多尔那里得来的,“它跟人一样机灵,总是摇头摆尾,总那么讨人喜欢,有那么一股热乎劲儿。要说牲口啊,才这么小就知道讨好,实在难得。奥克达夫夫人,我得走了,我可没有时间闲聊,这不,眼看就十点钟了,我不光是炉子没有升旺,还有一堆芦笋要削呢。”
\par “什么!弗朗索瓦丝,又是芦笋!你今年真得了芦笋病了,早晚让咱们家的那几位巴黎人吃倒胃口!”
\par “才不会呢,奥克达夫夫人,他们可爱吃哩。等他们从教堂做完弥撒回来,一定胃口大开,您瞧着吧,他们保管吃得津津有味。”
\par “这会儿,他们一定已经在教堂里了;你最好别耽误工夫,赶紧张罗午饭去吧。”
\par 正当我姨妈同弗朗索瓦丝这么东一句西一句闲扯的时候,我同外祖父母和父母一起在教堂做弥撒。我多么喜欢那座教堂呀,如今想起来犹历历在目!我们进教堂时必经的古老门楼,黑石上布满了坑坑点点,边角线已经走样,被磨得凹进去一大块(门楼里面的圣水池也一样),看来进教堂的农民身上披的粗呢斗篷,以及他们小心翼翼从圣水池里撩水的手指,一次次在石头上轻轻擦过,年复一年地经过几个世纪,最终形成一股无坚不摧的力量,连顽石都经受不住,给蹭出了一道道深沟,好比天天挨车轮磕撞的界石桩子,上面总留有车轮的痕迹。教堂里掩埋着贡布雷历代神甫高贵尸骨的墓石,像是为祭殿铺下的地板,更增添了萦绕遐迩的灵气;可如今这片片墓石已失去死寂坚硬的质地,因为岁月已使它们变得酥软,而且像蜂蜜那样地溢出原先棱角分明的界限,这儿,冒出一股黄水,卷走了一个哥特式的花体大写字母,淹没了石板上惨淡的紫堇;而在别处,墓石又被紫堇覆盖得不见天日,椭圆形的拉丁铭文更显得缩成一团,使那几个缩写字母平添一层乖张的意味,同一个字里有两个字母挨得特别近,而其他的字母却被大大地拓开了距离。教堂里的彩绘玻璃窗,只要外面稍有阳光,便能闪耀光彩,所以尽管外面天色阴沉,教堂里却总是光辉灿烂;有一面彩绘玻璃窗,从上到下只被一个人物形象所占满,那人的模样跟纸牌上的大王相似;他就在上面顶天立地站着,教堂的拱顶成了他的华盖。教堂里平常不做功德法事时,中午时分,他便笼罩在斜照的蓝色的反光中(那样的日子难得遇到,教堂里空空荡荡,空气清新,阳光照在瑰丽的陈设上,显得更加堂皇,也更有人情味,再加上石雕和彩色玻璃,这里简直变得像一家中世纪风格的旅馆的接待厅,几乎具有供人歇宿的意味)。那时你能看到萨士拉夫人跪在那里咕哝几句祷文,她旁边的祈祷桌上放着一包捆扎好的点心,那是她刚从对面的糕点铺买的,准备拿回家去当午饭。另一面彩绘玻璃窗上是一座粉红色的雪山,山下是打仗的场面;它好像是雪山喷出的凌乱的雪珠直接打到玻璃上凝结而成的霜冻,又像玻璃窗上残留的雪花,只是这片片雪花被一道霞光抹上了一层红晕(无疑,就是这道霞光,把祭台的彩屏照得格外绚丽,好似这上面的五光十色,不是早就涂在石料上的颜色,倒像由外面射来的一道随时准备放出异彩的光芒当场抹上去似的)。每一面彩色大窗全都历史悠久,处处显得生意盎然,数百年的积尘银光闪闪;这一面面由彩色玻璃交织而成的亮晶晶的大挂毯,已被岁月磨蚀得经纬毕露。其中有一面窗像长条的棋盘,由百十来块长方形的小玻璃拼成,主调是蓝色的,像当年供查理六世用来解闷的一副大纸牌;但是,也许因为有一道光芒倏然闪过,也许因为我的转动的目光透过那面忽明忽暗的彩色长窗,看到了一团跃跃蹿动、瑰丽无比的烈火,顷刻间那面彩色长窗忽然迸射出孔雀尾羽那样变化多端的幽光,接着它颤颤悠悠地波动起来,形成一丝丝亮晶晶的奇幻的细雨,从岩洞般昏暗的拱顶,淅淅沥沥地沿着潮湿的岩壁滴下。我随着手执经卷的长辈往前走,仿佛走进了五光十色的岩洞,四周是诡异的钟乳石,多彩多姿;刹那间那一片片菱形的小玻璃显得清澈透明,像镶嵌在一枚硕大无朋的胸章上的蓝宝石那样坚硬,然而你又明明可以感到,在它们的后面,还有一件更令人钦慕的东西,那就是偶尔一露的阳光的微笑。在这片沐着宝石般湛蓝柔和的光波中,它是那样清晰可辨,跟广场石板上或集市草堆中的阳光一样。在复活节前我们到达贡布雷的最初几个星期天,虽然大地仍是光秃秃的、黑黝黝的,但阳光的微笑却给了我们安慰,它在这里,像历史上圣路易的子孙们遇到过的那个载入史册的春天一样,使装点着忘我草的那面金碧辉煌的大彩窗放射出灿烂的光芒。
\par 两幅立经挂毯描绘爱丝苔尔\footnote{爱丝苔尔:《圣经》中的人物。传说她是犹太人的孤女,被波斯王阿絮埃吕斯选入宫中,得宠,立为王后。奸臣哈曼怂恿波斯王杀尽境内的犹太人,爱丝苔尔施计揭露哈曼的阴谋,终使犹太种族免于灭绝。这个故事详见《圣经》中的《爱丝苔尔书》。}受冕的场面(根据传统,阿絮埃吕斯王的相貌被描绘得像一位法国国王,而爱丝苔尔的形象则同国王所宠爱的盖尔芒特家的某位贵夫人相似),挂毯上的颜色已褪得模糊不清,倒给画面增添一种表现力,一种立体感,一种亮度:爱丝苔尔唇上的淡红色越出了嘴唇的轮廓线;她的连衣裙上的黄色,显得那么滑腻,那么厚实,仿佛已板结成块,吹来一股气流就能把它整块掀掉似的。在这幅丝线和羊毛交织成的挂毯的下半部,树木还绿得那样鲜艳,可是上半部已经“年久色衰”,因而深色树干上发黄的高枝,苍白得十分显眼,好像有一道无形的阳光,以强烈的斜照,把它们晒黄,晒褪了它们一半的颜色。这一切,尤其是教堂里那些珍贵的文物,原先是由历史上的名人传下来的,他们在我的心目中几乎成了传奇人物(那个精雕细刻的金十字架,据说是圣埃罗瓦\footnote{圣埃罗瓦(约558—660):著名金器匠人,创建索里尼亚克修道院,后被奉为金银匠和铁匠的守护神。}的杰作,由达戈贝\footnote{达戈贝(七世纪初—639):法国国王(629—639)。}敕赐教堂的,还有日耳曼路易\footnote{日耳曼路易(804—876):东法兰克国王(817—843)和日耳曼国王(843—876)。}的王子们的合葬墓,墓身由斑石砌成,上面镶着金丝彩釉的青铜雕刻),正因为有这些东西,我们在教堂就座之后,我才有如临奇境之感,就像乡下人走进神仙到过的山谷,能在一块岩石上,一棵树身上,一片水塘中,惊喜地发现神仙经过的明显的痕迹。凡此种种,都使这座教堂在我的心目中与城里的其他地方完全有别:这座建筑可以说占据了四维空间——第四维就是时间,它像一艘船扬帆在世纪的长河中航行,驶过一柱又一柱,一厅又一厅,它所赢得、所超越的似乎不仅仅是多少公尺,而是一个朝代又一个朝代,它是胜利者。它把严酷粗野的十一世纪,隐匿在厚实的墙壁中,沉重的拱梁下填满了大块碎石,把风洞堵得严严密密,只有门廊附近登上钟楼的楼梯才在墙上破开一条深深的槽口,露出一点往昔的遗迹。但是,即使在那里,也有重重叠叠哥特式的、风姿绰约的拱门,一个挨着一个地挡着,让外人一眼看不到楼梯,好比一群千娇百媚的大姐姐,笑吟吟地挡住了身后土里土气、哭哭啼啼、衣衫寒酸的小弟弟。教堂的塔楼,直刺青天,高高地屹立在广场之上;它当年曾静观过圣路易的英姿,今天似乎仍看得到他的风采。教堂的地下室深深地陷入中世纪的黑夜中;戴奥多尔和他的姐姐摸索着把我们领到幽暗的拱顶下,天花板上鼓出一道道粗壮的筋脉,像一只巨大的蝙蝠张开的翼膜。两位领路人用一支蜡烛给我们照亮了西格贝王\footnote{西格贝(?—509):莱茵河下游普利安法兰克人的国王,公元496年前后,在今科隆一带曾击败日耳曼族中骁勇善战的阿拉芒人。509年为其子所杀。}的小公主的坟墓,坟墓中央有一个深坑——像墓穴的遗迹——据传那是由一盏水晶灯落下时砸出来的:“法兰克公主被杀的当夜,原来由金链吊在现在后殿那个地方的一盏水晶灯忽然脱钩落下,灯罩没有破碎,火焰也没有熄灭,只是砸进了石头,灯的分量居然使顽石塌陷。”
\par 贡布雷教堂的后殿,能正经地提到它吗?它那么粗糙,毫无艺术可言,甚至没有半点宗教情调。从外面看,由于它对着的那个十字路口在下坡,它的外墙底下垫了一层乱石砌成的墙基,石头东一块西一块地凸出在外,毫无教堂的特色。窗户好像开得很高很高,总的看起来,不大像教堂,倒像监狱。不用说,后来当我想到我生平所见到过的其他教堂的富丽堂皇的后殿,我从来没有想到把它们同贡布雷教堂的后殿进行比较。只是有一回,我在内地的一条小胡同的拐角处,发现三条胡同的交叉口,有一面粗糙的高墙,上面的窗户也开得很高,跟贡布雷教堂后殿的那面墙的外观一样不成比例。那时,我没有像在参观夏特勒大教堂或者兰姆大教堂时那样细细探究宗教感情在那些建筑物中怎样有力地得到了体现,我只是情不自禁地叫了声:“教堂!”
\par 教堂!它同住宅紧挨紧连;在圣伊莱尔街,它的北门介于两家紧邻之间:一边是拉班先生的药房,一边是卢瓦索夫人的住宅。它同这两家墙挨墙,没有丝毫距离,它就像贡布雷的普通居民之家,如果贡布雷的街上编有门牌号码的话,它也可以有个门牌号码;邮差早晨送信的时候,在走出拉班先生的药房,还未走进卢瓦索夫人的住宅之前,似乎本应该在它的门口停一停的;然而在教堂和非教堂之间,却有一道我的思想始终不能逾越的界线。尽管卢瓦索夫人的窗前有几棵倒挂金钟,习惯于不知趣地纵容耷拉着脑袋的枝叶到处乱蹿,那上面的花朵开到一定时候,总迫不及待地要把自己的红得发紫的面孔贴到教堂阴沉的墙上去凉快凉快,我觉得倒挂金钟并不因此而沾上灵气;在花朵和它们所投靠的阴沉的墙面之间,我的肉眼虽看不到有半点间隙,但是在我的心目中,却存在着一个不可逾越的深渊。


\subparagraph*{1}


\par 圣伊莱尔街的钟楼,老远就能看到;在贡布雷市容还没有出现的远方,它那令人难忘的面貌就已经露出地平线了。复活节的那个星期,当火车把我们从巴黎送到这里的时候,我的父亲看见它轮番地驰过地平线上的每一层折痕,钟楼上的风信鸽朝东南西北四方转动。父亲说:“好,把毯子都收起来,咱们到了。”有一次,我们到离贡布雷很远的地方散步,有一段道路很狭窄,旋而豁然开朗,眼前出现一大片四周被枝柯参差的森林团团围住的平地,只见圣伊莱尔街钟楼细巧的塔尖,冒出在树梢之上;它呈淡红色,显得那样宜人,那样苗条,亭亭玉立在天边,仿佛有谁故意在这幅尽是天然景物的图画的天空部位,用指甲抠出一道艺术的记号,作为表明有人居住的唯一标志。再靠近些,就能看到四方形塔楼的残迹了。半圮的塔楼仍簇拥钟楼而立,只是比它要矮些;塔身石块上的暗红的色调,尤其令人惊叹。在秋雾凄迷的早晨,那情状宛如一派彤云叆叇的葡萄园上兀立着一堆攀满红色爬山虎的废墟。
\par 我们回家的时候,外祖母常常让我在广场上滞留片刻,好看看教堂的钟楼。塔楼上的窗户两个一组,分层排列,间距规整而独具一格,人的五官若具有这种比例才显得端庄而美丽。从楼上,每隔一阵飞出一群暮鸦;它们呱呱地转圈翩跹,好似原先听凭它们扑腾腾栖落的古塔,忽然变得难以安身,仿佛隙缝间释放出某种动荡不停的元素,把它们从塔里轰了出来。待它们把暮霭苍茫的淡紫色帷幕到处划遍之后,又突然安静下来,钻回塔里去栖息;充满凶兆的塔楼重新变成安居的福地。有几只乌鸦散歇在小钟楼的塔尖,看上去一动不动,说不定它们正盯住一只小虫,准备下喙,就像稳坐钓鱼台的渔夫准备抬竿,停歇在浪尖的海鸥准备啄鱼似的。不知为什么,我的外祖母觉得圣伊莱尔钟楼没有一丝一毫庸俗、浮夸和鄙吝之气,因为她喜爱自然景物和天才的作品,并认为唯有自然和天才之作才富于有益的影响;至于自然景物,当然不可假手人工,比如我的姑祖母的园子经园丁一弄,自然反而受到糟踏。这教堂无论从哪方面看,都显得从本质上就与别的建筑不同,而真正意识到它别具一格,确定它的存在具有个性、敢于独树一帜的则是它的钟楼。为教堂立言的,也是这座钟楼。我尤其相信,我的外祖母在贡布雷钟楼的身上,模糊地见到了她心目中最可贵的东西,那就是既自然又不凡的气派。她对建筑学一窍不通,但她说:“孩子们,你们尽管可以笑我,也许从规范上说,这座钟楼并不美,但是它老态龙钟的怪样,我看了很受用。我甚至相信,倘若它会弹钢琴的话,一定不会弹得干巴无味的。”她望着塔身,眼睛顺着砖石的坡度,顺着塔身优雅的张力向上望去,只见斜线越往上越靠近,就像合十祈祷的双手;我的心似乎同箭一样地向上飞去,她的目光也随着塔身跃然上升;她对已经风化的古老的石塔露出友好的微笑,当时仅仅在塔尖还残留着些许夕阳。自从塔身进入这一光照区之后,每一片石头便被阳光照得轻飘飘起来,仿佛突然间显得又高又远,像一首歌用提高八度的尖音来演唱一样。
\par 是圣伊莱尔钟楼,使城里的各行各业、每时每刻和各种观点,都具有形式、取得结果和得到认可。从我的房间望去,我只能见到它外铺石板的塔基;但是,在炎热的夏季的某个星期天早晨,我一看到那些石板像一团黑色的太阳在烨烨放光,我就会想:“天哪!九点钟了!如果我想要在去教堂做弥撒之前还有时间向姨妈请安的话,那现在就得做准备了。”因为我确切地知道太阳照临广场时是什么颜色,我感觉得到外面的气温和市场上的尘埃,感觉得到妈妈在做弥撒前会去买东西的那家店铺门前的遮篷的投影。店堂里有一股未经漂白的本色布的气味,妈妈也许去买块手绢之类的东西,店掌柜会绷直了身子吩咐伙计拿出货来给妈妈挑选,他自己则准备关店门,而且早已到后面去穿好了节日的上衣和洗净了双手。他有每隔五分钟就搓一次手的习惯,即使遇到最不痛快的场合,他也要踌躇满志地、精明强干地搓他的那双手。
\par 做完弥撒,我们走进店堂,吩咐戴奥多尔给我们一份比平时要大的奶油圆面包,因为我们的表亲趁着好天气从梯贝齐赶来同我们一起吃午饭。那时我们眼前的钟楼周身披着灿烂的阳光,金光闪闪、焦黄诱人,简直像一块硕大无朋的节日奶油面包,它的塔尖直戳蓝色的天空。黄昏时,当我散步归来,想到呆会儿我得向母亲道晚安,而且将一整夜见不到她,这时钟楼反倒因为白日已尽而显得格外温柔,它倚着苍白的天空,像靠在深褐色的丝绒坐垫上似的,天空在它的压力下微微塌陷,仿佛为它腾出地方安息,并且裹住了它的四周;围着塔身飞翔的鸟类的叫声更衬托出它的寂静,更拔高了它的尖顶,使它具有某种难以言传的意味。
\par 即使我们走到教堂后面某条已经看不到教堂的街上,那里房舍的布局似乎也是由钟楼在哪里出现而定的;也许它出现在看不到教堂的地方才更显得惊心动魄。当然,另有不少钟楼在这类景观中比它壮丽,我的脑海里就有好几幅钟楼屹立在鳞次栉比的屋顶之上的图景,但它们同贡布雷阴沉街景中出现的那座钟楼相比,艺术上各有异趣。我永远也忘不了巴尔贝克附近有一座属诺曼第省的引人入胜的城市,城里有两所十八世纪留下的、款式宜人的府邸,从许多方面说,我喜欢这两处建筑,并且打心眼儿里崇拜。从那个有一溜台阶通往河沿的花园看去,一座哥特式教堂的塔尖恰恰夹在它们中间。教堂本身被那两所府邸遮去,但塔尖却像它们楼面的屋顶,像加在楼顶的装饰,但是,它的格局又是那样不同,那样可贵,那样多姿,那样娇艳,那样光鲜,使人一下子便看出它同下面的建筑并无关系,正等于在海滩上两块并列的漂亮的卵石之间,夹着一只尖塔形的、色泽鲜艳的贝壳,它那红得发紫、带有涡纹的尖头,同卵石毕竟不构成一体。甚至在巴黎,在最丑陋的地区,我记得有一个窗户,从那里望出去,是一幅由好几条街道的凌乱的屋顶组成的画面,你可以在前景、中景、甚至远景的某个层次,看到一座紫色钟楼的圆顶,有时它发红,也有时,茫茫雾霭从灰濛濛中离析出黑影,洗印出最精美的“照片”,使它呈现为高雅的黑色,这就是圣奥古斯丁教堂的钟楼,它使巴黎的这一景象,具有皮兰内西\footnote{皮兰内西(1720—1778):意大利版画家和建筑师,他的版画作品有组画《监狱》和《罗马风光》等。}笔下的某些罗马风光的特征。但是,无论我的记忆用哪一种笔法来描绘当年所见的情景,我都无法把失去多年的感触在记忆的版画中重现。感触使我们端详一件事物不仅把它当做观赏的对象,而且相信它是独一无二的。所以没有一幅记忆的版画能独立地保全我内心生活的某一完整的部分,如同我忆及从贡布雷教堂后面的街上所见到的钟楼的种种景象,那样完整地保留着当年的心境。五点钟看到它,那是上邮局去取信的时候,只见它在左面离我们几幢房屋远的地方,突然孤零零地矗起它的塔尖,超过一溜屋脊;如果返身想去问候萨士拉夫人的近况,那么你眼前的那溜屋脊就会随着你走下另一面的斜坡而降低,你知道得在钟楼过后的第二条街拐弯;如果你还朝前走,向车站那边走去,你侧眼看看钟楼,它就会向你展示新的屋脊和新的楼面,就像某种固体在它演变的某一时刻突然被人发现;或者,你从维福纳河的沿岸看去,教堂的后殿显得在高处蹲着。它那鼓起的肌肉仿佛迸发出钟楼借以向空中发射箭头的力量。总之,无论你在哪里,你的眼光都得落到钟楼的身上,它总高踞于一切之上,在一个意想不到的高处把房舍召集到它的跟前。在我的心目中,它像上帝的手指;上帝本人可能隐迹于芸芸众生之间,我并不会因此而混淆上帝与凡人的区别。直到今天还是一样,倘若我在内地的哪一座大城市,或者在巴黎我不熟悉的哪一个地段,为我“指点迷津”的路人把远处某家医院的钟楼或者某所修道院里高高顶着僧帽帽尖的钟楼作为标志指给我看,告诉我该走那条街,我的记忆会立刻在那钟楼的楼身,发现一些蛛丝马迹,同我所钟爱、现在已经消失的钟楼的外貌,多少有相似之处。如果那路人回过头来,看看我有没有走错路,他会惊讶地发觉,我已把该走的路和该办的事置诸脑后,一连几个钟头呆立在钟楼前苦思冥想地追忆,而且在我的内心深处感到从遗忘中夺回来的地盘逐渐变得结实,并得到重建。于是,我大概比刚才问路的时候更为焦虑地在寻问自己的道路,我转过一条街……但是……这是在我自己的心中寻问。
\par 在回家的路上,我们经常能遇到勒格朗丹先生。他在巴黎当工程师,所以除了休假之外,他只能在星期六晚上到贡布雷的庄园来,呆到星期一早晨再走。他是那种除了科技专业在行,而且成绩出色之外,还具有其他文化修养的人,例如文学、艺术方面的修养;这对他们所从事的专业完全无用,只在谈吐方面可资益助。这些人比许多文学家更有文采(那时我们并不知道勒格朗丹先生作为作家也颇有名气,当我们得知有位著名的音乐家曾经根据他的诗谱过曲,我们还大吃一惊呢),也比许多画家更“出手不凡”;据他们自己想,他们眼前的生活对他们并不合适,因而他们对待实际从事的职业,要么夹杂着幻想而漫不经心,要么高傲地、鄙夷地力求做好,既隐忍苦衷,又兢兢业业。勒格朗丹先生高高的个子,风度潇洒,留着两撇长长的淡黄色的小胡子,显得既有思想又很精明;蔚蓝色的目光透出看破一切的神情。他举止彬彬有礼,谈锋之健是我们前所未闻的。他在我们全家人的心目中是生活高雅的精英人物的典型,我们总引以为楷模。我的外祖母只嫌他一点不足,就是他说起话来过于讲究,有点像书面语言,不像他戴的大花领结总那样飘逸而自然,不像他身上那件学生装式的单排扣上衣总那样洒脱而随意。我的外祖母还因为他经常攻击贵族、攻击摆阔讲排场、攻击趋炎附势,而且措辞激烈,感到惊讶。她说:“圣保罗说到有种罪过不可原谅,一定是指这类恶习。”
\par 追求虚荣是我的外祖母所无法体会、甚至无法理解的一种感情,所以她认为完全不必这样大动肝火去贬斥它。况且,既然勒格朗丹先生的姐姐嫁给了巴尔贝克附近一位下诺曼第省的贵族,他还这样激烈地攻击贵族,甚至埋怨革命没有把他们全都推上断头台,我的外祖母认为未免有失厚道。
\par “朋友们,你们好!”他迎上前来,对我们说,“你们住在这里真是有幸;明天我得返回巴黎,钻到我的窝里去了。啊!”他又堆起他独有的、稍带讥讽、略含失意、更有点漫不经心的微笑补充说道,“当然,在我家里,没用的东西倒应有尽有,唯独缺少最必要的东西——一大片像这样的蓝天。小伙子,尽量在你的生活里始终保持一片蓝天吧,”他转身对我说,“你有一颗难能可贵的心,你具有艺术家的天赋,别让它缺少应有的东西。”
\par 我们一回到家里,我的姨妈就派人来问:古比尔夫人做弥撒是不是迟到了。我们无法回答,反而给她增添烦恼:我们告诉她说,有个画家去教堂临摹坏家伙希尔贝的彩绘玻璃窗了。于是弗朗索瓦丝立刻被派往杂货铺打听,结果一无所获,因为戴奥多尔不在。此人身兼两职,在教堂他是唱诗班成员,在杂货铺他是店堂伙计,既能从教堂里得到消息,又同社会各集团的人都打交道,所以城里的事他无所不知。
\par “唉!”我的姨妈叹了口气,“我真希望欧拉莉快点来。其实只有她才能告诉我真相。”
\par 欧拉莉是个又瘸又聋、爽直泼辣的老姑娘,从小在拉布勒东纳里夫人家帮工,夫人死后,她也随即“退休”,在教堂旁边找到一间房子往下,经常出来做做礼拜,在没有礼拜的时候,她自己默默祈祷,或者给戴奥多尔搭把手,帮点忙;其余时间,她用来探望几位像我姨妈那样的病人,她把做弥撒和做晚祷的时候所发生的事情告诉我的莱奥妮姨妈。她本来有一笔老东家给的年金养老,不过她倒不轻视捞外快,常常到本堂神甫或者贡布雷僧侣界的其他头面人物那里去搜罗些内衣被单来浆洗。她身穿披风,头戴白色小便帽,打扮得跟吃教会饭的人差不多。皮肤病使她的一部分面颊和弯曲的鼻梁呈现凤仙花那样鲜艳刺目的桃红色。她的来访一向是莱奥妮姨妈的一大乐事,因为除了本堂神甫之外,姨妈早已把其他客人逐个拒之于门外了,她认为那些人错就错在属于她所憎恶的两类人之列:第一类人最差劲,是姨妈首先要甩开的,他们劝她不要“顾影自怜”,还鼓吹“阳光下走走,吃点带血的烤牛肉,比卧床和服药对她更有补益”之类的邪端异说,尽管有人采取消极态度,只以某种形式的沉默表示不赞成姨妈的做法,或者笑笑表示怀疑;至于另一类人,看来真以为姨妈的病情比她自己估计的还要严重,至少同她自己所说的一样严重。比如,姨妈几经斟酌,听从了弗朗索瓦丝殷切的劝说,允许他们上楼来看望她,他们中就有人表现得太辜负姨妈的抬举,居然怯生生地说:“您不认为遇到好天气出去稍微活动活动会好些吗?”有人倒相反,听姨妈说罢“今天我很不好,很不好,要完了,可怜的朋友们呀”,他们竟接茬说:“啊!身体不好嘛!不过您这样也还能拖一阵呢。”上述两种人,虽然表现不同,有一点倒肯定一样,那就是从此被拒于门外。当我的姨妈从床上看到圣灵街有这号人显然正前来看她,当她听到门铃已被拉响时,她的脸上顿时出现害怕的表情。如果说,弗朗索瓦丝见此情状觉得有趣,那么,她更为姨妈总有巧妙办法把他们打发走而拍手称快,更为他们没有见到姨妈,反而碰了一鼻子灰而乐不可支。她打心眼儿里佩服我的姨妈,她认为自己的女东家比那些人要优越,所以才不愿让他们登门。
\par 总而言之,我的姨妈既要求人家赞成她卧床服药的做法,又要求人家同情她的病痛,还要求人家说些宽心话,担保她早晚会康复。
\par 而欧拉莉对此最在行。我的姨妈尽管一分钟之内能说上几十遍:“我完了,可怜的欧拉莉。”欧拉莉准能答上几十遍:“奥克达夫夫人,您对自己的病知道得这么透彻,那么您准能活上一百年,就像昨天萨士兰夫人对我说的那样。”(欧拉莉的坚定不移的信念之一,就是认准了萨士拉夫人其实叫萨士兰夫人,尽管经验无数次地对她进行纠正,仍不足以打破她的这一信念。)
\par “我倒不求活上一百年。”我的姨妈说;她不喜欢人家用确切的日期来判定她能有的寿限。
\par 此外,欧拉莉还善于给我姨妈解闷,又不让她累着。这是谁都没有的本领。所以她的来访对于姨妈来说是莫大的愉快。她每星期天必来,除非有意外事缠身。对欧拉莉又将来访的期望,开始着实让我姨妈高兴好几天,可惜这很快就转化为痛苦,就像挨饿的人饿过了头,虽说欧拉莉才晚来一小会儿。等待欧拉莉的兴奋心情拖延过久就变成不堪忍受的折磨;我的姨妈不停地看钟点、打哈欠,一阵阵感到心力交瘁、支持不住了。要是欧拉莉来访的门铃声直到天黑,在我的姨妈已无指望的时候才打响,她反倒感到伤心难受了。事实上,每个礼拜天,她最牵肠挂肚的一件事不过是欧拉莉的来访。吃罢午饭,弗朗索瓦丝急于等我们早早离开饭厅,她好赶上楼去“忙乎”我的姨妈。但是(尤其自从晴朗的天气在贡布雷定居下来之后),当正午时分的崇高的钟声给圣伊莱尔塔楼上音响的王冠缀上十二朵转瞬即逝的小花、使袅袅余音在我们的餐桌边,在也是亲切地来自教堂的圣饼的附近,缭绕萦回了很久之后,我们仍久久地坐在饰有“一千零一夜”图画的平底碟前懒得动弹,因为炎热,尤其是因为吃得太饱,我们无力离席。所谓太饱,因为,除了鸡蛋、排骨、土豆、果酱、烤饼等几道已经不必预告、每餐必备的食品外,弗朗索瓦丝还根据庄稼地和果园的收成,海鲜捕捞所得,市场供应,邻里馈赠,以及她自己的烹调天才所能提供的东西,另外添几道菜,因此,我们的食谱,就像十三世纪人们在大教堂门上雕刻的四面浮雕一样,多少反映了一年四季和人生兴衰的节奏。添一条鲜鱼,因为鱼贩子担保它特别新鲜;添一只火鸡,因为她赶巧在鲁森维尔的市场上碰上一只肥美的;添一道骨髓蓟菜汤,因为她以前没有用这种做法给我们做过;添一盘烤羊腿,因为去外面透过新鲜空气之后一定胃口大开,况且到吃晚饭足足有七小时,有足够的时间把羊腿烤到骨脱肉酥;菠菜是为了换换口味;杏子是因为刚刚上市,街上还难得见到;醋栗是因为再过半个月就吃不上了;草莓是斯万先生特意送来的;樱桃是园子里那棵两年不结果的樱桃树又重新结出的第一批果实;奶酪是我一向爱吃的;杏仁糕是她昨天定做的;奶油圆球面包倒是我们的贡献。上述各道食品吃罢之后,专为我们做的、特别是专门献给我识货的父亲品尝的巧克力冰淇淋端了上来,那是弗朗索瓦丝别出心裁、精心制作的个人作品,就像一首短小、轻盈的应景诗,其中凝聚着作者的全部才智。谁要是拒绝品尝,说什么“我吃完了,不想吃了”,谁就立刻沦入“大老粗”之列,正等于艺术家送他一幅作品,明明价值在于作者的意图和作者的签名,他却只看重作品的重量和作品所用的材料。甚至在盘子里留下一滴残汁,也是不礼貌的表示,其程度相当于没有听完一首曲子,当着作曲家的面站起来就走一样严重。
\par 我的母亲终于对我说:“得了,别没完没了地在这儿待着了,要是你嫌外面太热,就上你自己的房间去,但是你得先透透空气,免得一离开餐桌就看书。”我于是坐到水泵和水槽附近的一条没有靠背的长凳上去。水槽像哥特式的井栏,雕有好几条火龙的图案,粗糙的石面上刻下了火龙的流线型的、包含寓意的体态,十分生动。长凳恰好在一株丁香树的树荫下;园子的这个角落有一扇便门开向圣灵街;在一片荒芜的土地上,矗立着一座独立的建筑,突出在正屋之外,门前有两级台阶,那是厨房外做粗活的小屋。从外面看去,可以影影绰绰看到里面的地上铺着斑岩一般闪闪发光的红色石板,这小屋与其说是弗朗索瓦丝的“洞府”,倒不如说更像供奉维纳斯女神的小庙,里面堆满了奶制品商人、水果店老板、菜贩子等人送来的供品,他们有些是从相当远的村落来的,就为了给“女神”献上他们田园里的时鲜。小屋屋脊上总有一只鸽子在咕咕啼叫。
\par 早先,我并不在这小庙周围的神圣的树林中久留,因为我在上楼读书之前,总要先到外祖父的兄弟阿道夫外叔祖父居住的楼下那间起坐间去呆一会儿。阿道夫外叔祖父是位老军人,以少将衔退休。他那间屋子难得照进阳光,即使窗户大开,听凭外面的热气进去,屋里也仍然无穷无尽地散发出一股幽幽的凉气,既有林区的风味,又有王政时代的盎然古风,好比走进猎场的废弃的楼阁,能让人的嗅觉久久地沉醉于梦境之中。但是,我不进阿道夫外叔祖的单间已有很多年了,因为他同我们家发生过一场误会,不再来贡布雷小住。这事是由我惹起的,经过情形如下:
\par 在巴黎的时候,家里每个月派我去看他一两次,那时他总是刚吃完午饭,穿着家常便服,侍候他的仆人穿的是紫白两色相间的条纹布工作服。外叔祖父咕哝着埋怨我好久没来看他了,没人理他了;他给我吃块杏仁饼或者一只橘子,我们穿过一间客厅,那里从来也没有人会停下坐一会儿;客厅里没有炉火,墙上装点着镀金的装饰线脚,天花板刷上蓝色,说是模仿天空;家具都蒙上了缎面垫套,跟外祖父家一样,只是这儿用的是大黄缎面;我们经过客厅,走进被外叔祖父称为“工作室”的那个房间。只见墙上挂了几幅版画,大凡是黑色衬底上有一位丰满、肉感、皮色粉红的女神,或驾一辆战车,或踩一只圆球,或在额前缀有一颗五角星;第二帝国时期这类画很受欢迎,因为一般认为画里有一种庞贝的情调。后来人们很讨厌这类画,有人之所以又开始喜欢起来,虽然说法不一,其实只有一个原因:这类画具有第二帝国的情调。我同外叔祖父一直坐在这里,直到他的听差替车夫来问什么时候用车。外叔祖父沉吟良久,在一边纳罕的听差如果稍有动弹,仿佛就会扰乱他沉思似的,于是他只得全神贯注地等待他作出始终如一的回答。外叔祖父经过一番周密的斟酌,终于说出了从来不变的决定:“两点一刻。”听差惊讶地重复了一遍,但决无二话:“两点一刻?……好,我告诉他去。”
\par 在那个时期,我热爱戏剧,但这只是柏拉图式的爱,因为我的父母还一直没有允许我去看戏,所以我把看戏的乐趣,想象得相当不符合实际;我几乎以为每个观众眼中的舞台布景,都像是通过立体镜才看到似的,只为他一个人存在,尽管同其他观众所看到的上千种其他景象大致一样,但各人所见只属各人。
\par 每天上午,我都要跑到广告亭去看看又有什么新戏预告。每一出预告的新戏都给我的想象提供种种梦想,而天下最无利害关系又最令人开怀的,莫过于这些梦想了;同组成剧名的每一个单字紧密相关的形象,还有墨迹未干、被糨糊弄得鼓鼓囊囊的海报的颜色,更助长了我的想象。海报上剧名赫然在目,除了《赛萨·奚罗多的遗嘱》或《欧迪普斯王》之类的古怪剧目外(这类剧目不会出现在“喜剧歌剧院”的绿色海报上,而只出现在“法兰西喜剧院”的酡红色的海报上),最大相径庭的要算《王冠上的钻石》和《黑色的多米诺骨牌》这两出戏的海报了:一张是发亮的羽白色,另一张像带有神秘色彩的黑缎。我的父母向我宣告:我第一次去剧院,必须就这两出戏中选一出。于是我接连对它们的剧名进行钻研,因为我的有关这两出戏的全部知识只是它们的剧名。我殚精竭虑地想逐一抓住它们可能给我带来的乐趣,然后进行比较,最后我费尽力气,把一出戏想象成光彩夺目、气宇轩昂,另一出戏则温情脉脉、缠绵悱恻,结果我还是不能决定我的取舍,正等于上最后一道甜食时,问我要牛奶米糕还是要奶油巧克力一样。
\par 我与我的同学们谈论演员,虽然那时我对演技还一无所知,却认为在艺术借以体现的一切形式中,演技是首要的形式,通过演技,我才第一次感受到什么是艺术,同样一段台词,这位演员和那位演员在朗诵方法和声调处理方面各不相同,我觉得其中最琐细的差别都具有无法估量的意义。我根据有关这一演员和那一演员的传闻,把他们按才艺的高低排了个先后,这些名单我成天独自默诵,最后在我的脑海中凝固,像结成了硬块,弄得我头脑僵硬。
\par 后来,我上中学,每当我趁老师转身的机会同一位新朋友窃窃私语时,我的第一问题总是问他是否去过剧院,是否认为最了不起的演员是戈特,其次是德洛内,等等。倘若他认为法布夫尔不如迪龙,或者德洛内名列戈克兰之后,那时我的心目中戈克兰便失去磐石般的坚固性,突然松动起来,退缩到二等,德洛内也取得了神奇的灵活性,丰富的活跃性,而屈居第四;这样的变动使我的头脑得到软化,得到滋养,竟有繁花似锦、生动活泼之感。
\par 虽说我对演员们如此着迷,虽说有一天下午我见到莫邦从法兰西剧院出来顿时感到爱的激动和爱的痛苦,但是当我见到某家剧院门前某位赫赫巨星的大名烨烨生辉,当我见到一辆马头上缀满玫瑰花的双座轿车从街上驰过,车窗里露出一位据我想可能是演员的女子的倩影,那时我内心的激荡更久久不能平息,我多么无能为力地、多么痛苦地努力设想她们的私生活啊!我虽把最有名的女演员按才艺的高低排出如下的名次:萨拉·贝恩纳特,拉贝玛,巴代,玛德莱娜·布洛昂,霞娜·萨马里,但是,无论先后我对她们全都关心。我的外叔祖父认识不少女演员和一些“交际花”,我分辨不清后者同女演员的差别。他把她们请到家中做客。我们之所以只在某些日子去看望他,是因为其他日子有那些女客登门,家里人一向不愿与她们打照面。至少我们家持这一主张,因为从我的外叔祖父那方面说,他跟那些可能从来没有结过婚的风流寡妇、跟那些虽大名鼎鼎、其实出身靠不大住的伯爵夫人过于随便的态度,他把她们介绍给我的外祖母时所说的奉承话,或者他把祖传的首饰送给她们,以巴结讨好,等等,早已不止一次引起他同我的外祖父之间的龃龉。平日交谈中如果出现某位女演员的名字,我常听到我的父亲笑着对我的母亲说:“这是你叔叔的一位女朋友。”当时我想,有多少大人物恐怕开始一连好几年都巴结不上那样的女人,给她写信不理,登门拜访,她又打发门房拒之门外:我的外叔祖父倒说不定有办法让我这样初出茅庐的青年免受这番折腾,他可以在自己的家里把我介绍给许多人都无法接近、但对他来说却是知心朋友的女演员。
\par 因此——我借口有一门课改了时间,不仅已经耽误了我好几次不能去看外叔祖父,而且以后还会没有空去——有一天(那并不是专门留给我们去看他的日子),我们家午饭比平时吃得早,我便趁机上街,并没有去看家里允许我单独去看的新戏海报,而是一口气跑到了外叔祖父那里。我注意到他家门口停着一辆双驾马车,马的护眼罩上,跟车夫上衣的扣眼上一样,插着一朵红色的康乃馨。我从楼梯上就听到一个女人的嬉笑声,等我一拉门铃,里面的声音反而戛然而止,一片寂静之后是连续的关门声。听差终于出来开门,见到是我,显得很尴尬,声称我的外叔祖父现在正忙着,恐怕抽不出身来见我。他正打算进去禀报,只听到里面传出刚才的女人的声音:“啊,不!让他进来;一分钟就行,我一定会很高兴的。从您的写字台上的那张照片来看,他跟他的妈妈,也就是您的侄女,长得很像,您的侄女的照片挨着的那张照片不就是他吗?我倒是想要见见这孩子,哪怕见一面呢。”
\par 我听到我的外叔祖父咕哝着表示不高兴;最后,听差请我进去。
\par 桌子上,有一盘跟平时一样的杏仁饼;我的外叔祖父仍穿着那件家常便服,但是在他的对面,坐着一位身穿粉红色丝绸长裙、脖子上挂着一条长长的珍珠项链的年轻女子,她正把最后一瓣橘子放进嘴里。我一时拿不定主意,不知该称呼她夫人还是小姐。我憋红了脸,不敢朝她那面看,生怕同她答话。我过去亲了亲外叔祖父。她笑眯眯地望着我。我的外叔祖父对她说:“这是我的侄外孙。”既没有告诉她我姓什么,也没有把她的名字告诉我,大约是因为自从同我的外祖父发生过龃龉之后,他尽可能避免家庭成员同他的这类朋友接触。
\par “他长得多像他的母亲。”那女的说。
\par “您也不过是在照片上见过我的侄女。”我的外叔祖父连忙粗声粗气地接口道。
\par “对不起,亲爱的朋友,去年您生病的时候,我在楼梯上曾经同她照过面。确实,我也只是一闪而过地瞅了一眼,你们这儿的楼梯又那么黑;但是,这一眼足以使我对她钦佩了。这瘦小的年轻人眼睛长得挺美,还有这儿,”她说着,用手指划了一下额头下面,“您的侄女儿是不是跟您同姓?”她问我的外叔祖父。
\par “这孩子更像他的父亲。”我的外叔祖父咕哝着说;他既不想提到我妈妈的姓,以间接地介绍我,更不想作进一步的说明,“他完全像他的父亲,也像我故世的母亲。”
\par “我不认识他的父亲,”穿粉红色长裙的女子微微歪着脑袋说道,“也从来没有见过您那位故世的母亲。我的朋友,您一定记得,咱们是在您遭受丧母之痛后不久才相识的。”
\par 我感到有些失望,因为这位少妇同我在家里见到过的其他标致女子,尤其是同我每逢大年初一都要去拜年的一位表亲家的千金并无二致。我的外叔祖父的这位女朋友,除了衣着更为讲究之外,那眼神也同样机敏而和善,表情既坦诚又动人。我在她身上没有发现女演员照片上一般有的那种使我倾慕的舞台风度,也没有看到应该同她的私生活相呼应的那种妖媚的表情。我难以相信她竟是交际花,而且如果我没有见到门口停着的那辆双驾轿车,没有见到她那身粉红色的丝裙和那串珍珠项链,没有早就听说我的外叔祖父尽结识些最高级的交际花,我恐怕更难相信眼前这位风韵不俗的女子就是其中的一位。但是,我不明白的是供她们住华屋、坐轿车,让她们打扮得珠光宝气,不惜为她们倾家荡产的金屋藏娇的百万富翁,又怎能从这样平凡、这样规矩的女子那里得到愉快呢?然而,想到她们私生活应有的情状,我更为她们的不道德感到迷惑不解。如果这种不道德具体化为一个特殊的形象出现在我的面前,那么这种不道德就会像一部小说、一件丑闻的隐秘部分那样地不露痕迹。但恰恰是那件丑闻使她们脱离了中产阶级的家庭和她们待人和善的父母,使她们扶摇直上地变为一代佳丽,出入交际场所,赢得显赫的名声。眼前的这位女子,面部表情和说话的声调同我所认识的其他许多妇女并无两样,这就使我不由得把她看做良家千金,其实她早已无家可依了。
\par 这时我们已经走进外叔祖父的工作室。我的外叔祖父请她抽烟,只因有我在场,他多少显得有些尴尬。
\par “不,”她说,“亲爱的,您知道我只抽得惯大公爵送给我的那种烟卷。我跟大公爵说了,您也馋那种烟卷。”说着,她从烟盒里掏出好几支印有金色外文字样的纸烟。忽然,她又说:“我一定在您这里见到过这孩子的父亲,他不就是您的侄女婿吗?我怎么能忘呢?他那样和气,我觉得他文雅极了。”她说得既谦虚又热情。但是,我深知父亲待人一向矜持冷漠,想到他当时一定绷着脸皮,现在却被说成文雅极了,我不禁狼狈不堪,因为他很可能表现得并不风雅,这种过高的评价,同他在礼节方面的欠缺实在太不相称。后来我才体会到,这些既无所事事又用心良苦的妇女所扮演的角色,其魅力之一正在于此:她们以她们的热情、她们的才能,以及优美的感情所具备的一种梦境和她们不必破费便可轻易到手的一种金玉般的华彩,像名贵而细巧的嵌饰,把男人们毛糙而缺乏磨砺的生活装点得富丽堂皇。对于梦境,她们同艺术家们一样,既不追求实际价值,也不让它局限于现实生活,例如我的外叔祖父穿着宽松的便服在吸烟室中接待的这位女士,她以娇美的体态,粉红色的丝绸长裙,周身的珠光宝气,以及她同大公爵的交情所散发出来的那种高贵气派,给烟雾缭绕的室内增添了异样的光辉;同样,她随口说了句对我父亲的评价,说得非常讲究,使这句话别具一格,有一种高雅的意味,再加上她以亮晶晶的目光看上一眼,等于给这句话镶上一颗光华熠熠的钻石,其中既包含谦恭之意,又透出感激之情,这句话从她嘴里说出便成了一件艺术珍品,一件“文雅极了”的宝贝。
\par “好吧,孩子,你该回去了。”外叔祖父对我说。
\par 我站起来,克制不住想去吻一下粉衣女郎的手,但,我觉得这样做恐怕过于孟浪,简直类似抢劫。我的心怦怦乱跳,心里盘算着:“该做还是不该做?”后来,我不再考虑该做什么,而是能做什么,我以一种盲目的、反常的动作,连刚才我找到的有利于这样做的种种理由也全都抛置不顾了:我上前抓住她伸过来的手,把它送到我的唇边。
\par “他多可爱啊!已经知道巴结女人喜欢了,这是跟他的外叔祖父学的。将来准成为十全十美的绅士。”她又咬文嚼字地加上这么一句,故意把绅士这个词儿说得带点英国口音。“用跟我们一衣带水的英国邻居的话来说,哪天他能不能过来喝a cup of tea?\footnote{英语:一杯茶。}到时候,上午给我发一封‘蓝笺’\footnote{蓝笺:市内电报的俗谓。}就行了,我准来奉陪。”
\par 当时我还不知道“蓝笺”是什么意思。她的话我有一半听不懂。我怕有些问话若不回答会有失礼貌,所以我始终全神贯注地听,结果感到非常吃力。
\par “不,不,这不可能。”我的外叔祖父耸耸肩膀,说道,“他忙得很,他很用功。他的功课门门得奖。”他又低声地——声音压得很低,怕我听见后纠正——补充说道。“谁说得准呢?也许他将来是雨果第二,或是福拉贝尔\footnote{福拉贝尔(1799—1879):法国历史学家,1848年任公共教育部长。}之类的人物。这您是知道的。”
\par “我崇拜艺术家,”粉衣夫人答道,“只有艺术家才了解妇女……只有他们和您这样出类拔萃的人才理解我们。原谅我的无知,朋友,福拉贝尔是何许人?就是您房里玻璃书柜上的那几本烫金的书籍的作者吗?您知道,您答应借我看的,我一定小心翼翼地爱护书籍。”
\par 我的外叔祖父最讨厌借书给别人,因而没有接话。他一直把我送到过厅。对粉衣夫人的爱慕弄得我晕头转向,我发疯似的吻遍了我外叔祖父沾满烟丝的两边腮帮。他相当尴尬地暗示我:希望我最好不要把这次来访告诉家里,但他又不敢明说。而我呢,我热泪盈眶地向他表示:他对我的一片好心,我铭感至深,总有一天要想办法报答。我倒确实铭感至深:两小时之后,我先是说了些闪烁其词的话,后来觉得并没有让我的父母明确地认识到我新近得到的器重,于是我想倒不如把话挑明,干脆把两小时以前去外叔祖父家的经过,详详细细地告诉他们,我没有料到这样做会给外叔祖父招惹是非。我本来没想给他添麻烦,怎么能料到这一着呢?我不能想象我的父母能从中找出毛病,因为我并不认为有什么不对,不是每天都会有这样的事情发生吗?——一位朋友来请求我们千万别忘了代他向某某女士表示歉意,因为他本人无法给她投书致意,而我们经常不把这种事放在心上,认为那位女士未必把他的沉默看得多重要,我们不觉得转致歉意能有多大意义。我也跟大家一样,总把别人的脑海想象成一件来者不拒的容器,对于注入的东西不会有什么特殊的反应;我从不怀疑,始终以为我把在外叔祖父家结识新朋友的消息灌进我父母的脑海,也就能如愿以偿地把我对这次介绍的善意判断转达给他们了。不幸的是我的父母在评价我的外叔祖父的行为时所遵循的原则,同我的期望完全南辕北辙。我的父亲和我的外祖父向我的外叔祖父提出措辞激烈的质问;我是间接听说的。几天以后,我在街上迎面遇到我的外叔祖父,他正坐在一辆敞篷车上。我感到痛苦、后悔、对他不起,我真想把这些感受告诉他。但我内疚之深、铭感之深,决不是摘帽致意所能表达的;我觉得这反倒会显得小家子气,甚至可能让外叔祖父看不出我对他感恩戴德,只以为我用通常的礼貌敷衍罢了。我决定免去这种不足以表达我内心感情的举动,我把脸扭了过去。我的外叔祖父却以为我为了服从父母的命令才不理他的,因此他对我的父母记恨在心。好多年后他才死去,我们一直没有再去看望他。
\par 所以,我就不再进入已经关闭的阿道夫外叔祖父的那间休息室了。我只在厨房外的小屋周围流连。这时弗朗索瓦丝出现在小庙前的平台上对我说:“我让帮厨的女工一会儿把咖啡和热水端去,我要赶紧去侍候奥克达夫夫人。”听她这一说,我决定回屋,直接到我的房里去读书。帮厨的女工是个有名无实的角色,是个常设的职位,承担着始终如一的任务,它通过体现它存在的一连串暂时的形态,保证了某种连续性和同一性,因为从来没有一个帮厨女工在我们家连续干满两年以上。我们吃了许多芦笋的那个年头,帮厨女工一般负责削芦笋皮。那是一个病病歪歪的女人,我们在复活节前后到达贡布雷的时候,她正怀着孕,而且已接近临产期。我们甚至奇怪:怎么弗朗索瓦丝还让她走那么多路,干那么多活,因为她的身前挂着的那只日见饱满的包袱,虽然有宽大的工作服罩在外面,仍能让人看出它已大到相当可观的地步,况且她开始步履艰难了。她那身衣裳使人联想到乔托\footnote{乔托(1266—1337):意大利画家。他的体积感、空间感以及对自然景物的偏爱,使他成为意大利绘画发展史上那一阶段的代表。他为帕多瓦的阿林娜圣母寺所作的壁画(约于1303至1305年间),是他传世的杰作之一。}的壁画中的几位象征性人物身上所穿的那种宽袖外套。这些壁画的照片,斯万先生曾经送给我过。使我们注意到这个特点的,也是他。每逢问起有关帮厨女工的近况,他总这么说:“乔托的‘慈悲图’近况如何?”也确实,那可怜的女工因怀孕而发胖,一直胖到脸上,腮帮结实得堆起了横肉,同画里那些更像接生婆的粗壮的处女不相上下;在阿林娜圣母寺的壁画中,她们是种种美德的化身。今天我才意识到,帕多瓦寺院里的那些善恶图,还从另一方面跟我们的帮厨女工相像。帮厨女工的形象由于腹部多了一件象征而变得高大起来,但她本人显然并不理解这一象征,她的脸上没有丝毫表情来传达它的美和它的精神意义,似乎她只是抱着一只普通的、沉重的包袱;同样,阿林娜圣母寺里那幅标题为“慈悲”的壁画,显然也没有让人家想到画中那位结实的主妇形象正是慈悲这一美德的化身(在贡布雷我的自修室的墙上就挂有这幅画的复制品),看来那张结实而俗气的面孔不可能表达任何慈悲的思想。多亏画家别出心裁的独创,她脚下明明踩着大地的宝藏,那表情却完全像在踩挤红的葡萄汁,或者更像跨上一堆装满东西的口袋往高处攀登;她把自己热烈的心献给上帝,说得更确切些,她在把心“递”给上帝,就像厨娘把起瓶塞的工具从地下室的气窗里递给正在楼下窗口向她要这件工具的人。“贪欲”这幅壁画,倒也许把贪欲的某种表现,描述得更为露骨。但是,象征也还是占据太多的地盘,而且表现得过于真实。对准“贪欲”的嘴唇嘶嘶吐信的蛇被画得很粗,把“贪欲”张得大大的嘴巴整个填满;为了把蛇含进嘴里,她的面部的肌肉全都鼓起来了,就像小孩儿吹气球一样,“贪欲”的注意力也引动了我们的注意力,全都集中在嘴唇的动作上,没有给贪婪的思想留下多少回旋的余地。
\par 尽管斯万先生对乔托的这几幅壁画推崇备至,我却在很长一段时期内无心欣赏;他送给我之后就一直挂在自修室墙上。“慈悲图”上没有慈悲;“贪欲图”则像仅在医学书上才能见到的插图,类似声门或小舌如何受到舌瘤的压迫,或者外科医生的器械如何插进口腔;而那位象征正义的女子,面色灰暗,五官端正而表情啬刻,这恰恰是我在做弥撒时所见到的贡布雷某些相貌漂亮、感情贫乏、虔诚刻薄的中产阶级小姐、太太的写照,而她们中有些人早就充当了不正义的后备军。后来我才懂得,这几幅壁画之所以诡谲离奇得动人心魄,具有特殊的美,是因为象征在其中占据了主要的地位;事实上象征并没有作为象征来表现,因为象征化的思想是无法表现的,在这里它是作为真实的来表现的,表现为具体的感受或物质的动作,这就使作品的含义更切题,更准确,也使作品的教益更实惠,更惊人。在可怜的帮厨女工的身上,情况也一样,人们的注意力不也是一再被日益变大的肚子吸引过去吗?还有,人之将死,想到的往往是实际的、痛苦的、昏暝莫辨的腑脏深处,往往想到死亡的阴暗面,这恰恰是帮厨女工所呈现的模样:她使我们严峻地感觉到这一面的存在,与其称之为死亡的抽象观念,倒不如说它更像一个要把我们压扁的包袱,一种令人喘不过气来的绝境,一种急需痛饮的干渴。
\par 帕多瓦寺院中的善恶图,肯定包含许多现实成分,因为在我看来,它们活生生得像我们家的怀孕的帮厨女工;而且我觉得那位女工身上也存在丰富的寓意。一个人的灵魂往往不参与通过自己才得以表现的美德,这种不参与(至少表面如此),除了有其美学价值外,也还包含一种真实,一种即使不是心理学的、起码也是面相术方面的真实。后来,我在实际生活中,曾多次有机会遇到过一些真正神圣的悲天悯人的化身,例如修道院里的僧尼。他们一般看来都兴致勃勃,讲究实惠,像忙忙碌碌的外科医生,既不动感情又果断利索,面对着人类的苦难,他们的脸上并无丝毫怜悯、同情的表示,也不怕去触及人们的痛处,那是一张张没有柔情、令人生畏的脸,因真正的善良而变得格外崇高。
\par 帮厨女工先端上咖啡(用我母亲的话来说,只配叫热水),然后又把热水(其实勉强有点热气)送到我们房里,这就无意中像谬误通过对比衬托出真理的光辉那样地更显示出弗朗索瓦丝的高明优越之处,那时我早已拿着一本书躺在我自己房里的床上了。几乎全都合上的百叶窗颤颤巍巍地把下午的阳光挡在窗外,以保护房内透明的凉爽,然而,有一丝反光还是设法张开黄色的翅膀钻了进来,像一只蝴蝶一动不动地歇在百叶窗和玻璃窗之间的夹缝里。这点光亮勉强够我看清书上的字迹,只有神甫街上加米拍打箱柜灰尘的声音,才让我感到外面的阳光有多灿烂(弗朗索瓦丝告诉加米:我的姨妈不在“休息”,可以暂勿噤声)。那一声声拍打,在炎热季节特有的訇然传音的大气中回荡,仿佛抖落下无数艳红色的星雨,一颗颗飞向远方。此外,还有一群苍蝇,像演奏夏季室内乐般在我的眼前演奏它们的小协奏曲,倒跟你在盛夏季节偶尔能听到乐师们演奏的曲调并不一样,但是能让你接着联想到人间的乐声;这种音乐由一种更加不可缺的纽带把它同夏季联系在一起:它从晴朗的日子里诞生,只能同晴朗的日子一起复活,它蕴含着晴朗的精魂,不仅能在我们的记忆中唤起晴朗的形象,还能证实晴朗已经归来,确实就在外面,而且已弥漫人间,唾手可及。
\par 我的房里的这种阴暗的清凉,就像大街阳光下的荫凉处,也就是说,虽暗犹明,同阳光一样明亮,并且给我的想象展示出夏季的全部景象;而倘若我在外面散步,我的感官恐怕也只能品享到其中的一些片断;因此,这种幽暗,同我的休息十分合拍,对于常常被书中的惊险故事所激动的我,休息也只像放在流水中一动不动的手掌,经受着急流的冲击和摇撼。
\par 但是,我的外祖母,即使天气热得彤云四起,即使暴雨骤来或者只是落下几滴雨点,她都要苦苦劝我出去走走。哪怕我不肯放下手里的书本,至少也得到花园里去阅读,坐在栗树下那个用草席和苫布搭成的凉棚里;我自以为那里足可避人耳目,躲过偶尔有人来访的干扰。
\par 我的思想不也像一个隐蔽所吗?我躲在里面感到很安全,甚至还可以看看外面发生的事情。当我看到外界的某一件东西,看到的意识便停留在我与物之间,在物的周围有一圈薄薄的精神的界线,妨碍我同它直接接触;在我同这种意识接上关系前,它又仿佛飘然消散,好比你拿一件炽热的物体,去碰一件湿淋淋的东西,炽热的物体接触不到另一件东西上的潮湿,因为在触及前水分总是先已气化。我在读书的时候,我的意识同时展现出多种不同的情景,它们斑驳杂陈地仿佛组成一幅五光十色的屏幕,上面展示出埋藏在我内心最深处的种种愿望,乃至于我在这花园角落里眼前所见的纯属外观的各类景象之中,最切近我内心深处、并不断活动着又统帅其余一切的,是我的信念和我的愿望:我相信我正读着的那本书里有丰富的哲理,蕴藏着美,我但求把它们占为己有,不管那是本什么书。因为,即使那本书我是在贡布雷镇上的博朗士杂货铺跟前一眼瞥见之后买的,那铺子离我家较远,弗朗索瓦丝不可能像上加米杂货铺那样去那里买东西,但他们的书籍品种比较齐全,赶得上文具店和书店,门口的那两扇门板,比教堂的大门更神秘,更引人浮想联翩,上面琳琅满目地挂着许多期刊和小册子,我发现那本书就挂在其间,我之所以选中它,是因为早先听到老师或者某位同学提到过,当时在我的心目中,那位同学看来已经深得真和美的奥秘,而我对真和美还只有模糊的感觉,只有一知半解,认识真和美是我的思想所追求的目标,虽然不很明确,我却念念不忘。
\par 我在阅读的过程中,这一中心信念不断地进行由表及里和由里及表的运动,以求发现真理,随着信念而来的是我积极参与的活动所产生的内心激荡,因为那些天下午我的曲折经历,常常比一个人整整一生的经历更为丰富、更为充实。我说的是我读的那本书里发生的种种事情;的确,受事件影响的人物,正如弗朗索瓦丝所说,并非“实有其人”。但是,一位真实人物的悲欢在我们心中所引起的各种感情,却只有通过悲欢的具体形象作媒介,才能得到表现;第一位小说家的聪慧之处就在于他了解到在我们激情的机制中,既然形象是唯一的要素,那么干脆把真实人物排除掉的那种简化办法,就是一项决定性的完善措施。一个真实的人,无论我们对他的感情有多深,总有相当大一部分是我们感官的产物,也就是说,我们始终无法看透,总有一种僵化的分量是我们的感觉所抬不动的。遇到有什么不幸落到这人的头上,我们固然也能为之而伤心,但是我们心目中他所遭受的不幸其实不过是整个不幸概念中的一小部分而已;甚至他本人也只能感受到整个概念的一部分。小说家的创举在于想到用数量相当的抽象部分,也就是说,用灵魂可以认同的东西来替换灵魂无法看透的部分。既然我们已经把这些新形态下的人物的举止和感情化作了我们自己的举止和感情,既然这些举止和感情是在我们的内心得到表现的,而且,当我们心情激荡地翻阅书中一页又一页的文字时,书中人物的举止和感情在我们的内心控制了我们呼吸的急缓和目光的张弛,那么,表面上的真实与否又有什么要紧呢?小说家一旦把我们置于那样的境地,也就是说,同纯属内心的种种境界一样,凡喜怒哀乐、七情六欲都得到十倍的增长,那么,他写的那本书就会像梦一样搅得我们心绪不宁,但是这比我们睡着时所做的梦要清晰明朗些,因而也留下更多的回忆,到那时我们的内心在一小时中可能经历到的各种幸与不幸,我们在实际生活中或许得花费好几年的工夫才能领略到其中的一二,而最激动人心的那些部分,我们恐怕终生都体会不到,因为幸也罢不幸也罢,在生活中都是缓缓地发生的,慢得我们无从觉察(例如:悲莫大于心死,可是我们只有在阅读时、在想象中,才体会到这种悲哀;现实生活中心灵的变化同自然界的某些现象一样,其过程相当缓慢,倘若我们有可能对变化中的每一个不同的状态逐一进行验证,那么我们连变化的感觉都会丧失殆尽的)。
\par 故事发生的环境已经不如书中人物的命运那样深入我的内心,但它对我的思想的影响,却远比我从书上抬眼看到的周围风物的影响要大得多。所以,有两年夏天,我在炎热的贡布雷的花园中,就因为当时阅读的那本书,我竟神往一片山明水秀的地方,希望在那里见到许多水力锯木厂,见到清澈流水中有好些木头在茂密的水草下腐烂,不远处有几簇姹紫嫣红的繁花沿着一溜矮墙攀援而上。由于我的思想中始终保留着这样的梦,梦见一位女士爱我,所以我对那片山川的神往也同样浸透了流水的清凉;而且无论我忆及哪位女士,那一簇簇姹紫嫣红的繁花立刻会在她的周围出现,好像专为她增添颜色似的。
\par 这倒不仅是因为我们梦见的某个形象总是带有明显的特征,总得到我们在遐想中偶尔衬映在这形象周围的各种奇光异彩的烘托而显得格外美丽,而是因为我读的那些书里所描述的风光,对于我来说,并非只在我的想象中才显得更加瑰丽,它其实跟我在贡布雷所见大同小异。由于作者的选词遣句,由于我在思想上对作者的描述像对一种启示那样地虔信,书中的景物仿佛就是大自然本身的一个真实可信部分,值得细细玩味、深深探究。我当时所处的环境,尤其是我们的那座花园,经过我的外祖母所鄙视的那位四平八稳、毫无才情的园丁整治过之后,从来没有给过我这样的印象。
\par 倘若我的父母允许我去实地考察我读到的书中所描述过的那些地方,我倒真可以认为自己向掌握真理跨出了不可估量的一步。因为如果一个人感到始终置身于自己的心灵之中,那么他不会觉得自己像置身于一座稳然不动的牢笼中一样,而会觉得自己像同牢笼一起卷入无休无止的飞跃,力求冲出牢笼,达到外界,同时惶惶若失地始终听到自己的周围回荡着一种声音,它不是外界的回响,而是内心激荡的共鸣。我们力求在因此而变得可贵的万物中重新找到我们的心灵曾经投射其上的反光;我们失望地发现在自然中万物仿佛失去了原先在我们的思想中由某些相近的观念所赋予的魅力;有时我们把这种精神力量全都化为光华熠熠的机敏,以影响我们明知在我们身外却又无法触及的他人。因此,我之所以总是围绕着我所爱的女人想象我最向往的地方,我之所以希望她来领我去游历那些地方,为我打开一条通往陌生世界的渠道,这并非出于偶然而简单的联想;不,因为我对游历和爱情的梦想只是我全部生命力所迸发出的同一股百折不挠的喷泉中的不同力矩罢了;今天我好比把一股表面看来屹然不动、映射出彩虹的水柱按不同高度划分成几截那样,人为地把我的这股生命力划分出不同的力矩。
\par 我继续出入于同时在我的意识中并存的各种境况,在得以展现那些境况的真实的视野之前,我终于得到了另一种快感,安坐的快感,呼吸新鲜空气的快感,不受来客骚扰的快感。当圣伊莱尔钟楼敲响下午一点,我更因发觉下午的时光已开始一截一截地被消耗而感到痛快,我数着钟声直到最后一响,计算已经消耗的总数。接着是漫长的寂静,允许我在蓝天下读书的那一整段时间仿佛也随之而开始,直到弗朗索瓦丝准备的那顿香喷喷的晚饭端上餐桌;我在阅读时追随书中主人公走南闯北弄得相当劳累,要由精美的晚饭来补偿我的辛苦。每过一小时钟声响一次,仿佛上一次的钟声离眼前才不久;一次次的钟声在天上挨得很近,我简直难以相信,在两个金色的刻度之间,那短短的蓝色弧线下,竟能容纳下整整六十分钟。有时候,敲得这么勤的钟声,这一次比上一次多了两响,那就是说这中间有一次钟声我没有听到,其间发生了什么事对于我来说等于没有发生;读得入迷就跟睡得很实一样具有神奇的魔力,我的耳朵像中了邪似的失去听觉,寂静的蔚蓝色表盘上的金色的钟点也抹得了无痕迹。星期天晴朗的下午多迷人啊!在贡布雷花园的栗树下,我精心地把个人生活中平庸的琐事统统抛开,用另一种曲折的生活,不同寻常的追求来加以充实,我向往着一个被纵横的流水滋润和灌溉的地方。美丽的星期天的下午啊,当我一想到你们,至今犹历历在目,确实,当初我把书一页页往下读的时候,白日的炎热在逐渐消散的时候,你们就已经把那种不寻常的生活裹了起来,让它逐渐地、一点一点地结晶。这个晶体变化极慢,里面贯穿着枝头的绿叶和你们静悄悄的、回荡着声响的、香气宜人的、透明的每一个钟点。你们把那种生活保存了下来。
\par 有几次,下午三四点钟光景,园丁的女儿发疯似的奔跑,打断了我的阅读。她跑得撞倒了一棵橘子树,自己也划伤了手指,还磕掉一颗牙。只听她喊道:“他们来了!他们来了!”她倒是为了让弗朗索瓦丝和我及时赶去,别错过看一场热闹。那几天驻防部队操练,要经过贡布雷市镇,通常他们走的是圣伊尔德迦尔特街。那时我们家的用人们正摆开一排椅子,坐在铁门外,观看贡布雷街上星期天的行人,同时也让过往行人观看他们。园丁的女儿从远处车站大街的两幢房屋的夹缝间,瞅见了盔甲的闪光。用人们匆忙收拾椅子走进铁门,因为经过圣伊尔德迦尔特街的全副戎装的士兵队伍将占据整条街的宽度,马队几乎要踩着人行道,擦过两边的房屋,浩荡而去,就像洪水涌来,河床显得过于狭窄,洪水难免溢出河堤。
\par “这些孩子怪可怜的,”弗朗索瓦丝刚刚赶到铁门边就已经流下眼泪来了,“可怜,他们的青春就像草场上的青草一样,都要给割尽了。一想到这里,我就像挨了一闷棍似的。”说着,她把手捂到胸口,以表示挨到闷棍的部位。
\par “看到这些小伙子舍生忘死,不是很壮观吗,弗朗索瓦丝太太?”园丁为了给她“鼓气”,这么说道。
\par 他的话没有白说。
\par “舍生忘死?可是人生在世,不求生还求什么?生命是善良的上帝赐给我们的唯一的恩典,从来只有一次。唉呀!上帝呀!他们倒还真的舍生忘死!我在一八七〇年见过;他们一个个都不怕死,那仗打得多惨!真是不折不扣的一群疯子。再说,他们不用人家耗费什么绳子来把他们绞死,他们哪是人呀,简直是狮子。”(对于弗朗索瓦丝来说,把人比作雄狮并没有丝毫恭维之意。)
\par 圣伊尔德迦尔特街的弯拐得太小,我们无法看到队伍从远处浩浩荡荡开来,而只是从车站大街那两幢房屋之间的夹缝中看到阳光下金光锃亮的头盔不断地起伏而过。园丁本想看看是不是还有那么多士兵要经过,可是日头晒得太狠,他都渴了。于是,他的女儿像杀出重围似的突然窜到街角,冒着九死一生的危险,从那里带回一瓶柠檬水和如下的消息:从梯贝尔齐和梅塞格利丝那边不断拥来的士兵足有上千人哩。已经讲和的弗朗索瓦丝和园丁讨论起战争时期应该怎么办的问题来了。
\par 园丁说:“您看到没有?弗朗索瓦丝,革命总比别的战争强,因为一宣布革命,只有愿意上前线的人才去打仗。”
\par “啊!对了,至少我是这样理解的,这干脆得多。”
\par 园丁认为战争一爆发,铁路交通全都中断。
\par “敢情,怕人乘火车逃跑呗。”弗朗索瓦丝说。
\par 园丁说:“嗨!他们可坏了。”因为他认定战争只是国家用来作弄百姓的恶作剧,既然它有法子这么办,谁也就甭想溜掉。
\par 但是弗朗索瓦丝要赶紧去侍候我的姨妈,我也要回到我读的那本书里去,用人们重新在门外坐定,观看由士兵们掀起的灰尘和激情慢慢消散,平静下来很久之后,贡布雷街上仍流动着不寻常的黑压压的人群,家家户户的门前都有一堆仆人,甚至主人坐着观望,连平时门口没有人的那几家也不例外,他们像门槛外缀上的一条边沿参差不齐的花边,又像大潮过后留在海滩上的水藻、贝壳等物组成的一条斑斓如锦的彩带。
\par 除了那样的日子外,我平日倒总能安心读书。只是有一次,斯万来访,打断了我的阅读。当时我正在读一位我以前从未拜读过的作家贝戈特的作品,斯万对我说的那番话,倒使我在很长一段时期内,不再在挂满一簇簇紫花的墙边发现我所梦见的妇女形象,而是在完全不同的背景上,在哥特式教堂的门楼前,浮现出她们的倩影。
\par 贝戈特的大名,第一次是由一位比我大几岁的同学告诉我的。他姓布洛克,我对他十分钦佩。他听说我欣赏《十月之夜》,便哈哈大笑,对我说:“你居然对缪塞之流入迷,趣味够低级的。他是坏蛋中的坏蛋,畜生中的畜生,不过我应该坦白承认,他,还有那个名叫拉辛的家伙,他们一生之中倒是各写下一句音韵铿锵的诗行,据我看,其最高价值在于它毫无意义可言。这就是‘白净的奥路索娜和白净的加米尔’,另一句是‘米诺斯和帕西法埃的女儿’。我的恩师,受到众神宠爱的勒贡特老爹,在他的一篇文章中引用了这两句诗,目的显然是为这两名恶棍开脱。顺便说一句,我手头倒有一本书,现在暂时没有空读,好像我的伟大的恩师曾经推荐过,他认为作者贝戈特写得非常精细;虽然他有时候宽容得无法解释,但他的话在我心目中等于德尔菲神庙\footnote{古希腊供奉太阳神的神庙。古代希腊人每遇大事,即赴神庙以求神谕。}发下谕示,你读读这些抒情的散文吧,要是领受了太阳神的指点写下《皆大欢喜》和《玛纽斯猎犬》这两篇韵文的音韵大师说得不假,那么亲爱的大师,你就能品尝到奥林匹斯山上的琼浆玉液了。”他起初用调侃的语气要我称他为大师,后来他也同样称我为大师,事实上,我们开这种玩笑多少有点意思,因为我们当时少年狂放,总认为称呼什么就真能成为什么。
\par 不幸的是,我一面同布洛克闲谈,一面却无法平息内心的混乱。他刚才说,美的诗句正因为它没有含义才更美,而我只希望从诗中寻找到真理的启示。我要他就此作出解释。事实上,布洛克后来再也没有被邀请到我们家来做客。开始他在我们家受到了热情的款待。这倒是真的,我的外祖父说过,我只要跟同学中的哪一位关系更为密切,把他领到家来,那总是个犹太孩子。原则上他倒并不因此而不快——他自己的朋友斯万也是犹太人血统,他认为一般说来我是在优秀的犹太孩子中选择朋友的。所以每当我领来一位新朋友,他几乎嘴里都要哼哼《犹太女郎》中的那句歌词“我们父辈的上帝哟!”或者“以色列,砸碎你的锁链!”当然,他只哼哼调门,但是我怕我的同学听出那段调门,给它配上歌词。
\par 我的外祖父在见到我的同学们之前,只要听说他们姓什么,尽管这些姓往往没有犹太特点,他也不仅能猜到我的那位朋友是犹太血统(事实上也真是犹太血统),而且还能看到他家里有什么地方招人讨嫌。
\par “今天晚上要来的你的那位朋友姓什么?”
\par “姓迪蒙,外祖父。”
\par “迪蒙!哦!要当心哪!”
\par 说着,他哼哼起来:
\refdocument{
    \par 弓箭手们,严阵以待!
    \par 悄悄注视,切莫等闲。
}
\par 待他巧妙地向我们提出几个比较确切的问题之后,他叫出声来:“当心啊!当心啊!”或者,如果他通过隐蔽的盘问,迫使已经进门的同学不知不觉自己说出是什么出身,那时,他为了表明已经不再存有疑问,就索性一面看着我们,一面声音轻得几乎让人听不到地哼起这样的歌词:
\refdocument{
    \par 怎么,您把这胆怯的犹太佬领到了我们这里!
}
\par 或者:
\refdocument{
    \par 希布伦,亲爱的山谷,我祖祖辈辈生息的地方。
}
\par 还可能是:
\refdocument{
    \par 是啊,我们是上帝优选的民族。
}
\par 我的外祖父的这类小怪癖倒并不意味着对我的同学有任何恶意。我的长辈之所以不喜欢布洛克,那是另有原因的。他一开始就招我的父亲讨厌。那回,我的父亲见他浑身湿透,关心地问道:
\par “布洛克先生,外面变天了吗?是不是下过一场雨?我真不明白,晴雨表上刚才表明是晴天呀。”
\par 但他得到的回答却是:
\par “先生,我绝对无法奉告是否下过雨,因为我一向把物质的琐事置之度外,以至于我的感官已经不必告诉我晴雨之类的变化。”
\par 布洛克走了之后,我的父亲对我说:“可怜的儿子,你的那位朋友是白痴。笑话!他居然都无法告诉我天晴天雨!这真是有意思极了!他是呆子!”
\par 后来布洛克又惹得我的外祖母不高兴,因为吃罢午饭,她说她有点不舒服,布洛克听罢居然抽抽搭搭地抹起眼泪来。
\par “这怎么可能是真诚的呢,”外祖母对我说,“因为他根本不认识我;要不然他是疯子。”
\par 总之,他让大家都不满意,因为那回他来吃饭迟到了一个半小时,而且身上溅满污泥。他不仅不道歉,反而说:
\par “我从来不受天气变化和公认的时间分割的约束。我宁可规劝世人使用鸦片烟枪和马来亚波刃短刀,但是,对于使用钟表和雨伞这两件害处多得无以复加而且市民气十足的庸俗工具,我一向是敬谢不敏的。”
\par 尽管如此,他本来还可以来我们家玩的。他固然不是我的长辈们希望我结交的朋友,他们后来也还相信他为我的外祖母身体不适而流下的眼泪未必是做假,但是他们凭本能或者凭经验知道,我们的感情冲动对于我们随之而来的行动,以及对于我们的实际作为并无多大的影响;尊重道德准则,忠于朋友,埋头干某项工作,切实奉行某一套制度,凡此种种的更牢靠的基础尚有赖于盲目的习惯,而不是一时的冲动和空泛的热情。比起布洛克来,他们倒更希望我结交这样的朋友——这些人所能给予我的不超过根据布尔乔亚的道德标准应给予朋友的限度,不会因为哪天多情多意地惦记起我,便送我一筐水果,也不会因为一时的感情冲动和凭空瞎想,为了让友谊所要求的义务的天平倾向对我有利的一边,而不惜弄虚作假,使我蒙受更大的损害。我们的怨尤也难以把这些本质同它们对我们的要求截然分开,我的姑祖母就是一个榜样。她同她的一个侄女多年不和,根本不理她,但她并不因此而改变自己的遗嘱,仍旧把全部财产留给她,因为这是她最近的亲属,“理应”如此。
\par 不过,既然我喜欢布洛克,我的长辈就不愿扫我的兴。最让我大费脑筋、苦恼至极的问题是我实在想不通为什么米诺斯和帕西法埃的女儿之所以美,全在于这种美毫无意义。这方面的苦恼大大超过后来同他的交谈所带来的麻烦,虽然我的母亲认为那些交谈都是有害的胡言。我们家本来还可以接待他的,但有一次饭后,他斩钉截铁地向我保证,他曾经听到人家确凿无疑地说到我的姑祖母年轻时是位风流女子,曾公开接受过人家的供养,正如他不久前对我所说,女人心目中只有爱情,谁都一样,她们尽管推拒,最终没有一个是攻不破的——这一信息后来对我的生活产生很大的影响,先是使我过得更加幸福,后来又让我落到更加不幸的地步。我忍不住把他的话都告诉了我的长辈,从此他们把他拒之门外,后来我在街上向他打招呼,他对我冷淡至极。
\par 但是,关于贝戈特,他的话倒一点不假。
\par 开头几天,作者的字里行间使我应该爱不释手的东西并没有浮现在我的眼前,就像一首乐曲,你听得只顾心醉神迷,还来不及品出妙处。我读的那本小说,虽已经同我难分难舍,但我误以为这兴趣只是由故事引起的,正如爱恋之初你天天赶到某处某个娱乐场所去消遣,去会见那个女人,你当时还以为只是娱乐本身吸引你呢。后来,我注意到贝戈特在一些地方爱用难得见到的、简直是古意盎然的词句,那几处形成一股和谐的暗流,一段含蓄的引子,从而使他的文风高雅起来;而且就在那些地方,他谈到了“人生空幻的梦”,“美丽的形态流溢出滔滔不绝的激流”,“知心和依恋的折磨如何空泛徒劳而又甜蜜消魂”,“震撼人心的塑像如何把教堂的外观点缀得格外崇高”。他用美妙动人的形象来表达一种对我来说全然新颖的哲理,那些形象可以说激起了竖琴的齐鸣,在悠悠乐声的烘托下,形象更显得崇高。在贝戈特的那些段落中,有一段我抽出来细细玩味,那是第三段或第四段吧,它所给予我的愉快同我在读第一段时大不一样,那种愉快我在内心深处更统一、更广阔,因而是一切障碍一切隔阂仿佛都已排除掉的那个部位所感受到的。因为——其实在开头几段引起我兴趣的,也正是他这种在遣字造句上唯求生僻的偏爱,这种回荡着悠悠乐声的音韵,这种唯心主义的哲理,只是我当时没有意识到而已——我一旦认出这些东西,我仿佛感到自己不再只是在读贝戈特的某一本书的某一个别段落,浮现在我思想表面的也不是一个纯属平面的形象了,而是一个“理想段落”,跟贝戈特的其他著作有着共同的特点,而仿佛同这个理想段落难以区分的其他类似的段落,一起形成一种厚度,一种体积,使我的心智也得以扩展。
\par 不只是我一个人崇拜贝戈特;我的母亲的一位女朋友很有学问,也偏爱贝戈特的作品;还有迪·布尔邦大夫,为了读完贝戈特的一本新作,不惜让病人在一边等待;贝戈特作品的风靡的种子是从迪·布尔邦大夫的候诊室、贡布雷市镇附近的一家花园中飞散开来的;当时还只是稀有的品种,今天已经风靡全球,欧洲、美洲,乃至于穷乡小村,到处都见得到这枝理想的、共同的花朵。我的母亲的女朋友,据说还有迪·布尔邦大夫,对贝戈特的著作中最为欣赏的东西,跟我之所好相同,那就是他字里行间那种行云流水般的旋律感,那些古意盎然的词句,还有一些尽管很简朴、很常用的短语,但是,他把它们放在显要的地位,从而仿佛有意表示出对它们的特殊的偏爱;总之,在哀怨的行文中,插进一两个唐突的字眼儿,一种粗声粗气的语调,不用说,他本人也一定感到自己最感人的魅力正在于此。因为,在他后来的几本书中,倘若赶上什么重要的真人真事,或者提到某一座著名教堂,他就中断叙述,插入祈求、呼号和滔滔不绝的祷告,让一股股这类的气息充分地得到发泄;而在他早期的著作中,这类气息始终是内在的,只由于表面的波动才泄露出一二分来;也正因为是半隐半现的,或许更柔美,更和谐,但毕竟人们无法确切地指出这一股股窃窃私语的气息是从哪里流出来的。作者得意之处也正是读者激赏之时。我对那几段文字能背得滚瓜烂熟。当作者重新拾起叙述的脉络时,我还感到扫兴呢。有些东西的内在的美,我一直还看不透,例如松林,霰雪,巴黎圣母院,《阿达莉》或《淮德拉》,他每当讲到这些,他都绘色绘声地以形象来引爆那种美,来打动我的心扉。所以我感到:宇宙之大,区区感官岂能得窥全豹,倘若没有他的引领,天地间有多少方面是我的残弱的感知所无从分辨的啊!我倒真希望听听他对于万物的见解,哪怕一种隐喻也罢,尤其是对于那些我或许有机会见到的东西,特别是法国的古建筑和某些滨海地区的风物,因为他在他的好几本书中一再提到它们,足见他认为这些事物中蕴藏着丰富的意味和丰富的美。可惜,他几乎对一切事物都讳莫如深地不予评述。我不怀疑,他的见解一定同我的见解完全不同,因为它来自我正设法攀登上去的那个陌生的世界。我坚信,我的种种想法在那位绝顶聪明的智者看来,纯属冥顽不灵,所以我干脆统统推翻。可是有一天我偶尔在他的一本书中发现了我过去也曾有过的想法,我的心一下子膨胀起来,简直好似有哪位天神大发慈悲,把那个想法归还给我,并宣布它是合情合理的、优美的。有时候,他书中某一页写的话,同我在失眠时夜里写给我的外祖母和母亲的信中意思完全一样,贝戈特的那页文字仿佛是放在我的那些信头上的提要汇编,甚至后来我自己开始著书的时候,有些句子我总觉得不够精当,下不了继续写的决心,我就从贝戈特的书里去寻找等同的写法。只有在他的作品中找到之后我才会感到高兴。等到我自己营字造句,一心想让行文恰如其分地反映出我的思想捕捉到的内容,同时又担心“落入窠臼”的时候,我才不着急呢!我细细掂量写的东西究竟是不是尽如人意。但实际上,我真正钟爱的,只是这类短语、这类观念。我搜索枯肠、永不满足的努力,本身标志着一种爱,一种没有欢乐、却很深沉的爱。所以,当我在另一位作者的著作中突然发现同样的短语,也就是说,当我们不必自己去字斟句酌,为一丝不苟而搔首踟蹰时,我才终于能痛快地品尝到其中的滋味,好比一名厨子,偶尔有一回不下厨,总算有暇尝尝美味佳肴。有一天,我在贝戈特的一本书中,读到一段挖苦老女仆的笑话,出自大手笔的庄重的语言,使讽刺的意味格外入木三分,我跟我的外祖母谈到弗朗索瓦丝时也常常说过这样的挖苦话;还有一次,我发现贝戈特并不认为在反映真实的作品中写进类似我曾有机会对我们的朋友勒格朗丹先生所作的评述会有伤大雅(对弗朗索瓦丝和勒格朗丹先生的评述是我最无顾忌地供奉给贝戈特的祭品,相信他一定会觉得兴味索然的),于是我突然感到,我的平庸的生活同真实的王国之间,并不像我过去所设想,隔着什么鸿沟,它们甚至在好几点上相互交叉,我有了信心,高兴得像伏在久别重逢的父亲怀里似的伏在书上哭起来。
\par 根据贝戈特的著作,我想象他是位病弱失意的老人,丧子之痛始终未平。因此我读他的散文,心中默默唱诵,也许唱得比文字本身更柔更慢,最简单的用语到我的嘴里也具有一种哀怨的调门。我最喜爱的,是他的哲理,我誓将终生奉行。它使我焦急地盼望早日达到上中学的年龄,好进哲学班上课。但是我只希望学校里时时处处只按贝戈特的思想行事。要是那时就有人对我说,我现在所倾心的思辨大师们跟贝戈特毫无共同之处,我会感到绝望的,正如一位堕入情网的人,本打算终生不变心地只爱一人,人家却预言他将来会另有几位情妇。
\par 有一个礼拜天,我正在园中读书,被斯万的来访打断。
\par “你读什么呢,能给我看看吗?哟,贝戈特写的?谁跟你提到他的作品的?”
\par 我告诉他:是布洛克。
\par “啊,对了,我有一次在这里见到过这个男孩子,他长得跟贝里尼画的穆罕默德二世一模一样。哦,像极了,同样是弧形的眉毛,弯曲的鼻梁和隆起的颧骨。等他长出两撇小胡子之后,那就是穆罕默德二世了。不管怎么说,他倒还有些鉴赏力,因为贝戈特是位很优雅的聪明人。”从来不提起自己的熟人的斯万,发觉我对贝戈特如此钦佩,便出于好心,为我破了一次例,说道:
\par “我跟他很熟,要是让他在你的书的扉页上写点什么能使你高兴的话,我倒是可以为你请他题词的。”
\par 我不敢接受他的好意,只是问了斯万好些有关贝戈特的问题:“您能告诉我他最喜欢哪位演员吗?”
\par “演员嘛,我不知道,但我知道他认为男演员里面没有人能同拉贝玛相提并论。他认为拉贝玛比谁都高出一筹。你看过她演的戏吗?”
\par “没有,先生。我的父母不让我去剧院看戏。”
\par “可惜。你应该要求他们允许你去呀。拉贝玛在《淮德拉》和《熙德》这两出戏里,可以说只不过是名女演员,但是,你知道,我一向不大相信艺术有什么‘高低之分’。”(我发现——而且过去他同我的两位姨祖母交谈时,这种表现已多次让我深感诧异——他每当谈及严肃的事情,用到某种说法,仿佛就某一重要问题提出某种见解时,总要用特别的、一字一顿的语调,挖苦似的把那种说法孤立开来,好像给它加上引号似的。这次提到“高低之分”,大有“正如荒唐的人所说”的意味。其实,既然荒唐,他又何必说呢?)他停顿片刻之后,又补充了一句:“像她最近演的那出戏,高雅的程度,赶得上任何一部传世杰作。我对此并不在行……我说的是……”他呵呵一笑,“例如《沙特尔的王后们》这出戏!”至此,我觉得,他这种害怕认真表达自己见解的态度,大约是高雅的表示,是巴黎派头,跟我的姨外婆们的不见世面的死心眼儿大相径庭;同时我还怀疑,这或许是斯万的生活圈子里的那伙人的一种思想的形式,他们对过去几辈人的抒情感叹有意来个反动,过分推崇一向受人鄙视的细节,乃至于否定一切“陈词滥调”。现在,我觉得斯万对待事情的态度有点让人感到难堪。他显然不想说出自己的见解,他只在能够提供细节的时候才侃侃而谈。但是,他难道不知道要求所提供的细节具有一定的意义不正等于宣扬某种见解吗?我又想到了那天晚上,我吃晚饭的时候心情很压抑,因为有客,妈妈不能上楼来吻我,说声晚安了;就在那天晚饭的餐桌上,斯万说,莱翁王妃家的舞会他并不放在心上。可是他成年累月偏偏都消磨在那样的吃喝玩乐中。我觉得这一切难以自圆其说。莫非他还保留着另一种生活,能最终正正经经地说出自己对一些事情的看法,不必打上引号地作出自己的判断,不必彬彬有礼地投身于他同时又称之为可笑的活动?我还注意到斯万同我谈论贝戈特的时候,语气中没有他惯有的特点,相反,同贝戈特的其他崇拜者,例如我母亲的那位女朋友,还有迪·布尔邦大夫的语气完全一样。他们提到贝戈特,同斯万一样,也说:“这人优雅而聪明,很有特点,有自己的一套叙述方法,有点过于讲究,但亲切宜人。看到他写的东西,不必看作者的署名,便能马上认出是他的作品。”但是谁也不会进而说:“他是位伟大的作家,才华横溢。”他们甚至不会说他有才气。他们之所以不这么说是因为他们心中无数。一位新作家的外观,明明同我们包罗万象的观念中标上“大才子”称号的模式完全吻合,我们却总是迟迟认不出来。恰恰是因为他的那副面貌是新的,我们才觉察不到他同我们心目中的“才华”完全相符。我们宁可说他独创、优雅、精致、豪放;最终有一天,我们才认识到这一切恰恰就是才华。
\par “贝戈特的作品中,有谈到拉贝玛的吗?”我问斯万先生。
\par “我想他在论拉辛的那本小册子中谈到过,不过大约早已售完。可能后来又重印过一回。我打听打听。况且你要什么,我都可以向贝戈特提,一年当中他没有一个星期不到我家来吃饭的。他是我女儿的好朋友。他们一起去参观历史古城、教堂、宫堡。”











\subparagraph*{2}

\par 因为我对于社会地位的高低毫无概念,所以长久以来,我的父亲认为我们不可能拜访斯万夫人和斯万小姐,我还因此而想象她们同我们隔得太远,反倒使她们在我的心目中增添了威望。我惋惜我的母亲不像斯万夫人那样染头发,抹口红,因为我听我们的邻居萨士拉夫人说过,斯万夫人这样做,倒并不是为了讨丈夫的喜欢,而是为了取悦于德·夏吕斯先生;我当时认为,我们在她的眼里,一定是不屑一顾的俗物;我之所以这样想,多半还因为听人说过,斯万小姐是位非常漂亮的姑娘。我常常梦见她,每次都把她设想成既骄纵任性又委婉动人。直到那天我才知道,原来她的地位如此难得,她享有那么多的特权却习以为常,当她问她的父母谁来吃晚饭的时候,她所得到的回答竟是那样高贵的客人的字字铿锵、金光闪闪的大名——贝戈特!那样的贵客对她来说只是家里的一位老朋友。我在餐桌上所能听到的只是姑祖母的议论,而与此相应的亲密的谈话,对她来说,却是贝戈特诉说自己书中没有论及的各种问题。我真恨不能亲聆他的高见呀!临了,她一旦要去参观什么古城,贝戈特总像下凡的神仙,载誉载辉地陪伴在斯万小姐的身边,虽说俗人不认识他。于是我感到跟她相比我显得多么粗俗无知,而她那样活着才多有价值。我强烈地体会到若能成为她的朋友该有多美,而这对于我来说又多不可能;因此我在满怀期望的同时又充满绝望。现在我一想到她,常常若有所见地看到她站在教堂前面,为我讲解塑像的意义,而且还面带对我嘉许的微笑,把我作为她的朋友介绍给贝戈特。各地大教堂在我的胸中引发出的种种优美的思绪,法兰西岛起伏的丘陵和诺曼第省坦荡的平原的妖娆风光,都以自己美丽的风采反射到我所构思的斯万小姐的形象上来:我真是一心只求爱上她了。为了产生爱情,必须有许多条件,其中最必不可少也最不费周折的要求,就是相信爱情能使我们进入一种陌生的生活,成为其中的一部分,即使自称以貌取人的妇女,也能在她所看中的那个男人的身上,发现一种特殊生活的气息。所以她们爱军人,爱救火队员,因为他们的制服使他们的外貌显得更可亲些;女士们认为在盔甲之下能吻到一颗与众不同、勇于冒险、侠骨柔肠的心;一位少年君主,年轻的王储,并不需要有端正的相貌,却能在他所访问的国度赢得最令人羡慕的艳福,而对于一位普通的情场老手来说,五官端正也许是必不可少的条件。
\par 我礼拜天在花园里读书,我的姑祖母是无法理解的,一星期七天,唯独那天是不准做任何正经营生的,所以她不做针线(平时,她又会对我说:“怎么,你又在看书消遣了,今天又不是星期天。”她给“消遣”这个字眼,加进了“孩子气”和“浪费时间”的含义)。我在读书的当口,我的姨妈莱奥妮正一面同弗朗索瓦丝聊天,一面等待欧拉莉来访。姨妈告诉弗朗索瓦丝说,她刚才看见古比尔太太走过,“没有带雨伞,穿的是那身从前在夏多丹做的丝绸长裙。倘若黄昏前她还有不少路要走的话,那身裙子恐怕要挨雨淋了。”
\par “可能吧,可能吧(意思是不见得吧)。”弗朗索瓦丝说,以免断然排除天色好转的可能性。
\par “你看,”姨妈拍了拍脑袋,说,“这倒提醒了我:我还没有打听到她是不是在领圣体之后才赶到教堂的呢。呆会儿我得问问欧拉莉……弗朗索瓦丝,你看:这钟楼后面的那团乌云,瓦片上的那点阴阳怪气的阳光,肯定天黑之前要下场雨,不可能就这样下去,天气太闷热了。雨下得越早越好,因为只要暴雨不来,我喝下去的维希圣水也就堵在胸口难以消化。”我的姨妈最后又补充这么一句。总的说来,她巴望维希圣水早早消化的急切心情大大超过唯恐古比尔夫人裙子淋湿的担心。
\par “可能吧,可能吧。”
\par “你知道,广场上要是下起雨来,可是没有什么地方好躲避的。怎么,都三点钟了?”我的姨妈脸色发白,突然叫出声来,“这么说,晚祷都开始了,我居然忘了服用蛋白酶!我现在才明白,怪不得维希圣水堵在胸口下不去呢。”说着,她急忙扑过去抓起一本紫丝绒封面、切口烫金的祈祷书,匆忙间把夹在书里标出节日祷文那几页的几张镶有发黄的纸花边的书签掉了出来。我的姨妈一面咽下蛋白酶,一面开始以最快的速度诵读经文,对其含义她多少有点糊涂了,因为她心神不定,不知道服用维希圣水之后,隔了那么久才服用蛋白酶,还能不能赶上药力,让圣水早早消化。“都三点钟了,时间过得真快,简直不可思议!”
\par 窗户上好像有什么东西碰了一下,接着又像有人从楼上的窗子里撒下一把沙子,簌簌地往下落,后来这落下的声音扩散开去,规整得有板有眼,变成了潺潺的水声,琤琤淙淙地响起来,像音乐一般,散成无数小点,到处盖满:下雨了。
\par “瞧!弗朗索瓦丝,我怎么说来着?下了!我觉得好像花园的门铃儿响了,快去看看这种时候能有谁来?”
\par 弗朗索瓦丝回来说:
\par “是阿梅代夫人(我的外祖母)弄响的门铃儿,她说她要出去散散步,雨可是下得很大。”
\par “我并不感到意外,”我的姨妈两眼朝上一翻,说道,“我一直说,她的精神跟大家不一样。在这样的时候,我倒希望往外跑的是我,而不是她。”
\par “阿梅代夫人总是同别人截然相反。”弗朗索瓦丝客气地说,算是留点余地,以便单独跟别的用人在一起的时候,好说她认为我的外祖母有点“神经病”。
\par “没有盼头了!欧拉莉不会来了,”我的姨妈叹息说,“准是这天气把她吓住了。”
\par “可是还不到五点钟呢,奥克达夫夫人,现在才四点半。”
\par “才四点半?居然已经需要撩起小窗帘让外面透点亮光进来。四点半就这样!现在离升天节只有八天了!啊,可怜的弗朗索瓦丝!准是善良的上帝生咱们的气呢。当今世人的作为也太过分了。就像我可怜的奥克达夫当年所说的那样,人们太不把上帝放在心上,上帝要报复的。”
\par 一片鲜艳的红润使我的姨妈的面容生动起来:欧拉莉来了。不巧的是,她刚进屋,弗朗索瓦丝也就跟着回来了。只见她满脸堆起微笑,目的在于主动地配合,以求同我的姨妈必定会有的喜悦取得一致,因为她有十分的把握,相信她要说的话必定让姨妈听了高兴。她一字一顿地说着,以此表明:她虽然使用间接语气,但是作为忠于职守的女仆,她说的只是转述来客的原话:
\par “要是奥克达夫夫人没有在休息,可以接见神甫先生,他将感到不胜荣幸。神甫先生不想有所打扰。神甫先生就在楼下,是我让他进客厅等候的。”
\par 事实上,神甫先生的访问并不像弗朗索瓦丝所设想的那样,能让我的姨妈感到有多高兴。她每当通报神甫来访,总认为脸上应堆起可掬的笑容才是,殊不知这副欢天喜地的模样同病人的心情并不完全合拍。神甫(是个好人,我一直可惜没有同他多谈,因为他虽不懂艺术,却精通词源学)惯于向参观教堂的贵客提供有关教堂的史料轶事(他甚至想写一本书介绍贡布雷教区的掌故),他总要没完没了地向姨妈作千篇一律的讲解,听得她又烦又累。当他的来访碰巧同欧拉莉赶在一起,我的姨妈干脆觉得他来得不是时候,很不知趣了。姨妈宁可多多利用欧拉莉的情报,却不喜欢同时来一大堆人。但她不敢不接见神甫;她只是向欧拉莉使个眼色,要她别同神甫一起走,等神甫走了之后,再呆一会儿。
\par “神甫先生,我听人怎么说来着,说有名画家在你们教堂里支上画架,临摹彩绘玻璃窗。可以说我活了这一大把年纪还从来没有听说过这类稀罕事儿!现在的世道人心都在想些什么!教堂里还有比这更可恶的事吗?”
\par “我倒不至于说这事有多可恶,因为圣伊莱尔好些地方值得参观;我的那座破落的大殿好些地方已老得不成样子,整个主教区里就只有我那座教堂没有翻修。天晓得我们的门廊有多脏,有多古老,但毕竟具有一种庄重的品格;至于说到那几块描写爱丝苔尔故事的壁毯,我个人认为不值两三文钱,可是识货的人一眼就看出,它们比森斯教堂的壁毯更有价值。此外,我承认,那几幅壁毯画除了某些细节很有写实风格之外,另一些细节还表现出一种真正的观察力。至于彩绘玻璃窗,那倒不提为好!难道在地面七高八低的教堂里保留那些透不进阳光的窗户,只让我都说不上是什么颜色的反光来弄花人们的眼睛是明智的吗?他们就是不肯换掉高低不平的石板,说是因为那里面埋葬着贡布雷历代神甫和布拉邦特历代君主——盖尔芒特家的爵爷们,也就是今天的盖尔芒特公爵和公爵夫人的直系祖先,因为公爵夫人本来就是盖尔芒特家的小姐,后来嫁给了她的堂兄。(我的外祖母一向不在乎人家的姓氏出身,结果弄得张冠李戴。每当听到德·盖尔芒特公爵夫人的名字,她总以为准是德·维尔巴里西斯夫人的亲戚,引得大家哄堂大笑,于是她引用一封请柬上的话来为自己辩护,说:“我仿佛记得帖子上有盖尔芒特这几个字来着。”有一回,我跟大伙儿一起反对她,因为我不能同意她当年的那位同寝室的朋友跟热纳维耶夫·德·布拉邦特公主的后代能有什么血缘关系。)您再看看鲁森维尔,如今只成了村落,而在古代,那地方因毡帽交易和钟表生意十分兴隆而曾经繁华一时。(我对鲁森维尔这一地名的由来没有把握。我主观地认为它本名鲁维尔,Radulfi villa“红城”,同夏多鲁的词源——Castrum Radulfi“红堡”相仿。但这是后话,以后再说。)现在把话说回来,那儿的教堂倒有非常华丽的彩绘玻璃窗,几乎全都是新的。那幅气宇不凡的《路易菲利浦幸驾贡布雷》,其实应该装在贡布雷教堂的窗户上才更为合适。有人说,那幅巨作赶得上鼎鼎大名的沙特尔大教堂的彩绘大窗。就在昨天,我还见到过贝斯比埃大夫的兄弟,他是这方面的行家,他认为那是幅上等精品。我问过那位艺术家,他看来倒很讲礼貌,而且据说作起画来着实得心应手、游刃有余。我问他:“这面玻璃窗明明比别的玻璃窗更暗淡,您又觉得它了不起在哪里呢?”
\par “我相信,只要您向主教大人提出要求,他不会拒绝给您换一面新窗的。”我的姨妈有气无力地说道,她已经开始想到自己马上就会感到累了。
\par “亏您还指望他呢,奥克达夫夫人,”神甫答道,“就是主教大人专为那面倒霉的玻璃窗说好话;他考证下来,窗上画的是热纳维耶夫·德·布拉邦特的直系子孙、盖尔芒特家的一位人称坏家伙希尔贝的爵爷,正得到圣伊莱尔降恩赦罪。热纳维耶夫·德·布拉邦特原本是盖尔芒特家的千金。”
\par “可是,我怎么不知道画里面有圣伊莱尔呢?”
\par “怎么没有?在彩窗的角上,您没有注意到有个穿黄色长裙的贵妇人吗?哎!她就是圣伊莱尔,您知道,在有些省份,人们称她为圣伊里埃,圣埃里埃,在汝拉省,还有人叫她圣伊里呢。那些得道的古人的名字,往往以讹传讹,出现好几种叫法,圣伊拉里乌斯这个名字衍生出来的这个大大走了样儿的称呼,还不算最出格的呢。好心的欧拉莉呀,就拿您的保护神圣欧拉莉亚来说吧,您知道她在勃艮第被人称呼什么?他们干脆叫她圣埃洛亚。女圣人变成了男圣人。您看见没有?等您死后,人家就会把您说成是男人。”
\par “神甫先生总有词儿来挖苦人。”
\par “希尔贝的哥哥结巴查理当年是虔诚的王子,他们的父亲疯子丕平接连发过几次精神病之后死了,那时查理还年轻。他年少气盛,掌管了至尊的权柄,心目中毫无法度,倘若他在什么地方,看到有谁的长相不合他的心意,他就下令把那个地方的男女老少统统杀尽。希尔贝为了对查理进行报复,放火烧掉了贡布雷的教堂,也就是原先的那座教堂;当年西奥德贝\footnote{西奥德贝(511—558):法国古代“东王国”国王,又称梯贝尔一世。}率领他的扈从廷臣离开他的乡间行宫(离此地不远,在梯贝齐,拉丁文叫西奥德贝齐阿喀斯),前去攻打勃艮第人之时,在圣伊莱尔的墓上发誓,倘若圣人在天之灵保佑他旗开得胜,日后他定将在这里建立一座教堂。原先的那座教堂就是这样建成的。希尔贝的一把火,把原来的教堂烧得只剩地下神殿,想必戴奥多尔领你们下去看过。后来希尔贝借助征服者威廉\footnote{征服者威廉(1027—1087):英国国王兼诺曼第大公。}(神甫念成纪洛姆)的兵力,击败了倒霉的查理,所以有不少英国人来这儿参观,但是希尔贝似乎不善于赢得贡布雷的民心,因为有一次他做完弥撒,刚走出教堂,贡布雷的百姓一拥而上,砍了他的脑袋。其他细节在戴奥多尔借给大家看的那本小册子里都有说明。
\par “但是,毋庸争辩,我们教堂里最为奇特的,是从钟楼顶上往四下看到的景色,非常壮观。当然,你们身体都不很结实,我不劝你们攀登钟楼里的九十七级台阶,其实,那只及著名的米兰大教堂的钟楼梯级的半数。不过,即使身体很结实的人,爬起来也够吃力的,尤其是想要不磕脑袋就得弯着腰走,而且一路上还得拿手里的东西去拨开蜘蛛网。总而言之,您得穿得厚实些,”他又补充了一句说(他没有发觉:他竟设想我的姨妈能去爬钟楼,这种想法引起她多大的气愤),“因为一到钟楼上面,穿堂风大极了!有人甚至感到透心凉,说简直觉得自己像死了一样。那也没关系,星期天照常总有一帮一帮的人,有的甚至从很远的地方来,登上钟楼欣赏极目远眺的美景,乘兴而来,如醉如痴而归。瞧着吧,下星期天要是天气不变,您在钟楼上准能见到人头挤挤挨挨的,因为那时正赶上升天节。说实话,从那上面俯瞰大地,真有飘飘欲仙之感,纵览八极,别有一番滋味。每逢天气晴和之日,您可以一直看到维尔诺叶。平时只能顾此失彼看到的这部分、那部分风景,届时都能尽收眼底了。例如维福纳河、同贡布雷比邻的圣达西兹的大沟小壑,以及横在它们之间的林木的屏障,还有舒子爵市(您也知道,古时候叫乌迪亚喀斯子爵市)的纵横的运河,都能一览无余。我每次去舒子爵市,都只能看到运河的一段,我转过一条街,就看到运河的另一段,而刚才的那一段就不见了。我虽然在脑子里想把两段运河联在一起,却收效不大。从圣伊莱尔钟楼望去,却是另一番景象。整片河网呈现在眼前,只是运河里的水看不出来,仿佛几道大缝把市镇切成几块,就像已经切开的面包似的,一块块虽仍挨在一起,但彼此都已分开。最好是您能分身有术,既在圣伊莱尔钟楼上,同时又置身于舒子爵市。”
\par 神甫的喋喋不休,使我的姨妈累得难以支撑,以至于他刚刚告辞,我的姨妈只好把欧拉莉也随即打发走了。
\par “听我说,可怜的欧拉莉,”她声音微弱地说着,同时伸手拿过钱包,掏出一枚硬币,“您祈祷的时候别忘了我。”
\par “哟!奥克达夫夫人,我真不知道该怎么办才好了,您是知道的,我又不是为了这个才来看您的!”欧拉莉不无埋怨地说道。她每次都跟头一回似的,总显得那么为难,那么尴尬,还挺不乐意,这使我的姨妈觉得好笑,但她并不因此而感到扫兴,因为,倘若有一天,欧拉莉不像平时那样显得无可奈何似的收下她塞过去的硬币,我的姨妈就会说:
\par “真不知道欧拉莉今天怎么啦。我今天并没有少给,她怎么不高兴?”
\par “我认为她没有什么不满足的。”弗朗索瓦丝叹了口气说。我的姨妈无论送给她和她的孩子什么东西,她都看做是不足挂齿的小费,而我的姨妈每星期天悄悄塞到欧拉莉这样不识抬举之辈手中、小得连弗朗索瓦丝看都无法看到的一点东西,弗朗索瓦丝都认为是把宝贝任意挥霍。她倒并不希望我的姨妈把赏给欧拉莉的钱赏给她。她但愿我的姨妈能把钱自己留着就行了,因为她知道主人若有钱,仆人在别人心目中的地位也高些,显得光彩。她,弗朗索瓦丝,在贡布雷、在舒子爵市以及在别的地方之所以大名鼎鼎、面上有光,皆因为我的姨妈拥有许许多多的农庄,本堂神甫又经常来访,而且一来就聊上半天,再加上我的姨妈平时饮用维希泉水的瓶数在这一带可算首屈一指。弗朗索瓦丝精打细算,都只为我的姨妈着想;她若经管这份产业(这恐怕是她梦寐以求的美差),她就会像母亲一样地不讲情面,不许外人染指,保管好家当。她知道我的姨妈手松得不可救药,动不动就给人东西;要是给有钱人送礼,倒也罢了,她还不至于认为算得上什么大错,也许她想,有钱人并不稀罕我姨妈的礼物,他们决没有因为受了礼才待她好的嫌疑。况且给萨士拉夫人、斯万先生、勒格朗丹先生、古比尔夫人,以及其他地位同我的姨妈相当,彼此又“很合得来”的殷实富户送礼,她认为这本来就是富人们光彩奕奕、与众不同的生活中司空见惯的规矩;他们打猎,举行舞会,彼此串门做客,她都笑吟吟地打心眼儿里钦佩。但是,如果我的姨妈的慷慨的受益者,不过是弗朗索瓦丝称之为“同我一样、甚至还不如我”的人,是那些她最瞧不起,而且不称她为“弗朗索瓦丝太太”,不承认自己“不如她”的人,那就另当别论了。每当她看到我的姨妈不顾她的劝告一意孤行地把钱白扔给(至少她这么认为)那些受之有愧的下人,她就觉得我的姨妈待她未免太薄,跟她想象中欧拉莉所得到的大笔大笔好处相比,主人给她的东西也太少了。据她设想,欧拉莉单凭每次来访所得到的赏钱,若想置份家当,贡布雷附近没有一处庄园她不能轻易买下的。事实上,欧拉莉对弗朗索瓦丝的巨额私房钱也作了同样的估计。平常欧拉莉一走,弗朗索瓦丝就不怀好意地估算她的赏钱总数。她既恨她又怕她。她在时,她认为自己不能不陪“笑脸”。她一走,她便立即恢复常态。的确,那时她决不直呼其名提到她,而是嚷着说些古代女预言家“箴言录”\footnote{女预言家的“箴言录”相传成书于公元六世纪,集录了流传于世的古代女预言家的预言。}里的话,或者引用具有普遍意义的格言,例如《圣经》传道书里的格言,其用意我的姨妈一听就明白。弗朗索瓦丝从窗帘边上往外看了看欧拉莉是否已经关上园门之后,说道:“溜须拍马的人总有办法上门捡便宜,等着瞧吧,上帝早晚有一天会惩罚他们的。”说着,她斜眼一望,就像一心为阿达莉着想的若阿斯在含沙射影地说:
\refdocument{
    \par 恶人的幸福像湍流,转眼即逝。\footnote{引自拉辛悲剧《阿达莉》。}
}
\par 但是,神甫也来凑热闹,在没完没了的絮叨把我的姨妈精力耗尽之后,弗朗索瓦丝随欧拉莉走出房门,说道:
\par “奥克达夫夫人,我也走了,您好好休息,您看上去很累。”
\par 我的姨妈没有回答,只舒了一口气,简直像吐完最后一口气似的阖上了眼睛。可是,弗朗索瓦丝刚刚下楼,便听到激烈的铃声四响,传遍全屋。我的姨妈在床上坐了起来,大声喊道:
\par “欧拉莉走了没有?你看我都忘了问问她,古比尔夫人是不是在弥撒献祭之前就赶到了教堂?你快去追她!”
\par 弗朗索瓦丝没有撵上欧拉莉,独自回来了。
\par “这真是太扫兴了,”我的姨妈连连摇头,说道,“就这件事儿最重要,我偏偏没有问!”
\par 莱奥妮姨妈的生活就这样日复一日地度过,天天如此;她装做轻蔑、其实很深情地把这种日子称之为“我的小日子”。她一天天过得那样温暖、那样单调。大家都在为她小心翼翼地保护这种“小日子”,不仅家里的人感到无法劝她采取更好的养生法,只好听其自然,尊重她的这套生活方式;即使在镇上,离我们家足有三条街远的包装工,在钉箱子之前,也得问问弗朗索瓦丝我的姨妈那时是不是正在“休息”。然而,这种常规生活那年却受到了一次骚扰,就像一颗长在暗处的果实,尽管无人理睬,却自发地生长,直到果熟蒂落。事情是这样的:帮厨女工有一天晚上突然临产,她疼得难以忍受,而贡布雷镇上偏偏没有接生婆,弗朗索瓦丝只得天没亮就赶到梯贝齐去请接生婆。帮厨女工大声叫疼,我的姨妈因而不得休息,去梯贝齐的弗朗索瓦丝尽管路程不长,却很晚才回来,我的姨妈惦记得要命。所以我的妈妈一早就对我说:“上楼去看看你姨妈,看她需要什么?”我走进外间,从开着的门往里间看,看到我的姨妈侧卧着,睡得正香;我听到她的轻轻的鼾声。我正打算蹑手蹑脚地走开,可是,一定是我弄出的声响闯入了她的睡乡,用开汽车的行话说,“变挡了”,因为鼾声忽然停顿了一秒钟,尔后又以低一点的调门继续呼噜不息;最后她醒了,侧过脸来,让我看到了她的表情。她脸上有一种恐怖的神色,显然她刚做了一个噩梦;她处的那个位置没法看到我,我也呆立在原地不知道该往前走还是往后退;但她显然已经恢复现实感,认识到刚才吓坏了她的幻觉实际上是假的;她莞尔一笑,表示高兴,也表示对上帝的由衷感激,因为多亏上帝,实际生活才不如梦那样残酷。这一笑使她的脸上掠过一丝光芒;她以为只有她一个人在场的时候,她习惯于自言自语;这时她悄声说道:“谢天谢地!除了临盆的帮厨女工吵闹以外,倒还没有别的烦心事儿。可不是吗?我梦见我的奥克达夫复活了,而且他要我天天散步!”她伸手想去抓桌上的念珠,但是睡意再次袭来,使她无力够到念珠:她又安心地睡着了。我轻步走出房去,无论她或是别人,谁都不知道我刚才听到了什么。
\par 当我说,除了像有人生孩子之类难得遇上的事情之外,一般没有别的变动打乱我姨妈的生活,其实我还没有述及她单调的生活中每隔一定时间总要反复出现另一种单调的变化,那就是每星期六,由于弗朗索瓦丝总要在下午去鲁森维尔的集市采购东西,所以午饭时间就提前一小时。我姨妈的生活每周一次受到这样的破坏,她已经习以为常,结果她比别人更离不开这种变化,用弗朗索瓦丝的话来说,她已经“习惯成自然”,甚至如果哪个星期六按平常时间开饭,她反而觉得“乱了套”,非得用另一天提前开饭作为补偿。对于我们大家来说,星期六提前吃饭则另有特殊的意义,我们觉得这样更随和、更可心。在离平时开饭还差一小时的时候,我们心想,再过几秒钟天香菜便可提前上桌,还能享用到格外开恩的摊鸡蛋和受之不当的炖牛肉。星期六的这种不对称的轮回成了一桩内政性、地方性、甚至全民性的小事件,它在平静的生活和闭塞的社会中,造成一种民族联系,由谈话、说笑以及有意夸张其辞的传说提供热门的主题:如果我们有谁具备史诗头脑,这个主题就能化为一系列传奇故事的核心。人们一早起床,还没有穿戴齐全,就开始无缘无故地感到一股团结的力量而精神抖擞起来,彼此和颜悦色地、诚恳地怀着乡土感情说道:“赶紧,别忘了今儿是星期六!”而我的姨妈甚至认为这一天比平常日子要长,她跟弗朗索瓦丝商量:“是不是给他们炖一块小牛肉?因为今天是星期六。”倘若哪位粗心大意的人,在十点半钟的时候掏出怀表一看,随口说:“还有一个半小时开饭。”那么,人人都会乐于告诉他:“怎么?您想什么呢?别忘了今儿是星期六!”直到一刻钟之后,当人们想到他竟如此粗心,还止不住会大笑一阵的,而且忘不了上楼去告诉我的姨妈,让她也开开心。那天连天空也改变了模样。午饭之后,意识到今天是星期六的太阳在天上多游逛了一小时。如果有谁一下想到早该出门散步,忽听得圣伊莱尔的钟声才响两下,不禁纳罕:“怎么?才两点钟!”(平日,两响的钟声在白茫茫的、细波粼粼的河边是见不到人影的,因为那时有人午饭还没有吃罢,有人午眠正酣,路上人迹罕至,连垂钓的人都离开了河岸,只有寂寞的钟声孤单单地驰过仅留剩几片懒云还没有离去的空阔的天边。)这时大家都会异口同声地对他说:“您所以产生错觉,是因为午饭提前了一小时。您知道,今天是星期六!”有一回,有个蛮子(凡不知道星期六特殊的人我们统称为蛮子)十一点钟来找我的父亲,见我们已上餐桌,大为惊讶,这于是成为弗朗索瓦丝一生中最开心的事情之一。发窘的来客不知道我们星期六提前开午饭的原因,固然为弗朗索瓦丝提供了笑柄,但她觉得更滑稽的是我的父亲的回答(当然,她充满了狭隘的地方观念);我的父亲居然没有想到那个蛮子可能不知内情,见他如此惊讶,竟没有向他作解释,只说:“您想嘛,今天是星期六!”弗朗索瓦丝每次讲到这里总忍不住笑出了眼泪。为了更加凑趣,她还添枝加叶胡编了好些那位不知星期六奥秘的来客的对答。我们不仅不拆穿她,反而觉得她编派得不够,对她说:“客人似乎还说了别的话。你上次讲得更详细。”连我的姑祖母都放下了手中的活计,抬眼从老花镜子上面看看大家。
\par 星期六还有一个特别之处,那是在五月,每逢周末,我们吃罢晚饭便出门去参加“玛丽月”\footnote{玛丽是基督的母亲,每年8月15日为她的纪念日。}的祈祷仪式。
\par 由于我们有时能遇到对“当今的思潮纵容青年不修边幅”颇持严厉态度的凡德伊先生,我的母亲总特别注意我的穿着。每次她必先审视一番之后,我们才去教堂。我记得我是在“玛丽月”开始爱上山楂花的。它不仅点缀教堂(那地方固然很神圣,但我们还有权进去),它还被供奉在祭台上,成为神圣仪式的一部分,同神圣融为一体。它那些林立在祭台上的枝柯组成庆典的花彩,盘旋在烛光和圣瓶之间;一层层绿叶像婀娜的花边衬托出花枝的俏丽,叶片之上星星点点地散布着一粒粒白得耀眼的花蕾,像拖在新娘身后长长的纱裙后襟上点缀的花点。但是,我只敢偷偷地看上一眼;我觉得这些辉煌的花彩生气蓬勃,仿佛是大自然亲手从枝叶间剪裁出来的,又给它配上洁白的蓓蕾,作为至高无上的点缀,使这种装饰既为群众所欣赏,又具备庄严神秘的意味。绿叶之上有几处花冠已在枝头争芳吐艳,而且漫不经心地托出一束雄蕊,像绾住最后一件转瞬即逝的首饰;一根根雄蕊细得好像纠结的蛛网,把整个花冠笼罩在轻丝柔纱之中。我的心追随着,模拟着花冠吐蕊的情状,由于它开得如此漫不经心,我把它想象成一位活泼而心野的白衣少女正眯着细眼在娇媚地摇晃着脑袋。
\par 凡德伊先生带着女儿坐到我们的旁边。他本是富裕门第出身,曾经当过我的两位姨祖母的钢琴老师,他在妻子死后得了一笔遗产,便退休住在贡布雷附近,是我们家的常客。可是后来由于他过分讲面子,用他的话来说,怕在我们家遇到“合乎时尚地同一位门第不当的女子结婚”的斯万,便不常来我们家了。我的母亲听说他也自己作曲,每当前去拜望时便客气地说,他应该给大家演奏几段他的大作。凡德伊先生或许对此很高兴,但是他太讲礼貌也太与人为善,简直谨慎得过了头;他总设身处地为别人着想,就怕按自己的想法办会招人讨嫌,即使让人家猜出自己的意图,他也担心人家觉得他过于自私。我的父母拜望他的那一天,我也跟着去了。他们允许我在外面等候。因为凡德伊先生在蒙舒凡的房屋正处于我所呆的那个灌木丛生的小山头下面,我在的地点恰好同他们家三楼的客厅相齐,离窗户才五十厘米。当仆人通报我的父母来访时,我看见凡德伊先生忙把一首曲子放在钢琴上显眼的地方。但是当我的父母走进客厅,他却又把曲谱收了起来,塞到角落里去。他一定怕我的父母以为他之所以见到他们如此高兴只是为了可以给他们演奏自己的作品。每当我的母亲拜访他时重新怂恿他演奏自己的作品,他总要埋怨说:“不知道谁把这谱子放在钢琴上了,它本来没有放在这里。”接着他就把话题转到与他关系不大的方面去。他唯一的激情是对女儿的疼爱。他的女儿长得像男孩子那么壮实,当父亲的却对她体贴入微,总要给她披上披肩之类的东西,唯恐她着凉,谁见到这种情景都不免要微笑的。我的外祖母提醒我们说:那位脸上布满雀斑的莽撞的女孩子,目光中往往流露出温柔、敏感、甚至羞怯的表情。她说话时自己也本着对方的精神来听,警惕自己的话里可能出现使人误会的言词。人们能像透过玻璃似的看到她那副假小子的“淘气”外表下,越来越清晰地显示出一位楚楚动人的少女的细腻特征。
\par 离开教堂前我正跪在神坛下,起身时我突然闻到山楂花发出的一阵阵巴旦杏那样的甘苦兼备的气味。这时我注意到山楂花的花瓣上有几处发黄的斑点,我想象这气味就是从那里散发出来的,就像从点心的焦皮下发出蛋黄的香味,从凡德伊小姐的雀斑下散出她双颊的异香。尽管山楂花兀自不语,但它不断释放出的这股香气好比活跃的生命在窃窃低诉,连祭台都像田野里受到昆虫触角拨弄的疏篱,为之微微颤动。我所以产生这样的联想,因为我看到几茎生气蓬勃的发红的雄蕊仿佛是今天才由昆虫变成的,仍保留着昆虫的青春的锐气和撩拨的能力。
\par 我们走出教堂,在教堂门口同凡德伊先生寒暄了几句。几个男孩子在广场上打架,凡德伊先生前去干预;他维护年纪小的,训斥年纪大的。倘若他的女儿用粗嗓门对我们说,见到我们很高兴,我们仿佛立刻能感觉到在她的粗犷的外表下隐藏着一位敏感得多的女孩子,正在为男孩般冒失的客套话而羞红了脸,因为那句话有可能让我们以为她有意讨好我们,好让我们请她来家做客。她的父亲过来给她披上外套,父女双双登上由女儿亲自驾驶的轻便马车,打道回蒙舒凡。至于我们,因为明天是星期天,要睡到上教堂做弥撒之前才起床,所以如果赶上月明星稀、气候暖和的日子,我的好大喜功的父亲就会让我们作一次途经“受难场”的长途跋涉。我的母亲辨识方向和认路的能力较差,她把这样的远距离散步简直看做战略天才指挥的远征,有时我们一直走到旱桥底下。从车站那边延伸过来的石砌的桥身,在我的心目中代表了逐出文明世界之外的痛苦的形象,因为每年从巴黎乘火车来到这里,总有人千叮万嘱,要我们千万注意不可坐过站,火车还没有到达贡布雷,我们就已做好下车准备,因为火车只停两分钟,尔后它就要驶上旱桥,开出基督教国家的疆界。贡布雷是我心目中的基督教世界的终点站。我们取道车站大街回家,镇上最漂亮的别墅全在这里。月光像建筑师于贝·罗贝那样,给每家花园里点缀上白石台阶、喷水池和半掩的栅门,但是它偏偏把电报局大楼吞噬掉了,只给它留下一根拦腰截断的柱子,亏得柱子上还保存下了不朽遗迹的壮美。我拖着沉重的脚步,昏昏欲睡;椴树的芳香仿佛是一种只有付出劳而无当的代价才能得到的报偿。稀疏的栅栏内被我们零落的脚步声所惊醒的看家狗此起彼落地吠叫起来。至今,我有时在晚上仍依稀听到这样的吠声,心想车站大街一定就隐藏在犬吠声中(贡布雷的公园也在那条街上),因为,无论身在何处,我只要听到犬吠声遥相呼应,眼前便出现车站大街,被月光照白的两排椴树和路旁的人行道都历历在目。
\par 突然间,我的父亲叫我们停下。他问我的母亲:“咱们现在走到哪儿了?”早已精疲力尽、但仍为我的父亲感到骄傲的母亲柔声细气地自认无知。父亲耸肩笑了。接着,他像从上衣口袋里掏出钥匙那样轻而易举地伸手一挥,我们家花园的后门便同圣灵街的街口一起应命来到我们的面前。我们走过了漫长的陌生的道路,抬头一看,原来后门已在路尽处等候我们归来。母亲钦佩不已,对父亲说:“你真了不起!”从那一瞬间起,我已不用自己费力走路了,只觉得是花园的土地在我的脚下移动,在这里我的一举一动都毋需着意留神,习惯把我搂进它的怀抱,像抱娃娃似的一直把我抱到我的床上。
\par 尽管星期六那天的活动要比平日提前一小时,再加上弗朗索瓦丝又不能在家侍候,对于我的姨妈来说,那天比哪天都要漫长,然而她却从星期一起就天天急切地盼望星期六,似乎那一天会有种种既新鲜又开心的乐趣,她那娇弱而狂热的身体也还经受得住。这倒并不是说她没有偶尔巴望发生更大的变化,不渴求与现状完全不同的改观,像有些人那样由于缺乏精力或想象力,单凭自己无法产生改变现状的动力,只求未来的分分秒秒以及拉响门铃的邮差带来新的——哪怕是坏的——消息,以便激动一番,痛苦一番;被幸福弄得沉默的敏感,像闲置已久的竖琴急切地渴望有人来拨弄,哪怕让粗暴的手把琴弦拨断;难以排除障碍的意志,得不到纵情向往、纵情受苦的权利,恨不能把控制自己的缰绳甩给急转直下的,甚至鲜血淋漓的事件去掌握。也许我的姨妈稍受劳累精力便会完全耗尽,只能靠休息才能逐渐恢复,养精蓄锐更需日长时久,像别人在活动中流露出来的剩余精力,她需要一连休养生息几个月才能蓄全;她既认识不到这样的精力,更无法决定如何使用。正等于想以奶油土豆来取代土豆泥的念头,日复一日萦绕在她的心头,终于使她对奶油土豆产生同她对百吃不厌的土豆泥一样好的胃口一样,我毫不怀疑她终究也会从她那样恋恋不舍的单调生活中萌生出对灾祸的期望,但愿顷刻间发生一场灾祸,迫使她一劳永逸地实现一种由不得她的变化,但她认为这对自己的健康有益无害。她固然真心实意地爱我们,但她也乐于为我们的夭折而痛哭;她的希望一定经常受到类似如下景象的纠缠:一场灾难突然发生在她自我感觉良好而且不出汗的时候,例如家里忽起大火,我们都被烧死,房屋也烧得片瓦无剩,她多亏及时起床才不慌不忙地逃离火场,等等,而且这类景象仿佛同作为副产品的种种长处联系在一起,长处之一在于能使她在久久的哀恸中切实体会到她对我们的全部依恋之情;长处之二是能让镇上的人们惊叹她的坚强,看到她虽不胜悲痛却勇敢地挺住,虽伤心欲绝但沉着地为我们入殓出殡;最难能可贵的长处是能迫使她在合适的时机及时地、不必牵肠挂肚地到米鲁格兰的庄园去消夏,她在那里的庄园风景优美,更有瀑布点缀。她独自在房中百无聊赖地寻乐解闷的时候一定对诸如此类变故的成效进行过深入的思考(开头的情景,始料不及的种种细节,宣告噩耗的用词以及令人终生难忘的语气,还有其他确凿无疑地打上死亡烙印的一切,凡与抽象推理演绎出的可能性绝然不同,起先一定使她痛不欲生过),但是,这类变故毕竟从来没有发生,她也只得降格以求,把她热衷于虚构的曲折情节引进自己的日常生活,好让日子过得有点意思。她有时心血来潮,突然假设弗朗索瓦丝偷她的东西。于是她不惜巧施心计,想以捉贼捉赃的办法来证实她的假设。就像她独自玩牌惯于同时兼打对家一样,她模拟弗朗索瓦丝尴尬地向她求饶,然后她又气愤地、火气十足地予以驳斥。如果赶巧这时有谁进屋,就会发现她正大汗淋漓,两眼放光,头上的假发也歪到了一边,露出光秃的前额。弗朗索瓦丝也许有时听出隔壁房内传来的,用词尖刻的挖苦话是针对她说的,但是,既然这些话仅停留在纯抽象的状态,小声说出来并不能增加它的现实意义,那么我的姨妈纵然编出一套又一套话,也不足以解她心头之恨。有时她甚至不满足于在床上“排练”,想正式演出。于是有一个星期天,她把里里外外的房门都给神秘地关上了,在房里跟欧拉莉进行密谈,她说她怀疑弗朗索瓦丝手脚不干净,她要辞退她;另有一次,她私下对弗朗索瓦丝说,她怀疑欧拉莉靠不住,以后打算不让她再登门了;过了几天,她又反悔自己不该同吃里扒外的内奸说私房话,一想到自己竟把这号人引为知己就要恶心;不过等到下一场演出,叛徒的角色又会分派给别人。但是,对欧拉莉可能引起的怀疑毕竟只是一时的,像一堆起火的麦秸,不经烧,转眼就烧光了,因为她到底不是家里的人。对弗朗索瓦丝就不一样了,我的姨妈时刻感到她就在这同一个屋顶下面。她若不是怕起床着凉,还真敢下厨房去证实一下自己的怀疑有无根据。如此日复一日,她的头脑里不再有别的牵挂,一心只想猜度弗朗索瓦丝这时可能在干什么,那时又可能企图隐瞒什么;弗朗索瓦丝面部一点细微而迅速的变化,话语中的一点自相矛盾,都逃不过我姨妈的注意,她能从中识破弗朗索瓦丝妄图掩盖的真实打算。她只消一句话便能使弗朗索瓦丝顿时吓得脸色变白,这种直戳对方心窝的做法似乎很使我的姨妈尝到一种残忍的乐趣,她能以此向弗朗索瓦丝表明自己早已看透对方的心计。等到下一个星期天——犹如那些重大的发现突然为一门新学科开辟出一片意想不到的研究领域,并使它走上正轨那样——欧拉莉作了一次揭发,证明我的姨妈原先的假设还远远赶不上实际的真相。
\par “弗朗索瓦丝现在一定心里有数了:您送她一辆马车。”
\par “什么?我送她一辆马车?”我的姨妈失声叫道。
\par “啊!我哪儿知道呀?只是猜想罢了。我见她坐着马车神气活现地去鲁森维尔采购东西,心想准是奥克达夫夫人把这马车送给她了。”
\par 这样一天天下去,弗朗索瓦丝和我的姨妈变得像野兽和猎人一样,时刻提防着对方耍心眼儿。我的母亲唯恐弗朗索瓦丝把提防发展为真正的仇恨,因为我的姨妈伤透了她的心。总之,弗朗索瓦丝越来越异乎寻常地注意我姨妈的每一句话和每一点表示,遇到有事要问,她总先反复斟酌应采取什么方式,待她话一出口,她便暗自留意我姨妈的反应,力求从脸部表情中揣度她的心思和她可能作出的决定。譬如说某位艺术家读了十七世纪的回忆录之后,一心想同太阳王攀附亲缘,便为自己编排家族世谱,使自己成为名门之后,或者同当今欧洲的某国君王搭上关系,满以为这才是条通行的正路,殊不知他等于缘木求鱼,不该拘泥僵死的形式,结果枉费气力却事与愿违;同样,一位身居内地的妇女,本来只不过听凭自己无法抵御的种种怪癖和百无聊赖中养成的坏脾气的摆布,从来没有想到过路易十四,但她发觉自己一天之内诸如起床、梳洗、用餐、休息之类极其琐细的活动,在一意孤行和专横任性方面竟同圣西蒙所说的凡尔赛宫的生活“机制”的实质略有异曲同工之妙,而且她还可以认为自己的沉默以及和善或高傲的细微变化,能引得弗朗索瓦丝沾沾自喜或惶惶不安,跟路易十四的廷臣乃至于王公贵族在凡尔赛御花园的曲径处递呈奏折时见到王上闭口不语、龙颜喜悦或傲然接纳而窃窃自喜或诚惶诚恐一样,确实,其效果是一样的。
\par 在我的姨妈同时接待本堂神甫和欧拉莉两人来访之后又休息了一阵后的那个星期天,我们全都上楼去向她道晚安。妈妈对姨妈总遇到同时接待多人的不幸遭遇表示同情和慰问,她柔声细气地对姨妈说:
\par “听说今天您这儿又给弄得乱哄哄的,您总是一下子有一大帮客人。”
\par 我的姑祖母打岔说:“人越多越热闹……”自从她的女儿病倒之后,她认为应该处处使女儿高兴,凡事总往好处说。可是我父亲那时偏要插话,说:
\par “我现在趁大家都在场,跟你们讲件事儿,免得以后跟每个人啰嗦一遍。勒格朗丹先生恐怕跟咱们有点不愉快,今天上午我跟他打招呼他才勉强点了点头。”
\par 我倒不必听父亲讲这件事的始末,因为我们做完弥撒遇到勒格朗丹先生的时候我正同父亲在一起。所以我就到厨房打听晚饭菜谱去了。我看菜谱跟人家看报一样是每天少不了的消遣,而且它跟戏单子一样能使我的精神兴奋。勒格朗丹先生走出教堂经过我们身边的时候,他正同附近一位与我们只是面熟的女庄园主并肩走着。我的父亲一面走一面向他打了个既友好又矜持的招呼,勒格朗丹先生稍有惊讶的神色,勉强地答礼,仿佛他没有认出我们是谁。他那种疏远的眼光只有不讲客气的人才会使用,仿佛忽然退缩到眼睛的深处,像从一条漫长得望不到头的路口远远地瞥上一眼,所以他只向你略略颔首,以便同他心目中木偶般的小人的比例相称。
\par 至于同勒格朗丹并肩而行的那位女士,倒是位受人尊敬、品行端正的人,所以不存在他可能有恋爱纠葛被人发现而感到尴尬的问题。我的父亲弄不明白的是他怎么可能引起勒格朗丹不满。“如果他真有所不满的话,那我就更为遗憾了,”父亲说,“因为在那一大群衣着讲究的人之间,他只穿件单排扣的小尺寸上装,领带也不挺括,颇有一种不事修饰、朴素自然的风度,一种近乎天真、落落大方的派头。”家庭会议的一致看法是认为我的父亲可能过于多心,要不然就是格勒朗丹当时心不在焉,想别的事。父亲的挂虑在第二天晚上被打消了。我们散步归来,在老桥附近遇到了勒格朗丹;他因为过节在贡布雷多盘桓了几天。他一见我们便迎上前来,向我们伸出手。“书迷先生,”他这话是对我说的,“你知道保尔·戴夏克丹的这句诗吗?——树林已经昏黑,天空仍碧青如洗。——不正是眼前这个时刻的精当的写照吗?你也许还没有读过保尔·戴夏克丹的作品;读点他的作品吧,孩子。有人告诉我,说他现在已经皈依布道兄弟会当修士了,不过他过去长期是一位笔触清丽的水彩画家……树林已经昏黑,天空仍碧青如洗……但愿天空对我们永远晴朗,小朋友;甚至我在这样的日落西山的年龄,尽管树林已经昏黑,夜幕即将降临,我这样遥望天际,也照样能得到慰藉。”说罢,他从口袋里掏出一支卷烟,久久凝视远方。“再见了,伙伴们。”他突然冒出这么一句话后便扭身走开了。
\par 平日当我下厨房打听菜谱的时候,晚饭已经下锅。只见弗朗索瓦丝像神话中自荐下凡当厨的巨人那样调动一切自然力量来作自己的帮手;她砸煤取火,给待烹的土豆提供蒸气,让上桌的主菜火候恰到好处,这些烹调杰作先已由她像陶瓷工那样在各种器皿中整理塑造,她用过大缸、大锅、小锅、鱼锅、炖野味的砂锅、做点心的模子、调蛋酱的小罐,以及一套各种尺码的平底煎锅。我的目光久久地停留在案板上。帮厨女工剥完的青豆一行行数目不等地排列在案,像正在开赛的台球桌上的绿色台球。不过,最使我悦目赏心的是那堆芦笋,从头到脚浸透了海青、桃红两色,上端的穗条一丝丝有如染上了浅紫和碧蓝,往下则好似虹彩递变,色层分明,直达污泥犹存的根部;这显然不是土壤之功,我觉得这些天成的光色恰恰泄露了一群狡黠的精灵的作为,仿佛是它们乐于化作菜蔬,好让人们透过这些厚实而可口的肉质伪装,从犹如曙光初现、彩虹渐显、暮霭覆天之时的光色转换中,瞥见它们可贵的本质。我在晚餐时食用过芦笋之后,这种本质我整夜都不难分辨;变幻的光色恰如莎士比亚神话故事里专爱恶作剧的小精灵,开尽既有诗意又很粗俗的玩笑,一夜间把我的夜壶变成了香水瓶。
\par 被斯万称作乔托“慈悲图”的帮厨女工受弗朗索瓦丝之命专削芦笋皮,一篮芦笋就放在她的身边。她那痛苦的神色仿佛表明她感受到人世间的种种苦难。芦笋淡红色的外皮上端有一圈蓝颜色,像是把芦笋头轻轻箍住的头饰,那上面细致入微地勾画出并列的一颗颗星星,宛如帕多瓦教堂的壁画“品德图”中缚在那女子头上的那圈花环,又像插在那女子的花篮中的成排的花朵。而这时弗朗索瓦丝正在烤鸡,只有她才善于烤得恰到火候;她的美名随着鸡肉的香味在贡布雷遐迩传播。等她把烤鸡端上桌面时,这种美味更显示出我对她品性的特殊感受中的温柔甜润的一面。她能把鸡肉烤得那样鲜嫩,鸡肉的香味于是在我的心目中成为她的一种美德所散发的芬芳。
\par 但是,那天我趁父亲就勒格朗丹一事向家庭会议进行咨询之际下厨探问菜谱,偏偏赶上乔托的“慈悲图”生育不久、体质尚弱、不能起床的日子。弗朗索瓦丝少了帮手干活,进度慢多了。我下楼时她还在面向后院的厨房外干粗活的小屋里杀鸡。她想从鸡耳下面割断喉管,鸡本能地、绝望地挣扎着,随之而来的是弗朗索瓦丝失态的叫声:“畜生!畜生!”由怒斥声所伴随的家禽的挣扎使我们的女仆的温柔甜润黯然失色,不如第二天晚餐桌上香喷喷的烤鸡那样给她脸上争光,因为烤鸡的外皮边上一圈金黄胜似绣上金丝花边的霞帔,那精美的酱汁淋漓而下,也像是从圣体盒里滴下的甘露。喉管割断之后弗朗索瓦丝把如注的鲜血盛入碗中,这时她仍余怒未消,跺了跺脚,怒目瞪视着冤家的尸体,最后骂了一句“畜生”!我浑身发抖,扭头上楼,恨不得马上叫人把弗朗索瓦丝赶出家门。但是,她若一走,谁给我做热乎乎的卷子?谁给我煮香喷喷的咖啡?甚至……谁给我烤那么肥美的鸡?……其实,这类卑劣的小算盘人人都打,跟我一样。因为,我的莱奥妮姨妈早已心中有数——只是我当时还不知道——她知道能为自己的女儿和子侄舍命而决无怨言的弗朗索瓦丝对别人却特别狠心无情。虽说如此,姨妈却仍然留用她,因为她固然认识到她心狠,却又器重她能干。我逐渐认识到弗朗索瓦丝温柔、虔诚和讲究德操的外表下掩盖着多少出类似厨房外那间干粗活的小屋中发生的悲剧,正如历史发现那些在教堂的彩画玻璃窗上被描绘成合十跪拜的历代男女君王,生前无不以血腥镇压来维护自己的统治一样。我终于明白弗朗索瓦丝除了自己的亲属外,对于别人的不幸,唯其遭难者离她越远才越能引起她的怜悯。她在报上读到陌生人遭难时会泪如雨下,待她一旦对那人的身世有了更为确切的了解后,她的泪水转眼便会干涸。帮厨女工分娩之后的某一天晚上忽然肚疼难忍,妈妈听到她哼哼叫疼,起床推醒弗朗索瓦丝,她却不为所动,声称帮厨女工哇哇叫喊无非装样罢了,她想叫人“侍候”呢。当初医生预计到这种情况,在我们家的一本医学书中夹上一张书签,把描述这类腹痛症状的那一页特别标出,以便我们及时查阅,采取应急措施。我的母亲叫弗朗索瓦丝把那本书拿来,嘱咐她切不可把书签弄丢。弗朗索瓦丝去了个把钟点还不回来;母亲又急又气,以为她又上床睡去了,便叫我亲自去图书室查找。我在图书室见到弗朗索瓦丝;她起先想看看书签标出的那一页的内容,待她读到发病时的临床描述,不禁呜呜地哭出声来,因为这恰恰是她所不知道的一种病症。而当她读到书中说到每一种疼痛的情状时,她都要失声叫道:“哎呀!圣母玛丽亚,慈悲的上帝怎么能让可怜的凡人经受这样悲惨的痛苦呀?唉!可怜的女人啊!”
\par 但是,当我把她叫走,当她回到“慈悲图”痛苦辗转的床前,她的眼泪顿时不流了;她平时的悲天悯人的恻隐之心,读报时常常流淌的同情泪,以及同舟共济、同病相怜的感情,统统被她抛诸脑后,只剩下半夜三更被一名帮厨女工折腾得无法安眠所感到的恼恨和气愤。医书上有关的描述虽曾使她失声痛哭,待她实地见到同样的痛苦时,她却只有不满的嘀咕,甚至狠心的挖苦。她以为我们已经走远,听不到她信口雌黄,便肆无忌惮地数落起来:“早知今天受这份罪,她当初就不该浪!既然当初贪图一时的舒服,今天又何必哭天喊地装蒜!不过,能跟这号货色鬼混的,也准是个上帝都讨厌的赖小子。哈!这正合上我过世的母亲乡间的一句老话,叫作相中狗屁股的人,眼里只认作是玫瑰。”
\par 然而,倘若她的外孙头疼脑热,她夜里觉也不睡了,也像得了病似的,连夜赶回家去看看有什么要她帮着去办的。尔后又在天亮之前连赶十六公里夜路回来上班。她对于家属的这种疼爱,这种但求自家门庭日后兴旺的心愿,在她对其他用人所采用的方针中由一条始终如一的原则表现出来了,那就是决不让别的用人踏进我的姨妈房间的门槛。不让别人接近我的姨妈几乎是她引为骄傲的头等大事,即便她病倒了,她也要硬撑着起床去侍候我的姨妈服用维希圣水,而决不许帮厨女工跨进她的女东家的房门。法布尔\footnote{法布尔(1823—1915):法国昆虫学家,科普读物作家;代表作为《昆虫记》。}曾经考察过一种膜翅目的昆虫,一种土居的黄蜂,它们为了在它们死后幼虫仍能吃到新鲜的肉食,不惜借助解剖学知识来发挥它们残忍的本性:它们用尾刺娴熟地、巧妙地扎进捕获到的象鼻虫和蜘蛛的中枢神经,使俘虏失去肢体活动的能力,又不影响到其他的生命功能;然后它们把瘫痪的昆虫放到它们所产的虫卵的旁边,好让幼虫一经孵化出壳就能吃到既无力抵抗也无法逃遁、只有乖乖听凭摆布、决无危害又不变味的活食。弗朗索瓦丝为了让别的用人无法在我们家长期呆下去,也总有一套巧妙而残忍的诡计来实现她这一持之以恒的愿望。我们直到好多年之后才知道原来那年夏天我们之所以吃那么多芦笋,是因为芦笋的气味能诱发负责削皮的帮厨女工的哮喘病,而且发作起来十分厉害,弄得那女工只好辞职不干。
\par 唉!我们必须义无反顾地改变对勒格朗丹的看法。在我的父亲与他老桥相遇、接着又不得不自认多心之后的某个星期天,教堂的弥撒刚刚结束,一种不那么神圣的气氛随同外面的阳光和嘈杂声一起拥进教堂,使得古比尔夫人和贝斯比埃夫人像走出教堂来到广场上似的同我们大声交谈起来(而不久前我刚进教堂时——我到得比平时晚——人人都目不斜视专心祈祷;若不是有人用脚拨开挡住我就座的小凳,我还真以为没有人看到我进来呢)。这时我们看到勒格朗丹正站在阳光灿烂的大门口;门楼外的台阶下是人声鼎沸、五光十色的集市。我们上回见过的那位夫人的丈夫正把勒格朗丹介绍给附近另一位大地主的妻子。勒格朗丹显得异乎寻常地活跃和讨好,他深深地鞠了一躬又往后一仰;身板仰到比原先更靠后的地位,这礼节想必是他的姐夫康布尔梅先生教的。他的腰板迅速一挺,臀部——据我猜想肌肉未必丰满——随即掀起一股强烈的波动。不知道为什么这种纯属物质的起伏,这种并不表达灵气、只受低下的献媚之心所驱使的肉体活动,竟突然会使我的思想意识到可能存在着另一位与我们所认识的朋友完全不同的勒格朗丹。那位女士请他给车夫捎句话,他立即喜孜孜地应命而去。他刚才被介绍时就挂在脸上的那种羞羞答答、俯首帖耳、喜笑颜开的表情,一直停留在他的眉宇间。他像做梦似的咧嘴笑着,又急急忙忙赶回到那位女士的跟前。由于他走得比平时快,肩膀便左摇右摆,十分可笑;他只管全力以赴地讨好,其他方面也就无暇顾及了,所以显得像一件受幸福驱动的无生命的机械玩具。这时我们已经走出教堂,正要从他的身边经过;那么有教养的他居然没有回头,他的目光像大梦未醒的人,直勾勾地盯着远方,对我们竟视而不见,也无从跟我们打招呼。他的表情还是那么天真单纯,那件款式随便的单排扣上衣在令人讨厌的讲究的衣着中间显得与场合不相称。被广场上的风所吹起来的那个花点大领结,依然像一面标榜孤傲和独立的高尚的旗帜飘动在他的胸前。
\par 我们刚到家门,妈妈发现忘了买奶油果子饼,便要父亲和我一起返身去吩咐点心铺立刻送来。我们在教堂附近同勒格朗丹迎面相遇。他用自己的马车载着刚才的那位女士朝我们来的方向驶去,经过我们的身旁时他并没有中止同那位女士的谈话,而只用他的蓝眼睛的眼角瞟了我们一眼,仿佛在眼皮底下同我们打了一个小小的招呼,脸上的肌肉却纹丝未动,车上的那位夫人很可能根本没有发觉他的这一举动,但是,他设法以感情的密度来补偿向我们表达友情所用的仅占他蓝眼睛小小的一角的狭小的地盘,他让这一瞟闪烁出他的全部风采,这已不止是活泼的闪光,而近乎狡黠了。他使友好的细微表现达到了极限:心照不宣的一瞥明眼人心领神会,总之凡灵犀相通的种种途径他都熟门熟路;他把友谊的保证提高到披露柔情、甚至宣告爱慕的高度。当时,他以对女庄园主的隐而不露的厌烦和纹丝不动的脸上那多情的一瞥来向我们表明心迹,也只有我们才能心领神会。
\par 就在那天的前一天,他要求我的父母让我去陪他吃晚饭。“来陪陪你的老朋友吧,”他对我说,“你就像是远方的旅客从我们一去不复返的国度送来的一束鲜花,让我闻闻从你的青春的远方送来的这些鲜花吧。许多年以前,我也曾经经历过群花争妍的春天。来吧,带着报春花、龙须菊和金盏花;来吧,带着巴尔扎克的植物志中象征挚爱的景天花,带着复活节前开放的雏菊和复活节前的最后一场小雪尚未融化时已经在你姑祖母家的花园中播散芳香的雪球花;来吧,带着百合花洁白的绸缎(那是配得上莎乐美那样娇美的身躯的裙料),带着蝴蝶花斑斓的彩釉,尤其要带来寒意犹存的料峭的清风,让它为一早就守候在门口的两只彩蝶吹开耶路撒冷的第一朵玫瑰。”
\par 家里的人起先拿不定主意,不知道该不该让我去陪伴勒格朗丹先生吃顿晚饭。倒是我的外祖母说什么也不愿意相信他会不讲礼貌:“你们自己也承认,他去教堂时穿得很朴素,跟讲排场的人不一样。”她还说,哪怕作最坏的估计,就算他是贪慕虚荣的人,我们无论如何也不宜显出有所察觉。说实话,连对勒格朗丹的态度最为反感的我父亲也许对他的举止的含义都还存有最后一点怀疑呢。他的言行不正显示了那种城府很深的人的品性吗?他的态度跟他以前的言论明明是脱节的;我们无法根据他的自白来证实我们的怀疑,因为他不会老实招供的;我们只能依靠自己的感觉。但是,仅仅根据片断的、不连贯的回忆,我们却没有把握确信我们的感觉会不受某种幻觉的愚弄。结果这些至关紧要的待人接物的态度往往只给我们留下一些疑团。
\par 我陪伴勒格朗丹在他家房前的平台上用晚餐;那天晚上月色清朗。“有一种幽静的美,是不是?”他对我说,“正如一位小说家所说,对我这样心灵受过创伤的人来说,只有幽暗与寂静最为相宜。你以后会读到他的作品的。你知道吗,孩子?一个人在一生之中会遇到那样的时候,你现在还体会不到,那时候眼睛只能容忍一种光明,那就是在这样月白风清的夜晚以幽暗提炼出来的光明;耳朵也只能听到一种音乐,那就是月光用寂静的笛子奏出的音乐。”我听着勒格朗丹娓娓道来,他的话我听了总觉得很入耳。但是我当时无法摆脱记忆的骚扰,我总忘不了最近第一次见到过的一位女士。我现在既然知道勒格朗丹同附近的一些贵族有交往,我想他或许认识那位女士,于是我鼓了鼓勇气问他说:“先生,您是不是认识……盖尔芒特家的那一位……那几位女主人?”这个姓氏一经被我说出口,我感到非常高兴,因为我总算对它采取了行动,把它从我的梦幻里拉了出来,赋予它一个客观的、有声的存在。
\par 但是,我发现我的朋友一听到盖尔芒特这个姓氏,他的蓝眼珠中央立刻出现一个深褐色的漏洞,好像被一根无形的针尖捅了一下似的,眼珠的其他部分则泛起蔚蓝色的涟漪。他的眼圈顿时发暗,他垂下眼皮,嘴角掠过一丝苦笑,很快又恢复了常态。他的眼神却像万箭穿胸的美丽的殉道者,依然充满痛苦。“不,我不认识她们。”他说,那语气不像一句简单的答话、普通的说明那样自然而流畅;他说得一字一顿,又点头又弯腰,好像在说一件别人不信、他为了说服对方不得不加以强调的事情,似乎他不认识盖尔芒特只是出奇的偶然;同时他又装成像不能回避某种尴尬局面似的,觉得与其遮掩不如痛快承认,好让人家觉得自己很坦然,并无丝毫勉强之处,而是轻松、愉快、由衷地直认不讳;再说同盖尔芒特没有联系的这件事情本身也并不使他感到遗憾,相反是符合他的心愿的,因为某种家庭传统,例如道德原则或不便明说的誓约之类毫不含糊地禁止他同盖尔芒特交往。“不,”他接着用自己的话来解释方才的语气,“我不认识她们,我也从来没想结识她们;我始终珍惜我享有的充分的独立。你知道,我其实多少是个雅各宾派。许多人劝我,说我不该不去结交盖尔芒特,说我把自己弄得粗野不堪,像头老熊。可是,这种名声我才不怕呢,恰如其分嘛!说实话,这人世间我几乎无所留恋,除了少数几座教堂,两三本书,四五幅画;还有这样的月夜,你的青春的微风把我的昏花的老眼已无法看清的鲜花的芳香吹到了我的跟前。”我当时弄不明白,为什么一个人必须坚持自己的独立才能不去拜望陌生人?这又在哪一点上使你显得像头笨熊?但是,有一点我是明白的,勒格朗丹说的不尽是实话,他并不像他所说的那样只爱教堂、月光和青春;他很爱住在宫堡里的贵族,他很怕招他们的讨厌,他甚至不敢让他们发现自己的朋友当中有布尔乔亚,有公证人和经纪人的后代,倘若真相不得不暴露,他宁可自己不在场,躲得远远的,让人“鞭长莫及”。他是贪图虚荣的人。当然,他在我的长辈和我都十分爱听的言谈中,决不会透露半点趋炎附势的痕迹。我若问他:“您认识盖尔芒特家的人吗?”巧于辞令的勒格朗丹就回答说:“不,我从来没想结识他们。”可惜的是,回答这话的他实际听命于被他深深地埋藏在心里、从不出头露面的另一位勒格朗丹,而这另一位却能说出有关我们心目中的他,以及有关他贪图虚荣的不少难避嫌疑的掌故来。其实,他刚才眼睛里出现的那个漏洞,他嘴边掠过的那丝苦笑,他语气中那样的过分强调,以及他一瞬间像势利殉道者那样万箭穿心般的痛苦情状,早已为另一位勒格朗丹作出了回答:“唉!你算是击中我的痛处了。不,我不认识盖尔芒特,别再揭我生平最疼痛彻骨的这块伤疤了。”这位桀骜不驯、气势汹汹的勒格朗丹虽无另一位勒格朗丹的美妙言词,却有人称之为“反射”的犀利无比的对应能力,故而巧于辞令的勒格朗丹还没有来得及堵住他的嘴,他已经抢先表了态,害得我们的朋友处心积虑,力求弥补“另一个自我”不慎造成的坏印象,却毕竟无济于事,充其量只能勉强遮掩罢了。
\par 这倒并不是说勒格朗丹怒斥别人附庸风雅是言不由衷。他无法知道自己也是那种人,至少靠他自己无法办到,因为我们向来只知道别人热衷于什么,至于自己醉心之所在,我们略知的一二也都是从别人那里听说的。七情六欲只通过间接方式、只通过想象影响我们,而想象早已用体面得多的中间动机替换掉了原始动机。勒格朗丹的势利之心决不会直接鼓动他去结交某位公爵夫人,而只会让他充满想象,使那位公爵夫人在他眼里显得集优雅品质于一身,他去接近她还自以为是仰慕一般俗人所无法赏识的她的才思和德操之类的动人品质,只有旁人才看清他其实同一般俗人不相上下,因为旁人了解不到他的想象力所发挥的中介作用,他们只看到勒格朗丹高攀贵族的活动以及与此相应的原始动机。
\par 现在我们家已对勒格朗丹先生不抱任何幻想了,同他也大大疏远了。妈妈每当发现他攀附高枝的新行径,总觉得十分有趣。勒格朗丹本人则矢口否认,他仍把势利称作罪不容赦的行为。我的父亲却不能这样坦然愉快地容忍勒格朗丹的假清高。有一年暑假,他们想让我同外祖母一起去巴尔贝克度假。父亲说:“我无论如何要把你们去巴尔贝克的这件事告诉勒格朗丹,我倒要看看他会不会主动地把你们介绍给他的姐姐。他一定还记得曾经跟咱们说过,他姐姐就住在离巴尔贝克才两公里的地方。”我的外祖母倒认为既去海滨浴场就应该从早到晚在海滩上呼吸带盐分的空气,没有熟人才好呢,因为互相串门拜访、结伴游览,会占去许多呼吸海风的时间,所以她主张不向勒格朗丹透露我们的度假计划,她甚至担心勒格朗丹的姐姐德·康布尔梅夫人不要偏在我们正打算去海边钓鱼的时候来到我们下榻的旅馆,害得我们只能关在屋里奉陪。妈妈对外祖母的担心付诸一笑,她认为这种危险的威胁性不大,勒格朗丹未必会殷勤到把我们介绍给他的姐姐。结果,我们虽说没有跟勒格朗丹谈及巴尔贝克,而他也从来没有想到我们会有去那儿的打算,有一天傍晚我们在维福纳河边遇到他时,他竟“自投罗网”了。
\par “今晚,云霞中有些非常美的紫色和蓝色,是不是,我的伙计?”他对我的父亲说,“尤其是那蓝颜色,与其说是空中的,倒不如说跟花朵一样,蓝得像瓜叶菊,挂在天上格外别致。还有那一小团桃红色的云彩,不也有花的色调吗?像石竹,像绣球。只有在英吉利海峡,在诺曼第和布列塔尼之间的海边,才能看到天空出现比这更富丽的花团锦簇般的云霞。那里,在巴尔贝克附近,离那一大片蛮荒之地不远的地方,有个风物秀丽的小海湾;那里熔金般的落日,奥吉谷地的夕阳,我倒并不在乎,因为它们并无多大特色也并无多大意趣;但黄昏时分在那片湿润的空气中,几秒钟之内天边就绽出一束束蓝的、粉的花朵,却美得无法比拟,而且往往要过好几个小时才会凋谢。有几朵云彩虽然不久就零落了,但它们的花瓣,鹅黄色的、桃红色的,洒得满天皆是,更是蔚为壮观。在那个人称银河湾的小海湾里,金黄色的沙滩仿佛比仙女星座里的金发仙女更情意绵绵,它们依偎着附近海边嶙峋的峭壁,贴着那一溜以海难著称的凶险的石岸,每年冬天有多少条顶风破浪的船只在那里触礁啊!巴尔贝克!我们的地球上最古老的地质架,名副其实的地表硬壳,大海由此浩淼,土地至此而尽。阿纳托尔·法朗士,我们的小朋友或许读过这位迷人作家的作品吧?他曾经非常精彩地把那个鬼地方描绘得终年烟雾茫茫,跟史诗《奥德赛纪》里奚美良人\footnote{公元前七世纪居住在小亚细亚的古老部落。}居住的地方一样。如今在巴尔贝克那片古老而迷人的土地上,已经层层叠叠地盖出了一批旅馆,但并没有破坏那里的景观,仅几步之遥便能置身于原始风味的壮丽景色之中,岂不美哉!”
\par “是啊!您在巴尔贝克有熟人吗?”我的父亲问道,“这小家伙正好要跟他的外祖母,也许还有内人一起到那里去住上两个月呢。”
\par 勒格朗丹望着我的父亲,忽然出其不意听到这句问话,他来不及把眼睛从我父亲的脸上移开,只好索性紧紧地盯着,嘴角泛起无可奈何的微笑。他望着我的父亲的眼睛,那表情既友好又坦诚;他倒不怕正视对方,仿佛对方的面孔已经变得透明,甚至使他看到了面孔后面掠过的一朵颜色艳丽的云彩,来为他提供心不在焉的借口,好有理由为自己申辩:当别人问他在巴尔贝克有无熟人的时候,他仿佛正心不在焉想别的事,以至没有听到问话。通常,他这样的眼光会引起对方发问:“您在想什么?”可是我的父亲有点恼火,偏要狠心地盘问到底:
\par “您那么熟悉巴尔贝克,您在那里有熟人吗?”
\par 勒格朗丹的微笑的目光作了最后的绝望的努力,达到柔和、迷人、坦诚和走神的极致。但他一定想到自己非作出回答不可了,便说:
\par “我哪儿都有朋友,只要那地方有几丛受伤的树,虽被斫伤却不倒下,彼此相依在一起,以悲壮的毅力齐声向并不怜恤它们的无情的苍天哀告。”
\par “我不是这个意思,”我的父亲像受伤的树一样顽强,像苍天一样无情地打断他的话说,“我是为了岳母一旦有事,不要感到举目无亲,所以才问您,您在那儿有没有熟人?”
\par “那儿,跟哪儿都一样,我谁都认识,又谁都不认识,”勒格朗丹不肯就此服输,答道,“那地方我很熟悉,人却所识无几。但是那里的景物本身同人差不多,同那些难能可贵、心灵纤细、遇到实际生活容易消沉的人一样。
\par 有时候,您会在悬崖上遇到一幢古堡,它悄立在路旁迎着红晕未消的晚霞,掂量自己的凄凉,那时金色的月亮已经升起,归航的船只拨开色彩斑斓的水面,把黄昏的火焰捧上桅尖,以黄昏的颜色染遍招展的旌旗;有时候,您能见到一幢普通的孤舍,模样多少有点丑陋,显得猥猥琐琐,但很有一点诗情画意,其中蕴蓄着谁都看不透的某种秘密,既有无穷的幸福,也有不尽的失望。”他接着又像马基雅维里\footnote{马基雅维里(1469—1572):意大利政治家,外交家,作家,传世的《君主论》被认为是他的代表作。他主张政治不受任何道德的束缚,为达到目的可不择手段。}那样颇有心计地补充说道:“那是个不实际的地方,是个纯属幻想的地方,让一个孩子去领略那里的风光很不妥当。我们这位小朋友已经具有感伤的倾向,他的心灵天生善于领会这类情调,我若为他选择一个散心的地方,决不会介绍他去那儿。那里充满情绵绵互诉衷肠、恨悠悠枉自惆怅的气氛,对我这样早已看破红尘的老朽来说可能还算适宜,对于气质尚未成型的孩子来说总是不健康的。相信我的话,”他着重地强调说,“那个海湾的水有一半已经是布列塔尼省流来的了。对于我这样心脏并非没有毛病的人来说,反正是那么回事儿,据说,那里的海水还有些镇静作用呢。不过有人还说未必。至于你这样的年纪,小家伙,医生是禁用那里的海水的。再见,各位芳邻。”他这么补了一句,便像往常那样有意逃避似的突然离开我们;才走几步,他又回过头来,向我们伸出医学权威的手指,把他的诊断作了如下的概括:“五十岁以前,不要去巴尔贝克,五十岁以后还得视心脏状况而定。”他大声向我们宣告。
\par 我的父亲后来遇到他时又老话重提,还用盘问折磨他,但照样白费工夫。勒格朗丹跟那种善于伪造古籍的骗子一样,自有一套本领和广博的学问,他只需使用其中的百分之一,便足以稳当地赚进一大笔钱,过上相当体面的日子。如果我们没完没了地盘问下去,他或许最终会胡扯一通景观伦理学或者下诺曼第天文地理学,但决不会向我们供认他姐姐的住地离巴尔贝克仅两公里,更不会义不容辞地为我们写封介绍信。倘若他有绝对的把握相信我们不会利用这类介绍信,他倒大可不必那样提心吊胆。按理说,根据平时的接触,他应该对我的外祖母的性格有所了解:我们怎么会利用这类介绍信呢?但他宁可避而不谈。
\par 平时散步,我们总是早早就回家了,以便在晚饭前上楼去看看莱奥妮姨妈。初春时节天黑得早,我们回到圣灵街时家里的玻璃窗上已反射出落日的余晖,而在十字架那边的树林里,一抹紫霞映在远处的池塘中,常常伴随着料峭寒意,红色的夕阳在我的心目中却同烤炉上的红色的火苗相关联,因为烤炉上的肥鸡对于我来说是继散步的诗情陶醉之后的另一种享受,使我得到解馋、温暖和休息的快乐。到了夏天,相反,等我们散步回来,太阳还没有下山。我们到莱奥妮姨妈的房里时,西斜的阳光正照到窗口,停留在大窗帘和帘绳之间,被分割成一束束、一条条,透过窗帘射进房来,给柠檬木的多屉柜镶嵌上一片片碎金,又像照射林中的草木丛似的,以耀眼的斜光细致入微地照得满屋生辉。但是,难得有那样的日子:我们回来时柜子上的临时嵌饰已经消失,我们到达圣灵街时,窗户上已经没有夕阳的反照,十字架树林那边的池塘也已经失去了夕阳的红光,甚至变成银白色;一道长长的月光,融入池塘的粼粼细波之中,并且铺满整个水面。每逢那样的日子,当我们走近家门时,就会看到门口有个人影;妈妈对我说:
\par “天哪!弗朗索瓦丝在等候咱们呢。你的姨妈不放心了;咱们回来得太晚了。”
\par 我们顾不得脱掉外衣,赶紧上楼,好让莱奥妮姨妈放心,并且以现身说法向她表明,同她想象的恰恰相反,我们一路上并没有遇到不测,只是去“盖尔芒特家那边”散步了。天晓得,我的姨妈也明白,上那边去散步什么时候回得来就说不准了。
\par “瞧,弗朗索瓦丝,”我的姨妈说,“我不是说着了吗?他们果然去盖尔芒特家那边了!天哪!他们一定饿坏了!你炖烂的羊腿搁了那么半天一定发硬了。这么说,回来就得一个小时!怎么,你们居然去盖尔芒特家那边散步了!”
\par “我还以为您知道呢,莱奥妮,”妈妈说,“我记得,弗朗索瓦丝是看见我们从菜园的小门出去的。”
\par 因为,在贡布雷附近,有两个“那边”供我们散步,它们的方向相反,我们去这个“那边”或那个“那边”,离家时实际上不走同一扇门:酒乡梅塞格利丝那边,我们又称之为斯万家那边,因为要经过斯万先生的宅院;另外就是盖尔芒特家那边。说实在的,我对酒乡梅塞格利丝的全部认识不过“那边”两字,再就是星期天来贡布雷溜达的外乡人,那些人,我们(甚至包括我的姨妈)全都“压根儿不认识”,所以凡陌生人我们都认为“可能是从梅塞格利丝来的”。说到盖尔芒特,后来我了解得更多一些,不过那是很久以后的事;当时,在我的整个少年时代,若说梅塞格利丝在我心目中像天边一样远不可即,无论你走多远,眼前总有一片已经同贡布雷不一样的地盘挡着你的视线,那么盖尔芒特对我说来,简直是“那边”的极限,与其说有实际意义,倒不如说是个概念性的东西,类似赤道、极圈、东方之类的地理概念。所以,说“取道盖尔芒特”去梅塞格利丝,或者相反,说“取道梅塞格利丝”去盖尔芒特,在我看来,等于说从东到西一样只是一种语焉不详的说法。由于我的父亲把梅塞格利丝那边形容成他生平所见最美的平原风光,把盖尔芒特那边说成典型的河畔景观,所以我就把这两个“那边”想象成两个实体,并赋予它们只有精神才能创造出来的那种凝聚力和统一性。它们的每一部分,哪怕小小的一角,我也觉得是可贵的,能显示出它们各自特有的品格,而这两处圣地周围的道路,把它们作为平原风光的理想或河畔景观的理想供奉在中央的那些纯属物质的道路,却等于戏剧艺术爱好者眼中剧院附近的街巷,不值一顾。尤其是我想到这两处的时候,我把我头脑里的这两部分的距离安置在它们之间,其实大大超过了它们之间的实际公里数;那是一种空想的距离,只能使它们相距更远,相隔更甚,把它们各各置于另一个层面。由于我们从来不在同一天、同一次、同时去两边散步,而是这次去梅塞格利丝那边,下次去盖尔芒特那边,这种习惯使它们之间的界线就变得更加绝对,可以说把它们圈定在相隔遥远的地方,彼此无法相识,天各一方,在不同的下午,它们之间决无联系。
\par 每当我们想上梅塞格利丝那边去(我们不会很早出门,即使遇上阴天也一样,因为散步的时间不长,也不会耽搁太久),我们就像上别处去一样,从姨妈那幢房子的大门出去,走上圣灵街。一路上,打火铳的铁匠铺老板跟我们点头招呼,我们把信扔进邮筒,顺便为弗朗索瓦丝捎口信给戴奥多尔,说食油和咖啡已经用完,然后,我们经过斯万先生家花园白栅墙外的那条路出城。在到那里之前,我们就闻到他家的白丁香的芬芳扑鼻而来,一簇簇丁香由青翠欲滴的心形绿叶扶衬着,把点缀着鹅黄色或纯白色羽毛的花冠,探出栅墙外。沐照丁香的阳光甚至把背阴处的花团都照得格外明丽。有几株丁香映掩在一幢被称为“岗楼”的瓦屋前,那是守园人住的小屋,哥特式的山墙上面罩着玫瑰色的清真寺尖塔般的屋顶。丁香树像一群年轻的伊斯兰仙女,在这座法国式花园里维护着波斯式精致园林的纯净而明丽的格局,同她们相比,希腊神话里的山林仙女们都不免显得俗气。我真想过去搂住她们柔软的腰肢,把她们的缀满星星般花朵的芳香的头顶捧到我的唇边。但是,我们没有停下。自从斯万结婚之后,我的长辈们便不来当松维尔做客了,而且为了免得让人误以为我们偷看花园,我们索性不走花园外那条直接通往城外田野的道路,而走另一条路,虽然也通往田野,但偏斜出去一大段,要远得多。那天,外祖父对我的父亲说:
\par “你记得吗?昨天斯万说他的妻子和女儿到兰斯\footnote{初版时,斯万妻女不是去兰斯,而是去沙特尔。后来普鲁斯特决定把1914年至1918年的大战也写进小说,故而把贡布雷改置于未来的战区之内,即朗市与兰斯之间(事实上,贡布雷镇是以沙特尔附近的伊利埃斯为原型的)。}去了,所以他要乘机去巴黎住两天。既然两位女士不在,我们不妨从花园那边过去,路近多了。”
\par 我们在栅墙外停了一会儿。丁香花已盛极而衰。有几株依然托出精致的花团,像一盏盏鹅黄色的吊灯,但枝叶间许多部分的花朵,虽然一星期前还芳香如潮,如今却已萎蔫、零落、枯黄、干瘪,只像一团团香气已消的泡沫。我的外祖父指点着对我的父亲说,自从他同斯万先生在斯万太太去世的那天在这里一起散步以来,这园内的景物哪些依旧如故,哪些已经改换模样。他抓住机会又把那天散步的经过讲了一遍。
\par 我们的眼前是一条两边种植着旱金莲的花径,它在阳光的直射下向高处伸展,直达宅门。右面则相反,花园在一片平地上铺开。被周围的大树覆盖的池塘虽是当年斯万老先生雇人开挖出来的,但这花园中最着斧凿痕迹的部分也只是对自然的加工;有几处天然特色始终在它们的范围内保持着独特的权威,它们置身于花园就像置身于没有经过加工的自然环境中一样,公然挑出自己本来就有的特色。展示这些天然特色极需一个僻静的环境,而在人工点缀之上它们自有一种孤幽的意韵:例如花径下的人工池塘边,两行交相栽植的勿忘我和长春花组成一顶雅致的蓝色花冠,箍住了水光潋滟的池塘的前额,菖蒲像轩昂的王公挥落它们的宝剑,一任他们统治水域的权杖上紫色、黄色的零落的百合花徽,散落在泽兰和水毛茛的头上。
\par 斯万小姐的远行使我失去了有幸在花径一见她的倩影的可怕的机缘。不能结识这样一位享有殊荣、与贝戈特为友、能同贝戈特一起参观各处教堂的少女,应算是有幸抑或不幸呢?因为若与她相遇,自惭形秽的我必受到她的轻视;可是,由于她不在,我虽生平第一次得到静观当松维尔园内景色的机会,却只觉得了无情趣。对我的外祖父和我的父亲来说,情况倒似乎相反,他们也许觉得女主人们不在反给整个庄园增添宜人的气氛,使它具有难得的美(犹如登山之日巧遇万里无云的好天气),因而今天到这边来散步就格外适时。我真盼望他们的算计落空,突然出现奇迹,让斯万小姐陪伴着她的父亲双双来到我们的眼前,使我们不及躲避,只好同她结识。这时我忽然发现草丛里有只篮子被遗忘在一根钓鱼杆的旁边,鱼杆上的渔漂还浮在水面。我赶紧设法转移我的外祖父和我的父亲的注意,生怕他们发现她可能在家的些许迹象。不过,斯万倒曾经跟我们说过,他这回出门有点不合时宜,因为家里有人住着。那么说,这鱼杆可能是哪位客人放的。花径间听不到有人走动的声音。一只不见踪影的鸟不知在丈量哪棵树的梢头,它千方百计地要缩短白昼的长度,用悠长的音符来探测周遭的僻静,但它从僻静中得到的却只是调门一致的反响,使周遭更安定、更寂静,仿佛它本来力求使一瞬间消逝得更快,结果反使那一瞬间无限延长了。天空变得凝滞,阳光径直射下,让人想躲也躲不开;小昆虫们无休止地骚扰平静的水面,沉睡的池水一定梦见了想象中的弥漫无际的漩涡,仿佛在迅速地把软木渔漂拖进倒映在水中的那片悄然的天空,从而更增长我初见渔漂时的惶惑之感,渔漂几乎垂直地浮在水面,似乎随时都会沉入水中,我已经顾不得自己既想结识斯万小姐又怕见她的双重心情,考虑是否该去告诉她鱼已上钩。这时,已经走上通往田野小路的我的外祖父和我的父亲惊讶地发现我没有跟在后面便转身叫我,我只得赶上前去。我觉得小路上掠过一股山楂花的香味。疏篱像一排教堂被堆积的繁花覆盖得密密匝匝,成了一座巨大的迎圣台;繁花下面,阳光像透过彩绘玻璃窗似的把一方光明照到地上;如胶似漆的芳香萦绕着繁花组成的圣台,我的感觉就如跪在供奉圣母的祭台前一样。花朵也像盛装的少女,一个个若无其事地捧出一束熠熠生辉的雄蕊;纤细的花蕊辐射开去,像火焰式风格的建筑的肋线,这类线条使教堂的祭廊的坡级平添光彩,也使彩绘窗上的竖梁格外雄健,而那些绽开的花蕊更有如草莓花的洁白的肉质花瓣。相比之下,几星期之后,也要在阳光下爬上这同一条小路的、穿着一色粉红的紧身衣衫、一阵轻风便可催开的蔷薇,将会显得多么寒伧、多么土气啊!
\par 我虽流连在山楂花前,嗅着这无形而固定的芳香,想把它送进我不知所措的脑海,把它在飘动中重新捉住,让它同山楂树随处散播花朵的、洋溢着青春活力的节奏相协调——这节奏像某些音乐一样,起落不定——而且山楂花也以滔滔不绝的芳香给我以无穷的美感,但它偏偏不让我深入其间,就同那些反复演奏的旋律一样,从不肯深入到曲中的奥秘处。我暂且扭身不顾,用更新鲜的活力迎向花前。我纵目远望,一直望到通往田野的陡坡;那陡坡在花篱以外,一株迷失路津的丽春花和几茎懒洋洋地迟开的矢车菊,以稀稀落落的花朵,像点缀一幅挂毯的边缘似的点缀着那片陡坡,挂毯上疏朗的林野图案一定显得格外精神吧;而更为稀疏的花朵像临近村口的孤零零的房舍宣告村落已近似的,告诉我那里有无垠的田野,起伏着滚滚的麦浪,麦浪之上是叆叇的白云。而在田野边缘孤然挺立的丽春花,凭借一堆肥沃的黑土,高举起迎风燃烧的火炬,我一见到它心头便怦然跳动,就像远游的旅人在一片洼地瞅见嵌缝工正在修理一艘曾经触礁的船只,还没有见到大海便情不自禁地喊一声:“大海!”
\par 然后,我又把眼光落到山楂花前,像观赏杰作似的,总以为暂停凝视之后再回头细看才更能领略它的妙处。但是,尽管我用手挡住周围的东西,只给眼前留下山楂花的倩影,但花朵在我内心所唤起的感情却依然晦暗不清,浑浑噩噩,苦于无法脱颖而出,去与花朵结合。那些山楂花无助于我廓清混沌的感情,我又无法仰仗别的花朵。这时,我的外祖父给了我这样一种愉快,其感觉好比我们看到我们所偏爱的某位画家的一幅作品,它同我们所熟悉的其他作品大不一样;或者我们忽然被人指引,看到那么一幅油画,过去我们只见过它的铅笔草图;或者听到那么一首配器华丽的乐曲,过去我们只听过它的钢琴演奏。外祖父指着当松维尔的花篱叫我,他说:“你是爱山楂花的,看看这株桃红色的刺山楂,多漂亮!”确实,这是棵刺山楂,但它是桃红色的,比白色的更美。它也穿了一身节日盛装,是真正的节日盛装啊!只有宗教节日才算真正的节日,不像世俗节日随便由谁胡乱定在某一天,既无节可庆,基本上又无庆可言的;然而,它那身打扮更富丽,因为层层叠叠缀满枝头的花朵,使满树像洛可可风格的花哨的权杖,没有一处不装点得花团锦簇,而且,更因为这些花是“有色”的,所以根据贡布雷的美学观点,它们的质地更为优良,这从市中心广场各家商店、乃至于加米杂货铺的售价贵贱即可窥其一斑:桃红色的饼干不是比别的饼干贵些吗。我自己也一样,认为抹上红色果酱的干酪更值钱,其实这无非是他们答应把捣烂的草莓浇在干酪上面罢了。而眼前的这株山楂偏偏选中了这样一种食品的颜色,这样一种使节日盛装更加艳丽的颜色(因为它让节日盛装显得品位更高雅)。这类颜色因为艳丽,在孩子们看来,仿佛格外美丽,也正因为如此,他们才觉得比别的颜色更充满生气,更自然,即使他们认识到颜色本身既不能解馋,也不会被裁缝选作衣料。自不待言,看到这些山楂花,我除了更加惊喜之外,同看到白色的山楂花一样,分明地感觉到它的喜气洋洋中并无丝毫的矫揉造作,没有人为加工的痕迹,全是大自然自发的流露,那种天真可掬之态,可与村中为在街旁搭一张迎圣祭台而奔忙的女商人,把满树堆砌,弄得既豪华又有乡土气的颜色过于娇艳的花朵相比。树冠的枝梢,像遇到盛大节日供在祭台上的,外面裹着纸质花边的一盆盆盆栽玫瑰,细长的梢头缀满了千百颗淡红的蓓蕾,有的已含苞初绽,好比一盏桃红色的石杯,让人影影绰绰地看出杯心的一点殷红,它们比花朵本身更透出刺山楂的特殊的精神和不可违拗的品性,它不论在哪里发芽,不论在哪里开花,只能是桃红色的;它挤在花篱之间跟盛装的姑娘跻身于只穿家常便服、不准备外出的妇女们之中一样;它已经为迎接“玛丽月”作好一切准备,甚至仿佛已经成为庆典的一部分;它穿着鲜艳的浅红色盛装,那样神采奕奕,笑容可掬——这株信奉天主的、娇美可爱的小树啊!
\par 花篱扶疏间,可以隐约看到园内有一条花草夹道的小径,除茉莉、三色堇和韭叶兰之外,还有紫罗兰打开了它们的钱包,像科尔多瓦\footnote{科尔多瓦:西班牙城市,以生产皮件著称。}的古老的皮件散播着芳香,颜色近似凋谢的玫瑰;一条长长的水管盘旋在砾石铺就的台阶上,扎满小孔的喷头在香气被水润透的鲜花的上面垂直地展开一面由彩色水珠组成的棱镜般的团扇。忽然,我惊得无法动弹了,仿佛眼前的景象不仅呈现于我们的视觉,还要求我们以整个身心来作更深入的感应。一位头发黄得发红的少女,显然刚散步归来,她手里拿着一把花铲,仰着布满雀斑的脸在看我们。她的黑眼珠炯炯闪亮,由于我当时不会、后来也没有学会把一个强烈的印象进行客观的归纳,由于我如同人们所说的,没有足够的“观察力”以得出眼珠颜色的概念,以致在很长一段时期内,每当我一想到她,因为她既然是黄头发,我便把记忆中的那双闪亮的眼睛想当然地记成了蓝色。结果,也许她若没有那样一双让人乍一见无不称奇的黑眼睛,我恐怕还不至于像当年那样地特别钟情于她的那双被我想成是蓝色的黑眼睛呢。
\par 我望着她,我的目光起先不是代替眼睛说话,而只是为我的惊呆而惶惑的感官提供一个伏栏观望的窗口,那目光简直想扑上去抚摸、捕捉所看到的躯体,并把它和灵魂一起掠走;接着,我担心我的外祖父和我的父亲随时都可能发现她,会叫我过去,让我离开她,于是我的目光不自觉地变得蛮横起来,硬是强迫她注意我,认识我!她却把目光朝前一看又往边上一瞟,看到了我的外祖父和我的父亲。她一定认为我们不值一理,所以她扭过脸去,冷淡而傲慢地侧身,使自己的容颜不留在我们的视线之内。但是我的外祖父和我的父亲并没有看见她,他们在继续往前走;于是她斜眼朝我望来。她没有特别的表情,甚至显得视而不见,但眉宇间有一种含而不露的微笑,两眼盯着我看。据我所掌握的有关礼貌方面的知识,她那种表情只能被认为是肆无忌惮的蔑视;她同时又做了个不体面的手势,根据我记忆中的那些交际标准解释,公然向不认识的人做出这种手势,只有一个含义,那就是故意侮慢。
\par “快啊,希尔贝特,快来,你在干什么呢?”一位我从来没有看见过的太太,穿着一身白色的衣裙,用权威的口吻,尖声地叫道。离她不远,还有一位我不认识的先生,身穿斜纹便装,盯着我看;他那对眼珠子简直像要从眼眶里蹿出来似的;小姑娘顿时收敛了笑容,拿着铲子走开了,也没有回头看我,她显得那么听话,那么有城府,让人捉摸不透。
\par 就这样,希尔贝特的名字传到了我的耳畔,简直像符咒一般,刹那间把一个模糊不清的形象变成了一个活生生的人,也许有一天还能使我重新见到她。就这样,这名字传了过来,就像绿色的喷水管中喷出的水珠,那样尖利、那样沁人心脾地洒在茉莉和紫丁香的花丛之上;它用纯洁的空气渗透它所经过的地区,并以缤纷的虹彩笼罩那个地区,它还以它所指的那位姑娘的神秘生活,把那个地区隔绝起来,成为有幸同她一起生活、一起旅游的人们专有的禁地;这一声呼唤在山楂花下,在我的肩头,表明了他们亲密的关系,表明他们同她、同她神秘的生活是亲密无间的,我更觉痛心,因为我无法进入那个神秘的天地。
\par 有那么一小会儿(当时我们正在走开去,我的外祖父悄声说,“斯万也怪可怜的,他们让他扮演什么角色!故意把他打发走,让她好跟夏吕斯厮混,那男的就是夏吕斯,我认得!还有那个小姑娘,也参与进这类丑事当中!”)我忽然产生如下的印象:希尔贝特的母亲口气那么厉害,她都不敢顶嘴,说明她并非高不可攀,也得听命于人;这个印象减轻了一点我的痛苦,给了我些许希望,也使我的爱恋之情有所收敛。但是,这种爱恋之情很快又在我的内心升腾起来,仿佛是一种反应,我的受到委屈的心想通过这一反应来同希尔贝特并起并坐,或者把她也贬到同样的水平。我爱她,我后悔当时没有来得及想到什么妙语气气她,让她伤心,迫使她记得我。我觉得她很美,所以我恨不能转身回去,耸耸肩膀对她喊一声:“您真丑,瞧您这怪样,叫我恶心!”然而,我没有这样做,只是走开了,心里留下了这个红头发、皮肤上布满红色雀斑、手里拿着一把铲子、笑着向我投来呆板而隐含深意的目光的少女的形象,并把它作为我这样年龄的孩子因无法违拗自然法则而不能得到的某种幸福的首例。她的名字在我和她一起听到呼喊的那片桃红色的山楂花下留下了芳香,这名字的魅力还将征服同它接近的一切;我的外祖父母有幸结识并没齿不忘的她的祖父母,崇高的经纪人的职业,以及她在巴黎居住的香榭丽舍大街的那个令人断肠的地区,都因与她有关而增光添彩。
\par “莱奥妮,”我的外祖父一回到家里便说道,“刚才你要是能跟我们一起散步才好呢。你一定不认得当松维尔了。可惜我不敢,不然我就折一枝你那么喜欢的桃红色的山楂花带回来送给你了。”我的外祖父跟我的莱奥妮姨妈讲述我们在散步中的见闻,既是为了哄她高兴,也许还因为我们没有完全失去希望,盼望哪一天能怂恿她下床,出门走走,况且我姨妈原先很喜欢斯万的那个宅院,斯万是她接见的最后一位客人,那时她早已闭门谢客了。而如今,倘若斯万前来探问她的近况(她是我们家唯一的斯万还要求见见的人),她会让人回话说,她累了,请他下次再来;同样,那天晚上,她听罢外祖父的叙述,便说:“是啊,等哪天天气好,我坐车去那儿的花园门口看看。”她这么说倒是诚心诚意的。她很想再见见斯万,重睹当松维尔的芳华;但是,她力不从心,真要这么做恐怕会累垮的。有时候,天气晴朗,她的精力多少充沛些,她起床梳妆;可是还没有跨出门槛她就感到累了,忙着要上床。在她身上,已经出现“人到老年万事休”的心境——只是比一般人来得早而已。她什么事都无心去做,只等着死亡临头,早早地把自己像蚕蛹一样地裹在茧中。我们可以看到,有些人寿命很长,但在他们的晚年,即使当年曾是形影不离的情侣,即使当年曾是心心相印的密友,到了一定年纪,他们也不再为聚首而离家远行,甚至不再互致信札,他们认定了在这尘世间他们已无心曲可通。我的姨妈大概也心中有数,她不会再见到斯万,不会再出门,但是这种我们可能觉得痛苦难忍的幽闭生活,她大概倒认为是合情合理的,因为她精力衰退,每天都感到困顿不济,不得不画地为牢约束自己;她每做一件事,每有一个举动,即使不感到痛苦,至少也感到吃力,这样,不活动、与世隔绝、悄悄度日,她反倒能得到摄生养息的舒适和悠闲。
\par 我的姨妈没有去看桃红色山楂花堆艳叠锦的花篱,但是,我每次都要问我的长辈:她会不会去?她从前是不是常去当松维尔?我想方设法抓住机会让他们提到斯万小姐的父母和祖父母,因为他们在我的心目中跟神仙一样伟大。斯万这个姓对我简直具有神话般的色彩,我跟我的长辈聊天的时候,我如饥似渴地盼望他们提到这个姓氏,虽然我自己不敢把它叫出口,但是我拐弯抹角地引导他们触及同希尔贝特和她的家族有点关系、甚至牵涉到她本人的一些话题,好让我感到离她不至于太远;我有时会突然迫使父亲开口,譬如说,我假装以为外祖父的职务早就是我们家祖传的行业,或者假装以为莱奥妮姨妈想要去看的那座花篱是在公家的地界内,我的父亲就会纠正我的说法,告诉我:“不对,这个职务原先是由斯万的父亲承担的,那座花篱在斯万家的花园里。”于是,我不得不狠狠地吸一口气,因为斯万这个姓,沉重地压在我心中永远铭记的那个部位,使我透不过气来,每当我听到它,总觉得它比别的一切更丰满;它之所以特别有分量,是因我每次都早已在心中呼唤过千遍万遍。它引起我一种快感;我深感愧疚的是竟敢向我的长辈们索取这种快感。由于这种快感如此巨大,他们得耗费许多精力才能使我得到,而他们并不能得到补偿,因为对于他们来说,这并无快乐可言。所以,我往往转移话题。出于谨慎,也出于顾忌。但是,当他们一说出斯万两字,我赋予这个姓氏的种种特殊的诱惑力又都活跃起来。那时,我突然感到,我的长辈们对它的魅力也不能无所感触,他们甚至站到了我的立场,发现我的着迷之处,不仅不责怪我,甚至同我共鸣,我简直就像把他们征服、把他们带坏似的感到无比内疚。
\par 那一年,我的父母比往常早得多地决定了回巴黎的日子,动身的那天早晨,为了照相,他们给我卷了头发,并小心翼翼地给我戴了一顶我从未戴过的帽子,给我穿了一件丝绒的外套。我的母亲到处找我,终于在与当松维尔相接的小陡坡上找到了我。当时我正流着眼泪,搂住了长满尖刺的树枝在向山楂树告别,而且,我跟悲剧中的王妃那样,只觉得无用的衣饰是不堪忍受的负担,把我的头发做成堆在额前的小鬈鬈,实在是多此一举,我并不感恩,反而恨恨地扯掉卷发纸,把它们同我的那顶崭新的帽子一起踩在脚下\footnote{这里,普鲁斯特间接地引用了拉辛的悲剧《淮德拉》中的台词:“这无用的衣饰,这层层的纱,压得我好苦!是谁以多事的手给我把头发卷成这样,并细心地把发卷优美地堆在额前?”(第一幕第三场)}。我的母亲并没有因为我流泪而感动,她看到我的帽子被踩扁了,我的外套给糟蹋了,不禁叫出声来。我听不见她的叫喊,只顾哭着说道:“我可怜的小山楂树啊,不是你们使我伤心,逼我走。你们从来也不让我痛苦!所以我将永远爱你们。”我一面擦着眼泪,一面对它们许愿说,我长大之后,决不像别人那样荒唐地过日子,即使在巴黎,遇到春天,我也不去拜客,不去听那些无聊的敷衍,而是要到乡下来探望第一批开花的山楂树。
\par 我们去梅塞格利丝那边散步时,一走进田野,就再也离不开田野了。风好像通过一条无形的小路,无时无刻不把田野吹遍,我觉得风是贡布雷独有的神仙。每年,我们一到贡布雷,为了切实感受一下我确已身临其地,我总要登高去寻觅风的足迹。它在犁沟里跑着,叫我跟在后面追赶,在梅塞格利丝那边,在那片鼓鼓溜溜的、几十里都不见沟壑的平原上,风总在人们的身边吹拂。我听说斯万小姐经常去朗市住几天,虽然离这儿有几十里之遥,由于中间没有阻隔,距离也就相对地缩短了。炎热的下午,我看到那同一股轻风从极目处吹来,把远方的麦梢压弯,然后像起伏的波浪驰遍寥廓的田野,接着它暖暖乎乎地、悄声细语地伏到我脚下的野草丛中。我与她共有的这一片平原仿佛使我们更接近,把我们联结在一起。我当时想,这股轻风曾从她的身边吹过,风的悄声细语传来了她的某些消息,只是我听不懂罢了。所以,风吹拂过我的跟前时我拥抱了它。左边有一个村庄,叫尚比欧村(本堂神甫称它为Campus Pagani——异教庄)。右边,在一片麦田的上面,遥遥可见圣安德烈教堂的两座钟楼,雕琢得很精致,颇有乡土风味,它们也跟麦穗似的,尖尖翘翘,瓦片蜂窝般地一格格紧扣成行,像正在变黄的麦粒。
\par 苹果树的树叶,长得与其他果树不同,一般人不会认错;在绿叶的衬托下,枝头间距对称地绽开一团团宽瓣的、白缎般发亮的花朵,或者半悬着一簇簇羞红的、欲开还闭的蓓蕾。在梅塞格利丝那边,我第一次注意到苹果树在阳光明媚的大地留下圆圆的树荫,夕阳在树叶下面斜投下一丝丝金线;我看到父亲用手杖截断那丝丝金线,而它们却宁折不弯。
\par 有时,下午的天空中出现苍白的月亮,像一朵白云在悄悄地运行,没有光泽,好比没有登台的女演员,穿着平时的服装,不事声张地悄悄坐在剧场里看看同行的演出,但愿不引人注意。我喜欢在画上、在书中见到月亮的形象,但是当年我所欣赏的那些艺术作品,与今天我觉得把月亮描绘得很美、甚至都认不出那是月亮的艺术作品,有多大的不同呀——至少在早年,在布洛克打开我的眼界,使我的思维更倾向于纤细的和谐之前是这样的。那些作品,例如森蒂纳的某部小说,格莱尔的某幅风景画,把月亮描绘成清晰地悬挂在天空的一弯银镰。诸如此类的作品同我自己心目中的印象一样地稚拙粗俗,我外祖母的两位妹妹见到我喜欢这类作品就很生气。她们认为,给孩子们看的作品,孩子们看后由衷地表现出欣赏趣味的作品,应该是一个人成年之后仍叹赏不已的作品。在他们的心目中美学价值一定是同具体的物质一样,眼睛一看便能感受到它的存在,不必在内心经过一些等价物的耳濡目染,慢慢酝酿成熟。
\par 凡德伊先生在蒙舒凡的住宅,面临一潭深涧、背靠灌木丛生的山坡,就在去梅塞格利丝那边的路上。所以,我们常在散步时遇到他的女儿驾驶一辆轻便货车飞快地从我们身边驰过。近年来,我们见她已不再独来独往,总有一位年纪比她大的女友陪伴着她,那人在这一带名声不好,后来搬到蒙舒凡定居。大家都说:“凡德伊先生准是被那女人的甜言蜜语迷住了心窍,才听不到人家背后的议论。他平时听到一句不得体的话都会面红耳赤的,如今居然允许自己的女儿跟那样的女人在家里出出进进,还说那女人不平凡,感情丰富,在音乐方面更有不同寻常的才情,可惜她过去没有得到发挥。他可能明明知道那女人并不关心他女儿的音乐修养,而是教唆她干别的事。”凡德伊先生倒真是这么说过;事实上,一个人凡同谁有过肉体上的关系,总能使那个人的亲属对他(或她)的精神品质产生由衷的钦佩。肉体之爱尽管受到那样不公正的诋毁,却能迫使每一个落入情网的人把内心的善良和献身精神表现得淋漓尽致,让他(或她)的亲朋好友感到光彩夺目。贝斯比埃大夫多亏他那副大脑门和那两条浓眉,可以随心所欲地扮演坏蛋,但他的模样却根本不像,所以不会有损于他作为大好人的不可动摇、但名不副实的声誉。他用粗鲁的语气说了下面这番话,巧妙地把本堂神甫和大伙儿逗得笑出了眼泪:“敢情!据说这娘儿们跟她的朋友凡德伊小姐在搞音乐。看来真让您感到意外。我反正不知底细。昨天,那个当爸爸的还跟我这么说呢。怎么说,那丫头爱好音乐没错,我不赞成压抑孩子的艺术天分。显然,凡德伊也不赞成,况且他自己还跟他女儿的女朋友一起玩音乐呢。哈!天晓得。他们家成了音乐窝了。你们笑什么呀?只是那帮人音乐玩得太过分。那天我在公墓附近遇到凡德伊老先生。他腿力不济,都站不稳了。”
\par 那一阵,我们发觉凡德伊先生遇到熟人便躲避,只要远远瞅见熟人,他就绕道走开;几个月里他明显地老了许多,愁眉苦脸。凡跟他女儿的幸福没有直接关系的事,他一概无心过问;他经常整天整天徘徊在亡妻的坟前。显而易见,他内心痛苦得要死;谁都不难推测,他对于流言蜚语并非一无所闻。他全都知道,还甚至相信这是事实。对于一般人来说,无论他的德操有多么高洁,遇到纠缠不清的情况,也许只能安之若素地同他一向深恶痛绝的劣迹朝夕相处,因为他无法识破那些披着伪装的劣迹,因为它们都是以特殊的形式出现在他的眼前的,他感到难受,却又无法判定:例如,某天晚上,他耳闻一些莫名其妙的话,目睹一些难以理解的举动,而说这些话、作这些举动的人,偏偏是他有种种理由应予以爱怜的人。但是,要逆来顺受,处于一般人错误地认为唯独吉卜赛人才有的那种处境,对于像凡德伊先生这样的人来说,会比别人更感到痛苦得多。癖好是自然天性在孩子身上诱发出来的东西,有时甚至只需调和父母的德操,就像调和孩子眼睛的颜色那样,便能诱发出一种癖好来,而每当这种癖好需要必不可少的场合和起码的安全时,就会出现吉卜赛人那样的处境。不过,凡德伊先生或许对他女儿的行为有所了解,他对于女儿的宠爱却并不因此而稍减。事实钻不进我们的信念的领域,既不会产生信念,也不会摧毁信念;它们尽管持之以恒地驳斥我们的信念,却不能动摇我们人的信念;倘若谁家连续遭难,疾病灾祸不断降临,也决不会使这家人怀疑上帝的仁慈和医生的高明。但是,当凡德伊先生以一般人的观点从名声的角度,为自己和自己的女儿着想时,当他力图使自己同女儿一起跻身于受到普遍尊敬的人们的行列,他就不免有社会成见,同贡布雷最敌视他的居民所抱的成见毫无二致,他发觉自己已经同女儿一起沉沦到最为人不齿的末流,于是他的举止近来变得自卑、谦恭,见到谁都像从下贱之处仰慕高高在上的贵人(尽管有人过去比他卑下得多),而且他还表现出一种竭力高攀的倾向,这是一切落魄的人必然会有的一种机械反应。有一天我们正同斯万先生在贡布雷镇上的一条街上走着,从另一条街上出来的凡德伊先生猛不防同我们迎面遇上,他不及躲避,斯万先生便同他聊了好久。斯万先生是那种见过世面的上流人,言谈举止透出体恤下情的仁慈,他不仅能把自己的道德偏见统统消除,还能从别人蒙羞的处境中找到可以宽恕的理由。这种宽厚的表示,他自己比受惠者更感到难能可贵,从而他的自尊心得到极大的满足。过去,他从未同凡德伊先生交谈过,今天,他在向我们告辞之前居然问凡德伊先生,能不能让他的女儿去当松维尔玩玩。这样的邀请在两年前肯定会使凡德伊先生大为恼怒的,可是今天他却为之感激涕零,并由此而认为自己受之有愧,切不可不知深浅地接受。他觉得斯万先生对她女儿如此厚道,这本身就是对他的一种体面的、亲切的支持;他想或许不乘机利用为好,心领他的好意岂不更美吗?
\par “他多风雅啊。”斯万向我们告辞之后,他连声叹道,那口气就像伶俐漂亮的平民女子对一位公爵夫人的风度佩服得五体投地似的,尽管公爵夫人又丑又老,她却打心眼儿里仰慕。凡德伊先生也怀有同样的激动。“他多风雅啊!可惜他同一个门户不当的女人结了婚,真令人痛心!”
\par 当时,最真挚的人言谈中也不免掺杂许多虚情假意,跟这个人说话的时候,总把对他的看法忘得一干二净,等他一走,又赶紧对他评头论足。我的长辈们同凡德伊先生一起惋惜斯万的婚姻不当,说它背离原则,不合规矩(他们甚至同凡德伊先生一起提到了那些原则和规矩,以表示他们跟他一样,都是规矩人),显然,言下之意,认为凡德伊先生家倒从没有类似的越轨行径。凡德伊先生没有让他女儿上斯万家去玩。倒是斯万先生因此而感到遗憾,因为,每当他遇到凡德伊先生,临分手时总要问问某一位也姓凡德伊的人的近况,他认为那人准是凡德伊先生的本家。临了,他还总不忘记问一句要紧话:什么时候凡德伊先生准备带他的千金光临当松维尔?
\par 由于去梅塞格利丝那边散步是我们到贡布雷镇外散步的两条路线中较短的一条路线,所以我们总在天气变化不定的日子才去,于是梅塞格利丝那边的天气经常是潮湿的,而我们的眼光也始终盯住鲁森维尔森林中的那片空地;森林里枝繁叶茂,必要时我们可以去避雨。
\par 经常是太阳藏在一片云彩的后面,云彩使太阳的脸庞改变模样,太阳又把云彩的边缘抹上黄色。田野虽依然明亮,但没有光彩,草木生灵似乎都悬在半空,鲁森维尔那边的小村落在天边精致而细密地刻下一幅鳞次栉比的白色屋脊的浮雕。一阵轻风惊起一只乌鸦,它扑扑地飞到远处又重新落下,远处白垩垩的天空把树林衬托得更加清幽,像老式房子里点缀炉壁的釉砖,蓝得发亮。
\par 有时候,眼镜铺橱窗里的晴雨表所预告的那场雨终于开始落下,雨点像列队飞翔的候鸟,密集成行地自天而降。它们彼此紧挨着,在迅速的飞驰中,没有一滴离队,每一滴雨水都不仅各守其位,还带动着后面的雨点紧紧地跟上,天色顿时像飞过一群春燕似的暗了下来。我们跑到林中去避雨。阵雨过后,偶尔还掉下几滴懒洋洋慢吞吞的雨点,我们也顾不得了,只管走出树林,因为那种雨点只在树叶间嬉戏。地上几乎已经干了,而树上倒还有不止一点两点在叶脉间追逐,或者挂在叶尖休息,迎着阳光闪烁,冷不防地从它停歇的枝头落下,滴到我们的脸上。
\par 我们还经常慌慌张张地跑到圣安德烈教堂的门廊下同圣徒和长老们的石雕塑像在一起避雨。那座教堂的法国风味多浓烈呀!门上的圣徒、国王、骑士,各人手执一枝百合花,或参加婚典,或出席葬礼,都惟妙惟肖地表现出在弗朗索瓦丝心目中他们所应有的那种神情。当年的雕塑师还刻画了亚里士多德和维吉尔作品中的故事场面,但是,手法上却与弗朗索瓦丝在厨房里随口提到圣路易往事的语气相仿,听她的语气好似她本人认识圣路易,对他的为人了如指掌,而且一般来说,提到他总是为了把他跟我的外祖父母作对比,照她看,我的外祖父母不如圣路易“公正”。我们可以感觉到,中世纪的石雕艺术家和中世纪的这位(一直活到十九世纪为我们掌勺烹调)女农民对于古代历史或基督教历史的概念,显然都既不准确又朴实单纯,他们的历史知识不是从书本中得来的,而是直接来自古老的、在口头代代相传、世世接续的传说,原先的模样虽说已经难以辨认,但它始终具有活跃的生命力。我从中认出另一位贡布雷的人物,他也在圣安德烈教堂的哥特时代的雕塑群像中得到了预示,那就是加米杂货铺的小伙计,年轻的戴奥多尔。弗朗索瓦丝居然也感到他是本乡本土、古道热肠的牢靠人,所以,当我的莱奥妮姨妈病情加重,弗朗索瓦丝单独一人已无法帮她翻身,抱她坐到靠椅上去的时候,她宁可去叫戴奥多尔帮忙,也决不让帮厨女工上楼去“讨好”我的姨妈。而那位平日被人们公正地看做捣蛋鬼的小伙子,内心却充满了圣安德烈教堂浮雕里的精神,尤其是充满了弗朗索瓦丝认为对“可怜的病人”、对她的“可怜的女东家”应该怀有的那种敬爱之情。他把我的姨妈的头扶上枕头的时候,脸上的表情既天真又热忱,跟浮雕中手持蜡烛围绕在虚弱的圣母跟前的天使一样,仿佛那些灰秃秃的石雕的面容跟冬天的树木似的,不过暂时处于一种休眠状态而已,早晚会在像戴奥多尔那样既敬畏神尊又透出狡猾、红得好比熟透的苹果似的千百张老百姓的脸上重新焕发出奕奕的生气。有一位女圣徒的形象,已经不再像那些天使一样依附在石头上了,而是从门廊的群像中脱身而出;她的身材比真人高大,端立在一座石基上,仿佛站在一张板凳上免得双脚沾到潮湿的土地似的;她的面容丰满,结实的乳房鼓起了胸口的衣衫,像装在麻袋里的成熟的果实;狭窄的脑门,短小而淘气的鼻子,深陷的眼窝,活脱是一副当地农家女的健壮、粗犷而泼辣的模样。造型上的这种惟妙惟肖,给塑像精微入理地注入一种我原先没有期望看到的柔美的情致。经常有几位村姑也像我们一样前来避雨,她们的音容体态更佐证了塑像造型的准确,正如在石刻的枝叶旁边的缝隙里长出的野枝野叶,仿佛有意要跟雕塑作个对比,以使人认识到艺术作品刻画得多么逼真。在我们的正前方,鲁森维尔遥遥可见,那儿是一片乐土呢,还是遭到天罚的罪恶之地?反正我从来没有进去过,有时我们这儿的雨已经停歇,鲁森维尔仍继续像《旧约》里说到的那个村子一样受到暴雨的惩罚,如注的雨水像一条条鞭子抽打着城里居民的房屋,有时它又得到了上帝的宽恕,重新露面的太阳把像祭台圣器上反光一样的长短不齐的金色光芒流苏般地垂到鲁森维尔的城头。
\par 有几次天气坏得无以复加,我们只能回家或者索性闭门不出。无论哪边的田野都阴沉沉、湿漉漉的,远远望去直如茫茫大海,几幢孤零零的房屋依附在黑暗和雨水半淹的山坡上,像一条条收起船帆的小舟在泛光,一动不动地停泊在茫茫夜海中,下一场雨,甚至下一场狂风暴雨更有何妨!夏天,恶劣的天气不过是晴朗天气的一时的脾气,表面的阴沉掩盖不住潜在的、固有的晴朗;同冬天的不稳定的晴朗大不一样,夏天的晴朗则在地上扎了根,化作茂密的枝叶;雨水滴在枝叶上,并不能损害枝叶的欣欣向荣,整个夏季,晴朗的天气把它的淡紫色或白色的旌旗插遍村里的大街小巷,招摇在房舍和花园的墙头。我坐在小客厅里读书,等着吃晚饭,听到如注的雨水从花园里的醋栗树上滴下,我知道瓢泼大雨只是使树叶更滋润、更油亮,那些树就像是夏天的抵押品,整夜经受着雨淋,为的是确保晴朗天气的延续不断;我知道,尽管下雨,明天当松维尔的白色栅墙上,心形的丁香叶依然会茂密地摇摆不停;我遥遥见到贝尚街的那棵杨树在暴风雨中痛苦而绝望地挣扎,我并不感到忧伤;我听到滚滚的响雷在花园那头的丁香树丛中驰过,我也不因此而惆怅。
\par 倘若大清早就阴雨不止,我的长辈们就放弃散步,那我也无法出门了。但是后来我习惯于单独一人到梅塞格利丝那边去散步。那年秋天,我们来到贡布雷奔丧,因为我的莱奥妮姨妈终于死了。她的死既证明了认为她所采用的疗法只会使她的健康每况愈下最终致死的说法言之有理,也证明了始终认为她害的不是臆症而是器质性病变的观点才是真知灼见;她这一死,原来的怀疑论者才不得不在事实面前认输。她的死只引起一个人的巨大的悲痛,这个人偏偏是没有文化的粗人。在我的姨妈病重不起的最后十五天中,弗朗索瓦丝日夜守护在她的身边,她不脱衣睡觉,也不让任何人去帮忙照料,直到姨妈下葬,她才与她分手。原来姨妈对弗朗索瓦丝疾言厉色,怀疑她居心叵测,对她常发脾气,使弗朗索瓦丝成天提心吊胆,过去我们以为她对姨妈一定暗怀恨心,此刻我们才知道,她怕姨妈其实是敬畏,是爱慕。那是她的真正的女主人,她在世时,尽打让人无法预料的主意,施让人难以抵挡的花招,但她天生的慈悲心肠,容易动情。如今,这样的女王,这样神秘莫测、至高无上的君主离开了人世,同她相比,我们在弗朗索瓦丝的心目中太渺小了。这以后,我们虽年年到贡布雷去度假,但要过好些年我们在弗朗索瓦丝的心目中才赢得我的姨妈当年享有的威望。那年秋天,我的父母忙于办手续、同公证人和佃户们交谈,很少有空外出;况且偶尔有空,天公又往往不作美,所以就常常让我独自到梅塞格利丝那边去散步。为了挡雨,我披上一件苏格兰大氅,我有意把它搭在肩上,因为我感到弗朗索瓦丝一见到苏格兰花呢上的方格子就会生气,我们无法跟她讲这样的道理,说衣裳的颜色同孝服没有关系,此外,我们对姨妈的死所表现出的悲伤,她也感到不满,因为我们没有举办大规模的丧宴,我们提到姨妈时没有用一种特别的声调,而且我甚至于有时候嘴里还哼哼歌曲。我相信,倘若有哪一本书,根据《罗兰之歌》或者圣安德烈教堂里那些浮雕的场面,提出这类服丧的观点,我会跟弗朗索瓦丝一样,认为非常动听,而且欣然同感的。但是,弗朗索瓦丝就在我的左右,于是总有一个魔鬼唆使我故意气气她,我抓住一点借口,就跟她说:姨妈死了。我之所以难过,是因为她虽然有些可笑之处,但毕竟是个好心肠的人,并不是因为她是我的姨妈;倘若她虽是我的姨妈,但我觉得她很讨厌,那么她死了我也决不会难过——这样的话,如果出现在哪本书里,连我也会觉得大逆不道的。
\par 如果那时弗朗索瓦丝像诗人一样,对于悲痛,对于家庭的悼念,只有一种流动不定的、模糊的意识,对我的那套理论无从对答,只是说:“我也说不清楚。”那我倒会无愧于贝斯比埃大夫的指教,通情达理地对她的自认无知,狠狠地挖苦几句,自鸣得意一番。倘若她又说:“她毕竟跟您沾亲带故,对亲友总还得尊敬才是。”那么我会耸耸肩膀,独自咕哝一句:“我真是好心到家了,跟这样信口雌黄的文盲白费口舌。”就这样,我采取一般人的狭隘观点来判断弗朗索瓦丝的优劣,扮演了那些最鄙视片面思想的君子在生活中遇到婆婆妈妈的场面时最可能扮演的角色。
\par 那年秋天,我觉得散步特别开心,因为我总是读了好几个钟头的书之后才出去散步的。整整一上午,我坐在大厅里读书,读得感到累了,我就把苏格兰大氅往肩上一披,出门散步去。我的身子经过长时间的静止,积累了充沛的活力,需要像被撒出手的陀螺一样,在转悠中消耗积聚的能量。房舍的外墙,当松维尔的花篱,鲁森维尔森林中的树木,蒙舒凡背后的灌木丛,都受到我的雨伞或手杖的抽打,都听到我的欢快的喊叫。这些喊叫,只是一些模糊的感触,还没有在光明中找到归宿,它们等不及得到缓慢而困难的澄清,宁可找一条立即宣泄的捷径。我们对内心的感情所作的所谓的表白,其实大多不过使我们借以解脱,让我们的感受以一种模糊的形式从我们的内心释放出来,而模糊的形式根本不能使我们认识到感受的真谛。当我试图总结一下我在梅塞格利丝那边究竟有何所得,我从意外的景色或者起码引起我感奋的原因中间究竟得到多少细小的新发现时,我不禁想起那年秋天,我散步走到蒙舒凡身后那片灌木丛生的山坡附近,第一次惊讶地发现我们的印象和我们习惯的表白之间有多不协调。我兴高采烈地同风雨搏斗了一个小时之后,来到了蒙舒凡池边一座瓦片覆顶的小屋前,那是凡德伊先生家的园丁放置园艺工具的小屋。太阳又重新露头,它的金色的光辉经过暴雨的洗涤,鲜亮地闪耀在天边,闪耀在枝头、小屋的墙上,以及依然湿润的瓦片和屋脊上。一只母鸡在屋脊上漫步。吹拂而过的风把生长在墙缝里的野草一片片拉平,母鸡身上的羽毛也全都竖立起来,像一簇没有感觉的、轻飘飘的东西似的,听凭来风直吹到羽毛的根部。阳光又使池水像镜子一样反照出池边的景物,小屋的屋顶在水面上形成一块桃红色的斑纹,过去我还从来不曾注意到有这样一块斑纹。我发现水面和墙面泛起苍白的微笑,同天空的微笑遥相呼应;我不禁激动万分,举起我已经收好的雨伞,啧啧地叫好。同时,我感到我不应该只限于叫出含义不清的啧啧声,而应该把我欣喜的根由弄明白。
\par 也是在那一次,我才知道同样的激动并不同时以预定的顺序在每一个人身上产生。这得多谢一位路过的农民,当时他脸色已经不很痛快,我手舞足蹈,差一点把雨伞打到他的脸上,他的脸色就更阴沉了。我高兴地说:“好天气,是不是,出来走走真痛快。”他的反应却很冷淡。后来,每当我看了半天书,有兴致想找人聊聊的时候,我所盼望同我聊聊的朋友总是谈兴已过,但愿别人让他安心看书。倘若我孝心勃发,想到我的父母,并决定做点最能博得他们欢心的事,他们总偏偏在那个时候指责我早已忘记的一件过错,他们偏偏赶在我打算扑上去吻他们的当口对我横加训斥。
\par 有时候,除了孤独给予我的激动外,还有另一种我无法判明的兴奋心情,那是由一种欲望引起的,我盼望眼前突然出现一位农家女子,好让我拥进怀里。在许多完全不同的思绪中间,突然萌生这样的念头,而且我都来不及确切地弄清它的来龙去脉,只觉得随之而来的快感不过是一切思绪所给予我的快感的一种升华。那时我所想到的一切——覆盖着瓦片的屋顶在水面上形成的桃红色的倒影,墙缝里的野草,我早就想去看看的鲁森维尔的村落,森林里的树木,教堂的钟楼,都由于我内心感受到那种新的激荡而具有进一步的价值,因为我认为正是这一切激起了我快感的升华,它像一股强劲的、神秘莫测的顺风,鼓满了我的风帆,仿佛要把我更快地送进这一切的怀抱。但是,盼望有姑娘出现的念头对于我来说固然给妖娆的自然增添某种回肠荡气的魅力,反之,大自然的魅力也让少女过于局限的妩媚得到了扩展。仿佛树木的婀娜也体现了姑娘的美,仿佛远眺所见的自然风光、鲁森维尔的村落、我那年所读过的书,都各有自己的精魂,而那精魂要由姑娘的一吻来传递给我似的,我的想象一经触及我的肉体感受,便取得了蓬勃的活力,它像电流传遍我想象所及的每一个角落,于是我的欲望再也没有局限了。在大自然的怀抱中浮想联翩时经常有这种情况,那时习惯的作用暂时中断,我们对事物的抽象概念也都被抛到一边,我们由衷地相信我们所在的那个地方,生命别具一格,自有它独特的个性,所以,我的欲望所召唤的姑娘对我来说并不是这类人物的一般典型,并不只是女性,而是这片土地的必然的、自然的产物。因为,在那时,凡身外之物,无论大地还是生灵,我都觉得格外可贵、格外重要,具有格外真实的生气;它们在成人的心目中就没有这么可贵、这么真实。而大地呀,生灵呀,那时与我紧紧相连。我想要见到梅塞格利丝或鲁森维尔的农家女,想要见到巴尔贝克的渔家女,正同我想见到梅塞格利丝的风光、巴尔贝克的景物一样。如果我随心所欲地改变她们所处的环境,那么她们可能给予我的愉快就会变得不那么真实,我甚至会对这种愉快失去信任。在巴黎结识一位巴尔贝克的渔家女或一位梅塞格利丝的农家女,简直就像得到我在海滩上从未见过的贝壳,收下一簇我在树林里没有遇到的蕨草,等于把环境给予我的愉快从她给予我的愉快中剔除,然而我想象中的她是被自然美景所簇拥的。倘若我在鲁森维尔的森林中徜徉,却碰不到一位可以拥抱的农家姑娘,那就无法认识森林隐秘的宝藏,无法认识它深层的美。我想象中只见那位姑娘周身披满树叶的投影,她在我的心目中本身就是一株当地生长的植物,只是在品位上比其他植物更高级,她的结构可以使我更深入地领略到当地的气息。我之所以那么轻易地认准这一点(而且相信她为了使我体会更深而给予我的爱抚也是别具一格的,除了她之外,别的姑娘不可能让我体会到那样的愉快),因为我在很长的一段时期内还很幼稚,还没有把赢得各种女人的心、从不同的女人那里得到的愉快加以抽象,还没有把这种愉快概括成一个普遍适用的概念:把不同的女人只看做取得同一愉快的工具,彼此可以任意变换。可是当时,我思想中的这种愉快甚至不是孤立地、与其他事物无关地、自成一格地存在着的,既没有为追求女人而追求的目的,也没有事先感到心乱如麻之类的经验。好似一想到它就能唾手可得;把它称作愉快倒不如称作姑娘的魅力更妥帖;因为我考虑的不是自己,而是如何超脱自己。这种暗自期待的、内在的、隐秘的快感,只在某些时候达到高潮,那就是当我们身旁的哪位姑娘含情脉脉地看着我们,吻我们,引起了我们另外的愉快的时候,那种愉快在我们的感觉中,尤其像一种感激涕零的冲动,感激她的由衷的善意,感激她对我们令人心醉的惠顾;我们把这种善意、这种青睐比作恩典,比作使我们得到满足的幸福。
\par 唉!我枉然地恳求鲁森维尔的塔楼,就像请求我唯一的知心朋友似的,请它让村里的姑娘到我的身边来,因为我在贡布雷家中楼上那间充满菖蒲花芳香的房间内,在那扇半掩半启的格子窗中间,只见到那座钟楼的塔影,我把最初在我内心萌动的种种欲念,都告诉了它;我本像探险的旅行家或者绝望得要自杀的人一样,在做出壮烈举动之前不免踌躇再三,而终于心灰意懒,想从自身中另辟蹊径,却又自以为面临山穷水尽的绝境;忽然,我发现,除了垂到我眼前的那株野生的黑加仑树的枝叶外,还有这样一条像蜗牛行迹似的大自然的脚印。而现在我哀求它,它却不予理睬。我白白地把我眼前的一大片田野盯住不放,我用我的眼光挤压这片田野,想从中挤出一位姑娘来,结果枉费精神。我虽然可以一直走到圣安德烈教堂的门廊下去碰运气,但是我从来只有跟外祖父一起去的时候,才能有把握地遇到农家姑娘,而那时又无法跟她交谈。我心神不定地盯住远方一棵树的树干,盼望从树后走出一位姑娘来,被我目光搜索的远方却始终不见人迹。天色渐暗,我无望地把注意力紧紧地贴住这片贫瘠的土壤,这片枯竭的大地,仿佛要从中吸出可能隐藏着的生灵;我不再兴高采烈、而是恼恨万分地敲打着鲁森维尔森林里的树木,从这些树木间不会走出什么活人来了,仿佛它们只是画在一片环形画布上的形象。我虽然不愿意在没有拥抱到我那么盼望拥抱的姑娘之前就甘心回家,但我毕竟不得不返回贡布雷;我无可奈何地认识到,半路上意外邂逅的可能性是微乎其微的。再说,即使半路上遇到她,我敢同她攀谈吗?我想,她或许会把我当做疯子;我不再相信我在那几次散步中所产生的不现实的欲念会得到别人的共鸣,不再相信这样的欲念在我的内心之外仍是真实的。我只觉得这是我的气质的产物,是纯主观的、无能的、幻觉的创造。这些欲念与大自然、与现实没有任何联系,于是现实失去了它的一切魅力和意蕴,只成了我的实际生活的一个沿袭的框架,正等于坐在车厢里的旅客为了消磨时间看一本小说,车厢就是那本小说的幻想世界的框架。
\par 几年之后我在蒙舒凡附近所产生的印象或许也是这样的,那时印象还很模糊,隔了很远我才猛然想到施虐狂这个概念。最终你会看到,这个印象对我一生起到至关重要的作用,虽然出自别的理由。那一天,天气很热,我的长辈们有事出门,白天回不来,就对我说,我愿多晚回家随我的便。我一直走到蒙舒凡的池塘边,我爱看池水中屋顶的倒影,我躺在以前我父亲拜访凡德伊先生时我在外边等他的那片灌木丛生的山坡上,居然睡着了。等我醒来,天几乎黑了。我正打算爬起来,这时,我看到了凡德伊小姐(至少我当时认为自己认出是她,因为我在贡布雷难得见到她,而且当初她还是个孩子,那时她已经开始长成一位少女了),她准是刚回家,离我才几厘米远,就在我的眼前,就在她父亲曾经接待过我的父亲、她用来当做自己的小客厅的那个房间里。窗户半掩着,房间里已经亮灯,我能看到她的一举一动,她却看不到我;但是我倘若踩响灌木丛的枯枝,她会听到声响,以为我有意躲在那里偷看她呢。
\par 她穿着孝服,因为她的父亲刚去世不久。我们没有去看她,我的母亲出于一种美德才不愿意去看她,对于母亲来说也只有这种美德才能限制她善良的宽宏,那就是廉耻心;不过她还是打心眼儿里可怜凡德伊小姐的。我的母亲念念不忘凡德伊先生凄凉的晚年,他对女儿既像母亲又像女佣那样照顾得无微不至,他的余生,先是为女儿操心,后来又陷入女儿给他引起的痛苦之中;老人在最后几年中满脸愁苦的情状,我的母亲一直历历在目;她知道,凡德伊先生放弃了把自己最后几首作品完整地记在乐谱上的计划,那些虽只是一位钢琴老教师、乡村教堂的管风琴演奏师的惨淡经营之作,本身想必没有多大价值,但我们并不小看它们,因为这些作品对于他来说意义重大,在他为女儿作出牺牲之前,它们曾是他苟活人世的理由,其中大部分甚至连音符都没有记下,只保留在他的脑海中,有一部分则分散地记在一些零碎的纸片上,笔迹不清,肯定要失传了。我的母亲还想到凡德伊先生无可奈何地放弃的另一件事,那就更惨不忍言:他不得不放弃对女儿日后取得既正派又受人尊敬的幸福前程的期望;这件事最伤透我的姨祖母们以前的这位钢琴老师的心,我的母亲一想到事情的来龙去脉,总不免扼腕叹息,她想凡德伊小姐一定也恨恨不已,当然苦涩之情完全不同,凡德伊小姐的伤悼中应夹杂着悔恨,因为她的父亲几乎是被她害死的。“凡德伊先生怪惨的,”我的母亲说,“他为女儿活着,也为女儿而死,却没有得到应有的报答。既然死了,他还能得到什么报答?怎么报答法?只有他的女儿才能报答他的恩情。”
\par 在凡德伊小姐的客厅靠里面那一头的壁炉架上,放着一帧她父亲的遗像。她一听到大路上传来辚辚的车马声,就赶紧过去把遗像拿过来,然后坐到长沙发上,拉过一张小茶几,把遗像放在上面,那情景跟当年凡德伊先生把他想演奏给我的父母听的曲谱放到自己的手边一样。不一会儿,凡德伊小姐的女朋友走进客厅,她打了个招呼,却没有起身,两只手还枕在脑后,而且把身子往沙发的另一头移了一移,仿佛给来客腾出地方坐似的。但是她立刻意识到她似乎应该对来客采取一种也许她自己认为是多余的态度。她想她的朋友可能更愿意坐得离她远些,她感到自己有失检点,敏感的心灵于是警觉起来;她又躺靠在整张沙发上,闭上眼睛,连打哈欠,表示她之所以躺下只是因为她想睡觉了。虽然在她跟那位女朋友的关系中不加掩饰的亲热占了上风,但是我发觉她的言谈举止,仍带有她父亲讲究繁文缛节、闪烁其词的特征;她经常欲言又止,突然拘谨起来。她刚闭上眼睛,又立刻起身,假装想去关窗,偏偏又关不上。
\par “让它开着吧,我热。”她的女友说。
\par “开着多别扭啊,人家会看见咱们的。”凡德伊小姐回答说。
\par 她一定猜到她的朋友会怎么想;她的朋友知道她这么说无非是有意逗她接话,说些她想听的话,但出于谨慎她又不便挑明,而是要对方主动地说出来。所以,当她急急忙忙地补充下面这句话的时候,她的眼神一定出现了当年我的外祖母特别赏识的表情,不过当时我还分辨不出来罢了。她急忙补充的话是:
\par “我说看见咱们,意思是看见咱们读书学习,想到人家的眼睛在瞅着咱们,咱们干什么他都看得一清二楚,这有多别扭呀。”
\par 她本性宽厚,更出于一种不自觉的礼貌,她没有把事先考虑好的话说出口,虽然她认为这些话是圆满实现自己愿望必不可少的。在她的内心深处,任何时候都有一位羞怯而恳切的处女,在哀求一个占了上风的粗鲁的兵痞子不要对她无礼,不要逼近她。
\par “对了,这么晚了,在这样人头挤挤的乡下,倒真会有人看咱们的,”她的女友挖苦道,“看见又怎么样!”她接着说(同时她认为在好心地说出这番话时有必要狡猾地挤挤眼睛,就好比在读一篇她明明知道凡德伊小姐爱听的文章,她偏要拿腔作调,读得玩世不恭),“谁爱看谁就看好了,这不更好吗?”
\par 凡德伊小姐哆嗦了一下,站起来。她那既拘谨又多情的心眼儿不知道该由衷地说些什么话才符合她七情六欲所需要的宣泄。她尽可能地超越自己真正的天性,找些风骚姑娘才说得出口的话来,她真巴望自己是这样的人;可是她自以为说得很自然的话到她嘴边却显得虚假不堪。她敢于说出口的那几句话,口气倒不小,其实很牵强,一向腼腆的习惯使她仅有的一点儿泼辣也无从发挥。只听她讷讷说道:“你既不冷,也不太热,你不愿意一个人待着读什么书吧?”
\par “我觉得小姐,您今天晚上有点儿春心荡漾。”她终于这样说道,大概是重复她曾经从她的女友口中听到过的一句话。
\par 凡德伊小姐感到她的女友在她的乔其纱胸衣的叉口处吻了一下;她像挨到什么东西刺了一下似的轻叫一声,便闪开了。于是两人跳着蹦着地追逐起来,宽大的袖子像翅膀一样在扇动;她们叽叽格格笑得像两只调情的小鸟。后来凡德伊小姐终于倒进沙发,她的女友立刻压在她身上,但是这位女朋友有意把背部扭向放着已故钢琴教师肖像的那张小桌。凡德伊小姐心中有数,除非她提请注意,否则她的女友是决不会理会那帧肖像的。所以她装做刚刚发觉似的对她的女友说:
\par “啊!我父亲的肖像在看着咱们呢!不知道谁又把它放在小桌上了。我说过多少遍,那儿不是放照片的地方。”
\par 我记得当年凡德伊先生关于琴谱也对我的父亲说过同样的话。那帧肖像一定习惯于被她们当做亵渎仪式的工具,因为那位女友的答话看来就是这类仪式的唱和,她说:
\par “让它待着吧!反正他不能再讨咱们的嫌了。你以为那老东西看到你在这儿,看到窗户敞着,还会哭哭啼啼地来给你披上外衣吗?”
\par 凡德伊小姐答道:“得了,得了。”这句稍有谴责之意的答话倒证明了她天性的宽厚,她这么说并不是因为人家用那种口吻谈论她的父亲,她听了生气(显然,不知出于什么奇奇怪怪的逻辑,每逢这样的时候总有一种感情她是习惯于埋在心里而不予表露的),而是因为这么说等于给自己一个约束,她的女友在想方设法给她提供快乐,她为了不显得只顾自己就有意给自己来点约束。然而,这种对亵渎言行的温和的折衷,这种娇声娇气的假怪嗔,对于她坦诚的天性来说,显得特别卑鄙,简直像男盗女娼之流的甜言蜜语;她偏偏想精通这类无耻之道。但是,她无法抗拒快乐的诱惑;有人对她温柔备至,她感到由衷地高兴,偏偏这人对无力自卫的死者如此刻薄。她跳起来坐到她的女友的腿上,天真地把头伸过去给她吻,好像她是她的女儿似的;同时她心花怒放地感到,她们俩这下子要狠心到底,一起到凡德伊先生的坟墓里去盗走他的父爱了。女友捧住凡德伊小姐的脸庞,在额上吻了一下,吻得那样温顺,因为她对凡德伊小姐非常疼爱,她想给如今成了孤儿的少女的凄楚生涯增加一些消愁解忧的乐趣。
\par “你知道我想给这老怪物来点什么吗?”她拿起肖像说道。
\par 她又凑到凡德伊小姐的耳边悄悄说了几句我听不到的话。
\par “哦!你不敢吧?”
\par “我不能啐?往这上面啐?”女友故意恶狠狠地说道。
\par 下文我就听不到了。因为凡德伊小姐无精打采、笨手笨脚、慌慌忙忙、一本正经、愁眉苦脸地过来关上了百叶窗。我总算知道了生前为女儿吃尽种种苦头的凡德伊先生死后得到了女儿什么样的报答。
\par 后来我倒曾经想过,即使凡德伊先生亲眼目睹方才的情景,他对自己女儿心地善良的信念也许照样不会丧失,甚至明明错了他还会坚信不移。当然,在凡德伊小姐日常的行为中,恶的表现极为彻底,一般人难以想象她怎么能坏到这种程度,简直跟施虐狂患者不相上下。让自己的女朋友朝生前一心爱她的父亲的遗像上啐唾沫,此情此景出现在大马路的剧院舞台上倒比出现在名副其实的乡间住宅里更合适。在生活中只有施虐狂才为情节剧提供美学根据。实际上除了施虐狂患者之外,一般姑娘纵然会像凡德伊小姐那样狠心不顾亡父的遗愿和在天之灵,但也不至于有意把自己的狠心概括成那样的一种行为,用那样浅近和直露的象征手法表现出来;在她们的行为中,大逆不道的表现总要隐蔽些,对别人遮掩,甚至自己也看不清楚,干了坏事自己并不承认。但是除了表现之外,在凡德伊小姐的心中至少一开始善恶并不混淆。像她那样的施虐狂都是作恶的艺术家;彻头彻尾的下流坯成不了这样的艺术家,因为对于他们来说恶不是外在的东西,而是天生的品性,同他们无法分离;他们决不会把品德、悼亡和孝顺父母之类看得神圣不可侵犯,所以当他们亵渎这类东西时也感觉不到大逆不道的痛快。而类似凡德伊小姐那样的施虐狂,则是一些单凭感情用事的人,生来就知廉耻,他们甚至对感官享受都视为堕落,当做只有坏人才能享受的特权。他们一旦在操行方面对自己作出让步,一旦放纵自己贪欢片刻,他们也总是尽量让自己和自己的对手钻进坏人的躯壳里去,甚至产生一时的幻觉,以为自己已经逃出拘谨而温顺的灵魂,闯进了一片纵欲的非人世界。我终于明白,凡德伊小姐一方面巴望如此,同时又发觉自己不可能得逞。她想让自己做得同父亲不一样的时候,她的言行偏偏使我想起她父亲的想法和说法。她所亵渎的东西,那夹在她与快乐之间妨碍她直接尝到甜头的东西,她偏要用来为自己取乐出力,这岂止是那帧照片,更是她自己同父亲酷肖的相貌,更是她父亲作为传家宝遗传给她的那双本来长在祖母脸上的蓝眼睛,更是她温文尔雅的举止;这些都在凡德伊小姐和她的劣迹之间横下了一套华丽的辞藻和一种与丑恶的行为格格不入的精神状态,使她认识不到自己的放荡同她平时奉行的许多待人接物的礼数有多大的距离。使她产生寻欢之念的,使她感到快活可心的,不是恶;在她的心目中,快乐倒不是好事。由于她每次纵情求欢所感到的快乐,始终与她贞洁的心灵平时所没有的一些坏思想形影相伴,从而她最终认为快乐之中存在某种邪魔,这种邪魔就是恶。也许凡德伊小姐觉得她的女友本质不坏,认为那些亵渎性语言并非发自她的内心。至少女友高兴吻她的脸,那脸上的微笑和眼神,也许全都是装的,却透露出邪恶的、下流的表情,一个心地善良、忍受痛苦的人决不会有那种表情,倒像生性残忍、贪图快乐的人才有的行状。可能她有过一闪之念,想象自己其实在寻开心,好比一位少女明明对有人野蛮地亵渎自己的亡父深感痛恨,却还在同如此丧尽天良的伙伴鬼混;也许她不至于认为恶是一片世上少有、不同寻常、异域情调的福地洞府,住到里面去有多么逍遥自在,可惜她不能在自己身上以及在别人身上发现对痛苦的麻木。有人故意制造痛苦,人们却对此无动于衷,称之为麻木也罢,称之为别的什么也罢,总之这是残忍的表现,是残忍的可怕的、持久的表现形式。
\par 如果说去梅塞格利丝那边散步是十分轻而易举的事,那么去盖尔芒特家那边散步就另当别论了。因为路程长,先要打听着实天气如何。要去就得等到看上去将有一连几个大晴天的日子;就得等到为“可怜的庄稼”操心的弗朗索瓦丝眼看平静而蔚蓝的天上只飘过几丝白云,对下雨已感绝望,唉声叹气地大声说道:“那几片云像不像把尖嘴探出水面嬉闹的海狗?嗨!它们倒是为种田人着想着想,让老天爷下点雨呀!等麦子长起来之后,雨又要滴滴答答没完没了地下个不停了,它都不知道下在什么上面,好像下在海里似的。”就得等到我的父亲从园丁和晴雨表那里一起得到同样的晴天预报;只有到那时,我们在吃晚饭的时候才会说:“明天倘若还是这样的好天,咱们就去盖尔芒特家那边散步。”第二天午饭吃罢之后,我们马上就走出花园的边门,踏进狭窄的、形成一个锐角的贝尚街。街上长满狗尾草,两三只黄蜂成天在草丛间采集标本,街面同街名一样古怪,我甚至觉得街道稀奇的特征和不近人情的个性全是由古怪的街名衍生而来的。在贡布雷镇,今天已无处寻觅这条街了,故道上盖起了学校。但是,正如维奥莱勒迪克\footnote{维奥莱勒迪克(1814—1879):法国大建筑师,曾负责修缮包括巴黎圣母院在内的许多中世纪建筑,他所编写的《十一至十六世纪法国建筑考据大全》及《文艺复兴以前的法国家具图录》两书,史料翔实,有极高的历史和艺术价值。}门下的学生们认为在文艺复兴时期的祭廊里以及在十七世纪的祭坛下能重新找出罗马时期唱诗班的遗迹,从而把整座建筑恢复到十二世纪时的原貌那样,我的联翩的浮想同样也不让新建筑有片石留下,它在旧址上重新开凿出、并且“按原样恢复”了贝尚街,况且贝尚街有足够的资料供恢复参考,从事古建筑修缮的人一般还掌握不到这样精确的历史资料:我的记忆保存下来的有关我童年时代的贡布雷的一些印象,也许是它仅存的最后的印象了,现在虽还存在,却注定不久会磨灭;正因为这是我童年时代的贡布雷,在自行消失之前,把那些动人的印象刻画在我的心上,好比一幅肖像本身已湮没无闻,但根据它的原作临摹下来的东西却显赫地流传于世一样。我的外祖母就喜欢送我这类作品的复制件,例如早年根据《最后的晚餐》和让迪勒·贝里尼\footnote{让迪勒·贝里尼(1429—1507):意大利威尼斯画派中的贝里尼家族的第二代画师。法国卢浮宫藏有他所作的《基督受难图》等画品。}原作刻制的版画,这些版画保留下了达·芬奇的壁画杰作和圣马克教堂的门楼至今已无处寻觅的原貌。
\par 我们从鸟儿街上的古老的鸟儿客栈门前走过。十七世纪时,蒙邦西埃家、盖尔芒特家和蒙莫朗西家的公爵夫人们的轿车曾驶进客栈的大院,她们来到贡布雷,有时是为了解决与佃户的争端,有时是为了接受佃户的贡奉。我们走上林荫道,圣伊莱尔教堂的钟楼在树木间显现。我真想能在那儿坐上一整天,在悠扬的钟声中埋头读书;因为,天气那样晴朗,环境又那样清幽,当钟声响起来的时候,仿佛它不仅没有打断白天的平静,反而更减轻白日的烦扰,钟楼就像没有其他事情可干的闲人,只管既悠闲又精细地每到一定的时刻分秒不差地前来挤压饱和的寂静,把炎热缓慢地、自然地积累在寂静之中的金色液汁,一点一滴地挤出来。
\par 盖尔芒特家那边最动人的魅力在于维福纳河几乎始终在你的身边流淌。我们第一次过河是在离家十分钟之后,从一条被称作“老桥”的跳板上过去的。我们到达贡布雷的第二天,一般总是复活节,听罢布道,倘若赶上天气晴朗,我就跑来看看这条河。那天上午大家正为过复活节这样盛大节日而忙乱着,准备过节使用的富丽的用品使那些还没有收起来的日常器皿显得更加黯然失色。已由蓝天映得碧绿的河水在依然光秃秃的黑色田亩间流淌着,只有一群早来的杜鹃和几朵提前开放的报春花陪伴着它,偶尔有一茎紫堇噘起蓝色的小嘴,一任含在花盏中的香汁的重量把花茎压弯。走过“老桥”,是一条纤道,每逢夏天,有一棵核桃树的蓝色的枝叶覆盖成荫,树下有一位戴草帽的渔夫,扎下根似的稳坐在那里。在贡布雷,我知道钉马掌的铁匠或杂货铺伙计的个性是藏在教堂侍卫的号衣或唱诗班孩子的白色法衣中的。唯独这位渔夫,我始终没有发现他真正的身份,想必他认识我的长辈,因为我们经过时,他总要抬一抬他的草帽。我本想请教他的姓名,可是总有人比画着不让我出声,怕我惊动正待上钩的鱼。我们走上纤道,下面是几尺高的岸坡。对面的河岸矮,是一片片宽阔的草地,一直延伸到村子边,延伸到远处的火车站。那里到处有贡布雷昔日领主的城堡的残迹,半埋在杂草中。中世纪时维福纳河是贡布雷抵御盖尔芒特的贵族首领和马丁维尔的神甫们进犯的天堑。如今只剩下箭楼的断瓦残砖给草地留下几堆不甚显眼的土包而已,还有几截雉堞围墙,当年弓弩手从那里投射石弹,哨兵从那里监视诺甫篷、克莱尔丰丹、马丁维尔旱地、巴约免赋地等盖尔芒特家族管辖下一切属地的动静,它们当年把贡布雷夹在中间。昔日的属地早已夷为平地,在这里称王称霸的已是教会学校的孩子,他们到这里来学习功课或作课间游戏。几乎已经埋入地下的往事像散步的人中途纳凉似的躺在河边,却使我浮想联翩,使我觉得贡布雷的这个名字的内涵不仅指今日的小镇,还包括另一座完全不同的城池,它那半埋在金盏花下的不可思议的昔日风貌牢牢地攫住了我的思绪。这里的金盏花多得数不清,它们选择这片土地,在草上追逐嬉戏;它们有的孤然独立,有的成对成双,有的结伴成群;它们黄得像蛋黄,而且光泽照人,尤其因为我感到它们只能饱我以眼福,却无法飨我以口福,我便把观赏的快乐积聚在它们的金光闪烁的表面,终于使这种快乐变得相当强烈,足以产生出一些不求实惠的美感来。我自幼年时起就这样做了:我从纤道上向它们伸出双手,我还叫不全它们的名字,只觉得跟法国童话里的王子们的名字一样漂亮动听;它们也许是几百年前从亚洲迁来的,但早已在村子里落户定居;它们对清贫的环境很知足,喜欢这里的太阳和河岸,对于远眺所及的车站的不起眼的景色,它们也决无二心,同时它们还像我们某些古画那样在稚拙纯朴中保留着东方的诗意的光辉。
\par 我兴致勃勃地观看顽童们放进维福纳河里用来装鱼的玻璃瓶。每只瓶里装满了河水,河水又把瓶子紧紧裹住;它们既是四壁透明得像是由一种凝固的清水做成的“容器”,同时又是沉进了一个更大的,由流动着的晶体做成的容器里的“内容”;它们在这里比在餐桌上更沁人心脾、更撩人欲念地体现出清凉的形象,因为在餐桌上,瓶水的清凉的形象始终只流溢在水和玻璃之间,我们的手不能在清淡的水中捕捉到清凉的形象,而我们的上腭也无法从凝固的玻璃中品尝到清凉的滋味。我打算以后再来时带上渔竿;我从野餐篮里面撕下了一块面包,把它搓成一团一团,扔进维福纳河,看来这足以在水中造成一种超饱和现象,因为河水立刻凝固了,在面包团四周无数细小的蝌蚪,凝聚成一个个椭圆形的小球,原先这些蝌蚪一定是散布在河水里的,肉眼看不到,但密度已达到结晶的临界线。
\par 不久,维福纳河的水流被水生植物堵塞了。起初,河里先是长出几株孤零零的水草,例如有那样一支水浮莲,水流从它的身边流过,可怜它在水流中间,很少得到安宁;水流把它从这边的岸沿冲到那边的岸沿,它像一艘机动渡船一样,无休无止地往返在两岸之间。被推向岸边的水浮莲的株茎,舒展,伸长,绷紧,以至于达到张力的极限;飘到岸边以后,水流又把它往回拉,绿色的株茎又开始收拢,把可怜的植物重新引回到姑且称之为它出发的地点,可安生不了一秒钟,它又得被反复地带来带去。我一次又一次地在散步时见到它,它总是处于同样的境地,这使我想起某些神经质的人(我的外祖父把我的莱奥妮姨妈也算在其中),他们年复一年地让我们看到他们一成不变的古怪习惯,他们每次都声称要加以改变,但始终固守不爽。他们被卡进了不痛快和怪脾气的齿轮之中,纵然使尽气力也难以脱身,只能更加强齿轮的运转,使他们古怪的、劫数难逃的保守疗法像钟摆一样地往复不已。那株水浮莲也是如此,也像这样不幸的病人,他们反复不休、永无止境的古怪的痛苦曾引起但丁的好奇,倘若维吉尔没有大步走开,迫使他不得不快快赶上的话,但丁还会没完没了地要那些受到这种痛苦折磨的人亲自诉说自己的病情和病因的,正如这时我的父母已经走远,我得快快跟上一样。
\par 但是,再往前去,水流渐缓,流经一座业主向公众开放的庄园;主人有偏爱浮莲水草之雅,以此装点庭院,在维福纳河水灌注的一片片池塘中,群莲争艳,真成了名实相符的赏莲园。这一带两岸树木葱茏,团团浓荫通常把水面映得碧绿,但有几次暴雨过后,黄昏分外恬静,归途中我发现河水蓝得透亮,近似淡紫,仿佛涂上了一层日本风格的彩釉。水面上疏疏落落地点缀着几朵像草莓一般光艳的红莲,花蕊红得发紫,花瓣边缘呈白色。远处的莲花较密,却显得苍白些,不那么光滑,比较粗糙,还有些皱皱巴巴,它们被无意的流水堆积成一团团颇有情趣的花球,真像是一场热闹的游乐会之后,人去园空,花彩带上的玫瑰零落漂浮在水面,一任流水载浮载沉。另有一处,仿佛专门腾出一角供普通的品种繁殖,那里呈现一派香芹的素雅的洁白和淡红,而稍往前看,一簇簇鲜花拥挤在一起,形成一块飘浮在水面的花坛,仿佛花园中的蝴蝶花,像一群真正的蝴蝶,把它们冰晶般透蓝的翅膀,停歇在这片水上花坛的透明的斜面上;说它是水上花坛,其实也是天上花坛,因为这花坛为花朵提供了一片颜色比花朵更富丽、更动人的“土壤”——水面;下午,它在浮生的花朵下像万花筒一般闪烁出其乐融融的、专注、静默和多变的光芒;黄昏,它像远方的港口,充满了夕阳的红晕和梦想,变幻无穷,同时又在色彩比较稳定的花朵的周围,始终与更深沉、更神秘、更飘忽不定的时光,与宇宙的无限取得和谐,在那时,它仿佛让这一切都化作了满天的彩霞。
\par 流出花园之后,维福纳河又滔滔转急。有多少回,我见到一位船夫,放下了船桨仰面躺在船中,听凭小船随波飘荡,他的头枕在船板上,只见到天空在他的上面慢慢地飘移,他的脸上流露出预想幸福和安详的表情;我若能随心所欲地生活,我多想仿效他那样的豁达坦荡啊!
\par 我们坐在岸边的菖蒲花丛中休息。在假日的天空,一朵闲云久久地徘徊。不时有一条闷得发慌的鲤鱼跃出水面,惴惴不安地透一口气。这正是野餐的时间。我们要在这儿呆好久才回家;在草地上吃点水果、面包、巧克力,圣伊莱尔教堂的钟声沿着地平线悠悠传来,声音虽弱,却依然浑厚而铿锵;它们从那么远的地方,穿透一层层的空气,却没有与空气混合,一道道声波的连续的颤动给钟声四周留下一条条棱纹,掠过花朵时发出阵阵共鸣,一直到达我们的脚边。
\par 有时,在林木围绕的水边,我们见到一幢被称作别墅的房屋,孤零零地隐匿在幽僻的地方,只有墙脚下的河流与它相伴。一位少妇独立在窗内,显得若有所思;从她的华丽的面罩来看,她不像本地人。她大约是如俗话所说来这儿“隐身”的。窗外,她所能见到的只有拴在门外的一叶扁舟而已。这地方无人知道她的姓名,尤其是无人知道她曾经爱过但早已无法继续挂在心上的那位男子的姓名,她一定因此而感到既苦涩又高兴。她心不在焉地抬眼望望,先听到岸边的树后有行人经过,然后才看到行人的模样;她可能心中有数,他们以前不认识、将来也不会知道谁是负心人,他们过去对她毫无印象,将来也未必有再见到她的机会。一般人认为,她离群索居,是有意远离能见到心上人的地方,哪怕远远一瞥,她也尽量躲开,故而避到根本没见过那人的这里来。而有一次,我散步回家,经过她明知自己所爱的人决不会出现的那条路,我见到她无可奈何地摘下了自己长长的、华而不实的手套。
\par 我们到盖尔芒特家那边散步,没有一次能走到维福纳河的源头;我经常想到源头去,在我的心目中,它简直是一种很抽象、意念很强的存在,倘若有人告诉我说,这源头就在本省,离贡布雷才多少多少公里,我一定会惊讶万分,其程度等于听人说地球上哪个确切的地点古时候曾是地狱的入口处。我们也从来没有能一直走到我非常想去的终点:盖尔芒特。我知道,那是领主盖尔芒特公爵和夫人的府邸;我知道他们是实际存在的真人,但是,一想到他们,我就时而把他们想象成壁毯上的人物,跟我们教堂里那幅名叫《爱丝苔尔受冕》的壁毯上的盖尔芒特伯爵夫人的形象一样;时而我把他们想象成色调变幻的人物,跟教堂彩色玻璃窗上的“坏家伙希尔贝”似的,我在取圣水的时候,他看上去是菜绿色的,等我在椅子上坐定之后,他又变成了青梅色;时而我把他们想象成完全不可捉摸,跟盖尔芒特家的远祖,热纳维耶夫·德·布拉邦特的形象一样——幻灯曾映照她的形象驰过我房内的帘幔,或者登上房内的天花板。总之,他们总裹着中世纪神秘的外衣,像受到夕阳的沐照似的,沉浸在“芒特”这两个音节所放射出来的橘黄色的光辉之中。但是,尽管如此,作为公爵和公爵夫人,他们在我的心目中毕竟实有其人,虽然他们与众不同,从另一方面来说,他们的公爵身份使他们的形象极度地膨胀,变得虚无缥缈,足以容纳下他们的爵号后面那个显赫世家的姓氏——盖尔芒特,容纳下“盖尔芒特家那边”所有的一切:明媚的阳光,维福纳河,河上的睡莲,岸边的大树,以及那么多晴朗的下午。我知道他们不仅有盖尔芒特公爵和公爵夫人的爵位,从十四世纪起,他们征服贡布雷的企图落空之后,便与大领主联姻,由此分封得到贡布雷的领主权,从而成为贡布雷最早的公民,也是唯独不在贡布雷定居的公民。他们兼任贡布雷伯爵,在他们的姓氏和身份中加进了贡布雷的地名,不用说,贡布雷所特有的那种离奇而虔诚的忧伤情调实际上也随之潜入他们的心中;他们是贡布雷市镇的主人,但是他们在镇上没有一所私宅,进入市镇他们大约只能呆在屋外,呆在街上,呆在天地之间,就像圣伊莱尔教堂彩绘玻璃窗上的那个坏家伙希尔贝,当我到加米杂货铺去买盐时,经过教堂的后身,抬头望去,却只能见到彩绘玻璃窗一片漆黑的反面。
\par 后来还有过这样的事情:在盖尔芒特家那边,我有时经过几片潮湿的小庄园,几簇色泽无光的花朵伸出栏外。我驻足停步,自以为得到了一个可贵的概念,因为我觉得眼前仿佛是我自从读到一位心爱的作家有关描述之后便日夜向往的那片河网地带的一角。贝斯比埃大夫曾同我们讲到了盖尔芒特宫堡花园里的花和花园里蜿蜒密布的小溪,我一面听着,一面想到了那位作家所描述的河网地带,想到了那片纵横密布着潺潺流水的虚幻的地方,从而盖尔芒特在我的脑海中改变了形象,我把盖尔芒特同那片虚构的景象等同起来。我想入非非地仿佛觉得盖尔芒特夫人一时心血来潮,对我钟情,邀我去玩;她一整天都陪伴我钓鱼。黄昏时,她拉着我的手,我们从她的家臣们的小花园前走过,沿着低矮的围墙,她指点我看垂挂在墙头的一簇簇紫色和红色的花朵,并告诉我这些花的名称。她要我说出我刻意经营的那些诗篇的主题。这类梦提醒了我:既然我想有朝一日当名作家,现在就该明确打算写什么。但是,我一旦扪心自问,力求找到一个可以容纳无限的哲学意蕴的主题,我的思路便停止了运作,只觉得自己眼前一片空白;我感到自己缺乏天才,也许我的脑子有什么毛病妨碍才能的发挥。有时我指望父亲帮我理顺这一团乱麻。他很有办法,在当政者跟前很吃香,甚至可以让我们拒不照办被弗朗索瓦丝说成跟生死一样无法抗拒的官方法令。在我们居住的那个地段,唯独我们家把“整修墙面”的规定推迟一年执行;他还为萨士拉夫人的想进水利部门工作的儿子取得部长的特许,提前两个月通过会考——考生名单本来是按姓氏第一个字母的顺序排列的,经过特许的萨士拉夫人的儿子的名字竟然列入姓氏以A开头的考生名单,而不列入姓氏以S开头的考生名单。假如我生了重病,假如我遭到强盗绑架,我坚信我的父亲有通天的本领,能写一封连上帝都无法推却的介绍信,最终使我的重病,我的被绑架,都不过是虚惊一场。我会不慌不忙地等待着必将转危为安的时刻,得到解救或治愈。也许我的缺乏才能,我为自己将来的作品寻找主题的时候在我思想中所出现的那个黑洞,同样无非是一种不牢靠的幻觉,只要父亲出面干预,这种幻觉就会烟消云散;仿佛他早已同官方和上帝达成默契,同意让我成为当代第一流的作家。但是也有这样的时候,我的父母见我老是落在后面而为我着急,那时我的实际生活仿佛已不再是我的父亲着意创作的作品,不再是他可以任意改变的产物,相反,它似乎被包括进与我格格不入的现实,没有任何办法可以对抗那种现实,我在其中也没有一个同盟军,除那种现实之外,别无它物。那时我就觉得我活在世上与常人无异,像大家一样,我会老,会死,我只是没有写作天赋的庸人中的一员。所以,我灰心丧气,从此放弃文学,虽然布洛克一再鼓励我。这种内心的、直接的体验,这种思想的空虚感,比一切人们可能给予我的溢美之词更有力量,等于一个坏人听到人家夸奖他的每一桩善举,他也不免良心发现,悔恨自己的无行。
\par 有一天,母亲对我说:“既然你老是提到盖尔芒特夫人……贝斯比埃大夫四年前为她治过病,照料得特别精心,如今大夫的女儿要结婚了,她一定会到贡布雷来参加婚礼的。你可以在婚礼上见到她。”有关盖尔芒特夫人的事,我听得最多的是贝斯比埃大夫的介绍,他甚至还给我们看了一期画报,那上面刊载了一张她在莱翁王妃家举行的化装舞会上穿着奇装异服拍摄的照片。
\par 在婚礼弥撒进行的当口,教堂侍卫移动了一下身子,使我突然看到坐在一间偏殿里的金黄色头发的贵妇人,她,鼻子大,一双蓝眼睛看起人来入骨三分,胸前蓬松的丝领结是浅紫色的,平整、簇新、光滑,鼻子边上有一颗小疱。她满脸通红,似乎很热,从那张脸上,我认出了与画报上那张照片相近的某些类似之处,虽然它已经像褪了颜色似的模糊不清,但是,就凭我在她脸上发现的特征,倘若我加以归纳的话,恰恰同贝斯比埃大夫在我面前描述的盖尔芒特夫人的特征完全一样:大鼻子、蓝眼睛;于是我心想:那位贵妇人跟盖尔芒特夫人长得很像;她坐着听弥撒的那个偏殿正是坏家伙希尔贝的偏殿,偏殿下已像蜂窝那样松散而发黄的古墓里,安息着布拉邦特古时世袭伯爵们的遗骸,我记得听人说过,那个偏殿是供盖尔芒特家的人到贡布雷来参加宗教仪式时专用的;而那一天,正巧是盖尔芒特夫人应该来的日子,在这个偏殿里只可能有一个女人同盖尔芒特夫人的照片相像,那就是她本人。我失望得很。失望在于我万万没有预料到她会是这样的;过去一想到盖尔芒特夫人,我总是用挂毯或彩色玻璃窗的色调在心中描绘她的形象,把她想象成另一世纪的模样,举止气派与活生生的人完全不同。我万万没有料到她会跟萨士拉夫人一样红光满面,打着浅紫色的领结,她的鹅蛋形的脸庞使我想起了我在家里经常见到过的一些人,我不禁顿生一丝稍纵即逝的疑惑:怀疑偏殿里的那位夫人从生成原则和分子构成上说也许同盖尔芒特夫人名实不符,她的体态完全不知道她头顶上的姓氏有多大的分量,恐怕与医生和商人的妻子属于同一类型。我惊讶地注视着她,脸上的表情等于在说:“原来如此,盖尔芒特夫人也不过如此!”她的形象自然同多次出现在我的幻想中的盖尔芒特夫人的形象毫无关系,因为她不同于我抽象地幻想出来的模样,她只是在一刹那之前,在教堂里,第一次突然出现在我的眼前;她的性质完全不同,不能由我任意着色,不像我想象中的人那样听凭音节流溢出来的橘黄色浸透全身,而是实实在在的真人,她身上的一切,包括鼻子一角正在发炎的小疱,都证实了她从属于生命的法则,好比一出戏演得再热烈迷人,仙女的裙褶以及她手指的颤动都揭示出一位活生生的女演员的实际存在,虽然看戏的人一时疑幻疑真,不知道眼前所见是否只是灯光投下的幻影。
\par 但同时,我努力给这个形象,给那只大鼻子和那双目光锐利的眼睛刻在我视野中的这个形象(也许正是那两样东西趁我还没有来得及想到眼前这位妇女可能就是盖尔芒特夫人的时候就出现在我的视野之内,并在上面刻下了第一道印记),给这个全新的、不可改变的形象粘贴上如下的说明:“这位就是德·盖尔芒特夫人。”然而我却不能使这样的认识同形象妥帖地相合,它们像两只隔着空当的圆盘,始终转不到一起。可是,过去我经常梦见、如今又亲眼目睹确实存在于我心外的这位盖尔芒特夫人,对我的想象力仍施加进一步的威力;我的想象力同与它的期望完全不同的现实一经接触,先是麻木了一阵,后来又开始作出反应,对我说:“盖尔芒特家早在查理大帝之前就声名显赫,对手下的属臣拥有生杀之权;盖尔芒特夫人是热纳维耶夫·德·布拉邦特的后代。她不认识、也不想认识这里的任何人。”
\par 啊!人类的目光享有多么美妙的独立性啊!它由一根松散的、长长的、有弹性的绳子系在人的脸上,因而它能远离人的面孔独自去扫视!盖尔芒特夫人的身体端坐在掩埋着她家祖先们的偏殿内,她的目光却到处转悠,顺着一根根柱子往上张望,甚至像在正殿徘徊的一束阳光那样停留在我的身上,只是这束阳光似乎意识到我在接受它的抚摸。至于盖尔芒特夫人本人,却端坐不动,好比一位母亲,自己的孩子在一边胡作非为地淘气,跟她所不认识的人多嘴多舌地答腔,她却视而不见,所以我就没法知道她赞成不赞成自己的眼光,趁自己的心灵懒得动弹之际这样到处游逛。
\par 然而我觉得要紧的是,在我把她看够以前她别走开,因为我记得多少年来我把见到她当做梦寐以求的一件大事,我的眼睛一见到她就再也离不开了,仿佛我每看一眼都能实实在在地把她的大鼻子、红腮帮以及足以说明她的脸庞特点的一切可贵的第一手资料,统统都贮存进我的记忆库里。当时在我脑海中凡与她有关的想法都使我感到她那张脸是美的——也许尤其是那种总不愿扫兴的愿望,是那种保存我们内心向往最美好事物的本能的表现,把她置于凡夫俗子之外,只凭草草看一眼,我最初有那么一瞬间曾把她同凡夫俗子混淆在一起,但毕竟眼前的她同我以前心目中的盖尔芒特夫人是一个人呀!偏偏当时有人在我周围悄悄议论:“她比萨士拉夫人好看,也比凡德伊小姐强一些。”我听了很生气,言下之意好像她们能跟她相比似的。于是我的目光注视她的金黄色的头发,她的蓝眼睛和她的脖子,由此排除了可能使我想到别人容貌的一切特征,看着这幅有意画得不完全的速写稿,我不禁叫出声来:“她多美呀!多雍容华贵!她准是盖尔芒特家的一位高傲的夫人,热纳维耶夫·德·布拉邦特的后代!”我当时的注意力全都集中在她的身上,简直把她孤立了起来,以至于今天我倘若回忆那天的婚礼,我再不记得其他参加婚礼的人的模样,只记得她以及那位教堂侍卫的情状,因为我问过教堂侍卫,那位夫人是不是盖尔芒特夫人;教堂侍卫给了我肯定的回答。说到她,我尤其历历在目的是她同大家一起鱼贯进入圣器室的情景。那一天刮着风,又时而来一阵大雨,炎热的、时有时无的太阳照亮了圣器室。盖尔芒特夫人同贡布雷的老百姓挤在一起,她连他们姓什么都不知道,但是他们的猥琐把她的崇高衬托得极其鲜明,以至于她不能不由衷地对他们怀有一种宽厚之心,而且她的既高雅又纯朴的举止,更使大家对她敬畏备至。一般人见到认识的人,目光中总故意地含有某种确切的含义;而她不能放出这样的目光,她只是让她的漫不经心的念头,化作她掩饰不住的粼粼蓝光,不断地流溢出来,她但愿这股光流,在流经那些小人物身边,并且随时都在触及那些小人物的时候,千万不要使他们感到局促不安,千万不要显得高傲冷淡。我至今犹历历在目的是,在浅紫的、蓬蓬松松的丝领结之上,她的眼睛流露出些许惊讶和略含羞涩的微笑;这微笑倒不是她有意给什么人看的,而是让每一个在场人都感觉到;那种气派就像一位女王谦逊地面对她的臣民,表现出她的爱民之心;这微笑落到了一直盯住她看的我的身上,她的目光蓝得好比透过“坏家伙希尔贝”那幅彩色玻璃窗射进屋来的阳光,它在做弥撒的时候停留在我的身上,我不禁想道:“她一定注意到我了。”我认准她喜欢我,她离开教堂后还会想到我的,甚至回到盖尔芒特以后她也许会为我而惆怅呢。我也立刻爱上了她,因为,若说一见钟情,有时候只须她像我想象中的斯万小姐的态度那样,对我们不屑一顾地瞅上一眼,我们心想这女人绝无可能倾心于我们,这些就足以使我们痴情相思了;但也有时,只须哪位女士像盖尔芒特夫人那样好心地瞧瞧我们,我们想她可以同我们两心相悦,这同样足以使我们魂牵梦萦。她的眼睛像一朵无法采撷的青莲色的长春花;我虽无法采撷,她却是馈赠给我的;已被一团乌云挡去半边的太阳,仍竭尽全力把光芒投射到广场上和圣器室,给为婚礼铺设的红地毯增添一种肉红色的质感,使羊毛地毯长出一片粉红色的绒毛,多了一层光亮的表皮;盖尔芒特夫人微笑着走在地毯上面,那种温柔、庄重、亲切的气氛,渗透了豪华而欢快的场面,类似歌剧《洛痕格林》\footnote{《洛痕格林》:瓦格纳的第一部突破传统形式的歌剧,1850年首演于魏玛,取材于德国传说:洛痕格林救出布拉邦特公主,并与她相爱、结婚,后又因出身问题,离开了她。}中的某些片段,类似卡帕契奥\footnote{卡帕契奥(1455—1525):意大利画家,是上面提到过的让迪勒·贝里尼的学生。}的某几幅油画,同样使人认识到波特莱尔\footnote{波特莱尔(1821—1867):法国诗人,《恶之华》的作者。}为什么能用甜蜜这个形容词来形容铜管乐的声音。
\par 从那天起,每当我去盖尔芒特家那边散步,我总比以前更为自己因缺乏文学禀赋,不得不断绝当大作家之念而痛心不已!我离开人群,独自在一旁遐思时,憾恨之情更使我苦楚难当,以致为了不再受这痛苦的折磨,我的理智索性采取有意止痛的办法,完全不去想诗歌、小说以及由于我才情寡薄而无从指望的诗一般的前程。于是,一个屋顶,反照在石头上的一点阳光,一条小路的特殊气息,忽然脱离一切文学的思考,与任何东西都无联系地使我感到一个特殊的快乐,使我驻步流连;我暂停观赏的另一个原因是由于这一切事物仿佛在我所见不到的隐秘之中蕴藏着某种东西,它们请我去摘取,我却竭尽全力而无处觅得。因为我感到这东西蕴藏在它们的内部,所以我一动不动地呆立在那里,用眼睛看,用鼻子嗅,想用自己的思想,钻进这形象和这气息的内部去。倘若那时我必须赶上我的外祖父,继续往前走,那么我就闭上眼睛,想方设法回忆方才所见的情景。我专心致志地、一丝不苟地追忆那屋顶的形状,那石头的微妙的细节;也不知为什么,我总觉得它们仿佛饱满得要裂开似的,仿佛准备把它们掩盖下的东西统统都交给我。当然,虽说能使我重新萌生当作家和诗人的希望的不是这些印象,因为它们总是同某个既无思考价值又同任何抽象真理无涉的个别对象相联系,但它们至少给了我一种无由的快感,一种文思活跃的幻觉,从而排遣了我的苦恼,排遣了每当我想为写一部巨著寻找一种哲学主题时所自恨不已的无能感。然而那些印象以具体的形态、色彩和气味迫使我意识到严峻的责任:我必须努力找到隐蔽其中的东西。但是这任务太艰巨了,我很快就为自己找到逃避努力、免去劳累的借口。幸亏那时我的长辈们在叫我了,我感到我当时不具备进行有效探究所必需的平静的心境,倒不如在回到家里之前索性不去想它为好,省得早早地徒劳无功。于是,我不再为外面裹着一种形式、一股香味、但里面又不知包藏何物的那件东西操心了;我心安理得,因为我正把受到形象外衣保护的那件东西带回家去呢,我感到它在形象的外衣下,同每逢大人允许我外出钓鱼的日子,我装进筐里还盖上保鲜的青草带回家来的鱼儿一样地鲜灵活泼。但是,回家之后,我就另有所思了,所以,那块阳光反照的石头,那片映在水面的屋顶,那悠悠的钟声,那草木的气息,还有许多各不相同的形象,也都在我的脑海中堆积下来,就跟我散步时采回来的各色野花和别人送我的各种东西堆积在我的房间里一样。而隐蔽在那些形象下的实况,我虽曾有所感,却始终缺乏足够的毅力去发现,后来也早都泯灭了。然而,有一次,我们散步的时间比平时长,在回家的中途遇见了驾车经过的贝斯比埃大夫。由于时近黄昏,大夫认出我们一行之后,便请我们上车;那次我又得到类似的印象,不过我没有轻易搁置一边,而是进行深一步的探究。我被安排坐在车夫的身旁。马车疾驰如风,因为贝斯比埃大夫在回到贡布雷之前还得在马丹维尔停留一会儿,去看望一名病人;他同我们讲定:我们在病人家门口等他。车到拐弯处,突然,我感到一阵特别的、与其他快感全然不同的喜悦,因为我远远望见了马丹维尔教堂的双塔并立的钟楼,而且随着马车的奔驰和夕阳的反照,那双塔仿佛也在迁移,及至后来,同它们相隔一座山岗、位于另一片较高的平川上的维欧维克的钟楼,竟似乎也同它们成了紧邻。
\par 我在注意到双塔塔尖形状的同时,目堵了它们轮廓的位移和塔面夕照的反光,我感到我领略不透自己的印象,总觉得在这种运动和这片反光中,有件东西既是双塔所包含的,也是它们所窃取的。
\par 这两座钟楼看来离我们还远,仿佛我们的马车并没有向它们驰去,等到转瞬间我们忽然在教堂前停车,我才大吃一惊。我不知道望到双塔时为什么那样地喜悦,而探究其原因又似乎非常艰难;我但求在脑海中贮存下这些阳光沐照的轮廓线,至少在目前不去想它。我倘若加以探究,那么两座钟楼定会同那么多的树呀、屋顶呀、气味呀、音响呀永远联结在一起,我之所以能从纷扰的万物中分辨出上面这些东西,是因为它们同那一片面目不清、我始终没有深入探究的平原有关。我跳下马车,在等待大夫的时候,同大人们一起聊天。后来我们又开始上路,我还是坐在车夫旁边的座位上。我回头看看双塔,稍微过了一会儿,我又在拐弯处最后看了它们一眼。车夫虽然不善于交谈,我说什么他都很少答腔。由于没有别人做伴,我只得与自己做伴,无可奈何地回忆我的那两座钟楼。不久,它们的轮廓,它们的阳光灿烂的表面忽然像有一层外壳似的裂开了,隐藏在里面的东西露出了一角。当时我顿生一念,在前一秒钟它还不存在,这时却形成一串词句,涌进我的脑海;初见双塔时我所感到的那种喜悦立即膨胀起来,使我像醉了似的再不能想别的事情了。当时,我们已经远离马丹维尔,我回头看去,又见到了双塔;这一次它们成了两条黑影,因为太阳已经下山。有好几次,道路转弯,把双塔从我的视线中抹去,后来,它们最后一次出现在地平线上,又终于在我的眼前完全消失了。
\par 我并没有想到隐藏在双塔之中的东西大概同漂亮的句子相类似,因为它是以使我感奋的词汇的形式出现在我的面前的,我向大夫借了纸和笔,也不管车行颠簸,我写了下面这一小段文字,以慰抚我激荡的心胸,以宣泄我满腔的热情;后来我找到了当时的原文,现在只作些许改动,转录如下:
\par “孤零零地从地平线上崛起、仿佛埋没在茫茫田野中的马丹维尔的双塔,高高地刺向蓝天。不久,我们看到三座塔影:一座迟来的钟楼,维欧维克的钟楼,摇身一转,站到了它们的面前,同它们会合在一起。时光流逝,我们的马车也在飞驰,然而鼎立的三塔始终在我们的眼前,像三只飞禽,一动不动地兀立平川,阳光下它们的身影格外分明。后来维欧维克的钟楼躲到一边,拉开了距离,马丹维尔的双塔依然并立,被落日的光辉照得纤毫可辨,甚至在离它们那么远的地方,我都能见到夕阳在塔尖的斜坡上嬉戏、微笑。我们花费了那么多的时间向它们靠拢,我以为还需许久才能到达它们跟前,忽然,车儿一拐,竟已经把我们送到塔下;双塔那样突然地扑面而来,幸而及时刹车,否则差一点撞在庙门上。我们继续上路;我们已经离开了马丹维尔,村庄陪我们走了几秒钟之后便消失了,地平线上只剩下马丹维尔的双塔和维欧维克的钟楼,它们在摇动着阳光灿烂的塔尖,向我们道别,目送我们奔驰远去。有时候,它们中一个隐去,让另外两个再瞅我们一眼;但是道路改变着方向,它们在阳光中像三枚金轴也随之转动,随后在我们的眼前消失。又过了一会儿,那时我们离贡布雷不远,太阳已经下山,我最后一次遥望它们,它们竟只像画在田野地平线之下的三朵小花了。它们也使我联想到传说中的三位姑娘,被抛弃在夜幕已经降临的荒野。正当我们的马车奔驰远去之际,我看到她们在怯怯地寻路,只见她们高贵的身影磕磕绊绊,后来就彼此紧挨在一起,一个躲到另一个的身后,在夕红未消的天边只留下一个婀娜卑谦的黑影,最终在夜色苍茫中消隐。”
\par 以后我一直没有再去想这段文字,可是,在当时,我坐在大夫的马车夫的旁边,那是他通常放鸡笼子的地方,笼里装满他在马丹维尔市场上采购来的鸡鸭,我坐在那地方写完了上述一段文字之后感到非常痛快,我觉得它巧妙、周全地把我从钟楼的纠缠中解脱出来,让我对钟楼所蕴藏的内涵也作了交待,我痛快得好比一只刚下过蛋的母鸡,直着嗓门儿唱了起来。
\par 在作这类漫步的时候,我能整整一天想入非非,想到能成为盖尔芒特夫人的朋友该有多快活,钓钓鳟鱼,乘一叶扁舟荡漾在维福纳河上;而贪图幸福的我,在那样的时刻,对生活别无他求,但愿此生天天下午如此逍遥。但是,在归途中,当我在左首瞥见一座农庄时,我的心突然怦怦乱跳,我知道不出半小时我们就到家了。这座农庄离另外两座挨得很近的农庄相当远,要进入贡布雷市区,只须经由农庄折入橡树夹行的林荫道,林荫道的一边是分属三户农家的果园,株距整齐的苹果树枝条垂地,斜照的夕阳给树荫勾画出日本风格的图案。每逢去盖尔芒特家那边散步的日子反正都是这样,回家之后不久就开晚饭,我刚吃完,他们就打发我去睡觉,要是赶上家里有客,我的母亲就不能离席,不能上楼来到我的床边同我道晚安。我悻悻然进入这个凄凉境界,同不久前我欢天喜地投入的那个快活境界相比,区别如此鲜明,犹如层云迭起的天边,一抹红晕被一道绿线或一道黑线所切断。红霞中有一只鸟儿在飞翔,眼看它将飞到尽头,几乎已经接近黑色区域,接着它飞了进去。盼望去盖尔芒特,盼望旅游,盼望幸福的念头刚才还纠缠着我,可现在我与它们相去万里;我已不觉得实现这些愿望有什么乐趣可言了。我甘心把这一切全都抛弃,只求能在母亲的怀里整夜哭泣!我瑟瑟发抖,我忧心忡忡地盯住了母亲的脸庞,今天晚上她不会到我的房里来了,独居孤室的景象已在我的脑海浮现,我恨不能一死了之。这种心境一直延续到第二天的早晨,当阳光像园丁架梯子似的把一道道光线靠到长满旱金莲的墙上(那些旱金莲一直缘墙而上,长到我的窗前),我连忙下床,赶快到花园里去,不再顾及黄昏又会引来同母亲分手的时刻。所以说,我是在盖尔芒特家那边学会辨别在某些时期内先后在我身上出现的各种不同的心境的,它们甚至在一天之内都各占一段时间,一种心境赶走另一种心境,就像定时发烧一样分秒不差;它们彼此相接,又彼此独立,彼此之间无法沟通,以致在某种心境之下,我无法理解、甚至无法想象在另一种心境之下我所期望或我所惧怕或我所做过的一切。
\par 因此梅塞格利丝那边和盖尔芒特家那边,对于我来说,是同我们各种并行的生活中最充满曲折、最富于插曲的那种生活的许多琐细小事紧密相连的,也就是同我们的精神生活有关。无疑,它在我们的心中是悄悄地进展的,而我们认为意义和面貌都发生变化的真理,为我们开辟新的道路的真理,我们其实早就为了发现它作过长期的准备,只是我们没有意识到罢了;而在我们的心目中,真理却只从它变得显而易见的那一天、那一分钟算起。当年在草地上嬉戏的花朵,当年在阳光下流淌的河水,曾与周围的风景相关连,而这些景物至今仍留恋着它们当年的无意识的或者散淡的风貌;不用说,当它们被那位微不足道的过客、那个想入非非的孩子久久地审视时,好比一位国王受到湮没在人群中的某位回忆录作者的仔细的考察那样,大自然的那个角落,花园里的那个地段未必能认为它们多亏那孩子才得以继续幸存在它们稍纵即逝的特色之中;然而,掠过花篱,紧接着由野蔷薇接替的那株山楂花的芳香、花径台阶上没有回音的脚步声、河中泛起扑向一棵水草又立即破碎的水泡,都一直留在我激荡的心里,而且连续那么些年都久久难忘,而周围的道路却在记忆中消失得无影无踪了。走过那些道路的人死了,甚至连对走过那些道路的人的回忆也都泯灭了。有时,延存至今的那一截片断的景物,孤零零地从大千世界中清晰地浮现,繁花似锦的小岛在我的脑海中飘动,我却说不出它来自何方,起于何时——也许干脆出自什么梦境。但是,我之所以要想到梅塞格利丝那边和盖尔芒特家那边,首先是把它们看做我的精神领域的深层沉淀,看做我至今仍赖以存身的坚固的地盘。正因为我走遍那两处地方的时候,我对物对人都深信不疑,所以唯独我经过那些地方时所认识到的物和人至今仍使我信以为真,仍使我感到愉快。也许因为创作的信心已在我的心中枯萎,也许因为现实只在我的回忆中成形,今天人们指给我看我以前未曾见过的花朵,我只觉得不是真花。沿途有丁香花、山楂花、矢车菊、丽春花和苹果树的梅塞格利丝那边,沿途有蝌蚪浮游的河流、睡莲、金盏花的盖尔芒特家那边,在我的心目中永远构成了我乐于生活其间的地域景象,在那里我首先要求的是能有地方钓鱼,有地方划船,有地方见到哥特式古堡的残迹,就像在圣安德烈那里一样,能在麦浪之间找到一座磨房般金光灿烂、乡土气十足的、雄伟的教堂。我如今漫游时偶尔还能在田野中遇见矢车菊、山楂树和苹果树,由于它们早印在我的心灵深处,与我的往事相处在同一层次,所以便直接同我的心灵相通。然而因为一地有一地的独特之处,所以我一旦萌生重访盖尔芒特家那边的愿望,即使那时有人领我到一条河边,河里的睡莲跟维福纳河的睡莲一样美,甚至更美,我也不能得到满足;同样,黄昏时回到家里,在忧虑袭来的时刻(后来这忧虑迁居进爱情的领域,变得同爱情难分难舍),我也不希望有一位比我的母亲更美丽、更聪明的母亲来同我道晚安。不,为了我能美滋滋地、安心地入睡,我需要的是她,是我的母亲,是她向我俯来的脸庞,在她的眼睛下面似乎有什么东西,可以算一种缺陷,但我也同样喜欢;除母亲之外,没有一个情妇能使我得到那样纤毫不乱的安宁,因为你即使信赖她们的时候都不免存有戒心,你永远不能像我接受母亲一吻那样得到她们的心;母亲的吻是完整的,不掺进任何杂念,绝无丝毫其他意图,只是一心为我。同样,我想重睹芳华的是我所认识的盖尔芒特家那边的景物——半路有座农庄,与另外两座紧挨在一起的农庄相距颇远,位于那条橡树成行的林荫路口;是那几片被夕阳照得犹如池塘一样反光、倒映出苹果树低垂枝杈的如茵的草地。这幅风景有时在夜间进入我的梦境,其独特的个性以一种近乎神奇的力量紧紧搂住了我,待我从梦中醒来时,却又无从寻觅。无疑,梅塞格利丝那边或盖尔芒特家那边只因为在我心上留下不同印象的同时也使我亲身体验到了这一切,所以这些不同的印象才牢固地铭刻在我心中,永远紧紧地连结在一起,从而使我今后的生活面临那么多的幻灭,甚至那么多的错误。因为,我经常想重新见到某人,却意识不到这仅仅是由于那人使我回忆起攀满山楂花的樊篱,因此我认为——同时也让别人相信——只需神游故地,便能重温昔日的残梦了。同样,即使我身临其境,今天在我可能同梅塞格利丝那边和盖尔芒特家那边有关的印象中,昔日的印象依然存在,只是那两个地方给我的印象提供了牢靠的基础、一定的深度和一种其他印象所没有的幅度;它们也使我的旧印象多了一种魅力,一种只有我才体会得到的意蕴。每当夏天的黄昏,和谐的天空响起猛兽吼叫般的雷鸣,在人人都埋怨风狂雨骤的时候,正是梅塞格利丝那边的昔日情景,驱使我独自透过落下的雨声,忘情地嗅到虽无形迹却长存于我心田的丁香花的芬芳。
\par 就这样,我往往遐思达旦,想到在贡布雷度过的时光,想到当年凄凉的不眠之夜,想到昔日的种种情景——是后来的一杯茶的味道(贡布雷人称之为“香味”),勾起了多少往事的生动形象——更由于回忆的连锁反应,使我想到早在我出生之前就已经发生、但直到我离开贡布雷多年之后才听说的有关斯万的恋爱经历,这在细节上不可能精确无误,因为我们有时对死了几百年的人的生平,更容易知道一些细节,而对我们最亲密的朋友的生活,反而不易得到详备的认识,故而精确之不可能,好比想从这个城市同另一个城市的人聊天,在人们不知道有什么途径可以扭转这种不可能的情况下,看来是无法进行的。这一切回忆重重叠叠,堆在一起,不过倒也不是不能分辨,有些回忆是老的回忆,有些是由一杯茶的香味勾引起来的比较靠后的回忆,有些则是我从别人那里听来的别人的回忆,其中当然还有“裂缝”,有名副其实的“断层”,至少有类似表明某些岩石、某些花纹石的不同起源、不同年代、不同结构的纹理和驳杂的色斑。
\par 当然,当天色徐明时,我似醒非醒的短暂的矇胧早已经消散。我知道我果然躺在某一间屋子里,因为在夜犹未央时我已经把这房间照原样设想过一番了;仅仅靠我的回忆或者凭我放在窗帘下的一盏微弱的油灯提示,我已经像维持窗门原始布局的建筑师和装璜匠那样地把整间屋子里的格局和家具设置都照原样想象得各在其位了。我把镜子架在原处,把柜子也放在它通常占据的地点。但是,阳光已不是我起初误以为的阳光,其实是黄铜帘杆上炭火余烬的反光了。当阳光像用粉笔在黑暗中刚划下第一道更正的白线时,原先被我错放进门框的窗户立刻带着窗帘脱框而跑;被我的记忆放错地方的书桌为了给窗帘让路也连忙把壁炉往前推,同时把过道那边的墙壁拨到一旁;一个小庭院稳稳当当地在一刹那之前为盥洗室所占据的地盘上落脚,而我在昏暗中所重建的那个寓所,被曙光伸出的手指在窗帘上方划下的那道苍白的记号赶得仓惶逃窜,挤进了我初醒时在回忆的漩涡中泛起的其他寓所的行列之中。

\subsubsection*{第二部\ 斯万之恋}


\par 要参加维尔迪兰家的“小核心”、“小集团”、“小宗派”,只要满足一个条件,但这是一个必不可少的条件,那就是要默认它的信条,其中有一条就是要承认当年得到维尔迪兰夫人宠爱的那位青年钢琴家既“压倒”普朗岱,也“压倒”鲁宾斯坦\footnote{普朗岱(1839—1934),法国钢琴家;鲁宾斯坦(1829—1894),俄国钢琴家、作曲家。}(维尔迪兰夫人说:“瓦格纳的曲子,再也不可能有人弹得像他那样好了!”),还要承认戈达尔大夫的医道比博丹\footnote{博丹(1825—1901),法国名医。}还要高明。随便哪个“新会员”,要是维尔迪兰夫妇不能说服他承认别家的晚会全都跟连阴天那样无聊乏味,那他马上要给轰出去。在这一方面,妇女要比男人难以驯服,她们不愿抛弃从事社交活动的好奇心,不愿放弃亲自到别的沙龙去体会是否比这里更有意思的意愿,而维尔迪兰夫妇感到这种探索精神,这种轻佻的邪魔可能传染开来,对这个小教会的正统教义会带来致命的打击,于是不得不把女性“信徒”一个一个赶了出去。
\par 除了大夫的年轻太太外,那里的女性“信徒”几乎就只剩下(尽管维尔迪兰夫人本人是个有德行的人,出自一个极其富有然而门第低微的正统的资产阶级家庭,但她也慢慢地跟这个家庭中断了一切联系)一个半上流社会中的人,叫作德·克雷西夫人,维尔迪兰夫人按她的小名管她叫奥黛特,说她是个“爱神”;另外还有一个是钢琴家的姑妈,仿佛原先是个看门的门房;她们对上流社会一无所知,头脑简单,很容易就相信萨冈亲王夫人和盖尔芒特公爵夫人只能花钱去雇穷人到她们家饭桌上去充数这种说法,也很容易就相信如果有人邀请她们到这两位贵妇人家去做客的话,这位当年的门房和这位轻佻的女人是会嗤之以鼻的。
\par 维尔迪兰夫妇从不请旁人吃饭,他们饭桌上的客人是固定的。晚会也没有一定的节目单。年轻的钢琴家只有在“来劲儿”的时候才演奏,本来嘛,谁也不能勉强谁,维尔迪兰先生不是常说吗:“在座的都是朋友,友情第一嘛!”如果钢琴家想演奏《女武神》中奔马那一段或者《特里斯坦》\footnote{《女武神》和《特里斯坦与依索尔德》都是瓦格纳的歌剧。}的序曲,维尔迪兰夫人就会反对,倒不是这音乐不中她的意,恰恰相反,那是因为它在她身上产生的效果太强烈了。“您非要我得偏头痛不可吗?您早就知道,每次他弹这个,我就得偏头痛。我知道会产生什么后果!明天当我要起床的时候,得了,晚安吧,谁也不来了!”他要是不弹琴,大家就聊天。朋友当中有那么一位,通常是他们那时宠爱的那位画家,如同维尔迪兰先生所说:“撒出一句扯淡的话,招得大家纵声大笑。”尤其是维尔迪兰夫人,她是惯于把表达那些情绪的形象化的说法落到实处的,有一天就因为笑得太过厉害,戈达尔大夫(当年还只是个初出茅庐的小伙子)不得不把她那脱了臼的下颌骨给托上去。
\par 晚礼服是不许穿的,因为大家都是“亲密伙伴”,不必穿得跟被他们避之若瘟神、只是在尽可能少举办而仅仅是为了讨好那位画家或者把那位音乐家介绍给别人时才组织的盛大晚会上邀请的那些“讨厌家伙”一样。其余的时间,大家就满足于猜猜字谜,穿着便服共进晚餐,决不让任何外人混入这个“核心”。
\par 随着这些“伙伴们”在维尔迪兰夫人的生活中所占的地位日益增长,凡是使得朋友们不能到她跟前来的事情,凡是使得他们有时不得空闲的事情,例如这一位的母亲,那一位的业务工作,另一位的乡间别墅或者什么病痛等等,就都成了叫人讨厌、该受指责的了。要是戈达尔大夫认为应该离开餐桌回到病危的病人跟前去的话,维尔迪兰夫人就会对他说:“又有谁知道,如果您今天晚上不去打扰他,也许对他反倒好得多;您不去,他可以好好睡一夜;明天您一早去,他的病也许已经好了。”十二月一到,一想起她的忠实信徒们要在圣诞和元旦那两天把她“撂在家里”,她就发愁。钢琴家的姑妈要他那天一定得到她母亲家去吃晚饭。
\par 维尔迪兰夫人厉声叫道:“如果你们元旦那天不随乡下人的习俗,不跟您的母亲在一起吃那顿晚饭,她就会死啦!”
\par 到了复活节前的那个圣周,她的不安情绪又起来了。
\par “您是个大夫,是科学家,是自由思想家,您当然跟平常一样,耶稣受难日那天是要来的哟?”她在组织“核心”的第一年以坚定的口吻对戈达尔大夫说,仿佛准能得到肯定的答复。不过她在等待那句答话的时候,还是不免有些担心,因为他要是不来的话,她就有孤独一人的危险。
\par “耶稣受难日那天我是要来的……来向您告别,因为我们要到奥维涅去过复活节。”
\par “到奥维涅?去喂跳蚤,喂虱子,敢情是大有好处!”
\par 沉默了一阵,她又说:
\par “如果您早点对我们说,我们也许会安排安排,跟你们在比较舒适的条件下一起去作这次旅行的。”
\par 同样,要是有哪位“忠实信徒”有个朋友,或者哪位“常来的女客”有个追求者,可能会拽住他们不让他们前来的话,维尔迪兰夫妇就会说:“好吧,把您的朋友带来吧!”他们倒是不怕女客有情人,只要她把他带到他们家来,在他们家谈情说爱,不至于因为爱他而不爱他们就行。他们会考验这位朋友,看他是不是能对维尔迪兰夫人推心置腹,有没有可能被接纳进这个“小宗派”。如若不然,他们就会把介绍他前来的那位信徒叫到一边,请他们跟他们的朋友或情妇闹翻。反之,那位“新来的人”也就会变成一个信徒。就这样,那一年当那位半上流社会中人对维尔迪兰先生说,她认识了一个很可爱的人,叫作斯万先生,同时暗示他很想受到他们接待的时候,维尔迪兰先生当场就把这个请求转告他的妻子。维尔迪兰先生向来是要等他的妻子拿了主意才拿主意的,他的特殊任务就是想方设法满足她以及她的信徒们的一切愿望。
\par “德·克雷西夫人有事跟你商量。她想把她的一个朋友斯万先生介绍给你。你看怎么样?”
\par “嗨,对这样一个完美无缺的人,我有什么不能答应的?您别谦虚了,我没有问您的意见,我就是要说您是一个完美无缺的人。”
\par “既然您那么说,”奥黛特以马里沃式的故作风雅的殷勤语调答道,说着又补充一句:“您是知道的,我可不是个fishing for compliments(沽名钓誉)的人。”
\par “好吧,如果您的朋友讨人喜欢,那就带他来吧。”
\par 诚然,这个“小核心”跟斯万常去的社交圈子毫无关系,而纯粹的上流社会人士也会觉得像他那样已经在上流社会里占有一个特殊地位的人,犯不上想方设法登上维尔迪兰夫妇的家门。不过斯万是那么爱女人,打他差不多认遍了贵族阶层的所有女子,她们已经再也没有什么可以教他的那一天起,他就把圣日耳曼区授给他的那些归化证书(差不多也就是贵族证书)仅仅看做是本身已经没有什么价值的流通证券或者信用证,倒是可以使他有条件到外省什么小地方,巴黎什么偏僻的地区去追求他看着漂亮的某个乡绅或者法院书记官的女儿了。当年欲念或者爱情在他身上激起的那种虚荣心,现在通过日常生活的习惯已经摆脱了,而正是这种虚荣心把他导向那个上流社会的生活,在无聊的逸乐中浪掷了他的聪明才智,把他在艺术方面的博学用于指导贵妇人购买绘画作品,布置她们的府邸。也正是这种虚荣心促使他在他爱上的不相识的女子面前,显摆单是斯万这个姓氏所表达不了的帅劲儿。如果那个不相识的女子出身低微,他就越发要显摆那个劲儿。正如一个有才气的人不怕在另一个有才气的人面前露拙一样,一个帅的人不怕一个阔老爷,而怕一个乡巴佬不领略他的帅劲儿。有史以来,人们出于虚荣而费的心机说的谎话,有四分之三是对地位比自己低下的人而发的。斯万在一个公爵夫人面前朴朴实实,不修边幅,而在一个女用人面前就要装腔作势,唯恐被她瞧不起。
\par 有很多人出于他们的社会地位造成的慵懒或者无可奈何的安于现状的心理,他们不去享受他们老死于其间的上流社会之外的现实生活为他们提供的乐趣,却退而求其次,一旦对那些平庸的娱乐以及还能忍受的无聊乏味的事情习以为常,就把这些称之为乐趣。斯万却不是这样的人。他不费心思去发现跟他在一起消磨时间的女人身上的美,却花时间去跟他一眼就觉得漂亮的女人在一起。而这些女人的美时常是相当俗气的,因为他本能地追求的体态之美跟他所喜爱的大师们所雕塑或绘出的女子的美恰恰背道而驰。后者深沉的性格或阴郁的表情使他的感官凝滞,而只要有健康、丰满而红润的肉体就足以使他的感官苏醒。
\par 如果在旅途中遇到一个他原不该去结识的人家,而其中有一个女人在他眼里显出他从未见识过的魅力,那么,要他保持矜持态度,消除她在他身上激起的欲念,用写信召唤一个旧情妇到身边来这种办法来替代他可能从那一位身上得到的乐趣,这在他看来就等于是在生活面前的怯懦的退让,是与不去游览这个地区,却把自己关在卧室里眺望巴黎的景色同样的对新的幸福的愚蠢的抛弃。他不把自己封闭在他的社会关系的圈子里,而是自己去创造,以便哪儿有个女人中他的意,就在哪儿另起炉灶,建立基地,就像探险家随身携带的装卸自如的帐篷一样。至于不能搬动的东西,或者不能换取新的乐趣的东西,不管别人看来是如何可贵,他都弃之如敝屣。不止一次,他跟一个公爵夫人相处多年,慢慢地激起了对方以身相许但苦于无机会满足的欲念,从而在她跟前赢得了信任,可是他却冒冒失失给她拍个电文,要她给他去封电报,让他立即跟她的一个管家联系,原来他在乡下发现了管家的女儿——这真像是一个饿得要死的人拿一粒金刚钻换一片面包!事情过后,他也不免哑然失笑,原来在他身上,虽然也有些难能可贵的高尚优雅之处,却也不乏粗野劲儿。再说,他属于这样一种有才气的人,他们在无所事事中度日,心想无所事事正好给他们的聪明才智提供跟搞艺术或学习同样值得注意的对象,心想“生活”本身包含比所有小说更有意思、更富有浪漫色彩的情景,就拿这种想法聊以自慰,甚至作为原谅自己的借口。至少他是这么说的,而且轻而易举地说服他社交界中最高雅的朋友们,特别是夏吕斯男爵。他常跟男爵讲一些妙趣横生的艳遇故事,自己也暗自得意,说是什么有回在火车上碰到一个女的,后来把她带到家里,发现她是一位君主的妹妹,当时欧洲政治的条条脉络全都掌握在她哥哥手心里,他自己也就对欧洲政治了若指掌,又说什么由于情况的极端复杂,有回他能否当上一个女厨师的情夫,要由教皇选举会议来决定等等。
\par 供斯万驱使,为他拉线搭桥的不仅有一大群他过从甚密的德高望重的太后、将军、院士,他所有的朋友也都不时收到他的来信,信上以外交手腕要求他们写封推荐信或介绍信,而在层出不穷的桃色事件中假借花样翻新的借口,这种手腕总是万变不离其宗,也就跟大白话一个样了。
\par 多年以后,由于他的性格当中有别的许多方面跟我相似而使我对它发生兴趣的时候,我时常听说,当他给我的外祖父(那时还不是我的外祖父,因为当斯万那段恋情开始从而在很长一段时期内不再寻花问柳的时候,我还没有出生呢)写信时,我外祖父一看信封上的笔迹,就高声叫道:“嗨!斯万又有求于我了,可得小心着点!”也许是出于不信任之感,也许是出于我们只把一样东西送给不需要它的人的那种潜意识的心理,我的外祖父母对他提出的最容易满足的要求报之以斩钉截铁的拒绝,譬如当他提出让他们把他介绍给每个星期天都到他们家吃晚饭的那个姑娘,而每当斯万重提这件事情的时候,他们只好假装已经很久没有见到这个姑娘,其实他们整个星期都在商量该邀请谁来陪她,结果时常是找不出任何人来,但却不跟那最乐于接受邀请的人打个招呼。
\par 有时候,外祖父母的朋友当中的某一对夫妇一直抱怨怎么老见不着斯万,会突然满意地宣布,说是斯万最近变得再可爱不过了,老是跟他们在一起。这么说也许多少还有点要激起我外祖父母对他们的羡慕的意思。我外祖父不愿破坏他们的乐趣,只是瞧着我外祖母哼道:
\refdocument{
    \par 这倒是怎样一个谜团?
    \par 我真是百思不得其解。
}
\par 或者:
\refdocument{
    \par 难以捉摸的幻象……
}
\par 或者:
\refdocument{
    \par 在这样的事儿当中,
    \par 最好是视而不见。
}
\par 几个月之后,如果我外祖父问起斯万的一个新朋友:“斯万怎么样了?您跟他还常见面吗?”对方就会拉长了脸:“嗨!您就别再提他了!”
\par “我还以为你们过往很密呢……”
\par 斯万在好几个月当中一直是我外祖母的表兄弟家的常客,差不多每天都在他们家吃饭。忽然有一天,他不去了,连个招呼也没打。大家还以为他病了呢,我外祖母的表妹正要打发人去打听他的消息,忽然在厨房里发现他的一封信,是厨娘不经意夹在她账本里的。他在信里告诉厨娘,说他就要离开巴黎,不能再来了。原来她是他的情妇,而在跟他们家中断来往的时候,他认为只有必要通知她一个人。
\par 如果他当时的情妇是社交界中的人,或者至少出身不太低微,处境不太特殊,不至于无法引入大雅之堂的话,那么他就会为了她而回到社交界去,但只是在她活动或者他领她去的那个特定的轨道上运行。“今晚就别指望斯万了,”人们说,“要知道,今天是他带那个美国娘儿们上歌剧院的日子。”他为她张罗请帖,到那些人数特别有限的沙龙去,那里有他的老朋友,有每周一次的聚餐,有牌局;每天晚上,当他把他那红棕色的头发梳上一梳,再稍为卷一下子以后,就挑上一朵花插在纽扣孔上,然后动身去找他的情妇,上他那小圈子里的某个女人家去一起吃饭;这时候,一想到他就要看到的那些他可以任意摆布的时髦青年会在他所爱的女人面前怎样对他表示钦佩和友情,他就会重新体味他原已感到厌倦的社交生活的魅力;这种生活的内容,一旦由他跟一种新的爱情结合起来,便被一个忽隐忽现的火焰所照亮,所温暖,在他眼里变得美好而可贵。
\par 这样的私通,这样的调情,每一次都是当斯万看到一张一眼就觉得迷人的脸,或是一个一眼就觉得迷人的身子时油然而生的梦想,或是完全或部分成为现实。可是有一天,当他在剧场里被一位往日的朋友介绍给奥黛特·德·克雷西的时候,事情就不一样了。这位朋友曾经对他说过,这个女的真是令人销魂,他也许可以跟她搞出点什么名堂,不过事情要比看起来难得多,所以把她介绍给他也就是帮了他一个大忙。在斯万看来,她当然不是不美,不过那是一种他不感兴趣的美,激不起他的任何情欲,甚至还引起他某种生理的反感;他觉得她是这样一种女人,每个人都可以举出几个样本来,每个人举的又都不同样,她们都是我们的感官所要求的那种类型的反面人物。要想中他的意,她的轮廓未免太鲜明突出,皮肤未免太纤细,颧骨未免太高,脸蛋未免太瘦长。她的眼睛倒是好看,但是大得仿佛在自身的重量下往下低垂,压着脸上的其余部分,使她总显得身子不舒服或者情绪不佳。在剧场那次相识以后不久,她就给他写了一封信,请他允许她来看看她极感兴趣的他的收藏,她说她“虽然无知,却对美的东西颇为爱好”,她设想他在家中“一杯清茶,满屋图书,一定非常舒适”,而等到她登门拜访以后,对他的了解就会更进一步,却也不掩饰她的惊讶,说他住的那个区不免有点寒碜,而“他是那么smart(帅),这个区与他实在太不般配了”。他后来让她去了,在分手的时候,她说她十分高兴能来拜访,遗憾的是呆的时间那么短促,说他给她留下的印象跟她认识的别的人都不一样,仿佛他们两人之间可以建立一点罗曼蒂克的联系;斯万听到这里微微一笑。他已经接近看破一切的岁数,懂得满足于为爱的乐趣而爱,并不太要求对方的爱;但是这种心心相印虽然已经不再像年轻的时候那样是爱情必然追求的目标,却依然还跟一些概念联系得如此紧密,还可能在爱情没有萌发之前成为产生爱情的根源。男人在年轻的时候渴望占有他所爱的女子的心,到了后来,只要你感觉到一个女子心上有你,就足以使你对她产生爱情。就这样,到了一定的岁数,由于你在爱情中追求的主要是一种主观的乐趣,你就会觉得对女性之美的爱好应该在爱情中起最大的作用,这时即使最初没有任何欲念的因素,爱情也会油然而生,但这是纯生理的爱。在人生的这个阶段,一个人已经多次被爱神之箭射中,爱情就不再在他惊诧和消沉的心面前,完全按它自己的不为我们所知又是无可抗拒的规律来运行了。我们也出来插上一手,用我们的记忆,用我们的主意来歪曲它。当我们看到爱情的一个征候的时候,我们就会想起,就会臆造出其他好些征候。既然我们已经掌握了爱情之曲,一字一句都铭刻在心,那就用不着一个女子唱出曲中的充满了对她的美的赞赏之情的第一句才能想起全曲。而如果她从曲子的中间开始——说什么两人心心相印,双方离了对方生活就失去意义等等——我们就会在应该接茬的地方,立刻参加跟对方的合唱。
\par 奥黛特·德·克雷西又去拜访斯万,以后的访问愈来愈频繁;每一次访问都使他重温在重逢时的失望之感:她那张面孔,他在两次相会的间隔中已经把它的特征差不多忘了,在印象里既不那么富有表情,也不那么暗淡无光(尽管她还年轻);当她跟他谈话的时候,他因她的美并不是他自然而然地偏爱的那种美而感到遗憾。再说,奥黛特的脸显得比实际上更瘦削更凸出,因为她的前额和面颊上部比较扁平,盖着一片当年时兴的前刘海,底下衬着假发卷,蓬松的发绺一直盖到耳边;至于她那长得绝妙的身材,很难看出它的完整性(那是由于当时时装式样的关系,虽然她是巴黎衣服穿得最讲究的妇女之一),因为她的胸衣凸成弧形,像是遮盖着一个假想中的腹部,下缘突然收缩,底下就是鼓得跟气球一样的双层裙子,使得她这个人看来仿佛是由互不相关的几截拼凑而成的;而裙边、荷叶边和坎肩又都一一自成体系,根据设计者的心血来潮或料子的软硬,或者紧贴着它们跟缎带的结子、花边的褶裥、垂直的蓬边相连的线条;或者紧贴着胸衣底下的鲸须片撑架,不管怎样,跟穿在衣服里的人是毫不合体的。衣服上的这些小装饰时而紧贴着她的身体,时而空空荡荡,这就决定她时而显得耸肩缩脖,时而像是深陷在衣服之中。
\par 但是,当奥黛特走了以后,斯万想起她曾对他说过,她觉得每次在等待他答应她再来之前这段时间是过得多么慢的时候,就不免微微一笑;他想起有次她请他不要让她等待过久的时候的那副焦急不安,腼腆羞涩的神色,还有她当时注视着他的那副带着胆怯的恳求的眼神,却使她在插在带有黑天鹅绒的飘带的白圆草帽上的纸蝴蝶花束下,显得非常动人。她也曾说过:“您就不能上我家去喝杯茶吗?”他借口正在进行关于弗美尔\footnote{弗美尔(1632—1675):荷兰风俗画家,亦作肖像及风景。}的研究,其实他已经中辍多年了。“我知道我是什么也干不了的,”她答道,“在您这样的大学问家跟前,我是微不足道的。在你们这些学者面前,我是井底之蛙。不过我还是非常想学习,想知道这些东西,想有人把我领进门。博览群书,埋头在故纸堆里,该多有意思!”她说话时那副自满的神气就跟一个衣着华丽的女人说她不怕脏,乐于干些像“亲自下厨”做菜这样的脏活时一样。“您也许会笑话我;阻碍您去看我的那个画家(她指的是弗美尔),我可从来没有听人说起过;他还活着吗?我能在巴黎见到他的作品吗?我很想了解一下您所爱的东西,很想猜一猜您这辛勤劳动的脑门里面装的是什么,您这永远在思考着的脑子里装的又是什么。要是能参与您的工作,那该是多美好的梦想啊!”他表示歉意,说他怕再结新交——出于对女人的礼貌,他当时说的是怕再遭一次不幸。“您怕堕入情网?真有意思,我可是求之不得,我都愿意付出自己的生命来求得一个寄托感情的对象。”她在说这话时的语气是那么自然,那么令人信服,连他也被感动了。
\par “您多半是为了哪个女的吃过苦头,就以为所有的女人都跟她一样。她没有能了解您;您是这样一个不同凡响的人。您的这种气质,我只看一眼就喜欢,我马上就充分感觉到您与众不同。”
\par “再说您哪,”他说,“我对女人还是非常了解的。您一定也有许多事儿要做,没有多少闲工夫的。”
\par “我?我从来也没有什么事儿要做!我总是有空的,您要找我,我总是有空奉陪的。无论是白天还是晚上,随便什么时候,您都可以来看我。如果您给我个信,我总是乐于来的。您同意吗?您要是能让我把您介绍给维尔迪兰夫人,那我就太高兴了,我是每天晚上都上她家去的。您想想,要是能在那里见到您,想到您是为了我而去的,那该多好!”
\par 当然,当他独自一人的时候,像这样回味他们的谈话,像这样想起了她的时候,他自然会把她的形象跟他在带有浪漫色彩的遐想中想起的别的许多女人的形象并列起来;然而,假如由于某一个偶然情况(或者甚至不需要这个偶然情况,因为当脑子里的一个潜在的心理状态突然冒头的时候,这时出现的情况可能对这个心理状态起不了任何作用),奥黛特·德·克雷西的形象居然占据了他的一切遐想,假如他的一切遐想已经跟对她的回忆密不可分,那么她体态上的缺陷就不再具有任何重要性,她的体态是否比别人的更合斯万的口味也就无关紧要,因为一旦成了他所爱的人的身子,它从此就是唯一能给他带来欢乐或痛苦的身子了。
\par 我的外祖父正好认识维尔迪兰一家,他现存的朋友当中哪一个也不知道这件事。但是他当时已经跟他称之为“小维尔迪兰”的那一位完全断绝了来往,认为他虽然还有百万家财,却已经沦为放荡不羁的败类了。有一天,他收到斯万一封信,问他能否把他介绍给维尔迪兰一家。外祖父叫了起来:“可得小心!可得小心!我一点也不觉得奇怪,斯万准是会走上这条道的。真是好地方!首先,我不能答应他的要求,因为我已经不认识这位先生了。再说,这事儿准跟女人有关系,我可不愿意牵扯进去。好嘛,斯万要跟小维尔迪兰那一伙泡在一起,咱们可有好戏看了。”
\par 外祖父给了否定的答复,只好由奥黛特亲自把斯万领到维尔迪兰家去了。
\par 斯万第一次去的那天,维尔迪兰夫妇饭桌上有戈达尔大夫夫妇、年轻的钢琴家和他的姑妈,还有当时得宠的那个画家;那天晚会上另外还去了几个忠实信徒。
\par 戈达尔大夫从来也拿不准该用什么口吻来回答别人的话,也弄不清对方究竟是开玩笑还是一本正经。他随时准备端出一副笑容,作出一个随机应变、昙花一现的微笑,又要带有一定程度的狡黠,万一对方说的是句玩笑话,也可免遭头脑过分简单之讥。由于他对对方的意图可能猜得不透,所以他不敢让他的微笑在脸上明确表现出来,总是显出一点犹疑不决,使人一眼就看出他是想提又不敢提“您这话可是当真?”这么一个问题。他对在大街上,甚至在日常生活中应该有怎样的言谈举止,也不比在沙龙中更有把握;他对行人、车马、所发生的事情总是报之以带有狡黠意味的微笑,这个微笑使他免遭举止失宜之讥,因为如果他的态度不合时宜,这个微笑就可以表示他早知如此,而他之所以采取这种态度,不过是开个玩笑而已。
\par 而在他觉得可以明白提出问题的一切事情上,大夫是不惜作出一切努力来增长知识,缩小他所不知道的事物的范围的。
\par 因此,他就遵照他那有远见卓识的母亲在他离开外省时给他的教导,每碰到有不知道的成语或者专有名词时,总要查找资料,把它弄个明白。
\par 说到成语,他总是不厌其烦地进行查考,因为他有时以为一个成语还有什么更明确的意义,总想弄清他最常听到的那些成语的精确含义,譬如什么la beauté du diable(青春美)、du sang bleu(贵族名门)、une vie de bâton de chaise(放荡不羁的生活、le quart d'heure de Rabelais囊中如洗、捉襟见肘的时刻)、être le prince des élégances(衣着华丽)、donner carte blanche(授以全权etrer duit a quia哑口无言)之类,还要弄清在怎样的情况下他可以拿来使用。要是没有成语可用,他就会用学来的一些双关语或者谐音词。当他听人在他面前提到新的人名的时候,他就满足于以带来疑问色彩的语调重复一下,心想这么一来就可以套出对方作出一番解释。
\par 他自以为对什么都能分析批判一番,其实这种批判精神他根本是欠缺的。有教养的人施恩于人却说得仿佛是他欠了对方的情(当然也不希望他当真相信),这种心思在戈达尔身上就是白费,他把所听到的话全按字面来理解。不管维尔迪兰夫人对他是怎样盲目地偏爱,虽然她依然觉得他很机灵,可是有次请他进包厢看萨拉·贝尔纳\footnote{萨拉·贝尔纳(1844—1923):法国名噪一时的杰出女演员。}的演出时,就闹过一次笑话。她很客气地说:“大夫,您惠顾光临,真是太好了,特别是我相信您一定常听萨拉·贝尔纳的戏;不过咱们的包厢离舞台也许太近了点儿。”而戈达尔大夫在步入包厢时嘴边挂着一丝微笑(准备根据权威人士是否跟他讲这剧的价值,或保持下去,或收敛起来)答道:“这个包厢敢情离舞台太近,而且现在大家对萨拉·贝尔纳已经有点厌倦了。不过您既然表示了要我来的愿望,对我来说,您的愿望就是命令。能为您效这么点劳,我实在太高兴了。您这么好,我怎能拂您的意呢?”这时候,维尔迪兰夫人也终于恼了。大夫接着又说:“萨拉·贝尔纳真是金嗓子,是不是?好些人写文章说她演起戏来十分卖力,真是满座生辉。这话说得好,是不是?”他原以为维尔迪兰夫人要夸他几句的,可是碰了一鼻子灰。
\par “我看哪,”维尔迪兰夫人后来对她丈夫说,“咱们不该那么谦虚,把咱们送给大夫的东西的价值说得那么低。他是个科学家,不通人情世故。他不识货,咱们怎么说,他就真以为是那么回事。”
\par “我一直不敢跟你说,”维尔迪兰先生答道,“我早就看出来了。”
\par 到了元旦,维尔迪兰先生就不送戈达尔大夫一颗值三千法郎的红宝石而说价值无几,而是买了一颗只值三百法郎的假宝石,却说是无价之宝。
\par 当维尔迪兰夫人宣布斯万先生晚上要来的时候,大夫大吃一惊,高声叫道:“斯万?”那话音简直有点近乎粗暴了,因为这位老兄总是自以为料事如神,对于小小不然的新闻也比谁都感到意外。看到没人搭理,他真是急不可耐,吼了起来:“斯万?斯万是谁?”等到维尔迪兰夫人说:“不就是奥黛特提起过的她的那位朋友吗?”他这才平静下来,直说:“噢!好,好!”至于那位画家,他很高兴看到斯万给领进维尔迪兰夫人的家门,因为他猜想斯万已经爱上了奥黛特,而他是乐于促成好事的。“再也没有比做媒更有意思的了,”他跟戈达尔大夫咬咬耳朵,“我已经做成多对了,甚至是在女人跟女人之间。”
\par 当奥黛特跟维尔迪兰夫妇说斯万很“帅”的时候,他们还担心他是一个“讨厌家伙”呢。哪知道他给他们的印象好极了;他们不晓得,这是由于他经常出入于上流社会的缘故。跟那些哪怕是聪明过人然而从来没有厕身社交界的人比起来,他多少具有进出过社交界的人士的一个优点,那就是不再由于一心要想进去,或者由于毫无根据的反感而歪曲它的形象,把它看成无足轻重。进出过社交界的人士,他们的风度中摆脱了一切冒充风雅的成分,摆脱了显得过分亲切的担心,呈现出潇洒自如,一举手一投足都显得优美,仿佛四肢灵活,做出的姿势恰如他们所愿,而身体的其余部分不会做出任何不合时宜的笨拙动作。社交界人士在向别人介绍给他们的不相识的年轻人优雅地伸出手来,或者是向别人为之介绍的一位大使不卑不亢地躬身时,那简直是一种基本的体操动作,在不知不觉之间,渗透到了斯万的整个社交生活中,因此当他面对像维尔迪兰夫妇和他们的朋友这些地位比他低下的人时,本能地表示出一种殷勤,主动接近他们,而这在他们看来,一个“讨厌家伙”是绝不会如此的。他对戈达尔大夫表示了片刻的冷淡:眼看这位大夫在他们两人还没有交谈以前就向他眯了眯眼,莫测高深地微微一笑(戈达尔管这种鬼脸叫“要来的都来吧”),斯万以为大夫多半曾经在哪个烟花场中见过他,可他自己极少涉足那种地方,也从来没有沉溺于花天酒地之中。斯万一想到这个联想有点不雅,特别是在奥黛特面前,她可能会对他产生不良的好印象,因此赶紧敛容。不过当他得悉在他身边的那位妇女就是戈达尔太太时,他心想她的丈夫是那样年轻,不至于在他妻子面前暗示那样的游乐,对大夫那种狡黠的神情也就不再作刚才那样的解释了。画家马上就邀请斯万跟奥黛特一起去参观他的画室,斯万觉得他这个人挺可爱的。“也许您得到的盛情款待比我当年还有过之呢,”维尔迪兰夫人以假装生气的口吻说,“他会把戈达尔的画像给您看的(这是她向画家订的货)。”她又提醒画家:“比施大师(‘大师’是她对画家的戏称),您可记着点儿,眼神要画得美,眼角要画得细巧逗人。您不是不知道,我要的主要是他的微笑,我请您画的是他微笑的肖像。”她认为她最后这句话说得十分巧妙,又高声重复一遍,让很多客人都能听见,甚至为此随便找出一个借口,让几个客人往她身边靠拢一些。斯万要求结识所有的人,甚至包括维尔迪兰家的一个老朋友,叫萨尼埃特的,他有广博的文献知识,拥有巨资,门第显赫,这些条件本该使他赢得尊敬,却由于他腼腆朴实、心地善良而丧失了。他说话的时候含含糊糊,然而这种含糊并不令人讨厌,因为它并不体现语言上的缺陷而是体现他的心灵,表明他依然还保持着纯真的童心。有些辅音他发不好,说明有些刺耳的话他是讲不出口的。当斯万请维尔迪兰夫人把他介绍给萨尼埃特先生的时候,请她把他们两个人的地位颠倒过来;维尔迪兰夫人果然说道:“斯万先生,请允许我把我们的朋友萨尼埃特介绍给您。”把“我们的朋友萨尼埃特”和“您”特别加重。斯万这就在萨尼埃特心中激起了一股暖流,可是维尔迪兰夫妇却从未向斯万透露过这点消息,因为他们多少有点讨厌萨尼埃特,不愿为他介绍朋友。而与此相反,当斯万恳切要求他们为他介绍钢琴家的姑妈时,他们就万分感动。这位姑妈总是穿着黑色的衣服,因为她觉得女人穿黑衣服好看,而且更加高雅;她脸色特别红润,就像刚吃过饭一样。她恭恭敬敬地向斯万哈了哈腰,马上又庄严地挺起身来。她所受的教育不多,又怕在语言上出错,因此发音故意含糊,心想万一说漏了嘴,也可以由于发音含糊而蒙混过去,不致被人家确切地辨认出来,结果她讲的话只是一片难以听清的沙哑声,难得冒出几个她确有把握的字眼。斯万心想可以在跟维尔迪兰先生谈话的时候,把她稍为讽刺一下,不料引起了对方的不快。
\par “她这个人可好极了!”维尔迪兰先生答道,“不错,她才貌并不惊人,这我同意;可是我敢向您担保,当您同她谈话的时候,她可是很讨人喜欢的。”
\par “这我毫不怀疑,”斯万赶紧让步,又说,“我刚才的意思只是说我并不觉得她‘超群出众’(他把这四个字特别强调),并不是对她不表赞赏。”
\par “还有让您吃惊的呢,”维尔迪兰先生又说,“她写得一手好文章。您从没有听过她侄子的演奏?那可是妙极了,大夫,您说是不是?斯万先生,您要我请他弹点什么吗?”
\par “那可是不胜荣幸之至……”斯万正要往下讲,大夫跟他做了个鬼脸,把他的话头打断。敢情大夫记得,在普通的会话里用强调语气,用庄严的形式,已经过时,所以一听到有人一本正经地用一个庄严的字眼(例如刚才的“荣幸”),就觉得说话的人有一副学究气。而如果这个字眼碰巧又在他所称之为陈词滥调之列,那就不管它是如何常用,大夫就认为这个句子必然滑稽可笑,赶紧自己接上茬,用上一句他以为对方想要讲的套话,其实对方连想都不曾想到。
\par “法兰西不胜荣幸之至!”他高举双臂,狡黠地高声大叫。
\par 维尔迪兰先生忍不住笑了起来。
\par “那几位先生在笑什么呢?看起来你们那个角落里全都是乐天派。”维尔迪兰夫人高声叫道。她又像孩子撒娇似的补了一句:“我一个人呆在这里受罚,你们难道还以为我挺高兴吗!”
\par 维尔迪兰夫人坐在一把打了蜡的瑞典式松木高椅子上,这是瑞典一位提琴家送给她的,虽然看起来像张板凳,跟周围古色古香的精美家具毫不相称,可是她还是把它保留下来;她的忠实信徒们不时给她送的礼品,她摆在外面,好让馈赠者认出时心里高兴。她也曾劝他们只送花和糖果,这些东西是不能长久保存的;可是说也没用,结果她家里慢慢地就堆满了脚炉、椅垫、挂钟、屏风、气压计、瓷花瓶,重复冗杂,杂乱无章。
\par 她坐在她那高高在上的位子上兴致勃勃地参加她的信徒们的谈话,为他们开的玩笑而心花怒放,不过自从那次笑得下颌骨都脱了臼以后,就再也不敢当真放声大笑,而代之以一个手势,表示她笑得连眼泪都流出来了,这就既不费力又无危险。要是哪位常客对某个“讨厌家伙”,或者对某个原是常客后来被打成“讨厌家伙”的人说上一句俏皮话,维尔迪兰夫人就会发出一声尖叫,把她那双已经开始蒙上一层白内障的小鸟似的眼睛紧闭,突然用双手将脸捂上,严密得什么也看不见,仿佛面前出现了什么猥亵的场面或者是要闪避一个致命的打击似的;她装出正在竭力憋着不笑出来,简直像是如果笑将起来,就会笑得昏死过去似的。维尔迪兰先生一直自以为跟他妻子一样和蔼可亲,可当真开怀大笑,马上就笑得喘不过气来,跟他妻子那些经久不息的假笑这种高招相比,真是望尘莫及,自愧不如,这是他最难过的一件事。维尔迪兰夫人则为她的信徒们的兴高采烈而飘飘然,为友好情谊、恶意中伤和斩钉截铁的断言所陶醉,她像一只吃了在热酒中泡过的食料的鸟,栖息在她那张高椅子上,为这充满着友情的气氛而抽噎。
\par 维尔迪兰先生请斯万允许他点上烟斗(“在这里的都是朋友,不必拘礼”),再请年轻的艺术家坐上琴凳。
\par “不,不,别麻烦他,他到这里不是来受折磨的,”维尔迪兰夫人高声叫道,“谁要折磨他,我可不答应。”
\par “可这怎么叫麻烦他呢?”维尔迪兰先生说,“我们发现的那个升F调奏鸣曲,斯万先生也许还没有听过;他可以为我们弹弹那首为钢琴改编的曲子。”
\par “啊!不,不,别弹我的那首奏鸣曲!”维尔迪兰夫人叫道,“我可不想跟上次那样,哭得得了鼻炎,外带颜面神经痛;谢谢了,我可不想再来一次;你们都是一片好意,可是该卧床一星期的不是你们!”
\par 这样一场小戏,每当钢琴家要演奏时总要演出一番,却总跟首次上演一样,观众都乐于观看,仿佛它说明女主人是何等独出心裁,她对音乐又是何等敏感。聚在她身边的人赶紧招呼在远处吸烟或者打牌的人,让他们往前靠靠,示意就要有什么重要的事情发生,还像在国会辩论时的关键时刻中那样,嚷道:“听着,听着!”到了第二天,他们还直为没有到场的人惋惜,说头天那场小戏演得比平常还有意思。
\par “好吧!好吧!”维尔迪兰先生说,“他就只弹行板吧!”
\par “只弹行板!你这是什么话?”维尔迪兰夫人高声叫道,“弄得我浑身瘫软的正是这段行板。你这位先生真是妙不可言!这不就等于说在《第九》里只听终曲,在《大师》\footnote{《第九》指贝多芬的《第九交响曲》,《大师》指瓦格纳的歌剧《歌唱大师》。}里只听序曲一样吗?”
\par 戈达尔大夫还是劝维尔迪兰夫人让钢琴家演奏,倒不是因为他认为音乐在她身上产生的激动是假装出来的,他知道她有些神经衰弱的症状,而是因为许多大夫都有这样一种习惯,当他们参加一个社交活动(他们认为它的成功与否至关重要),而他们奉劝暂时忘掉消化不良或者头痛的那个人又是这个活动的关键人物时,马上就把疾病的严重性说得缓和一些。
\par “您今天是不会闹病的,”他对她说,一面向她递眼色示意,“再说,如果您闹病了,我们也会照料您的。”
\par “真的?”维尔迪兰夫人答道,仿佛在这样的盛情所展现的希望面前,只好退让了。也许同时也因为,当她说她会病倒的时候,有时是忘了这是一句谎话,是一种病态心理。而病人时常不愿意为了少发病而处处小心提防,很容易相信他们可以不受惩罚地做他们高兴做而常常因此而得病的事情,只要能把自己的命运交到一个强者手里,自己不必费力,就可以凭一句话或者一颗药丸而复原就行了。
\par 奥黛特已经走到钢琴旁边的一张毛毯面子的沙发跟前,坐了下来。
\par “这是我的安乐窝。”她对维尔迪兰夫人说。
\par 维乐迪兰夫人看到斯万坐在一把椅子上,就请他站起来:
\par “您在那里不舒服,您还是坐到奥黛特身边来吧。奥黛特,您能腾点地方给斯万先生吗?”
\par “多漂亮的博韦毛毯。”斯万在坐下以前说,他竭力要显得亲切。
\par “啊!您欣赏我的沙发,我真高兴,”维尔迪兰夫人答道,“您如果还想看到一张跟这张同样好看的沙发,那我就劝您趁早打消这个念头。这种款式的沙发,他们从来就没有做过第二张。那些小椅子也都是珍品。您一会儿可以去看看。每一个青铜铸件都是跟椅子上的图形相配的;如果您有意看一看,您既能学到东西,又能得到享受,准能感到没有白费时光。
\par 您请看看这椅子的镶边,那‘熊与葡萄’红底上的小葡萄藤,画得多好!您说呢?我说他们画画可真有一手!这葡萄是不是叫人馋涎欲滴?我丈夫硬说我不喜欢吃水果,因为我吃得没有他多。其实不然,我比你们诸位都贪吃,只不过我不想把水果吃进嘴里,我要用眼睛欣赏。你们笑什么?你们可以问问大夫,他可以告诉你们,葡萄是我的泻药。有人用枫丹白露的白葡萄治病,我是拿这博韦罩毯治病。斯万先生,您走以前一定要摸摸椅子背上的青铜铸件,是不是又细又光?不要紧,您尽管用手摸好了。”
\par “好嘛!维尔迪兰夫人要摸青铜铸件,”画家说,“我们今晚就听不成音乐了。”
\par “您住嘴,您这个坏坯!”她又转过身来对斯万说,“我们女人哪,连一点最起码的快感都不让享受。这世上有谁的皮肉有这么细!想当年维尔迪兰先生对我醋劲儿挺大,唯恐失去我的时候——得了,别打断我的话,你可别说你从来没有吃过醋……”
\par “我可什么也没说。大夫,我请您作证,我说什么没有?”
\par 斯万出于礼貌,还在抚摩那些青铜铸件,不敢马上撒手。
\par “得了,您往后再抚摩吧;现在到了别人爱抚您,让您一饱耳福的时候了;我想您准会喜欢的;就是这位年轻人来承担这项任务。”
\par 等到钢琴家演奏完毕,斯万对他就比对在座的任何人都更亲切了。这是什么道理?
\par 原来头年他在一次晚会上听人用钢琴和小提琴演奏了一部作品。起初,他只体会到这两种乐器发出的物质性的音质。而当他在小提琴纤细、顽强、充实、左右全局的琴弦声中,忽然发现那钢琴声正在试图逐渐上升,化为激荡的流水,绚丽多彩而浑然一体,平展坦荡而又像被月色抚慰宽解的蓝色海洋那样荡漾,心里感到极大的乐趣。在某一个时刻,他自己也不能清楚地辨认出一个轮廓,也叫不上使他喜欢的东西到底叫什么名字,反正是突然感到着了迷,他就努力回忆刚才那个乐句或者和弦(他自己也说不清);这个乐句或者和弦就跟夜晚弥漫在潮湿的空气中的某些玫瑰花的香气打开我们的鼻孔一样,使他的心扉更加敞开。可能是因为他不知道这是什么乐曲,所以他得到的印象是如此模糊,一种也许正是真正的纯粹音乐的印象,是局限于这个范围,完全别具一格,不能归之于任何别的种类的印象。这样一种印象,在一刹那间,可以说是“无物质的”印象。当然这时我们听到的音符,按照它们的音高和时值,会在我们的眼前笼罩或大或小的空间,描画出错综复杂的阿拉伯式的图案,给我们以广袤或纤小、稳定或反复无常的感觉。然而这些感觉在我们心中还没有牢固地形成,还不足以会被紧接而来的,甚至是同时发出的音符所激起的感觉淹没以前,就已经消逝了。而这种印象却还会继续以它的流动不定,以它的“淡入或淡出”,掩盖那些不时冒出、难以区别、转瞬即逝、只能由它们在我们身上产生的特殊的快感才得以辨认的,无法形容、无法记忆、无法命名、不可名状的主题——即使我们的记忆,像一个在汹涌的波涛中砌造一个建筑物的牢固基础的工人一样,能为我们提供那些逃遁的乐句的仿制品,却无法使我们能把它们随之而来的乐句加以比较,加以区别。就这样,当斯万感觉到的那个甘美的印象刚一消失,他的记忆就立即为他提供了一个记录,然而那是既不完全又难持久的记录;但当乐曲仍在继续时,他毕竟得以向这记录投上一瞥,所以当这同一个印象突然再次出现时,它就不再是不可捕捉的了。他可以捉摸这个印象的广度,捉摸与它对称的改编乐句,捉摸它的记谱法,捉摸它的表现力;他面前的这个东西就不再是纯音乐的东西,而是帮助他记住这音乐的图案、建筑物和思想了。这时候,他就能清楚地辨认出那个在片刻之间、在音响之波中升腾而起的乐句。它立刻唤起他一些奇妙的快感,他感到这是除了这个乐句以外任何别的东西都不可能给予他的,因此对它产生了一种从未体验过的喜爱。
\par 这个乐句以缓慢的节奏把他领到这里,把他领到那里,把他领向一个崇高、难以理解,然而又是明确存在的幸福。突然间,正当这个乐句把他领到一个地方,而他在休息片刻后正准备随它继续前进时,它却猛地变换方向,以速度更快的细碎、凄然、温和而无休止的运动,把他带向新的境界,随即又消逝了。他热切地祈望着第三次再见到它。而它果然又重现了,然而并没有对他作出什么更明确的启示,在他身上激起的快感也没有以前那样深刻。可是当他回到家里,他却需要它:他仿佛成了这样一个人,他在马路上瞥见的一个过路的女子在他的生活中注入了一种崭新的美的形象,这个形象强化了他自己的感情,可他是否还能重逢他已经爱上但却连姓名都还不知道的那个人,连他自己也不清楚。
\par 对这个乐句的爱仿佛在一瞬间在斯万身上产生了恢复已经失去了的青春的可能性。很久以来,他就弃绝了把生活跟一个理想结合起来的念头,只把它局限于追求日常乐趣的满足,而他认为——虽然没有正式地对自己这样说——这种情况到死也不会改变了;更进一步,他既然再也不会感到头脑里有什么崇高的思想,于是就连天下是否有这样的思想存在也不再相信,虽然他还不能完全予以否定。因此,他就养成了逃避存在于琐碎不足道的思想之中的习惯,也就不再去追究事物的原委。同样,他也不再自问是否再参加社交生活,但却确信如果接受邀请就应该应邀前往,而如果临时不能赴约,就应该给主人留张名片;同样在谈话中间他竭力不对任何事物畅谈由衷的见解,只是提供一些本身能多少说明问题,而他自己无需倾其所知的细节。他对菜肴的烹调方法,对某个画家的生卒年代,对他的作品的标题却是了如指掌。有时,他情不自禁地对某一作品,对某种人生观发表见解,但语含讽刺,仿佛他对自己所说的话也并不完全赞同。然而,就像某些多病的人到了一个新的地方,接受一种新的治疗方法,身体上莫名其妙地自发出现一种新的变化,就仿佛觉得自己的病大为减轻,因而开始看到今后与前完全不同的生活的可能性一样,斯万这一回也通过对他所听到的那个乐句的回忆,通过他为了看一看是否还能发现这个乐句而请人演奏的某些协奏曲,在他自己身上发现了以前不再相信的一个看不见的现实;此外,仿佛音乐对他那干涸的心有一种治疗的作用似的,他也重新产生了把生活奉献给某一目标的愿望,甚至是力量。然而,他没能弄清他那晚听的那部作品出于谁手,也没能找到那部作品,结果也就把它忘了。他倒是在那个星期里碰到了那天跟他一起参加那个晚会的几个人,问过他们;可是好几个人都是在演奏完了才到的,或者没有到演奏就已早退;有几个人在演奏时倒是在场,不过在另外一个角落里聊天,另外有几个人倒是听了,可是也是听而不闻。至于晚会的主人,他们只知道这是一部新作品,是他们约请的音乐家们自己提出要演奏的,而这些音乐家到外地巡回演出了。斯万有一些音乐界的朋友,可是他尽管记得这乐句使他产生的无法表达的特殊的乐趣,尽管眼前能看到这个乐句描绘出来的形象,却不能把它哼给他们听听。后来,他也就不再去想它了。
\par 而今晚在维尔迪兰夫人家,年轻的钢琴家刚开始弹了几分钟,斯万忽然在一个延续两小节的高音之后,看到他所爱的那个轻盈的、芬芳的乐句从这拖长的、像一块为了掩盖它的诞生的神秘而悬起的有声之幕那样的音响中飘逸而出,向他款款接近,被他认了出来——这就是那个长期隐秘、细声细气、脱颖而出的乐句。这个乐句是如此不同凡响,它的魅力是如此独一无二,任何别的魅力都无法替代,对斯万来说,就好比在一个朋友家中的客厅里突然遇到他曾在马路上赞赏不已,以为永远也不能再见的一个女人一样。最后,这个不倦的指路明灯式的乐句随着它芳香的细流飘向远方,在斯万的脸上留下了他微笑的痕迹。这次他可以打听这个不相识的人的姓名了,原来这是凡德伊的《钢琴小提琴奏鸣曲》的平板。他把它记住,从此就可以在家里随时重温,研究它的音乐语言,掌握它的秘密了。


\paragraph*{1}

\par 因此,当钢琴家演奏刚完毕,斯万就走到他跟前,向他致谢,那种热烈劲儿,维尔迪兰夫人看了十分高兴。
\par “这是何等的魅力!”她对斯万说,“小伙子对这个奏鸣曲理解得十分透彻,是不是?您从来没有想到钢琴能达到这么高的境界吧!说真的,那里面什么都有,就是没有钢琴声。每次听的时候,我都以为是听一支管弦乐队在演奏。甚至比管弦乐队奏得还美,还完整。”
\par 青年钢琴家躬了躬身,面带微笑,一板一眼地说,仿佛是在念一句警句似的:
\par “您太过奖了。”
\par 维尔迪兰夫人对她的丈夫说:“来,来,给他来杯橘子水。他该得这份奖赏。”斯万则对奥黛特叙说他爱上那句乐句的经过。这时候维尔迪兰夫人说道:“哎,奥黛特,看样子他在跟您讲什么知心话呢!”奥黛特答道:“对了,是知心话。”斯万很欣赏她的直爽。他接着打听凡德伊是怎样一个人,有什么作品,这部奏鸣曲是什么时期写的,他当时写那个乐句的时候要表达什么思想,这是他特别要弄清楚的。
\par 当斯万说这个奏鸣曲真美的时候,维尔迪兰夫人高声叫道:“您说得不错,它真美!您不该说您原来不知道这首奏鸣曲,您没有权利不知道这首奏鸣曲。”画家接茬说:“啊,是啊,这是一部了不起的作品,这当然不是什么大路货,不是什么‘通俗作品’,这是对我们这些懂艺术的人能产生强烈印象的作品。”所有这些人全都自诩能欣赏这个音乐家,可是他们全都从来没有向他们自己提出斯万刚才那些问题,因此谁也答不上来。
\par 甚至当斯万就他心爱的那个乐句发表一两点见解的时候,维尔迪兰夫人却答道:“嗨,您说逗不逗?我可从来没有注意到;我呀,我不喜欢吹毛求疵,不喜欢过问那些鸡毛蒜皮的事儿;这里的人谁也不喜欢费工夫去钻牛角尖,我们家可没有这样的毛病。”这时候戈达尔大夫张着大嘴以赞赏的眼光注视着她,满腔热情地听她一口气说出那么多的成语。他跟他的太太都有某些出身低微的平民百姓的那种世故,对他们回到家里相互承认并不懂得的音乐作品以及比施“大师”的绘画,都避免发表意见,也不假装能够欣赏。广大群众只能从他们已经慢慢接受了的那种艺术当中的老一套的东西里领略大自然的魅力、美和形象,而有独创性的艺术却正在抛弃这些老一套的东西,所以作为广大群众在这方面的代表,戈达尔夫妇既不能在凡德伊的奏鸣曲中,也不能在那位画家的肖像画中发现他们所理解的音乐的和谐和绘画之美。钢琴家演奏的时候,他们觉得他是在钢琴上随便弹上几个音符,这是他们已经习惯的形式所无法联系起来的,而画家只是在画布上随意抹上点颜色而已。当他们在画布上辨认出一个人形时,他们也觉得它笨拙俗气,也就是说,缺乏他们用来观察路上的行人的那个习惯画法所显示的优美,也觉得它不真实,仿佛比施先生不懂得一个人的肩膀是怎么长的,也不知道女人的头发是不会长成淡紫色的。
\par 信徒们散开了,大夫感到这是一个好机会,正当维尔迪兰夫人就凡德伊的奏鸣曲讲完最后一句话的时候,他就像刚学游泳的人挑选没有太多人瞧着他的时候才跳下水一样,突然下定决心叫道:“是啊,这就是一个所谓di primo cartello(第一流)的音乐家!”
\par 斯万就只打听出凡德伊这首奏鸣曲是最近发表的,在一个思想很先进的音乐派别中引起强烈的反响,而广大群众却根本不知道有这么回事。
\par “我倒是认识一个叫凡德伊的人。”斯万说。他想到的是我外祖母的妹妹们的钢琴教师。
\par “也许就是他?”维尔迪兰夫人叫道。
\par “啊,不!”斯万笑着答道,“如果您见过他,您就不会提出这样的问题了。”
\par “可提出问题就是解决问题嘛!”大夫说。
\par “也许是他的一个亲戚,”斯万又说,“说起来也真够惨的,一个天才竟会是一个老傻瓜的堂兄弟。果然如此,我就情愿受一切折磨,也要让这老傻瓜把我介绍给奏鸣曲的作者。先得接受去找这老傻瓜的折磨,真是件可怕的事情。”
\par 画家知道凡德伊这会儿病得很厉害,博丹大夫都担心救不活他了。
\par “怎么?”维尔迪兰夫人叫道,“居然还有人找博丹看病!”
\par “啊,维尔迪兰夫人!”戈达尔拿腔拿调地说,“您忘了您是在说我的一个同行,说得更正确些,是我的一个老师。”
\par 画家早就听说凡德伊的精神都快错乱了。他说这从他那首奏鸣曲的某些片段中就可以看得出来。斯万也并不觉得这种看法荒谬,不过却为之不安,因为一部纯粹的音乐作品本来就不包含任何逻辑关系,言语中逻辑关系的错乱表明说话的人神经不正常,但他总认为在一首奏鸣曲中显示出来的错乱却是跟一条狗或者一匹马的精神错乱(尽管当真可以观察出来)同样神秘的东西。
\par “您就别在我跟前提您的什么老师了,您比他高明十倍,”维尔迪兰夫人这样回答戈达尔大夫,用的是一个敢于坚持己见,敢于顶撞持不同意见者的口吻,“您至少不会治死您的病人。”
\par “夫人,他可是位院士,”大夫以嘲讽的口吻反驳道,“如果一个病人乐意死在一个科学泰斗手中的话……一个人要是能说:‘是博丹在给我治病。’那就更光彩了。”
\par “啊!更光彩?”维尔迪兰夫人说,“敢情现在生病还有什么光彩不光彩的,真是新鲜事儿……您可把我逗死了!”她突然双手捂脸叫了起来,“我这个老傻瓜还在跟您正儿八经地讨论呢,竟没有看出您是在愚弄我。”
\par 至于维尔迪兰先生,他觉得为了这么点儿小小不然的事儿就哈哈大笑,未免有点讨人嫌,就猛抽一口烟斗,不无伤心地心想在对人和蔼可亲上面怎么也赶不上他的妻子了。
\par 当黛奥特跟她道晚安告别时,维尔迪兰夫人对她说:“我们很喜欢您的朋友。他很爽直,很可爱;您要是还有这样的朋友介绍给我们,尽管带他们来好了。”
\par 维尔迪兰先生却指出斯万对钢琴家的姑妈并不欣赏。
\par “我想这是因为他对咱们这个环境还不熟悉的缘故,”维尔迪兰夫人答道,“你可不能指望他第一次来就跟戈达尔一样跟这里的人一个调子,戈达尔参加我们这个小圈子已经好几年了。第一次不算数,只能算是了解了解情况。奥黛特,他答应明天跟我们一起到夏特莱剧院去,您是不是去接他一下?”
\par “不,他不要我去接。”
\par “那就随你们吧。但愿他不要临时甩掉我们!”
\par 出乎维尔迪兰夫人意料之外,他从来没有把他们甩掉过。随便他们到什么地方,他都奉陪,或是到郊区的饭馆(还不到时令,去得较少),而更常去的是戏院(维尔迪兰夫人很爱看戏)。有一天维尔迪兰夫人在她家里对斯万说,碰到什么戏的首场演出,或是盛大的节日活动,要是有一张特别通行证就非常管用,甘必大\footnote{甘必大(1838—1882),法国资产阶级政治活动家,第二帝国时期共和派左翼领袖。1870年巴黎被普军围困时曾到外地企图组织新军抗击普军。在反对保皇党恢复帝制,捍卫第三共和国方面有功,逝世时任政府总理。}葬礼那天就因为没有这么一张东西而添了不少麻烦。斯万从来没有提起他那些显赫的朋友,只提那些没有多大声望的,认为后一种关系如果加以隐瞒,未免不够正派;而在圣日耳曼区他就认为跟政界的交往无需隐瞒。这次却冲口而出:
\par “这事儿就交给我了,等《达尼谢夫》重新上演的时候,您就能拿到手了。我明天正好要到爱丽舍宫跟警察总监一起吃饭。”
\par “什么,在爱丽舍宫?”戈达尔大夫高声叫道,简直像是雷鸣一般。
\par “对了,在格雷维先生那里。”斯万答道,对他刚才那句话产生的反应多少有点窘色。
\par 画家对大夫开玩笑说:“您这倒是少见哪!”
\par 一般说来,戈达尔每次听人作出什么解释的时候,总是连声说“好,好”,也不显露什么表情,可是这一次,斯万最后这句话却没有跟往常一样让他安下心来,而是使他万分震惊,敢情跟他同桌吃饭,既无官衔又无任何名声的这个人竟跟国家元首来往呢。
\par “怎么?格雷维先生?您认识格雷维先生?”他对斯万说,那副吃惊和怀疑的神气就仿佛是爱丽舍宫门口站岗的门警碰上前来求见共和国总统的陌生人时一样:根据对方的言语,他明白他是何许人,满口答应他即将受到总统接见,其实却把这可怜的精神病患者领到拘留所的特别诊室去。
\par “我认识他,可不很熟,我们有些共同的朋友(他不敢说出威尔斯亲王的名字),再说,他很好客,那里的饭局也没有多大意思,菜很简单,席上也从不超过八个人。”斯万答道,他竭力把他跟共和国总统的交往中可能在对方看来过分眼花缭乱的事情略去不提。
\par 戈达尔当真信了斯万的话,当真以为格雷维先生的邀请没有什么了不起,并不是什么众所追求而是唾手可得的东西。从此以后,他就对斯万或者别的什么人去爱丽舍宫不再感到惊讶,甚至对他应邀参加那样乏味的宴会表示同情了。
\par “啊,好,好!”他说,那口气就仿佛是个海关关员,刚才还对你表示怀疑,听了你的解释以后,就在你的签证上盖上章,没有打开你的箱子就让你过去了。
\par “您说那里的宴会没有多大意思,我相信也是这样;您去参加这样的宴会,真是难能可贵。”维尔迪兰夫人说,在她眼里,共和国总统是个特别可怕的讨厌家伙,因为他手里掌握着诱惑人和强制人的手段,要是她拿来对付她的信徒的话,那是会叫他们退避三舍的,“听说他耳背得厉害,吃饭还用手指头呢。”
\par “本来嘛,上那儿去,您是不会玩得痛快的。”大夫带着点怜悯说。当他想起一桌只有八个人的时候,又问道:“莫非那是知己朋友间的便酌?”那种热心劲儿与其说是出之于好奇,倒不如说是出之于一个语言学家的钻研精神。
\par 然而共和国总统在他心目中的威望最终毕竟还是胜过了斯万的谦虚和维尔迪兰夫人的恶意,戈达尔在每次聚餐的时候总要关切地问道:“咱们今晚能见到斯万先生吗?他跟格雷维先生有私交。我想他就是一个大伙所说的gentleman(绅士)吧?”他甚至送给他一张牙科展览会的请帖。
\par “有了这张请帖,您还可以带别人进去,不过不能带狗。您知道,我所以说这个话,是因为我有几个朋友不知道这个规定,临时添了麻烦。”
\par 至于维尔迪兰先生,他可注意到了斯万有这样强有力的朋友而以前一直没有说起,这一发现在他妻子身上产生了何等不良的印象。
\par 要是没有安排外出活动的话,斯万就到维尔迪兰家中参加这个小圈子的活动,不过他只是到晚上才来,而且尽管奥黛特一直恳求,他也没有答应跟他们在一起吃晚饭。
\par “如果您愿意的话,我可以跟您单独吃饭。”她对他说。
\par “那维尔迪兰夫人呢?”
\par “啊,那很简单。我只消跟她说我的衣服还没有做好,我的马车来晚了就行了。总有办法应付的。”
\par “您真好。”
\par 不过斯万心想,如果让奥黛特知道(他只同意在晚饭后同她见面),他还有比跟她在一起更大的乐趣的话,那么她在他身上不久就更要得寸进尺了。再说,他早已爱上了一个长得鲜艳丰满得像一朵玫瑰花似的小女工,她的体态之美远过于奥黛特,他宁愿在黄昏时分跟她在一起,然后再去跟奥黛特相会。出于同样的理由,他从来没有答应奥黛特上他家去接他一起到维尔迪兰家去。小女工总是在他家附近他的马车夫雷米知道的一个街角等他,到时候登上车来,坐到斯万身旁,在他怀里一直呆到维尔迪兰家门口。等他进客厅的时候,维尔迪兰夫人指着他早上送去的玫瑰花对他说:“我可要说您了。”同时指着奥黛特身边的位子叫他坐下,这时钢琴家正为他们两个人演奏凡德伊的那个乐句——它仿佛是他俩爱情的国歌。他总是从小提琴的震音部分开始,有几拍是不带伴奏的,占着最显著的地位;然后这震音部分仿佛突然离去,而那个乐句就像霍赫\footnote{霍赫(1629—1677),荷兰画家,善于表现室内光的效果。}室内画中的物体由于半开着的狭窄门框而显得更深远一样,从遥远的地方,以另一种色彩,在柔和的光线中出现了;它舞姿轻盈,带有田园风味,像是一段插曲,属于另一个世界。这个乐句以单纯而不朽的步伐向前移动,带着难以用言语形容的微笑,将它的优美作为礼品向四面八方施舍;可是斯万现在却仿佛觉得这个乐句原来的魔力顿然消失了。这个乐句仿佛认识到了它所指引的那种幸福的虚妄。在它轻盈的优美之中已经有点万事俱休的感觉,就好像是随着徒然的遗憾之情而来的超脱之感。不过对他来说,这些都无关紧要,他不大去考虑这个乐句本身,不大去考虑这个乐句对那在创作时并不知道世上有斯万和奥黛特存在的那位音乐家意味着什么,也不大去考虑它对今后几百年的听众意味着什么,而只把它看做是他的爱情的一种证明,一种纪念品,足以使维尔迪兰夫妇、使这位年轻的钢琴家想起奥黛特,想起他斯万,同时把他们两人连结在一起。甚至他也打消了请一位音乐家把那首奏鸣曲整个演奏一遍的打算(奥黛特一时心血来潮,曾经这样要求过的),以至于在全曲当中他依然只知道这一段。奥黛特也附和着说:“咱们干吗要其余部分呢?这才是咱们那一段。”更进一步,后来他都苦于思索了,以致当这个乐句在他们耳畔掠过,离他们虽是那么近,可又像是在无穷远处,虽是为他们而奏,却又不认识他们的时候,他都感到遗憾了,为这个乐句有一种含义,有一种内在的、不变的而又不为他们所知的美而感到遗憾——就像是当我们收到我们所爱的女子送来的珠宝或者所写的情书时,我们会怪怨宝石的水色和语言中的词语为什么不纯粹是由一段短暂的恋情和一个举世无双的情人的精髓所构成一样。
\par 他时常在到维尔迪兰家去以前跟那个年轻女工在一起呆的时间太久,以致钢琴家刚把那个乐句演完,他就发现奥黛特回家的时刻马上就要到了。他总是把她送到凯旋门背后拉彼鲁兹街她那小住宅的门口。也许正是因为这一点,正是为了不要求她给以全部特殊优遇,他才牺牲早些看到她,跟她一起到维尔迪兰家去这个对他来说并不那么必要的乐趣,而保留伴送她回家的特权——这是她十分领情而他也更为重视的一项特权,因为这样,他就会感到没有别人看到她,没有人介入他们两人之间,而且在跟她分手以后,也没有人妨碍她在精神上与他同在。
\par 就这样,她每晚都坐斯万的马车回去。有一晚,当她从车上下来,他跟她说“明天见”的时候,她快步跑到房子前的小花园里采摘最后一朵菊花,在车走动以前送到他的手里。他在归途中一直吻着这朵花,过了几天,花枯萎了,他就小心翼翼地把它收在写字台里。
\par 可是他夜晚从不踏进她的家门。只有两个下午,他去参加了在她看来是如此重要的活动——吃午茶。在这里的这些小街上,几乎全都是一所挨着一所的矮小住宅,只是偶尔有几家昏暗的小铺子(这是这个过去名声不佳的地段的历史遗迹)打破这种单调一致。这些小街的寂静和空荡、花园和树上残留的白雪、冬季的衰败景象,城市中保留下来的自然景色,这些都为他在进门时感到的温暖和看到的花朵增添了神秘的色彩。
\par 奥黛特的卧室位于高出于街面的底层,面临着与前街平行的一条狭窄的后街;卧室右边是一道陡直的楼梯,两旁是糊着深色壁纸的墙,墙上挂着东方的壁毯、土耳其的串珠、一盏用丝线绳吊起的日本大灯(为了避免来客连一点西方文明的现代化起居设备都享受不到,点的是煤气)。这道楼梯一直通到楼上的大小客厅。两间客厅前面有个狭小的门厅,墙上装着花园里那种用板条做的格子架,沿着它的整个长度摆着一个长方形的木箱,里面像花房里那样种着一行盛开的大菊花,这在那年月还是比较罕见的,虽然还没有日后的园艺家培植的那样巨大。斯万看了虽然有些不快,因为种大菊花是头年才在巴黎流行开的风尚,但这回看到这些在冬季灰暗的阳光中闪烁的短暂的星辰发出的芬芳的光芒,在这间半明半暗的小屋中映出一道道粉红的、橙黄的、白色的斑纹,心里还是很高兴的。奥黛特穿着粉红色的绸晨衣接待他,脖颈和胳膊都裸露着。她请他在她身边坐下,那是在客厅深处的许多神秘的隐秘角落之一,有种在中国大花盆里的大棕榈树或者挂着相片、丝带和扇子的屏风挡着。她对他说:“您这么坐着不舒服,来,我来给您摆弄一下。”她面带那种行将一显身手的得意的微笑,拿来几个日本绸面垫子,搓搓揉揉,仿佛对这些值钱东西毫不在乎,然后把它们垫在斯万脑袋后面和脚底下。仆人进来把一盏盏灯一一放好,这些灯几乎全都装在中国瓷瓶里,有的单独一盏,有的两盏成双,都放在不同的家具上(也可以说是神龛上),在这冬季天已近黄昏的苍茫暮色中重现落日的景象,却显得更持久,更鲜艳,更亲切——这种景象也许可以使得伫立在马路上观赏橱窗中时隐时现的人群的一个恋人遐想不已。奥黛特这时一直盯着她的仆人,看他摆的灯是不是全都摆在应有的位置。她认为,哪怕只有一盏摆得不是地方,她的客厅的整体效果就会遭到破坏,她那摆在铺着长毛绒的画架上的肖像上的光线就会不对劲儿。所以她急切地注视这笨家伙的一举一动,当他挨近她那唯恐遭到损坏而总是亲自擦拭的那对花瓶架时,就严厉地申斥他,赶紧走上前去看看花是否被他碰坏。她觉得她那些中国小摆设全都有“逗人”的形态,而兰花,特别是卡特来兰,也是一样,这种花跟菊花是她最喜爱的花,因为这些花跟平常的花不同,仿佛是用丝绸、用缎子做的一样。她指着一朵兰花对斯万说:“这朵兰花仿佛是从我斗篷衬里上铰下来似的,”话中带着对这种如此雅致的花的一番敬意;它是大自然赐给她的一个漂亮的、意想不到的姐妹,在实际生活中难以觅得,而它又是如此优雅,比许多妇女都更尊贵。因此她在客厅中给它以一席之地。她又让他看画在花瓶上或者绣在帐幕上的吐着火舌的龙、一束兰花的花冠、跟玉蟾蜍一起摆在壁炉架上的那匹眼睛嵌有宝石的银镶单峰驼,一会儿假装害怕那些怪物的凶相,笑它们长得那么滑稽,一会儿又假装为花儿的妖艳而害臊,一会儿又假装忍不住要去吻一吻被她称之为“宝贝”的单峰驼和蟾蜍。这些做作的动作跟她对某些东西的虔诚恰成鲜明的对比,特别是对拉盖圣母的虔敬。当她在尼斯居住时,拉盖圣母曾把她从致命的疾病中拯救过来,因此她身上总是带着这位圣母的金像章,相信它有无边的法力。奥黛特给斯万递上一杯茶,问他:“柠檬还是奶油?”当他回答是“奶油”的时候,就笑着对他说:“一丁点儿?”一听到他称赞茶真好喝的时候,她就说:“您看,我是知道您喜欢什么的。”的确,斯万跟她一样,都觉得这茶是弥足珍贵的,而爱情也如此需要通过一些乐趣来证实它的存在,来保证它能延续下去(要是没有爱情,这些乐趣就不成其为乐趣,也将随爱情而消失),以至当他在七点钟跟她分手,回家去换上晚间的衣服时,他坐在马车上一直难以抑制这个下午得到的欢快情绪,心想,“能在一个女子家里喝到这么难得的好茶,该多有意思!”一个钟头以后,他接到奥黛特的一张字条,马上就认出那写得大大的字,她由于要学英国人写字的那种刚劲有力,字写得虽不成体,却还显出是下了工夫的;换上一个不像斯万那样对她已有好感的人,就会觉得那是思路不清、教育欠缺、不够真诚、缺乏意志的表现。斯万把烟盒丢在她家里了。她写道:“您为什么不连您的心也丢在这里呢?如果是这样的话,我是不会让您收回去的。”
\par 他的第二次访问也许对他来说更加重要。跟每次要见到她时一样,他这天在到她家去的途中,一直在脑子里勾勒她的形象;为了觉得她的脸蛋长得好看,他不得不只回忆她那红润鲜艳的颧颊,因为她的面颊的其余部分通常总是颜色灰黄,恹无生气,只是偶尔泛出几点红晕;这种必要性使他感到痛苦,因这说明理想的东西总是无法得到,而现实的幸福总是平庸不足道的。他那天给她带去她想看的一幅版画。她有点不舒服,穿着浅紫色的中国双绉梳妆衣,胸前绣满了花样。她站在他身旁,头发没有结拢,披散在她的面颊上,一条腿像是在舞蹈中那样曲着,以便能俯身看那幅版画而不至太累;她低垂着头,那双大眼睛在没有什么东西使她兴奋的时候一直现出倦怠不快。她跟罗马西斯廷小教堂一幅壁画上叶忒罗的女儿西坡拉\footnote{西坡拉是《圣经》中犹太人领袖摩西的妻子,叶忒罗是她的父亲,摩西的岳父。}是那么相像,给斯万留下了深刻的印象。斯万素来有一种特殊的爱好,爱从大师们的画幅中不仅去发现我们身边现实的人们身上的一般特征,而且去发现最不寻常的东西,发现我们认识的面貌中极其个别的特征,例如在安东尼奥·里佐\footnote{安东尼奥·里佐,十五世纪意大利建筑师、雕塑家。}所塑的威尼斯总督洛雷丹诺的胸像中,发现他的马车夫雷米的高颧骨、歪眉毛,甚至发现两人整个面貌都一模一样;在基兰达约\footnote{基兰达约(1449—1494),意大利画家,米开朗琪罗年幼时曾从他学画。}的画中发现巴朗西先生的鼻子;在丁托列托\footnote{丁托列托(1518—1594),意大利文艺复兴后期威尼斯画派重要画家之一。}的一幅肖像画中发现迪·布尔邦大夫脸上被茂密的颊髯占了地盘的腮帮子、断了鼻梁骨的鼻子、炯炯逼人的目光,以及充血的眼睑。也许正是由于他总是为把他的生活局限于社交活动、局限于空谈而感到悔恨,因此他觉得可以在大艺术家的作品中找到宽纵自己的借口,因为这些艺术家也曾愉快地打量过这样的面貌,搬进自己的作品,为作品增添了强烈的现实感和生动性,增添了可说是现代的风味;也许同时也是由于他是如此深深地体会到上流社会中的人们是这么无聊,所以他感到有必要在古代的杰作中去探索一些可以用来影射今天的人物的东西。也许恰恰相反,正是因为他具有充分的艺术家的气质,所以当他从历史肖像跟它并不表现的当代人物的相似中看到那些个别的特征取得普遍的意义时,他就感到乐趣。不管怎样,也许是因为一段时间以来他接受了大量的印象,尽管这些印象毋宁是来自他对音乐的爱好,却也丰富了他对绘画的兴趣,所以他这时从奥黛特跟这位桑德洛·迪·马里阿诺(人们现在多用他的外号波堤切利\footnote{波堤切利(1445—1510),意大利文艺复兴时期的画家。}来称呼他,但这个外号与其说是代表这位画家的真实作品,倒不如说是代表对他的作品散布的庸俗错误的见解)笔下的西坡拉的相像当中得到的乐趣也就更深,而且日后将在他身上产生持久的影响。现在他看待奥黛特的脸就不再根据她两颊的美妙还是缺陷,不再根据当他有朝一日吻她时,他的双唇会给人怎样的柔软甘美的感觉,而是把它看做一束精细美丽的线,由他的视线加以缠绕,把她脖颈的节奏和头发的奔放以及眼睑的低垂连结起来,连成一幅能鲜明地表现她的特性的肖像。
\par 他瞧着她,那幅壁画的一个片段在她的脸庞和身体上显示出来;从此以后,当他在奥黛特身畔或者只是在想起她的时候,他就总是要寻找这个片段;虽然这幅佛罗伦萨画派的杰作之所以得到他的珍爱是由于他在奥黛特身上发现了它,但两者间的相像同时也使得他觉得她更美、更弥足珍贵。斯万责怪自己从前不能认识这样一个可能博得伟大的桑德洛爱慕的女子的真正价值,同时为他能为在看到奥黛特时所得的乐趣已从他自己的美学修养中找到根据而暗自庆幸。他心想,当他把奥黛特跟他理想的幸福联系起来的时候,他并不是像他以前所想的那样,是什么退而求其次地追求一个并不完美的权宜之计,因为在她身上体现了他最精巧的艺术鉴赏力。他可看不到,奥黛特并不因此就是他所要得到手的那种女人,因为他的欲念恰恰总是跟他的美学鉴赏背道而驰的。“佛罗伦萨画派作品”这个词在斯万身上可起了很大的作用。这个词就跟一个头衔称号一样,使他把奥黛特的形象带进了一个她以前无由进入的梦的世界,在这里身价百倍。以前当他纯粹从体态方面打量她的时候,总是怀疑她的脸、她的身材、她整体的美是不是够标准,这就减弱了他对她的爱,而现在他有某种美学原则作为基础,这些怀疑就烟消云散,那份爱情也就得到了肯定;此外,他本来觉得跟一个体态不够理想的女人亲吻,占有她的身体,固然也是顺理成章的事,可是也并不太足道,现在这既然像是对一件博物馆中的珍品的爱慕饰上花冠,在他心目中也就成了该是无比甘美、无比神妙的事情了。
\par 正当他要为几个月来把全部时间都用来看望奥黛特而后悔的时候,他却又在想在一件宝贵无比的杰作上面花许多时间是完全合乎情理的事情。这是一件以另有一番趣味的特殊材料铸成的杰作,举世无双;他有时怀着艺术家的虔敬、对精神价值的重视和不计功利的超脱,有时怀着收藏家的自豪、自私和欲念加以仔细观赏。
\par 他在书桌上放上一张《叶忒罗的女儿》的复制品,权当是奥黛特的相片。他欣赏她的大眼睛,隐约显示出皮肤有些缺陷的那纤细的脸庞,沿着略现倦容的面颊的其妙无比的发髻;他把从美学观点所体会的美运用到一个女人身上,把这美化为他乐于在他可能占有的女人身上全都体现出来的体态上的优点。有那么一种模糊的同感力,它会把我们吸引到我们所观赏的艺术杰作上去,现在他既然认识了《叶忒罗的女儿》有血有肉的原型,这种同感就变成一种欲念,从此填补了奥黛特的肉体以前从没有在他身上激起的欲念。当他长时间注视波堤切利这幅作品以后,他就想起了他自己的“波堤切利”,觉得比画上的还美,因此,当他把西坡拉的相片拿到身边的时候,他仿佛是把奥黛特紧紧搂在胸前。
\par 然而他竭力要防止的还不仅是奥黛特会产生厌倦,有时同时也是他自己会产生厌倦。他感觉到,自从奥黛特有了一切便利条件跟他见面以后,她仿佛没有多少话可跟他说,他担心她在跟他在一起时的那种不免琐碎、单调而且仿佛已经固定不变的态度,等到她有朝一日向他倾吐爱情的时候,会把他脑子里的那种带有浪漫色彩的希望扼杀掉,而恰恰是这个希望使他萌生并保持着他的爱情。奥黛特在他心目中的形象已经到了固定不变的地步,他担心他会对它感到厌倦,因此想把它改变一下,就突然给她写了一封信,其中充满着假装出来的对她的失望和愤懑情绪,在晚饭前叫人给她送去。他知道她将大吃一惊,赶紧给他回信,而他希望,她因失去他的这种担心而使自己的心灵陷入矛盾之时,她会讲出她还从来没有对他说过的话。事实上,他也曾用这种方式收到过她一些前所未有的饱含深情的信,其中有一封是一个中午在“金屋餐厅”派人送出的(那是在救济西班牙木尔西亚水灾灾民日),开头写道:“我的朋友,我的手抖得这么厉害,连笔都抓不住了。”他把这封信跟那朵枯萎的菊花一起收藏在那个抽屉里。如果她没有工夫写信,那么当他到维尔迪兰家时,她就赶紧走到他跟前,对他说:“我有话要对您讲,”他就好奇地从她的脸上,从她的话语中捉摸她一直隐藏在心里没有对他说出的是什么。
\par 每当他快到维尔迪兰家,看到那灯火辉煌的大窗户(百叶窗是从来不关的),想到他就要见到的那个可爱的人儿沐浴在金色的光芒之中时,他就心潮澎湃。有时候,客人们的身影映照在窗帘上,细长而黝黑,就像绘制在半透明的玻璃灯罩上的小小的图像,而灯罩的另一面则是一片光亮。他试着寻找奥黛特的侧影。等他一进屋,他的眼睛就不由自主地发出如此愉快的光芒,维尔迪兰对画家说:“看吧,这下可热闹了。”的确,奥黛特的在场给这里添上了斯万在接待他的任何一家都没有的东西:那是一个敏感装置,一个连通各间房间,给他的心带来不断的刺激的神经系统。
\par 就这样,这个被称之为“小宗派”的社交机构的活动就为斯万提供跟奥黛特每天会面的机会,使他有时能以假装对跟她见面不感兴趣,甚至是假装以后不想再跟她见面,但这些都不会产生什么严重后果的,因为尽管他在白天给她写了信,晚上一准还是会去看她,并且把她送回家去的。
\par 可是有一回,当他想起每晚总少不了的伴送时忽然感到不快,于是就陪他那小女工一直到布洛尼林园,好推迟到维尔迪兰家去的时间。就这样,他到得太晚,奥黛特以为他不来了,就回家了。见她不在客厅,斯万心里感到难过;在此之前,当他想要得到跟她见面的乐趣时,他总是确有把握能得到这种乐趣的,现在这种把握降低了,甚至使我们完全看不到那种乐趣的价值(在其他各种乐趣中也是一样),而今天才是第一次体会到了它的分量。
\par “你看见没有,当他发现她不在的时候,那张脸拉得多长!”维尔迪兰先生对他的妻子说,“我看他是爱上她了。”
\par “什么拉得多长?”戈达尔粗声粗气地问。他刚去看一个病人,现在回来找他的妻子,不知道他们讲的是谁。
\par “怎么?您刚才在门口没有碰上斯万家中最漂亮的那一位?”
\par “没有。斯万先生来了?”
\par “才呆了一会儿。斯万刚才可激动,可神经质了。您看,奥黛特走了。”
\par “您是说,她现在已经跟他打得火热,已经到了‘人约黄昏后’的阶段了?”大夫说,对他用的暗喻洋洋得意。
\par “不,绝对不是。咱们关起门来说说,我觉得她处理不当,简直是个傻瓜,实在是个傻瓜。”
\par “得了,得了,得了,”维尔迪兰先生说,“你知道什么呀?他们两个之间什么关系也没有?咱们又没有去看过,咱们怎么知道?”
\par “要是有什么的话,她是会对我说的,”维尔迪兰夫人郑重其事地反驳道,“我对你们说吧。她什么事情也不瞒我。她这会儿没有人,我跟她说过,她应该跟他睡觉。可她说她不能,她虽然钟情于他,可是他在她跟前总是畏畏缩缩的,她也就不敢大胆了。她还说她并不以那样一种方式来爱他,他是一个柏拉图式的情人,她不愿玷污她自己对他的感情。这都是她的话。斯万这个人倒恰恰是她所要的那种人。”
\par “对不起,我的意见可跟你不一样,”维尔迪兰先生说,“这位先生并不完全合我的心意,我觉得他有点摆架子。”
\par 维尔迪兰夫人整个身体都僵直了,脸上现出一副死气沉沉的表情,仿佛她已经变成了一座雕像,这么一来倒显得她没有听到那叫人无法忍受的“摆架子”三个字。对他们“摆架子”,那不就表明他比他们“高明”吗?
\par “不管怎么说吧,如果他们之间没有什么关系,我也并不认为那是因为这位先生认为她是个贞洁的女人,”维尔迪兰先生酸溜溜地说,“不过,这倒是真的,他仿佛觉得她是个聪明人。不知你有没有听到那天晚上他是怎样跟她谈凡德伊的奏鸣曲的;我是衷心喜欢奥黛特的,可是跟她讲什么美学理论,那才是天字第一号的大傻瓜呢!”
\par “嗨,别说奥黛特的坏话,”维尔迪兰夫人装出孩子撒娇的样子说,“她是很可爱的。”
\par “那也不妨害她可爱呀!我并不是说她的坏话,我只是说她既不是个贞洁的女人,也不是个聪明的女人。”他又对画家说,“说到底,她贞洁不贞洁又是什么大不了的事儿呢?贞洁了,她也许就远不如现在这样可爱了,是不是?”
\par 斯万在楼梯平台上碰到了维尔迪兰家的听差头,刚才他上楼的时候,他正好离开了一会儿。奥黛特临走时托他告诉斯万(这已经是一个钟头以前的事情了),假如他来,就对他说,她可能在回家以前先上普雷福咖啡馆喝杯巧克力。斯万马上到普雷福咖啡馆去,可是马车每走一步都被别的车辆或者过街的行人挡住;要不是怕招惹警察干涉,时间会耽误得更久的话,他真想把他们碾死。他计算他所费的时间,把每一分钟都延长几秒,唯恐时间跑得太快,这样他就可以相信有更多的机会到得早些,还能找到奥黛特。突然间,就像一个发烧的病人刚从睡梦中醒来,意识到他刚才反复出现在脑海而难以从中分辨出自己的那些梦幻是何等荒谬一样,斯万也在自己身上发现,自从在维尔迪兰家里听到奥黛特已经走了的消息以后,他脑子里盘算的思想是何等异乎寻常,他心里的那种痛苦又是何等前所未见,他只是在此刻才发觉,仿佛他是刚从梦中醒来一样。什么?所有这些烦躁不安,全都是因为他要到明天才能见到奥黛特,而这不正是他在一个钟头以前在到维尔迪兰家去的路上所盼望的事情码?他不得不看到,把他载到普雷福咖啡馆去的这辆马车依然如故,可是他自己已经不再是原来那样一个人了,他已经不是单独一人,现在另有一个人和他在一起,这个人附在他身上,和他融而为一,也许不再能摆脱,不得不像对待一个主人或者一种疾病那样来与之周旋了。然而自从他感觉到有一个新人就这样附到他身上那一刻起,他也就感到生活更有意思了。能不能在普雷福咖啡馆见到她,他心中完全无数(这等待是如此折磨着他,以至在见到她以前,他方寸已乱,既不能思想,也不能回忆什么来使他的脑子平息下来),然而果然能够见到她,这次会见很可能跟往常一样,并没有什么了不起。跟每天晚上一样,一见到奥黛特,向她那变化无常的脸悄悄地投过一瞥,他马上就把视线转向他方,免得她从中看出有什么欲念的成分,而不再相信他并没有任何的私心杂念;这时他就不再有工夫去想她,而一心盘算找出什么借口来使他可以不立即离开她,同时不露声色地确保第二天能在维尔迪兰家中再次看到她,也就是说找出什么借口来把跟这个可接近而不敢拥抱的女子的不能开花结果的聚首而激起的失望与折磨在当时持续下去,并在第二天重新品尝。
\par 她不在普雷福咖啡馆。他决心到环城马路所有的饭店去找她。为了争取时间,当他到一些饭店去的时候,他就打发他的马车夫雷米(里佐画中的洛雷丹诺总督)上另一些饭店,如果他自己找不着,就到指定的地点去等马车夫。马车夫不见回来,斯万心里直翻腾,仿佛一会儿看到他回来说:“夫人在那里。”一会儿又看到他回来说:“夫人哪个咖啡馆里也找不着。”眼看天色已晚,也许今晚可能以跟奥黛特相会而告终,这就可以结束他的焦灼;也许不得不死了今晚找到她的念头,只好未曾相遇而黯然回家了。
\par 马车夫回来了,可是当他在斯万面前停下的时候,斯万并没有问他“找到夫人没有?”却说:“明天提醒我去订购劈柴,看来家里的快用完了。”也许他心里在想,如果雷米在哪个咖啡馆看到了奥黛特还在等他的话,那么这个倒霉的夜晚就已经被一个业已开始的幸福的夜晚取而代之了,他就用不着匆匆忙忙地奔向那已经到手、妥善收藏、万无一失的幸福了。不过这也是出之于惯性的作用;有些人的身体缺乏灵活性,当他们要躲避一次冲撞,把他们行将烧着的衣服从火苗边拽开,要作出一个紧急的反应时,他们却不慌不忙,先把原来的姿势保持一会儿,仿佛要从这个姿势中寻得一个支点,一股冲力似的。斯万这会儿则是在心灵中缺乏这么一种灵活性。假如车夫对他说“夫人在那里”的话,他多半也会这样回答:“啊!好,好!让你跑了这么多路,我没想到……”并且继续谈订购劈柴的事,免得让他看出自己情绪的激动,同时让自己有时间从不安转入幸福。
\par 车夫再一次回来告诉他,哪儿也找她不着,并且以老仆人的身份,提出自己的意见:
\par “我想先生只好回家了。”
\par 当雷米带来他最后的、无法改变的回音时,斯万尽可以装出满不在乎的样子,可是这次当他看见他打算要他放弃希望,不再寻找时,他可就装不出来了。他高声叫道:
\par “不,我们一定得把这位夫人找到;这是最重要不过的事情。要是她没有见着我,她会十分懊恼的,这可是件大事,她会生我的气的。”
\par “我可不明白,这位夫人怎么会生气,”雷米答道,“是她没等先生就走了,是她说要到普雷福咖啡馆,而她又不在。”
\par 这时四面八方的灯火都纷纷熄灭了。在林荫大道的树荫下,在神秘莫测的黑影中,越来越稀少的行人在踯躅,几乎分辨不出来。不时有个女人的身影走到斯万跟前,在他耳边嘟囔两句,请他送她回家,把斯万吓了一跳。他惶惶不安地从这些暗淡的身子边擦过,仿佛是在黑暗的王国,在鬼魂丛中寻找欧律狄克\footnote{欧律狄克是希腊神话中歌手俄耳甫斯的妻子,被毒蛇咬伤而死,为了把妻子找回,俄耳甫斯亲身到了冥界。}一般。
\par 在产生爱情的种种方式中,在传播大恶的种种媒介中,有一种是再有效不过的,那就是有时掠过我们体内的强烈的激动之流。我们这会儿乐于与之相处的那个人,她的命运就算是定了,我们从此爱的就是她了。在这以前,她是否比别人更合我们的心意,甚至仅仅是跟别人同等程度地合我们的心意,这都无关紧要。重要的是我们对她的兴趣应该专一。假如她不在我们身边,而我们对跟她相处的种种乐趣的追求,在我们身上突然由一种急迫的需要取而代之时,这个条件就实现了。这个需要以她本人为对象,这是一种荒谬的需要,是这个社会的法律所不允许实现、所难以宽解的一种需要——这就是要占有她的那种荒唐的、痛苦的需要。
\par 斯万让雷米带他到最后几家还没有关门的餐馆;这是他冷静地设想中的那个幸福得以实现的唯一条件。现在他不再掩饰他内心的激动,不再掩饰他对这次相会是何等的重视,于是答应他的马车夫,如果得以成功,就给以重赏,仿佛除了他自己以外再加上另一个人抱着成功的愿望,就可以使奥黛特出现在内环路上的某一个餐馆似的——哪怕她这时已经回家睡觉了也罢。他一直赶到金屋餐厅,两次走进托尔多尼饭店,都没有找着;他又从英国咖啡馆出来,惊慌失措地大踏步赶到在意大利人大道一个街角等着他的马车那里,可就在这时候,他迎面撞上了一个人,她就是奥黛特。她后来解释道,她在普雷福咖啡馆没有找到座位,就上金屋餐厅吃饭去了,她坐在一个凹角里,没有被他看到。她正在找她的马车。
\par 她根本没有想到会在此时此地碰上他,因此大吃一惊。而他呢,他跑遍了整个巴黎城,也并不是因为他认为有可能碰上她,而是因为要是死掉这颗心的话,对他自己未免太残酷了。他的理智一直认为今晚这份快乐是不可能实现的了,现在它却成了再现实不过的东西;他自己并没有去忖度种种可能来促成这份快乐的实现,它纯粹是外来的东西;他也用不着绞尽脑汁来赋予它以现实性,这现实性是它自己产生出来的,是自己向他投来的。这个现实光芒四射,驱散了像梦幻一样飘荡在他心中的孤独之感;而在这个现实之上,他在无意之中构筑起幸福的遐想。这就像一个在晴朗的日子到达地中海岸的旅客一样,对他刚离开的地方是否存在有所怀疑,这时他不去回顾这些地方,却听任迎面而来的海水的明亮而始终如一的蔚蓝色光芒照得自己眼花缭乱。
\par 他跟她一起登上她的马车,让他自己的车子跟在后面。
\par 她手上拿着一束卡特来兰,斯万透过她的花边头巾,看见她头发上也有同样的兰花系在用天鹅的羽毛制成的羽饰上。她在披巾下穿的是一件黑丝绒的袍子,下摆张成三角形,露出白罗缎的衬裙,在袒胸的上衣口有一块也是白罗缎的插绣,上面也插了几朵卡特来兰。她刚从跟斯万的不期而遇的惊讶中恢复过来,马就踢到了什么障碍物,闪向一旁。他们两人都给震得离开了原来的位置,她一声尖叫,吓得心突突地跳,连气也喘不过来。
\par “没有什么,”他对她说,“别害怕。”
\par 他扶住她的肩膀,把她的身子紧紧靠在自己胸前,又说:
\par “千万别说话,只消用手势回答我的问题就行了,免得您喘得更厉害。您上衣口上的花给震歪了,我来给您扶一扶正,您该不介意吧?我怕您的花会掉出来,我想把它插牢一点儿。”
\par 她从来没有见到男人对她这么彬彬有礼过,微笑着答道:
\par “不,哪儿会啊,我怎么能介意呢?”
\par 他却为她的回答而显得很难为情,这也许是由于他自己刚才提出了一个借口却又装得十分诚恳,甚至已经开始相信自己确是诚恳而难为情吧。他叫道:
\par “啊!不,不,千万别说话,您会喘得更厉害的,您只消做个手势就行了,我会明白您的意思的。您果然不介意?您看,您身上有一丁点儿……我想是一丁点儿花粉;您同意我用手把它掸掉吗?我不会使很大劲的,我把您弄痛了吗?也许我把您弄痒痒了?我并不想碰袍子的丝绒,免得把它弄皱了。不过您看,这些花实在应该固定一下,要不然就要掉出来了;我这就把它们插进去一点……您说实话,我还不至于招您讨厌吧!我想闻一闻,看看花的香气是不是全都跑了。什么味儿也闻不见。跟我说实话吧。”
\par 她微笑着耸耸肩膀,仿佛是说:“您真傻,您明明知道我很高兴。”
\par 他用另一只手沿着奥黛特的面颊轻轻地抚摸;她睁眼注视着他,带着佛罗伦萨那位大师所画的女人(他觉得她跟她们是相像的)那种含情脉脉而庄重的神情;她那两只跟画上的女人们相像的明亮秀气的大眼睛仿佛要跟两颗泪珠那样夺眶而出。她粉颈低垂,就跟异教画和基督教画中所有的女子一样。她这时的姿态当然是她惯常的姿态,但她也深深知道这个姿态是适合于当时的场合的,而她也注意着别忘了摆出这样一副姿态;她似乎需要竭尽全力来保持面部的位置,仿佛有一股看不见的力量把它吸引到斯万那边去。当她不由自主地把她的脸迎向斯万的双唇时,斯万用双手把它捧住,保持一段距离。他要让奥黛特有时间来回味一下她久已追求的梦想,来亲眼看到它的实现,就好像人们邀请受奖的孩子的母亲亲眼看看她钟爱的孩子的成就似的。也许斯万自己还有意要好好最后一次凝视一下他迄今还没有占有,甚至还没有吻过的奥黛特的脸,就好像是一个人在离别一个地方时要好好看一下他就要永远离开的那个景色一样。
\par 不过他在她跟前依然还是如此腼腆,以至在那晚以为她摆弄卡特来兰花开始,以占有她的身体告终之后,往后那几天,他还是使用同一个借口,这也许是因为他怕冒犯她,也许是因为怕露出撒谎的马脚,也许是因为缺乏提出比这更高的要求的勇气(其实他是可以再次提出的,因为奥黛特第一次并没有感到不快)。如果她上衣胸口戴着卡特来兰花,他就说:“今晚真不幸,您的卡特来兰花用不着重新摆弄,不像那晚那样乱,然而这一朵仿佛不太正。我倒想闻闻它们是不是特别的香。”要是她没有戴花呢;他就说:“哦!今晚没有卡特来兰花,没法子摆弄了。”就这样,在一段时间内,头一晚那个程序就一直没有变动,总是以用手指和嘴唇轻轻抚弄奥黛特的胸口开始,每次的接吻和拥抱也总是以这样的抚弄为先导;很久以后,当摆弄卡特来兰花(或者类似的礼节)早已过了时,“摆弄卡特来兰”这个暗喻却成了他们习惯性地用来代表肉体的占有这种行为(其实也无所谓占有不占有了)的普通词语,长期留在他们的言语之中,来纪念那个早已被遗忘了的习俗。也许用这种特殊的说法来表达“性关系”,其意义跟它的各种同义词不完全一样。我们尽可以对女人已经感到厌倦,尽可以把跟各种不同类型的女人的交欢看成是并没有什么两样,早就知道是怎么一回事,但是如果那女人不是那么容易到手——或者我们认为不是那么容易到手——以至我们必须在与她的交往中制造一个突如其来的插曲,就像斯万第一次通过摆弄卡特来兰那样,那么这种交欢就会变成一种新鲜的乐趣。斯万那晚急切地盼望着的(他心想如果奥黛特中了他的计,那她是猜不出来的),正是从卡特来兰的宽大的浅紫色花瓣中能结出占有这个女人之果;他那晚感到,而奥黛特也许只是因为没有充分意识到才予以默认的那种乐趣,在他的心目中因此就是一种迄今没有存在过,而是他试图创造出来的乐趣,是一种完全与众不同,完全新鲜的乐趣(正如上帝创造出来的第一个人见到地上的天堂中的花儿时所感到的一样)——他给它起的那个特殊的名称也保留了这点痕迹。
\par 现在,每天晚上,当他把她带回她家时,他就总得进去;她时常穿着晨衣把他送出来,一直送到他的马车边,当着车夫的面和他吻别,说:“给人瞧见了,又有什么关系?”他不上维尔迪兰家去的那些夜晚(自从他可以在别的地方和她相会,这种情况就不时发生了),他到上流社会的社交圈子里去的那些夜晚(这也越来越难得了),她就请他不管时间早晚,在回家前一定先上她家去。这是春天,一个晴朗而寒冷的春天。在从晚会上出来的时候,他登上他的四轮敞篷马车,把毛毯盖到腿上,对跟他同时回家,请他跟他们一道走的朋友们说他不能从命,说他去的是另一个方向,而车夫就扬鞭策马快走,反正他知道该上什么地方。朋友们都感到惊讶,斯万敢情变了。再也收不到他要求介绍女人的信了。他不再注意别的女人,避免到能碰见女人的地方。在餐馆里,在乡下,他的举止也全然变了;朋友们原来可以据以把他辨认出来,也以为今后将永远不变的那种举止也不知哪里去了。一种一时的异常的性格不仅能取代正常的性格,也能消除正常的性格直至此时所表现的恒常的外部特征,激情在我们心中造成的变化也是如此!与此相反,现在却有一件事情是不变的,那就是不管斯万晚上到哪里,他必然要去跟奥黛特相会。把他和她相隔开的这段路程就是他每天必不可少地要走一次的路程,仿佛这是他生命历程中无法避免的一个下滑的陡坡。说实在的,当他在哪个晚会上呆的时候过久时,他时常也想直接回到家里,不再跑这一趟远程,到第二天再去看她;单凭在这么晚的时候不辞辛劳地上她家去,并且猜想跟他道别的朋友们准会窃窃私议:“他是身不由己,准有个娘们强迫他不管时间早晚都得上她家去。”这一点,就使他感到他自己是在过着堕入情网的人们的生活,不惜为感官享受的追求而牺牲休息和利益,准是着了魔了。然而他尽管未假思索,却确信这时她准在等着他,决不跟其他人在别的什么地方,而他准能在回家以前见到她的面,这个信念消除了那晚奥黛特不在维尔迪兰家时他那种焦躁不安的情绪,这种情绪固然早已淡漠,然而随时还会重现,而他现在心中是如此宁静,简直可以说是一种幸福。奥黛特之所以在他心中占有如此重要的地位,也许正应该归功于那晚的焦躁不安。通常,别人跟我们是如此无关,以至当其中有一个人能主宰我们的哀乐时,我们就会觉得他仿佛是属于另一个世界,满身都是诗情画意,能把我们的生活化为一片我们与之同在的感情的海洋。有时,当他在晴朗的寒夜,从他的马车上眺望皎洁的月亮照射下的空无一人的街巷时,他就想到那张跟月色同样明亮而略带玫瑰色的脸,它有一天曾突然从他的脑际浮现出来,从此就将神秘之光投向这个世界。如果他在奥黛特打发她的仆人去睡觉以后到达,他就在按小花园的门铃之前,先到后街去,那里相邻的住宅的窗户全都一模一样,也全都一片漆黑,唯有她卧室那一扇还亮着。他在窗框上敲敲,她就答应一声,然后到大门背后等着。她的钢琴上摆着她喜爱的乐谱,《玫瑰圆舞曲》啦,或是塔里亚菲科\footnote{塔里亚菲科(1821—1900),法国歌唱家及作曲家。}的《可怜的疯子》(她在遗嘱上写明,在葬礼上要奏这个曲子),他却要她弹凡德伊那个乐句,虽然奥黛特弹得很不怎么样,但我们对一部作品的最美好的印象时常是得之于笨拙的指头在走调的钢琴上弹出的不符要求的音响的。他深深地感觉到,他那份爱情是在别处无法找到与之相应之物的东西,是除了他自己以外再也没有人能验证的东西;他也明白,奥黛特的素质也不足以解释他为什么对在她身边度过的时光是如此重视。时常,当他十分冷静地用理性来考虑的时候,他也想不再为了这假想的乐趣而在学问方面和社交方面作出这么重大的牺牲了。但当他一听到凡德伊的那个乐句,它就会在他心中腾出足以容纳它的空间,他的心胸就会因而扩大,为某一种形式的享受留出位置——这种享受也是在它自身之外无法找到与之相应之物的,然而不像爱情的享受那样是纯粹个人的事情,却像一个高出于具体事物的客观现实那样摆在斯万面前。凡德伊那个乐句在他身上唤起了这种对未曾体会过的魅力的渴求,却没有给他带来什么明确的东西使他得以满足。因此,那个乐句在斯万心中消除了对物质利益的关怀,消除了人皆有之的那些考虑所留下的空白,却并没有找到东西来填补,斯万便尽可以在那里镌刻上奥黛特的名字。此外,奥黛特的感情中有所欠缺、有所令人失望的地方,那个乐句也会来加以弥补,注入它那神秘的精髓。当他谛听这个乐句时,从他的脸上仿佛可以看出他正在吸着一种麻醉剂,使他的呼吸更加深沉。音乐给予他的那种转瞬即将化为一种真正的热望的乐趣,在这样的时刻,确实像是我们在做香料的实验时的那种乐趣,像是当我们接触一个不是为我们所造的世界时的那种乐趣——这个世界,在我们看来没有形式,因为我们看不见它;没有意义,因为它为我们的理智所不能掌握;我们只能通过一种感官才能到达那里。斯万的眼虽是敏锐的绘画鉴赏家的眼,他的脑子虽是人情世故的精细的观察家的脑子,它们却从此要带上无法消除的无聊乏味的生活的痕迹;当他感到自己变成了一个与人类无关的人,盲目的人,失去了逻辑能力的人,几乎变成了一个荒诞的传说中的独角兽,变成了仅仅通过听觉来感知世界的怪物时,这对他来说倒是可贵而神秘的休息。既然他要在这乐句中搜寻他的智力所不能及的意义,他就需要以何等的沉醉来不让他的心灵得到理性的任何帮助,来使他的心灵单独通过这乐音之廊,通过这乐音的阴暗的过滤器啊!他已经开始意识到,在这乐句甘美的乐音底下隐藏着怎样的苦楚,也许还是难以消除的隐痛,然而他并不以为苦。让这乐句说什么爱情是脆弱的吧,他的爱情却是如此牢固!他玩弄这乐句散发出的忧郁之情,感觉到它正在流经他的身体,然而总觉得它却像是使他的幸福感更深刻更甜蜜的一种爱抚。他让奥黛特十次、二十次地重复这个乐句,要求她在弹奏的同时不停地吻他。每一个吻都激起另一个吻。啊!在谈恋爱的初期,亲吻是如此自然地诞生!吻一个接着一个,要把一个钟头之内接的吻一个一个数出来,那跟把五月间原野上的鲜花一朵一朵数出来同样困难。这时,她假装要停下来,说道:“你搂着我,叫我怎么弹呀?我可没法子同时兼顾,你倒打定主意,我是该弹那句乐句呢,还是该跟你亲热?”他生气了,她却哈哈大笑,接着是一阵急风骤雨般的亲吻。要不然的话,她忧郁地看着他,他这就又看到她那张值得进入波堤切利的《摩西壁画》的脸,于是把奥黛特的脖颈摆弄一下,让它保持必要的倾斜;当他按照十五世纪西斯廷小教堂的墙上那样用色粉颜料把她的肖像画好以后,想到她这会儿就在身旁,坐在钢琴边,随时准备接受亲吻和交欢,想到她是个有血有肉的人,活生生的人时,他就如痴如狂,双眼圆睁,下巴伸出像是要吃人,扑到波堤切利笔下这个少女身上,拧她的面颊。等他走出了她的家门,又回来把她吻了又吻,因为他刚才一时想不起来她身上的气味或线条的某一特征;当他登上马车,踏上归途,他为奥黛特祝福,因为她同意他每天都去,而这样的聚会,他想并不会给她带来多大的欢乐,却由于可以使他免于产生妒意(再也不会吃像那晚在维尔迪兰家没有见到她时的那种苦头了),而能帮助他不必再遭那样的危机(那第一次是如此痛苦,也该是唯一的一次),就能度过他生命中的那一连几个小时的不同寻常,简直是如痴如狂的时刻,就像他乘车在月夜穿过巴黎的街道时那样。当他在归途中看到月亮现在已经移转,几乎已经靠近地平线时,也想到他的爱情也遵照一些不变的自然规律,自问他现在正在经历的这个时期能否长时持续下去,那张可爱的脸儿的地位是否会越来越下降,越来越失去它的魅力,不久就会从他的脑际消失。自从斯万堕入情网,他感到事物是有魅力的,正如他年轻时自以为是艺术家时那样;然而这不再是同样的魅力,现在的魅力,只有奥黛特才能赋予各种事物。青年时期的灵感被后来的放荡生活驱散了,现在他觉得又在他身上重新萌发,不过这些灵感全都带有特定的生活的反映和印记;现在当他独自一人在家跟复原中的心灵共同度过漫长的时刻时,他感到一种神妙的乐趣,他又逐渐恢复成为他自己,不过是处于另外一种地位了。
\par 他只是在晚上才到她家去,不知道她白天干点什么,也不知道她过去是怎么回事;他连一点点情况都不了解,而这样一些情况时常会促使我们去想象我们所不知道的事情,推动我们去打听的。因此他从来也不问一问她在干些什么,她过去的经历又是怎样。有时他也想起,几年以前,当他还不认识她的时候,有人曾经跟他说起过一个女的(如果他记得不错的话,应该就是她),说她是一个妓女,是一个由别人供养的情妇,总之是这样一种女人,由于跟她们很少来往,他只能认为她们具有某些小说家的想象力久已赋予她们的那一套根本反常的性格。想到这里的时候,他也总是一笑了之。他心想,要正确评断一个人,只消一反众人对他的毁誉就可以了。奥黛特跟那样一种性格是风马牛不相及,她善良、纯真、热爱理想、几乎不会撒谎;譬如,有一天为了跟她一起去吃饭,他要她写信给维尔迪兰夫妇,说她有病,等到第二天维尔迪兰夫人问她好一点没有,他亲眼看见她面红耳赤,说话结结巴巴,脸上不由自主地反映出撒谎是何等难受和痛苦,而当她在答话中就头天的病编造一些细节时,她又仿佛以哀求的眼神和悲伤的声调,请求对方饶恕她言词的虚伪。
\par 难得有些日子,她在下午到他家来,打断他的遐想或对弗美尔的研究(这是他最近才恢复的)。仆人通报克雷西夫人在他的小客厅。他就上客厅去见她,等他把门打开,奥黛特一看见他,她那粉红色的脸上就挂上一丝微笑,嘴唇的曲线、两眼的神色、面颊的轮廓也都变了。当他一个人在家的时候,她的微笑就浮现在他眼前——前一天的那个微笑,某一次迎上前来时的那个微笑,那天在马车上问她是否同意为她摆弄卡特来兰花时作为回答的那个微笑;奥黛特在其他时间的生活,他一无所知,仿佛是出现在中性的,没有色彩的背景上的无数的微笑,就像华托的一些素描习作当中,从各种位置,各个方向,用三色铅笔在淡黄色的纸上绘出来的笑容。但是,在斯万以为是一片空白的奥黛特的那一部分生活方面(因为他想象不出,然而他心底里又不信那会是一片空白),有一天,有那么一位朋友(他早料到他们两人在相爱,在谈到她的时候只敢说些无关紧要的事),说他那天早上看见奥黛特走在阿巴蒂西街上,穿了一件饰有臭鼬皮的披肩,戴了一顶伦勃朗式的帽子,上衣上别着一束紫罗兰。这番描写使得斯万深为震惊,因为这就使他突然发现奥黛特除了跟他在一起以外别有一番生活;他要弄明白她穿了这套他从来没有见过的衣服倒是要取悦于谁;他下定决心要问她那时是到什么地方去的,仿佛在他的情妇的平淡无奇的生活中(简直是并不存在的生活,因为这是他所不能目睹的),除了对他的微笑以外,唯有这件事是最重要的——戴了一顶伦勃朗式的帽子,上衣上别着一束紫罗兰外出。
\par 除了请她弹奏凡德伊那乐句而不要弹《玫瑰圆舞曲》外,斯万并不试图让她演奏他自己所爱好的曲子,也不试图纠正她在音乐和文学方面的低劣趣味。他很明白,她并不是一个智力高超的人。当她说她是多么希望他跟她讲讲伟大的诗人们的时候,她心想这就可以知道许多像博雷利子爵\footnote{博雷利子爵是平庸的专写社交生活的诗人。}那一套浪漫的英雄诗体了,甚至还更加动人。至于弗美尔,她问斯万这位画家是否吃过哪个女人的苦头,是不是哪个女人启发他画的画,而当斯万说这些问题谁也不清楚的时候,她对这位画家也就不感兴趣了。她常说:“我相信,如果诗歌真实,诗人说的全是他们所想的话,那就再也没有比这更美的了。可是诗人时常是最斤斤计较的人,这方面嘛,我倒是知道一点。我有个朋友,她爱过一个那样的诗人。他在诗里谈的尽是什么爱情哪,天空哪,星星哪。好!她可大上其当!这位诗人花了她三十多万法郎。”如果斯万想教她什么叫作艺术美,教她诗歌或者绘画该怎么欣赏的话,那就要不了多一会儿她就不爱听了,直说:“啊……我原来可没有料到是这么回事。”他感觉得出她是多么失望,因此宁愿撒谎,说他刚才所说的都算不了什么,都是鸡毛蒜皮,说他没有时间深入谈下去,还有好些东西没说呢。可她赶紧就说:“什么?还有好些东西?……你倒说说看。”可是他不说,他明知道他要说的在她心目中是多么无关紧要,跟她所希望的相距又是多么遥远,决不会像她设想的那样耸人听闻,那样激动人心;他也怕她对艺术的幻想破灭了,对爱情的幻想也会同时破灭。
\par 确实,她觉得斯万在智力上并不像她原来设想的那么高明。“你总是那么含蓄,我简直是莫测高深。”斯万对金钱毫不在乎,对每个人都亲切,对人体贴,对这些,奥黛特越来越赞叹不已。一个比斯万伟大的人物,譬如说一个学者,一个艺术家,当他为周围的人赏识的时候,在他们的情感当中证明这个人的智力果然超群的时候,时常不是他们对他的思想如何赞赏——因为他们根本不能理解这些思想,而是对他的优良品质的尊重。使得奥黛特对斯万产生尊敬之情的也是他在上流社会中的地位,不过她也并不指望斯万把她引进上流社会中去。也许她感觉到,斯万并不能在上流社会中取得成功,她甚至担心,他只要一谈起她,他的朋友就会透露出她唯恐别人知道的关于她的一些情况。因此,她要他答应决不要提起她的名字。她说,她之所以不到上流社会的社交界去,是因为她曾经跟一个女的吵翻了,而这个女的为了报复,说过她的坏话。斯万反对这种说法,他说:“可并不是每一个人都认识你那位朋友啊。”“不,坏话传千里,人心又都那么坏。”斯万虽然不明白那是怎么回事,却也认为“人心都那么坏”和“坏话传千里”这两句话一般说来总是对的;这样的事例有的是。奥黛特那档子事是不是也是这样的一个事例呢?他心里存着这样一个问题,但是存不了多久,因为他自己的心情也挺沉重,就跟他父亲当年面临难题时一样。再说,上流社会的社交界使得奥黛特如此害怕,也许她就不会产生进入这个社交界的强烈愿望;这个社交界跟她所了解的相去是如此之远,她是不会对它有个清楚的认识的。奥黛特在某些方面依然还是很纯朴的,譬如她跟一个歇业的女裁缝还保持着友谊,差不多每天都爬那又陡又暗又脏的楼梯去看她,然而她还是拼命追求派头,不过她所谓的派头跟上流社会人士的概念并不一样。对后者来说,派头产生于很少数一些人,由他们推广及于一定范围,离他们这个中心越远就越削弱,只是扩及他们的朋友或他们的朋友的朋友这个圈子里。而这些人可说是登记在册的。这个名单上,上流社会中的人士能数得出来,他们对这样的事情无不知晓,从中提炼出一种口味,一种分寸,以至像斯万这样的人,只要从报上看到某次宴会有哪些人参加,用不着求助于他对社交界的那套知识,立刻就能说出这个宴会是怎样一种派头的宴会,这就跟一个文学家一样,只要听你念出一句句子,马上就能精确地评定出作者的文学价值。奥黛特属于缺乏这种概念的人之列(不管上流社会人士对他们是什么看法,这样的人多得出奇,社会各阶层里都有),他们心目中的派头根本不一样,按照他们所属的社会阶层而具有不同的样子,但都有这样一个特点——不管是奥黛特梦寐以求的也好,戈达尔夫人为之倾倒的也好——那就是人人都能直接学会。上流社会人士的派头,说实在的,也是人人都能学会的,但需要一定时间。当奥黛特说某人“只到够派头的地方去”的时候,斯万就会问她所谓的“够派头的地方”是什么意思,她就会带着小看他的意思答道:
\par “够派头的地方就是够派头的地方呗!像你这样的岁数,还问人什么叫够派头的地方,你叫我怎么说呢?譬如说吧,星期天早上的皇后大道,五点钟时的湖滨,星期四的伊甸剧院,星期五的跑马场,还有舞会……”
\par “什么舞会?”
\par “巴黎的舞会呗,我说的当然是够派头的舞会。对了,埃班谢,你是知道的,他在一个证券经纪人那里工作;你也一定知道,他是巴黎最知名的人物之一。这个金发的大高个小伙子,穿得真帅,纽孔上总戴一朵花,短外衣是浅颜色,背上有条缝;他带着他那个‘老来俏’,哪出戏的首场演出也落不了。嗯,他有天晚上就办了一个舞会,全巴黎所有够派头的人物都去了。我也真想去,可要进场就得出示请帖,我可没能弄着。不过,我幸亏没有去,去了也是挤死人,什么也瞧不见。最多也只能吹嘘吹嘘参加过埃班谢的舞会罢了。我这个人哪,你不是不知道,我可不是那种爱虚荣的人!再说,在一百个说参加过那个舞会的女人当中,至少有一半是撒谎。不过,像你这么帅得不能再帅的人怎么也没有去,我真纳闷!”
\par 斯万也不打算改变她对什么叫作派头的看法;心想他自己对派头的看法也未必就对,也同样荒唐,同样无关紧要,毫无必要来灌输给他的情妇,因此过了几个月,她对他交往的人是否感兴趣,全看他们能不能给他送来赛马的入场券、戏剧首场演出的门票了。她希望他保持一些能派用场的关系,可是自从她有回在街上看到维尔巴里西斯侯爵夫人穿着一件黑毛料的衣服、戴了一顶有帽带的软帽以后,就认为斯万交往的那些人未必就够派头。
\par “亲爱的,可她看起来像个剧场里的引座员,像个老看门的!这算什么侯爵夫人!我不是侯爵夫人,可你要叫我穿着这么一套破烂衣服上街,打死我也不干!”
\par 她也不明白斯万为什么住在奥尔良滨河路,她嘴里不说,心里可觉得这种地方跟他这么帅的人不般配。
\par 当然,她自称爱好“古董”,说起她喜欢花整天的工夫到寄售店去“收集小摆设”,去搜寻“古玩”的时候总是眉飞色舞。虽然她对白天干什么事讳莫如深,从来不回答这方面的问题,从来“不作任何汇报”,简直把这当做是荣誉攸关的事情,当做是一种家规,但是有一次还是对斯万说她曾应邀到一个朋友家里,她家里什么都是“古色古香”的。斯万问她是哪个时代的,她说不上来,想了半天才说是“中世纪”的,其实她的意思是说她家的墙上装了细木护壁板而已。不久以后,她又对他说起这位女友,还补充了一句说:“她家的餐厅是十八世纪风格的!”说话的语调有点吞吞吐吐,然而那神气又很肯定,仿佛是在谈起这么一个人:你头天晚上还跟他在一起吃饭,可从来没有听人说过他的名字,而宴会的东道主又认为他是这么知名,以为听话的对方是肯定知道他是何许人的。她觉得那餐厅太难看了,墙上光秃秃的,仿佛房子还没有盖完似的,妇女在那里也显得难看,这种摆设是决不会时兴的。后来,她第三次提起这个餐厅,还把设计这个餐厅的人的姓名和地址写了出来,说等到她有钱的时候,她也要叫他来看看能不能给她也这么搞一下,当然不是照那老样,而是她早就梦寐以求的那样一间餐厅,可惜她的住房太小,装不下带那么高的餐具架的文艺复兴式的家具,还有像布卢瓦宫堡里的那种壁炉。就是那一天,她在斯万面前说出了她对他在奥尔良滨河路的住宅的看法;因为他曾批评她的女友不搞路易十六时期的风格(尽管这种风格搞的人少,却挺美的),而是搞仿古式的。奥黛特是这么对他说的:“你总不能要求她跟你一样住在破烂的家具和磨光了的地毯中间吧!”在她身上,中产阶级的讲求体面毕竟还是占了轻佻女子的业余爱好的上风。
\par 她把那些爱收集小摆设,爱诗歌,鄙视斤斤计较,追求荣誉与爱情的人看成是高出于他人的杰出精英。其实也用不着当真有这些爱好,只要口头上这样说说就行;谁要是在饭桌上说他喜欢闲逛,喜欢上老铺子抚摩积尘盈寸的旧货,说他在这商业的时代永远也不会吃香,因为他向来不计私利,身上犹有古代遗风,那她回家就说:“这个人可值得敬仰,他感情是多么丰富,我原来真没想到!”而她对他的好感就油然而生。可是与此相反,像斯万这样的人,他们真有那些爱好,可嘴上不说,就要遭到她的冷淡。不错,她也不得不承认斯万不重金钱,然而她马上就撅起嘴来找补一句:“在他身上,这可是另外一回事。”敢情对她的想象力起作用的不是不计私利的实际行动,而是嘴上说说的空话。
\par 斯万自己也感到他时常不能使她梦寐以求的事情如愿以偿,他想尽办法使她至少乐于跟他在一起,竭力不去反对她那些庸俗的思想,不去反对她在种种场合表现出来的低劣趣味,反而像欣赏一切出之于她的东西一样欣赏这种趣味,甚至为之所迷,认为这个女人的本质正是通过这样一些特征表现出来,成为可见的事物。因此,当她要去看《黄玉王后》\footnote{法国作曲家维克多·马塞(1822—1884)的作品。}上演而面有喜色的时候,或者当她担心要看不上花展或者赶不上王家街茶座的有英国松饼和吐司的午茶时(她认为一个有风度的女人是应该每场必到的),斯万就会跟我们大家看到天真活泼的孩子或者呼之欲出的肖像时那样兴高采烈,感到他的情妇的心情在脸上表露无遗,禁不住上前去捧起来亲吻。“啊!小奥黛特要我领她去看花展,她要让大伙欣赏欣赏她的美貌,好极了!我不能不从命,我一定领她去。”斯万的眼睛有点近视,他在家里工作时不得不戴眼镜,出外参加社交活动时就戴单片眼镜,这样可以多保留一点本来面目。当她第一次看到他戴单片眼镜的时候,她不禁喜形于色:“男人戴了这个,真是没得说的,太帅了!你这么一戴,多漂亮!真是十足地道的绅士。就差一个称号了!”说的时候不免有点遗憾之情。他也喜欢奥黛特讲这样的话,就好比如果他被一个布列塔尼女子爱上的话,他也是乐于看见她戴上当地那种特殊的头饰,乐于听她说她信鬼的。斯万也跟许多人一样,他们对艺术的爱好的发展是与肉欲无关的,直到那时为止,在他对两者的满足之间一直存在着奇怪的不协调现象;他在越来越粗俗的女人陪伴下享受越来越精细的艺术作品的魅力,带上一个小女仆到包厢里看他想看的颓废戏剧的演出或者去看印象派画展,心里还深信如果带去的是一个有教养的女子,她也未必多懂一些,然而不会像小女仆那样老老实实地不妄加评论。不过自从他爱上奥黛特以后,跟她抱有同感,努力使两人一条心,这对他来说就成了一种甜蜜的事业,因此他竭力喜欢她所爱的东西,把不仅模仿她的习惯而且接受她的观点看成是一种乐趣,更因为她的这些习惯和观点并不是她聪明才智的产物,而仅仅起着使他想起她的爱情这么一种作用,所以他的这种乐趣也就更加强烈。他之所以再次去看《塞尔施·巴尼娜》\footnote{《塞尔施·巴尼娜》,是根据乔治·奥内同名小说所编的剧本。}的演出,找机会去听奥利维埃·梅特拉\footnote{奥利维埃·梅特拉(1830—1889)是奥黛特所喜爱的《玫瑰圆舞曲》的作者。}指挥乐队,都是出之于对接受奥黛特的一切观点的乐趣,出之于得以同意她的一切爱好的感觉。她所爱好的作品和地方具有使他跟她更接近的魅力,跟那些更美的但是和她联系不起来的作品和地方所固有的魅力比起来,在他眼里显得更加神秘。此外,年轻时搞学问的信念已经越来越淡漠,饱经沧桑的人的怀疑主义不知不觉地也渗入了这样的信念,他心想(由于经常这么想甚至还说),我们所爱好的对象本身并没有什么绝对的价值,一切都依时代、阶级而异,都是一时的风尚,最庸俗的风尚也不比被认为是最高贵的风尚价值小些。奥黛特对能否弄到美术展览会剪彩典礼的请帖那份重视,本身并不比他当年跟威尔士亲王同桌吃饭感到的乐趣更可笑;同样,他也并不觉得她对蒙特卡洛或里基山\footnote{蒙特卡洛是摩纳哥大公国的一个城市,以其赌场而知名。里基山在瑞士,海拔1800米,景色优美。}的赞赏就比他自己对荷兰(在她想象中是丑陋的)和对凡尔赛(她认为是凄凉的)的爱好来得没有道理。因此他就不到后两个地方去。心想这是为了同她抱有同感,只爱她所爱的地方。
\par 他喜欢奥黛特周围的一切,喜欢能够看到她、跟她谈话的一切场合,因此也喜欢维尔迪兰家的那个社交团体。跟他们在一起的一切游乐活动——聚餐、音乐、游戏、化装宵夜、郊游、戏剧,甚至是难得为那些“讨厌家伙”举办的“盛大晚会”当中,总有奥黛特在场,总能看到奥黛特,总能跟奥黛特谈话,而维尔迪兰夫妇在邀请斯万参加时又把这些看成是对他的无法估量的恩典,这就使得斯万在这“小核心”里比在任何地方都更感到惬意,竭力为核心里的人摆出一些好处,心想他这辈子都会有兴趣参加这个社交圈子的活动的。然而他从来不敢想象(怕常想就会对他的预料产生怀疑)他会永远爱奥黛特,不过,假如他一直同维尔迪兰家交往(这种设想,从原则上来说,跟他的理智的抵触要少些),那么他在将来总是可以继续每晚都看到奥黛特的;这也许并不等于永远爱她,但就目前来说,当他还爱她的时候,他所求的也就是不至于有朝一日看不到她罢了。他心想:
\par “多可爱的环境啊!这里的生活才是真正的生活!这里的人比上流社会中的人更聪明,更爱艺术!维尔迪兰夫人虽然有些夸大其词,未免可笑,却又是对绘画和音乐怀有何等真诚的爱好,对美术和音乐作品是何等热爱,又是何等乐于取悦于艺术家啊!她对上流社会人士的观感固然不很对头,然而上流社会人士对艺术界的看法又何尝正确?可能我不太想在跟他们的谈话当中增长多少才智,虽说戈达尔总爱来一些愚蠢的文字游戏,我却非常乐于跟他交谈。至于那位画家,当他想一鸣惊人的时候表现出的那种矫揉造作劲儿固然有点讨厌,却是我所认识的最有头脑的人之一。再说,在这里人人都感到自由自在,可以无拘无束,用不着装模作样而做他不愿做的事情。在这客厅里,人们的心情每天都是何等愉快啊!除了少数例外情况,我一定不到别的什么地方去了。我将在这里慢慢培养我的习惯,度过我的一生。”
\par 他以为维尔迪兰夫妇固有的品质其实只是他出于对奥黛特的爱而在他们家中体会到的种种乐趣在他们身上的反映,所以当这种种乐趣越来越增长时,那些品质也就变得越来越当真、越深刻、越重要了。由于维尔迪兰夫人不时为斯万提供唯一能为他带来幸福的机会,某天晚上奥黛特跟某一位客人聊的时间多了一些,而斯万感到心焦,一气之下就不主动问她是否同他一起回去的时候,维尔迪兰夫人总是对奥黛特说:“奥黛特,您不送斯万先生回去吗?”从而使他心里平静下来,感到快活;由于那年夏季行将到来,斯万心里直打鼓,不知奥黛特是否会撇开他单独出去度假,不知他是否还能每天都跟她见面,而正是维尔迪兰夫人邀请他们两人都上她乡间的别墅度假的;于是这些都在不知不觉间让他的感激之情和利害观念渗入他的理智之中,影响他的思想,居然宣称维尔迪兰夫人有一颗“伟大的心灵”。要是他在卢浮宫美术学校的老同学谈起某些杰出的艺术家的话,他会答道:“我百倍地喜欢维尔迪兰夫妇。”而且他还会用以前从来没有过的庄严口吻说:“他们是高尚的人,而高尚这种品德是这世上唯一最重要的东西,是区别人的唯一标准。你看,这世上一共只有两种人:高尚的和不高尚的。我已经到了这样一个年龄,应该下定决心,一劳永逸地决定应该敬爱哪些人,应该蔑视哪些人,下定决心永远站在受人敬爱的人们那一边,同时为了弥补跟另一种人在一起浪掷了的时间,至死也不离开受人敬爱的人们。”我们有时说一件事情,并不因为这件事情是真的,而只是因为说了痛快,而且当我们自己说的时候,还仿佛觉得这话是出之于他人之口。这种情况,我们自己也并不时常意识到。斯万这时正是以我们在这种情况下的心情接着往下说:“好吧!事情就这么定了,我这就决定只爱心灵高尚的人们,从此只在高尚的环境中生活。你问我维尔迪兰夫人当真聪明不聪明?我可以向你保证,她的行为表明她心灵高尚,而要是思想不高超的话,心灵是不会达到这样的高度的。诚然,她对艺术的理解是深刻的,然而她最可爱的地方并不在这里,她那么巧妙、那么高明地为我尽力,她对我的关怀,她为我所作的既崇高又亲切的一举一动,显示出任何哲学教科书所不及的对人生的深刻理解。”
\par 他也许还能承认,在他父母的老朋友当中也有同维尔迪兰夫人一样纯朴的人,在他年轻时的同伴当中也有像他们那样热爱艺术的人,在他的熟人当中也有心灵高尚的人,然而自从他崇尚纯朴、艺术和心灵高尚以来,他却从没有再看到他们。而这些人不认识奥黛特,同时即使他们认识她,也不会费尽心机来促成他跟她的接触。
\par 这么一来,在维尔迪兰夫妇这个圈子里,像斯万这样爱他们,或者自以为爱他们的忠实信徒恐怕再也数不出来了。然而当维尔迪兰先生说斯万并不合他胃口的时候,他不仅说出了他自己的想法,也猜到了他妻子的心思。很显然,斯万对奥黛特的感情太特殊,他是不会向维尔迪兰夫人透露他俩之间的秘密的;也很显然,他又是以如此的谨慎来对待维尔迪兰夫妇的好客,时常以他们意想不到的理由就不上他家吃饭,他们只能认为他是不想回绝哪个“讨厌家伙”的邀请;也很显然,尽管他十分小心谨慎地提防,他们还是慢慢地发现他在上流社会里有显赫的地位;所有这一切都促使他们对他恼火。然而最深刻的原因还不在这里,而是因为他们很快就感觉到在他灵魂深处还保留着一个别人无法进入的王国,依然还默默地认为萨冈亲王夫人并不可笑,认为戈达尔的玩笑并不逗人,总而言之,虽然他对他们一贯殷勤亲切,从来不公开反抗他们的信条,但他们却不能使他衷心接受,不能使他彻底归化,这在别人身上还是从来没有见过的。他们原本可以原谅他跟一些“讨厌家伙”来往的(在他心底里,他却也是千百倍地更喜欢维尔迪兰夫妇和他们的“小核心”的),只要他做出个好榜样来,当着那些信徒的面背弃那些家伙就行了。然而他们也明白,要他发誓跟那些人断绝来往,那是不可能的。
\par 奥黛特请求他们邀请的那个“新人”,虽然她自己也只见过几次面,他们却对他寄以很大的希望,这跟对斯万是何等的不同!这位“新人”就是福什维尔伯爵。原来他正是萨尼埃特的连襟,这使那些信徒不胜诧异:这位老文献家态度那么谦卑,他们原以为他的社会地位要比他们低微,不料却出自一个富有而且几乎是贵族之家。当然,福什维尔浑身散发出冒充风雅的气味而斯万则不是;当然,他决不能像斯万那样,把维尔迪兰家这个圈子看得比任何别的地方都高出一筹。然而他缺乏斯万那种心计,不像他那样,对以维尔迪兰夫人为首的那些人指责他所认识的人们的明显错误时避免随声附和。至于画家有时发表的自命不凡的夸夸其谈,戈达尔所开的庸俗的玩笑,斯万虽然跟他们两个都要好,可以原谅他们,然而鼓不起勇气,也没有那份虚情假意来为他们叫好,而福什维尔却是那样愚钝,虽然并不懂得画家谈的是什么,竟为之倾倒,对戈达尔的玩笑也听得津津有味。正是在福什维尔在维尔迪兰家吃第一顿饭的桌上,两个人之间的差异全都暴露了出来,突出了福什维尔的品质,也加速了斯万的失宠。
\par 那天晚上,餐桌上除了常客之外,还有一位巴黎大学的教授,名叫布里肖,他是在温泉跟维尔迪兰夫妇认识的。要不是校内教务繁忙,研究工作又重,闲暇时间很少的话,他是很乐意常上他们家来的。他对人生有这样一种好奇之心(也可以说是迷信),这种好奇心跟人们对他们的研究对象的一定程度的怀疑态度相结合,就会在任何一行一业中,使得某些聪明人(譬如不信医学的医生,不信拉丁文翻译练习的中学教员)博得思想开阔、头脑敏锐、甚至高人一等的美名。他装模作样地在维尔迪兰夫人家中搜求他在讲哲学,讲历史时可资对照的当今实例,首先他认为哲学和历史都无非是为人生之途作准备,其次他也认为在这小宗派里可以看到以前仅仅在书本里看到的东西,现在在行动中表现出来;最后可能也是因为他从小就被灌输了对某些人的尊敬之情,而且在不知不觉之中把这种尊敬之情一直保持在心头,现在他却想剥去他自己大学教授的外衣,跟这些人一起放肆放肆——其实这些言行之所以显得是放肆,也仅仅因为他道貌岸然地穿着大学教授的外衣的缘故。
\par 刚一开饭,坐在维尔迪兰夫人(她可为了这位“新人”的光临而在衣装打扮上没有少下工夫)右首的德·福什维尔先生就对她说:“您这件白外衣(robe blanche)可真是独出心裁。”那位大夫一直好奇地打量着这位被他称之为“姓氏中带‘德’字的人”,目不转睛地盯着他,总想找机会引起他的注意,跟他拉上关系,这时抓住了blanche这个字,头也不抬地说:“Blanche?Blanche de Castille?(布朗施?布朗施·德·卡斯蒂利亚?)\footnote{布朗施·德·卡斯蒂利亚(1185—1252),法国国王路易八世之妻,路易九世(即圣路易)之母,曾两度为摄政王后。}”,然后继续低着头左顾右盼,既拿不稳大伙对他这句话会有什么反应,又流露出洋洋自得的神气。斯万苦笑了一下,表明他认为这种用同音异义字进行的文字游戏实在荒唐,而福什维尔则恰如其分地流露出一种欢快情绪(那种真诚坦率着实叫维尔迪兰夫人看了高兴),表明他既欣赏大夫所说的那句话的精巧,自己又精于为人处世之道。
\par “您觉得这位科学家怎么样?”她问福什维尔,“跟他在一起,你就没法子接连谈上两分钟的正经话。”她又转过脸来对大夫说:“您在医院里是不是也这么老开玩笑?这么着,倒是不至于整天闷得慌。我看我也该申请住进您的医院才是。”
\par “我想我刚才听见大夫说起了那个老泼妇布朗施·德·卡斯蒂利亚——请原谅我这么说话。夫人,我说得对不对?”布里肖问维尔迪兰夫人。维尔迪兰夫人喜不自禁,两眼紧闭,双手捂住脸,格格地闷声直笑。“天哪!夫人,我不想故作惊人之举,来吓唬现在在座而鄙人有所不知的虔敬的贵宾们……不过,我得承认咱们这个难以用言语形容的雅典式共和国——啊,那是十足地道的雅典式共和国,它的第一个警察头子正是这位采取愚民政策的卡佩家族的女人。就是这么回事,我亲爱的主人,就是这么回事,没有错。”他以铿锵有力的声音,一个音节一个音节地吐出他对维尔迪兰先生提出的反对意见的回答。“《圣德尼编年史》\footnote{《圣德尼编年史》即《法兰西编年史》,13世纪编于圣德尼市。}这部作品所提供的资料的可靠性是毫无问题的,它在这一点上就留下了不容置疑的证据。这位圣者的母亲哪,不信教的无产者再也挑不出比她更好的保护人了;她不但生了一个被称为圣者的儿子,还培养了一批蹩脚的圣者(絮谢尔\footnote{絮谢尔(约1081—1151),圣德尼市的教士,路易六世及路易七世时的大臣,在法国王权的加强方面起过极为重要的作用。}就是这样说的),以及一些圣伯尔纳\footnote{圣伯尔纳(1090—1153),中世纪神学家,在法国政教冲突中帮助巴黎主教反对路易六世及路易七世。鼓吹神秘主义,极力反对阿伯拉尔“理解而后信仰”的主张。}之流;谁沾上她的边都难免挨骂。”
\par “这位先生是谁?”福什维尔问维尔迪兰夫人,“他说起话来气儿还挺粗的。”


\paragraph*{2}

\par “怎么?您不认识这位大名鼎鼎的布里肖?他在全欧洲都是遐迩闻名的。”
\par “噢!他就是布里肖!”福什维尔高声叫道,他刚才并没有听真。接着又双眼圆睁瞧着那位客人对维尔迪兰夫人说,“您待会儿跟我详细介绍介绍。能跟一位名人同桌吃饭,总是很有意思的。您邀请的客人都经过精心挑选,在您这里是决不会厌烦的。”
\par “是的,尤其是他们都有一种安全感,”维尔迪兰夫人谦虚地说,“他们想谈什么就谈什么,大家畅所欲言,从来不会冷场。布里肖今天谈的还不怎么样;有一天在这里可是说得有声有色,叫你简直要拜倒在他脚下。要是在别人家里,他可就变了样了,机智也没有了,话就跟牙膏一样,你不挤就出不来,他甚至会变成一个讨厌家伙。”
\par “这倒真怪!”福什维尔不胜诧异地说。
\par 布里肖那样的机智,尽管跟真正的才智并不矛盾,可在斯万年轻时交往的那些人眼里会被看成是纯粹的愚蠢。而教授才华横溢,很多被斯万认为是有才的上流社会人士是会羡慕的。然而这些人士早已把他们的好恶,至少是与社交生活,甚至是与社交生活相连而其实应该属于才智领域的东西(例如谈吐)有关的好恶都灌输给了斯万,因此他只能认为布里肖开的玩笑既是学究气十足,又庸俗粗鲁得令人作呕。再说,他习惯于彬彬有礼,对那位狂热的民族主义的教授对任何人说话时的那种粗鲁甚至是大兵式的口吻也大为反感。最后,也许他那天晚上看到维尔迪兰夫人对奥黛特一时心血来潮带来的这位福什维尔表现得那么殷勤亲切,因此失去了平常那种宽容。奥黛特在斯万面前也显得有点不自在,来到的时候曾问他:“您觉得我那位客人怎么样?”
\par 福什维尔是他早就认识了的,可这是他第一次发现他居然能得到一个女人的好感,而且长得还相当漂亮,就没有好气地答道:“真恶心!”他倒不是为了奥黛特的缘故而心怀妒意,不过那天他不像往常那样高兴,所以布里肖讲起布朗施·德·卡斯蒂利亚的母亲,说她“跟金雀花朝的亨利生活在一起多年才嫁给他”这个故事时,他想让斯万敦促他接着讲下去,就对他说:“斯万先生,是不是?”那口吻倒像是在对乡巴佬讲话,或者是给大兵打气似的,斯万就说,他很对不起,他对布朗施·德·卡斯蒂利亚毫不感兴趣,倒是有话要跟画家说。这就杀了布里肖的威风,使得女主人大吃一惊。原来画家那天下午去看了一位艺术家的画展,那是维尔迪兰夫人的朋友,前不久死了的。斯万想通过画家(他的鉴赏力斯万是很欣赏的)了解一下那位艺术家,他在前几次展览中震惊了观众的精湛技巧,在最后几幅作品中是否更进了一步。
\par “从这一观点看来,真是了不起,然而我并不觉得这种艺术形式很‘高级’。”斯万面带微笑说。
\par “高级……高到九天之上。”戈达尔煞有介事似的举起双臂插上这么一句。
\par 举座纵声大笑。
\par “您看,我说得对不对,跟他在一起就没法子说正经的,”维尔迪兰夫人对福什维尔说,“在谁也预料不到的时刻,他冷不丁给你来上一句笑话。”
\par 然而她也注意到,只有斯万没有开颜。相反,他对戈达尔当着福什维尔的面笑他,感到很不满意。而画家吗,如果只有他跟斯万在场的话,是会帮他说句话的,现在却宁可就已故的大师的技巧说上两句,以此来博得席上的人的赞赏。
\par “我一直走到画幅跟前,”他说,“想看看到底是怎么画的;我都把鼻子尖顶上去了。嗨!谁也说不上那是用什么画的,是胶?是宝石?是胰子?是青铜?是阳光?还是屎巴巴?”
\par “再添一得十二!”大夫待了会儿叫道,谁也不明白他插这么一句话是什么意思。
\par “看样子是什么也没有用,”画家接着说,“这儿的谜跟《夜巡》和《摄政王后》那两幅画同样难解,那手法比伦勃朗\footnote{伦勃朗(1606—1669),荷兰画家,将意大利画家卡拉瓦齐的明暗对比法加以发展,形成独特的风格。《夜巡》为其杰作之一。}和哈尔斯\footnote{哈尔斯(约1580—1666),荷兰肖像画家和风俗画家,笔法流畅,有节奏感,色彩简朴而明亮,对后来欧洲绘画技法的改进有较大启发。《摄政王后》即出其手。}还要高明。这幅画真是了不起!”
\par 正如歌唱家已经唱到他所能唱到的最高音而只好改用假嗓子哼下去一样,他这会儿也只好含笑低语,仿佛那幅画美得反而有点可笑似的:
\par “味儿好闻,上脑,叫你透不过气来,叫你全身痒痒,可你又说不上那是用什么画的,这简直是巫术,是骗术,是奇迹(说到这里他放声大笑),是不老实!”他打住话头,庄严地抬起头来,以尽量显得悦耳的深沉的低音找补一句,“可又是如此正派!”
\par 除了当他说到“比《夜巡》还强”时引起维尔迪兰夫人的反对(她把《夜巡》跟《第九交响曲》和《萨摩色拉斯的胜利女神雕像》,看成是世上最伟大的三件杰作),提到巴巴这两个字时引起福什维尔环顾全桌,看他们对这话的反应,并且含蓄地、宽宏大量地微微一笑以外,其余的时间,席上的人除了斯万以外,全都着了魔似的盯着那位画家。
\par 等他说完话,维尔迪兰夫人眼看德·福什维尔先生第一次光临在餐桌上就如此兴致勃勃,高兴极了,她高声叫道:“你们看,他说得那么来劲,我真高兴。”又对她丈夫说:“你这是怎么啦?目瞪口呆地待在那里!你是听呆了。画家先生,他倒像是第一次听您说话似的。刚才您讲话的时候,他是一个字一个字都记在心间,赶明儿要他复述您的话,他准一个字儿也落不了。”
\par “不,我这并不是扯淡,”画家说,他对他的成功十分得意,“看样子,你们以为我这是吹牛,是骗局;那我就领你们去看看那画展,到时候你们再看我是不是夸大其词;我敢担保,你们看了比我还要兴高采烈!”
\par “可我们并不认为您是夸大其词,我们只是要您别忘了吃菜,要我丈夫也别忘了吃菜。再给比施先生来点诺曼第板鱼,他盘子里的已经凉了。我们不忙,别那么急着上菜。色拉待会儿再上吧。”
\par 戈达尔夫人向来谨慎,沉默寡言,可是当她灵感一来,想起一句得体的话,她也不乏自信。她感到这句话会一鸣惊人,这就使她产生了信心,而她这么做并不是为了自己出风头,更多地是为了有助于她丈夫的事业。维尔迪兰夫人刚提起“色拉”这两个字,她就赶紧抓住机会:
\par “莫非这是日本色拉?”她转过脸来,朝着奥黛特低声说道。
\par 这话虽然说得含蓄,却显然是跟最新一上演就轰动一时的小仲马的那个剧本有关,她为说这既得体又大胆的话感到高兴,却也有点不好意思,像个天真无邪的小姑娘似的笑了起来,笑声是那么轻,然而难以遏制,过了好一会儿才止住。
\par “这位夫人是谁?她可很有机智。”福什维尔说。
\par “不,不过各位如果星期五一起光临,我们给各位准备日本色拉。”
\par 戈达尔夫人对斯万说:“先生,说起来也许您会觉得我太土。我到现在还没看过那脍炙人口的《弗朗西伊翁》\footnote{《弗朗西伊翁》,小仲马于1887年发表的剧本。}呢。大夫已经看过了,我记得他对我说过,他是有幸跟您一起看的,我也觉得他不必为了陪我而去订票再看一次。当然,在法兰西剧院的晚上是从来不会虚度的,演出总是非常精彩,不过我们有很好的朋友(戈达尔夫人很少举出具体的姓名,只说“我们的朋友们”或者“我们的一位朋友”,拿腔做调,学着那不屑提那些不足道的人的姓名的那副架子,那种派头),他们有包厢,常想着带我们去看值得一看的新戏;我相信我迟早总会有机会去看《弗朗西伊翁》的,到时候就可以提出我自己的看法了。不过我可得坦白承认,我是够傻的,在我所到的沙龙里,大家都在谈论那个倒霉的日本色拉。”看到斯万对她那件新闻并不如她所期望的那样感兴趣,她又加上一句:“大伙甚至已经开始有点谈腻了。可也得承认这有时也会引出一些挺有意思的想法。譬如说吧,我有一个女友,很漂亮,很吸引人,很出名,可也很怪,她说她就叫她家的厨子做过那种日本色拉;小仲马在剧本里说要搁什么,她就叫搁什么。她邀请了几位朋友去品尝。我可没有被邀请的福气。不过有一天她跟我们大伙都说了,看来那种色拉难吃得要命,把我们乐得眼泪都笑出来了。当然,关键在于你讲的可乐不可乐。”看到斯万毫无笑容,她最后讲了这么一句。
\par 她心想也许是因为斯万不喜欢《弗朗西伊翁》的缘故,便又说道:“我想我也许会失望的。我不信它会比得上德·克雷西夫人崇拜得五体投地的《塞尔施·巴尼娜》。不过总还有些地方可以发人深思;可是在法兰西剧院的舞台上讲什么色拉的做法,那可未免太……而《塞尔施·巴尼娜》呢,就跟一切出之于乔治·奥内之手的作品一样,总是写得那么好。我不知道您看过《铁厂老板》没有,跟《塞尔施·巴尼娜》相比,我还更喜欢这一部呢。”
\par “对不起,”斯万语带讽刺地说,“我要坦白承认,我对这两部杰作,都同样不欣赏。”
\par “那您认为这两部作品有哪些毛病呢?您的意见就不会改变了吗?您是不是觉得惨了点儿?是吗,我总说,小说和剧本是没法讨论的。各有各的看法。我最喜欢的,您可能觉得讨厌。”
\par 福什维尔这会儿叫斯万,这就把戈达尔夫人的话给打断了。刚才当她大谈特谈《弗朗西伊翁》的时候,福什维尔在维尔迪兰夫人面前对画家的演讲大为赞赏。
\par 画家话刚讲完,他就对维尔迪兰夫人说:“这位先生口才真好,记忆力真强!真是少见。哎呀,我要是能这样就好了!他可以当个优秀的传教士。他跟布里肖先生真可说是旗鼓相当;我简直说不上这一位是否比教授更能说会道些。他出口成章,不那么咬文嚼字。虽然他有几个字眼说得未免太俗,可这也是时下的风尚。说起话来这么滔滔不绝的人可并不常见,这位先生倒叫我想起当年在团里一起服兵役的一个伙伴。随便谈起什么东西,譬如说这只杯子吧,他都可以给你说上几个钟头;不,不,不,干吗要谈杯子呢,我怎么这么傻!那就说滑铁卢战役吧,或者随便什么题目吧,他都会跟你提起一些你连想都想不到的事情。对了,斯万也跟我在一个团里,他应该认识他。”
\par “您跟斯万先生常见面?”维尔迪兰夫人问道。
\par “不。”德·福什维尔先生说。他为了更容易接近奥黛特,便想得到斯万的好感,所以要抓住这个机会讨他的好,提提他那些显赫的朋友,不过要以上流社会人士的身份来谈,带上善意的议论的口吻,不能显得像是庆贺他有这样意想不到的成功似的,“斯万,我跟您从不来往,是不是?再说,谁能有办法见着他?这家伙成天跟拉特雷默伊耶家,跟洛姆亲王夫妇这些贵人厮混在一起……”这指责可真是太离奇了,这一年来斯万几乎除了维尔迪兰家以外哪家也不去,可是他们一听这些他们所不认识的人的名字就气得默不作声。维尔迪兰先生怕这些“讨厌家伙”的名字,尤其是当着他那些忠实信徒的面毫无顾忌地吐了出来,肯定会在他妻子身上产生不良印象,于是赶紧悄悄地向她投过充满关怀和不安的一瞥,但只见她脸上露出一副不屑理睬的神气,对听到的新闻毫不为之所动,不仅作哑而且装聋。当我们听到哪个做了错事的朋友在谈话间吐出几句辩解的话时,我们不也是宁可假装没有听见,也不愿显得是听到了而不反驳,显得是认可了吗?当别人在我们面前提到一个我们忌讳听到的忘恩负义之徒的名字时,我们不也宁可假装没有听见吗?维尔迪兰夫人为了让她的沉默不至显得是表示同意,而只是像无生命的物体那种无意识的沉默,霎时间脸上看不出半点生气,甚至可说是纹丝不动;她那鼓脑门就像是一件圆雕作品,跟斯万厮混在一起的拉特雷默伊耶之流的名字是钻不进去的;她那微皱的鼻子露出两个鼻孔,也好像是用什么东西塑出来的一样。她那微张的嘴巴像是有话要说。全身上下看来就只是一团蜡、一个石膏面具、一个建筑用的模型、一个工业展览馆里展出的胸像——在这胸像面前,观众肯定要驻步观赏雕塑家是怎样把维尔迪兰家人压倒拉特雷默伊耶家人和洛姆亲王家人以及世上所有的“讨厌家伙”的威严表现出来,从而为这尊坚硬的白石像注入了几乎能与教皇相媲美的尊严。不过,大理石终于活了过来,说是只有不爱挑挑拣拣的人才能上那些人家去,因为那边的女人总是喝得醉醺醺的,男人无知得把corridor念成collidor。
\par “任你给我多少钱,我也不让这样的人上我家来。”维尔迪兰夫人最后说,狠狠地盯着斯万。
\par 钢琴家的姑妈高声叫道:“你们看!我真不明白,这样的人居然还能找到人来跟他们聊天!要是我的话,我准会吓得要死,准要倒大霉!怎么还能有人野成这个样子,跟在他们屁股后面转?”维尔迪兰夫人当然不敢希望斯万会那么顺从,来学这位没头脑的太太。可他至少可以像福什维尔这样来回答吧:“天哪!她可是位公爵夫人呢!有些人还是看重这些玩意儿的。”若真如此,维尔迪兰夫人至少可以这样回对:“就让他们大沾其光吧!”然而斯万却不这样,他只是微微一笑,那神气仿佛是说,他根本没法子把这么点玩笑认真看待。维尔迪兰先生还是时不时悄悄地看他的妻子,黯然看着,也完全理解她这时感到一个宗教裁判所的法官未能消除异端邪说时的那种愤怒,而为了试着让斯万收回前言(因为一个人坚持自己意见的勇气在对方看来总是出之于对利害的计较,总是怯懦的表现),他就招呼斯万:
\par “您就把您对他们的看法坦率地说出来吧,我们是不会告诉他们的。”
\par “我压根儿就不是怕公爵夫人(如果你们说的是拉特雷默伊耶家的话)。我敢说,谁都喜欢上她家去。我并不是说她这人很‘深刻’(他把‘深刻’二字读得仿佛是一个滑稽可笑的字眼似的,因为他的言谈中还保留着往日说俏皮话这种习惯的痕迹,不过由于最近生活中出现了新气象,对音乐热爱起来,这种习惯一时有所消失,所以发表意见时也不乏热情了),不过,说真心话,她是个聪明人,而她的丈夫是个正直的文人。他们俩都很可爱。”
\par 维尔迪兰夫人心想单凭这么一个不忠实的信徒,她就无法保持小核心内部思想的统一;她对这个居然看不出他的话使她如何痛苦的顽固分子满腔怒火,忍不住从心底里发出吼声:
\par “您要是这么看待他们,那是您的事。可至少别在我们面前说出来。”
\par “这全看您所说的聪明是怎么回事,”福什维尔说,他也想一露锋芒,“斯万,您所理解的聪明才智倒是怎么回事?”
\par “对了!”奥黛特叫了起来,“这些大问题,我请他给我讲一讲。他就是不肯。”
\par “哪来的事!”斯万否认。
\par “就是这么回事!”奥黛特说。
\par “您是不是认为聪明才智就是能说会道,就是钻进上流社会的本领?”福什维尔说。
\par “快把您的甜食吃完,好撤掉您的碟子,”维尔迪兰夫人话中带刺地对萨尼埃特说,他这会儿正陷入沉思,停下了刀叉。维尔迪兰夫人也许是对刚才她自己那口吻有点不好意思,又找补一句:“没关系,您尽管慢用。我这话是对别人说的,为了好上下一道菜。”
\par “那位可爱的无政府主义者费纳龙\footnote{费纳龙(1651—1715):法国散文作家,其小说《忒勒马科斯历险记》反映作者谴责暴君穷兵黩武,为害人民的情绪。}给聪明才智下过一个很怪的定义呢。”布里肖一板一眼地说。
\par “听着,”维尔迪兰夫人对福什维尔和大夫说,“他要把费纳龙对聪明才智下的定义告诉咱们了,这真有意思,这样的机会真是难得。”
\par 然而布里肖却要等斯万先生讲出他自己对聪明才智所下的定义。斯万不吭声,维尔迪兰夫人原想让福什维尔欣赏的唇枪舌剑也就此告吹了。
\par “你们看,这跟对我一样,”奥黛特赌着气说,“我倒挺高兴的,总算他认为不够格跟他讨论的还不止我一个。”
\par “塞维尼夫人这个冒充风雅的婆娘说过,她为能结识拉特雷默伊耶家人而感到庆幸,因为这对她的农民有好处。维尔迪兰夫人刚才说得那么不足称道的拉特雷默伊耶家族莫非就是他们的后裔?”布里肖一句一顿地问道,“不错,侯爵夫人还有另一个理由,在她看来,比刚才所说那个理由还要重要,那就是因为她骨子里是个文抄公,把抄放在首位。拉特雷默伊耶夫人交游广泛,消息灵通,塞维尼夫人经常寄给她女儿的日记当中有关外交事务方面的消息,都是得之于拉特雷默伊耶夫人的。”“不,我就不信他们是一家人。”维尔迪兰夫人冒说一句。
\par 萨尼埃特自从急急忙忙把还装满了菜的碟子交给侍役长以后,一直一言不发,陷入沉思,现在忽然哈哈大笑,讲了一段故事,说是他曾经跟拉特雷默伊耶公爵一起吃过一顿饭,发现这位公爵居然不知道乔治·桑是个妇女的笔名。斯万对萨尼埃特是有好感的,认为应该就公爵的文化修养问题向他提供一些情况,说明公爵会无知到如此地步,这根本是不可能的事;然而他说到半截就打住了,他明白萨尼埃特并不需要这些证明,他自己也明知道那故事并不真实,是他刚刚编造出来的。这位老好人一直苦于被维尔迪兰夫妇看成是个沉闷乏味的人;那天晚上意识到自己比平常还要无聊,所以不愿终晚不能博人一笑。他很快就投降了,为没有达到预期的效果而神色沮丧,最后恳求斯万别再继续进行已经毫无必要的驳斥:“好了,好了;再怎么说,即使是我错了,总也不算是什么罪过吧。”那口吻是如此软弱可怜,斯万都恨不得说他讲的那故事既真实又有趣。大夫一直听着他们两人说话,心想这正是说Se non e vero\footnote{Se non e vero,e bene trovato,意大利成语,意为即使这不是真的,至少是挺巧的。}的机会,但对这成语的意义不太拿得稳,又怕用错了出乖露丑。
\par 吃完晚饭,福什维尔主动走到大夫跟前:
\par “维尔迪兰夫人倒也还长得不错,再说,跟这个女人还可以谈得来,对我来说,这就够了。当然,她已经开始有点儿上年纪了。可德·克雷西夫人呢,这小女子可长得挺机灵的;哈,你一眼就能看出她跟美国人一样精明。我们正在谈德·克雷西夫人呢,”最后这句话是对维尔迪兰先生而发的,这时他正叼着烟斗过来,“我想,就女人的身段而言……”
\par “我倒真想跟她床上见呢。”戈达尔赶紧插上一句。他早就在等待福什维尔喘一口气,好让他乘机插进这一句由来已久的笑话,唯恐谈话一转题,错过了好机会,而他说这句话的时候故意拿腔拿调,来掩盖通常背人家的句子时感情的缺乏和情绪的激动。福什维尔是知道这句笑话的,听了立即就明白戈达尔的意思,感到很可乐。维尔迪兰先生也乐不可支,他不久前发现了表达他的欢快的一种方式,跟他妻子的有所不同,可同样既简单又明了。他跟一般放声大笑的人一样先仰面耸肩,马上又来一阵咳嗽,仿佛是因为笑得太厉害,给烟斗里的烟呛了一样。他继续把烟斗叼在嘴角,让那假装的窒息和狂笑无限期地保持下去。就这样,他和维尔迪兰夫人(她这时正在对面听画家讲一个故事,先把双眼闭上,再用双手捂脸)就像是舞台上的两个假面具,以不同方式来表示高兴。
\par 维尔迪兰先生没有把烟斗从嘴里拿出来,这可做对了,因为戈达尔这时要出去方便方便,低声说了他不久前才学到,可每次上同一地方都必说的那句笑话:“我得去找奥马尔公爵\footnote{奥马尔公爵(1822—1897):法国国王路易·菲利浦的四子,将军兼史学家,在阿尔及利亚殖民战争中建有功勋,以“去找奥马尔公爵聊一会”表示“出去方便方便”,来历不详。}聊一会。”这就把维尔迪兰先生的阵咳又引发了出来。
\par “你就把烟斗拿下来吧,你这么忍住不笑,会把你憋死的。”维尔迪兰夫人对他说,她这会儿正来给大伙斟酒。
\par “您的丈夫真是讨人喜欢,他机智超群,”福什维尔对戈达尔夫人说,“谢谢夫人。像我这样当过兵的,是不会拒绝喝一杯的。”
\par “德·福什维尔先生认为奥黛特很可爱呢。”维尔迪兰先生对他的妻子说。
\par “她正想哪天跟您同吃一顿午饭呢。我们来安排,可别让斯万知道了。他会泼冷水的。当然,您尽管来吃晚饭,我们希望能经常看到您。美好的季节就要来到了,我们就可以常在户外吃饭了。您该不至于讨厌到布洛尼林园去吃饭吧?好,好,那好极了!”她又向年轻的钢琴家嚷道:“您今晚不干点儿活吗?”这是为了在像福什维尔这样一位要人面前,既显示她的聪明才智,又显示她对信徒呼来喝去的威风。
\par “德·福什维尔先生刚才说你的坏话呢。”戈达尔夫人当她丈夫回到客厅时对他说。
\par 他可从晚饭开始到现在,脑子里始终在想着福什维尔高贵的出身,这时对她说:“我现在正在给一位男爵夫人治病,她叫普特布斯男爵夫人;普特布斯家人参加过十字军东征,是不是?他们在波美拉尼地区有个湖,比协和广场还大十倍。男爵夫人闹的是关节炎。她可是个可爱的女人。我想,她也是认识维尔迪兰夫人的。”
\par 过了一会儿,当福什维尔单独跟戈达尔夫人在一起的时候,他又继续发表对她丈夫的评价:
\par “他这个人真有意思,看得出来,他交游甚广。好家伙,大夫知道的事情真多!”
\par “我这就给斯万先生弹那首奏鸣曲的乐句。”钢琴家说。
\par “啊!老天!该不是那支《奏鸣蛇》吧?”福什维尔问道,一心想引人注目。
\par 戈达尔大夫从来没有听过这么一个用谐音字进行的文字游戏,不明白这是什么意思,还以为是福什维尔先生说错了呢。他赶紧走到他跟前去纠正这个错误。
\par “不,没有什么叫‘奏鸣蛇’的,只有响尾蛇\footnote{奏鸣蛇在原文中为“Serpent a sonates”,响尾蛇为“serpent a sonnettes”。}。”他热情急切,得意洋洋地说。
\par 福什维尔给他解释了一下这个文字游戏的由来。大夫脸红了。
\par “您该承认这挺逗吧,大夫?”
\par “啊!这我早就知道。”戈达尔答道。
\par 他们这就不再吭声了。这时那个小乐句在小提琴部高出两个八度的颤抖的震音的陪送下出现了——这就像是在山区,人们在高得令人晕眩、仿佛是凝滞不动的瀑布背面,看到在两百尺之下,一个正在散步的孤独女子的纤小的身影。这乐句在那透明连绵、高昂而汹涌澎湃的背景之中,从遥远的地方款款而来,优美无比。斯万这时心底里在跟这个乐句窃窃私语,仿佛它是他爱情的知情人,是奥黛特的一个朋友,来嘱咐他不必把这个福什维尔放在心上。
\par “啊!您来晚了,”维尔兰迪夫人对一位应邀仅仅在餐后“剔牙”时分才到的信徒说,“刚才有位布里肖先生在这里,那份口才,真是无与伦比!可惜他已经走了。您说是不是,斯万先生?我想您这是跟他第一次见面吧。”她说这话是为了提醒斯万,他之所以有缘认识他,全是凭了她的关系。“咱们这位布里肖可爱极了,是不是?”
\par 斯万很有礼貌地躬了躬身。
\par “不吗?您对他不感兴趣?”维尔迪兰夫人冷冰冰地问他。
\par “不,夫人,挺感兴趣,我高兴极了。不过他也许有点过分专断,也许有点儿过分嘻嘻哈哈,不合我的口味。我倒希望他有时谦虚一点,文雅一点,不过看得出来,他知道很多东西,看起来也是个好样儿的。”
\par 晚会结束得很晚。戈达尔对他的妻子说:
\par “难得看到维尔迪兰夫人有像今晚这么大兴头的。”
\par “这位维尔迪兰夫人到底是何许人物?金玉其外,败絮其中?”福什维尔问画家,一面邀他坐他的车回去。
\par 奥黛特不无遗憾地眼看着福什维尔离去,她不敢不跟斯万一起回去,可是在车上她一直很不高兴,当他问她,他是不是该进屋时,她说,“当然。”可又不耐烦地耸了耸肩膀。当客人都走光了的时候,维尔迪兰夫人问她丈夫:
\par “你有没有注意到,当我们提到拉特雷默伊耶夫人的时候,斯万直傻笑。”
\par 她可注意到斯万和福什维尔在提到这个名字的时候,好几次都把“德”字省掉了。她毫不怀疑他们这是为了显示自己并不拜倒在头衔之下,她自己也想效法他们那种矜持,然而又拿不稳该用什么语法形式来表达这份感情。结果还是她那错误的语言习惯占了她那反封建的共和主义情绪的上风,她有时说les de la Tremoille,有时又学咖啡馆里的歌星或者漫画作家给漫画写说明文字时的样子,把de字来个元音省略,说什么les d’La Tremoille,不过说了以后马上就加以改正,还是说“拉特雷默伊耶夫人”。她又嘲讽地找补一句:“斯万却爱管她叫公爵夫人。”脸上那个微笑表明她不过是重复斯万的话,并不承认这个既幼稚又可笑的称呼。
\par “不瞒你说,我觉得他傻极了。”
\par 维尔迪兰先生答道:
\par “这位先生不坦率,总是那么假惺惺,总是那么吞吞吐吐。老是两面不得罪。这跟福什维尔是多么不同!福什维尔有什么就说什么,不管你爱听不爱听他所说的话。他不像那一位,从来都是真真假假。而且奥黛特似乎也更喜欢福什维尔,我觉得她是对的。再说斯万在咱们面前摆出一副上流社会人士的架子,摆出一副公爵夫人的保卫者的架子。那一位可真有爵位,他是福什维尔伯爵。”他的话音是那么柔和,仿佛他对这个伯爵领地的历史了若指掌,给予它以极高的评价。
\par “我跟你说吧,”维尔迪兰夫人说,“他居然敢含沙射影地恶毒攻击布里肖,其实说的都是些荒唐可笑的话。当然,那是因为他眼看布里肖得到满座欢迎,攻击他就是攻击咱们,就是破坏咱们的聚会。我感觉得出来,这小子一出这大门,准把谁都说得一钱不值。”
\par “我不早跟你说了吗?”维尔迪兰先生答道,“这家伙不得志,看什么都眼红,都妒忌。”
\par 事实上,没有哪一个“信徒”的心地有像斯万那样好的;只不过所有的人都小心翼翼地把他们的恶意用众所周知的笑话,用一点儿感情,用一点儿真挚掩盖起来罢了;而斯万不屑于用什么“我这不是想说什么坏话”这样的陈词滥调来掩饰,所以他的任何含蓄都被看成是阴险恶毒的表现。有一些不同凡响的作家,他们的任何大胆言论都激起公众的反感,因为他们不屑迎合公众的趣味,不为公众提供他们习以为常的老生常谈;斯万之所以激怒维尔迪兰先生,也是这个道理。跟那些作家一样,正是斯万言语中的不落俗套使别人觉得他别有用心。
\par 斯万对他在维尔迪兰家面临的失宠的威胁依然一无觉察,他身堕情网,继续把他们那些可笑的言行加以美化。
\par 他通常只在晚上才跟奥黛特有约会,唯恐白天也上她家去会使她感到厌烦,但他却希望她老念着他,所以随时都找机会引起她对他的思念,但当然是以叫她感到高兴的方式。如果他从花店或者珠宝店的橱窗面前走过,视线被一棵小树或者一颗珠宝所吸引,他马上就会想到把它送给奥黛特,心想当她体会到他在得到这些东西时的乐趣时,就会使她对他更加温存,他就会马上叫铺子派人送到拉彼鲁兹街去,因为每次当她收到他什么礼物的时候,他总感觉他自己就在她身边一样。他尤其希望她能在离家外出以前收到这些礼物,这样当她在维尔迪兰家看到他的时候,她的感激之情就会化为对他更热烈的接待,甚至如果送货的人等不及的话,她还会在晚餐前打发人送封信给他,或者亲自到他家来道谢。从前他体会到她的性格当中有些令人反感的地方,现在则竭力从她的感激之情中探索她以前还没有对他流露过的深藏的感情。
\par 她时常手头拮据,为债主所逼而向他求助。他总是乐于效劳:凡是能使奥黛特看出他是如何爱她,或者只是看出他对她能产生影响,能有些用处的事,他都是乐于从事的。当然如果有人在开始时对他说,“她看中的是你的地位”,现在对他说,“她之所以爱你是为了你的财产”的话,他是不会相信的,不过既然人们设想她是由于像追求风雅或金钱这样强有力的东西而跟他关系密切,感觉到他们两人紧密相连,他对那种说法也并不会过分表示不满。即使他认为他们所说的是对的,那么当他发现奥黛特对他的爱除了基于她对他的感情和在他身上发现的品质以外,还有一个更持久的支柱——利害关系时,他也是不会难过的。这种利害关系足以使她试图跟他中断来往的日子永远也不会到来。此刻,他不断送她礼物,为她效劳,那就除了他自己的人品、聪明才智和无所不用其极的取悦于她的强烈愿望外,他还可以依靠另外一些有利条件。这种堕入情网的乐趣,仅仅是为了爱情而活着的乐趣,他有时也怀疑它是否现实,但他作为精神享受的爱好者而为此付出的代价越多,就越是觉得它的价值高昂——我们不是也看到有些人怀疑大海的景象和澎湃的波涛声是否当真美妙,不惜每天花一百法郎租一间海滨旅馆的房间去观赏,从而不但得以信服,而且他们自己超凡脱俗的品格不也得到了肯定吗?
\par 有一天,正当他陷入这样的沉思的时候,忽然想起了从前曾经有人说奥黛特是一个由情人供养的女人,那时他再次把“由情人供养的女人”这个奇怪的修辞学上的拟人表达法,这个像居斯塔夫·莫罗\footnote{居斯塔夫·莫罗(1826—1898),法国画家。}画的幻像那样,镶嵌有同宝石缠绕在一起的毒花,由难以识别、恶魔般的成分构成的闪闪发光的混合物跟奥黛特加以对比了:奥黛特,在她的脸上他可是亲眼目睹那对不幸者的怜悯之情,对不公正的事情的愤慨,对施恩者的感谢,就如同他从前在他自己的母亲,在他的朋友们的脸上看到的表情一样;奥黛特,她的话语时常是跟他自己最熟悉的事物有关,譬如他的收藏、他的卧室、他的老仆人,收存着他的股票的那位银行家,这时,银行家这个形象忽然提醒他该上他那里取点钱了。可不是吗,他上个月给了她五千法郎,如果这个月给她的物质困难的帮助没有那么多,而她想要的那串钻石项链也不给买,那他就不会看到那使他如此幸福的她对他的慷慨大度的赞赏与感激之情,甚至当她看到这种慷慨的表现越来越少,可能会以为他对她的爱情已经淡薄了。想到这里,他突然自问,这是否正是“供养”她呢?(仿佛“供养”这个概念可以出之于一些既不神秘又不反常的成分,且是属于日常私生活的范畴,例如那张普普通通撕破了又粘上的一千法郎的钞票,他的男仆在为他付了当月家用和房租以后塞在他的旧书桌的抽屉里,斯万取出跟另外四张一起送给奥黛特)他也自问,自从他认识奥黛特以来,在他看来跟她毫不相容的“由情人供养的女人”这个词能否用到奥黛特身上(因为他一刻也不曾设想在他之前她会接受任何人的金钱)。但他不能再顺着这个思路想下去,因为他生来就是懒于思维,这股懒劲也是一阵阵的,说来就来,这会儿正是来到的时候,于是就马上把他的智慧之火全部熄灭,就像后来到处用电气照明的时代,一下子就能把全家的灯统统灭掉一样。他的思想在黑暗中摸索了一会儿,他摘下眼镜,擦擦镜片,用手揉揉眼睛,直到找到一个新的思想时才重见光明——这新的思想就是下个月给奥黛特的不是五千而是六七千法郎,好给她来个出乎意料之外,感到异常的快乐。
\par 晚上,当他不呆在家里等着上维尔迪兰家去跟奥黛特相会,或者上布洛尼林园特别是圣克鲁他们爱去的露天餐厅用餐时,他就上他从前作为座上常客的那些上流社会人家去吃饭。他不愿跟那些人脱离接触,也许他们哪天会对奥黛特有些用处,同时也正是由于有了他们,他才时常得到她的欢心。而且,他对上流社会的豪华生活早就有了习惯,就在对它产生厌恶之情的同时,也觉得有过这种生活的需要,以至就在他把最简朴的陋室,跟王公宅第同等看待时,他的感官也是对后者如此习以为常,因此在步入前者时总会感到一定程度的不快。对那些在六楼套房里举行舞会(“请由右门洞登楼,六楼左门”)的小资产者,跟在巴黎举办最豪华的节日活动的帕尔马公主之间,他也有类似的不同观感,那类似的程度是他们难以相信的。当他在主妇的卧室里跟那些当爸爸的人站在一起的时候,他是不会有参加舞会的感觉的,而眼看洗脸盆上盖满了毛巾,床铺改为衣帽间,堆满了大衣和帽子,他就难免产生透不过气来的感觉,就跟用了半辈子电灯的人们闻见冒烟的油灯或者流油的蜡烛味儿时的心情一样。
\par 在他上街吃饭的日子,他让车夫在七点半套车,他一面穿衣服,一面惦记着奥黛特,这样他就可以不至有孤独之感;经常想着奥黛特,使得远离她的时刻也就跟在她身旁时有着同样的特殊的魅力。他登上马车,感到思念奥黛特的思绪跟一头爱畜一样也已经跳上车来,蜷伏在他膝上,将伴着他入席而不被同席的客人所发觉。他抚摸它,在它身上焐暖双手,当他感到有些郁闷时,不禁起了一阵轻微的战栗,缩起脖子,皱起鼻翼——这在他身上是前所未有的——同时把那小束耧头菜花插在纽孔上。一个时期以来,尤其是自从奥黛特把福什维尔介绍给维尔迪兰夫妇以后,斯万感到有些难过忧伤,很想到乡间休息一下。但奥黛特在巴黎,他连离开巴黎一天的勇气也鼓不起来。天气温暖,这是春季最美好的日子。他虽然是在穿过这个石头城到某个围有栅栏的公馆去,可是他在眼前看到的却是他在贡布雷的那座花园,在那里,一到下午四点钟,你还没有走到种龙须菜的畦田,从梅塞格利丝田野那边来的微风就阵阵送香,你在绿树棚下就感到阵阵清凉,就跟在四周都是毋忘我花和葛兰花的池塘边一样。当他在池塘边吃饭的时候,桌子周围全是由他的园丁精心编在一起的醋栗和玫瑰。
\par 晚饭后,如果布洛尼林园或者圣克鲁的约会时间约定得早的话,他就离开饭桌马上就走,尤其是在浓云密布,有可能下雨,“信徒们”会提前回家的时候。有次洛姆亲王夫人家的晚饭吃得较晚,斯万在咖啡还没有端上以前就向主人告辞,赶到布洛尼林园的岛上去跟维尔迪兰家聚会,使得亲王夫人说:
\par “真是的,要是斯万大上三十岁,膀胱又有毛病,那他溜得那么早还情有可原。他真是不把咱们放在眼里。”
\par 他心想,他虽不能到贡布雷去享受这明媚的春光,总可以在天鹅岛或者圣克鲁观赏观赏。不过他的脑子整个儿都给奥黛特占着了,连是不是曾闻到树叶的清香,是不是曾看到皎洁的月光都说不上来。迎接他的是餐厅钢琴上奏出的那首奏鸣曲的小乐句。要是没有钢琴的话,维尔迪兰夫妇不惜费神叫人从卧室或者饭厅搬一架下来,这倒不是因为斯万已经重新博得了他们的好感,根本不是这么回事。为别人提供一点别出心裁的乐趣,哪怕这人并不是他们所喜欢的人,即使在进行准备的阶段,这想法也会在他们身上引发一些对人亲切友好的美好感情——哪怕是昙花一现。有时他也想,又是一个春宵要过去了,他强制自己去注意一下树木和天空。可是他一心思念着奥黛特,难以安下心来。一些时间以来,他那种焦躁不安的情绪又无法摆脱,这就使他不能取得接受大自然的景象所必需的宁静和安逸的心境。
\par 有天晚上,斯万应邀和维尔迪兰夫妇共进晚餐,在进餐时说他第二天要参加当年同在一起服兵役的老战友的聚会,奥黛特在饭桌上当着福什维尔(他现在已经是忠实信徒之一了)、当着画家、当着戈达尔的面说:
\par “是啊,我知道您明天有宴会;那我就只能在我家里见到您了,可别来得太晚啊!”
\par 虽然斯万从来没有因为奥黛特对任何一位信徒有交情而当真感到不快过,但当他听到她当着所有人的面,毫无顾忌,若无其事地承认他俩每天晚上有约会,承认他在她家里的特殊地位,承认她对他的偏爱时,心里感到特别温暖。当然,斯万也常想,奥黛特根本不是一个了不起的女子,他对她处于无比优越的地位,当他看到她当着众信徒的面洋洋自得时也并不感觉有任何特别得意的地方;但自从他发现奥黛特在许多男人眼里是一个令人神魂颠倒的女子,一个希望能弄到手的女子以后,她的身子在他们身上产生的魅力在他的心中唤起了一种折磨人的渴望,要对她的心的每一个细胞都彻底加以控制。他首先把晚上在她那里度过的时刻看做千金难买的时刻,让她坐在他的膝上,讲讲她对这样那样事情的看法,自己则历数在这世上现在还不肯放手的是哪些财富。因此,在那顿晚饭以后,他把她拉到一边,一个劲儿对她表示谢意,力图让她知道怎样按照他所表示的感激之情的程度,估摸出她所能为他提供的各种乐趣的大小高低——其中最大的乐趣是当他对她的爱继续下去而可能招致情敌的时候,能得到无需吃醋的保证。
\par 第二天宴会结束时,大雨倾盆,他却只有那辆四轮敞篷马车;有位朋友提出用他的轿车送他回家。奥黛特昨天既然要他去,那就表明她不会等待别人,斯万原可以放心大胆地回家睡觉而不必冒雨前往的。然而,如果她看到他并无意坚持每天毫无例外地都跟她在一起度过后半夜的话,那就有可能当他特别要同她一起欢度良宵的时候,她却另有约会了。
\par 他过了十一点才到她家,当他连声抱歉没能早些来时,她却抱怨时间实在太晚,又说刚才风狂雨暴,她不舒服,脑袋疼,只能陪他半个钟头,到十二点就要请他回去;过不多久,她就累得要命,想去睡觉了。
\par “那么今晚就不摆弄卡特来兰花了?”他对她说,“我倒真想好好摆弄一下呢!”
\par 她撅起嘴,神经质地说:
\par “不,亲爱的,今晚就不摆弄卡特来兰花了,你看我不是不舒服吗!”
\par “也许摆弄一下对你倒有好处,不过我也并不坚持!”
\par 她请他在走以前把灯灭掉,他亲自把帐子放下再走。可是当他到了家里,他忽然想起奥黛特也许今晚在等什么人,累是装出来的,请他把灯灭了只是为了让他相信她就要睡着,而等他一走,就立即重新点上,让那人进来在她身边过夜。他看看表,离开她差不多才一个半小时,他又出去,雇上一辆马车,在离她家很近的一条跟她住宅后门(他有时来敲她卧室的窗,叫她开门)那条街垂直的小街停下;他从车上下来,街上是一片荒凉和黑暗,他走了几步路就到了她家门口。街上所有的窗户都早就一片漆黑,只有一扇窗,从那像葡萄酒榨床里压挤神秘的金黄色的果肉的木板那样的百叶窗缝里溢出一道光线。在如此众多的别的夜晚,当他走进街口老远就看到的这道光线,曾使他心花怒放,通知他“她在等着你”,而现在却告诉他“她正跟她等待的那个人在一起”而使他痛苦万分。他想知道那个人是谁;他沿着墙根一直悄悄走到窗口,可是从百叶窗的斜条缝里什么也瞧不见,但听得见在夜的沉寂中有喃喃的谈话声。
\par 当然,看到这道光线,想到在窗框后在它的金色的光芒中走动的那一对男女,想到在他回家以后来到的那个人暴露了。奥黛特的虚伪暴露了。她正在跟那一位共享幸福生活的这阵窃窃私语也暴露了,他是何等的痛苦啊。然而他还是为他来了而高兴:促使他从家里出来的那份折磨心情,由于越来越明朗而不再那么强烈,因为奥黛特的生活的另一面,当时对它突然产生了怀疑而又无可奈何,现在却明摆在他的面前,被那盏灯照得一清二楚,被囚在这屋里而不自知,而他只要高兴,就可以进去把它捉拿归案。他也可以像平常晚来时一样,去敲敲百叶窗;这样,奥黛特至少可以知道他已经掌握情况,看到了那道光,听到了他们的谈话;而他呢,刚才还在设想她正跟那一位在笑他蒙在鼓里,现在却要眼看他们当场认错,上了被他们认为远在千里之外的他的圈套。也许,他在这几乎是令人惬意的时刻所感到的并不是什么怀疑和痛苦的消失,而是一种属于智力范围的乐趣。自从他爱上奥黛特以后,他以前对事物的浓厚兴趣有所恢复,但这也限于跟对奥黛特的思念有关的事物,而现在他的醋意激起的却是他在好学的青年时代的另一种智能,那就是对真情实况的热烈追求,但那也限于跟他与他的情妇之间的关系有关的真情实况,仅仅是由她的光辉所照亮的真情实况,一种完全是与个人有关的真情实况,它只有一个对象,一个具有无限价值,几乎是具有超脱功利之美的对象,这就是奥黛特的行动、跟她有联系的人、她的种种盘算、她的过去。在他的一生中的其他任何时期,他总认为别人的日常言行没有什么价值,谁要是在他面前说三道四,他总觉得没有意义,即使听也是心不在焉,觉得自己此刻也成了一个最无聊的庸人。可在这奇怪的恋爱期间,别的一个人竟在他身上产生如此深刻的影响,他感到在他心头出现的对一个女人的最微不足道的事情的好奇之心,竟跟他以往读历史的时候一样强烈。凡是他往日认为是可耻的事情:在窗口窥看、巧妙地挑动别人帮你说话、收买仆人、在门口偷听,现在就都跟破译文本、核对证词、解释古物一样,全是具有真正学术价值的科学研究与探求真理的方法了。




\paragraph*{3}

\par 他正要抬手敲百叶窗那片刻,想到奥黛特就要知道他起了疑心,到这里来过,在街上守候过,不禁产生了一阵羞耻之心。她曾经对他说过,她对醋心重的人,对窥探对方隐私的情人是多么讨厌。他就要干的事情确实是笨拙的,她从此就要讨厌他了,而在他没有敲百叶窗之前,尽管她欺骗他,可能还是爱他的。人们为图一时的痛快而牺牲多少可能的幸福啊!
\par 但要弄清真情实况的这种愿望却更加强烈,在他看来也更为崇高。他知道,他不惜生命代价去核实的这个真情实况在这露出道道光线的窗户背后就能读出,这就好比是一部珍贵文献的烫金封面,查阅文献的学者对它底下的手稿的艺术价值是不会不动心的。他对这以如此温暖、如此美丽的半透明的物质制成的这个独一无二、稍纵即逝、宝贵异常的稿本的真情实况,急切地渴望着要了解。再说,他所感到自己高出于它们的地方——他又是如此需要有这样的感觉——也许与其说是他知道它们,倒不如说是他可以在它们面前显示他知道它们。他踮起脚。敲窗户。人家没有听见,他敲得更响,谈话戛然而止。只听得有个男人的声音,他竭力去辨认到底这是他所认识的奥黛特的哪个朋友的声音:
\par “谁啊?”
\par 他拿不稳是谁的声音。他再一次敲百叶窗。窗开了,接着是百叶窗也开了。现在可没法后退了,因为她马上就要知道真相,而为了不至显得过分狼狈,醋心太重,又太好奇,他只好装出一副若无其事的样子,欢快地叫道:
\par “别费事了,我路过这里,看见有光,想问问您是不是已经好些了。”
\par 他抬头一看,只见两位老先生站在窗口,其中一位举了盏灯,这就把房间照亮了——一间陌生的房间。平常在很晚的时刻到奥黛特家来时,他总是凭着在所有一模一样的窗户当中唯一有光这一点来认出她的窗户,这次却弄错了,敲了隔壁那家的那一扇。他连声道歉着走开,回到家里,直为好奇心得到满足,又无损于他俩之间的爱情而感到高兴,同时也为在如此久长的时期内假装对奥黛特的一定程度的冷淡以后,现在并没有使她通过他的醋心的发作,发现他的爱情过分强烈,从而今后会对他降温而感到高兴。
\par 这段经历,他没有跟她说起过,自己也不再去想它。但是有时脑子一动,就把这潜伏在脑海深处的对这件事情的回忆勾了起来,栩栩如生,只好重新把它埋得更深,这时他就突然感到强烈的痛苦。这仿佛是一种肉体的痛苦,斯万的思想无法使它减轻;然而如果这是一种肉体的痛苦的话,它至少与思想无关,思想总还可以仔细端详它,发现它已经减弱,已经一时消失。可是他那种痛苦,每当思想念及的时候,只能使它重新出现。想要不去想它,实际上是再一次想到它,他为此而更加感到痛苦。当他跟朋友们谈话的时候,他忘了他的痛苦,可是别人不经意间讲出的一句话会使他突然失色,就好像是一个伤员被冒失鬼触到了伤处一样。当他离开奥黛特的时候,他心情愉快,感到心地宁静,他回忆她在谈起别的男人时的带有讽意的微笑,和对他的充满温情的笑容;回忆她怎样把头低垂下来,几乎是不由自主地俯向他的双唇,好像是第一次在马车中时那样;回忆起当她在他怀中时像是怕冷一样怎样把脑袋紧紧靠在他的肩上,两眼向他投来无神的目光。
\par 然而他的醋意却和他的爱情仿佛是如影随形,马上就出来为她今晚向他投来的微笑提供一个副本,来了一个颠倒,变成是对斯万的嘲笑而充满着对另一个人的爱;她的脑袋低垂下来也是俯向别人的双唇,而她对他的一切温情的表现也都以别人为对象了。他从她家里带回的一切令人销魂的印象现在都仿佛变成了一个室内装饰师提供的一些草图,一些方案,使得斯万据以设想她可能在别人面前表现出来的热烈的、狂喜的举止。这样,他都为在她身边体会到的每一个乐趣,为他自己设想出来的每一个爱抚的动作(他还如此有欠谨慎,告诉她这些动作是如何使他欢快),为他在她身上发现的每一个优美之处感到后悔,因此他知道,过一会儿,这些又都会成为她手中用来折磨他的新的刑具。
\par 当斯万想起几天以前,他突然初次发现奥黛特眼中短促的一瞥;这一回忆使得那个折磨显得更加残酷。那是在维尔迪兰家晚饭之后发生的。福什维尔也许是感觉到他的连襟萨尼埃特在他们家并不得宠,想把他嘲弄一番,自己出出风头;也许是因为萨尼埃特刚对他说了些什么傻话而感到恼火,尽管在座的旁人都没有听见,更不会知道说话的人在无意中刺伤了什么人;也许是早就蓄意要把对他自己的底细了解得一清二楚,有时一见面就感到不舒服的这个老好人轰出这个家门,所以十分粗暴地回答萨尼埃特的笨拙的话,居然把他骂将起来,而由于对方害怕、软弱、哀求,他越骂越加大胆,弄得这个可怜虫在问了维尔迪兰夫人他是否还该呆下去而得不到答复时,只好热泪盈眶,嘟嘟囔囔地走开了。奥黛特无动于衷地看着这个场面,但当门在萨尼埃特背后砰的一声关上的时候,她脸上通常的表情仿佛是降下好几档,以便在卑劣方面能跟福什维尔媲美。她的眸子里闪现出一个狡黠的微笑,这对福什维尔的大胆行动是个祝贺,对它的牺牲品则是嘲讽;她向他投过同谋作恶的一瞥,仿佛是说:“要是我看得不错的话,他这下可完蛋了。您看见他那副尴尬的样子没有?他都哭了。”福什维尔看到她这眼神,突然收起怒容(或者是假装出来的怒容),微笑一下答道:
\par “他只要学得讨人喜欢一点,还是可以来的,不管年老年少,接受个教训总是有好处的。”
\par 有一天斯万下午出去访客,那人没有在家,他就想去奥黛特家,虽然他从没有在这时候去过,但他知道她这时准在家里,或者午睡,或者写信,然后用午茶;他想在这时候去看她该很有意思,也不至于打扰她。看门人说他想她是在家的;他按门铃,仿佛听到有声音,有人走动,却没有人来开门。他又着急又气恼,就上那宅子后门那条小街,走到奥黛特卧室的窗口;窗帘挡着,里头什么也看不见;他使劲敲窗玻璃,叫唤;没有人来开窗。他只见有些街坊探出头来瞧他。他走了,心想他刚才也许是听错了,其实并没有什么脚步声;然而他总是放心不下,脑子没法想旁的事情。一个钟头以后,他又回来,看到了她,她说刚才他按铃的时候是在家的,只是睡着了;铃声把她吵醒了,她猜想是他,赶紧跑上前去,可他已经走了。她也听到了敲后窗玻璃的声音。斯万马上就在她这话里听出那些被人当场抓住的撒谎的人为了自我安慰而在他们所编的谎话当中插进去的一点真情实况,他们心想这点真情实况编进去了就可以使谎言显得逼真。当奥黛特做了什么要瞒着别人的事情,她当然是要把它深藏心中的,然而当她一旦面临她所要瞒着的那个人时,她的心就乱了,她的思想就散架了,她编造和推理的能力也都瘫痪了,脑子里成了真空,然而又必须说点什么,能想得起来的却正好是她要隐瞒的,因为这需要隐瞒的事情是真实的,所以是唯一留存在脑际的东西。她从中取出一点本身并不重要的细节,心想这个细节经得起检验,不像虚假的细节那么危险。她心里想:“再怎么说,这是真实的,这就是一个优点,他尽管去打听,结果总会承认这是真的,是不会使我露馅的。”她错了,正是这个使她露了馅;她没有意识到,这个真实的细节有一些棱角只有跟经她任意阉割了的相关细节才能接合得天衣无缝,而不管她把那个真实细节插在怎样的编造出来的细节中间,这些细节总会以其过分夸大其词,或者由于还有一些没有补好的窟窿而暴露出那个真实的细节跟它们并不构成一体。斯万心想:“她承认听见我按门铃,听见我敲窗子,又心想是我,想要见到我。可这跟她没有叫人开门这个事实不协调啊。”
\par 可是他并没有把这个矛盾点出来,心想让奥黛特说下去,她也许又会撒什么谎,可能为真情实况多少提供一点线索;她一个劲儿说,他也不去打断她,而以又渴望又痛苦的心情听着她对他讲的那些话,感觉到它们像圣殿前的幕布一样,模模糊糊地掩盖着,依稀地勾画出那个无限宝贵,然而可惜又无法探得的真情实况(她在说话时确实在遮遮掩掩)——那就是刚才在他三点钟来到的时候,她到底在干些什么。这个真情实况,他也许永远只能掌握一些谎言,一些不可思议、无法判读的历史遗迹了,它仅仅存在于捉摸它而无法估量其价值的那个人的隐秘的记忆之中,可她是不会泄露给他的。当然,他有时也想,奥黛特的日常活动也未必值得那么热切地关注,她可能跟别的男人之间的关系,一般地说,也不至于使一个有思想的人产生如此强烈的忧伤,以至想去殉什么情。他这就认识到,他身上那种关注、那种忧伤只不过是一点小毛病,一旦过去了,奥黛特的一举一动,她给他的那些吻,依然会跟别的那些女人的动作和亲吻一样,不至勾起他伤心的回忆。然而当他认识到他的这种痛苦的好奇心的根子就在他自己身上时,这却并不能使他觉得把这种好奇心看成至关重要,竭尽全力去满足它就是什么违反理性的事情。这是因为,像斯万这样岁数的人,他们的人生哲学已经和年轻人不一样了;尤其是斯万,受到当代哲学的影响,也受到洛姆亲王夫人那个圈子的影响,在那里,大家认为一个人的才气跟他对一切事物的怀疑成正比,认为只有在每一个人的个人爱好中才能找到真实的和不容争论的东西。像他这样岁数的人生哲学是实证的,几乎是医学的哲学,他们不再显露他们追求什么目标,而试图从逝去的岁月中探得一些可以被他们认为是他们身上的特征性的、恒久的习惯和激情的残余,而他们首先关注的是他们现在的生活方式能不能符合那些习惯和激情。斯万认为承认由于不知道奥黛特干了些什么而感到痛苦是明智的,就跟他承认潮湿的天气会加剧他的湿疹一样;他也认为在支出中拨出一大笔钱来收集与奥黛特的日常生活有关的情报(缺了就会使他感到不幸)是明智的,他对那些有把握得到乐趣(至少是在堕入情网之前)的其他爱好,例如收藏艺术和美味佳肴,不也是这样做的吗?
\par 那天当他要跟奥黛特道别回家时,她请他再呆一会儿,在他要开门出去的时候,甚至拽住他的胳膊热烈挽留他。可是他并不在意,因为在一次谈话里众多的手势、言语、细微的事件当中,我们不可避免地对隐藏着我们的疑心所要探索的真情实况的那些手势等等视而不见、听而不闻,发现不了有什么足以引起我们注意的东西,而对没有什么内容的那些反倒全神贯注。她一再对他说:“你从来都不在下午来,难得来一次,我又没有见着你,你看多倒霉!”他明知道她对他的爱还不至于深到对他的来访未晤感到如此强烈的遗憾的地步,不过,她的心肠还是好的,也有心取得他的欢心,当她引起他不快的时候,他时常也确实难过,所以这次没能使他得到同她相处一个小时的乐趣,她心里难过也是很自然的,但这个乐趣在他看来会是一个很大的乐趣,在她心目中却未必如此。事情本来就没有什么了不起,她却一直显得很痛苦的样子,这就使得他不胜诧异了。她那副面容就比平常更使他想起《春》的作者、那位画家\footnote{指意大利文艺复兴时期画家波堤切利(1500—1571)。}笔下的妇女们的面容。她这时就有着她们在让孩提时的耶稣玩一只石榴或者看到摩西向马槽中倒水时那副沮丧伤心的表情,仿佛心中有着不堪承受的痛苦。她这种忧伤的表情,他以前是见过一次的,却忘了是什么时候。突然间,他想起来了:那是她有一次为了跟斯万在一起吃饭,第二天对维尔迪兰夫人撒谎说是头天有病才没有上她家去。说实在的,哪怕奥黛特是世上对自己要求最严格的女人,也用不着为了这么一点并无恶意的谎话感到如此悔恨。不过奥黛特常撒的谎并不是那么无可指责,它们是用来遮掩她跟某些朋友之间的一些麻烦事儿的。因此,当她撒谎的时候,心里是胆怯的,感到自己难以自圆其说,对所撒的谎能否奏效缺乏把握,心力交瘁得简直要像有些没有睡好的孩子那样哭将起来。此外,她也知道她的谎言通常是要严重伤害对方的,而谎要是撒得不周到,她又要落入对方的摆布之下。因此,她在他面前既感到谦卑,又感到有罪。而当她撒的是社交场合中毫无所谓的谎的时候,通过一些联想,一些回忆,她也会感到疲惫不堪,感到做了一件坏事的悔恨之情。
\par 她这时对斯万撒的倒是怎样折磨人的谎,居然使得她眼神如此痛苦,嗓音如此哀婉,仿佛是在求饶,仿佛都要难以自持了。就在这时候,他听到一阵铃声。奥黛特还在说下去,可她的话语已经成了一阵呻吟:她为没能在下午见到斯万,没能及时为他开门的这种遗憾简直成了一件终身憾事了。
\par 只听得大门又关上了,还有马车的声音,看来是有人折回去了——多半是一个不能让斯万见面的人,刚才别人跟他说奥黛特没有在家。斯万心想,仅仅在通常不来的时刻来这么一次,他就打乱了她那么多不愿让他知道的安排,心里不免有些泄气,甚至是苦恼之感。然而他还是爱奥黛特的,脑子里时时刻刻都在想着她,对她的怜悯之心油然而生,喃喃地说:“可怜的小宝贝!”当他离开她的时候,她把桌子上的好几封信交给他,问他能不能顺便为她投邮。他把这些信带走,回到家里才发现还留在身上。他又回到邮局,从衣兜里掏了出来,在扔进信箱之前先把地址瞧上一眼。全都是写给供应商的,只有一封是写给福什维尔的。他把这一封留在手里,心想:“我要是看一看信里说的是什么,就能知道她怎么称呼他,用什么口气说话,两个人之间是不是有什么关系。我要是不看一看,也许倒是对奥黛特不关心的表现,因为我这疑心也许是冤枉了她,徒然使她难过,把信看一看是消除这个疑心的唯一的办法,而信一旦寄走,我的疑心不消除,她也只能一直难过下去了。”
\par 他离开邮局,身上带着那封信回家。他点上一支蜡烛,把信封挨到烛光边(信封他是不敢拆的)。先是什么也看不见,不过信封很薄,用手摁在里面的硬卡片纸上还是可以看出最后几个字。那是一句平平常常的结束语。如果不是他来看她写给福什维尔的信,而是福什维尔来看她写给斯万的信的话,那他是会看到一些无比亲热的话语的!信封比里面装的卡片大,他用大拇指使卡片滑动,把一行行的字移到信封上没有夹层的那一部分,这是唯一能透出里面的字迹的那一部分。
\par 尽管如此,他还是看不太清楚,不过这也没有什么关系,反正他已经看到了足够多的文字,明白信里没有什么了不起的内容,跟什么恋情根本不沾边;这是跟奥黛特的舅舅有关的什么事儿。斯万在有一行的开头看到了“我怎能不”这几个字,可不明白奥黛特怎能不干什么,可忽然之间,刚才没有能辨认出来的几个字看清楚了,这就把全句的意思弄明白了:
\par “我怎能不去开门,那是我舅舅。”原来当斯万按门铃的时候,福什维尔在她家,是她把他打发走的,所以他听到了脚步声。
\par 这时他就把全信都读完了;在信末她为对他如此失礼而致歉意,还告诉他,他把烟盒丢在她家了,这也是斯万第一次来时她信上的那句话,不过那次还加了一句:“您为什么不连您的心也丢在这里呢?如果是这样的话,我是不会让您收回去的。”而对福什维尔则没有这样的话:没有任何迹象表明他们当中有什么勾搭。说真的,福什维尔比他受骗受得还更厉害,因为奥黛特在给他的信上说来客是她的舅舅。总而言之,在她心目中,是他,斯万,占有更多的地位,也是为了他,她才把那一位打发走的。然而,要是奥黛特和福什维尔之间没有什么的话,为什么她没有马上开门,为什么要说“我怎能不去开门,那是我舅舅”呢?要是她那会儿没做什么不好的事,福什维尔又怎能相信她不马上去开门的道理?斯万愣住了,既难过,又惶惑,然而面对奥黛特放心大胆地交给他的这个信封,却又感到高兴,因为她绝对相信他是个正派人,然而通过信封那个透明的窗口,除了他心想永远也不会弄清楚的那个秘密之外,也向他泄露了奥黛特生活的一角,仿佛是为未知的王国打开了一道透亮的窄缝。这时候,他的醋意为这一发现而大为兴奋,这醋意似乎有它自己独立的生命,自私心很强,对一切足以滋养它的东西全都贪而食之,甚至是损害斯万自己也在所不惜。现在这醋意就有了它的食料,斯万也就每天都为奥黛特在下午五点钟左右接待什么人而操心,想方设法去打听福什维尔这个时候在什么地方。这是因为,斯万对奥黛特的爱情还保持着开始时那样的特点,他既对奥黛特如何度过她的一天一无所知,脑子又懒于用想象去填补这个空白。首先,他不是对奥黛特的全部生活有所猜疑,而是仅仅对她一天中的某些时刻,在这些时刻中有某种情况(也许是经过曲解了的)使他猜想奥黛特会对他不忠。他的这种猜疑就像章鱼一样,最初伸出一只触手,又伸出第二只,再伸出第三只,先牢牢地固着于下午五点钟这个时刻,其次,是另一个时刻,然后又是另一个时刻。然而斯万是不会无中生有地编造出他自己的痛苦之情的。他的那些痛苦之情无非是来自外界的某种痛苦之情的回忆和继续。
\par 而外界的一切却给他带来一次又一次的痛苦。他想把奥黛特跟福什维尔隔离,把她带到南方去些日子。可他又想所有在旅馆里的男人都会追求她,她也会追求他们。他自己过去在旅途中也总是追求新欢,到人头攒动的地方,而现在人家却觉得他有点离群索居,回避社会,仿佛曾经惨遭社会的伤害似的。当他把每一个男人都看成是奥黛特潜在的情人的时候,他又怎能不厌恶人类呢?就这样,斯万那份醋劲儿就比当初他对奥黛特的欢快强烈的欲念更进一步地促成他性格的改变,使得他在别人眼里彻底变了样,连表现出他的性格的那些外部特征也都完全变了。
\par 就在他读了奥黛特给福什维尔的那封信的一个月以后,斯万去参加维尔迪兰家在布洛尼林园设的一次晚宴。正当大伙要散席的时候,他注意到维尔迪兰夫人跟几个客人交头接耳,看来他们是要提醒钢琴家第二天参加夏都那个聚会;而斯万呢?他可不在应邀之列。
\par 维尔迪兰夫妇压低嗓门说话,用词也含含糊糊,那位画家却粗心大意,高声叫道:
\par “到时候什么灯也别点,让他在黑暗中弹《月光奏鸣曲》,咱们好好欣赏欣赏月色。”
\par 维尔迪兰夫人看到斯万就在跟前,脸上做出一副表情,既要示意说话的人住嘴,又要让听话的人相信这事与她无关,然而这个愿望却被她那木然无神的双眼淹没了,在她那目光中,无邪的微笑背后掩盖着同谋的眼色;这种表情是发现别人说漏了嘴的人都会采取的,说话的人也许不会马上认识到,听话的人却立刻就心里有数了。奥黛特突然变了脸色,仿佛是觉得做人实在太难,只好听天由命。斯万心急如焚,盼着赶紧离开餐厅,好在路上向她问个明白,劝说她明天别上夏都去,或者想法让他也应邀前往,同时希望自己的焦躁不安能在她的怀中得以平静下来。总算到了叫马车的时刻。维尔迪兰夫人对斯万说:
\par “再见了,希望不久就能再看到您。”一面试图用亲切的目光和假惺惺的微笑来避免他注意到她不像往常那样说:“明儿个夏都见,后天上我家。”
\par 维尔迪兰夫妇让福什维尔登上他们的车,斯万的车停在他们的车后面,他在等着让奥黛特上去。
\par “奥黛特,我们送您回家,”维尔迪兰夫人说,“福什维尔先生旁边还有个位置呢。”
\par “好的,夫人。”奥黛特答道。
\par “怎么?我一直以为是由我送您回家的。”斯万高声叫道,也顾不得挑选委婉的词语了,因为这时车门已经打开,他早已等得不耐烦,决不能就这样单独回家。
\par “可维尔迪兰夫人要我……”
\par “得了,您就独自回去吧,我们让您送她的次数够多的了。”维尔迪兰夫人说。
\par “我可有要紧的事跟德·克雷西夫人说呢。”
\par “您给她写信好了。”
\par “再见。”奥黛特向他伸出手来说。
\par 他想微笑,可脸色实在难看。
\par “你看见没有?斯万现在居然对咱们这么不讲客气。”当他们回到家里的时候,维尔迪兰夫人对她丈夫说。“咱们送奥黛特回家,看样子他简直恨不得把我一口吞下去似的。实在是太不礼貌了!他干脆把咱们说成是开幽会馆的得了!我真不明白,奥黛特怎么能受得了他那种态度。他那副神气完全是等于说:‘你就是我的人。’我要把我的想法告诉奥黛特,我希望她能明白我的意思。”
\par 过了一会儿,她又怒气冲天地找补了一句:“哼!这畜生!”她不自觉地(也许是出于为自己辩解的需要)用了一头垂死的无辜牲口在最后挣扎时激起宰杀它的农民用的话语,就像弗朗索瓦丝在贡布雷宰那只硬不肯咽气的母鸡时那样。
\par 当维尔迪兰夫人的马车走了,斯万那辆车向前挪动时,他的车夫瞧着他,问他是不是病了,或者发生了什么祸事。
\par 斯万把他打发回去,他宁可走一走,就徒步回到布洛尼林园。他高声自言自语,那语调就跟他一个时期以来历数维尔迪兰家那个小核心的妙处和这对夫妇的宽宏大量时一样,多少有些做作。奥黛特的言语、微笑和吻,他从前觉得是如此甜蜜,现在如果以别人为对象的话,他就会觉得是何等可憎,同样,维尔迪兰家的客厅,他刚才还觉得是如此有意思,它散发着对艺术的真正爱好,甚至是一种精神贵族气派的风味,现在则因奥黛特去见、去自由地爱的已不是他而是另外一个人了,所以也就向他暴露出它的可笑、愚蠢、无耻了。
\par 他带着厌恶的心情在脑子里设想他们明天在夏都举行的晚会。“首先是挑了夏都这么个地方!那是刚打烊的绸布商光顾的地方!那些人满身都是市侩气,简直不像是有血有肉的真人,而是拉比什\footnote{拉比什(1815—1888):法国剧作家,一生写有一百七十三部喜剧。}剧本中的人物!”
\par 去参加的人多半有戈达尔夫妇,可能还有布里肖。“这些小人物搅和在一起,也真够滑稽的,他们要是明天不在夏都聚会,简直觉得自己就要完蛋了!”老天哪!还有那位画家,那位爱拉皮条的画家,他会邀请福什维尔跟奥黛特一起去参观他的画室的。斯万想象奥黛特会穿上对郊游来说是过分时髦的服装,“她这个人就是庸俗,这可怜虫实在是太傻了!!!”
\par 他仿佛听到维尔迪兰夫人饭后开的玩笑,不管这些玩笑以哪一个讨厌家伙为目标,在过去总是能逗他乐的,因为他看到奥黛特为之发笑,跟他一起笑,她的笑声简直跟他自己的笑声融为一体。现在他感到人们会以他作为笑料来引奥黛特发笑。“这是何等令人厌恶的欢快!”他说,嘴撅得简直叫他感觉到脖子上紧张的肌肉都蹭到衬衣领子了。“怎么?一个按上帝的形象创造出来的人竟能从这么令人恶心的笑话中找到笑料?任何一个鼻子稍为灵一点的人都会皱起眉头躲避这样的熏天臭气的。一个人怎么能不懂得,当她居然耻笑一个曾经正大光明地向她伸出手来的同类时,她就堕落到了万劫不复的泥坑?这简直是不可思议!那些家伙是在九泉之下叽叽喳喳,口吐无耻谰言,而我是在九天之上,维尔迪兰那婆娘拿我开的玩笑是溅不到我身上来的!”他昂首挺胸,高声喊道。“上帝可以作证,我是诚心诚意地想把奥黛特从那腐恶的泥坑里拉出来,把她带到高贵些、纯洁些的环境中去的。但是人的忍耐总是有限度的,我的忍耐也已经到头了。”他说,仿佛要把奥黛特从这挖苦人嘲讽人的环境中解救出来的这个使命产生已久,而并不是仅仅几分钟以前的事情似的,仿佛他赋予自己这样一个使命,并不是在他认为那些挖苦嘲讽的话可能以他为对象,而且旨在把奥黛特从他身边拉走那个时刻才开始似的。
\par 他看到钢琴家准备演奏《月光奏鸣曲》,看到维尔迪兰夫人害怕贝多芬的音乐可能刺激她的神经时装出的那副嘴脸。“笨蛋!骗人精!”他高声叫道,“这还叫什么热爱艺术!”她会在奥黛特面前巧妙地说福什维尔的好话(就跟她从前时常说他的好话一样),然后对她说:“您在您身边给福什维尔先生腾点地方好吗?”“在黑暗中!这拉纤人!这皮条客!”“拉皮条的”——他也把那种催一对男女默默地坐下,一起遐想,相对而视,拉起手来的音乐叫作“拉皮条的”。他觉得柏拉图、博叙埃\footnote{博叙埃(1627—1704):法国作家、宣道者。}以及法国的老式教育对待各种艺术的严峻态度不无道理。
\par 总而言之,维尔迪兰家那种生活,原来被他称之为“真正的生活”的,现在在他心目中成了再糟糕不过的生活;他们那个小核心成了最次最次的社交场所。他说:“一点儿也不错,那是社会阶梯中最低的一层,是但丁《神曲》中最低下的那个境界。毫无疑问,但丁那段令人敬畏的话就是针对维尔迪兰夫妇的!说来说去,上流社会的那些人,尽管不无可以指责的地方,却跟这一帮流氓不一样,当他们拒绝结识这一伙,不屑于玷污自己的指头去碰他们的时候,还是很明智的。圣日耳曼区的那句箴言Noli me tangere(不要摸我)\footnote{耶稣复活时,首先看见他的是抹大拉的马利亚(即《路加福音》中原为妓女,后被耶稣感化改恶向善的马德莱娜)。耶稣对她说:“不要摸我,因为我还没有升上去见我的父。”后来用来指不可接触的人或物。}是何等富有真知灼见!”他这时早就离开了布洛尼林园的小径,差不多已经到家了,然而他还没有从痛苦中醒悟过来,还没有从言不由衷的醉狂中清醒过来,他说话时那种不真实的语调和造作的铿锵还在不时加强他的这种醉狂,他依然还在夜的沉寂中滔滔不绝地慷慨陈词:“上流社会的人们也有他们的缺点,这我比谁都看得清楚,然而他们毕竟还是有所不为的。我交往过的时髦女子远不是完美无缺,然而她骨子里还是有细腻的感情的,所作所为讲求正直,不管出现什么情况,她都不会背叛你,这就足以在她跟维尔迪兰这个泼妇之间划出一道不可逾越的鸿沟。维尔迪兰!这是怎么样的姓氏\footnote{维尔迪兰原文为Verdurin,与purin(粪尿)音相近。}!嘿!他们简直是那一号人当中登峰造极,无与伦比的样板!谢天谢地!现在还来得及悬崖勒马,不再跟那一伙无耻之徒,那一伙粪土垃圾厮混在一起。”
\par 然而,斯万没有多久以前还认为维尔迪兰夫妇身上有的那些美德,即使他们当真具有,但如果他们不曾促成并且保护他的爱情的话,还是不足以在斯万身上激起那种为他们的宽宏大量所感动得如醉如狂的境界,同时这种境界如果是通过别人的感染而得的话,这个人也只能是奥黛特。同样,如果维尔迪兰夫妇没有邀请奥黛特跟福什维尔一起去而把他斯万撇开的话,那么他今天在这对夫妇身上发现的背德行为(即使果然如此),也不足以激起他如此狂怒,严厉指责他们“无耻”。毫无疑问,假如斯万在说话的时候避免使用对维尔迪兰这个圈子充满厌恶,对摆脱这个圈子表示欣喜之情的那些字眼,说的时候又不是那么装腔作势,不是为了发泄怒火而是为了表达思想的话,那么他的话语是会比他的头脑更富有远见的。当他沉溺于那番谩骂的时候,他的脑子里想的多半是一个完全不同的对象,因此他一回到家,刚把大门关上,就拍了一下脑门,吩咐把大门重新打开,这回却是以很自然的语调叫道:“我相信我已经想出了明天应邀去夏都参加晚餐会的办法了。”可是这办法并不灵,斯万并没有接到邀请。原来戈达尔大夫被召到外省去看一个重病人,已经多日没跟维尔迪兰夫妇见面,那天也没能到夏都去,晚餐会的第二天他到他们家入席时问道:“那么咱们今天晚上就见不着斯万先生了?他不是有个密友在当……”
\par “我相信他是不会来了!”维尔迪兰夫人高声叫道,“上帝保佑,别让我们再见到这个又讨厌,又愚蠢,又没有教养的家伙。”
\par 戈达尔听了这话,既是大吃一惊,又是俯首听命,仿佛是听到了始料所不及却又明摆在面前的一个真理;他只好既激动又畏怯地把鼻子埋在菜盘里,连声说道:“噢!噢!噢!噢!噢!”中气一点点地衰竭,嗓音一声比一声低沉。从此斯万要上维尔迪兰家去,就根本没有门儿了。
\par 就这样,原来把斯万和奥黛特撮合在一起的这个客厅现在却成了他们约会的障碍。她再也不能像他们初恋时那样对他说:“反正明儿晚上能见面,维尔迪兰家有个晚餐会。”而是:“明儿晚上见不了面了,维尔迪兰家有个晚餐会。”要不然就是维尔迪兰夫妇要把她领到喜歌剧院去看《克莉奥佩特拉之夜》,斯万就会在奥黛特眼里看到恐慌的神色,唯恐他求她别去,而在不久以前,当这样的神色掠过他情妇的脸时,他是禁不住要赐她一吻的,现在它却只能把他激怒了。他心想:当我看到她想去听这种臭大粪似的音乐时,我感到的不是愤怒,而是悲哀,不是为我自己,而是为她;每日相会已六个多月,她竟还没有脱胎换骨,主动地抛弃维克多·马塞\footnote{维克多·马塞(1822—1884),法国音乐家,《黄玉王后》、《克莉奥佩特拉之夜》的作者。}的音乐!特别是居然还不明白,在某些晚上,一个感情比较细腻的人是应该能够应别人的要求,放弃某种乐趣的。哪怕只是从策略上考虑,她也应该说“我不去了”,因为别人是根据她的回答来评定她的心理素质,而且“一旦作出结论就永远难以改变”。他先说服自己,他只是为了能对奥黛特的精神素质作出较有利的评断,才希望她那晚陪着他而不去喜歌剧院,然后拿同样的道理来说服奥黛特,说话时跟刚才说服自己时同样地言不由衷,甚至更有过之,因为他这时还想利用她的自尊心来打动她。
\par “我向你发誓,”他在她临动身上剧场去的时候说,“当我请你别去的时候,如果我是一个自私的人的话,我倒希望你拒绝我的要求,因为今晚我有一大堆事情要做,如果你出乎我意料之外地答应我不去的话,我倒会自找麻烦的。不过我自己的事情,我自己的乐趣并不就是一切,我得为你着想。也许会有那么一天,你离开了我,你那时就有权利责备我,说当我感觉到出之于我对你的爱而应该向你提出严厉的意见的关头,却没有及时提醒你。你看《克莉奥佩特拉之夜》(这是怎么样的标题!),跟这个问题毫无关系。我必须知道的是你到底是不是最没有头脑,甚至是最没有魅力的一个人,到底是不是不能抛弃一种乐趣的一个可鄙的人。如果你是这样的话,别人怎么能爱你呢?因为你连一个人,一个实实在在的,虽然不完美,然而至少是可以完美起来的人都不是。你就成了一滴没有一定形体的水,沿着别人安排的坡面滑下去,你就成了一条没有记忆,不会思想的鱼,在鱼缸里活一天,就上百次地撞那玻璃,一直认为那也是水。我并不是说听了你的回答我马上就会不再爱你,不过当我明白你不像人样、不求上进的时候,你就不会那么迷人,你明白不明白?当然,我原想把要你打消去看《克莉奥佩特拉之夜》(是你逼我玷污了自己的嘴来说出这个肮脏的名字的)的念头看成是一件无关紧要的小事,而心里却仍然希望你去。不过我还是决定要像我刚才那样来考虑问题,要从你的回答中引出那样的严重后果,所以我觉得还是提醒你为好。”
\par 奥黛特早就显得越来越激动,越来越犹豫了。虽然她不明白这篇演讲的意义何在,却知道这是属于指责或祈求的“空论”和演戏一类的东西;她看男人来这一手看惯了,用不着去注意话语的细节,就可以得出这样的结论:如果他们不爱你,就不会讲出那番话来,而既然他们爱你,那就无需照他们的话去做,事后他们只能更加爱你。因此她原本是会泰然自若地听斯万说下去的,只不过时间在流逝,他要再多说几句,她就不免要误了序幕——她带着一个温柔、执著而暧昧的微笑把这意思对他说了出来。
\par 从前他曾对她说过,最能导致他中止对她的爱的,就是她不肯抛弃撒谎这个恶习。他对她说:“你就不能明白,即便单单从娇媚的观点来看,你要是堕落到撒谎的地步,你会失去多少魅力?老老实实讲真话,你又可以补赎多少过失!说实在的,你真没有我原来想象的那么聪明!”斯万把她为什么可以不必撒谎的理由一条一条列举出来,可是毫无用处:奥黛特心里如果有一整套关于撒谎的理论的话,斯万那些理由也许可以把它摧毁掉,然而奥黛特又没有这么一套理论;她只要求每次做了一件不希望斯万知道的事情时不告诉他就是了。因此,对她来说,撒谎是一种特定的手段;她是用这一手段还是说实话,也完全取决于一种特定的理由,那就是斯万发现她没有说实话的可能性是大还是小。
\par 就体态而言,她正经历着一个糟糕的阶段:她发胖了;过去那种富有表情而引人怜爱的妩媚,那带着惊诧而若有所思的眼神,仿佛都随着青春一起消逝了,而斯万却正是在发现她没有从前那么好看的时候觉得她更足珍贵。他时常把她久久凝视,想捕捉过去在她身上看到的妩媚,但是枉然。但他知道,在这新的蛹壳下跳动着的还是奥黛特那颗心,她那变化不定、难以猜透、遮遮掩掩的天性依然如故,这就足以使他继续以同样的激情来力图把她征服。他再看看她两年前的相片,回想起她当时是何等地秀色可餐。这就多少给了他一点安慰,为她操那么多心并没有白费。
\par 当维尔迪兰夫妇把她带到圣日耳曼、夏都、牟朗去的时候,如果天好,他们时常临时提出在那里过夜,到第二天再回来。钢琴家的姨妈在巴黎,维尔迪兰夫人总设法劝说他别为老人担心:
\par “您一天不在她身边,她会感到高兴的。她知道您跟我们在一起,怎么会担心呢?再说,有什么事都有我在担待呢。”
\par 如果她此计不成,维尔迪兰先生就问问他身边那些忠实的信徒,有谁需要向家里送个信的,然后迈过田野,找个电报局发封电报,或者找个人捎封信回去。奥黛特总是谢绝,说是没有什么人需要通知,因为她早就跟斯万说过,当着众人的面给他送这种信,就等于是暴露了自己。有时她一连外出好几天,维尔迪兰夫妇带她上德勒去看坟场,或者按画家的建议,上贡比涅森林去观赏日落,然后一直走到比埃尔丰城堡。
\par “唉,她原本是可以跟我一起去参观这些真正的历史建筑物的;我学了十年的建筑,随时总有一些最有身份的人求我陪他们上博韦或者圣卢德诺去,但我只愿意跟她一起去,可她却跟那些再粗野不过的人先后在路易菲利浦和维奥莱勒迪克的臭大粪面前心醉神迷!我认为用不着是个艺术家就能做出那种东西,而且即使判断力不是特别强,也不至于选中茅房去度假,去就近闻闻大粪啊。”
\par 当她到德勒或者比埃尔丰城堡去了以后——糟糕的是她不答应他跟她一起去,说是那样可能给她带来“不良后果”——他就埋头读最令人陶醉的爱情小说,查火车时刻表,想办法在下午、晚上,甚至是当天早上就赶去和她相会。办法?这不是什么办法不办法的问题,而是要得到批准。火车时刻表跟各趟列车并不是为狗编制的。用印刷成表的形式告诉广大公众,有一趟列车早八时开往比埃尔丰,四时到达,这就是说上比埃尔丰是件合法的行为,无需奥黛特的同意;这也是一个可能以与奥黛特相会的愿望完全无关的事情为目的的行为,因为每天都有不认识奥黛特的人登上车厢,人数是如此之多,以至有必要把机车升起火来。
\par 总而言之,如果他想到比埃尔丰去,她可也没法阻拦。他也当真感到有上比埃尔丰去的欲望,而如果他不认识奥黛特,一定也就去了。很久以来,他就想对维奥莱勒迪克的复原工作有一个更精确的概念。天气这么好,他迫不及待地想到贡比涅森林里去散散步。
\par 真是倒霉,唯独这个地方今天对他有诱惑力,而奥黛特却偏偏不让他去。今天!如果他不顾她的禁令而去,那他今天就能见着她。如果她在比埃尔丰碰上的是别人的话,她会高高兴兴地对他说:“怎么?您也来了!”就会邀他到她跟维尔迪兰夫妇下榻的那个旅馆去看望,可如果是斯万,那她就会生气,就会以为他在盯她的梢,对他的爱就会有所减弱,也许会在见到他时气得扭头就走。等到回来的时候也许会对他说,“那我就连旅行的自由都没有了!”而事实上倒是他自己连旅行的自由都没有了!
\par 他忽然想起,要想上贡比涅和比埃尔丰而不显得是去找奥黛特,那就要让他的朋友福雷斯代尔侯爵陪他同往,他在附近有所别墅。当斯万把这个打算告诉他的时候(可没说出他的动机),他喜不自禁,这是十五年以来斯万第一次答应去看他的产业;斯万不愿意在那里长住,只答应在那里待上几天,一起散散步,游览游览。斯万都已经想象自己跟福雷斯代尔到了那里了。哪怕是在那里见到奥黛特以前,哪怕是在那里见不着她,他也将是多么幸福;能在这一块土地落脚,在那里,即使还不知道她将在哪一个确切的地点,在什么时候出现,他就已经到处都感到她蓦然出现的可能性在突突搏动:在那由于是为了她才来参观而显得美丽的城堡的天井里,在他觉得如此充满浪漫气息的城市的每一条街上,在被浓厚柔和的落日染红了的森林中的一条路上——这些是无数交替使用的掩蔽所,他那飘泊无定、繁殖倍增的幸福的心怀着希望用并不可靠的分身之术前来躲藏。“千万别碰上奥黛特和维尔迪兰夫妇,”他会对德·福雷斯代尔先生说,“我刚听说他们今天恰好就在比埃尔丰。在巴黎有的是时间见面,何必离开巴黎来证明彼此寸步不离?”他的朋友也会纳闷,为什么一到那里他就不断改变计划,走遍贡比涅所有旅馆的餐厅却打不定主意在哪家坐下,其实哪家都没有维尔迪兰夫妇的踪迹,而他那副神色却像是在寻找他口说要回避的人物,而且一旦找到还要躲避,因为如果他当真碰到那一帮人,他是会装模作样地避开的;只要他看到了奥黛特,她也看到了他,尤其是让她见到他并不在牵挂她,他就心满意足了。不,她是会猜到他是为了她才到那里去的。所以等到德·福雷斯代尔当真来找他一起动身的时候,他却说:“真抱歉!我今天不能上比埃尔丰去了,奥黛特正好在那里。”斯万可还是感到幸福,因为在芸芸众生当中唯独他一个人那天没有上比埃尔丰去的自由,那是因为他跟奥黛特的关系跟任何人都不一样,他是她的情人,而对他的行动自由的这种限制只不过是他如此珍惜的那种奴役、那种爱情的形式之一。肯定还是别冒跟她吵嘴之险为妙,还是耐心一点,等她回来。那些日子,他一直俯身在贡比涅森林的地图上,仿佛那是一张爱情国的地图,身边全是比埃尔丰城堡的照片。她有可能回来的日子一到,他就又把火车时刻表打开,计算她可能乘哪一班,而如果在那边多耽搁一些时间,又还有哪几班可乘。他呆在家里不出门,唯恐来电报时不在家,天黑了也不睡觉,怕她乘末班车回来,为了给他来个意外而在半夜里来看他。正在这时他听到有人在按门铃,可是很久没人去开,他想把门房叫醒,同时到窗口去叫奥黛特(如果是她的话),因为哪怕他亲自下楼嘱咐他们十次,他们还是可能对她说他不在家的。原来是个仆人回家。他听到马路上马车不停地飞驰过去,这他以前是从来没有注意过的。他只听得每辆车从远处过来,越来越近,驶过他的门口而不停下,带着不是属于他的信息奔向远处。他等了整整一夜,毫无结果,原来维尔迪兰夫妇他们提前回来,奥黛特在中午就回到了巴黎;她不想通知他;不知干点什么好,就独自一人上戏院看戏,这会儿早就回家上床睡着了。
\par 她连想都没有想他。像这样连斯万的存在都忘却的时刻对奥黛特却更有好处,这比她的全部风情更有助于把他的心系住。因为这样斯万就生活在如此强烈的痛苦的激动之中,就像那晚他在维尔迪兰家没能见着她,找她找了一整夜一样,结果促使他的爱情在他心中萌生开花。我童年在贡布雷时,有过一些幸福的白天,忘了痛苦,而这些痛苦之情直到晚间才又回来。斯万不曾有过这样的白天,他的白天不是在奥黛特身边过的;有时他想,让一个这么漂亮的女人在巴黎单独出去未免太不谨慎,这就跟把一只装满珠宝的盒子摆在马路中央一样。因此他对所有的行人都感到愤慨,把他们全都看成是小偷。然而他们的面貌是集体的,也是无形的,他怎么也想象不出来,所以也就激不起他的醋意。斯万绞尽脑汁,累得用手揉揉眼睛,叫道:“老天保佑!”人们在殚思竭虑来弄清外部世界的现实性或者灵魂的不朽性这样的问题以后,总是要求助于老天爷来缓解缓解疲惫不堪的脑子的。然而对不在身边的那个女人的思念跟斯万生活中再平常不过的行动——吃饭、收信、上街、上床睡觉,通过由于这些动作都是在她不在场的情况下进行的这种遗憾之情而不可分离地连结在一起,就跟玛格丽特·德·奥地利在为纪念她的丈夫美男子菲利贝而修建的勃鲁教堂\footnote{美男子菲利贝(1480—1504)是萨瓦公国的大公。勃鲁在安省首府布雷斯堡,地处巴黎东南422公里,教堂建于1506年至1536年间。}中,为了表示对他的怀念,到处都把他们两人姓名的开头字母交织在一起刻下来一样。有些日子,他不呆在家里而上附近一家餐厅去吃饭,这餐厅的烹调曾得到他的赏识,而现在他去则完全是出之于既神秘又荒谬,被人称之为浪漫色彩的理由;那是因为它(现在依然存在)冠有奥黛特住的那条街的名字:拉彼鲁兹。有时,当她短期出外,总要在回到巴黎几天之后才想起通知他。她干脆就说她是刚乘早车回来的,再也不像从前那样费神去多少找点真情实况来掩饰。这些话都是谎话,至少对奥黛特来说是谎话,站不住脚,不能像真话那样在她到火车站的回忆中找到支持;她在说那番话的时候,甚至懒得在脑子里编造一幅她声称是在下火车时干了些什么的景象。而在斯万的脑子里,她那些话却顺利通行,毫无障碍,扎下了根,那不容置疑的真实性是如此坚不可摧,如果哪位朋友对他说,他也是乘那班车来的并没有碰见奥黛特,那他就会深信是那位朋友记错了日子或者钟点,因为他的说法跟奥黛特的话不相符合。奥黛特的话,他只有在她未说之前就怀疑她要撒谎时才显得是谎话。要让他相信她在撒谎,事先的怀疑是个必要的条件。这同时也是一个充分的条件。这时奥黛特所说的一切就都可疑。只要听到她说一个男人的名字,那肯定就是她的一个情人;这个假设一旦成立,他得花几个星期才能把它消除;有一回他甚至找私家侦探去打听一个不相识的人的地址和每天的活动,非要这个人外出旅行他才能松口气。可后来才知道,此人却是奥黛特的一个叔叔,都死了二十年了。
\par 虽然她一般不同意他跟她一起在公共场所露面,说是会遭人闲话,可是有时候他也跟她一样同时应邀参加某个晚会,如在福什维尔家、在画家家、在哪个部举办的慈善舞会上,那时他就跟她在一起了。他见到她,可不敢待下,唯恐显得是在窥视她跟别人在一起时的乐趣,在他的想象里,这种乐趣是没有穷尽的,因为他从来没有看到它终了时的情况,因为他自己只能独自一人回家,惶惶不安地上床睡觉。几年以后,当他到贡布雷我们家去吃晚饭的那些夜晚,我也有这样的经历。有这么一两回,他通过这样的夜晚,也体验到一种可以称之为平静的欢乐(如果不因不安情绪突然消除而产生过分强烈的冲击的话),因为它使我们的心得到宁静:他有天到在画家的画室中举行的晚会上呆了一会儿,正准备要走,奥黛特这时化装成一个光彩照人的外国人,向周围的男人(而不是向他)含情脉脉,兴高采烈,简直像是预告就在这晚会上或是别的什么地方(也许是狂欢舞会,一想到她要去,他就不寒而栗)将有什么风流艳事发生,而这种高兴劲儿比看真正的肉体的结合更能激起斯万的妒意,因为他对后者比较难以想象;他都已经准备迈过画室的大门了,忽然听到奥黛特叫他:“您能不能等我五分钟,我马上就走,咱们一起回去,您把我送到家。”这几句话砍掉了晚会那叫他惊恐不安的结局,使得晚会在他回想当中竟是那么纯洁无邪,也使得奥黛特的回家不再是一件难以设想的可怕的事情,而成了甘美的现实,而且就跟日常生活的一部分那样摆在他的面前,摆在他的车中;这几句话也剥去了奥黛特那过分光耀夺目,过分欢快的外貌,揭示出她刚才只不过是一时化了装,而且是为了他的,并不是为了什么神秘莫测的乐趣,而对这种化装,她也已经厌倦了。
\par 确实有那么一天,福什维尔要求坐斯万的车回去,当车到了奥黛特家门口,他又要求让他也进去,奥黛特指着斯万对他说:“啊,这可得听这位先生的。您去问他吧。要不就进去坐一会儿,可别太久了,我要提醒您,他喜欢安安静静地跟我谈话,不喜欢在他来的时候来客人。啊!您要是像我那么了解他就好!My love(亲爱的),谁也没有我那么了解您,您说是不是?”
\par 斯万见她当着福什维尔的面对他说出这样表示偏爱的亲切话语,心里自然感动,不过如果她也能说某些批评建议的话,那就更好了,例如“星期天的那个晚宴,您准还没有给人回音呢。您要不爱去就别去,可别失礼”;或者是“您有没有把您关于弗美尔的那篇论文留在这里?明天不是可以多写一点吗?真是个懒骨头!我得督促督促您才是!”这样的话就表明奥黛特了解他在上流社会的应酬,了解他艺术论文进展的情况,表明他们两个人有着共同的生活,说这话的时候,她向他投来一个微笑,通过它,他感觉到她是整个身心都属于他的。
\par 在这样的时刻,当她为他们冲橘子汁的时候,像调得不好的反光镜先在墙上一个目标的周围投上一些古里古怪的大影子,然后慢慢收缩,最后集中消失于目标那一点那样,他对奥黛特的那些变幻无定的可怕的看法也逐渐消失,最后跟站在斯万面前的她那迷人的身体结合起来了。他忽然起疑,在奥黛特家中灯下度过的这个时刻也许并不是摆上道具,搬上蜡果,专门为他彩排的时刻(其目的在于掩盖他不断想着然而又得不出明确概念的那个可怕的微妙的东西,也就是当他不在那儿的时候,奥黛特到底过的是怎样的生活——她的真正的生活),而当真是奥黛特的真正的生活;如果他不在的话,她可能把这同一把扶手椅推到福什维尔跟前,倒给他的也不是别的什么特殊饮料,而就是这种橘子汁;奥黛特生活于其中的世界并不是他成天在确定其位置在何方面,也许仅仅存在于他想象之中的那个可怕的超自然的世界,而确确实实是这现实的宇宙,它并没有什么特殊凄惨的气氛,而是包括他就要去就座写字的那张桌子,他将有机会品尝的饮料,包括所有那些他既怀着好奇和赞叹又怀着感激之情去观赏的事物,因为这些事物在像海绵吸水那样吸收他的梦幻,把他从梦幻中摆脱出来的同时,它们自身也得到了充实;它们也向他指出他的梦幻的看得到摸得着的现实性,引起他的思想的注意;这些事物的形象在他眼前越来越鲜明生动,它们同时也使他困惑的心越来越安定下来。啊!要是命运能允许他跟奥黛特两个人只有一个住处,在她家里就是在他自己家里;在问仆人午餐吃什么时,得到的回答就是奥黛特的菜单;如果奥黛特早上想到布洛尼林园大道散步,他作为丈夫,尽管不想出去,也得陪着她并且在当她太热的时候给她拿着斗篷;晚饭以后,如果她想穿着便服呆在家里,他就得呆在她身边做她要他做的事情;那么,他生活中的那些鸡毛蒜皮的事儿,现在看来是那么乏味,到时候就同时也成了奥黛特生活的一部分,即使是最家常的那些细节,例如包括着那么多的梦幻,体现了那么多的意愿的那盏灯、那杯橘子水、那张扶手椅等等,到时也会变得无比的甘美,分量也会大得出奇!
\par 然而他又心想,他这样就要惋惜失去的安谧和宁静,这两者对爱情可不是有利的气氛。当奥黛特对他来说不再总是一个不在身边、随时怀念的想象中的人物时;当他对她的情感不再是那奏鸣曲的乐曲激起的那种神秘的慌乱,而是深情,而是感激;当他们两人之间建立了正常的关系,结束他的热狂和忧伤时;那时候,奥黛特的日常生活活动在他心目中就不会显得那么重要——他已经多次起过疑心,透过信封看她给福什维尔的信那天就是一例。他冷静地观察自己的病痛,仿佛是在自己身上进行预防接种,以便进行研究;他心想,当他病愈以后,奥黛特做什么事情就与他无关了。然而在他的病态中,说实在的,他对这病愈的害怕不亚于死亡,因为这样的病愈就等于是宣告他现在的一切的死亡。
\par 经过这样的安静的夜晚,斯万的疑心平定下来了;他为奥黛特祝福,第二天一早就派人把最好的首饰送到她家,因为她在前夕的那些好意的表现,在他身上激起的是感激之情,或者是看到这些表现能再现的愿望,或者是需要有所宣泄的爱情的高潮。
\par 可是,也有时候,痛苦之情揪住了他的心,他想象奥黛特是福什维尔的情妇,想象他自己没有被邀请的那次夏都的活动的前夕,他们两个从维尔迪兰家的马车里看着他带着连他的车夫都发现了的那种绝望的神色请她跟他一起回去,结果自己单独一人垂头丧气地回家那会儿,当她叫福什维尔看他那副神色,对他说:“嗨!看他气成那个样子!”的时候,她的眼神准跟福什维尔在维尔迪兰家中赶走萨尼埃特那天一样,闪闪发光、不怀好意、狡黠而微斜的。
\par 那时,斯万就讨厌她了,心想:“我也未免太傻了,花钱为别人买乐趣。她还是留点儿神为妙,别把绳子绷得太紧,等我急了是会一个子儿也不给的。无论如何,额外的优惠得暂时停付了!可就在昨天,当她提到想上拜罗伊特度音乐节时,我却傻得对她说什么要在近郊租一座巴伐利亚国王的漂亮城堡,两个人去住。幸好她并没有显得过分兴奋,也没说是去还是不去;但愿她拒绝吧,我的老天爷!她对瓦格纳的音乐就跟鱼对苹果一样,沾都不沾,一连两个星期跟这么个人听音乐会,敢情是妙不可言!”而他的恨就跟他的爱一样,需要发泄,需要行动,他都乐于把他那往坏处想的想法推得更远,设想奥黛特已经背叛他,这就更加讨厌她了,而如果他这些想法一旦得到证实(这是他力图信服的),就会找机会来惩罚她,把他那一腔怒火在她身上发泄。他都快要设想他就要收到她的信,向他要钱把拜罗伊特附近那个城堡租下,同时通知他,他自己不能去,因为她已经应承了福什维尔和维尔迪兰夫妇,要邀请他们前往。啊!他倒真希望她能有这么大的胆子!到时候给她来个回绝,给她来封报复性的回信,该是多么痛快!他都已经在挑选字眼,甚至高声念了出来,仿佛当真收到了她那封来信似的。
\par 这封信第二天果然来了。她说维尔迪兰夫妇和他们的朋友们表示有意去听瓦格纳作品的演出,而她平常经常在他们家受到接待,如果他肯给她送这笔钱的话,她就也将得到接待他们的乐趣。她只字没有提到他;不消说,有他们那些人在场就排除了他去的可能。
\par 头天晚上逐字逐句想好的那封可怕的回信(他可不敢指望这封信当真用得上),现在他却有派人把它给她送去的乐趣了。糟糕的是,凭她手头现有的钱,或者很容易就找来的钱,只要她想租,在拜罗伊特还是租得起房子的,虽然她不懂得巴赫和克拉比松\footnote{克拉比松(1808—1866):法国作曲家。}之间有什么区别。不过,凭她这点钱,她的生活就得省着点儿。他这回要是不送她几张一千法郎的钞票,她就没法每晚在她租的城堡里组织豪华的晚餐会,会后也许她还会心血来潮(可能以前还不曾有过),投入福什维尔的怀抱。反正这次见鬼的旅行,他斯万是决不出钱的!——啊!要是有办法阻止,那该多好!要是她在动身前崴了脚,要是能出高价买通送她上火车站的马车夫,把四十八小时以来在斯万眼中的这个背信弃义的女人,双眼里含着投向福什维尔的同谋的微笑的女人奥黛特送到一个地方关些日子,那该多好!
\par 可是她这副形象从来都不会保持很久;过了几天那闪亮狡猾的目光就失去了光辉和欺骗性,那对福什维尔说:“嗨!看他气成那个样子!”的可恶的奥黛特的形象开始淡化,开始消失。这时,另一个奥黛特的脸庞逐渐重新出现,在一片光明中缓缓地升起;这个奥黛特虽然也向福什维尔投去微笑,可只有在向斯万投去的微笑中才含有柔情;当她说,“可别太久了,当这位先生要我呆在他身边的时候,他是不大喜欢来客人的。啊!您要是像我那么了解他就好了!”的时候,不就是这样吗?当斯万对她体贴入微时,当在重要关头唯有他可以信赖而向他求教时,她的微笑不也就是这样吗?
\par 这时,他就会自问,他怎么能对这样一个奥黛特写那么一封侮辱性的信;毫无疑问,她是从来也不信他会写出这样一封信的,而这一封信就使他通过他的慷慨忠诚而在她的尊敬之情中占有的崇高的、唯一的地位上降了下来。她对他的爱就将不似往日了,正是因为他身上有福什维尔和任何别人所不具有的那些品质,所以她才爱他。正是由于这些品质,所以奥黛特才时常对他体贴入微;这些表现,当他心怀妒意时是不把它们当做怎么回事的,因为它们不是情欲冲动的表现,所代表的与其说是情爱倒不如说是柔情,可是当他的疑心逐渐消除(时常得力于阅读美术著作或者跟朋友谈话后的心平气和),使得他的激情不那么要求回报时,他就开始感到这些表现是何等可贵。
\par 在经过这番动摇以后,奥黛特自然回到了斯万的妒意把她一度拨开的那个位置,进入他觉得她动人的那个角度,他就在脑子里设想她是多么温情,眼睛里露出一副心甘情愿的神色,长得又是那么漂亮,他禁不住把他的双唇向她伸去,仿佛她当真在场,能够接受拥抱似的;而他对这迷人的善良的一瞥报之以感激之情,仿佛她刚才当真看了他一眼,仿佛刚才这一瞥并不是为了满足他的愿望而由他的想象力描绘出来的似的。
\par 他该给她造成了何等的痛苦!当然,他有充分的理由对她不满,但如果他不是那么爱她的话,这些理由还不足以使他对她不满到如此程度。他对别的一些女人不是也曾抱怨得厉害吗,而今天既然已经不再爱她们,对她们也就没有什么愤怒可言了,当她们找上门来时,不是照样可以乐于为她们效劳吗?如果有朝一日他对奥黛特采取这样不关痛痒的态度,那他就会理解,当初纯粹是出于醋意才使得他觉得她那想法如此恶劣,如此不可原谅,而那种想法骨子里还是十分自然,倒也显出一番好心,只是未免幼稚,无非是想在机会来临时能向维尔迪兰夫妇还一还礼,尽一尽地主之谊而已。
\par 他又从与爱情和醋意的观点相对立的观点来评断奥黛特,在想问题的时候力求公平,要考虑到种种可能性:他假设他从来没有爱过她,在他心目中跟任何别的女人都一样,她的生活并不因为他不在场而两样,并不是背着他,冲着他编织起来的。
\par 为什么要认为她在那边会跟福什维尔尝到她在他身边从未尝到过的令人陶醉的乐趣呢?这不完全是他的醋意凭空编造出来的吗?无论是在拜罗伊特也好,在巴黎也好,如果福什维尔想到他斯万的话,只能是把他看成在奥黛特的生活中占有重要地位的人,万一他们两人在她家相遇,他得为他斯万让路。福什维尔跟奥黛特之所以能不顾他的不乐意而在那里洋洋自得,那是由于他阻止不力所造成,而如果他对她的计划表示赞成的话(这计划原也是无可非议的),那她仿佛就是按他的旨意而去的,就会有被派去的感觉,被安顿在那里的感觉,而得到对那么经常接待她的人们予以回报的乐趣,也就得感谢斯万了。
\par 如果不让她生着他的气,没有跟他见面就走,如果给她把那笔钱送去,鼓励她作这次旅行,想法使旅行更加愉快,那她就会高高兴兴地,满怀感激之情跑向前来,而他也就会得到差不多一个星期来没有得到的跟她见面的那种欢乐,这是任何别的事物都无法替代的。只要斯万不带嫌恶之情去想象她,他就会在她的微笑中看到她善良的心,把她从任何别的男人手中夺回的愿望除了出之于爱情以外并不再含有醋意,那么这份爱情又恢复了对奥黛特的容貌身体给予他的种种感觉的爱好,恢复了对把她的一颦一笑、声调升降当做戏剧来欣赏,当做现象来探究这种乐趣的爱好。这种与众不同的乐趣结果在他身上产生了一种对奥黛特的需要,而这种需要也只有她亲自光临或者收到她的来信才能满足;这个需要跟斯万当年迈入崭新的生活阶段时那另一个需要几乎是同样不计功利,几乎是同样富于艺术色彩,而且是同样反常,那时斯万在度过多年枯燥沉闷的生活后忽然来了一个精神上充溢得泛滥的阶段,而他并不知道他的内心生活这种出乎意外的充实丰富从何而来,正如一个身体衰弱的人忽然逐渐健壮发胖,一时仿佛要走上彻底痊愈的道路一样——当年这个需要也是脱离外部现实世界而在他心中发展起来的,这就是欣赏音乐和了解音乐的需要。
\par 就这样,通过他的病痛的化学机理,他在以爱情制造了醋意之后,又开始制造对奥黛特的温情和怜悯了。奥黛特又恢复成为动人、善良的奥黛特。他为曾对她如此狠心而感到内疚。他希望她来到他的身边,而在她来之前先给她一些乐趣,好在见面时看到由感激之情塑造出来的她的面容和微笑。
\par 奥黛特拿得稳再过几天他准会前来请求和解,温柔驯从如前,所以也早就不怕使他不快,甚至不怕惹他一下,而且如果觉得时机合适也会拒绝赐予他最弥足珍贵的那种特殊优遇。
\par 也许她并不知道,当他跟她吵架的时候,当他对她说不再给她钱,要给她点苦头吃吃的时候,他并不是说着玩的。也许她更不知道,在另外一些场合,当他为了他俩的关系的长远利益,为了向她表明他可以离开她,破裂随时可能发生而决心在一段时间内不上她家去的时候,他也是真心实意的,如果说对她不见得是这样,至少对他自己是如此的。
\par 时常是事后一连几天,她不再给他增添什么新的烦恼;他也明知道最初几次见面不会得到多大的欢乐,也许倒会招来点不愉快的事情,搅乱他心底的宁静,所以写信给她,说他忙得不可开交,原定去看她的那些日子都不行了。可信刚发出,却接到她的来信,不约而同,正好也是请他推迟原定的约会。他心里不免纳闷,这倒是怎么回事?猜疑和痛苦揪住了他的心。心乱如麻,他再也不能遵守刚才在心境平静时许下的诺言,他赶忙跑到她家,要求在随后几天里天天去看她。即使不是她先给他来信,即使她回信说是同意几天不见面,他在家里也呆不住,非得去看她不可。这是因为,跟斯万的预料完全相反,奥黛特的同意使得他心里的盘算乱了套。有些人占有一种东西,为了要知道如果他一时失去了这样东西,有什么情况可能发生,他就把这样东西从他脑子里排除出去,让脑子里的其他东西都保持原样。然而少了一样东西并不仅仅意味着这样东西的不存在,并不只是一个部分的缺乏,这是整个其余部分的大动乱,这是一个无法从旧态中预见的一个新的状态。
\par 另外一些时候则与此相反:奥黛特正准备出外旅行,他在找了一个借口跟她口角一番以后,决心在她回来以前,既不给她写信,也不去看她,这就使得一次暂别看来像是一场了不起的不和(他在期待从中得到好处,而她也许以为这是一场无可挽救的不和),而这次暂别的大部分时间由于奥黛特外出旅行而不可避免,他不过是促使它早开始几天罢了。他都已经在设想奥黛特怎样为既不见他人又不见他信而焦急不安,苦恼万分,而奥黛特的这个形象平息着他的妒意,使他更容易习惯于不跟她见面了。他同意的这次暂别长达三周之久,脑子里一出现跟奥黛特重见这个念头就被他打将下去,然而也有时候,在他思想深处也为能在她回来时见到她而感到高兴,不过他也多少带点焦急地自问是否自愿把这如此易于熬过的禁欲时期更延长些日子。这段时期迄今还只过了三天,他以前也时常有不见奥黛特的面达三天以上,但都不像现在这样是事先安排下来的。然而有时心里的小小不痛快或者身上的小小不舒服促使他把现在这个时刻看成是例外的、出轨的时刻,是通权达变的精神容许他去接受一种乐趣带来的安抚,容许他给意志力放假(直至有必要恢复)的时刻;这种不痛快或者不舒服使意志力停止活动,不再起什么强制作用;有时他忽然想起有点什么事情忘了问奥黛特,例如她是否已经想好,她的马车要漆成什么颜色,或者买的股票是要普通股还是优先股(有机会向她表示一下他不见她的面也能活下去固然不错,然而如果日后马车要重漆一次,股票没有股息,那就糟了),这时候去看她这个念头就跟刚撒手的橡皮筋或者从刚打开盖的气压机中出来的空气一样,猛一下从远处闯进现在这个领域,来到立即有可能实现的领域。
\par 去看奥黛特这个念头又回到心间,不再遇到什么阻力,而这念头也变得如此不可抗拒,以至斯万觉得一天又一天地挨过跟奥黛特分离的十五天还比较容易,而等他的车夫把车套上,把他送到她家,要在焦急不安和欢欣雀跃中度过的那十分钟反倒十分难熬;在这段时间里,为了向她表示他的温情,他千万次地重温同她重新见面这个念头——正当他以为她还远在他方的时候,她却突然归来,现在回到他的心间。这是因为,这个念头现在找不着想方设法抵制去看奥黛特这个念头以制造障碍这样一种愿望;这种愿望在斯万身上已经不复存在,因为自从他向自己证明(至少他自己是这样想的),他是如此轻而易举就能抵制这个念头以来,他就觉得把暂别的尝试推迟进行并没有什么不便之处,反正他现在觉得只要他愿意,就有把握来实施了。同样也是因为,去看奥黛特这个念头现在重新出现在他心头时总带有新意,带有诱惑力,带有尖锐性——这三者以前都是被习惯磨平了的,现在则通过这不是三天而是十五天的禁绝(一次禁绝的期限不是按它实际已经延续了多久,而应该按预定的期限来计算的)而重新获得力量;同时从不付太多代价就牺牲了的期待中的乐趣当中却产生了他无法抵御的意想不到的幸福。最后,去看奥黛特这个念头现在重新出现在他心头时总伴随着斯万要知道当奥黛特在得不到他的音信时想些什么、做些什么的渴望心情,以至他行将发现的是一个几乎陌生的奥黛特的令人神魂颠倒的启示。
\par 而她呢,她早就认为他拒绝给钱不过是个假动作,来问车漆什么颜色,买哪样的股票都不过是个借口,她无需把他经历的这些情绪的发作的各个阶段从头到尾回顾一下;根据她对这些的认识,她无需了解它的来龙去脉,只相信她早就知道的那一点,也就是那必然的、万无一失、从来不变的结局。如果从斯万的观点来看,这种看法是不完全的——虽然也许可能是深刻的。斯万显然认为他不被奥黛特所理解,这就好比是一个有吗啡瘾的人深信他是正要摆脱他的顽固恶习时由于外界因素而受阻,或者是一个肺结核患者深信他正要最终痊愈时突然遭到意外的不适,全都感到自己不被医生所理解,认为医生对那些所谓偶然事件重视不足,把它们都看成恶习或病状用来掩盖自身的东西,而当病人自己陶醉于即将恢复正常或者即将得到痊愈的美梦时,他们的恶习或病状实际却继续无可挽救地压在他们头上。事实上,斯万的爱情已经到了这般地步,内科大夫和最大胆的外科医生(在某些疾病方面)都会自问,除掉这样一个病人的恶习或者根除他的疾病是否还合情合理,甚至是否还有可能。
\par 确实,斯万对他这份爱情的深广并没有直接的意识。当他想猜度猜度的时候,他时常觉得这份爱情仿佛已经衰退了,几乎已经化为乌有;譬如说,在他爱上奥黛特以前,他对她那富有表情的面部线条,她那并不鲜艳的脸色并不怎么喜欢,几乎可说是有点厌恶,现在有些日子也会发生这种情况。“当真是有了进步,”他在第二天心里就会这么想,“当我仔细捉摸的时候,我发现昨晚在她床上几乎感觉不到任何乐趣:也是怪,我总觉得她长得丑。”的确,这也是实话,这是因为他的爱已经大大超出了肉欲的领域。奥黛特的身体已经不占很多的地位。当他抬头看到桌子上奥黛特的相片时,或者当她来他家看他时,他很难把这照相纸上的或者那有血有肉的面容跟在他心头的那份难以平静的痛苦的不安心情之间划上等号。他几乎是不胜诧异地心想:“是她!”就像是有人突然把我们身上的某种疾病拿到体外来给我们看,而我们觉得它跟我们所闹的那种病并不相像一样。他试图弄清楚这到底是什么东西;那是有点像爱情、像死亡的东西,而不是跟疾病的概念依稀相似的东西;那是我们经常对之表示怀疑,经常予以深究,唯恐掌握不了它的实质的东西——那是人的品格之谜。而斯万的爱情这个病已经大大扩散,已经跟他的一切习惯、一切行动,跟他的思想、健康、睡眠、生活,甚至是身后的遗愿如此紧密相连,它已经跟他合而为一,不可能从他身上剥离而不把他自身整个毁坏,用句外科大夫的话,他的爱情已经无法再动手术了。
\par 由于有了这份爱情,斯万过去的那些兴趣已经衰退到这般地步,以至当他偶尔回到上流社会时(心想他那些社会关系就跟奥黛特不能确切知道其价值的钻石的精美托座一样,可以在她的心目中抬高他的身价,而如果这些社会关系没有因为那份爱情而贬值的话,这种想法也许是对的:原来在他心中,这份爱情把任何与之有关的事物的价值都贬低了,因为它把它们都说得没有那么可贵),他所感到的除了身处她所不认识的地方和不认识的人中间的那种忧伤外,还有在阅读或欣赏某些表现有闲阶级的消遣的小说或画幅时可能体味到的那种超然的乐趣:譬如他在家里就喜欢在他最喜爱的作家之一圣西门的作品中读与凡尔赛宫日常生活、德·曼特农夫人\footnote{德·曼特农夫人(1635—1687):法国作家斯卡龙之妻,孀居后,进王宫负责路易十四子女的教育,于1684年与路易十四秘密结婚,对王国政治颇有影响,著有《书信集》。}的菜单以及吕里\footnote{吕里(1632—1719):法国作曲家,法国歌剧的奠基人。}谨慎的吝啬与大摆排场时同样的兴趣来检查他家中日常生活安排是否顺当,他自己的衣着和仆役们的号衣是否漂亮,他家的资金投放得是否妥善。斯万过去那些兴趣的衰退也不是绝对的,而他之所以要体味体味这新的乐趣,那是为了能以一时躲避到他自己心中还没有被他的爱情、他的忧伤触及的那些屈指可数的地方。在这一点上,我的姨姥姥所说的那个“小斯万”的性格(跟夏尔·斯万的更有个人特色的性格不同)正是他现在最乐于具备的性格。有一天,帕尔马公主过生日(她能弄来盛大的节日欢庆活动的入场券,所以间接地对奥黛特也有用处),他想给她送点水果,可不太清楚该上哪里去订,就托他母亲的一个表妹去办理。这位姨妈写信告诉他,她给他买的水果不是在一个地方买的,葡萄购自克拉波特水果店(这是这一家的名牌商品),草莓和梨分别采自饶雷和谢费水果店(那里的最好),“所有果子都经我一一检验。”果然,公主在谢函中说草莓是多么的香,梨是多么的可口。特别是“所有的果子都经我一一检查”这句话给了他莫大的安慰,把他的心带到了他很少光顾的领域——在富有的相当有地位的资产阶级家庭中,对“常用地址”的了解以及上商店订购商品这套知识是世代相传的,他作为这样一个家庭的继承人,这套知识是随时会为他效劳的。
\par 的确,他早已忘了他是那个“小斯万”了,所以当他一时间内重新成为这个“小斯万”时,竟感觉到这个乐趣比他平常感到的并也早已无动于衷的那些乐趣都要强烈;资产者(对他们来说他从来都是那个“小斯万”)的殷勤要比贵族的亲切稍逊一筹,然而却更讨人喜欢,因为资产者的殷勤跟对人的尊敬之情是结合在一起的,所以无论哪位亲王殿下给他来的信,请他参加的什么招待会,在斯万心目中都不如他父母亲的老朋友请他担任证婚人或者仅仅参加婚礼的邀请信更弥足珍贵;他父母亲的这些老朋友,有的一直还跟他见面,譬如我的外祖父头年还曾请他参加我母亲的婚礼;另外有些只跟他有一面之交,但对已故斯万先生这位可尊敬的继承人还是彬彬有礼的。
\par 但由于他跟上流社会人士年代久远的亲密相处,他们在一定程度上也是他的住处、仆人和家庭的一部分。当他想起他那些显赫的朋友时,他觉得他们也跟上代传给他的美好的地产、精致的银餐具、好看的桌布一样,都是一种依靠,一种提供舒适的设备。当他想到,万一他在家里忽然病倒时,他的仆人前去求援的必然是夏特勒公爵、罗伊斯亲王、卢森堡公爵和夏吕斯男爵\footnote{都是斯万的朋友,其中夏特勒公爵(1840—1910),是法国国王路易·菲利浦之孙,巴黎伯爵之弟。},想到这里,他就像我们家的弗朗索瓦丝知道她来日将用绣了她自己的姓名,没有打过补丁的细布(或者缝补得如此精巧,显示出那双巧手的高超技艺)裹了入殓时同样感到安慰——这是她的心神往已久的裹尸布,虽不值钱,但已经足够体面,可以心满意足了。尤其是,在他所有与奥黛特有关的行动和思想当中,斯万总有一个没有明确说出来的占主导地位的想法,那就是认为他自己在她的心目中,也许比任何人,比维尔迪兰家最讨厌的忠实信徒都要亲些,然而并不是她最乐于相见的一个——当他想到那么一群人认为他是鉴赏趣味最高的一个人,是他们竭力要拉拢,为见不到他而感到遗憾的一个人时,他就相信这世上是另有一种更幸福的生活的,几乎已经感到尝试尝试这种生活的欲望,就如同一个卧床多月,饮食受到严格控制的病人,从报上看到正式宴会的菜单或者到西西里岛的旅游广告时一样跃跃欲试。
\par 如果说他是为了不去拜访他在上流社会中的朋友们而为自己辩解的话,他在奥黛特面前竭力为自己编造种种理由却是千方百计为了要去看她。而且他还得为此而掏腰包(到了月底时常还得想一想,是否太打扰她,去看她的次数是否太多了,给她四千法郎是否太少),每次还得找个借口,带点礼物送去,想出点她要听的消息,或者去找德·夏吕斯先生(有回在上她家去时在半路上碰到,硬要斯万陪着他去)。要是没有任何借口的话,他就请德·夏吕斯先生上她家去,让他跟她在漫谈中,说是突然想起有话要跟斯万说,请她打发人去把他马上请来她家;大多数时候是斯万在家里白等,德·夏吕斯先生晚上来跟他说,他这一计没有成功。结果呢,她现在时常离开巴黎,即使在巴黎时也很少跟他见面,而当年爱他的时候,却老说“我总是有工夫的”,或者说“别人的闲言碎语我才不管呢”,现在可好,每当他想跟她见面的时候,她要么提什么人言可畏,要么推说有事。当他说到要同她去看什么义演,参加美术预展,观看剧本的首场演出时,她就说他想把他们之间的关系暴露在光天化日之下,说他把她当做姑娘家看待。事情发展到了这等地步,为了免于哪儿也找不着她,斯万有天就上贝尔夏斯街我外叔祖父阿道夫住的那套套房去找他,请他对奥黛特施加影响;他知道她是认识并且很喜欢我外叔祖父的,他从前也是她的朋友。当她在斯万面前谈起我外叔祖父时,她总是像吟诗一样说话:“啊!他哪,他可不跟你一样,他对我的友情是多么纯洁、伟大、高尚!他可不会这么小看我,想跟我在随便什么公共场所一起露面。”斯万感到有点为难,不知道在我外叔祖父跟前谈奥黛特时该把调子定得多高。他先说她人品是如何优秀,她的人情味是如何超出常人,她的品德是如何非言语所能形容,又如何非任何概念所能概括。“我想跟您谈一谈。奥黛特是怎样一个可爱的人,怎样一个高出于所有女人的人,怎样一个天使,您是知道的。您也知道巴黎的生活是怎么回事。您跟我所认识的那个奥黛特,并不是每一个人都认识的。所以嘛,有些人就觉得我在扮演一个可笑的角色;她都不答应我在外边,在剧场碰见她。她对您是那么信任,请您在她面前为我说几句话,告诉她别以为我在街上给她打个招呼就会给她带来什么灾难。”
\par 我外叔祖父劝斯万过些日子再去看奥黛特,她只会因此而更加爱他,又劝奥黛特,斯万爱在哪儿跟她见面,就让他在哪儿跟她见面。几天以后,奥黛特对斯万说,她大失所望,原来我外叔祖父跟所有的男人没有什么两样:他不久前想对她强行非礼。斯万一听就要去找我外叔祖父算账,奥黛特把他劝阻了,可是当他碰见我外叔祖父时还是拒绝跟他握手。斯万原希望,假如他能再次看到我外叔祖父,跟他私下谈谈,弄清他跟她当年在尼斯时的生活有关的一些流言蜚语,因此就更加后悔跟我外叔祖父阿道夫闹了不和。我外叔祖父当年是常在尼斯过冬的。斯万心想:他也许正是在那里认识奥黛特的。有人在他面前漏了点话锋,是关于某个人的,这个人可能曾经是奥黛特的情人,这就使得斯万大为震惊。有些事情,在他知道以前,听起来可能觉得再可怕也不过,再难以置信也不过,一旦知道了,就永远跟他的愁思结上不解之缘,他承认它们,而且不再能相信它们没有存在过。只不过每一件事情都把他对他情妇的看法作出一点修正,从此难以改变。有一阵子,他都认为,以前他没有料到奥黛特会那么轻佻,现在她的轻佻却几乎尽人皆知,而当她在巴登和尼斯度过的几个月当中,她的风流是出了名的。他想跟几个纨袴子弟接近接近,向他们打听打听;可他们知道他认识奥黛特;而且他自己也担心这会使他们重新念叨她,又来缠她。直到那时之前,一切与巴登或者尼斯这两个五方杂处的城市生活有关的事情在他心目中比什么都无聊乏味,可忽然听说奥黛特从前曾经在这两个游乐城市过花天酒地的生活之后,他却怎么也闹不清那仅仅是为了满足她对金钱的需要呢(现在有了他,这个问题就不再存在了),还是只因为一时心血来潮(这可还会出现的)。现在他带着无能为力、莫名其妙的强烈的不安心情,俯身下视吞没了“七年任期”\footnote{指麦克马洪担任总统的七年期间(1873—1879)。}最初几年的那个无底洞,在那些年代中,人们在尼斯的英国人大道上过冬,在巴登的椴树荫下度夏,而他却觉得这些年月是个虽然痛苦然而辉煌的深渊——诗人是会这样说的:他会把当年蔚蓝海岸报纸上的琐闻回顾一番,只要它们能帮助他对奥黛特的微笑或者眼神——依然还是如此善良朴实——有所了解,他会比他作为美学家,为了深入理解波堤切利的《春》、《美丽的伐娜》、《维纳斯的诞生》而研究十五世纪佛罗伦萨的资料时还要热心。他时常一言不发地瞧着她,陷入沉思之中,这时她就对他说:“你怎么愁眉苦脸的!”不久前,他还把她看成是个很好的人,跟他认识的最好的女人一样的一个女人,现在却想她是一个由情人供养的女人;与此相反,有时他先看到的是跟那些专门吃喝玩乐的纨氾子弟,跟那些男不男女不女的家伙厮混在一起的奥黛特·德·克雷西,然后他又看到了这张表情如此温和的脸,想到了如此善良的性格。他心想:“就算尼斯所有的人都认得奥黛特·德·克雷西吧,又有什么了不起?那些流言蜚语都是别人编出来的;”他心想那种传说就算是确有其事吧,也是外在于奥黛特的东西,并不像怙恶不悛的本性那样是内在的东西;终于被勾引干了坏事的那个人,那是一个长着一双漂亮的眼睛,有着一颗对别人的痛苦充满怜悯之情的心,还有一个他曾搂在怀里任意摆弄的顺从的身子的女人;假如他能使自己成为她须臾不可缺的人的话,有朝一日他就可以把她整个身心完全占有。她现在就在那里,时有倦容,脸上这会儿倒显不出她在全神贯注于折磨着斯万,又叫人捉摸不透的那些事情;她用双手把头发往后一捋,额头和脸面都显得更宽了一些;就在这时候,一个平淡无奇的念头,一个善良的情感突然像一道金光一样从她眼里迸发出来,任何人在休息或沉思一阵以后都会这样的。像笼罩着云霞的灰色田野在日落时分突然开朗一样,她的脸也顿时露出喜色。奥黛特这时的内心生活,她憧憬的那个未来,斯万是但愿能够与她共享的;看来这是没有受到任何倒霉的骚动的影响。这样的时刻是越来越难得出现了,可每次出现都不无裨益。斯万通过他的记忆,把这些断片连缀起来,删去两次之间的间隔时间,铸就一个善良的、宁静的奥黛特的金像;为了这个奥黛特,他后来作出了牺牲,这是另一个奥黛特所没有得到的(我们在这部作品的第二卷里将要谈到)。这样的时刻可真是难得了,连见她面的机会都不多了!就是他们晚间的约会,她也总要到最后一分钟才说出她能不能答应,因为她认为他反正总是有空而她得拿准了除他以外没有别人提出要来才行。她总推说她得等待一个对她至关重要的回音,而即使当她派人叫斯万来了,晚间的聚会也已开始,只要有朋友请奥黛特陪他们上剧场或者去吃夜宵,她也总是不胜雀跃,匆匆忙忙地着装。她把衣服一件一件地穿上,每一个动作都加快斯万离开她、并使她一溜烟地跑开的那个时刻到来;等到衣服穿好,她最后一次把聚精会神、熠熠生辉的目光投向镜子,在嘴唇上抹点口红,在前额上做个发髻,然后叫人把那件缀了金流苏的天蓝色晚大氅拿来。斯万满面愁容,她都无法抑制她的不耐烦的心情,说道:“我一直陪你陪到最后一分钟,敢情你就是这样来谢我!我想我对你够好的了。下次我可再也不那么傻了!”有时他冒着惹她生气的危险,决心要弄明白她上哪儿去,他甚至幻想跟福什维尔结盟,心想也许他能为他提供情况。再说,当他知道她是跟哪些人在一起度过晚间时,那就不大可能会在他所有的朋友当中找不到知道(哪怕是间接地知道)她是跟哪个男人出去,同时探得某些情况的人。当他给某个朋友写信,请他设法弄清某一点时,他就如释重负,不必再向自己一提再提那些得不到答案的问题,而把四出打听之劳托付给别人。其实当斯万多了解一点情况的时候,他也并不就舒坦些。知道一件事情并不等于阻止一件事情发生,不过我们所知道的事情,我们总可以把它们掌握住,虽不是掌握在手中,至少是掌握在脑子里,在那里,我们就可以任意予以支配,这种情况给了我们一个幻觉,仿佛对它们能有所为。每当德·夏吕斯先生跟奥黛特在一起的时候,斯万就高兴。他知道,在德·夏吕斯先生和她之间是不会发生什么事情的,而德·夏吕斯先生之所以跟她一起出去,那是出于他对斯万的友情,他也会把奥黛特干了些什么原原本本地告诉他。有时她斩钉截铁地告诉斯万,说她某一晚没有可能跟他会面,看她那样子是非出去不可的,斯万就想尽办法让德·夏吕斯先生腾出时间来陪她。到了第二天,他不好意思向德·夏吕斯先生提很多问题,只是假装没有太听明白他的回答,硬要他再说一遍,在每句答话后他感到越来越宽慰,因为他知道奥黛特一晚参加的都是无伤大雅的游乐。“小梅梅,我可不太明白……你们不是一出她家就奔格雷凡蜡人馆的。你们先上别的地方去了。没有?哟!那就怪了!小梅梅,您真把我逗死了。她接着又上‘黑猫’,真是个怪念头,这主意是她出的吗?不?是您。那就怪了。这倒果然不是个坏主意,她在那里准有许多熟人?不?她跟谁也没有讲话?这就神了。你们俩就这么呆在那里?周围一个人也没有?这景象我倒能想象得出来。您真好,我的小梅梅,我真喜欢您。”斯万感到松了一口气。他有时心不在焉地跟一些不知道他跟她那档子事的朋友聊天,偶尔听到像“我昨天看见德·克雷西夫人来着,跟一位我不认识的先生”这样的句子;这样的句子马上就在斯万的心里化为固态,硬化成为水垢,划破他的心,从此不再离开,而像“她谁也不认识,跟谁也没有讲话”这样的语句在他心里又是流动得何等顺利,何等润滑,何等通畅,又是何等易于吸收!不过再过一会儿,他又心想,奥黛特大概觉得他挺乏味,不然怎么宁愿去找那样的乐趣也不愿意跟他在一起呢?那些乐趣没有什么了不起,这固然使他安了心,却也使他痛苦,仿佛是被人出卖了似的。
\par 即使他无法知道她上哪儿去了,这也足以使他心中的焦虑平静下来;对这种焦虑,奥黛特的在场,在她身边的温馨之感是唯一的特效药(这种特效药久而久之加重了病痛,然而至少暂时可以镇一镇痛);只要奥黛特同意他呆在她家里等她回来,也就够了;在这宁静的等待的时刻里,另外一些由于某种魅力、某种魔法而在他心目中显得与众不同的时刻会来与之交融在一起。可是她却不同意,他只好回自己家去,在路上强制自己考虑种种方案,不去想奥黛特,甚至在宽衣的时候也在咀嚼着欢快的想法;他满怀明天能看到什么杰作的希望上了床熄了灯;可是一等他为了准备睡觉而中止对自己感情的控制(这种自我控制早已习惯成自然,连他自己也意识不到了),他就感到身上一阵寒战,不由得哽咽起来。他也不想问个为什么,擦擦眼睛,含笑对自己说:“敢情好,我都得了神经病了!”然后他还是不禁怀着极度的厌倦想到明天还得重新开始设法打听奥黛特到底干了些什么,设法运用一切影响,力求跟她见面。这种无休无止、毫无变化、毫无结果的活动,对他来说是一种如此严酷的必需,以至有一天,当他看到腹部长了一个肿块的时候,他都为这也许是个致命的肿瘤而高兴万分,心想从此就可以不必再做任何事情,听凭这疾病的支配,成为它手中玩弄的对象一直到那为时已经不远的末日。在这个时期,他虽然没有明确承认,却时常但愿死期早临,而这与其是为了摆脱这深刻的痛苦,倒不如说是为了摆脱他所作的努力的单调乏味。
\par 然而他还是希望能活到他不再爱她的时候,那时她就没有任何理由向他撒谎,他也就终于可以知道那天他在下午去看她的时候,她是否正和福什维尔睡觉。时常在一连几天当中,对她爱着另外一个男人的怀疑使他不再向自己提出那跟福什维尔有关的这个问题,把这问题几乎看得是无关紧要,这就像是老毛病呈现出新的形式,仿佛使得我们暂时摆脱了旧的病状。甚至也有些日子,他不为任何怀疑所苦,自以为已经痊愈,然而到了第二天早上醒来时,他又在同一部位感到同样的痛苦,而这种感觉在头天白天仿佛已经在各种不同的印象的急流中冲淡了。其实这个痛苦的位置并没有转移,正是这个剧烈的痛苦把斯万弄醒了。
\par 每天萦绕在他脑际的这些如此重大的事情(他见多识广,知道那些事情无非是寻欢作乐罢了),奥黛特却从不提供任何情况,他也不能经久不息地老在想象,想着想着脑子也就空转了;这时他用手指揉揉疲乏的眼睑,就好像是擦擦夹鼻眼镜的镜片一样,然后彻底停止思想。在这一片茫茫之上却不时浮现出一些事情,隐隐约约地通过奥黛特而与她的一些远亲或者昔日的朋友有关,这些人她时常提起,说是由于接待他们而不能见他的;在斯万心目中,这些人似乎构成奥黛特的生活的固定的、不可或缺的框架。由于她不时对他说起“我跟我的女友上跑马场的日子”时的特殊声调,所以当他有病,他想到“奥黛特也许会到我家来”时,忽然想起那天正好就是那个日子,他就心想:“啊!不行,这就不必请她来了,我怎么早没有想到,今天是她跟女友上跑马场的日子。还是等待时机提点能办得到的事情吧;提出一些不能被接受,肯定要遭回绝的事情,会有什么好处?”落到奥黛特头上而斯万不得不依从的那个上跑马场去的义务,在他看来不仅是不可抗拒,而且它的必要性仿佛使得所有跟它直接间接有关的事情都成为合情合理又合法的了。如果有人在街上跟奥黛特打了招呼,引起他的妒意;如果她回答这个人的问题时把这位陌生人跟她对他常谈的两三样重要义务联系起来,譬如她说“这位先生那天跟陪我上跑马场的那个朋友坐在同一个包厢”时,这个解释就消除了斯万的怀疑,认为奥黛特那位女友除了奥黛特以外还邀了别的客人是不可避免的事情,却从来也没想这些客人是怎么样的人,而且即使想了也是想不出来的。啊!他是多么想认识把奥黛特带到跑马场去的那位女友,多么希望她也能把他带去!他是多么愿意把他所有的亲友来换一个能常见着奥黛特的人,哪怕她是一个修指甲的也好,是个店员也好!他愿为她们花费比为王后们还要多的钱。她们身上也体现了奥黛特的一部分生活,难道这不正是对他的痛苦的镇痛剂吗?要是能在那些由于兴趣一致或者由于同样纯朴的天性而跟奥黛特保持友好往来的小人物家中愉快地度日,那该多好!他是多么希望能从此搬到奥黛特从不带他去的那所虽然肮脏然而值得羡慕的房子的六楼长住,他情愿在那里假装是那个歇手不干的小女裁缝的情人,从此每天都能接待奥黛特来访!在这些平民区里,生活虽然简朴贫困,然而甘美、宁静而幸福,他真愿意永远住下去!


\paragraph*{4}

\par 还有时候,她在碰到斯万以后又有一个他所不认识的男人向她走来,这时他可以在奥黛特的脸上看到那天他去看她而福什维尔也在场时她脸上那种愁容。不过这种情况是罕见的,因为在不管有什么事情要做也不管旁人的闲言碎语而跟他会面的日子里,奥黛特主导的情绪是自信和泰然自若:想当年她刚认识他的时候,无论是在他身边还是不在他身边而给他写信的时候,她总是那么怯生生的(“我的朋友,我的手抖得这么厉害,连字都写不了了”——她至少是这样说的,而且这种感情总有一点是真的,才有夸大的基础)。那时候她是喜欢斯万的。我们颤抖,不是为了自己,就是为了所爱的人。当我们的幸福不再掌握在他们手里的时候,我们对他们就能泰然处之,就能从容自如,就能无所畏惧。当她现在跟他说话,给他写信的时候,就不再用那些制造他是属于她的那种幻想的字眼,不再在谈到他的时候拼命找机会用“我的”等字样,例如什么“您是我的一切,这是我们的友谊的香水,我把它留下”诸如此类的话;她也不再跟他谈起什么前途,谈起什么死亡,说得好像他们不但同命运,还将要同生死似的。想当年,他无论说什么,她总是赞赏地答道:“您,您这个人就是跟常人不一样嘛。”她瞧着他那稍微有点秃顶的长脑袋(那些知道斯万的成就的人心想:“要说漂亮,他算不上漂亮,可是要说帅,你瞧他那头发,那单片眼镜,那微笑!”),急于要知道他是怎样一个人而不是力求当上他的情妇,她说:“我要是能知道这脑袋瓜里想的是什么,那该多好!”现在啊,不管斯万说什么,她答话时总有时带点气恼,有时则显出一副宽宏大量的样子:“啊,你这个人总是跟别人不一样!”现在她瞧着他那操心操得稍显苍老的脸(现在所有的人都是读了说明书才发现一部交响音乐作品的主旨,知道孩子的父母是何许人才发现他哪些地方像他父母,凭着这么一点本领,说:“要说丑,他并不算丑,可他就是那么可笑,你瞧他那单片眼镜,那头发,那微笑!”凭着他们的想象,仅仅隔了几个月时间,就画出了一条分界线,一边是情人的面貌,一边是王八的嘴脸),说:“这脑袋瓜里想的是什么,我要是能够改变,叫它合情合理,那该多好!”
\par 斯万依然还是相信他所希望的事情是会实现的,奥黛特对他的举止虽然也引起他的怀疑,但他还是热切地对她说:
\par “如果你这么想,你就能办得到。”
\par 他试图向她解释,除她以外的别的女人都求之不得地献身于安慰他、控制他、督促他这个崇高的使命,而且指出,在她们手里,这个崇高的使命对他来说只不过是对他的自由的既不慎重又难以忍受的冒犯。他心想:“要是她不多少有点爱我的话,她是不会存改造我的愿望的。要改造我,她就必须跟我有更多的往来。”就这样,他就把她对他的责备看成是对他感兴趣,也许还是爱他的表现;的确,她现在对他的责备越来越少了,以至他都只好把她不让他干这干那看成是这样的表现。有一天,她对他说她不喜欢他的马车夫,说他挑拨斯万找她的碴,至少他在执行斯万的命令时不够严格,不够恭敬。她感觉到他希望从她嘴里听到“下回别让他送你上我家了”这样的话,正如他希望受她一吻一样。那天她情绪好,所以终于对他说了;他很感动。到了晚上,当他同德·夏吕斯聊天的时候(在他面前谈她可以毫无顾忌,而他即使是跟不认识她的人所谈的话,也都或多或少地与她有关),他对夏吕斯说:
\par “我想她还是爱我的;她对我那么好,对我所做的任何事情都是不会漠不关心的。”
\par 如果当他跟一个要在半道下车的朋友一起登上他的马车时,那位朋友说:“怎么回事?怎么不是洛雷丹诺驾车?”斯万在回答的时候又是高兴,又有点惨然:
\par “嗨!乖乖!跟你说吧,当我上拉彼鲁兹街的时候,我是不让洛雷丹诺驾车的。奥黛特不喜欢我带洛雷丹诺去,她觉得他跟我不般配。唉!女人嘛,你有什么办法?我知道她会很不高兴的。好吧!我就只好带雷米了,要不然可就好看了!”
\par 奥黛特现在对斯万这种漠不关心、冷冷冰冰,甚至急躁易怒的态度,斯万自然感到痛苦;然而他并不知道他痛苦到什么程度,因为奥黛特对他冷淡是一天一天、一步一步发展起来的,他只是在把她今天是怎样跟她开始又是怎样加以对比时才能测出这变化是何等之深。而这变化就是他那日日夜夜在折磨着他的深刻而隐秘的创伤;当他一感到他的思想就要触及这个创伤时,他就赶紧把它扭转方向,免得过分痛苦。他只能泛泛地说“从前有个时期奥黛特是比现在更爱我的”,可是他从来想不出那个时候的一个具体图景。在他的工作室里有一个五斗柜,他尽量不去看它,出出进进宁可拐一个弯,因为在一只抽屉里藏着他第一次送她回家时她送给他的那枝菊花,还有写着“您为什么不连您的心也丢在这里呢?如果是这样的话,我是不会让您收回去的”,以及“不管是在白天还是晚上几点钟,只要您需要我,随时给我打个招呼,我就奉陪”这些字样的信,同样,在他心里也有一个地方是他不让他的思想接近的,在必要时就来一大段拐弯抹角的道理来避免他的思想经过这个地方:这个地方就是对往日幸福日子的回忆。
\par 可是有天晚上,当他到上流社会中去的时候,他这个煞费苦心的谨慎却破产了。
\par 那是在圣德费尔特侯爵夫人家中,是那一年她请人去听将在她举办的义演上出场的音乐家演奏的一系列音乐会的最后一次。斯万本想以前各次全都去参加的,却一直下不了决心,直到穿衣准备去参加最后那次时,正好夏吕斯男爵来访,男爵说如果他陪他前往能使他不至过分厌倦,过分闷闷不乐的话,就愿意陪他上侯爵夫人家去一遭。斯万却说:
\par “跟您在一起,我多么高兴,您是想象不出来的。然而最使我高兴的还是您能上奥黛特家去一趟。您知道,您对她是能产生崇高的影响的。
\par 我想她今晚在上那位歇业的女裁缝家去以前是不会外出的,而您要是能陪她去,她是会高兴的。无论如何,您在这以前会在她家找着她,想法让她高兴,好好说服她。您要是能为明天安排点她喜欢的活动,咱们三个人一起参加,那就太好了。同时也设法探一探口风,看今年夏天能干点什么,看她有什么想法,想不想咱们三个人一起乘船旅行一番什么的。至于今晚嘛,我不指望能见到她;如果她要我去,或者您能找到什么借口,您就打发人上圣德费尔特侯爵夫人家给我送个信,如果过了十二点,那就送到我家。谢谢您为我费心,您知道我是多么爱您。”
\par 男爵答应在把斯万送到圣德费尔特府门口以后就去看奥黛特。到了侯爵夫人的家,斯万心想有夏吕斯在拉彼鲁兹街陪着奥黛特,也就放心了,而对一切与奥黛特无关的东西,特别是对上流社会社交生活中的那些东西则索然乏味,还带着点儿忧伤,这倒使得这些东西具有了我们不再孜孜以求的事物在它们本来面目下出现时的魅力。一下车,迎面就是女主人要在喜庆之日给客人看到的她们家生活概貌的第一场景,在这里,她们竭力保持服装与布景的原样,斯万看到巴尔扎克笔下的“老虎”\footnote{王政复辟时期,站在马车座位后面专司开闭车门的年轻侍从。}的后裔们,这些穿着制服的侍者,这些通常跟随主人外出散步的跟班,一个个穿靴戴帽,有的呆在公馆门前的大街上,有的呆在马厩跟前,就像排列在花圃门口的花匠一样,倒也挺有意思。他一向喜欢把活人跟博物馆里的肖像相比,现在这种比较更加经常,而且随时随地都在进行了:现在他已经脱离上流社会生活,这上流社会生活在他心头就仿佛成了一系列的组画。当他过去混迹上流社会时,他穿着大氅走进门厅,脱去大氅穿着燕尾服出去,从来也不知道在这里发生什么事情,在这里呆的两分钟时间里脑子里或者还想着刚离开的那个晚会,或者想的是马上就要进去参加的那个庆典,今天则是第一次注意到那一群东零西散、服装华丽而无所事事,专门坐在板凳或衣柜上打盹儿的侍从怎样被他这位姗姗来迟的客人惊醒,挺起他们高贵的猎兔狗般敏捷的身躯,站立起来,把他团团围住。
\par 其中有一个长相特别凶狠,很像文艺复兴时期某些画有酷刑的场面当中的执刑人,他毫不容情地向斯万走来,接住他的衣物。他的眼神虽似钢铁般坚硬无情,棉纱手套却是那样柔和,当他走近斯万的时候,他仿佛是对斯万其人表现出蔑视而对他的礼帽则颇为尊敬。他小心翼翼地把礼帽接住,动作准确细致,优雅动人,然后把礼帽递给他的一个下手,这是一个新手,腼腆胆怯,两眼滴溜溜的,射出愤怒的光焰,像刚被关进笼子的野兽那样惴惴不安。
\par 几步之外,一个穿着号衣的彪形大汉站在那儿出神,像尊塑像那样无所事事,动也不动,仿佛是曼坦那\footnote{曼坦那(1431—1506):意大利文艺复兴时期巴杜亚派画家。}最嘈杂喧闹的画幅当中那个纯粹是点缀用的武士一样,正当别人冲向前去,在他身旁忙于厮杀的时候,他却倚在盾牌上若有所思;这个大汉超脱于在斯万身边忙忙碌碌的那群伙伴之外,仿佛他对这个场景不感兴趣,只是以他凶狠的蓝眼睛漫不经心地瞧着,似乎那是“无辜婴儿的屠杀”或者“圣雅各的殉难”\footnote{无辜婴儿的屠杀指以残暴著称的犹太国王希律(前39—前4在位)对无辜婴儿的屠杀。雅各是耶稣十二使徒之一,被希律之孙希律亚基帕一世杀死于耶路撒冷。}似的。他倒仿佛当真属于那个已经消失了的家族,那个也许仅仅在圣芝诺教堂祭坛后部装饰屏上以及埃尔米塔尼教堂壁画上(斯万是在那里跟这个家族接触的,这个家族还在那里沉思)才存在的家族;这个由古代雕像与大师\footnote{指曼坦那。}的巴杜亚模特儿或者丢勒笔下的撒克逊人相结合的产物的家族。他那棕红色的头发天然是鬈曲的,抹着润滑油而粘在一起,那发髻鬈得雄浑有力,就像曼图亚那位画家\footnote{指曼坦那。曼图亚为意大利北部城市,公爵府饰有曼坦那的壁画。}不断研究的希腊雕像上的发髻一样;希腊雕刻在创始时虽只处理人像,却也善于从人的简单的线条中提炼出丰富多彩的形式,仿佛从整个生物界中都有所借取,就说是那一头头发吧,它那平缓的起伏,发髻尖尖的角,发辫上冠冕式装饰三层重叠排列就既像是一团海藻,一窝鸽子,又像是一片风信子花,也像是盘成一团的蛇。
\par 还有一些仆役,也都是身材魁梧,站在那宏伟壮观的台阶石级上,像大理石雕像那样一动也不动,纯粹起着装饰的作用,把这台阶点缀得简直跟公爵府的“巨人台阶”一般;斯万走上这台阶,心想奥黛特还从来没有涉足此间,不禁有些忧伤。啊!与此相反,要是他能登上那歇业的小女裁缝那昏暗的发出难闻的气味,一不小心就会摔倒的楼梯,他又该多么高兴!他要是能在奥黛特去她那小阁楼的日子同去消磨晚间的时刻,他都乐于付出比歌剧院包厢一星期还多的钱;即使是奥黛特不去的日子,他也可以跟经常和她见面的人们谈起她,和他们生活在一起;这些人由于经常和她见面,他认为他们身上藏有关于他的情妇的生活当中的更真实、更难以取得、更神秘不可测的东西。在这歇业的女裁缝这个恶臭但值得羡慕的楼梯上,由于另外没有一条专供仆役或者送货者用的楼梯,所以每到晚上,家家门口的擦鞋垫上都摆着一只脏的空奶罐,在斯万此刻登上的这个华丽而可恶的台阶上,在左右两侧不同的高度上,在门房的窗户或者套房的入口,在墙上形成的每一个凹处则都站着一个门房,或者是管家,或者是账房,分别代表着他们经管的府内业务,同时也是向来客表示敬意(他们也都是些体面的人物,每星期都有一部分时间在他们自己的产业上过着多少独立的生活,像小业主那样在家吃饭,有朝一日也许会到一个知名的医生或者实业家那里去服务),他们兢兢业业地谨守人们在让他们穿上这辉煌的号衣以前给他们的种种教导,这号衣他们也是难得穿上身,穿着也并不太舒服;他们站立在各自的门洞的拱廊底下,穿得鲜艳夺目,却多少带点市民的憨厚劲儿,仿佛是神龛里的圣像似的;还有一个身材高大的瑞士卫兵,打扮得跟教堂侍卫一样,在每一位来客走过他跟前时用手杖在地面上敲打一下。斯万在一个脸色苍白,像戈雅\footnote{戈雅(1746——1828),西班牙画家,对欧洲十九世纪绘画有深刻影响。}画中的圣器室管理人或者剧中公证文书誊写人那样,脑后用缎带扎着一条小辫的仆役陪伴下走到台阶顶上,到了一张办公桌跟前,那里有几个当差的像公证人那样,端坐在登记簿前,见斯万来到就站起身来,把他的名字登下。他这就穿过一个小前厅。有些人把某些房间专门为摆某一件艺术品而布置起来,就用这件艺术品来命名,故意弄得空空荡荡,不摆任何别的东西,而这个小前厅就是这样一间屋子,在进口处站着一个年轻的仆役,就像本韦努多·切利尼\footnote{本韦努多·切利尼(1500——1571),意大利雕塑家。}雕塑的一尊无比珍贵的武装卫士塑像一样,上身微向前倾,在红色的衬领中伸出一张更加红润的脸蛋,仿佛赫然烧着一团炽热、腼腆和热忱的火焰;他以强烈、警觉、发狂的目光穿透那挂在演奏音乐的客厅门口的奥比松挂毯,仿佛是以军人的沉着或不可思议的诚心——这是警觉的象征、期待的化身、暴乱的纪念——像哨兵那样从炮楼顶上监视着敌人出现或者像天使那样在大教堂顶上等待着最后审判时刻的来临。现在斯万只消迈进举行音乐会的大厅了,有个身背钥匙串链子的掌门官弯腰为他把门打开,仿佛是将城门的钥匙呈献给他似的。但斯万这时想的却是他可能去的那所房子(假如奥黛特许可的话),而擦鞋垫上空奶罐这个形象使他突然感到一阵恶心。
\par 迈过了那条挂毯,仆人的场面让位于客人的场面,斯万很快就发觉男宾都很丑陋。男性面貌之丑,他是知之已久了,可是自从他发现男人的相貌的基础在于五官线条的独立自主性(仅受美学关系的调节)以后,男性面貌之丑对他来说又成了新鲜事物了——在这以前男人的相貌对他来说本是用来辨认某一个人的符号,而这个人或则代表一系列值得追求的欢乐,或则代表应予驱避的烦恼,或则代表应该还报的礼数。斯万在他身边的人们身上,现在再也找不出一样东西不具有一定的个性了,就算是许多人都戴的单片眼镜吧,在他心目中过去至多只是表明他们戴单片眼镜罢了,现在也已经不再是人所共有的习惯而都各有特征了。也许是因为他现在只把正在入口处聊天的弗罗贝维尔将军和布雷奥代侯爵看成是一幅画当中的两个人物,而他们过去很长一段时间内对他来说却是把他介绍进赛马俱乐部,在几次决斗中帮过他忙的有用的朋友,所以将军那单片眼镜,那像一片弹片那样嵌在他那庸俗、带着伤疤、洋洋得意的脸上,那像希腊神话中的独眼巨人的那只独眼那样在前额中央独树一帜的单片眼镜,现在在他眼里却成了一个吓人的伤疤,受这样的伤固然是光荣,在别人面前显示出来却不大体面;至于德·布雷奥代先生,为了参加社交活动,增加节日气氛,除了戴上珍珠色手套、高级黑礼帽、白领带以外,也戴上一副单片眼镜来替代平常的夹鼻眼镜(斯万自己也是这么做的),像显微镜下的一张切片那样紧贴在镜片背面的是他那其小无比的眼睛,眼里射出亲切的目光,不时流露出微笑,对天花板之高,晚会的欢乐气氛,节日的安排和清凉饮料的质量表示满意。
\par “啊!原来是您哪!真是半辈子没有见着了。”对斯万说这话的是将军,他看到斯万愁眉苦脸,以为他也许是生了一场重病才离开了社交界,便找补上一句:“您现在气色不错嘛!”这时候德·布雷奥代先生则问一个刚把单片眼镜(这是他唯一用做心理观察和无情分析的工具)戴上眼角的专写社交生活的小说家:“怎么?您老兄到这里有何贵干?”这位小说家煞有介事,故作玄虚地答道:
\par “我在观察哪!”他的小舌音发得很重。
\par 福雷斯代尔侯爵的单片眼镜很小,镜片没有边框,像不知从何而来,又不知是何质地的一块多余的软骨一样嵌在眼皮里,弄得眼睛不停地、痛苦地抽搐,给侯爵脸上平添了几分带有阴郁色彩的细腻感情,使得妇女们深信他一旦失恋了是会感到非常痛苦的。德·圣冈代先生那副单片眼镜则跟土星一样,周围有个很大的环,它是那张脸的重心所在,整个脸随时都围绕它而调整,那个微微翕动的红鼻子,还有那张好挖苦人的厚嘴唇的嘴巴总是竭力以它们做出的怪模样来配合那玻璃镜片射出的机智的光芒;这副单片眼镜也引起那些轻佻的赶时髦的女郎的遐想,梦想从他那里得到矫揉造作的献媚和温文尔雅的逸乐;而那位大鲤鱼脑袋和鼓包眼睛的德·巴朗西先生戴着他那副单片眼镜在人群中慢慢地走来走去,时不时地松开他那下巴骨,仿佛是为了确定行进的方向似的;他那副模样就像是脸上只带着他那玻璃大鱼缸任意的,也许是象征性的,用于窥一斑而知全豹的一片玻璃——斯万十分欣赏乔托在帕多瓦一个教堂画的《罪恶》和《德行》这些画,他这就想起了“不义”身边那支绿叶葱葱的枝条,它象征着隐藏着它的巢穴的那些森林。
\par 在德·圣德费尔特夫人的恳求下,斯万走向前去,为欣赏由长笛演奏的《俄耳甫斯》\footnote{德国歌剧作曲家格鲁克(1714—1787)作。}中的一个曲子而在一个角落坐了下来,眼前只有两位年纪已经不算很轻的夫人并坐在一起,一位是康布尔梅侯爵夫人,一位是弗朗克多子爵夫人,她们是表姊妹,时常手提提包,在她们的女儿的陪伴下在晚会上像在火车站那样你找我,我找你,直到她们用扇子和手绢指着两个相连的空位置时才安静下来:德·康布尔梅夫人跟别人来往不多,很高兴能有德·弗朗克多夫人做伴,后者却很有名望,当着她那些漂亮朋友的面陪一位跟她曾一起度过童年的默默无闻的夫人,自以为这事儿做得很有风度,很独出心裁;斯万皱起眉头冷眼瞧着她们两位听长笛独奏后面那段钢琴插曲(李斯特的《圣法兰西斯跟鸟儿说话》),看那位名手令人为之眩目的指法:德·弗朗克多夫人是心急如焚,两眼射出发狂的光芒,仿佛钢琴家手指飞奔的那些琴键都是一架架高耸的秋千,一失足就能坠入八十米深的深渊,她同时向她的邻座投去惊讶怀疑的目光,仿佛在说:“能演奏到这等地步,简直是难以置信。”德·康布尔梅夫人摆出一副受过良好音乐教育的架式,脑袋跟节拍器的摆那样在打着拍子,从一个肩头晃到另一个肩头,摆动得那么大那么快(两眼则投出那不再去追究所受的痛苦也不想去加以控制,只满足于说一声“这又有什么办法”的受苦受难的人的茫然的目光),随时都牵动她上衣皱边上的钻石,也叫她不得不经常去摆正插在头发上的黑葡萄串,但并不因此而中断它越来越快的摆动。在德·弗朗克多夫人身旁,稍前一些的是加拉东侯爵夫人,她成天念念不忘的是她跟盖尔芒特家族的亲族关系,这为她的沙龙以及她个人大为增色,却也多少使她有点丢脸,因为这个家族中最显赫的人都多少有点回避她,这也许是由于她为人有点讨厌,也许是由于她名声不是太好,也许是由于她出于地位较低的一支,也许是根本没有任何理由。当她跟她不相识的人在一起的时候,譬如此刻在德·弗朗克多夫人身边的时候,她就苦于不能把她跟盖尔芒特家族的亲族关系用明白无误的词句标榜出来,就像东正教教堂的拼花图案上用直行的文字写在圣者身旁注出他们所说的话语一样。她此刻想的是,自从她表妹洛姆亲王夫人结婚六年以来,还从没有邀请过她,也没有来看望过她。想到这里,她满腔怒火,却也不无自豪之感,这是因为,如果有人奇怪怎么在洛姆亲王夫人家见不着她,她就可以说那是为了避免在那里碰上玛蒂尔德公主\footnote{玛蒂尔德公主(1820—1904):热罗姆·波拿巴亲王之女,她家的沙龙在第二帝国时期颇为知名。},而万一碰上了,那可是她那极端正统主义的家庭所决不能原谅的;这样一来,她也终于把这当做是她不上她表妹家去的理由了。她可也记得,她自己曾多次问过洛姆亲王夫人,她怎样才能跟她见面,然而到底得到了什么答复,印象已经模糊,只是常常嘀咕:“再怎么说,这第一步总不该由我迈出,我比她大二十岁呢。”以此来冲淡这令人羞辱的回忆。靠了这内心独白的力量,她傲慢地把双肩往后一甩,简直使它们脱离了她的胸部,她的脑袋也几乎跟肩膀齐平了,不禁叫人想起餐桌上插在骄傲的山鸡上那只带羽毛的鸡头。倒不是说她苗条得像只山鸡,她可是生来矮胖粗壮,大有男子气概;不过多年所受的凌辱却使她的脊梁挺直了起来,就好像是不幸长在崖边的树木为了保持平衡而向后往斜里生长一样。为了安慰自己不能跟盖尔芒特家族中其他人处于平等地位,她只得经常念叨,她之所以不常去看他们,那是由于她那毫不妥协的原则性和自豪感,久而久之,这种想法居然塑造了她的体态,使她产生了一定的仪容,平民百姓把它看成是上等人家的特征,有时也在俱乐部那些先生昏花的老眼里激起一刹那的欲念。谁要是把德·加拉东夫人的谈话加以分析,把每一个词语出现的频率统计出来,从而找出破译密码的关键,那就会发现即使是最常用的词语,出现的次数也不会多于“在盖尔芒特堂兄弟家”、“在盖尔芒特姑妈家”,“埃尔赛阿尔·德·盖尔芒特的健康”、“盖尔芒特表妹的浴盆”这些词语。当人们跟她谈起一个知名人士时,她总答道,她个人并不同他相识,然而在她盖尔芒特姑妈家却碰到过上千次,而且在回答的时候语调是那么平淡,声音是那么沉重,显然表明她个人之所以并不同他相识,还是出于那些根深蒂固不可动摇的原则;她那向后拱的双肩依靠的就是这些原则,就仿佛体操教练为了锻炼你的胸廓而让你依靠平衡木一样。
\par 大家原本没有料到会在德·圣德费尔特夫人家见到洛姆亲王夫人的,那天她可当真来了。她原是屈尊光临的,为了表示她并不想在客厅中显摆自己的门第,她是侧着身子进来的,其实面前既没有人群挡道,也没有任何人要她让路;她故意呆在客厅尽头,摆出一副适得其所的神气,仿佛是一个没有通知剧院当局而微服亲自在剧院门口排队买票的国王似的;为了不突出她在场,不招引众人的视线,她一个劲儿低头观察地毯上或她自己裙子上的图案,站立在她认为是最不显眼的地方(她清楚地知道,德·圣德费尔特夫人只要一瞥见她,一声欢呼,就会把她从那里拉将出去),就在她所不认识的德·康布尔梅夫人身旁。她观赏这位爱好音乐的邻座表演的哑剧,但并不去模仿她。这并不是说,洛姆亲王夫人这回拨冗来德·圣德费尔特夫人家待上五分钟,就不愿意尽可能表现得和蔼可亲,使她对主人的这番恩惠显得加倍地可贵。不过她生来就讨厌她所谓的“浮夸”,坚持不做出与她生活于其间的那个小圈子的“派头”不相适应的举动,虽然这些举动对她也不免产生诱惑,因为在与新环境(哪怕它比自己所在的环境低微)接触时,即使是最自信的人们也会产生一种模仿心理(同羞怯有点相近)。她首先心想,这乐曲也许跟她迄今为止所听的音乐不是一个路子,是否有必要手舞足蹈,又想如果不手舞足蹈是否表示自己不懂得这音乐,对女主人是否有失礼仪:结果她只好采取折衷办法来表达她这些相互矛盾的思想感情,一会儿一面以不动声色的好奇盯着她那狂热的邻座,一面扶扶肩带,摸摸她那金黄色头发上镶有钻石的珊瑚或者珐琅小球(这使她的发型显得既朴素又好看);一会儿用她的扇子打打拍子,但为了显示她不受乐曲的支配,并不按着节拍来打。钢琴家弹完了李斯特的一个曲子,又转入肖邦的一支序曲,这时德·康布尔梅夫人朝德·弗朗克多夫人投去温情的微笑,它既载着对往日岁月的回忆,也显示出行家满意的心情。她在年轻时就学会怎样抚爱肖邦那些婉转曲折、特别长大的乐句,它们是如此自由、柔和,如此易于感受;它们在开始时总在寻觅试探,力图逸出出发时的方向,在远离人们以为它们将到达之处,却总是在奇想的歧途上徘徊良久才更坚定地回来击中你的心坎——这回来的路程是事先精密地筹划了的,就像是一只水晶杯子,一响起来就不由你不发出一声惊叹。
\par 她生活在一个交游极窄的外省家庭里,几乎从不参加舞会,沉醉于庄园的孤寂生活之中,把所有那些想象中的舞伴的舞步或者放慢或者加速,像拨弄花瓣那样把他们挨个儿拨弄,暂时离开舞会到湖畔松林中去倾听狂风呼啸,突然看到有一个身材修长,嗓音既悦耳却又古怪又走调,戴了一副白手套的小伙子向她走来,跟人们梦想中这人世间的情人不大一样。可是今天呢,这种音乐的美已经过时,失去了鲜艳的色彩。几年来已经不再博得行家的重视,已经失去了原有的名声,原有的魅力,即使是口味平庸的听众从中得到的乐趣也平平常常,不屑一谈了。德·康布尔梅夫人回过头来偷看一眼。她知道她年轻的儿媳妇(她对她的婆家倒是满怀敬意的,但她既懂和声又认识希腊字母,在精神方面的事物上有她自己的看法)是看不起肖邦的,听到肖邦的音乐就头痛。她是个瓦格纳迷,这会儿跟一帮同她年纪相仿的人坐在远处,这下德·康布尔梅夫人摆脱了她的监视,可以尽情陶醉在她甘美的印象之中了。洛姆亲王夫人也有同样的感受。她虽然没有音乐的禀赋,可在十五年前也曾跟圣日耳曼区的一位钢琴教师学过,这位天才女性到了老年,生活贫困,在七十岁上重操旧业,教她从前的学生的女儿和孙女儿辈。她现在已经不在世了。可她的方法,她那美妙的琴声有时还在她的学生的指上重现,甚至还在那些早已平庸不足道,放弃了音乐,几乎连钢琴盖都早就不再打开的学生的指上重现。因此,洛姆夫人还能恰如其分地摇头晃脑,能正确欣赏钢琴家所演奏的那首她都能背得出来的序曲。开头那个乐句的最后半段都在她嘴上油然哼出来了。她喃喃自语:“真是美妙极了。”这“美妙”两字是带着这样深挚的感情,她都感到自己的双唇神秘地在翕动,同时也不由自主地在视线中注入了茫然的感伤色彩。德·加拉东夫人这会儿却暗自嘀咕,碰见洛姆亲王夫人的机会是如此难得,真是叫人恼火,因为她真想在亲王夫人跟她打招呼的时候不予理睬,用这样的办法来教训教训她。她不知道她这位表妹这会儿就在这里。德·弗朗克多夫人一点头,使她看到了亲王夫人。她立即奔到她的跟前,也顾不得对别人的打扰了;她想保持那副高傲冷淡的神气,好提醒大家,无论是谁,要是在她家里有可能面对面碰上玛蒂尔德公主的话,她是不愿意同这样的人打交道的,再说就岁数而言,她跟她也不是同一代人;不过她也想冲淡这副高傲而有保留的神气,说几句话来表明她来找她是事出有因,同时迫使亲王夫人不得不讲几句话;因此,德·加拉东夫人一到她表妹跟前,就绷着脸,无可奈何地伸出一只手问她:“你丈夫怎么样?”那语调充满了担心,倒仿佛亲王得了什么重病似的。亲王夫人以她特有的方式哈哈大笑,这一笑既是为了让别人知道她在讥笑某人,又是为了把她面部的线条都集中到她那生动活泼的嘴唇和炯炯有神的眼睛周围,从而使自己显得更美。她答道:
\par “再好也没有了!”
\par 说罢又笑了起来。这时德·加拉东夫人挺起上身,板起脸,仿佛还在为亲王的健康状况担忧,对她表妹说:
\par “奥丽阿娜(这时德·洛姆夫人以惊讶和含笑的神色瞧着一个看不见的第三者,仿佛是要请他证明,她可从来没有许可德·加拉东夫人直呼其名),我很希望你明晚能上我家小坐片刻,听一听莫扎特的五重奏,有单簧管。我想听听你的意见。”
\par 她好像不是在提出一次邀请,而是要对方帮个忙,要听听亲王夫人对五重奏的意见,仿佛是她的新厨娘创造出一道新菜,很希望听到美食家的意见似的。
\par “我知道这首五重奏,我可以把我的意见马上告诉你:我是喜欢它的!”
\par “嗯,我丈夫身体不怎么好,他的肝……要是他能见着你,他会是非常高兴的。”德·加拉东夫人接着说,现在是用爱德这个道理来将亲王夫人的军,要她在晚会上露面。
\par 亲王夫人不喜欢对人说她不愿意上他们家去。她每天总是给人写信表示歉意,说她怎么因故不能出席他们的晚会(其实是不想去),什么婆婆突然来家啦,小叔有所邀请啦,要上歌剧院啦,要去郊游啦,如此等等,不一而足。她这就让许多人听了心里高兴,以为她跟他们是愿意交往的,而她之所以不能应邀参加都是因为亲王府临时有事冲突,而把这样的事来跟他们举办的晚会相提并论,实在是很给他们面子的。亲王夫人出自盖尔芒特家族那个才气横溢的小集团,头脑机敏,谈吐不凡,情感高尚——这种精神可以上溯至梅里美,最后表现于梅拉克和阿莱维\footnote{梅拉克(1831—1897),法国剧作家;阿莱维为其合作者。}的戏剧之中;亲王夫人甚至把这种精神运用于社交关系之中,移之于礼仪之间,使之尽量明确实在,接近于实际。她决不会费许多唇舌对一个家庭主妇说她是多么想参加她家的晚会;她认为跟她谈些能否左右她前往的琐碎小事更加亲切。
\par “你听我说,”她对德·加拉东夫人说,“明儿晚上我可得上一个朋友家去,把这日子定下可费了事了。她要是领我们去看戏,那我就怎么想去你家也去不成了;如果我们在她家待着,我知道除了我们就没有旁人,我倒可以向她告辞。”
\par “对了,你看见你的朋友斯万先生没有?”
\par “没有,可爱的夏尔哪,我都不知道他这会儿在这里,我得想办法让他见到我才是。”
\par “说来也真怪,他怎么会到圣德费尔特婆娘家来,”德·加拉东夫人说,“我知道他可是个聪明人(其实她的意思是说“他可是个耍弄阴谋诡计的人”),这可也挡不住他这个犹太人踩进两个大主教的妹妹和嫂子的大门!”
\par “说句不嫌丢丑的话,我并不觉得这是什么令人震惊的事情。”洛姆亲王夫人说。
\par “我也知道他已经改了宗,连他的父母和祖父母也都已经改了宗。不过据说改了宗的人比没有改宗的人还要依恋他们原来的宗教,说那不过是虚晃一枪,不知道是否当真?”
\par “这问题我可不了解。”
\par 钢琴家要演奏肖邦的两支曲子,弹完前奏曲以后马上就开始弹一首波洛涅兹舞曲。不过自从德·加拉东夫人告诉她表妹,此刻斯万也在场以后,哪怕是肖邦起死回生,亲自来弹奏他的全部作品,洛姆亲王夫人也不会听它半句的。人类分成两拨,一拨只对他们不认识的人感兴趣,而在另一拨人身上,这种兴趣只对他们认识的人才有。亲王夫人属于后一拨。跟圣日耳曼区的许多妇女一样,她无论到什么地方,只要她那小圈子里有谁也在场,虽然对他没有什么特别的话要说,却也能把她的注意力全部占据,其余的一切她就全然不顾了。从那时起,亲王夫人一心存着能被斯万看到的希望,一个劲儿左顾右盼(就像是一只被驯养的小白鼠,驯养员拿一块糖一会儿伸向它的鼻子,一会儿又往后缩回),脸上是万千默契的线条,可就是跟肖邦的波洛涅兹舞曲传达的感情没有任何关系;她的脸总是探向斯万所在那个方向,如果斯万挪个地方,她也就随之挪动她那怀有深情的微笑。
\par “奥丽阿娜,你可别生气。”德·加拉东太太这个人时常为了图一时的痛快,说上几句不中听的话,宁可牺牲她在社交界里辉煌的前途,牺牲她有朝一日在社交圈子里出出风头的希望。这时她说:“有人说斯万先生这号人在家里是接待不得的,是不是这样?”
\par “这你比谁都更清楚,”洛姆亲王夫人答道,“你不是邀请过他五十回,他连一回也没上你家去过吗?”
\par 在离开这位受了侮辱的表姐时,她又哈哈大笑,激起了那些听音乐的人的反感,却引起了德·圣德费尔特夫人的注意。她出于礼貌,坐在钢琴旁边,直到那时才瞥见了亲王夫人。德·圣德费尔特夫人原本以为她还在盖尔芒特照料她那生病的小叔子呢,现在见她来了,自然分外高兴。
\par “怎么?亲王夫人,您来了?”
\par “对了,我刚才坐在一个犄角里,听了不少好东西。”
\par “怎么,您已经来了好一会儿了?”
\par “对了,已经来了好一会儿了,可我觉得才只一会儿,只是因为没有看见您才觉着慢。”
\par 德·圣德费尔特夫人想把她的扶手椅让给亲王夫人,夫人说:
\par “不必,不必!干吗要换呢?我坐哪儿都挺好的。”
\par 为了表现她贵妇人的朴实,她故意找了把没有靠背的小凳子:
\par “得了,这张软垫凳子就好极了,坐在上面我可以把上身挺直。啊!天哪,我在这里叽叽喳喳的,人家都要嘘我了。”
\par 这时钢琴家正加快速度,他那音乐激情正处于高潮之中,一个仆人正端着一方盘的清凉饮料递给客人,茶匙丁当直响,德·圣德费尔特夫人跟每次晚会一样,挥手叫他走开,可他老瞧不见她的手势。有个新娘子,遵从年轻女子不应该面有厌烦之色的教导,老是高高兴兴地面带笑容,两只眼睛直在寻找女主人,好用她的眼神来向她表达感激之情,感谢她在举办这样的盛典时还想起了她。她虽然比德·弗朗克多夫人要镇静一些,但在欣赏乐曲的时候也不是毫无不安的心情;不过她所担心的不是钢琴家本人,而是那架钢琴,它顶上摆着一支蜡烛,每当弹到最强音时烛火都会跳动起来,即使不至于会把灯罩烧着,至少会在红木琴台上留下几点蜡泪。到了最后,她忍不住了,登上琴台那两级台阶,快步向前把那蜡台的托盘撤走。但她的双手刚碰到托盘,乐曲最后一个和弦就响了起来,一曲告终,钢琴家站起身来。再怎么说,这位年轻妇女的大胆的首创精神,她跟钢琴家短时间内在台上的同时出现,在在座者的心中普遍产生了良好的印象。
\par “亲王夫人,您瞧见这位妇女了吗?”德·弗罗贝维尔将军问洛姆亲王夫人。他是过来跟亲王夫人打招呼的,德·圣德费尔特夫人刚走开一会儿:“真希罕!莫非她也是艺术家?”
\par “不,她是康布尔梅家的新媳妇,”亲王夫人随便这么一说,马上又找补一句:“我这是重复我听来的话,她究竟是谁,我一点概念也没有,我背后有人说他们是德·圣德费尔特夫人乡下的街坊,不过我不信真有谁认识他们。他们多半是‘乡下佬’!再说,我不知道您是不是经常出入于这个了不起的社交场所,我可对这些了不起的人姓甚名谁毫无概念。您想他们在参加德·圣德费尔特夫人的晚会以外的时间干些什么呢?她多半是靠了这些音乐家,这些舒服的椅子,还有可口的饮料才把他们吸引来的。应该承认,这些‘贝卢瓦家的客人’\footnote{贝卢瓦是专门出租椅子的商人。}倒是挺不错的。她居然当真有这股勇气每星期都出钱把这些凑热闹的租到家里来。真是不可思议!”
\par “嗯,康布尔梅可是个响当当的姓氏,又古老。”将军这么说。
\par “说它古老,我不反对,”亲王夫人冷冰冰地答道,“不过这名字读起来不和谐。”她把“和谐”两字读得特别重,仿佛是带了引号的,这又是盖尔芒特这个小圈子里的人说话的矫揉造作的一种表现。
\par “您这话当真?她可是美得可以入画。”将军说,他的视线一刻也不离开德·康布尔梅夫人,“您不这么认为吗,亲王夫人?”
\par “她太爱出头露面,我觉得像她这么年轻的人,这就不太好了;我想她还不是我的同龄人。”洛姆夫人答道(这最后一句话,同样也可以出之于加拉东和盖尔芒特之口)。
\par 亲王夫人看到德·弗罗贝维尔先生还在目不转睛地瞧着德·康布尔梅夫人,半是出于对这位夫人的恶意,半是出于要对将军表示殷勤,说道:“这对她丈夫可是不太好了!我很遗憾,并不认识她,否则我就可以把她介绍给您,看来您是被她迷上了。”其实她要是当真认识这位青年妇女,她是不会这么干的,“现在我不得不跟您道别了,今天是我的一个朋友的生日,我得去祝贺她。”她说这话时的语调既朴素又真实,表明她就要去参加的这个社交集会既是一个令人生厌的仪式,又不能不去,而她的光临是会令人感动的。“再说,我得去接巴赞,我到这儿来的时候,他去看他的朋友去了。我想您是认识他们的,他们的姓跟一座桥的名称一样,叫耶拿。”
\par “耶拿,这首先是一次胜利的战役的名称,亲王夫人,”将军说,“我是个老兵,首先想到的就是这些。”他一面说,一面把单片眼镜摘下来擦一下,就像是给伤口换块纱布似的。这时亲王夫人本能地扭过头去说“帝国时期封的贵族嘛,那当然是另外一回事,不过他们这伙人倒都是好样儿的,他们当年打起仗来都是英雄。”
\par “我对英雄是满怀敬意的,”亲王夫人说,那口气里多少有点讽意,“我所以没有跟巴赞一起上那位耶拿亲王夫人家去,根本不是因为我瞧不起他们,完完全全因为我不认识他们。巴赞认识他们,非常喜欢他们。不,不,并不像您所想的那样,这里头并没有什么爱情问题,我没有什么可反对的!再说,真要是有那样的事,我反对又有什么用?”她无可奈何地找补上这一句。谁都知道,自从洛姆亲王娶了他那秀色可餐的表妹,打第二天起就不断地对她不忠。“话又说回来了,这并不是那么回事,他们都是他老早就认识的人,对他很有好处,我也觉得这是件好事。我先来跟您讲讲他们的房子……您想想,他们的家具全都是帝国时期的式样!”
\par “亲王夫人,这是自然的,这是他们祖父母传下来的。”
\par “我也不是不知道,可这也挡不住这些家具样子丑陋。一个人家里可能没有好看的东西,这是可以理解的,然而至少不应该有滑稽可笑的东西。不瞒您说,我还从来没见过比那种可怕的式样更做作,更土气的东西呢,那五斗柜上居然装饰着澡盆那么大的天鹅头呢!”
\par “不过我想他们家里也有些好东西,譬如有一张精工镶嵌的桌子,有个什么条约就是在那张桌子上签字的。”
\par “啊!他们家是有些有历史意义的东西,这我承认。可是这些东西并不美……而是可怕!我自己也有些这样的东西,是巴赞从蒙代斯吉乌家继承来的。所不同的是,这些东西我们都收藏在盖尔芒特家里的顶楼上,谁也瞧不见。得了,得了,问题不在这里。假如我认识他们的话,我是会跟巴赞一起奔他们家去看他们,看他们家的狮身人面像,看他们家的铜器的,可我不认识他们!我从小就被教导说,上不认识的人家去是不礼貌的(她讲到这里的时候装出一副孩子气)。我是一向遵从这个教导的。哪有正派人让一个不相识的女人进他们家的?我要去了,岂不是要吃闭门羹吗?”
\par 这当然是种假设,讲到这里,她微微一笑,她那蓝眼睛盯着将军,这时带着梦幻般温柔的表情,就使得那微笑更美更俏了。
\par “啊!亲王夫人,您明明知道,您要去了,他们是会喜出望外的……”
\par “是吗?那是为什么?”她急忙问道,这也许是为了不显出她明明知道这是因为她是法国最高贵的贵妇人之一,也许是因为这话出之于将军之口而高兴,“那是为什么?您怎么知道?他们也许会把这看成是再讨厌也不过的事情呢。我不知道是不是这样,不过就我来说,跟我认识的人打交道都已经叫我烦透了,要是叫我跟我不认识的人打交道,哪怕是跟英雄好汉,我都要疯了。再说,除了像您这样早就认识的老朋友以外,我不知道英雄气概在社交界能起多大作用。请客吃饭有时都已经烦人了,如果还要伸出胳臂来邀斯巴达克\footnote{古代罗马奴隶起义领袖。}入席,那就……我也决不会邀请费森谢特里克斯来当第十四位\footnote{费森谢特里克斯,古代高卢将军,政治家,率领高卢人抵御凯撒。在西方,十三是个不祥的数字,碰到一桌十三人时,临时邀一人入席凑数。}。我想我可以请他来参加人数众多的晚会,可我又不组织这样的活动……”
\par “啊!亲王夫人,您这位盖尔芒特家人可真是货真价实。盖尔芒特家人的风趣,您身上可是充分体现出来了!”
\par “大家都说盖尔芒特家人的风趣,我真不明白那是为什么。难道您还认识别的有风趣的盖尔芒特家人吗?”说到这里的时候她哈哈大笑,眼睛鼻子都挤到一块来体现她的高兴劲儿,双眼炯炯有神,射出只有赞美她的风趣或美貌的言语(哪怕出自亲王夫人自己之口)才能激起的愉快的光芒。“嗳!斯万像是在那里跟您的康布尔梅打招呼呢;喏,他在圣德费尔特婆娘身边,您瞧不见!您可以请他把您介绍给她。得快着点儿,他要走了。”
\par “您有没有瞧见他那脸色是多么难看?”将军说。
\par “可怜的夏尔!啊!他终于来了,我都以为他不愿意见我的面呢!”
\par 斯万非常喜欢洛姆亲王夫人,看到她就想起跟贡布雷相邻的盖尔芒特,想起他如此热爱,而只是为了不愿离开奥黛特才不再回去的那片土地。他善于使用半是艺术性,半是情场用的语言来取悦亲王夫人。当他一时返回他久违的社交圈子时,自然不免要应用一番:
\par “啊!”他话是对德·圣德费尔特夫人说的,可又是说给洛姆夫人听的,“原来可爱的亲王夫人在这里!诸位,她是专程从盖尔芒特来听李斯特的《圣法兰西斯跟鸟儿说话》的,时间仓促,她只能跟美丽的山雀一样,随便捡几个李子,捡几个山楂插到头上就来了;现在还有几滴露珠,一点白霜,冷得公爵夫人直呻吟呢。真漂亮,亲爱的亲王夫人。”
\par “怎么?亲王夫人是专程从盖尔芒特来的?真是太棒了!我真抱歉,我原来还不知道呢。”德·圣德费尔特夫人天真地叫道。她对斯万的风趣话是不大习惯的。当她仔细看亲王夫人的头饰时她又说:“倒是真的,这是模仿……该怎么说呢?不像是栗子,这想法真是妙极了!可亲王夫人是怎么知道我的节目表的呢?音乐家们连我都没有告诉呢。”
\par 当斯万在一个惯常用情场的言语交谈的妇女身边时,他是常讲一些连上流社会中的许多人都不懂得的微妙的话的。他不屑于跟德·圣德费尔特夫人解释,说他是用隐喻说话的。至于亲王夫人呢,她都哈哈笑开了,因为斯万的风趣在她那个圈子里是深受赞赏的,也因为每当听到赞美她的话时,她总觉得这话是无比的优美,也总是令人忍俊不禁。
\par “好极了!夏尔,我这些小山楂果子合您的心意,我真高兴!您干吗跟那位康布尔梅人打招呼,莫非您也是她在乡间的街坊?”
\par 德·圣德费尔特夫人见到亲王夫人很乐意跟斯万聊天,就走开了。
\par “您自己不也是吗,亲王夫人?”
\par “我?莫非这些人到处都有乡间别墅?我倒真想能跟他们一样!”
\par “他们不是康布尔梅人,那时在康布尔梅的是她的亲戚;她娘家姓勒格朗丹,常到康布尔梅去。我不知道您知不知道您自己还是康布尔梅伯爵夫人,教务会还欠您一笔租金呢!”
\par “我不知道教务会欠我什么,可我知道本堂神甫每年向我借一百法郎,这笔钱我以后是不想再借出了。再说,这些康布尔梅人的名字也真能吓人一跳,结尾倒是干脆,可是并不高明!”她笑着说。
\par “开头也并不更高明些。”斯万答道。
\par “敢情这是两个缩略词拼起来的!”
\par “这准是一个怒气冲天却又讲体面的人创造出来的,他不敢把第一个词说完。”
\par “可既然他不能自己把第二个词说出来,他又何不把第一个词说完,一了百了呢?咱们这是在大发雅兴,开起玩笑来了,亲爱的夏尔——不过现在老见不着您,真够伤脑筋的。”她以温存的语调找补一句:“我是多么喜欢跟您聊聊天。您想想,我都没法子让弗罗贝维尔这笨蛋明白康布尔梅这个名字为什么能吓人一跳。生活这个东西也真是可怕。只有看到您的时候,我才不感到厌烦。”
\par 这当然不是真话。不过斯万跟亲王夫人对小事情的看法是一致的,结果连说话的方式甚至读音都非常相似,要不然正是这个相似导致他们看法的一致。这种相似倒并不太引人注目,因为他们两个人的声调迥然不同。不过只要你能在想象中把斯万的话语里他那洪亮的嗓音跟话语从中吐出的两撇小胡子去掉,你就可以发现这些语句、音调的这些变化,全都是盖尔芒特小圈子那一套。可在大事情上,斯万跟亲王夫人就毫无共同之处了。不过自从斯万如此消沉,随时总感到就要哭出声来以后,他总像一个杀人凶犯需要把他犯的罪行诉说出来一样,需要把他自己的苦楚倾吐一番。听到亲王夫人说到生活这个东西也真是可怕时,他感到得到一点安慰,仿佛亲王夫人跟他说起了奥黛特似的。
\par “对啊!生活这个东西真是可怕。咱们得时常见见面,亲爱的朋友。跟您在一起,好就好在您不是个嘻嘻哈哈的人。咱们可以一起度过一个愉快的晚上。”
\par “那是当然,您为什么不到盖尔芒特来呢,我婆婆会高兴得要死的!这地方景色不美,不过我敢说这地方并不令人不快,我讨厌‘风景如画’的地方。”
\par “这我相信,你们那地方好极了,”斯万答道,“此刻对我来说都已经太美,太热闹了,反正这是一个使人幸福的地方。这也许是因为我在那里生活过,所以连那里的一草一木都能跟我说得上话。当微风拂面,麦穗荡漾的时候,我就感觉到有人要来,将要收到什么消息;还有河边那些小房子……我该是多么不幸,如果……”
\par “哦!亲爱的夏尔,留点儿神,那夜叉朗比荣婆娘瞧见我了,快把我挡住,告诉我她家发生了什么事,我都搞糊涂了,是她把女儿嫁出去了,还是给她的情夫找了个妻子,我闹不清了;也许是把她的女儿嫁给了她的情夫?啊!我记起来了,是她被她那亲王丈夫休了……您装着给我讲话,省得这位贝雷妮丝\footnote{犹太希律王族的公主,与狄度热烈相爱,狄度曾欲娶之为妻,但在即罗马帝位后,因罗马人的反对被迫将她遣走。拉辛作有同名悲剧,高乃依则作为英雄喜剧著有《狄度与贝雷妮丝》。}来请我去吃饭。再说,我也得走了。您听我说,亲爱的夏尔,这回总算见着您了,您就不能跟我一起上帕尔马公主家去?她会是多么高兴,再说巴赞也要跟我在她家碰头的。要不是梅梅带来点您的消息……您想想,我现在根本就见不着您!”
\par 斯万没有答应;他早就告诉德·夏吕斯先生,他一离开德·圣德费尔特夫人家就直接回家去,他不想为了上帕尔马公主家去就看不到他一直在期待着的、由仆人送去或者留在门房里等待着他的那张便条。那天晚上洛姆夫人对她的丈夫说:“可怜的斯万哪,他还是那么亲切可爱,不过看样子挺倒霉的。您过几天会看到他的,他答应最近上咱家来吃饭。一个那么聪明的男人,为了那样一种女人而苦恼,我觉得真是荒唐。那女人一点儿意思也没有,有人说她是笨蛋。”说这种话,得有未堕入情网中人的那种清醒才行,这样的人认为一个有才智的人只能为值得为之憔悴的人才憔悴;要是有人为霍乱菌这样渺小的东西而甘愿染上霍乱,岂不是咄咄怪事!
\par 斯万想走,可就在终于可以脱身的时候,弗罗贝维尔将军却请他把德·康布尔梅夫人介绍给他,他就不得不跟他回到客厅去找她。
\par “我说啊,斯万,我宁愿安安稳稳在家里当这个女人的丈夫,也不愿被野蛮人宰了,您说呢?”
\par “被野蛮人宰了”这几个字刺痛了斯万的心;他马上就感到需要继续和将军谈一谈:
\par “是啊,很多人就是这样结束了自己的一生的。譬如说,您肯定知道,那位由迪蒙·德·乌维尔\footnote{迪蒙·德·乌维尔(1790—1842):法国航海家。}把他的骨灰带回来的那位航海家拉彼鲁兹(斯万讲到这里的时候感到很幸福,仿佛他是在说起奥黛特)。他是个好样儿的,我对他很感兴趣。”说到这里他都有点伤感了。
\par “啊!没有错。拉彼鲁兹谁不知道?有条街都是以他的名字命名的。”将军说。
\par “您认识拉彼鲁兹街上的人?”斯万兴奋地问。
\par “我就认得德·尚利福夫人,她是那位好样儿的肖斯比埃尔的妹妹。她有天举办了一个戏剧晚会,挺好的。她的沙龙今后会是很出色的,您瞧吧!”
\par “啊!她住在拉彼鲁兹街!这条街挺讨人喜欢的,挺美,挺冷清。”
\par “不,您大概有些时候不去了;现在不冷清了,那个区到处都在盖房子。”
\par 斯万最后把德·弗罗贝维尔先生介绍给年轻的德·康布尔梅夫人,这是她首次听到将军的大名,她匆匆摆出一个愉快和惊讶的微笑——这是对一个从来没有听说过的人的微笑;她新婚不久,对这家的朋友还不认识,别人领到她面前的每一个人,她都以为是家里的朋友,心想要是能装出自从她嫁到这家以后就常听人说起他的话,那就显得很得体,所以就不无犹豫地伸出手来,这犹豫既说明她在克服她早就学会了的含蓄,也说明那由于战胜了这犹豫而发自内心的友好情谊。就这样,她的公婆(她依然认为他们是法国最显赫的贵人)说她是个天使;他们特别要显示他们之所以挑中她做他们的儿媳妇,正是由于他们看中了她的人品,而不是她家巨大的家财。
\par “一眼就可以看出您有音乐的天赋,夫人。”将军对她说,不露痕迹地提起刚才蜡台托盘那档子事。
\par 音乐会继续进行,斯万知道他在这个新节目没有结束以前是脱不了身的。跟这些人一起被囚禁在这间屋里,他感到痛苦,他们的愚蠢和可笑刺痛着他的心,更何况他们不知道他在爱着一个人,而且即使知道,也不会感到兴趣,只能是笑他幼稚,惋惜他做出这等傻事;他们把他的那份爱情表现为只为他一个人存在的主观状态,缺乏任何外在的东西向他证明这是一个客观存在;他特别感到痛苦的是,他的奥黛特决不可能来到,所有的人和所有的东西对她都一概陌生,她完全不能涉足这个地方,而他还要持续流放下去,以至于乐器的声音简直要使他叫喊起来。
\par 突然间,奥黛特仿佛进来了;看到她的出现,他简直肝肠寸断,不由得把手捂住心口。原来小提琴奏出了高音,连绵缭绕,仿佛若有所待,这等待在继续下去,怀着已经瞥见它等待的对象从远处走将过来的激奋维系着那高亢的乐音,同时作出最大的努力持续到它的到达,在自身消失以前接待它的光临,竭尽全部余力为它敞开大路,让它过来,就好像我们用双手撑着一扇大门,阻止它自行关闭似的。斯万还没有来得及明白过来,还没有来得及对自己说“这是凡德伊的奏鸣曲中那小乐句,别听了”这句话时,直到那晚之前还得以掩埋在他心灵深处的对往昔奥黛特还爱着他的那些日子的回忆,却上了突然射出的一道光芒的当,以为爱情的季节已经回来,在他的心中又苏醒过来,振翅飞翔,向他纵情高唱已被忘却的幸福之歌,全然不怜悯他当前的不幸。
\par 过去他也常说“在我幸福的时日”、“在我得到她的爱的时日”,这些都是抽象的词语,说的时候也不感到特别难受,因为他脑际并没有在其中注入什么与过去有关的事物,只有一些虚妄的片断,并不保存什么实在的东西,而这一次重新找到的却是把失去的幸福中那特殊的、易于消失的精髓永远固定下来的一切东西;一切又都在他眼前重现:她扔进他的马车并被他举到嘴唇边的那朵菊花的雪白的卷曲的花瓣,上面写着“在给您写这信时我的手颤抖得多么厉害”的印有凸起的“金屋”两字的信纸,以及当她以恳求的口吻向他说:“我想不用再等多久您就会打发人来找我的吧”时那紧蹙的双眉;他又闻到在洛雷丹诺去给他找那个小女工、前理发师为他理发时,烫发钳发出的气味。那年春天暴雨来得如此频繁,他在月色下坐在他那四轮敞篷马车里冷得直哆嗦地回家;心理的习惯、季节的印象、皮肤的反应,这些东西构成一张大网,在一连好几个星期当中把他的整个身子都罩上了。在那时,他尝到那些除了爱情别无他事的人的种种乐趣,肉欲的追求也得以满足。他曾以为他可以永远如此,将来无需领略其中的痛苦;现在奥黛特的魅力跟那个像一个模糊的光晕那样笼罩着他的可怕的恐惧相比,已经微不足道了,而这光晕就是不能每时每刻都知道她在干些什么,不能随时随地占有她的那种焦躁不安。唉!他想起了她高叫“我随时都可以同您见面,我什么时候都是有空的!”时的那种语调,然而现在她却什么时候都没有空了!她对他的生活的兴趣和好奇,对答应她介入他的生活这种热切的愿望(他当时却怕它会引起可厌的打扰)也不复存在了!当初她必须苦苦哀求,他才答应让她领到维尔迪兰家去:当初他每月只让她上他家去一次,而她总得反复强调她梦寐以求的两人天天见面这个习惯将给她带来何等的快乐(而他却认为那是枯燥乏味的苦差使)之后,他才勉强答应她的要求,后来她却对这种习惯感到厌恶,彻底摆脱了,可他却已经把它看成是无法遏制的痛苦的需要。他记得当他第三次见到她时,她曾一再问道:“为什么不让我更经常地来看您?”他当时殷勤有礼地笑着答道:“我是怕来日徒然自苦呀!”唉!现在呢?她倒还是有时从饭店或者旅馆用带衔的信纸写封信来;可这些衔头上的一个个字都像火一样烧他的心。“这是在符耶蒙旅馆写的?她上那儿去干什么?跟谁去的?干了些什么?”他想起了意大利人大街正在一盏盏熄灭的煤气街灯,那时他已经失去了一切希望,竟在那几乎是神乎其神的夜里,在影影绰绰的人影中把她找着了(那天夜里,他几乎没有问如果去找她,又如果把她找着的话,是否会引起她的不快;他心里是那么确有把握,当她看见他,跟他一起回去时,她准会感到最大的快乐),而现在这个夜晚确实已经属于一个神秘的世界,它的大门已经全都关上,他再也无法重新进去了。斯万现在一动也不动地面对这重温的幸福,只见有一个不幸的人引起他的怜悯之心(因为他没有马上把他辨认出来),为了免得别人看见“他俩”热泪盈眶,便把头低了下去。这个人就是他自己。
\par 等他明白过来以后,他那怜悯之心也就随之消失,然而他妒忌她曾经爱过的另一个自己,妒忌他过去时常认为(然而心里也并不过分难过)“她也许在爱着”的那些人,因为他心中关于爱的空泛的概念(其实其中并没有爱情)已经由充满着爱情的菊花的花瓣和“金屋”餐厅信纸上的笺头取而代之了。他的痛苦之情愈来愈强烈,他抬手擦一擦前额,把单片眼镜摘下,擦拭擦拭镜片。毫无疑问,如果他这会儿能看到他自己的话,他会把他刚才像是摘下一个讨厌的念头那样摘下的单片眼镜,像是擦拭掉烦恼那样用手绢擦拭那蒙上水气的镜片的单片眼镜,补充到他刚才——加以区别的那一系列单片眼镜行列中去的。
\par 在小提琴声中——你如果看不到乐器的话,你就不能把所听到的声音跟乐器的形象联系起来,而乐器的形象是能改变乐器的音色的——有着跟次女低音一样的声音,使人产生有一位女歌唱家来参加这个音乐会的幻觉。你抬起眼来,却只见到那精致得跟中国珠宝盒一样的琴身,而且有时还能听到美人鸟迷人的歌声;有时也似乎听到被俘获的精灵在这中了魔法的颤抖的宝盒中,就像一个淹没在圣水缸里的魔鬼的挣扎声;有时又仿佛有一个神乎其神的纯洁的生灵在空中飘荡,展现它那看不见的启示。
\par 与其说乐师们在演奏那个乐句,倒不如说他们在举行为召唤这个乐句出现所需的仪式,在诵念为使它出现并使它的奇迹得以延续一些时间所需的咒语;斯万现在不再能看到它,除非它属于一个紫外线的世界,他在离它越来越近时却一时失明,只感到这一变化使他的精神为之一爽;他现在感到这个乐句出现在他面前,像是他的爱情的保护神和知情人,为了能在大庭广众之中走到他的跟前,把他拉到一边跟他絮语,而用这有声的外形把自己乔装打扮起来。当这乐句从他身边飘然而过,轻盈、安神,像鲜花的清香那样悄悄私语,倾心相诉,他仔细谛听每一个字,直惋惜话语如此迅速地飞逝,不由自主地用嘴唇去亲吻那和谐的,正在消逝的形体。他现在已经不再有遭流放的孤独之感了,因为乐句在跟他说话,悄悄地谈到了奥黛特。因为他现在不再像过去那样以为这乐句不认识奥黛特和他了。它曾如此经常地目睹过他俩在一起时的欢乐情景!不错,它也时常提醒他这种欢乐的不实在,会稍纵即逝。甚至就在那时,他也在乐句的微笑中,在它清澈的促人醒悟的声调中窥出了痛苦的苗头,而他今天从中觅得的却几乎是高高兴兴的听天由命的甘美。当年这乐句曾跟他谈起过悲伤的事,他自己虽未被波及,只见到乐句带着微笑把它们在它曲折湍急的激流中冲泻而下,而现在这些悲伤的事却是他亲自尝过的了,而且没有希望得以摆脱。这乐句仿佛也像当年说到他的幸福时一样,对他说:“这有什么关系?这算不了什么。”斯万心里第一次浮现对这位凡德伊,对这位本身多半也曾尝过苦涩滋味的,从不相识的崇高的兄长的怜悯与柔情;他度过了怎样的一生?他是从怎样的痛苦中汲取了神般的力量,汲取了无穷的威力来创作的?当这小乐句对他谈起他的痛苦的虚妄时,斯万体味到这箴言的甘美,但就在片刻以前,当他从把他的爱情看做是无关紧要的闲事的那些不相干的人的脸上窥出这种意思的时候,他却觉得这条箴言难以容忍。那是因为那个小乐句,与此相反,不管它对心灵的这些状态的短暂易逝表示了什么见解,它从中所看到的却跟这些人不一样,并不是没有实际生活那么严肃的东西,相反却是远远高出于生活的东西,是唯一值得表现的东西。这个小乐句试图模仿,试图再创造的是内心哀伤的魅力,而且要再现这种魅力的精髓;除了亲身感受这种魅力的人之外,任何别人都认为它是不能传达,也是毫无价值的;这个小乐句却把它的精髓抓住了,把它化为可以看见的东西。它使得它的听众只要多少有点音乐细胞,就能承认这种魅力的价值,尝到它的神奇的甘美,然而日后在他们身畔看到的每一个特定的爱情当中,他们却又看不到这种魅力了。当然,这小乐句把这种魅力组编起来的形式是不能化为逻辑的推理的。但一年多以来,对音乐的爱好向他揭示了他心灵中的许多宝贵财富,至少在一段时间之内在他身上生根发芽,斯万从此就把音乐的主旨看成是真实的思想,是另一个世界、另一种类型的思想,蒙着黑影、不为人所知、智力所不能窥透的思想,然而这些思想依然是完全可以相互区别,各有不同的价值与意义。自从他在维尔迪兰家那次晚会上请人把那乐句再奏一遍以后,他竭力想弄清这乐句是怎样像一股清香、一次搂抱那样迷惑他、缠绕他的,他终于意识到那个收缩了的、冷冰冰的甘美之感得之于组成这乐句的那五个间距很小而其中两个又不断重复的音符;可事实上他不知道,他这番推理并不是从这小乐句本身得来,而是在认识维尔迪兰夫妇以前的一个晚会上第一次听到这个奏鸣曲时,由于懒得动脑筋而用来解释他所探索的音乐这个神秘实体的简单的标准。他也知道,在他回忆之中的钢琴的乐声就越发歪曲他观察与音乐有关的事物的观点,而且展现在音乐家面前的天地并不是仅有七个音符的可怜的键盘,而是一个无限宽广的键盘,几乎还完全未为人所知,只是星星点点地散布着千千万万表现温柔、激情、勇气和安谧的琴键,中间被层层从未被我们探索过的黑暗所阻隔;这些琴键彼此之间有天地之别,只为少数伟大的艺术家所发现;他们在我们心灵深处唤醒了跟他们发现的主题相应的情感,告诉我们,在我们原以为空无一物的心灵这个未被探索,令人望而生畏的黑暗中却蕴藏着何等丰富多彩的宝藏而未为我们所知。凡德伊就是这样的音乐家中的一个。他那个小乐句虽然为我们的理性设置了一层薄膜,但我们还是可以感到它如此充实、如此明确的内容,它又给这内容以如此新鲜、如此独特的力量,使得听众把乐句和凭智力获得的思想一视同仁地保存在心中。斯万每次想到这个乐句,就仿佛是想到了爱情观和幸福观,马上就能从中体会到它的特点,就如同一想起《克莱芙公主》和《勒内》\footnote{《克莱芙公主》作者是法国十七世纪女作家拉法耶特夫人,被认为是法国第一部心理小说杰作。《勒内》则是十九世纪法国浪漫主义作家夏多布里昂的作品。}这两个标题就知道它们的特点一样。即使在他不想到这个小乐句时,它也跟一些无可替代的概念(例如光、声、凹、凸、肉欲这些概念)处于同等地位,潜伏在他的心灵之中,而我们的内心世界之所以如此多彩多姿,绚丽斑斓,正是由于这些丰富的精神财富。假如我们一命归天,我们也许就将失去这些财富,它们也许会自行消失。但只要我们活着,我们就不可能不认识它们,正如我们不可能不认识一个具体的物体一样,也正如当我们的房间里点上了灯,虽然屋里的物体都变了样,对黑暗的回忆也已不复存在,我们却不可能怀疑灯光的存在一样。就这样,凡德伊的这个乐句,正如《特里斯坦》\footnote{全名为《特里斯坦与依索尔德》,是十九世纪德国作曲家瓦格纳所作歌剧,歌颂死亡和黑暗,充满叔本华的悲观主义色彩。}的某个主题(它为我们表现了心灵的感受)一样,也歌颂死亡,也体现了相当动人的人生景象。这个乐句的命运,日后是要跟我们的心灵的现实联系在一起的,它是我们心灵的最特殊,又最各不相同的装饰物之一。也许只有虚无才是真实的东西,而我们的梦幻并不存在,然而那时我们就会感到,那些与我们的梦幻相关连而存在的乐句和概念也就不复存在了。我们终究会死去,但是我们手上有这些神奇的俘虏作人质,他们将在我们生存的机会丧失时继续存在下去。有了他们,死也就不会那么凄伤,不会那么不光彩了,甚至不会那么太肯定了。
\par 斯万相信那个乐句的确存在着,他没有错,当然,从这个观点来看,它是人间的东西,然而它却属于一种超自然的创造物的世界;我们虽然从来没有见过这种创造物,但当有某位探险家探索这不可见的世界,捕捉到一个这样的创造物,从他进入的这个神奇世界中带到我们这个尘寰的上空闪耀出片刻的光焰,我们看到时是会欣喜若狂的。凡德伊用他那个小乐句所做的就是这样一件工作。斯万感到,作曲家只是以他的乐器把它揭露出来,使它成为清晰可见,以他如此轻柔、如此审慎、如此细腻、如此稳健的手忠实描绘出它的轮廓,使得音响随时变幻,有时变得模糊黯淡以表现一个幽影,而当它必须勾勒奔放的轮廓时又重新活跃欢腾起来。斯万相信那个乐句确实存在,这有事实可以证明:如果凡德伊看见那个乐句,把它的形式描绘出来的能力较差,而竭力在一些地方凭他臆想添上几笔来掩饰他观察的不到和技巧的欠缺,那么,任何一个耳朵稍为灵敏一点的音乐爱好者就会发现他的骗局。
\par 乐句消失了。斯万知道,它还将在最后一个乐章的结尾出现,其间要隔着很长一段乐曲,而维尔迪兰夫人家中那个钢琴家老是把这一段跳过。这一段里有一些美妙的思想,斯万在第一次听时未能辨认出来而现在却发现了,仿佛这些思想在他记忆的衣帽间中突然把掩盖着它的新颖之处的外衣脱掉了似的。斯万听着那分散的主题组成乐句,正如三段论法中的前提演绎为必然的结论,他亲眼目睹这乐句的生成。他心想:“噢!凡德伊的大胆敢情跟拉瓦锡\footnote{拉瓦锡(1743—1794):法国化学家,建立了化学命名法,发现氧在燃烧中的作用,提出物质守恒定律。}和安培\footnote{安培(1775—1836):法国物理学家、数学家,电动力学的创始人。}一样,都是得之于天才的启发!他试验并发现了掌握着那未为我们所知的力量的规律,把他信赖不移但永不能见的无形的巨车,驶过从未探测过的地域,奔向那唯一可能的目标!”斯万在最后一段开始时听到的钢琴与小提琴之间的对话是多么美啊!虽然摒弃了人间的词语,却并不像人们想象的那样让幻想主宰一切,恰恰相反,这里却排除了幻想;从来也没有像这里这样更迫切需要对答的语言,然而问题从来也没有像这里这样提得如此贴切,回答也从来没有像这里这样明确。首先是钢琴独自哀怨,像一只被伴侣遗弃的鸟儿;提琴听到了,像是从邻近的一株树上应答。这犹如世界初创的时刻,大地上还只有它们两个,也可以说这犹如是根据造物主的逻辑所创造,对其余的一切都关上大门,永远是只有它们俩的世界——这奏鸣曲的世界。钢琴紧接着又为那个看不见的、呻吟着的生灵倾诉哀怨,可那生灵到底是什么?是一只鸟?是那小乐句还是不完整的灵魂?还是一个仙女?那叫喊声来得是如此突然,提琴手得赶紧抓起琴弓来迎接。真是一只神奇的鸟儿!提琴手像是想迷住它,驯服它,抓住它。它已经深入到他的心灵,由它召唤的那个小乐句已经使得提琴手那当真着了魔的身体像通灵者一样颤动起来。斯万知道这小乐句就要再次向他倾诉了。而这时他自己早已分裂成为两人,以至在等待他即将面临这乐句的时刻到来时,不禁哽咽起来,就像我们在读到一行美妙的诗句或者听到一个伤心的消息时那样——而且并不是当我们只身独处的时候,而是仿佛在把这诗句或这消息告诉给我们的朋友们的时候,在他们身上,我们看到我们自己成了一个情绪能影响他们的第三者。乐句又重新出现了,但这次是高悬空中而且一动也不动地仅仅持续了片刻,立即又消逝了。它延续的时间是如此短暂,斯万的眼睛连眨都没眨一下。它还像一个完整充实的虹色水泡那样悬着。又像一道彩虹,光泽逐渐减弱黯淡,然后又升腾起来,在最后归于消失以前,大放前所未见的异彩:它原先还只露出两种色彩,现在又添上棱镜折射出的所有绚丽多彩的琴弦,奏出动人的曲调。斯万不敢动弹,他也希望别人也都像他那样安安静静,仿佛稍有动静就会破坏这随时都会消失的美妙脆弱的、神乎其神的幻景。说真的,谁也不想开口。那一个不在场的人(也许是一位死者,因为斯万不知道凡德伊是否还在人世)的美妙得难以言传的话语,在这些祭司们的头上回荡,足以吸引住在场的三百人的注意,把这个召唤阴魂的乐台化为举行神奇仪式的庄严的祭坛。就这样,当乐句终于结束,只剩下袅袅余音在随后取而代之的旋律中回荡时,斯万先还为那愚蠢得出了名的蒙特里安德伯爵夫人在奏鸣曲还没有完全终止时就俯过身来对他讲说她的感想而恼火,后来却禁不住微微一笑,也许是为在她的话语中发现了她自己所未曾体会到的更深的含义而高兴。伯爵夫人对演奏者的高超演技赞叹不已,冲着斯万嚷道:“真是奇怪啊,我从来没有见过这么神的……”她怕把话说得太绝,又找补了一句:“只有招魂时用的灵动台才是例外!”
\par 从这次晚会以后,斯万明白奥黛特往日对他的感情是永远不会恢复了,他过幸福生活的希望是再也不能实现了。有些日子,她偶尔对他亲切温柔,多少对他表示一点关心;他把她这些回心转意的表面的、虚假的表示一一记下,就好比那些侍候着身患绝症行将离世的病人的朋友,怀着那种充满温情和怀疑色彩的关切以及毫无希望的欢乐,记下这样的话当做无比宝贵的事实:“昨天他都自己会算账了,指出了我们计算中的一个错误;他还高高兴兴地吃了一个鸡蛋,如果消化得好,我们明天想给他一块排骨试试。”尽管他们自己也明明知道,对于一个死亡已经不可避免的人来说,这样的事情已经毫无意义。斯万心里当然也明白,如果他现在离开奥黛特生活的话,他对她就会越来越淡漠,就会乐于看到她永远离开巴黎;到时候他自己就会有呆在巴黎的勇气,可是他却没有勇气先走开。
\par 斯万原也常有这样的想法。现在他已经恢复对弗美尔的研究,他至少应该再到海牙、德累斯顿、不伦瑞克去些日子。他深信,在哥德斯密特拍卖时由毛里茨博物馆\footnote{在海牙。}当做尼科拉斯·马斯\footnote{尼科拉斯·马斯(1632—1693):荷兰画家。}的作品买去的那幅《狄安娜的梳妆》,实际出自弗美尔之手。他很想就地进行一番研究来加强他的信念。然而当奥黛特在巴黎的时候(甚至当她不在的时候),要她离开巴黎,在他看来可是一个如此残酷的计划,他是明知自己永远也下不了决心去实现,所以才能经常放在心里盘算的——换到一个新地方,我们的感觉还没有被习惯冲淡,我们随时都会唤起原有的痛苦,使它加剧。不过他有时还在睡梦中萌生外出旅行的打算(全无影响根本是不可能的),居然还得以实现。有天他梦见他要外出一年,倚在车厢窗口,冲着站在月台上哭着向他道别的青年,劝那青年跟他一起上路。列车晃动,他也惊醒了,意识到他并没有出家门,而且当晚,第二天还有以后几乎每天都会见到奥黛特。那时,梦境依然萦回在他心头,他赞美自己那些优越的条件,使他生活不必依赖他人,可以呆在奥黛特身边,使得她允许他有时去看她;他把他这些优越的条件列举一番,其中有:他的社会地位、他的财产(她时常有迫切需要,所以不能同他破裂,而且耳闻她有跟他结婚的意思),他跟德·夏吕斯先生的交情(虽然其实并没有使他从奥黛特那里得到多大好处,但他是他们俩共同的朋友,奥黛特对他很是敬仰,有这样一位朋友在她面前说他的好话,他想着也不无温馨之感),还有他自己的聪明才智,他是全部用来每天安排巧计,使得奥黛特觉得有他在身边陪伴虽不一定是赏心快事,至少是必不可少的。他想,要是这些条件全都没有的话,他会变成什么样子;他想,要是他像许多人那样贫穷、低微、一无所有,不得不有什么工作就干什么工作,或者是依赖父母或妻子,他早就不能不离开奥黛特,心有余悸的那场梦就会变成现实。他心想:“人总是生在福中不知福。他们也决不像他们自己所想的那么不幸。”但他又想,他现在这种生活已经持续了好几年了,他所期望的也就是这种生活能持续下去,继续牺牲他的工作、他的乐趣、他的朋友,最后是牺牲他的一生来每天都期待一个并不能给他带来任何幸福的约会;他还想,他这样做是不是错了,凡是促进他俩的关系,防止其破裂的一切是不是在毁坏他的前途,他所应该期求的是不是正是他现在庆幸仅仅是梦中发生的事情,也就是他的离去?他心想,人总是生在祸中不知祸,他们也决不像他们自己所想的那么幸福。
\par 有时他盼望她在意外事故中没有痛苦地死去,因为她是从早到晚总在外面,在街上,在大路上的。当她安然无恙回来时,他不禁赞叹人的身体是如此灵活和结实,总能摆脱一切灾难(自从斯万有了这个隐秘的念头以后,他觉得这样的灾难是数不胜数的),使得人们天天都能几乎不受惩罚地从事他们撒谎、追求欢乐的勾当。斯万对由贝里尼作肖像的穆罕默德二世深表同情,后者对他的一个后妃爱得发狂,就用匕首把她刺死。据为他作传的威尼斯人不加掩饰地说,这是为了求得他心灵的平静。然后斯万又为他只想到自己而深自愧恨,觉得他居然把奥黛特的生命视若草芥,自己感到痛苦也是活该,一点也不值得怜悯。
\par 既然他不能义无反顾地离开她,那么,假如他继续见到她而不分离的话,至少他的痛苦终将减弱,而他的爱情也许终将熄灭。既然他不愿永远离开巴黎,他就希望她永不离开。既然他知道她每年离开巴黎时间最长是在八九两月之间,那么他眼前还有好几个月的余暇来把这苦涩的念头溶解在他脑子里遥想的时日当中,这些时日和当前的时日一模一样,在他饱含哀愁的心中流逝,透明而寒冷,然而并不引起他过分强烈的痛苦。但这心中构想的未来,这条无色而奔放的长河,奥黛特的一句话就把它击中,像一块寒冰似的把它堵住,阻止它流动,使它整个凝冻起来;斯万突然感到心里堵满了一块巨大而坚不可破的东西,挤压他身体的内壁,直到使他全身爆裂:原来奥黛特带着狡黠的微笑对他说:“福什维尔到圣灵降临节时要出外旅行。他要到埃及去,”斯万顿时就明白,这话就意味着“到圣灵降临节时我要跟福什维尔到埃及去”。果不其然,过了几天,斯万问她:“嗯,你那天说要跟福什维尔同去的那次旅行怎么样了?”她冒冒失失地答道:“对了,亲爱的,我们十九号就动身,我们会寄给你金字塔的图片的。”那时他想弄清楚她是不是福什维尔的情妇,要当面问个明白。他知道她迷信,有些伪誓是不会起的,而且迄今为止,他一直担心当面问她会使她恼火,遭她讨厌,然而现在他已经失去了得到她爱的一切希望,这种担心也就不复存在了。
\par 有一天,他收到一封匿名信,说奥黛特曾是无数男人的情妇(信上列举几个人,其中有福什维尔、德·布雷奥代先生,还有那位画家),还是一些女人的情妇,而且还进妓院。他为在他的朋友当中居然有人会给他写这样一封信而感到痛苦(从信上的某些细节看来,写信的人对斯万的私生活是很了解的)。他琢磨这是谁干的。他从来没有猜测过别人在背后干些什么,从来没有怀疑过别人那些跟他们的言语挂不上钩的行动。德·夏吕斯先生、洛姆亲王、德·奥尔桑先生,他们当中哪一位也从来没有在他面前说过他们赞成写匿名信的话,他们所说的都表示他们是强烈谴责匿名信的,这样一种卑劣的行径莫非出自他们公开的性格背后的什么地方?他看不出有什么理由把这种无耻勾当跟他们当中任何一人的品格联系起来。德·夏吕斯的性格有点不正常,然而基本上是善良厚道的;洛姆亲王虽然冷漠,但身心健全,为人正直。至于德·奥尔桑先生,斯万从来没有见过有谁,即使是在最惨的处境中,会看到他讲出言不由衷的话,做出不得体、不妥当的举止。有人说德·奥尔桑先生在跟一个富有的女人的关系当中有不正当的表现,斯万总难于理解,每当他想到他的时候,他总不得不排除他那个坏名声,认为它跟他那些数不胜数的高尚正直的表现无法协调。斯万一时觉得他的脑子越来越糊涂,他就想点别的事情,好看得清楚一些。过了一会儿,他又有勇气来继续那番思考了。他刚才既不能怀疑任何人,到这时候就只好怀疑所有的人了。归根到底,德·夏吕斯先生是爱他的,心地不坏。然而他有神经病,当他明天听说斯万病了的时候,他可能会难过得哭将起来,然而今天呢,也许出于妒忌,也许出于气愤,一时心血来潮,就要对他使坏。说到头,这号人最糟糕。洛姆亲王对他的爱当然远不及德·夏吕斯先生,但也正由于此,他对斯万也就没有那么强烈的感情;再说,他生性冷漠,既不会做出豪迈之举,也不会干出卑鄙龌龊的勾当;斯万都后悔尽跟这一号人泡在一起了。他又想,阻止一个人对他周围的人使坏是同情之心,而他终究只能保证本性跟他相同的人有这样的心,譬如就心地善良来说,德·夏吕斯先生就是这样一个人。对斯万造成这样一种痛苦,单单这一个念头就会使德·夏吕斯先生产生反感。然而对一个感情冷漠,不怎么太通人情的洛姆亲王来说,在不同的本质的驱使下,可能会干出什么事来,谁又能预料到?心地好是最主要的,德·夏吕斯先生的心地就不错。德·奥尔桑先生心地也不错,他跟斯万的关系虽不亲密但还是真诚的,是由于他们对什么事情都有一致的想法,所以乐于在一起絮叨;他们之间的关系比较平和,不像德·夏吕斯先生那样激昂,那样易于做出一时冲动的事情来,不管是好事还是坏事。如果说有谁是斯万过去一直感到能被他所了解,能身受其体贴爱护的话,那就是德·奥尔桑先生了。不错,不过他过的那种不大体面的生活又如何解释呢?斯万现在感到遗憾,他从前竟从来没有予以考虑,时常还以开玩笑的口吻说什么他只有在流氓集团里才能看到强烈的同情和尊敬的感情。现在他却想,人们判断别人,从来都是根据他们的行为,这并不是没有道理的。只有行为才有意义,我们说的和想的都算不了什么。夏吕斯和洛姆可能有这样那样的缺点,可他们是老实人。奥尔桑也许没有缺点,可他不是老实人。他可能又一次干了坏事。斯万又把雷米怀疑起来,不错,他只可能是授意别人去写,但他显然觉得那路子是走对了。首先,洛雷丹诺有理由恨奥黛特。其次,我们的仆人地位比我们低,以为我们除了家产之外还有什么财富让他们眼红,除了缺点之外还有什么罪恶让他们瞧不起,又怎能设想他们最后不会干出我们上等人干不出的事来呢?斯万还怀疑我的外祖父呢。斯万每次求他帮忙,他不总是拒绝吗?而且以他那资产阶级的脑筋,还以为这都是为斯万好呢。斯万还怀疑贝戈特,怀疑画家,怀疑维尔迪兰夫妇,而在怀疑之中他再一次赞赏上流社会人士真是聪明,他们不愿和艺术界的人士打上交道,而在艺术界里这样的事不仅可能发生,甚至也许被认为是巧妙的玩笑而受到肯定;但他这时也想起了那些波希米亚人,他们的行动是何等光明正大,而与此恰成鲜明对比的是贵族阶级,他们在手头缺钱,又要摆阔气、花天酒地时又是如何经常背弃原则,便宜行事,简直是尔虞我诈!总之,这封匿名信表明他认识一个能干得出这等卑鄙行径的人,然而他看不出为什么这样的卑鄙心理就更有可能隐藏在热心肠人、艺术家、贵族的心灵深处(为他人所探测不出),而不是在冷漠的人、买卖人、仆役的心灵深处。应该采用什么标准来判断一个人呢?归根结蒂,他所认识的人中间,没有哪一个是不能做出可耻的行动来的。是不是应该跟他们全都不再来往呢?他闹不清楚了;他一再抬手拍拍脑门,用手指擦拭单片眼镜的镜片,心想有一些并不比他差的人也跟德·夏吕斯先生、洛姆亲王和别的一些人交往,这就表明,即使他们并不是不可能做出可耻的行动,至少每个人都必须遵从的那个生活的必然是要求我们跟并非不可能做出可耻的行动的人们交往的。于是他就跟所有他怀疑过的朋友继续握手,只是带点保留态度,认为他们也许曾经想陷他于绝望之境——不过这种保留态度也只是徒具形式罢了。
\par 至于信的内容,他并不为之不安,因为其中列举奥黛特的罪状没有一丝真实的影子。斯万跟许多人一样,懒得动脑筋,也缺乏想象力。他清楚地知道,人们的生活充满着矛盾,这是一条普遍真理,但具体到特定的人身上,他就把对方生活中他所不知道的部分,设想成跟他所知道的那部分完全一致,他借助于对方跟他讲的话来设想他没有跟他讲的那些话。当奥黛特在他身边的时候,如果他们谈起别人有什么不正当的举止或者粗俗的情感的话,她总是用斯万的父母从小教导他而他也始终恪守的原则来指责他们的;再说,她也爱摆弄个花,爱喝杯茶,关心斯万的工作。因此,斯万就把奥黛特的这些习惯推而广之于她的生活中的其他部分,当他要想象她不在他身边时是什么情景的时候,他就在脑海里重复她那些姿态。假如别人描绘的情景跟她在他身边(或者毋宁说是曾经那么长时期地在他身边)的情景一样,然而是跟另外一个男人在一起,那他是会感到痛苦的,因为在他心目中,这个形象是逼真的。然而要说她进妓院,跟一些女人在一起狂欢作乐,过着卑鄙下流、荒淫无耻的生活,那就是荒诞无稽的胡说八道;谢天谢地,他想象中的朵朵菊花,她每日品饮的杯杯红茶,她在不义之举面前的填膺义愤,是不可能给这一派胡言的实现留下余地的。不过他也时不时地告诉奥黛特,别人是怎样出于恶意,把她的所作所为说给他听的;同时他也顺带用上点他偶尔听到的无关紧要然而却是真实的细节,仿佛他对奥黛特的全部生活都了如指掌,只是秘而不宣,无意中露了这么一点,让人以为他掌握什么情况,其实他既不了解,甚至连想都没有想到;而他之所以经常恳求奥黛特不要歪曲事实,只是为了——不管他自己意识到与否——让奥黛特把她的所作所为全都告诉他罢了。不错,他也常对奥黛特说,他爱真诚坦率,其实,他是把他所爱的真诚坦率看成是一个能把他情妇的日常生活向他密报的拉皮条的人。因此,他对真诚坦率之爱并非超脱功利,也未能使他的人品变得更加高尚。他所珍爱的真实是奥黛特告诉他的真实;而为了得到这个真实,他不惜借助于谎言,而他却经常对她说,谎言是如何陷人于堕落之境的。总之,他撒起谎来并不亚于奥黛特,因为他比她更不幸,也不比她少自私些。而奥黛特呢,当她听斯万对她本人讲起她干过的一些事情时,总是带着一副猜疑的神色瞧着他,偶尔露出愤怒之情,来遮掩她的羞耻之心。
\par 有一天,正当他难得心境平静了一个长时间而未生妒意的时候,他接受洛姆亲王的邀请,晚间陪他去观剧。他想知道上演的是哪个剧本,就把报纸打开,泰奥多尔·巴里埃尔的《大理石姑娘》这个名字赫然跃入眼底,狠狠地击中他的心坎,他不由得倒退一步,扭过头去。“大理石”这个词往常是如此经常映入他的眼帘,以至反倒是一晃而过,视而不见,现在在它出现的那个地方却像在舞台脚灯照射之下,突然如此夺目,叫他马上想起了奥黛特有次给他讲起的那个故事,说的是有回她跟维尔迪兰夫人一起上工业展览馆参观,这位夫人对她说:“你小心点儿!我可是知道怎样把你融化掉的。反正你不是大理石做的。”奥黛特当时对他说这不过是开个玩笑,斯万也没怎么在意。那时候他对她的信任比现在强多了。而那封匿名信却恰恰讲到了这一号恋情。他不敢抬眼看报,把它打开翻过一篇,躲开《大理石姑娘》这几个字,开始心不在焉地读起各省新闻来了。芒什省有暴风雨,第厄普、卡布尔、布士伐尔遭灾。他又怔了一下。
\par 布士伐尔这个名字叫他想起了这个地区的另一个地名,叫布士维尔;后者又与布雷奥代这个名字相关,他常在地图上看到,可这是第一次注意到它跟他的朋友德·布雷奥代先生的名字一样,而那封匿名信上说他也曾是奥黛特的情夫。再怎么说,对德·布雷奥代先生的指责并非全不可信;而说她跟维尔迪兰夫人有暧昧关系,那就完全不可能了。奥黛特固然有时撒谎,可不能从中得出结论,说她从来不讲真话,在她跟维尔迪兰夫妇讲过的话,以及她自己向斯万转述的那些话中,他也曾听到过女人们由于生活经验的缺乏和对罪恶的无知而开的一些没有多大意思然而不无危险的玩笑(这些话显示了她们的清白)。她们这样的人,譬如说奥黛特吧,她比谁都更不至于对另一个女人产生狂热的恋情的。与此相反,当她把她在转述时无意间在他心中引起的怀疑加以否定时的那种愤怒之情,倒是跟所知道的他这位情妇的格调和气质相一致的。然而在此刻,由于一阵突如其来的醋意——这就好比一个刚想到一个韵脚的诗人或者一个仅仅掌握一个零星观察结果的学者,忽然得到一个思想或者找到一条规律,给了他们以全部的力量——他第一次想起了奥黛特早在两年前跟他讲的一句话:“哦!维尔迪兰夫人哪,这会儿心里就只有我一个,我成了她的心肝宝贝,她吻我,要我陪她去买东西,要我对她以你我相称。”当时他根本没有想到这话跟奥黛特在他面前为了掩饰那有伤风化的勾当而讲的那些话有什么关系,只觉得这证明她俩交情很深罢了。现在维尔迪兰夫人对奥黛特那种柔情的印象突然跟她这番味道不正的话结合起来了。他脑子里再也无法把那印象跟这番话分离开来,只见两者在现实中也交织在一起,那种柔情给那些玩笑话注入了认真的要紧的东西,而那些玩笑话也就使那种柔情显得不那么清白了。他直奔奥黛特家。他离她远远地坐下。他不敢拥抱她,拿不稳这一吻在她或他身上激起的将是深情还是怒火。他沉默不语,眼睁睁地瞧着他们之间的爱情死去。他忽然下定了决心。
\par “奥黛特,”他对她说,“亲爱的,我明知道我使你讨厌,可我还得问你点事情。你还记得我曾经怀疑过你跟维尔迪兰夫人之间有什么关系吗?告诉我,到底有没有?跟她或者别的女的有没有?”
\par 她撅起嘴摇摇头,这是人们回答别人“您来看节日游行吗?”或者“您来看阅兵吗?”这样的问题,表示不去或者讨厌这些事情时常用的姿势。这种摇头,通常是用来表示不愿参加未来的活动的,因此在否定过去的事情当中也掺入了一点犹疑的味道。再说,这种摇头只表示这事对个人合适不合适,并不表示对它的谴责或者从道德观点出发认为它不可能的。斯万见她作出否认的姿态,心里明白这也许反倒是真事。
\par “我早就跟你说过了,你不是不知道。”她又找补了一句,一脸气恼和倒霉的神色。
\par “不错,我知道,不过你是不是确实拿得稳?你别说什么‘你不是不知道’,你说‘我从来没有跟哪个女人干过那档子事’。”
\par 她像背书一样重复了一遍,语含嘲讽,也显出她是要把他打发走:
\par “我从来没有跟哪个女人干过那档子事。”
\par “你能凭你的拉盖圣母像起誓吗?”
\par 斯万知道奥黛特是不会凭这个圣母像起伪誓的。
\par “啊!你把我折磨得太苦了!”她叫道,一面闪到一边,仿佛是要躲开这个问题似的,“你有完没有完?你今天是怎么啦?莫非是下定决心要我讨厌你,恨你?好嘛,我正要跟你和好如初呢,而你却这样来谢我!”
\par 可斯万不想把她轻易放过,坐在那里像个外科医生那样,等待刚才打断手术进行的那阵痉挛过去,继续开刀:
\par “你以为你说了我就会对你有一星半点的怨恨,那你可错了,奥黛特,”他以想说服人的虚情假意的轻声柔语对她说,“我跟你说的都是我知道的事情,而我知道的事情比我说出来的要多得多。这些事儿都是别人对我说的,只有你的坦白才能减轻我对你的恨。我所以生气,不是由于你的行动,我既然爱你就会原谅你的一切,而是由于你的虚伪,你那毫无道理的虚伪,使得你一个劲儿否认我所知道的事情。当我见到你在我面前坚持我明明知道是假的事情,还要起誓赌咒,你怎能叫我继续爱你呢?奥黛特,这时刻对你我都是痛苦的折磨,别让它再继续下去了。只要你愿意,一秒钟就能了事,到时候你就永远解脱了。你指着圣母像告诉我,你是不是干过那档子事。”
\par “我压根儿也不知道,”她愤怒地叫道,“也许很久很久以前,连我自己也莫名其妙呢,可能有这么两三回。”
\par 斯万早就盘算过各式各样的可能性。现在的现实却跟那些可能性并无丝毫关系,就跟我们身上挨了的一刀跟在我们头顶上飘动的浮云并无丝毫关系一样——“两三回”这几个字确像是一把尖刀在我们的心上画了一个十字。“两三回”这几个字,单单是这几个字,在我们身体之外发出的这几个字,居然能跟当真触到我们的心一样,把它撕碎,居然能跟吃的毒药一样使我们病倒,真是一件怪事!斯万不由自主地想起在德·圣德费尔特夫人府里听到的那句话:“自从看了招魂时用的灵动台以来,这是我见过的最神的奇迹了。”他现在感到的痛苦远远超过了他的想象。这倒不仅仅因为当他对奥黛特最不信任的时刻,他难以想到她在恶行这条路上能走得那么远,而也是因为,即使当他设想这等事的时候,那也是模糊的不肯定的没有感受到从“可能有这么两三回”这几个字当中散发出来的那种特殊的恐惧,没有当你首次听到你得了某种疾病时那种从未体会过的特殊的残酷。他这种痛苦完全来自奥黛特,然而奥黛特在他心目中并不因此而有欠可爱,反而更弥足珍贵,仿佛是痛苦越深,唯有这个妇女身上才有的那种镇痛剂和解毒剂的价值也水涨船高。他要给她以更多的照顾,仿佛突然发现自己身上的某种病痛比原来设想的还要严重。他希望她说曾干过“两三回”的那种丑事不再重犯。为此,他必须密切照看着她。人们常说,你要是向你的朋友指出他的情妇犯了什么过错,只能使他跟她更加接近,因为他是不会信你的,而他如果信了你,那就跟她贴得更紧了!斯万心想,他怎样才能保护她呢?他也许能使她不受某一个女人的影响,可是还有几百别的女人呢!他也想起,在维尔迪兰家没有找见她的那晚,他曾一时起念要去占有另一个女人(其实是办不到的),现在看来这念头是何等荒唐。幸好在这像一伙伙入侵者那样刚侵入斯万的心灵的新的痛苦底下,还有一层由天性构成的基础,它历史悠久、温和宁静、一声不响地在起着作用,犹如一个受了伤的器官的细胞立即来修补遭到损坏的组织,也犹如一个瘫痪的肢体上的肌肉总有恢复原有机能的趋势。他心灵中的这些资格较老、土生土长的居民,一时间把斯万的全部力量投入这不声不响的恢复元气的工作——正是这样的工作使得一个康复中的病人,使得一个刚接受过手术的病人一时感到安详。这一次跟平常不一样,这种由于精疲力竭而感到的松弛,与其说是出现于他脑际,倒不如说是出自他的心田。生活中所有曾经一度存在过的东西都一一在心中重现,而还是那份痛苦之情,就像是一头垂死的牲口为似乎已经终止的抽搐的惊跳所驱,刚平静了一会儿,又来到斯万的心上画了一个十字。他猛然想起那些月夜,他躺在他那辆驶往拉彼鲁兹街的敞篷马车上,纵情畅想恋人的种种欢乐,全然不知这些欢乐将必然带来什么毒果。但所有这些念头都仅仅一闪而过,也就是把手举到心口,缓过气来,强自微笑来掩盖他的痛苦那一会儿工夫罢了。这时他都已经又开始提出他的问题来了。他的醋意为了给他这样一个打击,使他经受还从未经受过的最惨烈的痛苦,简直比一个死敌还要不惜费上九牛二虎的气力,这时依然觉得他受的苦还不够,还要想方设法让他受到更深的创伤。他的醋意像一个邪恶的鬼神给他以启示,把他推向毁灭的边缘。如果说他受的罪在开始的时候还并不很重的话,那不是他的错,而仅仅是奥黛特的错。
\par “亲爱的,”他对她说,“现在就算完了;对了,那人我认识吗?”
\par “不,我发誓根本没有那么回事,我刚才是言过其实了,我并没有走到那一步。”
\par 他微微一笑,接着说下去:
\par “听便,没有关系,不过你不能把她的名字告诉我,实在遗憾。你要是能把她是怎么样一个人跟我讲讲,那就省得我再在这方面费心思了。这是为你好,你说了,我不是就不再麻烦你了吗?心里有什么事,一旦弄明白了,就像是一副担子落了地。要是琢磨不出是怎么回事,那才难受呢。不过你刚才对我已经就不错,我不愿再烦你了。我衷心感谢你对我的好处。这就算完了。只不过还有一个问题:那是几时的事情?”
\par “啊,夏尔!你真是烦死我了!那是早辈子的事了。我压根儿就从来没有再想过。你不把那些念头重新塞到我脑子里来就不罢休是不是!你这是有心使坏,无意中干了蠢事,没有你什么好处。”
\par “啊!我刚才只是想知道这是不是在我认识了你以后发生的事情。事情仍然就是在这里发生的了?你就不能告诉我那是哪个晚上,好让我想想那天晚上我在干什么?奥黛特,我的宝贝,倒是跟谁?那你是不可能记不起来的。”
\par “我也不知道,真的!我想是在布洛尼林园,有个晚上你上岛上去找我们来着。你先在洛姆亲王夫人家里吃了晚饭,”她说,很高兴能提供一个能证实她的话的精确细节,“在邻桌上有个我很久很久没有见过的女人。她对我说:‘跟我上那边岩背后去看湖光月色吧。’我打了个哈欠,答道:‘不,我累了,在这里挺好。’她说月色从来没有那么好过。我说:‘扯淡!’,我知道她想干什么。”
\par 奥黛特讲这番话的时候,差不多一直是嘻嘻哈哈的,也许因为她觉得这很自然,也许因为她想这样就可以让事情显得不怎么严重,也许是为了掩盖她的羞色。但当她看到斯万的脸色时,她就换了腔调:
\par “你这个坏家伙,你拿折磨我来寻开心,逼我编些谎话来好叫你让我安生!”
\par 对斯万的这个打击比第一个还要使他难以忍受。他从来没有料到这是一件离现在如此之近的事情,她却一直瞒过了他,他一直没能发现;这并不是在他所不知晓的过去,而是在他记得如此清楚的那些夜晚,是他跟奥黛特一起度过的那些夜晚,是他原以为了如指掌而现在回想起来却隐藏着欺骗和丑恶的那些夜晚;在这些夜晚中间忽然裂了一个大口子,就是在布洛尼林园中的那个时刻。奥黛特虽然不算聪明,但本身还是有魅力的。她刚才边比画边讲述那个场面时是何等的简洁,使得斯万气喘吁吁地仿佛身临其境:奥黛特的哈欠,那岩壁。他还听到她回答“扯淡”两字——不幸的是,答话时是高高兴兴的。他感到今晚她是不会再说什么了,这会儿不可能再等到有什么新的透露,就说:“可怜的小宝贝,原谅我吧,我知道我委屈你了,得了,我再也不去想它了。”
\par 不过她还是看到他的双眼死死盯着他所不知道的事情,盯着他们过去的那段恋情;在他的记忆中已经模糊因而显得既单调又平和的那段恋情,现在却被在洛姆亲王夫人家那顿晚宴后,在布洛尼林园岛上月光下的那一分钟,撕出了一道裂口。然而他早就养成了这样的习惯,总是把生活看得是饶有兴趣,总是要为在生活中稀奇古怪的发现赞赏不已,因此尽管难受得甚至认为这样的痛苦无法再忍受下去,心里却想:“生活这个东西真是叫人惊讶不已,它保留着许多妙不可言的意外;看来恶习这个东西散布起来比人们预料的要广泛些。这个女人我一直是信任的,看样子她是如此纯朴,如此正派,纵然有些轻佻,可她的各种爱好还是正常健康的。我根据一封不大可信的揭发信,盘问她一下,她承认的那点东西就透露了超出于我所能设想的情况。”然而他不能局限于她那几句没有多大意义的话。他要设法把她所说的话的价值弄个一清二楚,看看是不是应该得出这样的结论,就是那些事儿她是常干的,今后还要再犯。他反复琢磨她说的那几句话:“我知道她想干什么”,“两三次”,“扯淡!”然而这些话在斯万脑海里重现的时候并没有解除武装,每句话都像是抓住一把刀,给他又扎上一下。在很长一段时间内,就像一个病人不得不每分每秒都做使他感到痛苦的动作一样,他也反复琢磨着那几句话:“我在这里挺好”,“真扯淡!”不过他的痛苦是如此之深,他不得不打住了。他感到奇怪,怎么他一直是如此轻松,如此愉快地评断的那些事儿,现在竟能变得像可能置人于死地的疾病那样严重?他也认识一些女人,原是可以请她们监视奥黛特的。可你怎能指望她们的观点会跟他现在一致,而不是停留在曾长期指导着他的色情生活的那个观点上,能不笑着对他说:“你这醋坛子,你想剥夺别人的乐趣?”他原先在对奥黛特的爱情中所得到的纯粹是优雅的乐趣,而现在也不知是什么闸门突然落下,把他投入这新的地狱界中,看不出如何才能出去。可怜的奥黛特呀!他并不怨她。这并不全是她的罪过。不是说当她几乎还是个孩子的时候,就被她的生身母亲在尼斯卖给了一个英国富翁吗?阿尔弗雷·德·维尼在《诗人日记》里那几句话,他原先读的时候是无动于衷的,现在却觉得其中含有何等痛苦的真实:“当你觉得爱上了一个女子的时候,你应该自己问问:她的周围环境怎样?她的经历如何?生活的幸福全系于此。”斯万感到惊讶,像“真扯淡!”“我知道她想干什么”这样一些在他脑子里一个字一个字迸出来的简单的句子,竟能给他造成这么大的痛苦。不过他也明白,他以为这些不过是几句简单的句子,其实却是把他在听奥黛特叙述她那档子事的时候所感到的痛苦之情包裹起来的甲胄,随时都还能袭上他的心头的;他现在感到的不正是那份痛苦之情吗?他现在明白了这点也是枉然。随着时间的推移,就算他把它忘了,宽恕了,依然还是枉然。当他在心里重温这几句话的时候,那份痛苦之情依然像奥黛特说他的那样,使他成为无知和轻信的人;他那强烈的醋意为了使他遭到奥黛特的坦白的打击,总是把他处在一个不知情的人的地位,以至过了好几个月,这段老故事依然像是一个突然的启示那样使他大吃一惊。他自己也诧异他的记忆怎么能有这样强的再创造力。只有等到这台发生器的能力随着年事的增长而逐渐衰退,他才能指望这份折磨有所减轻。然而每当奥黛特所说的话折磨他的力量有点枯竭的时候,斯万脑子里原先较少萦回的话,就由一句几乎是新的话来接班,并以它的全部力量来予以打击。在洛姆亲王夫人家吃晚饭那晚的回忆是痛苦的,但那还只不过是他的痛苦的中心。痛苦从这里辐射出去,及于前前后后的日子。不管他的回忆触到哪一点往事,整整一季,维尔迪兰夫妇如此频繁地在布洛尼林园岛上吃晚饭的情景都刺痛他。这痛苦是如此之深,以至醋意在他心中激起的好奇之心渐渐地被在满足它们时将遭受的新的折磨的担心所抵消。他意识到奥黛特在遇见他以前的那段生活,他以前从来没有下工夫去了解的那段生活,那并不是他泛泛地看上一眼的一段抽象的时期,而是充满着具体事件的特定的岁月。在对这些岁月有所认识的过程中,他真怕这个此刻看来没有色彩,平稳流逝而可以忍受的过去的岁月会具有看得见的淫秽的形态,具有一副与众不同的恶魔般的面貌。他还是不打算去对她那段过去多作设想,这倒不是由于懒于动脑,而是怕增加苦恼。他希望有朝一日,他终于能在听到“布洛尼岛”,“洛姆亲王夫人”这些名字时能不再感到往日的伤心,同时也感到,在他的痛苦之情刚过去时就激奥黛特说出一些足以使这份痛苦之情以另一形式重现的新的话语、地点名称,以及各种情况是并不明智的。
\par 然而他所不知道的事情,他现在怕知道的事情,却往往是由奥黛特自发地,在无意中向他泄露的;奥黛特的恶习在她的实际生活跟斯万过去以为,现在还时常以为他的情妇过的那种相对无邪的生活之间,划出了一条鸿沟,连奥黛特自己也不知道它到底有多宽。一个染有恶习的人,在他不希望会怀疑他有这样的恶习的人们面前总是装出道貌岸然的样子的,但他意想不到他这些恶习(他感觉不到它们的持续生长)会怎样使他逐渐离开正常的生活方式。在他俩同居期间,在奥黛特心中,一方面有向斯万掩盖的一些行动的回忆,另一方面有些行动渐渐接受前者的影响,受到前者的感染而她自己并不以为怪,同时这些行动也不会在她心中接受培育的那个部位发生爆炸;但是如果她要把这些事讲给斯万听了,那他就会被这些事情泄露出来的气氛大吃一惊。有一天,他想问问奥黛特——倒没有刺痛她的意思——她是否跟皮条客打过交道。说实在的,他相信她是不会和她们打过交道的;他在读那封匿名信的时候,脑子里曾经闪过这个假想,然而那仅仅是文字的机械的反映,并没有信以为真,可依然还留在脑际。斯万要把这个虽然只是块死疙瘩,可毕竟还是恼人的怀疑摆脱掉,希望奥黛特能把它连根拔除。“啊!不!这并不等于说我没有被她们缠过,”她说,那微笑当中流露出一点自负和得意,竟忘了斯万看了会觉得奇怪,“昨天还来了一个,等了我两个多钟点,说是我开多大价都行。看样子是有个外国大使对她说了什么:‘您要是不把她给我找来,我都要自杀了。’我先让人对她说我不在家,后来只好亲自出来把她打发走。我真希望你那会儿在家看看我是怎么对待她的。我的女仆在隔壁屋里听我说话,后来说我当时扯开嗓门大叫:‘我已经对您说了,我不愿意!这是什么鬼主意,我可不乐意!我想干什么,不想干什么,总有我的自由吧!如果我要钱的话,我可……’我已经告诉门房以后别让她进来了,就说我在乡下。啊!我是多么希望你当时躲在什么地方听着。我相信你是会满意的,我亲爱的。你看,你的小奥黛特也有她好的一面,尽管有人说她的坏话。”
\par 她以为他已经发现了这些过错,所以承认下来,对斯万来说,这种坦白不但没有结束他旧的怀疑,反而成了新的怀疑的起点。这是因为她的坦白从来不会跟他的怀疑完全一致。奥黛特尽管从她的坦白当中抽去了最主要的部分,但在次要的东西里还是有些斯万从来没有想象过的东西,正由于其新而使他难以忍受,也使他的醋意的方程式中的已知未知各项起了变化。她这些坦白,他是再也不会忘掉的。他的心把它们装载起来,把它们抛下,又把它们抱到怀中摇晃,像是浮在河面的死尸。她的坦白使他的心中了毒。
\par 有一次她对他讲到救济西班牙木尔西亚水灾灾民日,那天福什维尔去看她了。“怎么,你那时候就认识他?噢!对了!不错,不错。”他赶紧改口,免得显得他不知道那件事情。他忽然想起,救济木尔西亚水灾灾民日那天正是收到他现在还珍藏着的她那封信的日子,那天她多半是跟福什维尔在金屋餐厅吃饭来着。想到这里,他不禁哆嗦起来。可她发誓说没有那么回事。“反正金屋餐厅叫我想起什么事情,后来知道那是谎话。”他说这话是为了吓唬吓唬她的。“对了,那天晚上你上普雷福咖啡馆找我,我说我刚从金屋餐厅出来,其实我并没有去。”她看他的神色以为他已经知情,所以说得很果断——与其说是出于脸皮厚,倒不如说是出于胆怯,怕斯万不高兴(由于爱面子又不想显露出来),还有就是想向斯万证明她也是能坦率的。就这样,奥黛特以刽子手操刀那种干净利索和力量打击了斯万,然而她倒并没有刽子手那样的残忍,因为她并不意识到她在伤害斯万;她甚至还笑出声来,可能主要是为了不在对方面前露出她的羞愧和窘态。“真的,我没有上金屋餐厅去,我是从福什维尔家出来。我当真到普雷福咖啡馆去了,这不是瞎扯,他在那里跟我碰头来着,请我上他家去看版画。可另外有个人来看他了。我跟你说我从金屋餐厅出来,那是因为我怕说了实话你要生气。你看,我这是为你好。就算是我当时错了,至少我现在对你说了实话。如果救济木尔西亚灾民日那天我真跟他在一起吃了饭,我瞒着你又有什么好处?再说,那会儿咱们两个也还不是太熟悉呢。是不是,亲爱的?”他向她尴尬地微微一笑,这些令人痛苦的话语忽然弄得他有气无力,像要垮下来了似的。原来就在他以为是十分幸福因而不堪回首的那些月份,在她爱他的那些月份,她已经在向他撒谎!除了在她跟他说是从金屋餐厅出来的那一刻(那是他们第一次“摆弄卡特来兰花”的那一晚),还该有多少时刻窝藏着斯万连想都没有想过的谎话啊!他想起她有一天对他说:“我只消跟维尔迪兰夫人说我的衣服还没有做好,我的马车来晚了就行了。总有办法应付的。”可能对他也是一样,她曾多次吐出几句话来解释她为什么迟到,说明改动约会时间的理由,这些话大概也出乎他当时意料之外地遮盖着她跟另一个人干的什么勾当,她对这个人也会说:“我只消跟斯万说我的衣服还没有做好,我的马车来晚了就行了,总有办法应付的。”在斯万最美好的回忆底下,在奥黛特以前对他所说的最淳朴,被他认为是无可置疑的福音书式的语言底下,在她向他讲述的日常活动底下,在最平凡无奇的地点——她那女裁缝家里、布洛尼林园大道、跑马场背后,他到处都感到可能有谎言的潜流存在,哪怕是最详细的日常生活情况的汇报也会留下空当,足以遮掩某些活动;他感到这谎言的潜流到处渗透,使得过去在他看来是最弥足珍贵的东西(最美好的良宵,奥黛特常在原定时间以外的时间离开的拉彼鲁兹街)也都变得丑恶了;这股潜流差不多到处都散布像他在听到她坦白关于金屋餐厅那档子事时感到的厌恶之情,也像“尼尼微的毁灭”\footnote{尼尼微为古代亚述帝国的首都,公元前612年被米堤亚和迦勒底联军所毁。}中那些伤风败俗的畜生一样,把他的过去这座大厦一块砖一块砖地震塌下来了。现在每当他想到金屋餐厅这个残酷的名称时,他都扭过头去,这就不像前不久在德·圣德费尔特夫人家的晚会上那样是使他重尝久已失去的一种幸福,而是向他重提他刚刚知情的一桩不幸。后来,无论是金屋餐厅这个名称也好,布洛尼岛这个名称也好,慢慢地都不再叫他伤心了。这是因为我们心目中的爱情和醋意都并不是一种连续的、不可分的、单一的激情。它们都是由无数昙花一现的阵阵发作的爱欲和各种不同的醋意构成的,只不过是由于它们不断地聚集,才使我们产生连续性的印象,统一性的幻觉。斯万爱情的存在,他的醋意的坚持是由无数欲念、无数怀疑的死亡和消失构成的,而这些欲念和怀疑全都以奥黛特为对象。如果他长期见不到她的话,那些正在死去的欲念和怀疑就不会被别的欲念和怀疑取而代之。而奥黛特的出现继续在斯万心中交替地播下柔情和猜疑。
\par 有些夜晚,她突然变得对他亲热异常,还敦促他赶紧抓住机会,否则良机难再;那时就得马上回到她家去“摆弄卡特来兰花”,而她那欲念来得如此突然,如此难解,如此迫不及待,她给他的那种种爱抚又是如此狂放,如此异乎寻常,以至这种突如其来、前所未见的温情反倒跟谎言和恶意一样使得斯万愁闷起来。有天晚上他就像这样奉奥黛特之命跟她回到家里,她又是吻他又是说些跟平常的冷漠恰成鲜明对比的充满热情的话语,他忽然觉得听到什么声音;他站起身来,到处寻找,没找到任何人,但也没有勇气坐回她的身边;她这时气得要命,摔碎一只花瓶,对斯万说:“你这个人真难侍候!”他却一直怀疑她是不是故意藏了一个人来激发他的醋意或者煽起他的怒火。
\par 有时他还上妓院去,想打听一点关于她的情况,当然不敢把她的名字说出来。老鸨对他说:“我这里有个小姑娘准能中您的意。”他这就跟一个感到莫名其妙的可怜的小姑娘有气无力地聊上个把钟头,也不干别的什么事儿。有天有个年纪很轻、秀色可餐的姑娘对他说:“我但愿能找到一个真正的朋友,他尽可放心,我再也不跟别的男人了。”“真的?你以为一个女人能被男人对她的爱情所感动,就永远不会对他不忠实?”斯万急切地问她。“当然咯,这得看她们的品格!”斯万禁不住在这些姑娘面前把洛姆亲王夫人听了都会高兴的话说了出来。他笑着对那位想找个男朋友的姑娘说:“你真好,你的眼睛蓝得跟你的腰带一个色。”“您的袖口也是蓝的。”“咱们在这样的地方谈这样的话,真是妙极了!我不打扰你吧?你也许有事儿要忙?”“不,我有的是时间。要是您打扰我的话,我是会直说的。恰恰相反,我很喜欢听您讲话。”“那我很荣幸。我们谈得挺投机的吧?”后面这句是对刚进来的鸨母说的。“是啊,我刚才还这么想呢。他们怎么那么老实!哼,这年月有人就是为了聊天才到我这儿来的。那天亲王就说了,在这里比在他老婆跟前好多了。看来这年头上流社会里的女人全都是那号人,说起来真丢人!我这就走了,我不在这里讨厌了。”她就撇下斯万跟那个蓝眼睛的姑娘。可他也立即站起身来跟这姑娘道别,他对她不感兴趣,因为她根本不认识奥黛特。
\par 画家病了,戈达尔大夫劝他到海上旅行旅行;好几个忠实信徒说要跟他一起去;维尔迪兰夫妇下不了决心单独呆在巴黎,就租上一条游艇,后来干脆买了下来,奥黛特这就经常出海了。每当她出去了一些日子,斯万就感到他开始摆脱她了,然而仿佛是精神上的距离跟物质上的距离恰成正比一样,一旦他知道奥黛特已经回来了,他在家里就呆不住,不能不去看她。有一次,他们以为是出去玩了一个月,可也许是路上受了什么诱惑,也许是因为维尔迪兰先生为了讨好他的太太而早有预谋,只是在路途上才慢慢向信徒们透露,他们从阿尔及尔到了突尼斯,然后又到意大利,再到希腊、君士坦丁堡,又到小亚细亚。旅行继续了将近一年。斯万感到绝对清静,几乎是非常幸福。虽然维尔迪兰夫人极力说服钢琴家和戈达尔大夫,说钢琴家的姑妈跟戈达尔的病人并不需要他们,而且维尔迪兰先生说巴黎正在闹革命,让戈达尔夫人回去有欠谨慎,然而维尔迪兰夫人到了君士坦丁堡也不得不把他们两个放回去。画家跟他们一起走了。有一天,在这三位旅客回到巴黎不久,斯万看到有辆上卢森堡公园去的公共马车,他正好要去办事,就跳了上去,刚好坐在戈达尔夫人对面;戈达尔夫人正在作她“每周”的探亲访友活动,穿戴齐全:帽子上插有羽毛,身穿绸长裙,手抄手笼,臂悬晴雨两用伞和名片夹,戴着浆洗得雪白的手套。如果天气晴和,她就带着这套标志,在同一区里徒步一家一家拜访,要是到另一个区去,那就利用公共马车作为中转。开初几分钟,她那作为女人的天然的亲切还没能够穿透小资产阶级妇女上过浆的那一层表膜,也还不大清楚是否该对斯万讲起维尔迪兰夫妇,她只好以她那缓慢、不自然但还温柔,有时被马车的嘎吱声完全淹没了的嗓音,倒还挺自然地把她一天之中爬上爬下跑的那二十来家人家当中听来的和自己照搬的话语挑选出来讲上一讲:
\par “先生,不用问,像您这样一个不甘落伍的人当然是已经上密里东去看了马夏\footnote{儒尔—路易·马夏(1839—1900):法国画家。}画的那幅肖像了,全巴黎城都趋之若鹜。您有什么高见?您是属于赞成派那个阵营呢,还是声讨派那个阵营?所有沙龙里都是众口一词地谈马夏这幅肖像;谁要不就马夏这幅肖像发表点意见,那就是不帅,不高雅,赶不上时代。”
\par 斯万说他还没看过这幅肖像,戈达尔夫人担心逼他这么坦白承认,会把他刺痛了,赶紧说:
\par “啊!很好,很好,至少您是坦白承认了,您并不因为没有看过马夏这幅肖像就感到丢脸。我觉得您这就很好。我呢,我倒是看了,真是见仁见智,有人说它有点过分精雕细刻,像是打成泡沫状的掼奶油,我呢,我觉得那幅肖像真是件理想的作品。当然,她跟咱们那位朋友比施画的蓝颜色、黄颜色的女人不一样。可我得向您坦白承认——您可能认为我是个老古板,可我是心口如一——比施的画我可并不懂。老天哪!他给我丈夫画的肖像的优点我不是不知道,那幅画画得没有他平常画得那么怪,可他居然把我丈夫的胡子画成蓝的!可马夏呢!我这会儿上我一个朋友家去,他是我丈夫的一个同行(能跟您同路真是莫大的荣幸),她的丈夫已经答应她了,如果他给选进了法兰西学院,就请马夏给她画像。当然,这是一个美妙的梦想!我还有一个朋友,说她更喜欢勒卢瓦\footnote{莫理斯·勒卢瓦:法国画家。}。我是个门外汉,也许勒卢瓦的学问比马夏更大。不过我觉得一幅肖像画的首要条件,特别是当它值一万法郎的时候,是要画得像,像得叫人看了舒服。”
\par 这些话无非都是帽子上羽毛的高度,名片夹上姓名起首字母组成的图案,洗染店用墨水在白手套上写的号码,还有在斯万跟前不便谈维尔迪兰夫妇这些情况下启发她说的,说完以后,眼看离波拿巴特街角还远,车夫一时还停不了车,她的心又启发她讲了另外一些话。
\par “我们在跟维尔迪兰夫人一起旅行的时候,先生您的耳朵该是一直热着的吧?”她对他说,“我们一直念叨着您来着。”
\par 斯万感到异常意外,他原以为在维尔迪兰夫妇面前是没有人会提他的名字的。
\par “而且,”戈达尔夫人接着说,“有德·克雷西夫人在场,那是再自然也不过的了。只要奥黛特在,她就不能不时时地讲起您。当然不是讲您的坏话。怎么!您不信?”看到斯万面露怀疑之色,她找补了那么一句。
\par 她深信自己是一片真诚,对所用的字眼也并没有添加任何不好的意思,只是跟大伙一样,把它用来表示把朋友们联系起来的那种感情而已。
\par “她可是爱您爱得很深呢!啊!当着她面谁也不能讲您的坏话,要不然的话,那可有你好看的!随便谈到什么,就说是看到一幅画吧,她就说:‘啊!要是他在的话,他就会告诉你们那是真的还是赝品。在这方面他是谁也比不上的。’她时时都在问:‘他这会儿在干什么?但愿他能下工夫干点活!这么有天赋的汉子,可那么懒,真是可惜!(您该不见怪吧?)我这会儿就看见他在我眼前,他在惦记着咱们,在琢磨咱们到了什么地方。’我当时就觉得她那话讲得好极了,原来维尔迪兰先生问她:‘您离他有几千里,您怎么能看到他现在在干些什么?’只听得奥黛特说道:‘情人眼里没有办不到的事情。’我起誓,我说这话并不是为了讨好您,您这位朋友可是不可多得的真正的朋友。而且我还要跟您说,如果您连这一点都不知道,您可是天下唯一的一个了。维尔迪兰夫人在最后一天都对我说(您知道,别离前夕聊起来总是更随便的):‘我并不是说奥黛特不爱我们,不过我们跟她说的话与斯万先生说的相比就没有什么分量了。’啊!好家伙,车夫把车停住了,聊着聊着我都差点儿要错过波拿巴特街了……劳您驾告诉我,我帽子上的羽毛正不正?”
\par 戈达尔夫人从她的手笼里把她那只戴了白手套的手抽了出来,伸向斯万,从那手中,除了那张转车车票外,还有一股高级生活的气派,跟洗染房的香味一起洋溢在车厢之中。斯万感到他心中充满了对她的亲切之感,同样也有对维尔迪兰夫人的亲切之感(也差不多同样有对奥黛特的,因为现在他对她的感情中不再掺杂痛苦的感觉,几乎就只是爱情了),这时他站在马车车厢外的平台上以充满柔情的目光看着戈达尔夫人雄赳赳气昂昂地走在波拿巴特街上,帽子上羽毛高耸,一手提着裙子,一手提着晴雨两用伞和露出姓名起首字母组成的图案的名片夹,走路时把个手笼在身前一摇一晃。
\par 戈达尔夫人真是比她丈夫还要高明的医疗专家,为了跟斯万心中对奥黛特存有的病态的情感相抗衡,她在它们之上嫁接了另外一些情感,那是感激和友好的正常的情感,是使得斯万心目中的奥黛特更富有人情味,与其他妇女更加相似的情感(其他妇女也是能启发他这样的情感的);这些情感促使他心目中奥黛特的形象起了根本的变化,恢复成为曾经被他平平稳稳地爱着的那个奥黛特;她有天晚上在画家家中的宴会之后带他跟福什维尔一起去喝一杯橙汁,他当时不是也预见到在她身边生活是能够幸福的吗?
\par 从前他也常不寒而栗地想过,有朝一日他也许会不爱奥黛特,他暗暗说应该警惕,一旦感到他对奥黛特的爱要离他而去时,就要把它紧紧抓住,拽将回来。可随着他爱情的衰退,保持爱情的愿望也随之衰退了。人是不能改变的,也就是说不能变成另外一个人而继续听从不复存在的那一个人的情感。有时他在报上见到被他怀疑曾经当过奥黛特情人的人的名字,这也会使他的醋意油然而生,不过这种醋意并不强烈,但表明他还没有完全摆脱他曾感到如此痛苦,也是他享到如此欢乐的时期,也表明人生路程上的一些偶然因素还可能使他悄悄地、远远地看到那个时期的优美之处;这醋意带给他的毋宁是一种可喜的激动,就像一个闷闷不乐的巴黎人离开威尼斯要回法国去,最后一只蚊子提醒他意大利跟夏天离他都还并不太远一样。而更多的时候,他正要与之告别的这段不寻常的岁月,当他作出努力,纵使不能继续滞留,至少在他还有可能的时候留下一个清楚的景象时,他却发现为时已经太晚了;他也想跟再看一眼行将消失的景象那样再看一眼他刚告别的这段恋情,可是一身而任两人,为已经不再具有的情感得出一个真实的景象却是如此困难,结果要不了多久脑子里就一片漆黑,眼睛也一无所见,他只好不再去看,摘下夹鼻眼镜,擦擦镜片;他心想还是休息一会儿的好,过一会儿也不为迟,这就没精打采地缩在角落里,跟那位昏昏欲睡的旅客一样,他正拉下帽子盖住眼睛,想在他感觉到正在越来越快地把他带离他曾长时间生活过的国家的这个车厢里睡上一觉,而他却曾默默许愿不让它在未曾最后道别以前就飞逝而过的。就跟那位直到进了法国国境才醒的旅客一样,当斯万偶然在身边找到福什维尔曾是奥黛特的情人的证据时,他发现自己毫不感到痛苦,他的爱情现在已经离他而去了,只是为它永远离开他时没有跟他打个招呼而感到遗憾。在首次吻奥黛特以前,他曾力图把她长久以来留给他的形象赶在这一吻的回忆日后使它变样之前铭记心中,同样,他也曾希望,能趁她还在,至少是在精神上能跟激起他的爱情、燃起他的妒火、给他带来痛苦、从此也将永不再见的奥黛特道别。
\par 他错了。几个星期以后,他还见到她一次。那是在他熟睡之际,在梦乡的暮霭之中。他正跟维尔迪兰夫人、戈达尔大夫、一个他认不出是谁的戴土耳其帽的年轻人、画家、奥黛特、拿破仑三世和我的外祖父一起散步。他们走的那条路俯瞰大海,一侧是悬崖,有时壁立千仞,有时仅及数尺,行人不断上坡下坡;正在攀登的人们就看不见已经下坡的游客,落日的余晖渐渐暗淡,看来黑夜立即就要笼罩四野。浪花不时溅到岸上,斯万也感到面颊上溅上冰冷的海水。奥黛特叫他擦掉,可是他办不到,因此在她面前他感到尴尬,何况他身上穿的还是睡衣。他但愿人们因为天黑而发现不了这个情况,然而维尔迪兰夫人却以诧异的目光久久凝视着他,而他只见她脸庞变形,鼻子拉长,还长上了一部大胡子。他转过脸去看奥黛特,只见她面颊苍白,脸上长着小红疙瘩,面容疲惫,眼圈发黑,然而她还是以充满柔情的目光看着他,双眼似乎要像泪珠一样夺眶而出,他感到他是如此地爱她,真想马上把她带走。奥黛特忽然转过手腕,看了一下手表,说一声“我该走了”,就以这同样的方式跟所有的人道别,也没有把斯万叫到一边,告诉他当晚或者哪一天在什么地方再见。他不好意思问她,他真想跟她一起走,却又不能不扮出一副笑容回答维尔迪兰夫人的问题,连头也不敢向奥黛特那边转去,可是他的心突突地跳得可怕,他恨奥黛特,真想把刚才还如此喜欢的她那两只眼睛抠掉,把她苍白的面颊抓烂。他继续跟维尔迪兰夫人一起上坡,也就是一步一步更远离在相反的方向下坡的奥黛特。时间才过了一秒钟,却仿佛她已经走了几个钟头。画家告诉斯万,她刚走不久,拿破仑三世也不见了。“他们肯定是商量好的,”他说,“他们准是要在崖脚下相会,却又顾到礼仪,不好意思两个人一起跟咱们道别。她是他的情妇。”那不相识的年轻人哭起来了。斯万竭力安慰他。“她还是有道理的,”他说,一面为他擦拭眼泪,一面给他摘了土耳其帽,让他更自在些,“我都劝过她十多次了。干吗难过呢?那个人是会理解她的。”斯万这是自言自语,因为他原先没能辨认出来的那个年轻人就是他自己;就像有些小说家一样,他是把自己的人格分配给了两个人物,一个是做梦的那个人,另一个是他所看见的站在他面前戴着土耳其帽的那个人。
\par 至于那个拿破仑三世,其实就是福什维尔;把某些概念模模糊糊地一联系,把男爵平常的面貌稍加改变,再加上交叉在胸前的荣誉勋位勋章的绶带,这就使得斯万给了他这样一个名字;实际上,梦中这个人物在他心目中所代表的,让他想起来的也正是福什维尔。在梦乡中的斯万从不完全的变幻着的形象中作出错误的推断,而且他这时也掌握一种创造的能力,能像某些低级生物通过简单分裂那样进行繁殖;他通过所感觉到的自己手掌的温暖造出一只他在想象中握着的另一人的手心,同时也通过自己都还没有意识到的情感和印象来勾勒出一些曲折情节,通过逻辑联系,在他睡梦中的一定时刻,构成必要的人物来接受他的爱或者促使他醒来。
\par 黑夜忽然降临,警钟响起,居民从烈焰冲天的房屋中逃出,奔跑着从他面前过去;斯万听到汹涌的波涛声,他的心也同样猛烈地在他胸膛里突突地跳着。突然间,他的心跳加速,他感到一阵说不出来的痛苦和恶心,一个满身是灼伤的农民在经过他面前时说:“您去问问夏吕斯吧,奥黛特是在他那里跟她的伙伴过夜的。他常跟她在一起,她跟他也无话不说。是他们放的火。”原来是他的男仆刚把他叫醒,对他说:
\par “先生,八点了,理发师也来了,我已经告诉他过一个钟头再来。”
\par 这些话穿透斯万沉浸其中的睡眠之波,在到达他的意识之前却产生了偏离,就像是一道光线在水底显得像是一个太阳一样,也正如片刻之前铃声在他梦乡的深渊之中变成了警钟的声音,闹出了火灾这档子事儿。这时候,他梦中的景色化为灰烬,他把眼睛睁开,最后一次听到大海远去的涛声。他摸摸面颊,是干的。然而他还记得那冰冷的水的感觉和盐的咸味。他下床穿上衣服。他之所以早早地把理发师叫来,是因为他头天给我外祖父写了信,说是下午要到贡布雷去,因为他听说德·康布尔梅夫人(也就是过去的勒格朗丹小姐)要在那里住几天。他回想起那年轻的妩媚的面孔,还有他久别了的乡间的妩媚的景色,两者对他产生了巨大的吸引力,促使他下定决心离开巴黎几天。种种偶然的机会使得我们跟某些人相逢,这机会并不跟我们爱他们的时间相一致,可能发生在爱情还没有开始以前,也可能在爱情已经泯灭以后又再重现;事后回想起来,在我们一生中后来注定要成为我们意中人的最初出现总是有预告或先兆的意义的。就这样,斯万常常回顾在剧场碰见奥黛特时她的形象,在那个晚上,他是根本没有想到以后会再见到她的;现在他也想到德·圣德费尔特夫人家那个晚会,他那晚把德·弗罗贝维尔将军介绍给德·康布尔梅夫人。我们生活中的利害关系是如此复杂,以至在同一情况下,尚未到来的幸福的基础已经在我们正在受着的痛苦加剧时奠定,这也并不罕见。这样的事情当然也会在德·圣德费尔特夫人府第以外在斯万身上发生。又有谁能知道,那天晚上他要是上别的什么地方,是否会有别的什么喜事,别的什么不幸,而往后被他看成是不可避免的事?不过,确确实实发生了的事情,他会觉得是不可避免的;他都差点儿要把那天打定主意去参加德·圣德费尔特夫人家的晚会看成是天意如此了:他这个人虽然渴望能欣赏生命丰富多彩的创造,却无法对一个难题(例如到底什么应该是最该企求的东西)长时间苦思冥想,只好认为在那晚感到的痛苦跟尚难预料然而已在萌生中的乐趣之间存在着必然的关联,只不过这痛苦与这乐趣之间的平衡太难保持了。
\par 醒来一小时后,当他指点理发师怎样使他的头发在火车上不致蓬乱时,他又想到他那个梦,又看到奥黛特苍白的脸色、瘦削的面颊、疲惫的脸庞、低垂的眼皮,仿佛全都就在他的眼前;奥黛特的万般柔情早已把斯万对她的执著的爱化为对她的首次印象的长期遗忘——自从他们最初相爱以来这些日子,在他刚才睡着时,他在记忆中都曾竭力搜寻它们的确切感觉,从那时以来他已不再注意到的东西也仿佛就在他的眼前。自从他不再感到不幸,道德修养也随之有所降低以来,粗野的话也不时涌上他的心头,他心里不禁咆哮起来:“我浪掷了好几年光阴,甚至恨不得去死,这都是为了我把最伟大的爱情给了一个我并不喜欢,也跟我并不一路的女人!”

\subsubsection*{第三部\ 地名:那个姓氏}

\par 在我无眠之夜最常回忆的那些卧室当中,跟贡布雷的卧室差别最大的要数巴尔贝克海滨大旅社那间了,这间屋的墙涂了瓷漆,就跟碧波粼粼的游泳池光滑的内壁一样,容有纯净、天蓝色、带盐味的空气,而贡布雷那几间卧室则洋溢着带有微尘、花粉、食品和虔诚味道的气氛。负责装饰旅社的那位巴伐利亚家具商让各间房间的装饰都有所不同,我住的那间沿着三面墙都有玻璃门矮书柜,按照它们所处的位置不同,产生出设计者未曾料及的效果,反映出大海变幻无常的景色的一角,这就像是在墙上糊上一层海青色的壁纸,只不过被书柜桃心木的门框分割成一片一片罢了。这样,整个房间就像是当今“现代款式”住宅展览会上展出的新型卧室,装饰着据说是能使居住者赏心悦目的艺术品,上面表现的题材则以住处所在地点的性质而异。
\par 而跟这真正的巴尔贝克最迥然不同的莫过于我在暴风雨的日子里常常向往的那个巴尔贝克了。在这样的日子里,风刮得那么大,弗朗索瓦丝领我上香榭丽舍时总嘱咐我别贴了墙根走,免得让刮落下来的瓦块砸着,还不胜感慨地谈到报上所说的那些陆地遭灾和海上翻船的消息。我倒极其希望能看到海上的风暴,倒不是因为这景象美,而是因为这是揭示大自然真实生命的时刻;或者可以这样说,我心目中美的景象是我确知并非为了取悦于我而人为地安排的景象,而是必然的、不可改变的景象——例如景色之美,或者伟大的艺术作品之美。我所感到好奇的,我所热切要认识的,都是我相信比我自己还要真实的东西,都是具有这样一种优点的东西,能向我显示某个伟大的天才的一点思想,显示自然不假人手而自行展现出来的力量或美惠。正如留声机唱片中孤立地播放出来的先妣美妙的嗓音并不足以减轻我们失去母亲的痛苦一样,用机械模仿出来的暴风雨也跟万国博览会上光彩夺目的喷泉一样引不起我丝毫兴趣。为使暴风雨绝对真实,我也要求这海岸是一条天然的海岸,不是哪个市政府临时挖出来的一条土沟。大自然在我心中激起的种种情怀,使我觉得它跟人用机械创造的东西截然不同。大自然带上的人工印记越少,它给我心的奔放留下的余地越多。我可早就记住了巴尔贝克这个名字,勒格朗丹说这个海滩紧挨着“那以沉船频繁而知名的丧葬海岸,一年当中倒有六个月笼罩着一层薄雾,翻腾着滚滚白浪”。
\par 他还说:“人们在那里比在菲尼斯泰尔(那里尽管现在旅馆鳞次栉比,依然未能改变大陆最古老的骨架)更能感觉到他们脚下就是法国大陆、欧洲大陆、古代世界大陆真正的边缘。这是渔民的最后一个营地,他们跟创世以来世世代代的所有渔民一样,面对海上的迷雾和黑夜的暗影这永恒的王国。”
\par 有一天在贡布雷,我在斯万先生面前谈起这巴尔贝克海滩,想从他嘴里探听一下这里是不是看最强烈的暴风雨的最理想的地点,他答道:“巴尔贝克吗,我是很熟悉的!巴尔贝克的教堂是十二三世纪建的,还是半罗曼式的,也许是诺曼第哥特式建筑物最奇妙的样品,可真是独一无二!简直像是波斯艺术。”直到这时为止,这个地区在我心目中仿佛只是属于遥远得无法追忆的远古的大自然,跟那些伟大的地质现象的历史同样悠久,也跟地上的海洋和天上的大熊星座一样置身于人类历史之外——就连那些未开化的渔民也跟他们所捕的鲸一样,对他们来说也无所谓中世纪不中世纪的问题。现在真像是喜从天降,忽然发现这个地区也走进了世纪的序列,经历过罗曼时代,忽然得知哥特式的三叶草也曾在一定的时刻来装点过这里蛮荒的石块,正如那虽然细小然而生命力旺盛的花草在春季来临时穿透终年不化的积雪,星星点点地散布在极地一样。哥特式艺术帮助我们确定这些地方和这些人的年代,同样这些地方和这些人也帮助我们确定哥特式艺术的年代。我试着在脑子里想象这些渔民的生活,他们在中世纪聚居在这地狱海岸的一角,在死亡的悬崖脚下,又是怎样小心翼翼地、出乎意料地尝试着建立起人与人之间的关系;我原来一直以为,哥特式艺术只有在城市中才有,现在它离开了城市,在我心目中就更是一个有生命的东西了,我可以看它怎样在特殊的条件下,在蛮荒的岩石上,萌芽生长,开出一朵尖尖的钟楼之花。有人领我去看巴尔贝克最著名的雕像的复制品,有毛发蓬松、塌鼻子的使徒,有门厅的圣母像,当我想到我有一天可以亲眼看到它们耸立在那永恒的带有咸味的浓雾之间,我都高兴得喘不过气来了。从此,每到二月间风雨交加但天气温和之夜,狂风在我心中呼啸,刮得它跟卧室的烟囱一样猛烈地晃动,也把上巴尔贝克一游的盘算注入我的心中,既要去看一看哥特式的建筑,也要去体验一下海上的风暴。
\par 我真想第二天就乘上一点二十二分那班其妙无比的火车;这班车的时刻表,无论是在铁路公司的公告牌上还是在巡回旅行的广告上读到时,我的心总不禁怦怦直跳:我仿佛觉得它在下午的某一个确定的点上,开了一道美妙的槽,画下了一个神秘的标志,自这里起,钟点改了方向,尽管也还通向夜晚,通向明晨,然而已经不是在巴黎看到的夜晚或明晨,而是在列车通过而你可以自行选择的若干城市之一中所看到的:列车在贝叶、古当斯、维特莱、盖斯当贝、邦多松、巴尔贝克、朗尼翁、朗巴尔、贝诺岱、阿方桥、甘贝莱都是要停的,还要潇洒地继续前进,为我提供更多的地名,叫我不知如何选择是好,因为我不能舍弃其中任何一个。然而甚至我都无法再等明天那班火车,如果父母亲答应的话,我想匆匆穿上衣服,当晚离开巴黎,明日清晨当太阳在呼啸的海面升起时就抵达巴尔贝克,我将在波斯风格的教堂里躲避那海面飞溅的浪花。但随着复活节假期日渐迫近,我父母亲答应我可以在意大利北部度假,于是那一直占据我整个心灵的暴风雨之梦,一心只想看浪涛从四面八方呼啸而来,汹涌升腾,在那些陡峭粗糙如悬崖、钟楼上有海鸟呼号的教堂旁边直冲最荒漠的海岸的梦想一下子就烟消云散了,失去了它全部的魅力,因为它同起而代之的春之梦截然对立,只能起削弱它的作用;这是最绚丽多彩之春,不是依然还有寒霜砭人的贡布雷的春天,而是将菲埃索尔\footnote{菲埃索尔在佛罗伦萨近郊。}的草地布满百合花和银莲花,使佛罗伦萨有像安吉利科修士\footnote{安吉利科修士(1387—1455):俗名古依多·第·彼埃特鲁,是意大利文艺复兴早期画家。}画中那样金光闪闪,光耀夺目的背景的春天。从这时起,我就觉得只有阳光、花香、色彩才有价值,景象的变换在我心中促成了愿望的彻底的改变,而且这改变来得如此突然,就像在音乐中时常发生的情形一样,也促成了我感情基调的彻底的变化。到了后来,只要天气稍为有些变动,就会在我心中激起那样的变化,用不着等到另一个季节的来临。这是因为,时常在某个季节的某一天,我们觉得它是另一个季节迷了路的一天,它使我们生活在那个季节,立即想起并且渴望那个季节特有的乐趣,把我们正在做的梦打断,把幸福日历中某一章的一页撕下,或者移前,或者挪后。不久,我们的舒适感或是我们的健康只能从这些自然现象中偶然取得微不足道的好处,直到有朝一日,科学能够充分掌握这些现象,任意予以制造,把呼唤雨雪阳光的本领交到我们手里,使它们免遭机运的监护,摆脱它的喜怒无常为止,同样,大西洋与意大利之梦的出现也就不再完全取决于季节和天气了。要使巴尔贝克、威尼斯、佛罗伦萨再现,我只消把它们的名字念上一遍,这些名字所代表的地方在我心中激起的愿望就凝聚在这几个音节之中。即使是在春天,只要在哪本书里见到巴尔贝克这个名字,就足以唤起我去看暴风雨和诺曼第哥特艺术的愿望;哪怕是个风雨交加的日子,佛罗伦萨或者威尼斯这个名字也会使我向往太阳、百合花、总督府或者百花圣母院。
\par 这些名字虽然从此永远吸附了我对这些城市所设想的形象,但这是经过改造了的形象,是依照它们自身的规律重现到我脑际的形象;这些名字美化了这些城市的形象,也使它跟这些诺曼第和托斯卡尼的城市的实际不相一致,而我想象中赋予的任意的欢快越是增长,来日我去旅行时的失望也越强烈。这些名字强化了我对地球上某些地方的概念,突出了它们各自的特殊性,从而使它们显得更加真实。我那时不把这些城市、风景、历史性建筑物看成是从同一块质料的画布上在不同的位置裁剪下来、赏心悦目的程度有所不同的画幅,我是把它们当中的每一个都看成是一个完全与众不同的陌生的东西,我的心灵渴望着它,乐于从结识它之中得到益处。当这些城市、风景、历史性建筑物冠以名称,冠以它们特有的名称,就跟人各有其姓名时,它们又取得了更多的个性。文字为我们提供事物的明白而常见的小小的图像,就像小学校墙上挂的挂图,教给孩子什么叫作木工的工作台,什么叫作鸟,什么叫作蚂蚁窝,反正把同一类东西都设想成是一模一样。而人名(还有城市的名称,因为我们是习惯于把城市看成是跟人一样各有不同,独一无二的)为我们提供的图像却是含糊的,它根据名字本身,根据名字是响亮还是低沉,选出一种颜色,把这图像普遍涂上,就像某些广告一样,全部涂上蓝色或者全部涂上红色,由于印刷条件的限制或是设计师的心血来潮,不但天空和大海是蓝的或红的,就连船只、教堂、行人也是蓝的或红的。自从我读了《巴马修道院》以后,巴马就成了我最想去的城市之一,我觉得它的名字紧密、光滑、颜色淡紫而甘美,如果有人对我说起我将在巴马的某一所房子得到安置,那他就使我产生一种乐趣,认为我可以住进一所光滑、紧密、颜色淡紫而甘美的住所,它跟意大利任何城市的房子毫无关系,因为我只是借助于巴马这个名字的密不通风的沉重音节,借助于我为它注入的司汤达式的甘美和紫罗兰花的反光而把它设想出来的。而当我想到佛罗伦萨的时候,就仿佛是想到一座散发出神奇的香味,类似一个花冠的城市,因为它被称之为百合花之城,而它的大教堂就叫作百花圣母院。至于巴尔贝克,它是这样的名字中的一个,正如古老的诺曼第陶器还保留着制造它的陶土的颜色一样,这些名字还体现着某种已经废除了的习俗、某种封建权利、一些地方的历史情况,还有某种曾构成一些古怪的音节的过时的读音方式,我也毫不怀疑还能从到达巴尔贝克时将为我斟上一杯牛奶咖啡、领我到教堂面前去看奔腾的大海的那位客栈主人嘴里听到;我要赋予他一副古代韵文故事中的人物那种喜欢争论,以及庄严肃穆的古色古香的派头。
\par 如果我身体日渐健壮,父母亲即使不答应我上巴尔贝克住些日子,至少同意我登上我在想象中曾多次搭乘的一点二十二分那班火车去见识见识诺曼第或者布列塔尼的建筑和景色的话,我就想在那最美的几个城市下车;然而我无法将它们加以比较,无法挑选,正如在并非可以互换的人们中间无法选取一样。譬如说吧,贝叶以它的尊贵的红色花边而显得如此高耸,它的巅顶闪耀着它最后一个音节的古老的金光;维特莱末了那个闭音符给古老的玻璃窗镶上了菱形的窗棂;悦目的朗巴尔,它那一片白中包含着从蛋壳黄到珍珠灰的各种色调;古当斯这个诺曼第的大教堂,它那结尾的二合元音沉浊而发黄,顶上是一座奶油钟楼;朗尼翁在村庄的寂静之中却也传出在苍蝇追随下的马车的声响;盖斯当贝和邦多松都是天真幼稚到可笑的地步,那是沿着这些富于诗意的河滨市镇的路上散布的白色羽毛和黄色鸟喙;贝诺岱,这个名字仿佛是刚用缆绳系住,河水就要把它冲到水藻丛中;阿方桥,那是映照在运河碧绿的水中颤动着的一顶轻盈的女帽之翼的白中带粉的腾飞;甘贝莱则是自从中世纪以来就紧紧地依着于那几条小溪,在溪中汩汩作响,在跟化为银灰色的钝点的阳光透过玻璃窗上的蛛网映照出来的灰色图形相似的背景上,把条条珍珠似的小溪连缀成串。
\par 这些形象之所以不会真实,还有另外一个原因,那就是它们必然是十分简单化了的;当然,我的想象力所向往,而我的感官只是很不完全地感知而且并未立刻感到乐趣的东西,我就把它打入名字的冷宫里了;当然,因为我也曾在这冷宫里积攒了梦想,所以那些名字现在就激励着我的愿望;然而那些名字也并不怎么包罗万象;我至多也只能装进每个城市的两三处主要的胜景,而这些胜景在那里也只能单独并列,缺乏中间的联系;在巴尔贝克这个名字当中,就像从在海水浴场卖的那种钢笔杆上的放大镜中,我看到一座波斯风格的教堂周围汹涌的海涛。但也许正因为这些形象是简化了的,所以它们在我身上才能起那么大的作用。有一年,当我的父亲决定我们要上佛罗伦萨和威尼斯度复活节假时,由于在佛罗伦萨这个名字当中没有地方装下通常构成一个城市的那些东西,我就只好以我所设想的乔托的天才,通过春天的芳香,孕育出一个超自然的城市来。既然我们不能让一个名字占有太多的空间与时间,我们至多只能像乔托的某些画中表现同一人物的先后两个动作那样——前一幅还躺在床上,后一幅则正准备跨上马背——把佛罗伦萨这个名字分成两间。在一间里,在一个顶盖之下,我观赏一幅壁画,那上面覆盖着一块晨曦之幕,灰蒙蒙的、斜照而逐渐扩展;在另一间里(当我想到一个名字时,我并不是想到一个不可企及的空想的事物,而是一个我行将投身其间的一个现实的环境,一个从未经历过的生活,我在这个现实环境中完整而纯净无瑕的生活,赋予最物质性的乐趣、最简单的场景以原始人的艺术作品的魅力),我快步迈过摆满长寿花、水仙花和银莲花的老桥,好早早地吃上正在等着我的那顿有水果,有勤地红葡萄酒的午餐。这就是我眼前所看到的(虽然我人还在巴黎),而并非真正在我身边的东西。即使是从单纯的现实主义的观点来看,我们所向往的国家在任何时刻也都比我们实际所在的国家在我们的实际生活中占有多得多的位置。显然,当我更仔细地想一想,在我说出“上佛罗伦萨、巴马、比萨、威尼斯去”这几个字时我脑子里到底想的是什么,这时候我就会明白,我眼前看到的根本不是一个城市,而是跟我已知的一切是如此不同,也是如此甘美,就跟从来都是生活在冬季傍晚的某些人突然看到那从未见过的新异奇迹——春之晨一样。那些固定不变的不真实的图景充斥于我的夜晚,也充斥于我的白昼,使得这个时期的我的生活不同于以前那些时期(在一个只从外面看事物,也就是说什么也看不到的旁观者的眼中,那些时期可能与这个时期并无不同),这就好像在一部歌剧中,一个富有旋律性的动机引进了一点创新之处,只看脚本的人体会不到,而呆在剧场外面一个劲儿掏出表来看钟点的人就更难以想象了。再说,就从单纯数量的观点来看,在我们的生活当中,日子也并不都是相等的。要度过一天,对像我这样多少有点神经质的人,就跟汽车一样,有着几种不同的“排挡”。有些日子坎坷不平,艰难险阻,爬起来是无休无止,而有些日子则是缓坡坦途,可以唱着歌儿全速俯冲。在这个月里,我把佛罗伦萨、威尼斯和比萨的形象当做一首歌曲那样反复吟咏而永不知满足,这些形象在我心中激起的愿望当中有着如此深刻的个人的东西,简直可说是一种爱情,对人的爱情——我一直相信这些形象是跟不以我的意志为转移的客观现实相符的,它们使我产生了早期基督徒在升入天堂的前夕所可能抱有的那种美妙的希望。由幻想创造出来而并未经感觉器官感知的东西,现在要用感觉器官去观看、去触摸(而且越是跟它们已知的东西不一样,诱惑力就越大),这里头存在的矛盾,我也不去管它了;正是提醒我这些形象是现实的那些东西最强烈地点燃着我的愿望,因为这仿佛是我的愿望可以得到满足的一个许诺。虽然我这种豪情是出之于要满足艺术享受的愿望,但就维持这个愿望来说,旅游指南却比美学书籍起的作用更大,而火车时刻表甚至更有过之。当我想起,佛罗伦萨这个在我的想象中可望而不可即的城市,如果在我心中把它跟我隔开的这段路程不能通行的话,我总可以“走陆路”绕个弯,拐一拐走到的,这时我就会心情激动。当然,当我赋予我就要看到的事物以重大的价值,反复思量威尼斯是“乔尔乔涅\footnote{乔尔乔涅(1477—1510):意大利文艺复兴时期威尼斯画派最优秀的画家之一。他的艺术对提香及后代画家影响很大。}画派的所在地,提香的故居,中世纪住宅建筑最完善的博物馆”时,我感到幸福。当我上街,由于天气的关系(早春来了几天后寒冬又忽然恢复,这在圣周时的贡布雷是常有的事)而走得很快的时候,我感到更加幸福——我看到马路两旁的栗树虽然沉浸在潮湿似水的寒气之中,却依然像毫不气馁,穿上盛装,准时赴宴的客人一样,照样开始用它们遭霜冻的嫩叶,装点这肃杀的寒气,这寒气虽然阻挠,然而无力遏制其生长的不可抗拒的青葱翠绿,这时我想佛罗伦萨的老桥已经堆满了风信子和银莲花,春天的太阳已经把威尼斯大运河的河水染成一片深蓝,染成一片碧绿,当它冲上提香的画作时,简直可以跟画上丰富的色彩比个高下。当我的父亲一边看气压计,为天气之冷而兴叹,一边却开始研究坐哪班车最好时,我真是抑制不住我欢乐的心情;我也知道,等到吃完午饭走进那染上煤灰的实验室,走进那能使周围的一切都变样的魔室,第二天醒来时就可以到达那“以碧玉为墙,以绿宝石铺地”的大理石和黄金之城了。这样,它跟百合花之城就不再仅仅是我任意置之于我的想象力面前的虚构的图景,而是存在于离开巴黎一段距离(要去的话就绝对必须迈过),存在于地球上某一定点而不是任何其他地点的了,总而言之,这两个城市是确确实实真实的城市。当我的父亲说“总之,你们在威尼斯可以从四月二十号呆到二十九号,然后在复活节的早晨就到佛罗伦萨”的时候,对我来说,这两个城市就更加真实了;他这几句话不仅使两个城市从抽象的空间当中脱离了出来,而且也使它们从想象的时间当中脱离了出来,在想象的时间中我们不是一次仅仅安排一个旅行,而是把别的几次旅行也同时安排在一起而并不以为怪,因为这些旅行仅仅是可能性而已——而且这想象的时间是完全可以再生的,你把它在这个城市里度过了,还可以在另一个城市再度;他这几句话也为这两个城市安排了特定的日子,这些日子就是证明在这些日子中所做的事情的真实性的证明书,因为这些独一无二的确定的日子用过以后就消失了,它们不再回来,你不能在那里度过以后又到这里再度;我感觉到,正是将近星期一洗衣店要把我溅了墨水的那件白背心洗了送回来的那一周,那两个皇后城市从它们当时还不存在于其间的理想的时间中走了出来,以最激动人心的几何学的方式把它们的圆屋顶和钟楼载入我个人的历史中去。然而我那时还只是在走向欢乐的顶点这条道路的途中;后来我终于到了这一点(直到那时,我才得到启示,在那汩汩作响、被乔尔乔涅的壁画映红了的街道上,下一周,也就是复活节的前夕,在威尼斯散步的并不是我不顾别人再三提醒而依然还设想的那些“威风凛凛,像海洋那样令人生畏,头戴闪耀着青铜光的盔甲,外披带褶的血红披风”的人,而在别人借给我的那张圣马克教堂的大照片上,摄影者照下来的头戴圆顶帽,站在门廊前的那个小人儿可能就是我了),这时我只听得父亲对我说:“大运河上这会儿可能还冷,你无论如何别忘了把冬大衣和厚上衣装进箱子。”听了这话,我简直是欣喜若狂了;我感到我突然穿进那些“仿佛是印度洋中的暗礁似的紫水晶石堆”之中,这是我直到那时一直以为是不可能的事情;我以远远超出我体力的动作,像剥一只无用的甲壳一样,除去我卧室里身边的空气,换上同等数量的威尼斯的空气——那是我的想象力注入威尼斯这个名字当中的海上的空气,是梦中的无法形容的特殊的空气;这时我忽然感到像是灵魂出窍,随之而来的是一阵恶心,就像人们刚得了一阵剧烈的喉痛时那样,家里人不得不把我扶到床上,我烧得那么厉害,大夫宣称不仅现在不能让我上佛罗伦萨和威尼斯去,而且即使我全好了,一年之内也不能打算外出旅行,也不能有任何激动。
\par 唉!我还被绝对禁止上剧场去听拉贝玛的戏;这位被贝戈特认为是有天才的卓越的艺术家,当她让我看到一些也许是既重要又美妙的东西时,原本是可以减轻我为没有能去佛罗伦萨和威尼斯,又不能去巴尔贝克而痛苦的心情的。家里只能退而求其次,让我每天到香榭丽舍公园去,由一个人陪着,不让我太累,这个人就是弗朗索瓦丝,她是在莱奥妮姨妈死了以后就一直侍候我们的。上香榭丽舍实在是我难以忍受的事情。只要贝戈特在他的哪部作品里描写过这个公园,我也许会产生结识它的愿望,正如我总想认识在想象中早就已经有了一个“副本”的东西一样。我的想象力使这东西保持温暖,赋予它一个个性,我就想在现实中找到这个东西;可是在香榭丽舍这个公园里,没有一样东西跟我的梦有任何联系。
\par 有一天\footnote{那是在1895年,“我”十五岁时。},正当我对木马旁边我们那老地方感到腻味的时候,弗朗索瓦丝带我越过那些由卖麦芽糖的女商贩等距相隔的座座堡垒构成的边境线,到邻近陌生的地区散步,那里是一张张从未见过的脸,还有山羊拉的小车来来往往;她然后回去把那靠在一丛月桂树上的椅子上的活计拿回来;在等待她的当口,我在那稀稀拉拉,剪得很短,又被太阳晒得枯黄的大草坪上走来走去,在这草坪的一端有一个池塘,塘边是座雕像,这时在小径那边,有个小姑娘正在穿外套,把球拍装进套子,以生硬的语调对正在喷泉的承水盘边打羽毛球的另一个红头发女孩说:“再见了,希尔贝特,我回去了,别忘了今天晚上我们吃了晚饭上你家去!”希尔贝特这个名字在我耳边掠过,它并不仅仅是提到一个不在场的人物,而是直接称呼讲话的对方,因此更有力地提醒我它所指的那个人的存在;它就这样在我耳边掠过,可说是以随着它的弹道曲线,随着它逼近目标而逐渐增长的力量而行动着——我感到,在它身上装载着呼唤她的那个朋友(当然不是我)对她所呼唤的对象的认识和印象,装载着当她念出这个名字时她对她们日常亲密的交往,对她们彼此间的串门所见到的全部景象,至少是保留在记忆中的全部景象,而我由于不能企及而为之感到痛苦的这份陌生的生活,对这个幸福的姑娘来说却是如此熟悉,如此可以操纵自如,她使我触及这份生活的表面而无法深入其中,她以她那一声叫喊把这份我所陌生的生活投进了寥廓的天空——希尔贝特这个名字,精确地触及了斯万小姐的生活中的一些肉眼不能见的点滴,使它们所发出的香泽在空中飘荡,其中也包括今晚晚餐以后在她家举行的那个聚会的芬芳——它也构成一片色彩斑斓的浮云,今晚在孩子和女仆群中悠然飘过,就同那在普桑所画的某个花园上空扬帆飞翔的云一样,跟歌剧中满载骏马和车辆的彩云那样反映出众神生活的场面——最后,它也在这块乱蓬蓬的草地上,在她所站的位置(这既是凋零的草坪的一角,又是打羽毛球那金发姑娘午后的一个时刻,她这时还在不停地发球,不停地接球,直到一个帽子上插着蓝色翎毛的家庭女教师来叫她才住手)投上一道美妙无比的鸡血石色的光带,像一个映像那样不可捉摸,像一块地毯那样覆盖在地面,而我不禁无休无止地在这道光带上拖着我那双恋恋不舍,亵渎神明的沉重的双脚踯躅,直到弗朗索瓦丝对我嚷道:“得了,把您短大衣的扣子扣上,咱们颠儿吧。”这时我生平第一次不无恼怒地注意到她的语言是如此粗俗,唉!帽子上没有蓝翎毛嘛!
\par 她倒是会不会再到香榭丽舍来呢?第二天,她没有来,可是后来那几天,我都在那里见到她了。我一直在她跟她的伙伴们玩的地方周围转悠,以至有一回,当她们玩捉俘虏游戏缺一把手的时候,她就叫人问我是不是愿意凑个数,从此以后,每当她在的时候,我就跟她一起玩了。但并不是每天都是如此;有时候她就来不了,或者是因为有课,有教理问答,或者是因为午后吃点心,总而言之,她的生活跟我的截然不同,只有那么两次,我才感觉到凝结在希尔贝特这个名字当中的她的生活如此痛苦地从我身畔掠过,一次是在贡布雷的斜坡上,一次是在香榭丽舍的草坪上。在那些日子,她事先告诉伙伴们,她来不了;如果是因为学业的关系,她就说:“真讨厌,我明天来不了,你们自己玩吧。”说的时候神色有点黯然,这倒使我多少得到一点慰藉;但与此相反,当她应邀去看一场日场演出而我有所不知而问她来不来玩的时候,她答道:“我想是来不了!我当然希望妈妈让我上我朋友家去。”反正在这些日子,我事先知道见她不着,可有些时候,她妈妈临时带她上街买东西,到第二天她就会说:“对了,我跟我妈妈出去了。”仿佛这是一件极其自然的事情,不可能构成任何人的一件最大的痛苦。也有碰到天气不好,那位老师怕下雨而不愿把她带到香榭丽舍来的。
\par 这么一来,当天色不稳的时候,我从大清早就一个劲儿抬头观天,注意一切征兆。如果对门那位太太在窗口戴上帽子,我就心想:“这位太太要出门了,所以这是个可以出门的天气,希尔贝特会不会跟这位太太一样行事呢?”可是天色逐渐阴沉下来,不过妈妈说只要有一丝阳光,天色还能转亮,但多半还是会下雨的;如果下雨的话,那干吗上香榭丽舍去呢?所以,从吃过午饭,我那焦躁不安的双眼就一直盯着那布满云彩、不大可靠的天空。天色依然阴沉。窗外阳台上是一片灰色。忽然间,在一块阴沉沉的石头上,我虽然没有见到稍微光亮一点的颜色,却感觉到有一条摇曳不定的光线想要把它的光芒释放出来,似乎在作出一番努力,要现出稍微光亮一点的颜色。再过一会儿,阳台成了一片苍白,像晨间的水面那样反射出万道微光,映照在阳台的铁栅栏上。一阵微风又把这条条光照吹散,石头又变得阴暗起来;然而这万道微光像已经被你驯养了似的又回来了;石头在不知不觉之中重新开始发白,而正如在一首序曲中最后那些越来越强的渐强音,通过所有过渡的音符,把唯一的那个音符引到最强音的地位一样,只见那块石头居然已经变成晴朗之日那成了定局、不可交易的灿烂金色,栏杆上铁条投上的影子现出一片漆黑,倒像是一片随心所欲不受约束的植被,轮廓勾勒得纤细入微,显露出艺术家的一番匠心和满意心情,而这些映照在阳光之湖上的宽阔而枝叶茂盛的光线是如此轮廓分明,如此柔软平滑,又是如此幸福沉静地栖息在那里,仿佛它们知道自己就是宁静和幸福的保证。
\par 这是信笔勾成的常春藤,这是短暂易逝的爬墙草!在许多人的心目中,是所有那些能攀缘墙壁或者装点窗户的草木当中最缺乏色彩,最令人凄然的一种;可对我而言,自从它在我们的阳台上出现的那一天,自从它暗示着希尔贝特也许已经到了香榭丽舍的那一天起,它就成了一切草木中最弥足珍贵的一种,而当我一到那里,她就会对我说:“咱们先玩捉俘虏游戏,您跟我在一边。”但这暗示是脆弱的,会被一阵风刮走,同时也不与季节而与钟点有关;这是这一天或拒绝或兑现的一个瞬即实现的幸福的诺言,而且是一个了不起的瞬即兑现的幸福,是爱情的幸福;它比附在石头上的苔藓更甜蜜更温暖;它充满生机,只要一道光线就可以催它出世,就可以开放出欢快的鲜花,哪怕这是在三九隆冬。
\par 后来,花草树木都已凋零,裹着万年老树树干的好看的绿皮也都蒙上了一层雪花。每当雪虽然已经不下,但天气还太阴沉,难以指望希尔贝特会出来的时候,我就施出计谋让妈妈亲口说出:“嗯,这会儿倒是晴了;你们也许可以出去试试,上香榭丽舍走上一遭。”在覆盖着阳台的那块雪毯上,刚露脸的太阳缝上了道道金线,现出暗淡的阴影。在那我们谁也没有瞧见,也没有见到任何玩罢即将回家的姑娘对我讲一声希尔贝特今天不来。平常那些道貌岸然可是特别怕冷的家庭女教师坐的椅子都空无一人,只有草坪附近坐着一位上了年纪的太太,她是不管什么天气都来,永远穿着同样一种款式的衣服,挺讲究然而颜色暗淡。如果权力操之我手的话,为了认识这位太太,我当时真会把我未来的一生中的一切最大的利益奉献出来。因为希尔贝特每天都来跟她打招呼;她则向希尔贝特打听“她亲爱的母亲”的消息;我仿佛觉得,如果我认识这位太太的话,我在希尔贝特心目中就会是另外一种人,是认识她父母的亲友的人了。当她的孙男孙女在远处玩的时候,她总是一心阅读《论坛报》,把它称之为“我的老论坛报”,还总以贵族的派头说起城里的警察或者租椅子的女人,说什么“我那位当警察的老朋友”,什么“那租椅子的跟我是老朋友”等等。
\par 弗朗索瓦丝老待着不动就太冷了,所以我们就一直走到协和桥上去看上冻了的塞纳河。每个人,包括孩子在内,都毫无惧色地接近,仿佛它是一条搁浅了的鲸鱼,一筹莫展,谁都可以随意把它剁成碎块。我们又回到香榭丽舍,我在那些一动也不动的木马跟雪白一片的草坪之间难过得要命,草坪四周小道上的积雪已经扫走,又组成了一个黑色的网,草坪上那个雕像指尖垂着一条冰凌,仿佛说明这就是她为什么要把胳膊伸出来的原因。那位老太太已经把她的《论坛报》叠了起来,问经过身边的保育员几点钟了,并一个劲儿说“您真好!”来向她道谢。她又请养路工人叫她的儿孙回来,说她感到冷了,还找补上一句:“您真是太好了,我真不好意思。”忽然间,天空裂了一道缝:在木偶戏剧场和马戏场之间,在那变得好看的地平线上,我忽然看见那位小姐帽子上的蓝色翎毛,这真是个难以置信的吉兆。希尔贝特已经飞快地朝我这个方向奔来,她戴了一顶裘皮的无边软帽,满面红光,由于天寒、来迟和急于要玩而兴致勃勃;在跑到我身边以前,她在冰上滑了一下,为了保持平衡,也许是因为觉得这姿势优美,也许还是为了摆出一副溜冰运动员的架势,她就那么把双臂向左右平伸,微笑着向前奔来,仿佛是要把我抱进她的怀中。“好啊!好啊!真是太妙了!我是另外一个时代的人,是从旧社会过来的人,要不然的话,我真要跟你那样说这真是太棒了,太够味了!”老太太高声叫道,仿佛是代表香榭丽舍感谢希尔贝特不顾天寒地冻而来似的。“你跟我一样,对咱们这亲爱的香榭丽舍是忠贞不渝的,咱们两个都是大无畏的勇士。我对香榭丽舍可说是一往情深。不怕你见笑,这雪哪,它叫我想起了白鼬皮来了。”说着,她当真哈哈大笑起来。
\par 这雪的景象代表着一股力量,足以使我无法见到希尔贝特,这些日子的第一天本会产生见不了面的愁苦,甚至会显得是一个离别的日子,因为它改变了我们唯一的见面地点的面貌,甚至影响到它能不能充当这个地点,因为现在起了变化,什么都笼罩在一个巨大的防尘罩底下了——然而这一天却促使我的爱情向前进了一步,因为这仿佛是她第一次跟我分担了忧患。那天我们这一伙中就只有我们两个人,而像这样跟她单独相处,不仅是亲密相处的开始,而且对她来说,冒着这样的天气前来仿佛完全就是为了我,这就跟有一天她本来要应邀参加午后一个约会,结果为了到香榭丽舍来和我见面而谢绝邀请同样感人肺腑;我们的友情在这毫无生气、孤寂、衰败的周围环境中依然生动活跃,我对它的生命力,对它的前途更加充满了信心;当她把小雪球塞到我脖子里去的时候,我亲切地微笑了,觉得这既表明她喜欢在这披上冬装,焕然一新的景区有我这样一个旅伴,又表明她愿在困境之中保持对我的忠贞。不一会儿,她那些伙伴就都跟犹豫不决的麻雀一样,一个接着一个来了,在洁白的雪地上缀上几个黑点。我们开始玩了起来,仿佛这一天开始时是如此凄惨,却要在欢快中结束似的,当我在玩捉俘虏游戏之前,走到我第一次听到希尔贝特的名字那天用尖嗓门叫喊的那个姑娘跟前的时候,她对我说:“不,不,我们都知道,您是爱跟希尔贝特在一边的,再说,她都已经在跟您打招呼了。”她果然在叫我上积满白雪的草坪上她那一边去。阳光灿烂,在草坪上照出万道金光,像是古代金线锦缎中的金线一般,倒叫人想起了金锦营\footnote{金锦营:1520年,法王弗朗索瓦一世与英王亨利八世在加来海峡某地聚会,拟签订盟约共同对付神圣罗马帝国皇帝查理五世。双方争奇斗艳,用金钱锦缎将营地装饰得金碧辉煌,而盟约却未订成。}来了。
\par 这一天开始时我曾如此忧心忡忡,结果却成了我难得感到不太不幸的一天。
\par 我都已经认为从此再也不会有一天看不见希尔贝特的了(以至有一回,我外祖母没有按时回来吃晚饭,我居然立即想道,如果她是被车轧死了,那我就不能上香榭丽舍去了。当你爱一个人的时候,你就不会对第二个人有什么爱了),然而有时从头天起,我虽然已如此焦急地等待,以至宁愿为这一时刻牺牲一切,但一旦当我就在她身边时,却并不感到这是幸福的时刻;我自己也明白,因为在我的一生当中,我只在这样的时刻身上才集中了热切细微的关注,这样的时刻本身是不会产生任何欢快的原子的。
\par 当我远离希尔贝特的时候,我需要能看见她,因为老是在脑子里想象她那副形象,想着想着就想不出来了,结果也就不能精确地知道我所爱的对象到底是什么样子。再说,她也从来没有对我说过她爱我。恰恰相反,她倒时常说她更喜欢某些男孩,说我是个好伙伴,乐于跟我一起玩,但我太不专心,不把心思都放在游戏上;而且她还时常对我作出明显的冷淡的表示,动摇我的信念,使我难以相信我在她心中的地位跟别人有所不同,如果我这份信念出之于希尔贝特对我的爱,而不是像事实那样出之于我对她的爱的话,那么这个信念就会是十分坚强,因为它是随我出之于内心的要求而不得不思念希尔贝特时的方式而异的。但我对她的感情,我自己还没有向她倾诉过。当然,在我每一本练习本的每一页上,我都写满了她的名字和她的住址,但当我看到我潦潦草草地勾画而她并不因此而想起我的这些字行,它们使她在我周围占了这么多显而易见的地位而她并不因此而进一步介入我的生活,我不禁感到泄气,因为这些字行所表示的并不是连看都看不见它们的希尔贝特,而是我自己的愿望,因此它们在我心目中就显得是纯粹主观的、不现实的、枯燥乏味的、产生不了结果的东西。最紧要的事情是希尔贝特跟我得见面,能够互相倾吐衷肠——这份爱情直到那时可说是还没有开场呢。当然,促使我如此急于要跟她会面的种种理由,对一个成熟的男人来说,就不会那么迫切。到了后来,等到我们对乐趣的培养有了经验,我们就满足于想念一个女人(就像我想念希尔贝特一样)这份乐趣,就不去操心这个形象是否符合实际,同时也就满足于爱她的乐趣,而无需确信她是否爱你;我们还放弃向她承认我们对她的爱恋这样一种乐趣,以便使她对我们的爱恋维持得更强烈——这是学日本园艺师的榜样,他们为了培植一种好看的花,不惜牺牲好几种别的花。当我爱希尔贝特那时节,我还以为爱情当真在我们身外客观实际地存在着;以为只要让我们尽量排除障碍,爱情就会在我们无力作任何变动的范围内为我们提供幸福;我仿佛觉得,如果我自觉自愿地用假装的不动感情来代替承认爱情这种甘美,我就不仅会剥夺自己最最梦寐以求的那份欢愉,也可以以我自己的自由意志,制造一份虚假的、没有价值的、与现实毫无关系的爱情,而我就会拒绝沿着它那条神秘的、命中注定的道路前进。
\par 但当我走到香榭丽舍,首先可以面对我的爱情,把这份爱情的非我所能控制而有其独立生命的原因加以必要的修正时,当我真的站到希尔贝特·斯万面前(这个希尔贝特·斯万,昨天我那疲惫不堪的脑子,已经再也想不起她的形象,我一直指望在再见到她时使这形象变得新鲜起来;这个希尔贝特·斯万,昨天我还同她一起玩来着,刚才我身上却有个盲目的本能促使我把她认了出来,打个招呼,这就跟我们走路这个本能一样,在我们还没有去想以前就先迈一只脚,再迈另一只脚),这时我忽然觉得,她跟我梦中所见的那个对象完全不一样。譬如说,昨天我脑子里记住的是丰满红润的面颊上的两只炯炯逼人的眼,现在希尔贝特固执地显现出来的那副面目却恰恰是我不曾想到的:一个尖尖长长的鼻子,再加面部的其他线条,构成了许多鲜明的特征,在生物学中简直可以用来与别的种属有所区别,使她成了一个尖鼻子类型的小姑娘。正当我准备利用这求之不得的时刻,根据我来以前在脑子里所准备、然而现在又不再见到的希尔贝特的形象,来帮我弄个一清二楚,使我在不在她身边的漫长时刻中,能确信我所记得的的确就是她,能确信我像写书那样日积月累地积累起来的爱情的确是以她为对象的,恰恰在这个时刻,她向我扔过一个球来,正像一个唯心主义的哲学家,他的肉体考虑到外部世界的存在,可他的头脑却不相信外部世界这个现实一样,刚才还没有把她确认为何许人就跟她打起招呼来的这个“我”,现在又赶忙叫我把她扔过来的球接住(仿佛她是我来与之游戏的游伴,而不是来与之聚首的一颗姐妹般的心灵似的),这个“我”也使得我出于礼貌,跟她说上千百句虽然亲切然而并无意义的话,但却阻止我在她走开之前,或者保持沉默,利用这机会把对我来说是必不可少然而时常逃逸的她在我脑中的形象固定下来,或者对她讲几句话,使我们的爱情能取得有决定意义的进展,而这种进展我总是今天推明天,明天推后天地不去积极争取的。
\par 我们的爱情毕竟也取得一些进展。有一天,我们跟希尔贝特一起一直走到跟我们特别友好的那些女商贩的木棚子跟前——斯万先生就是在她那里买香料蜜糖面包的。为了卫生的缘故,这种面包他每天吃得很多,因为他患有种族遗传性的湿疹,又闹便秘。希尔贝特笑着把两个小男孩指给我看,这两个孩子看着像是儿童读物里说到的调色专家和博物学家。其中之一不要红颜色的麦芽糖,非要根紫的不可,另一个则双眼含泪,拒绝女仆想给他买的那只李子,他以感人的语调解释道:“我挑中那一只李子,是因为它上面有个蛀洞!”我花了一个苏买了两个弹球。我满怀深情地瞧着放在一只木钵子里的两颗玛瑙球,闪闪发光,老老实实地监禁在钵子里;我觉得它们非常宝贵,一则是它们像小姑娘那样笑容可掬,满头金发,二则它们每个都值五十生丁。希尔贝特家里人给她的钱比我多得多,我希望她能把两个全买下来,把它们从监禁之中解脱出来。这两颗玛瑙球既透明晶莹,又像生命那样朦胧不清,要问我哪一个更美,我实在不想贬一褒一。可是我还是指着跟她的头发同样颜色的那一颗。希尔贝特把它拿了出来,看到上头有道金色的纹,吻了一吻,把这囚徒赎了出来,然后马上就把它交给了我,说:“拿着,它是您的了,给您,留作纪念吧。”
\par 又有一次,正当我一心想看拉贝玛在一出名剧里的演出时,我问她有没有贝戈特谈拉辛的那本小册子,因为市面上买不着了。她要我把书的全名告诉她,我当晚就给她打了一份电报,把我那早就在练习本上画过不知多少次的“希尔贝特·斯万”这个名字写在封套上。第二天,她就把她找到的那本书用浅紫色的缎带扎上,用白蜡加封带给了我。“您看,这正是您要的那本。”她说,一面从她的手笼里把我给她的那份电报抽了出来。这封气压传递的函件昨天还不代表什么东西,只不过是我写的一张蓝纸,可自从投递员把它交给希尔贝特家的门房,有个仆人把它送进她的房间,就变成了这个无价之宝,成了她那天收到的一份气压传递的急件——那上面尽是邮局盖上的圆圈,邮差用铅笔添上的字迹,这些都是邮途完成的记号,是外部世界的印记,是象征生命的紫罗兰色的腰带,它们是第一次来赞许、维持、提高、鼓舞我的梦想,我连自己所写的稀稀拉拉,模模糊糊的字迹都辨认不出来了。
\par 有天她又对我说:“您哪,您尽管叫我希尔贝特好了,可我还是叫您的教名。不然就太别扭了。”可有一段时间,她还是继续用“您”称呼我,当我提醒她的时候,她笑笑,然后编了一句像我们在学外语语法时除了练习用某个新词以外别无任何其他目的的句子,用我的小名结尾。当我后来回想我当时的感受时,我还有这样一个印象,仿佛我曾一度赤条条地被她衔在嘴里,不再具有像她同学们那样的社会身份,当她叫我的姓的时候,也不再具有我父母那样的社会身份,而她的双唇,当她有点像她的父亲那样,作出努力来把她所要强调的词语加以重读时,又仿佛是在剥去我的衣服,就如同剥去一只水果的皮,只吃它的果肉一样,而她的眼神,跟她的言语变得同样更加亲切,也就更直接地投上我身,并且随之以一个微笑,以表明她的认真、乐趣,甚至是感激之情。
\par 然而就在那时,我也不能体会这些新的乐趣的价值。这些乐趣并不是由一个我所爱的女孩给爱着她的我的,而是一个跟我一起玩的女孩给那脑子里对真正的希尔贝特毫无印象,也缺乏一颗能体会这幸福的价值的心(唯有这样一颗心才能体会这份价值)的另一个我的。即使是当我回到了家里,我也品尝不出这些乐趣,因为我每天不得不把对希尔贝特作一番认真、沉静、幸福的凝视的希望推到明天,也希望她终于能表白她对我的爱,把她迄今把这份爱隐藏起来的原因讲个明白;也正是这种必要,使我把过去看得无足轻重,一心只向前看,把她对我的种种友好表示并不仅仅看作是一般的表示,而把它们看成是一层一层台阶,使我可以步步升高,终于达到迄今还没有进入的幸福境界。
\par 她有时给我一些友好的表示,可有时也显得并不乐意跟我见面,这叫我难过,而这种情况时常正是在我认为最能实现我的希望的那些日子发生。我确信希尔贝特要到香榭丽舍去,我感到一阵欢快,而且觉得它预示着一个巨大的幸福。当我一早走进客厅去亲吻妈妈时,她早就整装待发,漆黑的发髻已经梳就,又白又胖的好看的双手犹有肥皂的香泽,只见钢琴上直挺挺地立着一个尘埃的光柱,又听得窗外有手摇风琴演奏《阅兵归来》这个曲子,我这才意识到就在昨晚,寒冬已经逝去,出人意料地迎来了灿烂的春天。当我们吃午餐的时候,住在我们对面的那位太太一开窗,就在刹那之间使得一道阳光从我椅子旁边掠过,一步就横扫整个饭厅,就在那儿开始午休,过了一会儿又回来继续休息。在学校里,当我上一点钟那堂课时,太阳以它金色的光芒照上我的书桌,使我十分焦躁不安,因为它像是在邀请我去过节,而我在三点以前又无法应邀,得等到那时候,弗朗索瓦丝才能到校门口来接我,一起走过那染上金色阳光,行人熙来攘往的街道,向香榭丽舍走去;马路两旁的阳台,像是被太阳从墙上卸了下来,冒着热气,像金色的云彩一样在房屋前面飘荡。唉!可在香榭丽舍,我没有看到希尔贝特,她还没有来到。我在这被看不见的太阳培育出来的草坪上坐着一动也不动,这太阳把各处的草尖都照得通红,在草坪上栖息的鸽子像是由园丁的镐头发掘到的这圣洁的土地上的一座座古代雕像,我双眼盯着地平线,随时都在等待希尔贝特的身影随着她的家庭女教师从那座雕像背后一起出现;那座雕像像是把她手上抱着的沐浴着阳光的孩子举向前方,让他接受太阳的祝福。《论坛报》的那位女读者坐在她那扶手椅里,还是在那老位置,她亲切地向一个园丁招手,对他叫道:“多美好的天气!”租椅子的女工走到她跟前收费,她做出千娇百态,把那张十生丁的租金券塞进她手套的开口处,倒仿佛这是一束鲜花,为了显示对赠予人的感激之情,要找一个最讨对方喜欢的地方插上似的。当她找到了这个位置,她把脑袋晃了一圈,把圆筒形皮毛围巾拽一拽,把露在手腕子那里那张黄色纸片的一端让她瞧一眼,脸上带着一个女人指着她的胸口对小伙子说“你看,这是你送给我的玫瑰花!”时的那种微笑。
\par 我领着弗朗索瓦丝去迎希尔贝特,一直走到凯旋门,可没有碰上她,我心想她准是不来了,就回到草坪那里去,可忽然在木马前面,那个尖嗓门的小女孩向我跑来:“快,快,希尔贝特已经来了一刻钟,都要走了。我们在等您玩捉俘虏呢。”原来刚才当我沿着香榭丽舍大街走的时候,希尔贝特从布瓦西当格拉街来了,小姐趁这好天气去为自己买点东西,而斯万先生也来找他女儿了。所以这就是我的不是了,我原不该远离草坪的,谁也不确有把握地知道希尔贝特准从哪条道来,是早还是晚,这一等待使我觉得不仅整条香榭丽舍大街跟整个下午都使我更加激动——它们像是一长段时空,在其中的每一个点,每一个时刻,希尔贝特的形象都可能出现——而且希尔贝特这个形象本身也使我更加激动,因为在这形象背后,我感到隐藏着的那支箭之所以不是在两点半而是在四点钟击中我心头的道理;她今天不是戴着体育锻炼时的贝雷帽,而是一顶出客的帽子;在大使剧院前面,而不是在两个木偶剧场之间出现,我这就依稀看到在我不能跟随希尔贝特时她干了点什么事情,又是什么事情使她不能不出门或者不能不呆在家里,我这就跟她那时对我来说是陌生的那部分生活的奥秘有了一点接触。当我按照那尖嗓门女孩的指示马上开始我们的捉俘虏游戏时,只见希尔贝特在我们面前是如此活跃莽撞,对那位读《论坛报》的夫人(她对她说:“多好的太阳,简直像是一团火”)恭恭敬敬地行了个屈膝礼,带着腼腆的笑跟她说话,那副拘谨的神气使我看到跟在她父母家里、在她父母的朋友身边、在外出访客、在我所不熟悉的她的那部分生活中的希尔贝特不一样的一个小姑娘,而也正是我所不熟悉的她的那部分生活的奥妙使我感到心中如此激动。但她那部分生活究竟是怎么回事呢?其中使我得到最深刻的印象的还是斯万先生,他过了一会儿就来接他的女儿来了。希尔贝特住在她父母家里,她在学习、游戏、交朋友等方面都是听他们话的,所以对我来说,斯万先生和斯万夫人身上有着一个难以企及的未知的事物,有着一种令人阴郁的魅力,这在希尔贝特身上也是一样,但他们比她更有过之,因为他们对她仿佛是全能的神,是她身上那种品质的根源所在。对我来说,凡是与他们有关的事情都是我经常关注的对象;斯万先生当年在跟我父母交往的时候是我时常见面的,但并没有引起我的好奇,现在在他到香榭丽舍来接希尔贝特的日子,我一看到他那顶灰色的帽子和那件披风式的短大衣时,心头就不禁突突地跳将起来,直到平静了下来,他那副容貌还像我们刚读了关于他的一系列作品,他那些最细微的特点还在使我们激动不已的一个历史人物那样感动着我。当我在贡布雷听人说起他跟巴黎伯爵之间的交往时,我仿佛觉得那跟我毫无关系,现在在我眼里却成了了不起的东西,仿佛除他之外再也没有谁跟奥尔良家族中的人相识的了;现在他混迹于在香榭丽舍熙来攘往的各色人等的浊流之中,观察他们而并不要求他们对他另眼相看(他穿戴得那样平常,谁也想不起要对他另眼看待),却正是那些交往使得他如此超凡出众。
\par 他对希尔贝特的伙伴们的问候彬彬有礼地还礼,即使对我也是如此,虽然他曾跟我家有过龃龉,不过看样子他也并没有把我认出来(这倒使我想起,他在乡间可是经常跟我见面的;这我还记得起来,不过记忆已经模糊,因为自从我见到希尔贝特以后,在我心目中斯万主要是她的父亲,不再是贡布雷的那个斯万;现在我把他的名字所归的类别跟当年它所纳入的那个系列中所容的概念完全不同,而当我现在必须想起他的时候,再也用不着那个系列了,因为他已经成了另外一个人;然而我依然还是通过一条人为的、次要的、横向的线把他跟我们家当年这位客人联系起来;既然除了在我的爱情还能从中得到好处这样一个范围以外,任何事物都没有什么价值,当我回顾那些岁月时,我是带着不能把它们一笔勾销的羞愧和遗憾之情的;现在在香榭丽舍站在我面前的这个斯万——幸好希尔贝特可能还没有对他提起我姓甚名谁,当年在他眼里我可时常是如此可笑,因为当妈妈跟他,还有爸爸和外祖父母一起在花园里的桌子上喝咖啡的时候,我常打发人去请妈妈上楼到我卧室里来互道晚安)。他对希尔贝特说,他可以让她玩一盘,可以等她一刻钟,然后就跟所有的人一样在铁椅子上坐下,用当年菲利浦七世经常紧握的那只手掏出钱来付租金,我们就在草坪上玩将起来,把那长着彩虹色美丽身体的鸽子轰向天空(它们的身体呈心形,是鸟类王国中的百合花),让它们栖息到安全的所在地,有的飞到大石钵上,低下头来,嘴巴看不见了,表示这里盛满了喂它们的水果或者谷粒;有的栖上雕像的前额,倒像是某些古代作品中为了使那千篇一律的石头的色调多少有点变化而添上的彩釉饰物,而当戴这饰物的是一个女神的时候,也就给这尊像添上一个特定的形容词(就跟我们凡人都有不同的名字一样),这就使它成了一个新的神祇。
\par 在这么一个阳光灿烂的日子,我的希望没有实现,我这天再也没有勇气把我的失望心情对希尔贝特掩藏起来了。
\par “我刚才正有许多话要问您呢,”我对她说,“我觉得今天这个日子对我们的友情有重要的意义,可您刚一到就要走了!明天想法子早点来,好让我跟您说说。”
\par 她脸上容光焕发,高兴得跳起来答道:
\par “朋友,明天您可别指望了,我来不了!下午有午茶会;后天也来不了,我要上一个朋友家窗口去看狄奥多西国王驾到的行列,好看着呢;后天要去看《米歇尔·斯特罗戈夫》\footnote{《米歇尔·斯特罗戈夫》是根据儒勒·凡尔纳同名惊险小说改编的剧本。},再过几天就是圣诞跟年假了。可能家里要把我带到南方去,那可就太棒了!只不过要是上南方去,我就要少得到一棵圣诞树;反正即使我呆在巴黎,我也不到这儿来了,我要跟妈妈串门去。再见了,爸爸在叫我了。”
\par 我跟弗朗索瓦丝从夕阳依然斜照的街道回家,然而却像是在一个欢庆活动已经结束了的夜晚似的。我都迈不开双腿了。
\par “这没有什么可奇怪的,”弗朗索瓦丝说,“今年天时不正,这个冬天太暖和。唉!上帝哪!到处都是闹病的穷人,简直是连天上也都乱了套。”
\par 我强压哽咽,在心里反复琢磨刚才希尔贝特兴高采烈地所说她好些日子来不了香榭丽舍那番话。然而只要当我一想到她的时候,自然而然地就有一股魅力充满我的心房;还有在跟希尔贝特的关系当中,由于我心头有这样一份创痛,我是不可避免地占有一个特殊的,也是唯一的地位(尽管是令人痛苦的),这地位跟那份魅力相结合,就在希尔贝特那份冷淡之中添上点罗曼蒂克的色彩,而在我的泪中也就出现了一丝微笑——这该是一个吻的怯生生的雏形吧。等到邮差送信的时刻到来时,这晚我跟每天晚上一样心想:“我就要收到希尔贝特的信了,她会告诉我,她从来没有中止对我的爱,她会向我解释是为了什么神秘的理由她才不得不直到此刻还把她对我的爱隐藏在心,装出为不能见着我而高兴,会向我解释是为了什么她才只扮演一个普通伙伴的角色的。”
\par 每天晚上我都乐于想象这样一封来信,我在心里默读,每一句话都背得出来。突然间,我怔住了。我明白,如果我接到希尔贝特的信的话,那决不会是这样一封,因为这封是我自己编出来的。从此以后,我就竭力不去想我希望她给我写的那些字眼,生怕老是这么念叨,结果恰恰把这些最弥足珍贵,最最盼望的词语从可能实现的领域中排除出去。即使出之于极不可能的巧合,希尔贝特写给我的信果然正好就像我自己编造的那样,能从中看出是我的作品,那我得到的将是收到一件出之我手的东西的印象,就不是什么真实的、新的、与我的主观思想无关、跟我的意志无涉、真正是由爱情产生的东西了。
\par 此刻我在重读一页,虽不是希尔贝特写给我的,却至少得自她手,那是贝戈特所写关于启发拉辛的古老神话之美的那一页,这本书一直跟那颗玛瑙球一样,摆在我手头。我的朋友为我搜求这部书,我很受感动;每一个人都要找出他的激情之所以产生的理由,直至认为在他所爱的对象身上具有在文学作品或者谈话中所说的那些值得人们爱的品质,同时通过模仿,把他所爱的对象身上的品质跟这些品质等同起来,使之成为他之所以有那份爱情的新的理由,尽管这些品质可能跟他不依赖他人教导而主动追求时所要求的品质截然相反,这就跟当年的斯万对奥黛特之美的美学性质一样。我呢,早在贡布雷时就爱上了希尔贝特,那时因为我对她的生活一无所知,希望自己能够投身进去,化入其中,把我那份自己已经感到毫不足道的生活舍弃,现在我则想,在我自己这个已经太熟悉,太不足道的生活当中,希尔贝特有朝一日可以来充当一个谦卑的仆人,成为我的得力助手,晚上可以帮我工作,看看我写的小册子里有没有错误,这该有无比的好处。至于贝戈特这位无比睿智,几乎超凡入圣的长者,我本是由于他才在认识希尔贝特以前就爱上她的,现在却是由于希尔贝特的缘故我才爱他本人。我以无比的乐趣读他所写的关于拉辛的篇页,我也以同样的乐趣瞧着她在把这本书送给我时那盖有白蜡印记,系有淡紫色丝带的包装纸。我吻着玛瑙球,这是我的朋友的心的最优秀的部分,是毫不轻浮十分忠贞的部分,同时虽然带有希尔贝特的生活中的神秘魅力,却一直呆在我的卧室里,与我同床而卧。但这块宝石之美,还有我乐于与之跟对希尔贝特的爱相联系的贝戈特作品之美,在我仿佛觉得希尔贝特对我的爱已经几乎化为乌有的此时此刻,这两种美却给它以凝聚之力,我发现这两种美比那份爱情出现得还早,跟这份爱情毫无相似之处,它们的内容取决于希尔贝特认识我以前早就存在的那份天才,取决于那些矿物学的规律,如果希尔贝特不曾爱我,这本书,这块石头也不会是另外一种样子,因此在这两者中间没有什么会给我带来任何幸福的信息。而我对希尔贝特的爱天天都在等待着第二天会得到希尔贝特的表白,每天晚上都把我在白天胡乱干的活计拆掉,而与此同时,在我心中暗处也有一个不相识的女工却不愿把我拆下的线扔掉,还要把它整理起来,全然无意取悦于我,也不为我的幸福着想,跟她干别的活时完全背其道而行之。这个不相识的女工对我对希尔贝特的爱情毫不感兴趣,也不首先就肯定我在被她爱着,却把希尔贝特做过的我认为无法解释的行动和已经得到我原谅了的她的过失都汇集起来。这样一来,两者就都具有了一定的意义。这样一种新的想法仿佛表明,当我看到希尔贝特不上香榭丽舍,而去看什么日场演出,或者跟她的家庭女教师去买什么东西,准备出门去度新年假期的时候,我就不该说她是什么轻浮或者是什么老实听话了。如果她爱我的话,她就既不会那么轻浮,也不会那么老实听话,而当她不得不听别人话的时候,那么在我见不着她的那些日子里,她心中应该同我一样地感到失望。这样一种新的想法还说明,既然我爱希尔贝特,我就应该懂得什么叫爱;这新的想法促使我注意到我老在想要在她心目中抬高自己的身价,因此力图说服母亲为弗朗索瓦丝买一件雨衣和一顶带蓝翎毛的帽子,或者别再让叫我害臊的这个女仆陪着上香榭丽舍(妈妈说我对弗朗索瓦丝不公道,说她是对我们家忠心耿耿的好人);这新的想法也促使我注意到,见到希尔贝特这个唯一的愿望使得我早在她走以前几个月就一心只想打听她什么时候离开巴黎,又上哪儿去,觉得如果她不在的话,那么世上最引人入胜的地方也只能算是一个隐遁之所,而只要能在香榭丽舍见到她,那我就愿意一辈子呆在巴黎;很清楚,我这个担心和愿望在希尔贝特的行动中是找不出来的。恰恰相反,她很喜欢她那家庭女教师,从来也不为我对这有什么看法而操心。她觉得,如果是为了陪小姐去买东西而不到香榭丽舍来,那是很自然的,而要是为了陪她母亲出去而不来,那更是惬意了。即使她同意我在同一地点和她度假,那么要选定这个地点,她至少得尊重她父母的意见,得考虑到她同我说过的那种种游乐,而决不会上我家里有意把我送去的那个地方。当她有几次对我说,她更喜欢另一个男朋友,或者她已经不像头天那么喜欢我,因为我粗心大意而叫她在游戏时输了一盘时,我就向她道歉,问她该怎么办才能重得她往日的欢心,使她喜欢我有过于任何别人;我希望她对我说她喜欢我本来就有过于别人;我恳求她说这句话,仿佛她可以随她高兴或者随我高兴,仅仅凭她根据我的行为是好是坏而说出来的几句话,就能随意变动她对我的感情似的。难道我那时不知道,我自己对她的感情不是既不取决于她的行为,也不取决于我的意志吗?
\par 在我心中暗处的那位不相识的女工所建立起来的新秩序还告诉我们,如果我们希望迄今为止伤了我们心的某个人的所作所为并非出于真心,那么它们就会射出一道我们的意愿无法熄灭的光芒,我们应该通过这道光芒,而不是通过我们自己的意愿去看看他明天的所作所为又将是怎样。
\par 这些新的话语,我的爱情是听到了的;这些话语使它信服,明天不会跟已逝的日子有什么两样;希尔贝特对我的感情已经年深日久,不可能有所改变,只能是冷漠而已;至于我对希尔贝特的爱情,爱着的只是我这一方面。我的爱情答道:“是的,对这份友情已经无计可施,它是不会改变的。”这样,明天一来(或者等个最近的节庆日子,等个周年纪念,或者是元旦,反正是与众有所不同的一个日子,到那时时间会抛弃过去的遗产,拒绝接受它留下的凄楚,另起炉灶),到那时,我会要求希尔贝特抛弃我们的旧友情,奠定我们新的友情的基础。
\par 我手头总有一张巴黎街道图,因为可以从中看到斯万夫妇所住的那条街,所以我觉得它装着一份财宝。出之于爱好,也出之于一种骑士式的忠诚,不管是谈到什么,我总要讲出这条街的名字,以至我父亲(他不像我母亲和我外祖母那样知道我在爱着一个人)问我:
\par “你干吗老是说起这条街?它没有什么特别的,只是因为紧挨着布洛尼林园,所以是个很宜人的住处,同样的街道也能数出十来处呢。”
\par 也不管是谈到什么,我总要引我父母说出斯万这个姓氏来;当然我马上就在心里默默地重复;不过我也需要听到它那悦耳的铿锵声,让我听听这个乐音——单是默读是不够的。再说,斯万这个姓氏虽然我早就知道,现在都像某些患失语症的人对最常用的词也感到新鲜一样,对我也成了一个新词。这词老在我的脑际,可我的脑子对它老是习惯不了。我把这个词加以分解,一个一个字母地拼读,它的拼法对我简直是个意外的发现。随着它变得越来越熟悉,我也就觉得它越来越不那么清白无瑕。我在听到这个词时所得的乐趣,我都心想它已经是如此有罪,仿佛别人已经猜透了我的心思,所以当我竭力把谈话向这方向引的时候,他们就转换话题。我一个劲儿转到跟希尔贝特有关的话头上来,老是重复那些话语——这些话在远离她的地方说出来,她也听不见,不过是些只能重复说明现状而不能改变现状的一无用处的话语——然而我仿佛觉得把希尔贝特身边的事这么折腾折腾,翻弄翻弄,也许可能从中得出点可喜的东西。我一再重复那位读《论坛报》的老太太对她的夸奖(我向我父母暗示,她是一位大使夫人,甚至是位亲王夫人),继续说这位老太太是多么美,多么大方,多么高贵,直到有一天我把从希尔贝特嘴里听到的她的名字说了出来——她叫布拉当太太。
\par “哈!现在我明白了!”我母亲尖叫起来,我感到自己脸上羞得发热,“你外祖父听了准要叫你小心又小心。你居然会觉得她长得美!她可长得实在难看,这辈子也没好看过。她是个执达吏的遗孀。你大概不记得了,在你小时候,我费了多少心血才阻止她来看你接受体育锻炼。我并不认识她,她可老是想跟我搭讪,假说是为了告诉我‘你长得好看得简直像个小美女’。这个女人从来都有那么一股子交结朋友的瘾;我一直这么想,她要是当真认识斯万太太,那她准是得了神经病了。因为这个女的虽然出身低微,可从来还没做过什么招人非议的事来。她就是一个劲儿要跟人拉关系。她长得难看,极其庸俗,而且爱惹是生非。”
\par 至于斯万,为了要使我自己长得跟他相像,我成天都在桌子边坐下,一个劲儿把鼻子拽长,一个劲儿揉眼睛。我父亲说:“这孩子傻了,简直讨厌透顶了。”我简直希望自己也跟斯万那样来个秃顶。我觉得他是如此不同凡响,有些我常交往的人居然也认识他,而且哪天都能碰巧碰上他,这简直令人难以置信。有一次,母亲正跟每天在吃晚饭时一样讲着她下午买了些什么东西的时候,忽然讲起:“对了,你们猜猜我在三区商店雨伞部碰见谁了?是斯万!”她讲的那些话本来对我是索然乏味,这下却催开了一朵神秘的鲜花!真是叫人听了既得到满足,又感到伤心,斯万今天下午怎么会在那人群里亮出他那神乎其神的身影去买一把雨伞!在那些同样与我无关的大大小小的事情当中,这一件事情在我心中激起了特殊的震动,我对希尔贝特的爱经常为之激荡。我父亲说我对什么都不感兴趣,因为当大家在谈狄奥多西二世国王此刻作为国宾和盟友在法国的访问将产生的政治影响时,我连听都不听。但与此相反,我是多么想知道当时斯万是不是穿着他那件披风式的短大衣!
\par “你们打招呼了吗?”我问道。
\par “那是当然。”母亲答道,她仿佛担心,如果她承认我们家对斯万冷淡的话,别人就会想法从中调解,超过她所希望的程度,反正她是不想认识斯万夫人的。“是他走上前来跟我打的招呼,我先没有瞧见他。”
\par “这么说来,你们并没有吵翻?”
\par “吵翻?干吗要吵翻?”她尖刻地回答,倒仿佛是我怀疑了关于她和斯万之间的和睦关系的神话,又试图来“拉拢”似的。
\par “他可能怪怨你不邀请他。”
\par “谁也用不着邀请所有的人,他邀请我吗?我不认识他的妻子。”
\par “可从前在贡布雷的时候,他是常来的。”
\par “好吧!在贡布雷的时候他来咱们家,在巴黎他有别的事儿要干,我也一样。不过我可以向你保证,我们压根儿也不像是两个吵翻了的人。我们在商店里一起呆了一阵子,直等到店员把他买的东西打好包为止。他向我打听你的消息,他说你跟他的女儿在一起玩……”母亲这么说着,原来斯万心里还有我呢,这真是个奇迹,叫我怎不惊奇,而且他了解的情况还相当全面,当我在香榭丽舍由于感情激动而在他面前哆嗦时,敢情他知道我姓什么,知道我的母亲是谁,而且除了知道我是他女儿游玩的伙伴以外,还掌握我外祖父母的一些情况,知道他们的家庭,知道我们住在什么地方,还晓得一些连我都可能不晓得的我们家当年生活的特点。不过我母亲在三区商店雨伞部被斯万瞧见,作为一个曾经与之有过共同往事的人物出现在他面前,使得他迎上前来跟她打招呼的时候,她可并没有觉得这次邂逅有什么特殊的魅力。
\par 无论是我母亲也好,还是我父亲也好,仿佛都并不觉得提起希尔贝特的祖父,提起这位证券经纪人来有什么特别的兴趣。我的想象力却从巴黎社交界中把某一个家庭单独抽出来,把它奉为神圣,如同它曾把巴黎这座石头城中的某所房子单独抽出来,把它的大门刻上花纹,把它的窗户彩绘装饰得十分华丽一样。不过这些装饰,只有我才看得见。我的父母认为斯万家住的那所房子跟林园区在同一时期盖的别的那些房子都一样,他们也觉得斯万家跟别的许多股票经纪人家都一样。他们对这个家庭的印象是好是坏,根据它在凡人共同的业绩中参与了几分,根本看不见它有什么独具一格的地方。即使他们发现了什么长处,他们也会在别处看到同样的,甚至犹胜一筹的优点。因此,当他们发现斯万家的位置好时,就说另外还有一所房子位置更好,然而这所房子跟希尔贝特毫无关系,或者是属于比她爷爷资金更雄厚的一些金融家的;万一他们要是一时跟我意见一致,那准是误会,立即就要纠正的。这是因为,我的父母不具备爱情赐给我的那种补充的、瞬时的感觉,所以发现不了希尔贝特周围任何新的品质——这就跟颜色领域里的红外线一样,在感情领域中也是属于肉眼所不见的一种。
\par 在希尔贝特早就通知我她不会来香榭丽舍的那些日子,我就想办法蹓个弯,走到离她所在的地方近一点的处所。有时我领着弗朗索瓦丝到斯万家所住的房子那里去朝圣。我让她把她从那家庭女教师那里听来的关于斯万夫人的话一而再,再而三地讲给我听。“看来她挺迷信的。哪天要是听到猫头鹰叫,或者墙里有钟表的滴答声,或者午夜看见一只猫,或者是木器发出吱吱嘎嘎的响声,那她是准不会外出旅行的。啊!她信教可虔诚了!”我对希尔贝特的爱是如此之深,当我在路上碰见她们家的老厨师头牵着狗出来溜达的时候,我也要带着深情把他那部花白胡须看上半天。弗朗索瓦丝说:
\par “您倒是怎么了?”
\par 然后我们就继续往前走,直到他们家马车出入的大门口,那里有一个跟任何看门人都不一样的看门的,他号衣上的饰带都浸透着我在希尔贝特这个名字里感到的那种令人忧郁的魅力,他仿佛知道我天生就不配进入他奉命守卫的那份神秘的生活,而一楼的那些窗户也仿佛有意识地关得严严实实的,在平纹细布的遮盖下,比任何其他窗户更不像希尔贝特的双眼那样炯炯有神。有时候,我们上环城马路去,我就在迪福街口站着,据说在那里时常可以看到斯万先生上他的牙科大夫诊所去;我的想象力把希尔贝特的父亲看得跟人间的任何人是如此不同,他在现实世界中的出现也会带来如此之多的神奇,以至在走到玛德莱娜教堂之前,当我一想到我们已经离那条可能出乎意料地见到奇迹出现的街不远,心里早就突突直跳了。
\par 然而更多的时候,当我见不着希尔贝特时,由于我听说斯万夫人几乎每天都沿着槐树路,在布洛尼湖岸边,还有在玛格丽特王后小道上散步,我就让弗朗索瓦丝领我上布洛尼林园去。在我心目中,这林园仿佛就是一座座这样的动物园,各色草木无不具备,种种景色层出不穷,翻过小山就看到洞窟、草原、巉岩、河流、沟壑、小丘、沼泽。然而游客也知道那都是为河马、斑马、鳄鱼、俄罗斯兔、狗熊和苍鹭所提供的嬉戏之所,所提供的合适的环境或者如画的背景;至于布洛尼林园,也是十分复杂,集结着许多自成体系的小世界——紧接着像弗吉尼亚州那种栽有美洲橡树这样的红色大树的农场就是湖畔一片松林,或者是一片高耸的乔木,从中突然窜出一位行色匆匆的女子,穿着一身柔软的裘皮衣服,两只眼睛炯炯有神——这是女人的花园;而槐树路,就跟《埃涅阿斯纪》中的爱神木路一样,为了她们就在两旁只种了一种树,这是一条著名的美人们散步的小径。孩子们老远看到岩顶就兴高采烈,他们知道海狮就要从这里跳进水里去,同样,早在走到槐树路以前,清香四溢的槐花也就叫我老远就感到马上就要接近那无与伦比的既强大又柔弱的植物实体,后来我越走越近,看到了树顶轻盈娇柔的叶丛,优雅而多少有些轻佻,线条妖艳,质薄料精,在叶丛中挂着万千白花,像是千百群振翅攒动的蜜蜂,还有这花的阴柔、闲逸而悦耳的名称,都使得我的心怦怦直跳,然而这里头却含有凡俗的因素,就像是那些华尔兹舞一样,我们记住的不是舞蹈本身,而是入舞厅时接待员高声叫出的漂亮的女宾的姓名。我听说,我将在那小径上看到一些打扮入时的美女,她们当中虽然有些还没有出嫁,然而别人不提则已,一提就总是跟斯万夫人一道提起,而且时常总是用她们的化名;她们如果换了什么新的姓名,那也仿佛是用来隐匿真实身份的假名,别人谈起她们来时是根本不用的,免得产生误会。心想在女人漂亮不漂亮的问题上,美是受一些神秘的法则所支配的,她们对此早已心领神会,也有办法来体现这美,所以我把她们的装束和车马的出现看作是一种启示,此外还有万千细节,我都寄予充分的信任,仿佛给这些转瞬即逝、游移不定的东西注入一个灵魂,使它们取得一件艺术杰作的完整一致。不过我要看的还是斯万夫人,我等着她走过来,心头激动得仿佛她就是希尔贝特似的。本来嘛,希尔贝特的父母,就跟她身边的一切一样,都浸透着她的魅力,跟她一样在我心头激起一份情感,甚至还有点令人痛苦的不安情绪(因为他们跟她的接触是她生活中内在的部分,是我所无缘介入的),而且,读者不久就会看到,我很快就明白,原来他们并不欢喜我跟她在一起玩,这就又添上了一份我们对那些能毫无限制地伤害我们的人的那种敬畏之情。
\par 有时,我看到斯万夫人穿一件普通呢子的波兰式连衣裙,头上戴一顶插着一支野鸡毛的无边小帽,胸口别一小束紫罗兰,仿佛只是为了抄近路早些回家似的,匆匆忙忙地穿过槐树路,而对坐在马车上老远认出了她的身影,向她打招呼而且心想谁也没有她那么帅的那些先生挤挤眼睛。这时,我就把简朴放在美学标准和社交条件的首位。然而有时我摆在首位的就不是简朴而是排场了,譬如说,当弗朗索瓦丝已经累得不行,直嘀咕说她迈不开腿了,而我还是逼她拖着脚步再陪我走上一个小时,终于在通往太子妃门那条小道看到——这形象在我看来就代表着王家的尊荣,是君王的驾临,是后来任何真正的王后都未能给我如此强烈印象的(因为我对她们的权力是有清楚的概念也有实际的体会的)——由两匹精壮矫健,像贡斯当丹·居伊\footnote{贡斯当丹·居伊(1805—1892),法国画家,作品中有多幅写其戎马生涯,代表作有《骑士》。}笔下那样的马拉着,御者座上坐着一位穿着哥萨克骑兵那样的皮衣的高大车夫,旁边是一个像已故博登诺尔\footnote{博登诺尔为巴尔扎克《加迪尼安亲王夫人的秘密》中的人物。}的侍从那样的青年侍者,我只见——说得更正确些,应该是我感到它的轮廓在我心头刻上了一个清晰而恼人的烙印——一辆无与伦比的维多利亚式四轮敞篷马车,车身比一般稍高,从最时新的豪华中又透出古雅的线条,车里潇洒地坐着斯万夫人,她的头发现在还是一片金黄,只有一绺灰的,束着一条狭窄的缎带,戴的经常是紫罗兰,从带上垂下长长的面纱,手上打着一把浅紫色的遮阳伞,嘴边挂着一个暧昧的微笑,我从中只看到王后那种仁慈,可也更加看到轻佻女子的撩拨,这是她轻盈优美地赐给跟她打招呼的人们的。这个微笑,对某些人是意味着:“我记得很清楚,真是太妙了!”对另一些人则是:“我何尝不想啊?咱们两个运气太坏!”对还有一些人则是:“好吧,我跟着这行列再走一段,一会儿就出来。”就是在陌生人身边走过时,她嘴边也挂着一个懒洋洋的微笑,仿佛是在等待哪个朋友或者想起哪个朋友;这丝微笑不禁令人赞叹:“她多美啊!”只对某一些人,她的微笑才是酸不溜丢、勉勉强强、畏畏缩缩、冷冷冰冰的,那意思是说:“好吗,你这个坏包,我知道你的舌头比毒蛇还毒,你那张臭嘴就是闭不住!可你以为我在乎吗?”戈克兰\footnote{戈克兰(1841—1909)为法国著名演员,以扮演费加罗·莫里哀剧中的仆人、罗斯丹《西哈诺·德·贝热拉克》中的西哈诺而知名。}跟一群听他侃侃而谈的朋友走过,以舞台上那种姿势向坐在马车上的人们挥手致意。可我一心想着斯万夫人,我装做没有瞧见她,因为我知道一到射鸽场那边,她就会叫车夫把车驶出行列,停下来好徒步走下小径。在我感到有勇气打她身边走过的日子,我就拽着弗朗索瓦丝上那个方向走去。果然过一会儿就老远看见斯万夫人在行人小径上向我们走来,她那浅紫色裙子长长的拖裾在身后拖着,那副衣装打扮在老百姓心目中是只有王后才有而又是别的妇女所不穿戴的。她有时垂下眼帘看看她阳伞的伞柄,对路过的行人毫不在意,仿佛她唯一的大事和目的就是出来活动活动,全然不想到众人都在看她,所有的脑袋都向她转将过来。可有时当她回过头来叫她那条猎兔狗时,她也不经意地向四周看上一眼。
\par 即使是那些不认识她的人也都注意到她身上有点与众不同,有点未免过分的地方,或者也许是由于一种心灵感应,就如同当拉贝玛演得最精彩时就连最无知的观众席中也会掌声雷动一样,感到她该是一个名人。他们心里纳闷:“她是谁?”有时也会问问行人,也会努力记住她的服饰,好向消息灵通的朋友打听个究竟。还有一些散步的人停下脚步,说道:
\par “您知道她是谁?是斯万夫人!您记不起来了?奥黛特·德·克雷西?”
\par “奥黛特·德·克雷西?我刚才也在嘀咕呢,那双多愁善感的眼睛……她现在可不是那么太年轻了!我记得我是在麦克马洪辞职那天\footnote{麦克马洪(1808—1898)是法兰西第三共和国的第二任总统,他本是君主派。1879年1月,当参众两院都由共和派控制时,麦克马洪被迫于1月30日辞职。}跟她睡觉的。”
\par “奉劝您别再向她提起。她现在是斯万夫人,她先生是赛马俱乐部的,是威尔士亲王的朋友。再说她还很漂亮呢。”
\par “不错,可您当年要认识她就好了,她那时那个美啊!她住在一所挺怪的小房子里,满是中国小摆设儿。我记得我们老是听到街上报童的叫喊声,后来她就催我起身了。”
\par 我也就没有再听那些往事,只感到她周围全都是关于她的卓著名声的窃窃私语。我的心焦躁地直跳,心想还得再过一会儿,所有这些人(很遗憾,他们当中还没有一个被我认为会瞧不起我的黑白混血银行家)才能看到这个他们一直未加注意的年轻人向这位以貌美、放荡、风度而遐迩闻名的女人致敬——说真的,我并不认识她,不过我认为我有资格这样做,因为我的父母认识她的丈夫而我又是她女儿的伙伴。我现在已经紧挨着斯万夫人了,我脱下帽子,伸长胳膊,久久地鞠一大躬,弄得她都忍不住微微一笑。有些人也笑了起来。至于她呢,她从来没有见我跟希尔贝特一起玩过,也不知道我姓甚名谁,在她心目中,我跟林园的看守、船夫、湖里的鸭子一样,是她在林园散步时的一个小角色,虽然见过但不知其姓名,所以也跟跑龙套的一样没有什么个性。有些日子我在槐树路上没有见着她,却在玛格丽特王后路上碰到,那里是那些希望单身独处或者希望显得是想单身独处的女人的去处;她总是单独呆不多一会儿,就有一个朋友来和她会合,他时常戴一顶灰色高顶礼帽,我不认识他,他跟她聊得很久,他们的两辆马车一直在他们身后慢慢跟着。
\par 布洛尼这个林子的这种复杂性使得它成了一个出于人手的产物,成了一个动物园或者神话中的园子:这种复杂性,我那年\footnote{那是在1913年,离“我”在这里见到希尔贝特那年(1895)已经十八个年头了。}在穿过林园到特里亚农去的时候又体会到了;那是十一月初的一个早晨,在巴黎,蛰居室内,匆匆逝去中的秋色近在身畔而你未能一顾,这就难免勾起你对落叶的眷恋之情,甚至可说是一种狂热,折腾得你难以入眠。在我那紧闭着的卧室里,一个月以来我就一直想去观赏,这落叶就经常在我的思想和我思维的对象之间出现,就跟有时当我们注视一个物体时在我们眼前跳跃的黄色斑点一样在我眼前盘旋纷飞。那天早上,耳听得不像前几天那样有雨声了,眼看晴朗的天就跟幸福的秘密从紧闭的嘴巴中泄露出来一样从关着的窗帘角边向我微笑时,我感觉到,我就可以欣赏这些枯黄的叶子,在灿烂阳光下超凡的美了;当年在孩提时听到狂风在壁炉里呼啸,可以强压自己到海滨去观赏的愿望,而现在却再也不能不去看看那些树木,我这就走出家门,穿过布洛尼林园上特里亚农去。这正是林园呈现出最丰富多彩的面貌的时刻和季节,这不仅因为这是它被分割得最厉害的时候,而且因为那是以另一种方式分割的。即使在那些可以看到一片广阔的空间的开阔地,面对着远处那些有的还保留着夏日的树叶,有的则已经秃光了的黑压压的树群,也还可以看见两行橙红色的栗树,仿佛这是在一幅刚开始落笔的画上,画家唯一上了油彩的部分,其余部分都还没有着色;这两行树把它们当中夹的那条道路伸向阳光灿烂之处,供日后添上的人物偶尔散步之用。
\par 再往远去,有个地方所有的树还都覆盖着绿叶,只有一棵小树,矮壮粗实,顶枝虽截却坚强不屈,迎风摇曳着它那一头难看的红发。还有的地方依然还是五月树叶开始苏醒时那副模样,有一棵白蔹的叶子简直是神了,像一株在冬季开花的红山楂一样满面笑容,打清早起就舒展怒放。这布洛尼林园一时看起来倒像是一个苗圃或者一个公园,为了什么植物学的原因或者是准备过什么节庆,在还没有拔除的同一种树木之间,刚栽上两三种名贵的品种,枝叶怪诞,仿佛是要在它们周围保留点间隙,疏通疏通空气,多留一些光照。就这样,这是布洛尼林园展现出种种特点,将最多的各不相同的部分组成一个复合的综合体的季节。这也是这样的一个时刻。在树木还保留着叶子的那些地方,当早晨的阳光几乎是水平地照射着的时候,这些树木仿佛又变了一种质地,而再过几个钟头,当薄暮来临,阳光像一盏灯从远处向树丛投上一个人造的温暖的反光,使树巅的叶子又发出强光,树木本身则像一支插着它那熊熊燃烧的顶部的无火的烛台时,这些树木仿佛又变了一种质地。在有的地方,阳光厚得像一层砖,跟饰有蓝色图案的波斯黄瓷砖一样,在空中胡乱涂抹在栗树叶上;在有的地方,树叶向天空伸出它们卷缩的金色的手指,阳光却插到它们与天空之间,把它们分隔开来。在一棵缠着野葡萄藤的树的半中间,阳光嫁接上并且催开了一大束红花,太耀眼,不可能辨别得太清楚,多半是康乃馨的一种变种。林园的各部分,夏季是一片苍翠,那么厚实,那么单调,现在各现本色了。从一些比较开阔的地方,几乎可以看到通向所有各部分的道路,也可以说是每一个浓密的叶丛都像一面往日王室的方形红旗一样,标志着通向各部分的道路。我仿佛在一幅彩色地图上看出哪是阿姆农维尔,哪是加特朗草地、马德里、赛马场、布洛尼湖滨。不时出现一些无用的建筑物,什么一个假的山洞啦,挪开树木腾出位置修的或者是在草地软绵绵、绿油油的平台上修的什么磨坊啦等等。可以感觉出来,林园并不仅仅是个林园,它还要适应与树木的生长毫无关系的一些用途;我心里感到的激奋也并不仅仅是由观赏秋色而产生,还出之于别的什么意念。这种愉快之源是我们的心虽然感觉得到却不知其原由,也不领悟这是任何身外之物所不能促其产生的!就这样,我以无法得到满足的温情注视着这些树木,这种温情迈过它们,在我不知不觉之中奔向这些树木每天都要荫庇几个小时的那些漂亮的散步的女子。我向槐树路走去。我穿过一些高大的乔木林,早晨的阳光将它们进行了新的区划,修剪了它们的枝条,把各式各样的树干结合在一起,组编成一个又一个的花束。阳光巧妙地把两棵树拉到一起,借助于它有力的光与影的大剪子,把每棵树的树干和树枝都剪去一半,然后把剩下的两个一半编织在一起,或者构成一根暗影的柱子,两边都是阳光,或者构成一团鬼魂似的光,它那看着别扭、颤动不定的轮廓四周镶嵌着一团黑影。当一道阳光把那些最高的树枝涂抹成金黄色时,它们就像是抹着一层闪闪发光的湿气,刺破整个乔木林浸沉于其间湿漉漉、翠绿色的大气圈,兀然耸立在空中。树木继续凭它们的生命活力活着,就在当它们光秃得没有一片叶子的时候,这生命活力依然发出更加夺目的光辉——或者是在裹着它们的树干的绿色绒鞘之上,或者是在一直长到杨树顶上、圆得跟米开朗琪罗那幅《创世记》中的太阳和月亮一样的槲寄生\footnote{槲寄生为常绿小灌木。}的白色绒球之中。可是,既然这些树木多年来可说是通过嫁接这种方式,跟那个女子有着共同的生活,它们就叫我想起了那个希腊神话中的山林仙女,想起那个行动矫健,面色红润的美丽的社交女子,当她走过的时候,它们以它们的树枝覆盖着她,使她也跟它们一样,领略这季节的法力;这些树木也叫我想起当我还年轻,还有所信仰的幸福岁月,那时我急切地来到这女性的美的杰作在这不知不觉地当了同谋者的叶丛之间一时展现出来的地方。然而,布洛尼林园的冷杉和槐树(它们比我就要到特里亚农去看的栗树和丁香还要撩乱我心),它们叫我向往的美却并不附着在我身外,并不附着在某一历史时期的回忆上,某些艺术作品之上,并不附着在门口堆放着金黄色的树叶的爱神之庙之上。我到了湖边,一直走到射鸽场。我心中的完美观,那时我觉得它体现在一辆维多利亚式敞篷马车的高度上,体现在那几匹轻盈得像胡蜂那样狂奔、双眼像狄俄墨得斯用人肉喂养的凶狠战马那样充血的骏马的精瘦上,而现在呢,我一心只想重新看到我曾经爱过的东西,这个念头跟多年前驱使我到这同样几条路上来的念头同样强烈,我真想再一次亲眼看一看斯万夫人那魁梧的车夫,在那只有他巴掌那么大、跟圣乔治一样稚气的小随从的监视下,竭尽全力驾驭那几匹振起钢翅飞奔的骏马。唉!如今只有那由留着小胡子的司机驾驶的汽车了,站在他身旁的是高如铁塔的跟班。我真想拿到眼前看看,现在女帽是否跟我记忆中那低矮得就跟一个花环那样的帽子一样迷人。现在女人戴的帽子都其大无比,顶上还装饰着果子和花,还有各式各样的小鸟。斯万夫人当年穿了俨然像王后一般的袍子也没有了,取而代之的是希腊撒克逊式的紧身衣服,带有希腊塔纳格拉陶俑那种皱褶,有时还是执政内阁时期的款式,浅底子的花绸上面跟糊墙纸那样缀着花朵。当年可能有幸跟斯万夫人在玛格丽特王后小道上散步的先生们头上,现在再也看不见有戴灰色高顶礼帽或其他式样的帽子了。他们如今是光着脑袋上街。眼前这景象中的形形色色的新玩意儿,我简直难以相信它们一个个都能站得住脚,都是一个统一的整体,甚至是否都有生命;它们支离破碎地在我眼前过去,纯属偶然,也无真实可言,它们身上也没有我的眼睛能以像往日那样去探索组合的任何美。女子都是平平常常,要说她们有什么风度,我是极难置信的,她们的衣着我也觉得没有什么了不起。当我们心中的一个信念消失时,有一个东西却还依然存在,而且越来越强烈,来掩盖我们丧失了的赋予新事物以现实性这种能力——这个东西就是对旧事物的偶像崇拜式的依恋,仿佛神奇之感不生自我们之身而存于这些旧事物之中,仿佛我们今天的怀疑有其偶然的原因,那就是众神都已死了。
\par 我心想:真是可怕!人们怎能觉得这些汽车跟当年的马车一样有气派呢?我也许岁数已经太大了,我可看不惯这么个世道,女人居然裹在都不是用衣料缝成的衣服里。当年聚集在这优雅的红叶丛底下的人现在都已烟消云散,庸俗和愚蠢取代了它们一度荫庇的精巧优美,再到这些树底下来又有什么意义?真是可怕!今天已不复有什么风度可言,我只好以思念当年认识的那些女子聊以自慰了。现在这些人出神地看着那些帽子上顶着一个鸟笼子或者一个果园的怪物,他们又怎样体会到斯万夫人头戴一顶普普通通的浅紫色带褶帽或者仅仅笔直地插上一支蝴蝶花的小帽时是何等迷人呢?在冬日的早晨,我碰上斯万夫人徒步行走,身穿水獭皮短大衣,头戴一顶普普通通的贝雷帽,只插两支山鹑毛,然而单凭她胸口那小束紫罗兰就可以想见她家里是温暖如春——那花开得如此鲜艳如此碧蓝,在这灰色的天空、凛冽的寒风、光秃的树木当中,它有着这样的魔力,就是仅仅把这季节和这天气当做一个背景,而实际却生活在人的环境之中。生活在这个女子的环境之中,跟那些在她客厅燃着的炉火旁边、丝绸沙发前面的花盆和花坛当中透过紧闭的窗户静静看着雪花纷纷落下的花儿具有同样的魔力:我那时的情感,又怎能叫那帮人理解?再说,对我来说,光让服饰恢复到当年那样子还是不够。一个回忆当中的各个部分是互相结合在一起的,而我们的记忆又保持这些部分在一个整体中的平衡,不容许我们有一丝克扣,有一毫抛弃,所以我都真想能在这些妇女当中哪一位家里度完这一天,面前一杯香茶,在漆着深色的墙壁的套间(就像是这篇故事的第一部分结束的次年斯万夫人住的那一套一样),墙上映照着橙色的火光,炉子里是一片火红,在那十一月的薄暮中闪烁着菊花玫瑰色和白色的光芒,而这时刻就跟我没有能得到我所向往的那些乐趣的那会儿相像——这点我们会在后面看到的。然而现在,这样的时刻虽然不会给我带来什么结果,我还是觉得它们本身就含有充分的魅力。我真想重新得到这样的时刻,完全跟我在回忆中的一样。唉!如今已经只有路易十六款式的房间了,四面都是点缀蓝色绣球花釉面的白墙。再说,现在人们都要很晚才从外地回到巴黎来。如果我写信给斯万夫人,请她帮我来把我感到已经属于遥远的岁月、属于已不容我追溯的年代的某些内容(这个愿望本身已无法实现,就如我当年徒然追求的那个乐趣一样无法得到)追补出来的话,她会从乡间的别墅回信,说她要到二月才能回来,那时菊花早已凋谢了。此外,我也真希望依然还是当年那些女子,那些服饰使我感到兴趣的女子,这是因为,在我还有所信仰的岁月,我的想象力曾把她们一一赋予个性,给她们每一个人都编上一篇传奇。唉!在槐树路,也就是《埃涅阿斯纪》中的爱神木路,我倒见到了几位,老了,都只是她们当年风韵的可怕的影子了,她们在维吉尔的树丛中徘徊踯躅,绝望地不知在搜寻些什么。她们都早就离开了,我可还在向那空无一人的小道打听。太阳隐藏起来了。大自然又开始统摄这个林园,把它说成是妇女乐园这种想法早已烟消云散;人工堆砌的磨坊上是一片十足的灰蒙蒙的天空;风吹皱了大湖,吹起了层层涟漪,倒像是一个真正的湖泊;大鸟迅捷地飞越林园,倒像是飞越一个真正的树林,一面发出尖叫,一面纷纷栖息在高大的橡树之巅;这橡树的树冠真像高卢时期德鲁伊特教祭司的花冠,而又以古希腊多多纳祭司的权威,仿佛在宣告这已经另作他用的森林已经荒无人烟,这倒有助于我明白在现实之中去寻找记忆中的图景是何等的矛盾,后者的魅力得之于回忆,得之于没有通过感官的感受。我当年认识的现实今日已经不复存在。只要斯万夫人不在同一时刻完全保持原有的模样到来,整条林荫大道就会是另一副模样。我们曾经认识的地方现在只处于这样一个小小的空间世界,我们只是为了方便起见,才给它们标出一个位置。它们只是构成我们当年生活的相邻的诸印象中间的一个小薄片;对某个形象的回忆只不过是对某一片刻的遗憾之情;而房屋、道路、大街,唉!都跟岁月一样易逝!


\subsection{第二卷\ 在少女们身旁}


\subsubsection*{第一部\ 斯万夫人周围}

\par 在商量请德·诺布瓦先生第一次来家吃饭时,母亲说,遗憾的是戈达尔教授目前在外旅行,她本人又完全断绝了与斯万的交往,否则这两位陪客会使那位卸任的大使感兴趣的。父亲回答说,像戈达尔这样的显赫上宾、著名学者,会使餐桌大增光彩。可是那位爱好卖弄、唯恐旁人不知自己结交了达官贵人的斯万,其实只是装模作样的庸俗之辈,德·诺布瓦侯爵会用“令人恶心”这个词来形容斯万的。对父亲的这个回答我得稍加解释。某些人可能还记得,戈达尔从前十分平庸,而斯万在社交方面既谦和又有分寸,含蓄得体。但是我父母的旧友斯万除了“小斯万”、赛马俱乐部的斯万之外,又增添了一个新头衔(而且不会是最后的头衔),即奥黛特的丈夫。他使自己素有的本能、欲望、机智服从于那个女人的卑俗野心,尽力建立一个适合于他伴侣的、由他们两人共有的新的地位,这个新地位大大低于他从前的地位。因此,他的表现判若两人。既然他开始的是第二种生活(虽然他仍然和自己的朋友单独来往。只要他们不主动要求结识奥黛特,他不愿意将她强加于他们),一种和他妻子所共有的、在新交的人之间的生活,那么,为了衡量这些新友人的地位,也就是衡量他们的来访给自己的自尊心所带来的愉快,他所使用的比较尺度不是自己婚前的社交圈子中最杰出的人物,而是奥黛特从前的朋友,这一点也就不难理解了。然而,即使人们知道他乐于和粗俗的官员以及政府部门舞会上的花瓶——名声不好的女人来往,但他居然津津乐道地炫耀某办公室副主任的妻子曾登门拜访斯万夫人,这未免使人愕然,因为他从前(至今仍然)对特威肯汉城\footnote{此城是法国奥尔良王族流亡英国的居住处。}或白金汉宫的邀请都曾潇洒地保持过缄默。人们也许认为昔日风流倜傥的斯万的纯朴其实只是虚荣心的一种文雅的形式,他们也许认为我父母的这位旧友和某些犹太人一样,轮流表现出他的种族所连续经历的状态,从最不加掩饰的附庸风雅、最赤裸裸的粗野,直到最文雅的彬彬有礼。然而,主要原因——而且这普遍适用于人类——在于这一点,即我们的美德本身并不是时时听任我们支配的某种自由浮动的东西,在我们的思想中,美德与我们认为应该实践美德的那些行动紧密相连,因此,当出现另一种类型的活动时,我们束手无策,根本想不到在这个活动中也可以实践同样的美德。斯万对新交无比殷勤,眉飞色舞地一一举出他们的姓名,这种态度好似那些谦虚或慷慨的大艺术家:他们在晚年也许尝试烹饪或园艺,为自己的拿手好菜或花坛沾沾自喜,只能听夸奖,不能听批评。但一旦涉及他们的杰作,他们是乐于倾听批评的;或者说,他们可以慷慨大方地赠送一幅名画,可是在多米诺牌桌上输了四十苏却满不高兴。
\par 谈到戈达尔教授,我们将在很久以后,在拉斯普利埃宫堡维尔迪兰夫人府上再次和他长久相聚。此刻,关于他,只需首先提请注意一点。斯万的变化严格说来无法使我惊讶,因为当我在香榭丽舍大街看见希尔贝特的父亲时,这变化已经完成,只是尚未被我看透罢了。再说他当时没有和我讲话,不可能向我吹嘘他那些政界朋友(即使他这样做,我多半也不能立即觉察到他的虚荣心,因为长时期形成的对某人的看法使我们视而不见,听而不闻。母亲也是一样,在三年里,她竟然没有觉察到侄女嘴上的唇膏,仿佛它溶解在流体之中无影无踪了。直到有一天,过浓的唇膏或者其他什么原因引起了所谓超饱和现象,于是从前没有看见的唇膏结成晶体,母亲突然看见了缤纷的彩色,大叫可耻,如同在贡布雷一样,并且几乎断绝了与侄女的一切来往)。戈达尔的情况却相反,他在维尔迪兰家目睹斯万跨进社交界的那个时期已经相当遥远,而岁月的流逝给他带来了荣誉和头衔。其次,一个人尽可以缺乏文化修养,尽可以做愚蠢的同音异词的文字游戏,但同时仍可以具有一种任何文化修养所无法取代的特殊天赋,例如大战略家或杰出医生的天赋。在同行们眼中,戈达尔不仅仅是靠资历而由无名小卒终于变为驰名欧洲的名医。年轻医生中之佼佼者宣布——至少在几年内,因为标准既然应变化之需要而诞生,它本身也在变化中——万一他们染病,戈达尔教授便是他们唯一能以命相托的人。当然他们愿意和某些文化修养更深、艺术气质更重的主任医生交往,和他们谈论尼采和瓦格纳。戈达尔夫人接待丈夫的同事和学生,盼望有朝一日丈夫能当上医学院院长。人们在晚会上欣赏音乐,戈达尔先生却无意聆听,而去隔壁的客厅里玩牌。然而他的好眼力、他诊断之敏捷、深刻、准确,令人赞叹不已。第三点,关于戈达尔教授对我父亲这种类型的人所采用的声调和态度,应该指出,我们在生活的第二部分所显示出的本质可能是第一本质的发展或衰败、扩大或减弱,但并不永远如此,它有时是相反的本质,是不折不扣的反面。戈达尔青年时代的那种迟疑的神情、过分的腼腆与和蔼曾使他经常受人挖苦,当然迷恋他的维尔迪兰家除外。是哪位慈悲为怀的朋友劝他摆出冷冰冰的面孔呢?由于他的重要地位,这样做是轻而易举的。在维尔迪兰家,他本能地恢复原貌,除此以外,在任何地方,他表现得冷若冰霜,往往是一言不发。而当他不得不说话时,他又往往采取断然的口吻,故意令人不快。他将这种新态度试用于求医者身上,既然求医者以前从未与他谋面,自然无法作比较。他们如果得知戈达尔并非生性粗鲁,准会大吃一惊。戈达尔极力使自己毫无表情。他在医院值班时,讲述同音异义的玩笑引起众人——从主任医生到新来的见习医生——捧腹大笑,而他的面部肌肉却纹丝不动。由于他剃去了胡须,他的面孔也完全变了样。
\par 最后说说德·诺布瓦侯爵为何许人,战前\footnote{指1870年普法战争前,法兰西第二帝国时期。}他曾任全权公使。五月十六日危机期间\footnote{指1877年5月16日法国内阁危机。}他任大使。尽管如此,使许多人大为吃惊的是,他后来曾多次代表法兰西出使国外执行重要使命,甚至赴埃及出任债务监督,并施展他非凡的财务能力,屡有建树,而这些使命都是由激进派内阁委任于他的。一般的反动资产者都拒绝为这个内阁效劳,更何况德·诺布瓦先生:他的经历、社会关系和观点都足以使他被内阁视为嫌疑分子。然而,激进派的部长们似乎意识到此种任命可以表明他们襟怀坦荡,以法兰西的最高利益为重,说明他们不同于一般政客,而当之无愧地被《辩论报》称为国家要人。最后,他们可以从贵族姓氏所具有的威望及剧情突变式的出人意料的任命所引起的关注中得到好处。他们明白,起用德·诺布瓦先生对他们有百利而无一害,他们不用担心后者会违背政治忠诚,因为,侯爵的出身不仅不引起他们的戒备防范,反而使他们放心。在这一点上,共和国政府没有看错。这首先是因为某一类贵族从童年时起就认为贵族姓氏是一种永远不会丧失的内在优势(他的同辈人,或者出身更为高贵的人对这种优势的价值十分清楚),他们知道自己大可不必像众多资产者那样费尽心机地(虽然并无显著效果)发表高见,攀交正人君子,因为这种努力不会给他们增添任何光彩。相反,他们一心想在身份比自己高的王侯或公爵面前抬高自己的身价,而要达到这一点,就必须往姓氏中添加原来所没有的东西:政治影响、文学或艺术声誉、万贯家产。他们无意在资产者所追求的、无用的乡绅身上浪费精力,何况得到一位乡绅的无实效的友谊并不会导致王侯的感激。他们将大量精力使用于能有助于他们担任使馆要职或参加竞选的政治家身上(即使是共济会会员也不在乎),使用于可以在自己的业务范围内帮助他们进行“突破”的、声誉显赫的艺术家或学者身上,简而言之,使用于一切促使他们扬名,促使他们与富人结成姻亲的人身上。
\par 德·诺布瓦先生从长期的外交实践中吸收了那种消极的、墨守成规的、保守的精神,即所谓“政府精神”,这是一切政府所共有,特别是政府之下各使馆所共有的精神。外交官的职业使他对反对派的手段——那些多少带有革命性的、至少是不恰当的手段——产生憎恶、恐惧和鄙视。只有平民百姓和社交界中少数无知者才认为所谓不同的类型纯系空谈,但就大多数情况而言,不同类型的相互接近不是出于相同的观点,而是出于同血缘的精神。像勒古费这种类型的院士是古典派,但他却为马克西姆·杜冈或梅西埃对维克多·雨果的颂词\footnote{即对浪漫主义的颂词。马克西姆·杜冈(1822—1894),法国作家;梅西埃(1829—1915),文学批评家。}鼓掌,却不愿为克洛代尔对布瓦洛的颂词\footnote{即对古典主义的颂词。克洛代尔(1868—1955),法国作家;布瓦洛(1636—1711),法国诗人。}鼓掌。同一个民族主义使巴雷斯\footnote{巴雷斯(1862—1923),法国作家,宣传民族主义。}与他的选民接近——后者对他和乔治·贝里先生\footnote{乔治·贝里,先为保皇派、右翼议员,后接受进步思想。}并不细加区别——却无法使巴雷斯和法兰西学院的同事们接近,因为后者虽然与他政见一致但精神迥异;他们甚至不喜欢他而偏爱政敌里博先生和德沙涅尔\footnote{里博(1842—1923),法国政治家,多次连任法国财政和外交部长。德沙涅尔,法国政治家,主张共和制,曾在1920年担任过几个月共和国总统。}先生;忠诚的保皇派感到与里博和德沙涅尔十分接近,而与莫拉斯及莱翁·都德相当疏远,尽管这两人也希望王朝复辞。德·诺布瓦先生寡言少语,不仅出于谨慎稳重的职业习惯,还由于言语在此类人眼中具有更高的价值、更丰富的含义,因为他们为使两个国家相互接近而作的长达十年的努力,在演讲和议定书中,也不过归纳为、表现为一个简单的形容词,它貌似平庸,但对他们却意味着整整一个世界。这位在委员会中以冷若冰霜著称的德·诺布瓦先生在开会时坐在我父亲旁边,因此人们纷纷祝贺父亲居然获得这位前大使的好感。父亲本人也感到惊奇,因为他脾气不太随和,除了一小圈知己以外,很少有人和他来往,他本人也确认不讳,他意识到外交家的殷勤是出于一种由本人决定好恶的完全独立的观点;当某人使我们厌烦或不快时,他的全部精神品质或敏感性就丧失作用,它们还不如另一人的爽直轻松能赢得我们的好感,虽然后者在许多人眼中显得空洞、浮浅、毫无价值。“德·诺布瓦又请我吃饭,真是件大事。”委员会里大家都很吃惊,因为他和委员会里的任何人都没有来往。“我敢肯定他又会和我讲关于一八七〇年战争的扣人心弦的事。”父亲知道德·诺布瓦先生也许是唯一一位提请皇帝注意普鲁士的军备扩张和战争意图的人;他知道俾斯麦对德·诺布瓦的智慧表示佩服。就在最近,在歌剧院为狄奥多西皇帝举行的盛大晚会上,报界注意到皇帝曾长时间接见德·诺布瓦先生。“我得打听皇帝的这次访问是否确实重要,”对外交政策颇感兴趣的父亲对我们说,“我知道诺布瓦老头守口如瓶,但他对我可无话不谈。”
\par 在母亲眼中,大使本人也许缺少最能使她感兴趣的那种智慧。应该说德·诺布瓦先生的谈话是某种职业、某个阶层、某个时期——对于这个职业和阶层来说,这个时期可能并未完全废除——所特有的古老的语言形式之大全,我未能将耳闻如实笔录下来,不免感到遗憾,否则我不费吹灰之力便能创造语言老朽这个效果,正如罗亚尔宫那位演员一样:有人问他从哪里找到那些令人惊奇的帽子,他回答说:“不是找来的。是保存下来的。”总而言之,我感到母亲认为德·诺布瓦先生有点“过时”。就举止而言,他并未使她不快,但就思想而言——其实德·诺布瓦先生的思想是十分时新的——或许远不如说就语言表述而言,他在她心目中毫无魅力。不过她感觉到,如果她在丈夫面前对那位对他表示如此少有的偏爱的外交家称赞一番,丈夫定会暗暗得意。她肯定了父亲对德·诺布瓦先生的好评,同时也引导他对自己产生好评,她意识到这是在履行职责:使丈夫愉快,就好比使菜肴精美、使上菜的仆人保持安静一样。她不善于对父亲撒谎,因此就培养自己去欣赏大使,以便诚心诚意地称赞他。何况,她当然欣赏他那和善的神情、稍嫌陈旧的礼节(而且过分拘谨,他走路时,高大的身躯挺得笔直,但一见我母亲乘车驶过,便将刚刚点着的雪茄抛得远远的,摘下帽子向她致意),他那有分寸的谈吐——他尽可能不谈自己,而且时时寻找能使对方高兴的话题——以及其速度令人吃惊的回信。父亲刚寄出一封信就收到回信,父亲看见信封上德·诺布瓦先生的笔迹,第一个反应是莫非这两封信恰巧错过了。难道邮局对他特别优待,加班为他收发信吗?母亲赞叹他虽百事缠身,却复信迅速;虽交游甚广,但仍和蔼可亲。她没有想到这些“虽然”其实正是“因为”,只是她未识别罢了,她没有想到(如同人们对老者的高龄、国王的不拘礼节、外省人的灵通信息感到吃惊一样)德·诺布瓦先生正是出于同一种习惯而既日理万机又复信迅速,既取悦于社交界又对我们和蔼可亲。再者,和所有过分谦虚的人一样,母亲的错误在于将与自己有关的事置于他人之下,即置于他人之外。她认为父亲这位朋友能即刻复信实属难能可贵,其实他每日写大量书信,这只是其中的一封,而她却将它视作大量信件中之例外。同样,她看不出德·诺布瓦先生来我家吃饭仅仅是他众多社交活动中之一项,因为她没想到大使昔日在外交活动中习惯于将应邀吃饭当做职责,习惯于表现出惯常的殷勤,如果要求他在我家一反常态地舍弃这种殷勤,那就未免太过分了。
\par 德·诺布瓦先生第一次来家吃饭的那一年,我还常去香榭丽舍大街玩耍。这顿饭一直留在我的记忆中,因为那天下午我总算能看拉贝玛\footnote{拉贝玛与后文提到的贝玛大妈是同一个人。在某些人名字前加上“拉”,是民间一种习俗用法。}主演的《淮德拉》\footnote{《淮德拉》,十七世纪古典主义剧作家拉辛的悲剧。}日场,还因为与德·诺布瓦先生的谈话使我骤然以新的方式感到:希尔贝特·斯万及她父母的一切在我心中所唤醒的感情与他们在其他任何人心中所引起的感情是多么地不同。
\par 新年假期即将到来,我也日益无精打采,因为希尔贝特亲口告诉我在假期中我再也见不到她,母亲大概注意到我的神气,想让我解解闷,有一天便对我说:“如果你仍然很想听拉贝玛的戏,我想父亲会同意的,外祖母可以带你去。”
\par 这是因为德·诺布瓦先生曾对父亲说应该让我去听拉贝玛的戏,对年轻人来说这是珍贵的回忆,父亲才改变一贯的态度——他反对我在他所谓的无聊小事(这种看法使外祖母震惊)上浪费时间并冒生病卧床的危险,并且几乎认为既然大使劝我看戏,那么看戏似乎成了飞黄腾达的秘诀之一。外祖母一直认为我能从拉贝玛的戏中学到许多东西,但是,为了我她放弃看戏,为了我的健康她作出巨大牺牲。此刻,她无比惊异,因为德·诺布瓦先生的一句话便使我的健康成为微不足道的东西了。她对我所遵守的呼吸新鲜空气和早睡的生活习惯寄托于理性主义者的坚定希望,因此认为打破习惯便会招来灾祸,她痛心地对父亲说:“您太轻率了!”父亲生气地回答说:“怎么,您现在又不愿意让他听戏!多么荒唐,您不是口口声声说听戏对他有好处吗?”
\par 德·诺布瓦先生在对我至关重要的另一件事上,改变了父亲的意图。父亲一直希望我当外交官,而我却难以接受。即使我在外交部内待一段时间,但总有一天我会被派往某些国家当大使,而希尔贝特并不住在那里。我愿意恢复从前在盖尔芒特家那边散步时所设想的、后来又放弃的文学打算。但父亲一直反对我从事文学,认为它比外交低贱得多。他甚至不能称它为事业。可是有一天,对新阶层的外交官看不上眼的德·诺布瓦先生竟对父亲说,当作家和当大使一样,受到同样的尊敬,施展同样的影响,而且具有更大的独立性。
\par “嗳!真没想到,诺布瓦老爹毫不反对你从事文学。”父亲对我说。父亲是相当有影响的人物,因此认为什么事情都可以通过和重要人物的谈话得到解决,得到圆满的解决,他说:“过几天,开完会后我带他来吃饭。你可以和他谈谈,露一手。好好写点东西给他看。他和《两个世界评论》的社长过从甚密,他会让你进去,他会安排的,这是个精明的老头,确实,他似乎认为外交界,在今天……”
\par 不会和希尔贝特分离,这种幸福使我产生了写篇好文章给德·诺布瓦先生看的愿望——而不是能力。我动手写了几页便感到厌烦,笔从我手中落下,我恼怒得哭了起来。我想到自己永远是庸才,想到自己毫无天赋,连即将来访的德·诺布瓦先生向我提供的永不离开巴黎的良机都没有能力利用。当我想到能去听拉贝玛的戏时,胸中的忧愁才有所排解。我喜爱的景色是海滨风暴,因为它最猛烈,与此相仿,我最喜欢这位名演员扮演的是传统角色,因为斯万曾对我说她扮演这些角色的艺术堪称炉火纯青。当我们希望接受某种自然印象或艺术印象从而获得宝贵的发现时,我们当然不愿让心灵接受可能使我们对美的准确价值产生谬误的、较为低劣的印象。拉贝玛演出《安德罗玛克》、《反复无常的玛丽安娜》、《淮德拉》,这是我的想象力渴望已久的精彩场面。如果我能听见拉贝玛吟诵这段诗句:听说您即将离我们远去,大人……\footnote{《淮德拉》第五幕第一场的台词。}等等,那我会心醉神迷;就仿佛在威尼斯乘小船去弗拉里教堂欣赏提香\footnote{提香(1477—1576),意大利画家。}圣母像或者观看卡帕契奥\footnote{卡帕契奥(1455—1525),意大利画家。}的系列画《斯基亚沃尼的圣乔治》一样。这些诗句,我已经在白纸黑字的简单复制品中读过,但我将看见它们在金嗓子所带来的空气和阳光中出现,好比是实现了旅行的梦想,我想到这里时,心便剧烈地跳动。威尼斯的卡帕契奥,《淮德拉》中的拉贝玛,这是绘画艺术和戏剧艺术中的杰作,它们所具有的魅力使它们在我身上富有生命力,使我感到卡帕契奥和威尼斯、拉贝玛和《淮德拉》是融为一体的。因此,如果我在卢浮宫的画廊里观看卡帕契奥的画,或者在某出我从未听说的戏中听拉贝玛朗诵,我便不会再产生美妙的惊叹,不会再感到终于看见使我梦绕魂萦的、不可思议的、无与伦比的杰作,其次,既然我期待从拉贝玛的表演中得到高贵和痛苦的某些方面的启示,如果女演员用她卓越和真实的艺术来表演一部有价值的作品,而不是在平庸粗俗的情节上添点儿真和美,那么,这种表演会更加卓越和真实。
\par 总之,如果拉贝玛表演的是一出新戏,我便难以对她的演技和朗诵作出判断,因为我无法将我事先不知道的台词与她的语调手势所加之于上的东西区别开,我会觉得它们和台词本是一体。相反,我能倒背如流的老剧本仿佛是特有的、准备好的广大空间,我能完全自由地判断拉贝玛如何将它当做壁画而发挥她那富有新意的创造力。可惜几年前她离开了大舞台,成为一个通俗剧团的名角,为它立下汗马功劳。她不再表演古典戏剧。我常常翻阅广告,但看到的总是某某时髦作家专门为她炮制的新戏。有一天,我在戏栏里寻找元旦那一周的日场演出预告,第一次看到——在压轴节目中,因为开场小戏毫无意义,它的名字显得晦暗,其中包含对我陌生的一切特殊情节——拉贝玛夫人演出《淮德拉》中的两幕,还有第二天第三天的《半上流社会》和《反复无常的玛丽安娜》。这些名字像《淮德拉》名字一样,在我眼前显得晶莹可鉴、光亮照人(因为我很熟悉它们),闪烁着艺术的微笑。它们似乎为拉贝玛夫人增添了光彩,因为在看完报上的节目预告以后,我又读到一则消息,说拉贝玛夫人决定亲自再次向公众表演往日创造的角色。看来艺术家知道某些角色的意义不仅限于初次上演、使观众耳目一新,或再次上演而大获成功。她将所扮演的角色视作博物馆的珍品——向曾经欣赏珍品的老一代或未曾目睹珍品的新一代再次展示的珍品,这的确是十分有益的。在仅仅用来消磨夜晚时光的那些演出的预告中,她塞进了《淮德拉》这个名字,它并不比别的名字长,也未采用不同的字体,但她心照不宣地将它塞了进去,仿佛女主人在请客人入席时,将他们——普通客人——的名字一一告诉你,然后用同样的声调介绍贵宾:阿纳托尔·法朗士先生。
\par 给我看病的医生,即禁止我作任何旅行的那位,劝父母不要让我去看戏,说我回来以后会生病的,而且可能病得很久,总之,我的痛苦将大于乐趣。如果我期待于剧院的仅仅是乐趣,那么,这种顾虑会使我望而却步,因为痛苦将会淹没乐趣。然而——正如我梦寐以求的巴尔贝克之行、威尼斯之行一样——我所期待于这场演出的,不是乐趣,而是其他,是比我生活的世界更为真实的世界的真理。这些真理,一旦被我获得,便再也不会被我那闲散生活中无足轻重的小事所夺去,即使这些小事使我的肉体承受痛苦。我在剧场中所感到的乐趣可能仅仅是感知真理的必要形式,但我不愿它受到影响和破坏,我盼望自己在演出结束以后才像预料中的那样感到身体不适。我恳求父母让我去看《淮德拉》,但是自从见过医生以后,他们便执意不允。我时时为自己背诵诗句:听说您即将离我们远去……我的声调尽量抑扬顿挫,以便更好地欣赏拉贝玛朗诵中的不平凡之处。她的表演所将揭示的神圣的美如同圣殿中之圣殿一样隐藏在帷幔之后,我看不见它,但我时时想象它的新面貌。我想到希尔贝特找到那本小册子中的贝戈特的话:“高贵的仪表,基督徒的朴素,冉森派的严峻,特雷泽公主及克莱芙公主\footnote{指古典悲剧女主人公淮德拉及小说人物克莱芙公主,这是两种不同的典型。},迈锡尼的戏剧\footnote{希腊初期文化。},戴尔菲的象征\footnote{戴尔菲是古希腊城,有太阳神阿波罗的圣殿。},太阳的神话。”这种神圣的美不分昼夜地高踞在我内心深处的、永远烛火通明的祭坛之上,而我那严厉而轻率的父母将决定我能否将这位女神(她将在原来隐藏着她无形形象的地方显露真面目)的美吸进,永远吸进我的精神之中。我的目光凝视着那难以想象的形象,我整日与家庭的障碍搏斗,但是当障碍被扫平,当母亲——尽管这个日场戏正好是委员会开会,而会后父亲将带德·诺布瓦先生来家吃饭的那一天——对我说:“唉,我们不愿意使你不高兴,如果你实在想去那就去吧。”当一直作为禁忌的戏院此刻只由我来决定取舍,我将不费吹灰之力便能实现夙愿时,我却反而犹豫不决,是该去还是不该去,是否除了父母的反对以外尚有其他否定的理由。首先,虽然他们最初的残酷让我讨厌,但此刻的允诺却使我觉得他们十分亲切。因此,一想到会使他们难过,我自己就感到难过,在这种情绪之下,生活的目的对我来说似乎不再是真理,而是柔情,生活的好与坏的标准似乎只是由我父母快活还是不快活而定。“如果这会使您不快活的话,我就不去了。”我对母亲这样说。她却反过来叫我不必有这种顾虑,这种顾虑会破坏我从《淮德拉》中得到的乐趣,而她和父亲正是考虑到我的乐趣才解除禁令的。这样一来,乐趣似乎成为某种十分沉重的义务。其次,如果看戏归来病倒的话,我能很快痊愈吗?因为假期一结束,希尔贝特一回到香榭丽舍大街,我便要去看她。为了决定看不看戏,我将这全部理由与我对拉贝玛完美艺术的想象(虽然它在面纱下难以看见)作比较,在天平的一端我放上“感到妈妈忧愁,可能去不了香榭丽舍大街”,在另一端放上“冉森派的严峻,太阳的神话”,但是这些词句本身最后在我思想中变得晦暗,失去了意义,失去了分量。渐渐地,我的犹豫变得十分痛苦,我完全可能仅仅为了结束这种犹豫,一劳永逸地摆脱这种犹豫而决定去看戏。我完全可能任人领到剧院,但不是为了得到精神启示和完美艺术的享受,而是为了缩短痛苦;不是为了谒见智慧女神,而是谒见在女神面纱之下偷梁换柱的、既无面孔又无姓名的无情的神明。幸亏突然之间一切都起了变化。我去看拉贝玛表演的夙愿受到了新的激励,以至我急切和兴奋地等待这个日场,原因是那天当我像每日一样来到戏剧海报圆柱前时(我像柱头隐士那样伫立在那里,这种时刻近来变得更严峻),我看到了第一次刚刚贴上去的、仍然潮湿的、详尽的《淮德拉》演出海报(其实其他演员并不具有足以使我作出决定的魅力)。这张海报使我原先犹豫不决的那件事具有了更为具体的形式,它近在眼前,几乎正在进行之中——因为海报上落款的日期不是我看到它的那一天,而是演出的那一天,而落款的钟点正是开幕的时刻。我在圆柱前高兴得跳了起来。我想,到了那一天,在这个准确的钟点,我将坐在我的座位上,等着拉贝玛出台。我担心父母来不及为外祖母和我订两个好座位,便一口气跑回家,如痴如呆地望着那句富有魅力的话:“正厅不接待戴帽的女士。两点钟后谢绝入场。”这句话取代了我脑中的“冉森派的严峻”和“太阳的神话”。
\par 可惜,这头一场戏使我大失所望。父亲提议在去委员会时顺便将外祖母和我带到剧场。出门时他对母亲说:“想法弄一顿丰盛的晚餐吧,你大概还记得我要带德·诺布瓦来吧。”母亲当然没有忘记。从前一天起,弗朗索瓦丝就沉浸在创造热情之中。她很高兴在烹调艺术上露一手,这方面她的确极有天赋。她听说来客是一位新客,更为兴奋,决定按她的秘方烹制冻汁牛肉。她对构成她作品的原料的内在质量极为关切,亲自去中央菜市场选购最上等的臀部肉、小腿肉和小牛腿,就好像米开朗琪罗当年为修建朱尔二世的陵墓而用八个月时间去卡拉雷山区挑选最上等的大理石。弗朗索瓦丝兴冲冲地出出进进,她那绯红的面孔不禁使母亲担心这位老女仆会累垮,就像美第奇陵墓的雕刻师\footnote{指米开朗琪罗。}当年累倒在皮特拉桑塔石矿里一样。而且从前一天起,她便吩咐人将那粉红色大理石一般的、她所称作“内约”的火腿,裹上面包屑送到面包房去烤。她第一次听人谈到“约克”火腿时,便以为自己听错了,以为别人说的是她知道的那个名字——她低估了语言的丰富性,也不相信自己的耳朵,怎么可能同时存在“约克”和“纽约”呢?真令人难以相信。此后,每当她听见或在广告上看见“约克”这个名字时,她便认为是“纽约”,并将“纽”读作“内”。因此她一本正经地对打下手的厨娘说:“你去奥莉达店买点火腿。太太一再嘱咐要‘内约’火腿。”
\par 如果说这一天使弗朗索瓦丝体验到伟大创造者的炽热信心,那么,我感受到的却是探索者的难以忍受的焦虑。当然,在听拉贝玛朗诵以前,我是愉快的。在戏院门前的小广场上,我感到愉快,两小时以后,路灯将照亮广场上栗树的细枝,光秃的栗树将发出金属般的反光。在检票员(他们的挑选、提升、命运全部取决于那位著名女演员,只有她掌握整个机构的管理权,而默默无闻地相继担任领导的经理只是有名无实的匆匆过客而已)面前,我感到愉快;他们索取我们的票,却不看我们,他们焦急不安:拉贝玛夫人的命令是否全部通知了新职工,他们是否明白决不能雇人为她鼓掌,是否明白在她上台以前不要关窗,而要在她上台以后关上所有的门,是否知道应在她身旁不引人注意的地方放上一罐热水以便控制舞台尘土。再过一会儿,她那辆由两匹长鬃马驾辕的马车将来到剧院门口,她将身着皮大衣由车上下来,不耐烦地回答别人的招呼,并且派一位随从去前台看看是否为她的朋友们保留了座位,并且打听场内的温度、包厢的客人、女引座员的服饰。在她眼中,剧场和观众仅仅是她将穿在外面的第二件衣服,是她的天才将通过的或优或劣的导体媒介。在剧场里,我也感到愉快。自从我得知大家共一个舞台时,与我幼稚的想象力长期所遐想的相反,我便以为,既然周围是人群,那么别的观众一定会妨碍你看得真切,然而,正相反,由于某种仿佛象征一切感知的布局,每个观众都感到自己处于剧场中心,这使我想起弗朗索瓦丝的话。有一次,我父母让她去看一出情节剧,座位在五楼,但她回来时说她的座位再好也没有了,她丝毫不感到太远,相反却感到胆怯,因为生动而神秘的帷幕近在咫尺。我开始听见从帷幕后面传来模糊的声音,音量越来越大,就像雏鸡在破壳而出以前发出的声响。此刻我更为愉快,因为虽然我们的目光无法穿透帷幕,但帷幕后面的世界正在注视我们。突然,来自帷幕后的声音显然向我们发出信号,它变成无比威严的三下响声,像火星上的信号一样动人心弦。幕布拉开,舞台上出现了十分普通的写字桌和壁炉,它们表明即将上场的不是我在一次夜场中所看见的朗诵演员,而是在这个家中生活的普通人;我闯入他们的生活中去,而他们看不见我。这时,我的乐趣有增无减,但它却被短暂的不安所打断,因为正当我屏息静气地等待开演时,两个男人走上了舞台,他们气势汹汹、大声吵嚷,剧院里的一千多观众听得十分清楚(而在小咖啡店里,要知道两个斗殴的人在说什么,必须问侍者)。这时,我惊奇地看到观众并不抗议,而是洗耳恭听,而且沉浸在一片寂静之中,偶尔从这里或那里响起笑声,于是我明白这两个蛮横无礼的人正是演员,明白那个称作开场戏的小戏已经开始了。接下来是长长的幕间休息,观众重新就座以后,不耐烦地跺起脚来。这使我很担心。每当我在诉讼案的报道中读到某位心地高尚者将一己的利益置之度外而为无辜者出庭辩护时,我总感到担心,唯恐人们对他不够和气,不够感激,不给他丰厚的酬劳,以至他伤心气馁而转到非正义一边。在这一点上,我将天才与德行相比,因此也同样担心拉贝玛会对缺乏教养的观众的无礼感到气恼,我真盼望她在观众席上能满意地认出几位其判断颇有分量的名流,因而不卖劲,以表示对他们的不满和蔑视。我用哀求的目光看着这些跺脚的野人,他们的愤怒会将我来此寻求的那个脆弱而宝贵的印象打得粉碎。最后,《淮德拉》的前几场戏给我带来愉快的时光。第二幕开始时,淮德拉这个人物还不出场。然而,第一道幕,接着第二道红丝绒幕——它在这位明星的表演中加强舞台深度——拉开,一位女演员从台底上场,容貌和声音酷似人们向我描绘的拉贝玛。这么说,拉贝玛换了角色,我对忒修斯的妻子\footnote{即淮德拉,下文中的希波吕托斯、奥侬娜、阿里西皆为《淮德拉》中的人物。}的精细研究算是白费工夫了。然而又一位女演员上场与第一位对话,我把第一位当做拉贝玛显然是弄错了,因为第二位更像她,而且朗诵的声调惟妙惟肖。这两位都往角色中增加了高贵的手势——她们撩起美丽的无袖长衣,使我明显地注意到这一点,并明白了手势和台词的关系——和巧妙的声调。它时而热情、时而讽刺,我明白了曾在家中读过但未加留心的诗句究竟何所指。但是,突然,在圣殿的红丝绒幕布的开启处(仿佛是镜框),出现了一个女人。于是我感到害怕,而这种害怕可能比拉贝玛本人还害怕。我害怕有人开窗从而使她感到不适;害怕有人搓揉节目单从而破坏她的某句台词;害怕人们为她的同伴鼓掌而对她的掌声不够热烈从而使她不高兴。我产生了比拉贝玛本人的想法更加绝对的念头,认为从此刻起,剧场、观众、演员、戏,以及我本人的身体都只是声音介质,只有当它们有利于抑扬顿挫的声音时才具有价值。这时我立刻明白我刚才欣赏片刻的那两位女演员与我专程前来聆听的这个女人毫无共同之处。然而我的乐趣也戛然中止。我的眼睛、耳朵、思想全部集中于拉贝玛身上,唯恐漏过任何一点值得我赞叹的理由,但一无所获。我甚至未在她的朗诵和表演中发现她的同伴们所使用的巧妙的声调和美丽的姿势。我听着她,就仿佛在阅读《淮德拉》,或者仿佛淮德拉正在对我讲话,而拉贝玛的才能似乎并未给话语增加任何东西。我多么想让艺术家的每个声音、每个面部表情凝住不动,长时间地凝住,好让我深入进去,努力发现它们所包含的美。我至少做到思想敏捷,在每个诗句以前准备好和调整好我的注意力,以免在她念每个字或作每个手势期间我将时间浪费在准备工作上。我想依靠这种全神贯注的努力,进入台词和手势的深处,仿佛我拥有长长的几个小时一样。然而时间毕竟十分短暂!一个声音刚刚传进我耳中便立刻被另一个声音所替代。在一个场面中,拉贝玛静止片刻,手臂举到脸部的高处,全身浸沉在暗绿色的照明光线之中,背景是大海,这时全场掌声雷动,然而刹那间女演员已变换了位置,我想仔细欣赏的那个画面已不复存在。我对外祖母说我看不清,她便将望远镜递给我。然而,当你确信事物的真实性时,用人为的手段去观察它并不能使你感到离它更近。我认为我在放大镜中所看到的不再是拉贝玛,而是她的图像。我放下望远镜,但我的眼睛所获得的那个被距离缩小的图像也许并不更准确。在这两个拉贝玛中,哪一个是真实的?我对这段戏曾寄予很大希望,何况她的同伴们在比这逊色得多的片断中曾不断向我揭示巧妙的弦外之音。我料想拉贝玛的语调肯定比我在家中阅读剧本时所想象的语调更令人惊叹,然而,她甚至没有达到奥侬娜或阿里西所可能使用的朗诵技巧,她用毫无变化的单调节奏来朗诵那一长段充满对比的独白,那些对比是如此令人注目,以致一位不太聪明的悲剧演员,甚至中学生,都不可能不觉察它的效果。她念得很快,当她念完最后一句话时,我的思想才意识到她在前几句台词中所故意使用的单调语气。
\par 终于,在观众狂热的掌声中,我最初的赞佩之情爆发了。我也鼓起掌来,而且时间很长,希望拉贝玛出于感激而更加卖力,那样一来,我便可以说见识过她最精湛的演技了。奇怪的是,观众热情激昂的这一时刻,也正是拉贝玛作出美妙创新的时刻(我后来才知道)。当某些超先验的现实向四周投射射线时,群众是最早的觉察者。例如,发生了重大事件,军队在边境上处于危急之中或者溃败,或者告捷,这时传来的消息模糊不清,未给有教养者带来任何重要信息,但却在群众中引起巨大震动。有教养者不免对震动感到吃惊,但当他们从专家那里获悉真实的军事形势以后,就不能不佩服民众觉察这种“光晕”(它伴随重大事件,在百里之外也可被人看见)的本领。人们获悉战争捷报,或者是在事后,在战争结束以后,或者是在当时,从门房兴高采烈的神气中感知。同样,人们发现拉贝玛演技精湛,或者是在看完戏一周以后从批评家那里得知,或者当场从观众的喝彩声中得知。然而,群众的这种直接认识往往和上百种错误认识交织在一起,因此,掌声往往是错误的,何况它是前面掌声的机械后果,正如风暴使海水翻腾,即使当风力不再增大,海浪也仍然汹涌一样。管他呢,我越鼓掌就越觉得拉贝玛演得好。坐在我旁边的一位普通妇女说:“她可真卖劲,用力敲自己,满台跑,这才叫演戏哩。”我很高兴找到这些理由来证明拉贝玛技艺高超,但同时也想到它们说明不了问题。农民感叹说:“画得多么好!真是妙笔!瞧这多美!多细!”这难道能说明《蒙娜丽莎》或本韦努托\footnote{本韦努托(1500—1571),意大利雕塑家。}的《珀耳修斯》吗?但我仍然醉饮群众热情这杯粗酒。然而,当帷幕落下时,我感到失望,我梦寐以求的乐趣原来不过如此,但同时,我需要延长这种乐趣,我不愿离开剧场从而结束剧场的经历——在几个小时里它曾是我的生活,我觉得直接回家好比是流放;幸亏我盼望到家以后能从拉贝玛的崇拜者口中再听到关于她的事,这位崇拜者正是那位使我获准去看《淮德拉》的人,即德·诺布瓦先生。
\par 晚饭前,父亲把我叫进书房,将我介绍给德·诺布瓦先生。我进去时,大使站起来,弯下他那高大的身躯向我伸出手,蓝色的眼睛关注地看着我。在他作为法兰西的代表的任职期间,人们往往将过往的外国人介绍给他,其中不乏多少有点名气的人物,甚至著名歌唱家;而他明白,有朝一日,当人们在巴黎或彼得堡提起这些人时,他便可以夸耀说曾在慕尼黑或索非亚和他们一同度过夜晚,因此他养成了这种习惯:亲切地向对方表示认识他有多么荣幸。此外,他认为,在外国首都的居留期间,他既能接触来往于各国首都的有趣人物,又能接触本地居民的习俗,从而对不同民族的历史、地理、风俗以及对欧洲的文化运动获得深入的、书本上所没有的知识,因此他在每个新来者身上应用尖锐的观察力,好立即弄清楚站在他面前的是什么人。长久以来,他不再被派驻国外,但每当别人向他介绍陌生人,他的眼睛便立即进行卓有成效的观察,仿佛眼睛并未接到停职通知,同时他的举止谈吐试图表明新来者的名字对他并不陌生。因此,他一面和气地、用自知阅历颇深的要人的神气和我谈话,一面怀着敏锐的好奇心,并出于他本人的利益而不停地观察我,仿佛我是具有异域习俗情调的、颇具教益的纪念性建筑物,或者是巡回演出的明星。因此他既像明智的芒托尔\footnote{芒托尔,古希腊神话中的智者。}那样庄严与和蔼,又像年轻的阿纳加西斯\footnote{阿纳加西斯,公元前六世纪哲学家。此处指十八世纪出版的《青年阿纳加西斯希腊游记》。}那样充满勤奋的好奇心。
\par 关于《两个世界评论》,他绝口不提为我斡旋,但对我过去的生活及学习,对我的兴趣,却提出了一系列问题。我这是头一次听见别人将发挥兴趣爱好作为合理的事情来谈论,因为在此以前,我一直认为应该压制兴趣爱好。既然我爱好文学,他便使话题围绕文学,并且无比崇敬地谈论它,仿佛它是上流社会一位可尊敬的、迷人的女士。他曾在罗马或德累斯登与她邂逅而留下美妙的回忆,但后来由于生活所迫而很少有幸再与她重逢。他带着几乎放荡的神情微笑,仿佛羡慕我比他幸运、比他悠闲,能与它共度美好时光。但是,他的字眼所表达的文学与我在贡布雷时对文学所臆想的形象完全不同,于是我明白我有双重理由放弃文学。以前我仅仅意识到自己缺乏创作的天赋,而现在德·诺布瓦先生使我丧失创作欲望。我想向他解释我的梦想。我激动得战栗,唯恐全部话语不能最真诚地表达我曾感觉到、但从未试图向自己表明的东西。我语无伦次,而德·诺布瓦先生呢,也许出于职业习惯,也许出于要人们所通常具有的漠然态度(既然别人求教于他,他便掌握谈话的主动权,听任对方局促不安、使出全身解数,而他无动于衷),也许出于想突出头部特点的愿望(他认为自己具有希腊式头型,尽管有浓密的颊须),当你向他阐述时,他的面部绝对地静止不动,使你以为面前是石雕陈列馆里一座古代胸像——而且是耳聋的!突然间,就像拍卖行估价人的锤声或者戴尔菲的神谕,响起了大使的回答,它令人激动,因为你从他那木然的脸上无法猜到他对你的印象或者他即将发表什么意见。
\par “正巧,”他不眨眼地一直盯着结结巴巴的我,突然下结论似的说,“我有一个朋友,他的儿子,mutatis mutandis\footnote{拉丁文,此处意为:基本上。},和你一样。(于是他用一种安慰的口气谈起我们的共同倾向,仿佛这不是对文学,而是对风湿病的倾向,而他想告诉我我不会因此丧生。)他放弃了父亲为他安排的外交仕途,不顾流言蜚语投身创作。当然他没有什么可后悔的。两年以前——他的年龄当然比你大得多——他发表了一部作品,是关于对维多利亚尼昂萨湖\footnote{维多利亚尼昂萨湖是赤道非洲的一个大湖。}西岸的‘无限性’的感触。今年又写了一本小册子,篇幅稍短,但笔锋犀利,甚至尖刻,谈的是保加利亚军队中的连发枪。这两本书使他成为了不起的人物。他已经走了一大段路,不会中途停下来的。在伦理科学院里,人们曾两三次提到他,而且毫无贬谪之意,虽然目前还未考虑提他为候选人。总之,他还不能算声誉显赫,但他的顽强搏斗已经赢得了优越的地位和成就。要知道成功并不总是属于那些骚动者、挑拨者、制造混乱者(他们几乎都自命不凡)。他通过努力一举成名。”
\par 父亲已经看见我在几年以后成为科学院院士了,因此十分得意,而德·诺布瓦先生又将这种满意推向高峰,因为他在仿佛估计自己行动后果的片刻犹豫以后,递给我一张名片,并说:“你去见见他吧,就说是我介绍的。他会给你一些有益的忠告。”他的话使我激动不安,仿佛他宣布了我次日就将登上帆船当见习水手。
\par 我从莱奥妮姨母那里继承了许多无法处置的物品和家具,以及几乎全部现金财产(她在死后表达了对我的爱,而在她生前我竟一无所知)。这笔钱将由父亲代管,直到我成年,因此父亲请教德·诺布瓦先生该向何处投资。德·诺布瓦先生建议购买他认为十分稳妥的低率证券,特别是英国统一公债及年息百分之四的俄国公债。他说:“这是第一流的证券,息金虽然不是太高,但本金至少不会贬值。”至于其他,父亲简略地告诉客人自己买进了什么,客人露出一个难以觉察的微笑,表示祝贺。德·诺布瓦先生和所有资本家一样,认为财富是值得羡慕的东西,但一当涉及他人的财产时,他认为以心照不宣的神气表示祝贺则更为得体。另一方面,由于他本人家财万贯,他便将远不如他阔气的人也看做巨富,同时又欣慰而满意地品味自己在财富上的优越地位。他毫不犹豫地祝贺父亲在证券的“结构”问题上表现出“十分稳妥、高雅、敏锐的鉴赏力”,仿佛他赋予交易证券的相互关系,甚至交易证券本身以某种美学价值似的。父亲谈到一种比较新的罕为人知的证券,这时德·诺布瓦先生便说(你以为只有你读过这本书,其实他也读过):“我当然知道啦,有一阵子我注意它的行情,很有趣。”同时露出对回忆入迷的微笑,仿佛他是某杂志的订户,一段一段地读过那上面长篇连载的最新小说。“我不劝阻您购买将发行的证券,它很有吸引力,价格也很有利。”至于某些老证券,父亲已记不清它们的名称了,往往将它们与类似的证券相混淆,因此便拉开抽屉取出来给大使看。我一见之下大为着迷;它们带着教堂尖顶及寓意图像的装饰,很像我往日翻阅的某些富于幻想的古老书刊。凡属于同一时期的东西都很相似。艺术家既为某一时期的诗歌作画,同时也受雇于当时的金融公司。河泊开发公司发行的记名证券,是一张四角由河神托着的、饰有花纹的长形证券,它立即使我回忆起贡布雷杂货店橱窗里挂着那些《巴黎圣母院》和热拉尔·德·内瓦尔\footnote{热拉尔·德·内瓦尔(1808—1855),法国著名作家。}的书。
\par 父亲瞧不起我这种类型的智力,但这种蔑视往往被亲子之爱所克制,因此,总的来说,他对我做的一切采取盲目的容忍态度。他不假思索地叫我取来我在贡布雷散步时所写的一首散文短诗。当年我是满怀激情写的,因此,我觉得谁读到它都会感动不已。然而,德·诺布瓦先生丝毫未被感动,他交还给我时一言不发。
\par 母亲一向对父亲的事务毕恭毕敬,此时她走了进来,胆怯地问是否可以开饭。她唯恐打断了一场她不应介入的谈话。此刻父亲确实在向侯爵谈到将在下一次委员会会议上提出的必要措施,他那特殊的声调使人想起两位同行——好比两位中学生——在外行面前交谈的口吻,他们由于职业习惯而享有共同的回忆,但既然外行对此一无所知,他们当着这些外行的面提起往事时只能采取歉然的口吻。
\par 此刻,德·诺布瓦先生的面部肌肉已经达到了完美的独立,因此他能够以听而不闻的表情听人说话:父亲终于局促不安起来:“我本来想征求委员会的意见……”在转弯抹角以后,他终于说道。可是,从这位贵族气派的演奏能手的面孔上、从他那像乐师一样呆滞地静等演奏时刻的面孔上,抛出了这句话,它不紧不慢,几乎用另一种音色来结束已经开始的乐句:“当然,您完全可以召集委员们开会,何况您认识他们每一个人,让他们来一趟就行了。”显然,这个结束语本身毫无新奇之处,但是,在它以前的那个状态使它显得突出,使它像钢琴上的乐句那样清脆晶莹,十分巧妙地令人耳目一新,就好比在莫扎特的协奏曲中,一直沉默的钢琴按规定的时刻接替了刚才演奏的大提琴。
\par “怎么样,对戏满意吗?”在餐桌前就坐时,父亲问我道。他有意让我显露一番,认为我的兴奋会博得德·诺布瓦先生的好感。“他刚才去听拉贝玛的戏了,您还记得我们曾经谈起过。”他转身对外交家说,采取一种回顾往事的、充满技术性的神秘语调,仿佛他谈的是委员会。
\par “你一定会十分满意吧,特别是你这是第一次看她演出。令尊本来担心这次小小的娱乐会有损于你的健康。看来你不是十分结实,一个文弱书生。不过我叫他放心,因为现在的剧场和二十年前可是大不一样。座位还算舒适,空气也不断更换,当然我们还得大大努力才能赶上德国和英国,他们在这方面,以及其他许多方面都比我们先进。我没有看过拉贝玛夫人演《淮德拉》,但我听说她的演技极为出色。你肯定很满意吧?”
\par 德·诺布瓦先生比我聪明千倍,他肯定掌握我未能从拉贝玛的演技中悟出的真理,他会向我揭示的。我必须回答他的提问,请他告诉我这个真理,这样一来,他会向我证明我去看拉贝玛演出确实不虚此行。时间不多,应该就基本点提出疑问,然而,哪些是基本点呢?我全神贯注地思考我所得到的模糊印象,无暇考虑如何赢得德·诺布瓦的赞赏,而是一心想从他那里获得我所期望的真理,因此我结结巴巴地讲着,顾不上借用现成的短语来弥补用词之贫乏,而且,为了最终激励他说出拉贝玛的美妙之处,我承认自己大失所望。


\paragraph*{1}

\par “怎么,”父亲恼怒地叫了起来,因为我这番自认不开窍的表白会给德·诺布瓦先生留下不好的印象,“你怎么能说你没感到丝毫乐趣呢?外祖母讲你聚精会神地听拉贝玛的每一句台词,瞪着大眼睛,没有任何观众像你那样。”
\par “是的,我的确全神贯注,我想知道她的出类拔萃表现在什么地方。当然,她演得很好……”
\par “既然很好,你还要求什么呢?”
\par “有一点肯定有助于拉贝玛夫人的成功。”德·诺布瓦先生说。他特别转头看着母亲,一来避免将她撇在谈话之外,二来也是认真地对女主人表示应有的礼貌,“那就是她在选择角色时所表现的完美鉴赏力,正是鉴赏力给她带来了名副其实的成功,真正的成功。她极少扮演平庸角色,这一次扮演的是淮德拉。再说,她的鉴赏力也体现在服装和演技中。她经常去英国和美国作巡回演出,并且大获赞赏,但是她没有染上庸俗习气,我指的不是约翰牛,那未免不够公允,至少对维多利亚时期的英国来说不够公允,我指的是山姆大叔。她从来没有过度刺目的颜色,从来没有声嘶力竭的叫喊。她那美丽的悦耳的声音为她增添光彩,而她对声音的运用竟如此巧妙,真可谓声乐家!”
\par 演出既已结束,我对拉贝玛的艺术的兴趣便不再被现实所压制和约束,它越来越强烈,但我必须为它寻找解释。再说,当拉贝玛表演时,她对我的眼睛和耳朵提供的是在生活中浑然一体的东西,我的兴趣仅仅予以笼统的关注,而未加任何区分或分辨,因此此刻,它在这番称赞艺术家朴实无华和情趣高尚的颂词中高兴地发现一种合理解释,它施展吸引力,将溢美之词据为己有,正好比一位乐天的醉汉将邻居的行为据为己有并大发感慨一样。“是的,”我心里想,“多么美妙的声音,没有喊叫,多么朴素的服装!挑了淮德拉这个角色,又是多么明智!不,我没有失望。”
\par 胡萝卜牛肉冷盘出现了。在我家厨房的“米开朗琪罗”的设计下,牛肉躺在如晶莹石英一般的、硕大的冻汁晶体之上。
\par “您的厨师是第一流的,夫人,”德·诺布瓦先生说,“难得呀!我在国外时往往不得不讲排场,因此我明白找一个高超的厨师多么不容易。您这真是盛宴。”
\par 的确如此,弗朗索瓦丝兴高采烈地为贵宾准备美餐,好显显身手。她卖力地重新施展她在贡布雷时的绝技,没有客人来吃饭时她已经不愿意这样费心劳神了。
\par “这是在夜总会,我是指最高级的夜总会,所尝不到的。焖牛肉,冻汁没有糨糊气味,牛肉有胡萝卜的香味,真是了不起!请允许我再加一点。”他一面说,一面做手势表示还要一点冻汁,“我真想尝尝府上的法代尔\footnote{法代尔,法国十七世纪大孔代亲王的著名膳食总管。}的另一种手艺,比方说,尝尝她做的斯特罗加诺夫\footnote{斯特罗加诺夫,为俄国财政家,以家族名字命名的这道菜是奶汁牛肉。}式牛肉。”
\par 德·诺布瓦先生为了替餐桌增添情趣,给我们端上了他经常招待同行的那些形形色色的故事。有时他引用某位政治家演说中可笑的复合句(此人惯于此道),句子既冗长臃肿,又充满自相矛盾的形象。有时他又引用某位文体高雅的外交家的明捷快语。其实,他对这两种文体的判断标准与我对文学的判断标准毫无共同之处。对许多细微区别,我毫不理解。他哈哈大笑加以嘲弄的字句与他赞不绝口的字句,在我看来,并无多大区别。他是另外一种人,关于我所喜爱的作品,他会说:“你看懂了?老实说,我看不懂,我不在行。”而我也可以以其人之道还治其人之身;他在反驳或演说中所看到的机智或愚蠢、雄辩或夸张,我都无法领会。既然没有任何可以被感知的理由来说明此优彼劣,那么这种文学在我眼中就更为神秘,无比隐晦。我领悟到,重复别人的思想,这在政治上并非劣势的标志,而是优势的标志。当德·诺布瓦先生使用报刊上随手拈来的某些用语,并且配之以强调语气时,这些用语一旦为他所用就变为行动,引人注意的行动。
\par 母亲对菠萝块菰色拉寄予很大期望。大使用观察者的深邃目光对这道菜凝视片刻,然后吃了起来,但保持外交家的审慎态度,不再坦露思想。母亲坚持要他再吃一点,德·诺布瓦先生又添了一次,但没有说出人们所期待的恭维话,只是说:“遵命,夫人,既然这是您的命令。”
\par “报上说您和狄奥多西国王作过长谈。”父亲说。
\par “不错。国王对面孔有惊人的记忆力。那天他看见我坐在正厅前排便想起了我,因为我在巴伐利亚宫廷里曾经见过他好几次,当时他并未想到东部王位(您知道,他是应欧洲大会之请而登基的,他甚至犹豫了很久才同意,他认为这个王位与他那全欧最高贵的家族不太相称)。一位副官走来请我去见国王陛下,我当然乐于从命。”
\par “您对他这次访问的结果满意吗?”
\par “很满意!当初有人担心这位年轻君主能否在如此复杂的形势下摆脱困境,这种担心是可以理解的。至于我,我完全相信他的政治嗅觉,而且事实远远超过了我的希望。根据权威方面的消息,他在爱丽舍宫的致词,从第一个字到最后一个字都是他亲自起草的,当之无愧地引起各方面的好感。这确实是高招。当然未免过于大胆,但事实证明这种胆略是对的。外交传统固然有其优点,但正是由于它,我们两国的关系笼罩在一种令人窒息的、封闭的气氛中,更换新鲜空气的办法便是打破玻璃窗,别人当然无法提出这种建议,只有狄奥多西可以这样做,而他确实这样做了。他那襟怀坦荡的态度令众人倾倒,他用词妥帖得体,不愧为母系是博学多才的王公贵族的后代。在谈到他的国家和法国之间的关系时,他用的是‘亲缘关系’一词,这种用词在外交词汇中极为罕见,但在此却极为恰当。你瞧文学毫无害处,即使对外交、对君主而言,”他最后这句话是对我说的,“当然,此事早有迹象,两个强国之间的关系原来就大有改善,但毕竟由他嘴里说了出来。他的话正是人们所期望的,而且用词巧妙,所以效果惊人。我当然双手赞同啦。”
\par “您的朋友福古贝先生多年来致力于改善两国关系,他一定很高兴吧。”
\par “当然,何况国王陛下像往常一样,有意让他喜出望外。再说,从外交部长开始,人人都大吃一惊,无一例外。据说外交部长对此事不甚满意。别人问他时,他提高嗓门,好让周围的人听见他那直言不讳的回答:‘我既未被征求意见,也未收到通知。’以此明确表示他与此事毫不相干。当然,这件事引起纷纷议论,”他狡黠地笑笑,然后又说,“我不敢担保那些将‘无为’奉为最高信条的同事不因此坐立不安。至于福古贝,你们知道他由于亲法政策而受到猛烈抨击,这使他很难过,何况此公心地善良,而且很敏感。这一点我可以作证。虽然他比我年轻许多,但我们是老朋友了,常有来往,我很了解他。再说谁不了解他呢?他的心灵清澈见底,这是他可以受指责的唯一缺点,因为外交家没有必要像他那样透明。现在有人提出派他去罗马,这当然是晋升,但也是‘啃骨头’。我这是私下对您说,福古贝虽然毫无野心,但对新职不会不高兴,他绝不会拒绝这杯苦酒。他也许会干出奇迹。他是孔苏尔塔\footnote{孔苏尔塔,意大利外交部所在地。}所赞同的人。对这样一位艺术家,法尔内兹宫和卡拉什走廊\footnote{法尔内兹宫,法国驻罗马使馆,其内有由十六世纪画家卡拉什装饰的走廊。}是最合适的地方了。至少不会有人恨他。而在狄奥多西国王周围,有一批依附于威廉街\footnote{威廉街是德国外交部所在地。}的奸党,他们顺从地执行威廉街的意图,千方百计地给福古贝捣乱。福古贝不但要对付宫廷阴谋,还要对付帮闲文人的辱骂。他们后来像所有被豢养的记者一样怯懦地求饶,但同时依然故我地刊登流氓无赖对我国代表的无理指责。在一个多月的时间里,敌人围着福古贝跳头皮舞\footnote{这是印第安人的舞蹈,胜利者在割下战败者的头皮以前围着他跳舞。}。”德·诺布瓦先生特别着重这最后一个词:“不过,俗话说:‘早有防范,免遭暗算’。他一脚踢开了诽谤辱骂。”他的声音更响亮,眼睛射出凶光,以致我们在片刻间停止了吃饭。“有一句漂亮的阿拉伯谚语:‘任凭群犬乱吠,商队依然前进。’”德·诺布瓦先生抛出这条谚语后瞧着我们,观察它在我们身上产生什么效果。效果显著。我们熟悉它,因为那一年它在有身份的人中间流行,而另一句谚语“种蒺藜者得刺”却被淘汰,因为它精力不足,不像“为人作嫁”那样永不疲劳、永葆活力。要知道这些社会名流的语言采取的是三年一换的轮种制的。德·诺布瓦先生在《两个世界评论》的文章中,擅长使用此种类型的引文,其实它们在有根有据、信息可靠的文章中完全是多余的。德·诺布瓦先生根本不需要这些装饰,只需挑选关键时刻——他也正是这样做的——就行了,如“圣詹姆斯\footnote{指英国外交部。}已感危机在即”;或者“歌手桥\footnote{指奥地利外交部。}群情激动,正不安地注视双头王朝的自私而巧妙的政策”;或者“蒙泰奇托里奥\footnote{指沙俄外交部。}发警报”;或者“乐厅广场\footnote{指意大利议院。}所永远惯用的两面手法”。即使是外行的读者,一看见这些用语便立即明白作者是职业外交家,并表示赞赏。但有人说他不仅仅是职业外交家,他的修养更为卓越,因为他对谚语的运用恰到好处,而其中最完美的典范是“正如路易男爵\footnote{路易男爵是法王路易十八和路易菲力普的财政大臣。}所说,您给我良好政治,我给您良好财政。”(因为当时还未从东方传来日本谚语“在交战中,多坚持一刻者必胜无疑”。)正是这种名人学者的声誉,以及漠然的面具下所隐藏的名副其实的阴谋天才,使德·诺布瓦先生成为伦理科学学院的院士,而且有人甚至认为他进法兰西学院也无不可,因为有一次,他在指出为了和英国和解而与俄国联盟的必要性时,竟然写道:“有一点应该让奥赛码头\footnote{指法国外交部。}的人明白,应该写进所有的地理课本中(这方面确有遗漏),应该作为中学毕业生获得业士学位的标准,那就是:如果说‘条条大路通向罗马’,那么,从巴黎去伦敦必须经过彼得堡。”
\par “总之,”德·诺布瓦先生继续对父亲说,“福古贝这次大为成功,甚至超过他自己的估计。当然他预料会有一篇十分得体的祝酒辞(在近年来的阴云以后这已算是了不起了),但没有想到比那更胜一筹。有几位当时在场的人对我说,祝酒辞的效果决非事后阅读所能领会的,国王堪称演说艺术家,他的朗读、停顿都很有讲究,让听众对各种言外之意及微妙之处心领神会。我听人讲过一件很有趣的事,它又一次证明狄奥多西国王充满那种颇得人心的青春风采。‘亲缘关系’一词可以说是演讲中的一大革新,您瞧,它将成为各个使馆长期议论的话题。国王陛下在吐出这个词时,大概想到会使我们这位大使欣喜异常——这是对他的努力、甚至他的梦想的公正的报偿,并且会使他获得元帅权杖——因此他半转身朝着福古贝,用奥丹尚家族那迷人的眼神盯着他,一个音节一个音节地说‘亲缘关系’这个十分恰当的、新颖不凡的词。他的声调表明他使用这个词是十分慎重的,他对它的分量了如指掌。据说福古贝激动得不能自抑,在某种程度上,我认为我能理解他的心情。据十分可靠的消息说,宴会以后,国王陛下走近夹在人群中的福古贝,低声对他说:‘您对我这个学生满意吗,亲爱的侯爵?’显然,”德·诺布瓦先生又说,“这篇祝酒辞的效力超过了二十年的谈判,它更加密切了两国之间的——用狄奥多西二世的生动语言来说——‘亲缘关系’。这仅仅是一个词,可是您瞧着吧,它会平步青云,全欧洲的报纸都在重复它,它引起了广泛的兴趣,发出了新的声音。话说回来,这是国王的一贯作风。我不敢说他每天都能发现如此纯净的钻石,但是,在他精心准备的演讲中,或者在他的即兴谈话中,他少不了塞进一句俏皮话,作为自己的标志——或者说签名。在这一点上,我决无偏袒之嫌,因为我一向反对这种俏皮话,二十句中有十九句都是危险的。”
\par “是的。我想德国皇帝最近的电报一定不合您的口味吧。”父亲说。
\par 德·诺布瓦先生抬眼看了一下天花板,仿佛在说:“啊!这家伙!首先,这是忘恩负义,不仅仅是错误,而且是犯罪,可以说是骇人听闻的蠢事!其次,如果没有人加以制止,那么这个赶走了俾斯麦的人\footnote{即德国皇帝威廉二世,他迫使俾斯麦辞职与英交恶。}很可能渐渐抛弃俾斯麦的全部政策,到了那时,谁也不知道会发生什么。”
\par “我丈夫告诉我,先生,说您可能在近两三年的夏天让他和您一道去西班牙,我真为他高兴。”
\par “是的,这是一个很诱人的计划。我很高兴,我很乐意和您一同旅行,亲爱的朋友。您呢,夫人,您打算怎样度假?”
\par “不知道。也许和儿子一同去巴尔贝克。”
\par “啊!巴尔贝克是好地方。几年以前我去过。那里正在兴建漂亮别致的别墅,我想您会喜欢那里的。不过,您能告诉我为什么看上这个地方吗?”
\par “我儿子很想看教堂,特别是巴尔贝克教堂。我最初有点担心,生怕旅途劳累,特别是吃住不便,会影响他的健康。不过最近听人说那里盖了一家很好的饭店,里面有他所必需的舒适设备,那么他可以住些时候。”
\par “啊!我得把这消息告诉一位对此很关心的女士。”
\par “巴尔贝克教堂很了不起吧,先生?”我问道,抑制心中的不快,因为在他眼中,巴尔贝克的魅力在于漂亮别致的别墅。
\par “不坏,确实不坏,不过,它毕竟无法和精雕细琢的真正珍宝相比,例如兰斯教堂、沙特尔教堂,以及珍品中之珍品——我最喜爱的巴黎圣礼拜堂。”
\par “巴尔贝克教堂的一部分属于罗曼式吧?”
\par “不错,是罗曼式,这种风格本身就极为古板,比不上后来的哥特式建筑。哥特式优美、新颖,石头都精雕着花边。巴尔贝克教堂的确有点与众不同,你既然到了那里,这个教堂当然值得一游。如果哪天下雨你无处可去,可以进去看看图维尔\footnote{图维尔(1642—1701),法国元帅。}的墓。”
\par “您出席昨天外交部的宴会了吗?我脱不开身。”父亲说。
\par “没去,”德·诺布瓦先生微笑着回答,“坦白地说,我没去,而是参加了另一个完全不同的晚会。我去一位女士家吃饭,你们大概听说过她,就是美丽的斯万夫人。”
\par 母亲控制住一阵战栗,因为她比父亲敏感,她已经为他即将感到的不快而担忧。他的不快往往最先被她感知,就好比法国的坏消息最先在国外,然后才在国内被人知晓。但是,她想知道斯万夫妇接待些什么人,于是便向德·诺布瓦先生打听他在那里遇见了谁。
\par “我的天……去那里的似乎主要是……男士们。有几位已婚男人,但他们的妻子身体不适,没有去。”大使用一种故作天真的微妙口吻说,而且环顾左右,他那柔和审慎的目光似乎想冲淡嘲弄,其实反而更巧妙地加强了嘲弄效果。
\par “应该说,”他继续说道,“公平地说,那里也有些女士,不过……她们属于……怎么说好呢,与其说属于斯万(他念成‘斯凡’)的社交圈子,不如说属于共和派。谁知道呢?也许有一天那里会成为政治沙龙或文化沙龙,而他们似乎也很满意。我觉得斯万炫耀得未免过分,老说某某人和某某人下星期邀请他们夫妇,其实,和这些人的交往有什么值得夸耀呢?他表现得既不稳重,又无趣味,几乎连分寸也不懂,像他这样的雅士竟然如此,不能不令人吃惊。他不断说:‘我们每晚都有宴请。’仿佛这很光彩,仿佛他成了新贵,其实他并不是。他以前有许多朋友,甚至许多女友。在这里我不想说得过头,也不想过于冒昧,但我认为在他的女友中,至少有一位(尽管不是全部或大部分女友)——而且身份显赫——是不会断然拒绝和斯万夫人结识的,那样一来,会有不少人成为帕尼尔热羊\footnote{法国十六世纪作家拉伯雷小说中的故事,帕尼尔热羊即指盲目模仿。},步其后尘。然而,斯万似乎未作过任何努力。噫,还有内塞尔罗德式布丁\footnote{以英国外交家内塞尔罗德命名的布丁(主要原料为栗子泥)。}!在这顿卢库卢斯\footnote{卢库卢斯为古罗马将军,以美食者著称。}式的盛宴以后,我看得去卡尔斯巴德\footnote{卡尔斯巴德,波希米亚地区疗养地。}疗养了。也许斯万感到阻力太大,无法克服。他这门婚事令人不快,这是肯定的。有人说那女士很有钱,这真是胡说八道。总之,这一切似乎叫人不大愉快。斯万有一位家产万贯而且声望极高的姑姑,她丈夫,就财富而言,可算实力雄厚。但是她不但拒绝接待斯万夫人,而且发起一场名副其实的运动,让她的朋友和熟人们都抵制斯万夫人。我这并不是说有哪一位有教养的巴黎人对斯万夫人有不尊敬的表示……不是!绝对不是!何况她丈夫是勇于决斗的人。总之,这位交游甚广,而且经常出入上流社会的斯万居然对这些至少可以称为三教九流的人大献殷勤,未免古怪。我以前认识他,他是一位素有教养,在最高级的社交圈里也闻名一时的人物,但他如今竟然感恩涕零地感谢邮政部办公室主任大驾光临,而且询问斯万夫人‘能否有幸’拜访主任夫人,这使我感到既吃惊又好笑。他大概不太自在,因为这显然是两个不同的世界。但是我认为他并不痛苦。在婚前的那几年里,那个女人确实玩了不少手腕来敲诈他。每当他拒绝她时,她便把女儿从他身边夺走。可怜,斯万这位雅士过于天真,他总是认为女儿的被劫持只是巧合,他不愿正视现实,而她还时时对他大发雷霆,所以当时人们想,一旦她达到目的,成为他妻子以后,她会更肆无忌惮,他们的生活会成为地狱。然而恰恰相反!斯万谈论妻子的口吻往往成为人们的笑柄,甚至是恶意嘲笑的口实。你总不能要求隐约感觉到自己当了……(你们知道莫里哀的那个词\footnote{即莫里哀用的“王八”一词。})的斯万大肆声张吧……不过,他把妻子说得那么贤慧,也未免过分。话说回来,这一切并不像人们想象的那么虚假,显然她对他是有感情的,只不过这是她所特有的、并非所有的丈夫都喜欢的方式。咱们这是私下说,既然斯万认识她多年,他又不是白痴傻瓜,他当然知道底细。我并不否认她水性杨花,可是斯万本人呢,按照你们不难想象的此刻满天飞的闲言碎语,他也喜欢寻花问柳。然而,她感激他为她做的一切,所以,和大家的担心相反,她变得像天使一般温柔。”
\par 其实奥黛特的变化并不像德·诺布瓦先生所想象的那么大,她以前一直以为斯万不会娶她。她曾含沙射影地说某某体面人和情妇结了婚,这时斯万总是冷冰冰地一言不发。如果她直截了当地问他:“怎么,他以这种方式回报为他奉献青春的女人,你不以为然,不认为了不起?”他最多只是冷冷地回答:“我没说这不好。各人有各人的做法。”她甚至几乎相信,正如他在气头上说的,他会完全抛弃她,因为她曾听见一位女雕刻家说:“男人什么都干得出来,他们无情无义。”奥黛特被这句深邃而悲观的格言所震动,并时时引用,奉为信条。她那失望的神气仿佛在说:“没什么办不到的事,我要碰碰运气。”而她以前所遵循的乐观主义的生活格言是:“对爱你的男人你可以为所欲为,他们是白痴。”她的面部表情只是眨眼睛,仿佛在说:“你别怕,他什么也不会摔碎的。”奥黛特的一位女友和一个男人同居,时间比奥黛特和斯万的同居期短,而且也没有孩子,但她竟让他娶了她,现在相当受人尊重,并被邀请参加爱丽舍宫的舞会。她对斯万的行为会作何想法呢?奥黛特为此很苦恼。如果有一位比德·诺布瓦先生思想更为深刻的医生,他大概会下诊断说奥黛特的乖戾来自这种屈辱和羞愧的感觉,她那穷凶极恶的外在性格并非她的本质,并非不治之症;他还会轻而易举地预言后来果然发生的事,即一种新的关系——婚姻关系——将使这些难以忍受的、每日发生的、但决非气质性的冲突奇迹般地立即销声匿迹。值得惊奇的是,几乎所有的人都对这门婚事感到惊讶,他们大概不明白爱情这个现象具有纯粹的主观性,它是一种创造,它将我们本身的许多因素附加在社会中某人身上,从而创造一个与这同名人毫不相似的人。人们往往感到不可理解:某人竟然在我们眼中如此举足轻重,其实他们和我们所见到的并非同一个人。然而,说到奥黛特,人们应该看出,虽然(当然)她对斯万的精神生活并未完全理解,但她至少知道他的研究题目及全部详情,她熟悉弗美尔弗美尔(1632—1675),荷兰画家。的名字如同熟悉她的裁缝的名字一样。她了解斯万的全部性格;这种男人的性格往往被世人忽视或嘲笑,只有在情妇或姐妹眼中它才具有真实的、可爱的形象。我们很珍惜自己的性格,甚至包括我们极想改正的性格,因此,当一个女人对此习以为常并采取宽容和善意打趣的态度(正如我们本人对它习以为常,我们的父母对它习以为常一样)时,老的爱情便像家庭感情一样温柔和强烈。当某人站在我们的角度来评论我们的缺点时,他和我们之间的关系便变得神圣了。在这些特点之中,有一些既涉及斯万的智力又涉及他的性格,而且,既然根源在于性格,奥黛特对它们最为敏感。她抱怨人们没有注意到:斯万在书信和谈吐中所表现的众多特点在他的创作和研究文章中也有所体现。她劝他更发挥这些特点。她之所以乐于这样是因为她在他身上所欣赏的正是它们,她爱它们是因为它们属于他,因此她自然而然地希望人们在他的作品中发现它们。也许她认为更为生动的作品能最后使他成名,并能使她实现她在维尔迪兰家所梦想的高于一切的事业:沙龙。
\par 有些人认为这种婚姻荒唐可笑,他们设身处地地自问:“如果我和德·蒙莫朗西小姐结婚,德·盖尔芒特先生会怎么想呢?布雷奥代会怎么说呢?”二十年前,斯万可能和他们具有同样的社会理想。他曾煞费苦心地加入赛马俱乐部,他曾盼望缔结一门显赫的婚姻,以巩固自己的地位,并最终成为巴黎最知名的人士。然而,和任何形象一样,婚事在当事人眼中的形象也必须不断从外界得到滋补,才不会逐渐衰败直至完全消失。你最炽热的愿望是对冒犯过你的人进行侮辱,可是,如果你换了一个地方,从此听不见人们谈起他,那么这个敌人在你眼中将最终变得无足轻重。当初,你是为了某些人而渴望进赛马俱乐部或法兰西研究院,但是,如果你和他们二十年不见面,那么,进入这个机构的前景将失去一切魅力。长期的爱情,如同退休、生病或改宗一样,以新的形象替代旧形象。斯万与奥黛特结婚,这并不意味着他放弃社交野心,因为奥黛特早已使他脱离(从俏皮的意义上讲)那种野心,而且,如果他尚未脱离,那么他更令人敬重,因为一般说来,不体面的婚事最受人敬重(所谓不体面,并非指金钱婚姻:由买卖关系而结合的夫妻最终都被上流社会所接纳,或是由于传统,或是由于先例,为了一视同仁),因为它意味着放弃优越的地位以成全纯粹感情生活中的乐趣。此外,与不同种族的人,大公夫人或轻浮女人结成配偶,与显贵女士或卑贱女人结婚(像孟德尔\footnote{捷克斯洛伐克僧侣孟德尔(1822—1884)曾对不同的植物杂交进行研究。}主义者所实行的或神话中所讲述的杂交一样),这可能给作为艺术家——甚至堕落者——的斯万带来某种快感。每当他考虑和奥黛特结婚时,他担心的只有一个人,就是德·盖尔芒特公爵夫人,而这并非出于附庸风雅,相反,奥黛特不把德·盖尔芒特夫人放在心上,她想到的不是居于广阔苍穹高处的那些人,而仅仅是直接在她头上的那些人。每当斯万遐想奥黛特成为他的妻子时,他总是想象如何将她,特别是女儿,引见给洛姆公主,后者在公公死后立即成为德·盖尔芒特公爵夫人。他不愿带她们去别的沙龙。他激动地幻想公爵夫人将如何对奥黛特谈到他,奥黛特又会说些什么。他幻想德·盖尔芒特夫人会喜欢希尔贝特,会溺爱她,会使他为女儿感到骄傲。他自得其乐地幻想引见的场面,连细节也十分精确,就好比买彩票的人仔细考虑万一中彩将如何使用那笔由他主观臆想的款项一样。如果说人们在作出决定时所臆想的形象往往变成这项决定的动机的话,那么,可以说斯万之所以娶奥黛特正是为了将她,将她和希尔贝特私下介绍给德·盖尔芒特公爵夫人(必要的话,永远没有别人知道)。下文中我们将看到斯万盼望妻子和女儿进入上流社会的这个唯一的雄心无法实现,并且遭到断然拒绝,因此,当斯万去世时,他以为公爵夫人将永远不会与她们结识。我们还将看到事实恰恰相反,正是在斯万去世以后开始了德·盖尔芒特夫人和奥黛特与希尔贝特的交往。他也许可以明智一些——在此暂不议论他对区区小事如此重视——无需对未来过于悲观,相信他所盼望的会见终将实现,只是他看不到这一天罢了。因果律最终能够产生几乎一切效果,包括原先被认为是不可能的效果,这个规律有时进展缓慢,由于我们的愿望——它竭力使它加快,结果适得其反——以及我们的存在本身而更加缓慢。因此,只有当我们停止希望,甚至停止生存时,它才得以实现。斯万从亲身经验中不是已经知道这一点了吗?他和奥黛特的这门婚事在他的生活中——预示在他死后将发生的事——好比是死后幸福。他曾狂热地爱她——如果说他并非一见钟情的话——而当他和她结婚时,他已不再爱她,他身上那个热切希望与奥黛特结成终身伴侣又如此绝望的人已经死去。
\par 我提到巴黎伯爵,询问他是否是斯万的朋友,因为我不愿话题从斯万身上扯开。“不错,是的。”德·诺布瓦先生转身对我说,蓝蓝的眼睛盯着我这个小人物,眼神中如鱼得水似的浮动着他巨大的工作才能和吸收能力。“哦,”他接着又对父亲说,“我给您讲一件有趣的事,这大概不算对我所敬重的亲王有所不恭吧(由于我的地位——虽然并非官方地位,我与他并无私人来往)。就在四年前,在中欧国家的一个小火车站上,亲王偶然看见了斯万夫人。当然,他的熟人中无人敢问殿下对她印象如何,那样未免太不成体统。不过,当她的名字偶尔在谈话中被提及时,人们从难以觉察但无可怀疑的迹象看出亲王对她的印象似乎不坏。”
\par “难道不可能将她介绍给巴黎伯爵?”父亲问道。
\par “咳!谁知道呢?王公们的事情难说。”德·诺布瓦先生回答道,“显贵们擅长于索取报偿,不过,有时为了酬赏某人的忠诚而甘冒天下之大不韪。显然,巴黎伯爵一直赞赏斯万的忠诚,何况斯万确实颇有风趣。”
\par “那您自己的印象呢,大使先生?”母亲出于礼节和好奇心问道。
\par 德·诺布瓦先生一反持重的常态,用行家的口吻热情地说:
\par “再好不过了!”
\par 老外交家知道,承认对某位女人产生强烈的兴趣,并且以打趣的口吻承认这一点,这便是谈话技巧中最受人赞赏的形式,因此他忽然轻声笑了起来,笑声持续片刻,他的蓝眼睛湿润了,露着红色细纤维的鼻翼在翕动。
\par “她十分迷人!”
\par “一位名叫贝戈特的作家也是座上客吗,先生?”我胆怯地问,尽量使话题围绕斯万。
\par “是的。贝戈特也在。”德·诺布瓦先生回答说,同时彬彬有礼地朝我这个方向点点头。他既然想对父亲献殷勤,便郑重其事地对待与父亲有关的一切,包括我这个年龄的(而且很少为他那个年龄的人所尊重)孩子所提的问题。“你认识他?”他用那双曾得到俾斯麦赞赏的、既深邃又明亮的眼睛凝视我。
\par “我的儿子不认识他,但十分钦佩他。”母亲说。
\par “啊呀!”德·诺布瓦先生说(他使我对自己的智力产生了最严重的怀疑,因为我所认为的世上最崇高的、比我本人珍贵千倍的东西,在他眼中却处于赞赏等级的最下层),“我可不敢苟同。贝戈特是我所称作的吹笛手。应该承认他吹得委婉动听,但是过于矫揉造作。毕竟这仅仅是吹笛,价值不大。他那些作品松松垮垮,缺乏所谓的结构。缺乏情节,或者说情节过于简单,更主要的是毫无意义。他的作品从根基上有缺陷,或者干脆说缺乏根基。在我们这个时代,生活越来越复杂,我们很少有时间看书,欧洲形势发生了深刻变化,并且也许即将发生更大的变化,我们面临各种带有威胁性的新问题,在这种时代,你们会和我一样认为作家应该是另一种人,而不是学究,因为学究热衷于对纯粹形式的优劣作空洞无用的讨论,而使我们忽略了随时都可能发生的蛮族入侵,外部和内部蛮族的双重入侵。我知道这是在亵渎那些先生所称作的‘为艺术而艺术’学派,神圣不可侵犯的学派,可是在我们这个时代,有比推敲优美文字更为紧迫的事等着我们。贝戈特的文字相当有魅力,我不否认,可是总的说来太造作,太单薄,太缺乏男子气。你对贝戈特的评价未免过高,不过我现在更理解你刚才拿出来看的那几行诗。我看不必再提它了,既然你自己也承认这只是小孩子胡写的东西(我确实说过,但心里绝不是这样想的)。对于过失,特别是年轻人的过失,要宽大为怀嘛。总之,种种过失,别人也有,在一段时期中以诗人自居的不仅仅是你。不过,你给我看的那篇东西表明你受到贝戈特的坏影响。你没学到他任何长处,我这样说想必你不会奇怪,因为他毕竟是某种风格技巧——尽管相当浮浅——的大师,而在你这个年龄是连它的皮毛也无法掌握的。但是你已经表现出和他一样的缺点——将铿锵的词句违反常理地先排列起来,然后才考虑其含意。这岂不是本末倒置吗!即使贝戈特的作品中,那些晦涩难懂的形式,颓废文人的繁琐词句又有什么意思呢?一位作家偶尔放出几支美丽的焰火,众人就立即惊呼为杰作。哪有那么多杰作呢?在贝戈特的家当中没有任何一本小说是立意颇高的成功之作,没有任何一本书值得放进书橱以引人注目。我看一本也没有。而他本人,比起作品来,更为逊色。啊!一位才子曾说人如其文,这话在他身上可真是反证。他和作品相去十万八千里。他一本正经、自命不凡、缺乏教养,有时十分平庸,和人说话时像是一本书,甚至不是他自己写的书,而是一本叫人讨厌的书(因为他的书至少不叫人讨厌),这就是那个贝戈特。这是一个杂乱无比而又过分雕琢的人,是前人所称为的浮夸者,而他说话的方式又使他说话的内容令人反感。我不记得是洛梅尼\footnote{洛梅尼(1815—1878),法国文学家。}还是圣勃夫\footnote{圣勃夫(1804—1869),法国文学家,文学批评家。}曾说过,维尼\footnote{维尼(1796—1863),法国作家,写过《桑马尔斯》及《红色封印》等小说。}也以类似的怪癖令人不快,但是贝戈特却从来没有写出像《桑马尔斯》及《红色封印》这样精彩的作品来。”
\par 德·诺布瓦先生对我刚才给他看的那段文字所作的议论令我无比沮丧,我又想起每当自己构思文章或者作严肃思考时总感到力不从心,于是我再次感到自己本是庸才,毫无文学天赋可言。往日我在贡布雷时曾有过某些微不足道的感受,曾读过贝戈特的某部作品,大概正是它们使我进入一种似乎颇有价值的遐想状态,而我的散文诗正是这种状态的反映。大使是明察秋毫的,他刚才本可以立刻抓住我在完全骗人的幻影中所找到的美,并予以揭露,然而,他没有这样做,而是让我明白我是多么微不足道(我被一位最好心的、最聪明的行家从外部进行客观评价)。我感到懊丧;自我感觉一落千丈。我的思想好似流体,其体积取决于他人提供的容量,昔日它鼓胀,将天才那支巨大容器填得满满的,今日它又缩小,骤然被德·诺布瓦先生关闭和限制在狭小的平庸之中。
\par “我和贝戈特的相识,”他又转头对父亲说,“对他,对我,都不能不说是一件尴尬的事(也是另一种方式的趣事)。几年以前,贝戈特去维也纳旅行,当时我在那里当大使。梅特涅克公主将他介绍给我,他到使馆来并希望我邀请他。既然我是法兰西的驻外使节,既然他的作品又为法兰西增光——在某种程度上,或者更确切地说,在微不足道的程度上——我当然可以抛开我对他私生活的不满。然而他并非独自旅行,所以他要求我也邀请他的女伴。我这人不爱假正经,而且,既然我没有妻室,我完全可以将使馆的门开得大一些。然而我忍受不了这种无耻,它令人作呕,因为他在作品中却大谈德行,甚至干脆教训人。他的书充满了永无休止的、甚至疲疲沓沓的分析,这是我们私下说,或者是痛苦的顾虑、病态的悔恨,以及由于鸡毛蒜皮的事而引发的冗长的说教(我们知道它值几文钱),而在另一方面,他在私生活中却如此轻浮,如此玩世不恭。总之我没有回答他。公主又来找我,我也没有答应。因此我估计此公对我不抱好感。我不知道他对斯万同时邀请我们两人的这番好意作何评价。或者是他本人向斯万提出来的,这也很难说,因为他实际上是病人。这甚至是他唯一的借口。”
\par “斯万夫人的女儿也在场吗?”我趁离开饭桌去客厅的这个机会向德·诺布瓦先生提出这个问题。这比一动不动地在饭桌上,在强烈的光线中提问更便于掩饰我的激动。
\par 德·诺布瓦先生似乎努力追忆片刻:
\par “是的,一位十四五岁的姑娘吧?不错,我记得在饭前别人把她介绍给我,说是主人的女儿。不,她露面的时间不长。她很早就去睡了,要不就是去女友家了,我记不清楚。看来你对斯万家的人很熟悉。”
\par “我常去香榭丽舍街和斯万小姐玩,她很可爱。”
\par “啊,原来如此!的确不错,我也觉得她可爱,不过,说真心话,她大概永远也比不上她母亲,这句话不至于刺伤你热烈的感情吧?”
\par “我更喜欢斯万小姐的面孔,当然我也欣赏她母亲。我常去布洛尼林园,就是为了碰见她。”
\par “啊!我要告诉她们这一切,她们会很得意的。”
\par 德·诺布瓦先生说这话时,态度与其他所有人一样(虽然为时不长)。这些人听见我说斯万是聪明人,说他父母是体面的经纪人,说他家的房子很漂亮,便以为我也会以同样的口吻来谈论同样的聪明人、同样体面的经纪人、同样漂亮的房子。其实,这好比是神经正常的人在与疯子交谈而尚未发现对方是疯子。德·诺布瓦先生认为爱看漂亮女人是理所当然的事,认为某人对你兴奋地谈起某某女士时,你便应该佯以为他堕入情网,和他打趣,并答应助他一臂之力,因此,这位要人说要向希尔贝特和她母亲谈起我(我将像奥林匹斯山的神化为一股流动的气,或者像米涅瓦\footnote{米涅瓦,罗马神话中的智慧女神。此处老者系指上文中提到的智者芒托尔。}一样化身为老者,隐身进入斯万夫人的沙龙,引起她的注意,占据她的思想,使她感谢我的赞赏,将我看做要人的朋友而邀请我,使我成为她家的密友),他将利用自己在斯万夫人眼中的崇高威信来帮助我。我突然感到无比激动,情不自禁地几乎亲吻他那双仿佛在水中浸泡过久的、泛白发皱的柔软的手。我几乎做出了这个姿势,以为觉察者仅我一人。对我们每个人来说,要对自己的言行举止在他人眼中的地位作准确判断确非易事。我们害怕自视过高,又假定人们生活中的众多回忆已经在他们身上占据极大的场地,因此我们举止言行中的次要部分几乎不可能进入谈话对方的意识之中,更不用说留在他们记忆之中了。其实,罪犯的假定也属于这同一类型。他们往往在事后修改说过的话,以为别人无法对证。然而,即使对人类千年的历史而言,预言一切都将保存的哲学可能比认为一切将被遗忘的专栏作家的哲学更为真实。在同一家巴黎报纸上,头版社论的说教者就某件大事、某部杰作,特别是某位“名噪一时”的女歌唱家写道:“十年以后有谁还记得这些呢?”而在第三版,古文学学院的报告常常谈论一件本身并不重要的事实,谈论一首写于法老时代的而且全文仍然为今人所知的、但本身并无多大价值的诗,难道不是这样吗?对短暂的人生来说,也许不完全如此。然而,几年以后,我在某人家里见到刚巧在那里做客的德·诺布瓦先生,我把他当做我所可能遇见的最有力的支持,因为他是父亲的朋友,为人宽厚、乐于助人,何况他由于职业和出身而言语谨慎,但是,这位大使刚走,就有人告诉我他曾提到以前那一次晚宴,并说他曾“看见我想亲吻他的手”。我不禁面红耳赤,德·诺布瓦先生谈论我时的语气以及他回忆的内容,使我愕然,它们与我的想象相去万里!这个“闲话”使我明白,在人的头脑中,分心、专注、记忆、遗忘,它们的比例多么出人意外,使我赞叹不已,就像我在马斯贝罗\footnote{马斯贝罗(1846—1916),埃及学专家。}的书中头一次读到人们居然掌握公元前十世纪阿苏巴尼巴尔国王邀请参加狩猎的猎手的准确名单!
\par “啊!先生,”当德·诺布瓦先生宣布将向希尔贝特和她母亲转达我的仰慕之情时,我说,“您要是这样做,您要是对斯万夫人谈起我,那我一生将感激不尽,一生将为您效劳!不过,我要告诉您,我和斯万夫人并不相识,从来没有人将我介绍给她。”
\par 我说最后这句话是唯恐对方以为我在吹嘘莫须有的交情。可是话一出口,我便感到它毫无用处,因为我那热情洋溢的感谢辞从一开始就使他降温。我看见大使脸上露出了犹疑和不满,眼中露出了下垂的、狭窄的、歪斜的目光(如同一张立体图中,代表某一面的远遁的斜线),它注视的仅仅是居于他本人身上的那位无形的对话者,而他们的谈话是在此以前一直和他交谈的先生——此处即为我——所听不见的。我原以为我那些话——尽管与我心中汹涌澎湃的感激之情相比软弱无力——可以打动德·诺布瓦先生,使他助我一臂之力(这对他轻而易举,而会令我欢欣鼓舞),但我立即意识到它的效果适得其反,甚至任何与我作对的人的恶言恶语也达不到这种效果。我们和一位陌生人交谈,愉快地交换对过路人的印象,而且看法似乎一致,认为他们庸俗,但是突然在我们和陌生人之间出现了一道病理鸿沟,因为他漫不经心地摸摸口袋说:“倒霉,我没带枪,不然他们一个也活不了。”和这种情景相仿,德·诺布瓦先生知道,结识斯万夫人,拜访她,这是再普通、再容易不过的事了,而我却视作高不可攀,其中必有巨大的难言之隐。因此,当他听见我这番话时,他认为在我所表达的貌似正常的愿望后面,一定暗藏着其他某种想法、某种可疑动机、某个以前的过失,所以至今才没有任何人愿意代我向斯万夫人致意,因为那会使她不高兴的。于是我明白他永远不会为我出这把力,他可以一年一年地每天与斯万夫人相见,也决不会——哪怕一次——提到我。不过,几天以后,他从她那里打听到我想知道的一件事,托父亲转告我。当然,他认为没有必要说明是为谁打听的。她不会知道我认识德·诺布瓦先生,也不会知道我热烈渴望去她家。也许这并不如我想象的那样倒霉。即使她知道这两点,第二点也不会增加第一点的效力,何况这个效力本身就是靠不住的,因为对奥黛特来说,既然她本人的生活和住宅引不起任何神秘的慌乱,那么,认识她并拜访她的人决不如我臆想的是什么神奇人物。要是可能的话,我真想在石头上写上我认识德·诺布瓦先生这几个字,然后将石头扔进斯万家的窗子。我认为,尽管传递方式粗野,这个信息会使女主人对我产生敬重而不是反感。其实,如果德·诺布瓦先生接受我的委托的话,它也不会有任何效果,反而引起斯万一家对我的恶感。即使我明白这一点,我也没有勇气收回这个委托(如果大使慨然允诺),没有勇气放弃乐趣(不论后果如何悲惨):即让我和我的名字在对我陌生的希尔贝特的家和生活中与她陪伴片刻。
\par 德·诺布瓦先生走后,父亲浏览报纸。我又想到拉贝玛。既然我看戏时所感到的乐趣远远少于我原先的估计,这个乐趣便要求被补充,并且无条件地吸收一切滋补。例如德·诺布瓦先生所赞扬的拉贝玛的优点,它被我一饮而尽,仿佛干旱的草地立刻吸收人们洒在上面的水一样。这时父亲将报纸递给我,指着上面一段小报道:“《淮德拉》的演出盛况空前,艺术界及批评界的名流前往观看。淮德拉的扮演者、久负盛誉的拉贝玛夫人获得她那辉煌事业中前所未有的成功。此次演出不愧为轰动戏剧界的大事,本报将作详细报道,在此只需指出,有权威的评论家一致认为,此次演出使淮德拉这个人物——拉辛笔下最美最深刻的人物之一——焕然一新,并且成为当代人有幸见到的最纯净、最杰出的艺术表演。”“最纯净、最杰出的艺术表演”,这个新概念一旦进入我的思想,便朝我在剧场中所感到的不完整的乐趣靠拢,并稍稍填补它的欠缺,而这种聚合形成了某种令人无比兴奋的东西,以致我惊呼道:“她是多么伟大的艺术家呀!”人们可能认为我这句话不完全出自内心。我们不妨想想许多作家的情况:他们对刚刚完成的作品不满意,但是如果他们谈到一篇颂扬夏多布里昂的天才的文章,或者想到某位被他们引为楷模的大艺术家(例如他们哼着贝多芬的乐曲并将其中的忧郁与自己散文中的忧郁作比较),那么,这种天才的概念会充塞了他们的头脑,因此,当他们回顾自己的作品时,也将天才的概念加之于它们,从而感到它们不再是最初的样子,甚至确信它们的价值,并会自言自语说:“毕竟不坏嘛!”然而他们并未意识到,在使他们得到最后满意的全部因素中,还有他们对夏多布里昂的美妙篇章的回忆,他们将这些篇章与自己的作品相提并论,而前者并非出自他们之手。我们不妨想想那些虽一再被情妇欺骗但仍然相信她们忠贞不渝的人吧。还有一些人时而盼望一种无法理解的幸存——例如含恨终身的丈夫想到已失去的、仍然爱着的妻子,或者艺术家想到将来可能享受的荣誉——时而盼望一种使人宽慰的虚无——因为他们回想起过失,如果没有虚无,他们在死后必须赎罪。我们再不妨想想那些旅游者,他们对每天的日程感到厌烦,但对旅行的总体美却兴奋异常。我们不妨问一问,既然各种概念共同生活于我们头脑里,那么,在使我们幸福的概念之中,有哪一个不是首先像寄生虫一样从邻近的不同概念索取自己所缺乏的力量呢?
\par 父亲不再提我的“外交官职业”,母亲似乎不太满意。我认为她感到遗憾的不是我放弃外交,而是我选择文学,因为她最关心的是用一种生活规律来约束我那喜怒无常的情绪。“别说了,”父亲大声说,“干什么事首先要有兴趣。再说他不再是孩子,他当然知道自己喜欢什么,恐怕很难改变。他明白什么是他生活中的幸福。”将来的生活幸福还是不幸福,暂且不谈,当晚我便由于父亲这番让我自己作主的话而感到烦恼。父亲突如其来的和蔼往往使我想扑过去亲吻他胡子上方红润润的脸颊,仅仅怕惹他不快我才不这样做。我好比是一位作者,他认为自己的遐想既然出于本人之手,似乎价值不大,但出版商竟然为它们挑选最上等的纸张,并且可能采用最佳字体来印刷,这不免使他惶惶然。我也一样,我问自己我的写作愿望确实如此重要,值得父亲为此浪费这么多善意吗?他说我的兴趣不会改变,我的生活将会幸福,这些话在我身上引起两点十分痛苦的猜想。第一点就是我的生活已经开始(而我每天都以为自己站在生活的门槛上,生活仍然是完整的,第二天凌晨才开始),不仅如此,将来发生的事与过去发生的事不会有多大差别。第二点猜想(其实只是第一点的另一种形式),就是我并非处于时间之外,而是像小说人物一样受制于时间的规律,而且正因为如此,当我坐在贡布雷的柳枝棚里阅读他们的生平时,我才感到万分忧愁。从理论上说,我们知道地球在转动,但事实上我们并不觉察,我们走路时脚下的地面似乎未动,我们坦然安心地生活。生活中的时间也是如此。小说家为了使读者感到时间在流逝,不得不疯狂地拨快时针,使读者在两分钟内越过十年、二十年、三十年。在一页书的开始,我们看见的是满怀希望的情人,而在同一页的结尾,他已是八旬老翁,正步履蹒跚地在养老院的庭院里作例行的散步,而且,由于丧失了记忆,他不理睬别人。父亲刚才说“他不再是孩子,他兴趣不会变了”等等,这些话使我突然间看到时间中的我,使我感到同样的忧愁,我虽然尚不是养老院里智力衰退的老头,但仿佛已是小说中人物。作者在书的结尾用极其残酷的、冷漠的语调说:“他越来越少离开乡间,终于永远定居乡间。”等等。
\par 这时,父亲唯恐我们对客人有所指责,便抢先对妈妈说:
\par “我承认诺布瓦老头,用你的话说,有点迂腐。他刚才说对巴黎伯爵提问会不成体统,我真怕你会笑出来。”
\par “你说到哪里去了,”母亲回答说,“我很喜欢他,他地位这么高、年龄这么大,还能保持这种稚气,这说明他为人正直又颇有教养。”
\par “不错。不过,这并不影响他的机警和聪明,这一点我最清楚,他在委员会上判若两人,”父亲抬高嗓门,他很高兴德·诺布瓦先生受到母亲的赞赏,并且想证明他比她想象的还要好(因为好感往往抬高对方,揶揄往往贬低对方),“他是怎么说的……‘王公们的事情难说……’?”
\par “对,正是这样。我也注意到了,他很敏锐,显然他的生活经验很丰富。”
\par “奇怪,他居然去斯万夫人家吃饭,而且还在那里遇见了正派人,公职人员。斯万夫人是从哪里弄来这些人的呢?”
\par “你没注意他那句俏皮话吗?‘去那里的似乎主要是男士们。’”
\par 于是两人都努力追忆德·诺布瓦说这话的声调,仿佛在回想布雷桑或迪龙\footnote{布雷桑、迪龙均为著名演员。}在表演《女冒险家》\footnote{法国剧作家奥吉埃(1820—1889)的作品。}或《普瓦里埃先生的女婿》\footnote{奥吉埃与桑都合写的五幕喜剧。}时的语调。然而,诺布瓦先生的用词所受到的最高赞赏来自弗朗索瓦丝。多年以后,每当人们提起大使称她为“第一流的厨师头”时,她还“忍俊不禁”。当初母亲去厨房向她传达这个称呼时,俨然如国防部长传达来访君主在检阅后所致的祝词。我比母亲早去厨房,因为我曾请求爱好和平但狠心的弗朗索瓦丝在宰兔时不要让它太痛苦,我去厨房看看事情进行得如何。弗朗索瓦丝对我说一切顺利,干净利索:“我还从来没遇见像这样的动物。一声不吭就死了,好像是哑巴。”我对动物的语言知之甚少,便说兔子的叫声比鸡小。弗朗索瓦丝见我如此无知,愤愤然地说:“先别下结论。你得看看兔子的叫声是否真比鸡小,我看比鸡大得多哩。”弗朗索瓦丝接受德·诺布瓦先生的称赞时,神态自豪而坦然,眼神欢快而聪慧——尽管是暂时的——仿佛一位艺术家在听人谈论自己的艺术。母亲曾派她去几家大餐馆见习见习烹调手艺。那天晚上,她把最有名的餐馆称作小饭铺。我听了甚为高兴,如同我曾发现戏剧艺术家的品质等级与声誉等级并不一致时那样高兴。母亲对她说:“大使说在哪里也吃不到你做的这种冷牛肉和蛋奶酥。”弗朗索瓦丝带着谦虚而受之无愧的神情表示同意,但大使这个头衔并未使她受宠若惊。她提到德·诺布瓦先生时,用一种亲切的口吻说:“这是一个好老头,和我一样。”因为他曾称她为“头”。他来的时侯,她曾经想偷看,但是,她知道妈妈最讨厌别人在门后或窗下偷看,而且会从别的仆人或门房那里得知弗朗索瓦丝偷看过(弗朗索瓦丝看见处处是“嫉妒”和“闲言碎语”,它们之作用于她的想象力,正如耶稣会或犹太人的阴谋之作用于某些人的想象力:这是一种无时无刻不在的、不祥的作用)因此她只是隔着厨房的窗瞟了一眼,“免得向太太解释”,而且,当她看见德·诺布瓦先生的大致模样和“灵巧”的姿势时,她“真以为是勒格朗丹先生”,其实这两个人毫无共同之处。“谁也做不出你这样可口的冻汁来(当你肯做的时侯),这来自什么原因?”母亲问她。“我也不知道这是从哪里变来的。”弗朗索瓦丝说(她不清楚动词“来”——至少它的某些用法——和动词“变来”究竟有什么区别)。她这话有一部分是真实的,因为她不善于——或者不愿意——揭示她的冻汁或奶油的成功诀窍,正如一位雍容高雅的女士之与自己的装束,或者著名歌唱家之与自己的歌喉。她们的解释往往使我们不得要领。我们的厨娘对烹调也是如此。在谈到大餐厅时,她说:“他们的火太急,又将菜分开烧。牛肉必须像海绵一样烂,才能吸收全部汤汁。不过,以前有一家咖啡店菜烧得不错。我不是说他们做的冻汁和我的完全一样,不过他们也是文火烧的,蛋奶酥里也确实有奶油。”“是亨利饭馆吧?”已经来到我们身边的父亲问道,他很欣赏该隆广场的这家饭馆,经常和同行去那里聚餐。“啊,不是!”弗朗索瓦丝说,柔和的声音暗藏着深深的蔑视,“我说的是小饭馆。亨利饭馆当然高级啦,不过它不是饭馆,而是……汤铺!”“那么是韦伯饭馆?”“啊,不是,我是指好饭馆。韦伯饭馆在王家街,它不算饭馆,是酒店。我不知道他们是否侍候客人用餐,我想他们连桌布也没有。什么都往桌子上一放,马马虎虎。”“是西罗饭馆?”弗朗索瓦丝微微一笑,“啊,那里嘛,就风味来说,我看主要是上流社会的女士(对弗朗索瓦丝来说,上流社会是指交际花之流)。当然哪,年轻人需要这些。”我们发觉弗朗索瓦丝虽然神情纯朴,对名厨师来说却是令人畏惧的“同行”,与最好嫉妒的、自命不凡的女演员相比,她毫不逊色。但我们感到她对自己这门手艺有正确的态度,她尊重传统,因为她又说:“不,我说的那家饭馆以前能做出几道大众喜欢的可口菜。现在的门面也不小。以前生意可好了,赚了不少的苏(勤俭的弗朗索瓦丝是以‘苏’来计算钱财的,不像败家子以‘路易’来计算)。太太认识这家饭馆,在大马路上,靠右手,稍稍靠后……”她以这种公允——夹杂着骄傲和纯真——的口吻谈到的饭馆,就是……英吉利咖啡馆。
\par 元旦来到了。我和妈妈去拜访亲戚。她怕累着我,事先就按照爸爸画的路线图将要去的人家按地区、而不是按亲疏的血缘关系分成几批。我们去拜访一位远房表亲(她住得离我们不远,所以作为起点),可是我们一踏进客厅,母亲便惊慌不安,因为一位好生疑心的叔叔的好友正在那里吃冰糖栗子或果仁夹心栗子,他肯定会告诉叔叔我们最先拜访的不是他,而叔叔的自尊心会受到伤害,因为他认为我们自然应该从玛德莱娜教堂到他住的植物园,然后是奥古斯坦街,最后再远征医学院街。
\par 拜访结束以后(外祖母免除了我们的拜访,因为那天我们要去她那里吃饭),我一直跑到香榭丽舍大街那家商店,请女老板将一封信转交每星期来买几次香料蜜糖面包的斯万家的仆人。自从希尔贝特使我十分难过的那一天起,我就决定在元旦给她写信,告诉她我们旧日的友谊与过去的一年一同结束了,我的抱怨和失望已成往事。从元月一日起,我们要建立一种崭新的友谊,它将异常牢固,任何东西也无法摧毁,它将十分美好,我希望希尔贝特殷勤照料它,使它永葆美丽,而且,万一出现任何威胁它的危险时,她必须及时告诉我,正如我答应要告诉她一样。在回家的路上,弗朗索瓦丝让我在王家街的拐角上停下,那里有一个露天小摊,她挑了几张庇护九世和拉斯巴耶\footnote{庇护九世为罗马教皇;拉斯巴耶(1794—1878)为法国著名记者及政治家。}的照片作为新年礼物,而我呢,我买了一张拉贝玛的照片。女演员的这张唯一的面孔,与她所引起的形形色色的赞誉相比,似乎显得贫乏,它像缺乏换洗衣服的人身上的衣服一样,一成不变而又无法持久。上嘴唇上方的那个小皱纹、扬起的眉毛,以及其他某些生理特征,它们总是一成不变,而且随时有被烧和被撞的危险。单凭这张面孔并不使我感到美,但我却产生了亲吻它的念头和欲望,因为它一定接受过无数亲吻,还因为它在“照片卡”上似乎用卖弄风情的温柔眼光及故作天真的微笑在召唤我。拉贝玛一定对许多年轻人怀有她在淮德拉这个人物的掩饰下所供认的种种欲念,而一切——甚至包括为她增添美丽,使她永葆青春的显赫声誉——能使她轻而易举地满足欲望。黄昏降临,我在剧场海报圆柱前停住,观看关于拉贝玛元月一日演出的海报。微风湿润而轻柔,这种天气我十分熟悉。我感到、预感到,元旦这一天和别的日子并无区别,它并非新世界的第一天——在那个新世界里,我将有机会重新认识希尔贝特,如同创世时期那样,仿佛过去的事都未发生,仿佛她有时使我产生的失望及其预示未来的迹象统统不存在了。在那个新世界中,旧世界的一切消失得无影无踪……除了一点:我希望得到希尔贝特的爱。我明白,既然我的心希望在它周围重建那个未曾使它得到满足的世界,那就是说我的心并未改变,因为我想希尔贝特的心也不可能改变。我感到新友谊与旧友谊并无区别,正如新年和旧年之间并不隔着一道鸿沟。我们的愿望既无法支配又无法改变岁月,只好在岁月毫无所知的情况下对它换一个称呼。我想将新的一年献给希尔贝特,将我对元旦的特殊想法刻印在元旦这一天上——好比将宗教重叠于盲目的大自然规律之上——但这都是徒劳和枉然。我感到它并不知道人们称它元旦,它像我所习惯的那样在黄昏中结束。微风吹着广告圆柱,我认出,我又感到往昔时光的那共同的永恒物质,它那熟悉的湿气和它那懵懂无知的流动性。
\par 我回到家中,我刚刚度过了老年人的元旦;老年人与年轻人的不同,不仅仅在于他们得不到新年礼物,而是在于他们不再相信新年。新年礼物,我倒是收到一些,但没有那件唯一能使我高兴的礼物——希尔贝特的信。不过,我毕竟还很年轻,我居然给她写了一封信,向她讲述我孤独的热情之梦,希望引起她的共鸣,而衰老的人们的可悲处在于他们根本不会写这种信,因为他们早已知道毫无用处。
\par 我躺下了,街道上一直持续到深夜的节日喧嚣使我无法入睡。我想到所有将在欢乐中度过这一夜的人,想到拉贝玛的情人或者那一群放荡者,他们一定在演出(即我在海报上看见的当晚的演出)以后去找拉贝玛。这个想法使我在不眠之夜更为激动不安,为了恢复镇静,我想对自己说拉贝玛也许并未想到爱情,但我说不出口,因为她所朗诵的仔细推敲的诗句,显然处处提醒她爱情是多么美妙,而她也深有感受,所以才表演出人所熟知的——但具有新威力和意想不到的柔情——慌乱心情而使观众赞叹不已,其实每位观众对此都有切身体会。我点燃熄灭的蜡烛,好再看看她的面孔。此刻它大概正被男人们亲抚,他们给予她并从她那里得到非凡而模糊的快乐(而我无法阻拦),这个臆想使我产生一种比色情更为残酷的激动,一种思念,它在号声(如同狂欢之夜及其他节日之夜里往往听到的号声)中更显得深沉;号声来自一家小酒店,毫无诗意,因而比“傍晚,在树林深处……”\footnote{法国诗人维尼(1797—1863)的诗《号角》。}更为忧郁。此时此刻,希尔贝特的信也许不是我所需要的。在紊乱的生活中人们的种种愿望互相干扰,因此,幸福很少降临在恰恰渴望它的愿望之上。
\par 天气晴朗时,我仍然去香榭丽舍大街。街旁那些精致的粉红色房屋展现在多变而轻盈的天空之下,因为当时水彩画展览风靡一时。如果我说当时我就认为加布里埃尔\footnote{加布里埃尔(1698—1782),著名建筑师,此处所指的建筑修建于十八世纪下半叶。}的建筑比四周的建筑更美,而且属于不同时代,那这是撒谎。我那时认为工业大厦,至少特罗卡德罗宫\footnote{工业大厦是为1855年博览会修建的;特罗卡德罗宫是为1878年博览会修建的,两者皆已拆毁。}更具特色,也许更为悠久。我的少年时光浸沉在激荡不定的睡眠之中,因此它在睡眠中所见到的这整个街区都仿佛是梦幻,我从未想到王家街居然有一座十八世纪的建筑。如果我得知路易十四时代的杰作圣马丁门和圣德尼门与这些肮脏街区里最新的建筑属于不同时期,那我会大吃一惊。加布里埃尔的建筑只有一次使我凝视良久,那时夜幕已经降临,圆柱在月光下失去了物质感的轮廓,仿佛是纸板,使我想到轻歌剧《俄耳浦斯游地狱》\footnote{作曲家奥芬巴赫的两幕四场轻歌剧。}中的布景,使我第一次感受到美。



\paragraph*{2}

\par 希尔贝特一直未回到香榭丽舍大街,而我需要看见她,因为,甚至她的面貌我也记不清了。我们以一种探索的、焦虑的、苛求的态度去看我们所爱的人,我们等待那句使我们对第二天的约会抱有希望或不再抱希望的话语,而在这句话来到以前,我们或同时或轮流地想象欢乐和失望,正因为如此,当我们面对所爱的人时,我们的注意力战战兢兢,无法对她(他)获得一个清晰的形象。这是一种由各种感官同时进行的、但又仅仅是试图通过视力来认识视力以外的东西的活动,它对一个活生生的人的千种形式、味道和运动也许过于宽容。的确,当我们不爱某人时,我们往往使她(他)静止。我们所珍爱的模特儿时时在动。我们的记忆中永远只有拍坏了的照片。我的确忘记了希尔贝特的面貌,除了她向我舒展笑颜的那神奇的瞬间——因为我只记得她的微笑。既然见不到那张亲爱的面孔,我便极力回忆,但也枉然,我恼怒地找到两张无用而惊人的面孔,它们精确之极地刻在我的记忆中:管木马的男人和卖麦芽糖的女贩。一个人失去了亲爱者,连在梦中也永远见不到她(他),却接连不断地梦见那么多讨厌鬼,更觉气恼,因为清醒时看见他们就已经难以容忍了。既然没有能力描绘痛苦思念的对象,人们便谴责自己不感觉痛苦。我也如此,既然我想不起希尔贝特的面貌,我几乎相信我忘记了有她这个人,我不再爱她。
\par 她终于回来了,几乎天天和我一起玩。我每天都希望明天能获得——从她那里获得——新东西。从这个意义上讲,我的爱情在日日更新。但突然又有一件事改变了每日下午两点钟我的爱情方式。是斯万先生发现了我写给他女儿的信,还是希尔贝特为了让我多加提防才将早已存在的情况告诉我呢?有一次,我对她说我十分钦佩她的双亲,她露出一种含糊的、有保留的、秘密的神气——在谈到她该做什么、买什么、拜访什么人时,她常常是这种神气——突然说:“你知道,他们可看不上你!”然后像滑溜溜的水精一样(这是她的习惯)大笑起来。她的笑声往往与话语极不协调,像音乐一样在另一平面勾画出另一个看不见的表层。斯万先生和夫人没有要求希尔贝特不再和我玩耍,但他们希望——她认为——这件事根本没有发生。他们不喜欢她和我来往,认为我品德不高尚,对他们的女儿只能产生坏影响。斯万认为我属于那类厚颜无耻的青年。在他的概念中,这种人憎恶自己所爱恋的少女的父母;虽然当面大献殷勤,背后却和她一起嘲笑他们,怂恿她将他们的话当耳边风,而等少女到手以后,甚至不许再与父母见面。与此种形象(最可鄙的人也决不会这样看待自己的)形成鲜明对比的,是我心中的感情。我对斯万充满了强烈的感情,我相信,如果他稍有觉察,定会懊悔对我判断失误,仿佛这是一桩错案!我大着胆子将我对他的这番感情写进一封长信,请希尔贝特转交给他。她答应了。可是,唉!出我意料,他竟以为我是一个更大的伪君子。我在十六页信纸中如此真实描述的感情竟受到他的怀疑。我那封热情而真诚的信,如同我对德·诺布瓦先生所讲的热情而真诚的话一样,毫无效果。第二天,希尔贝特将我领到小径上一大丛月桂树后面,那里很僻静,我们每人挑一张椅子坐下,她告诉我她父亲看信时耸肩说:“这一切毫无意义,反而证明我看得准。”我自信动机纯洁、心地善良,因此更为恼怒。我的话居然未触及斯万的荒谬错误的一根毫毛!他当然是错误的,我深信不疑。既然我对自己的慷慨感情的某些不容置疑的特点作了如此精确的描述,而斯万仍然不能立即根据这些特点来辨认我的感情并请求我宽恕他的错误,那么一定是因为他本人从未体验过如此崇高的感情,所以也无法理解别人会有这种感情。
\par 也许仅仅因为斯万知道慷慨只是我们自私的感情在未被分类定名以前所经常采取的内部形式,也许他认为我对他的好感只是我对希尔贝特的爱情的简单效果(及热情的肯定),而我将来的一切行为将不可避免地取决于这个爱情,而不取决于由此派生的我对他的崇拜。我不可能同意他的预言,因为我还不能将我的爱情与自我分开,还不能从实验的角度估计后果。我灰心失望。我得离开希尔贝特片刻,因为弗朗索瓦丝在叫我。我得陪她去那间带有绿色金属网纱的小亭,它很像废置不用的、老巴黎征收入市税的哨亭,不久以前在它的内部修建了英国人称作的盥洗室,而法国人一知半解地追求英国时髦,称它为“瓦泰尔克洛泽”\footnote{即英文Water-Closet的法语发音。}。我在门廊里等待弗朗索瓦丝,潮湿而陈旧的墙壁散发出清凉的霉味,使我立刻将希尔贝特转达的斯万的话所带来的忧虑抛在脑后,并使我充满了乐趣,这不是那种使我们更不稳定的、难以被我们挽留和驾驭的乐趣,而是一种相反的、我可以信赖的、牢固的乐趣,它美妙、温静、包含丰富而恒久的真实,它未被说明,但确凿无疑。我真希望像往日到盖尔芒特那儿去散步一样,努力探求这种强烈感受的魅力,一动不动地待在那里去审询这古老的气息,它邀请我深入它未揭示的真实之中,而不要我享受它附加给我的乐趣。可就在此刻,小亭子的老板娘,一位满脸脂粉、戴着红棕色假发的老妇对我说话了。弗朗索瓦丝说她“家庭蛮不错”,因为她的女儿嫁给了弗朗索瓦丝所称作的“富家子弟”,他与工人有天壤之别,正如圣西门认为公爵与“出身下层”的人有天壤之别一样。当然,这位老板娘在干这一行以前大概命运多舛,但弗朗索瓦丝肯定说她是侯爵夫人,属于圣费雷奥家族。这位侯爵夫人叫我别待在凉处,甚至为我打开一扇门说:“您不想进去?这间很干净。不用给钱。”她这样做也许是和古阿施糖果店的小姐一样。每次我们去订东西,她们总是从柜台上的玻璃罩下面取出一块糖递给我,可惜妈妈不许我接受。她也许还像那位卖花的、别有用心的老妇人,当妈妈为“花坛”挑选鲜花时,这位女人一面给我送秋波,一面递给我一枝玫瑰花。总之,如果说“侯爵夫人”喜欢男童,向他们打开男人们像狮身人面像一样蹲着的石墓小间的门的话,那么,她在这种慷慨之举中寻求的不是腐蚀的尝试,而是寻求向所爱者乐善好施而不图回报的乐趣,因此,我在她那里从未见过别的主顾,只有一个年老的公园看守。
\par 片刻以后,我和弗朗索瓦丝一起向“侯爵夫人”告别,然后我又离开弗朗索瓦丝去找希尔贝特。我发现她正坐在月桂花丛后面的椅子上。这是为了不被她的同伴看见,她们正在玩捉迷藏。我走去坐在她身旁。她将头上的软帽拉得很低,几乎遮住了眼睛,仿佛在“窥视”。我第一次在贡布雷看见她时,她就是这种梦幻的、狡猾的眼神。我问她有没有办法让我和她父亲当面谈谈。她说她曾向父亲提过,但他认为毫无必要。“拿着,”她接着说,“拿走你的信,我得去找同伴了,既然她们找不到我。”
\par 如果此时此刻,在我尚未拿到信(如此诚恳的信居然未能说服斯万,简直不可思议)以前,斯万突然来到,我也许会看到他的话不幸而言中。希尔贝特在椅子上仰着身子,叫我接信却不递给我,于是我凑近她,我感到她身体的强烈吸引力,我说:
\par “来,你别让我抢着,看看谁厉害。”
\par 她把信藏在背后,我的手掀起她垂在两肩的发辫,伸到她颈后。她披着垂肩的发辫,也许因为这适合她的年龄,也许因为母亲想延长女儿的童年,好使自己显得年轻。我们搏斗起来,弓着身子。我要把她拉过来,她在抵抗。她那张由于用力而发热的脸颊像樱桃一样又红又圆,她笑着,仿佛我在胳肢她。我将她紧紧夹在两腿之间,好似想攀登一株小树。在这场搏斗之中,我的气喘主要来自肌肉运动和游戏热情,如同因体力消耗而洒出汗珠一样,我洒出了我的乐趣,甚至来不及歇息片刻以品尝它的滋味。我立刻将信抢了过来。于是,希尔贝特和气地对我说:
\par “你知道,你要是愿意,我们可以再搏斗一会儿。”
\par 也许她朦胧地感到我玩这个游戏有另一层未言明的目的,不过她没有看出我的目的已经达到。我唯恐她有所觉察(片刻以后她作了一个廉耻心受到冒犯的、收缩而克制的动作,可见我的害怕不无道理),便答应继续玩搏斗,免得她认为我并无其他目的,而信既已抢到手,我便只想安安静静地待着。
\par 在回家的路上我突然看出,突然想起,那间带金属网纱的小亭子的凉爽、略带烟炱味的气息使我接近了一个在此以前隐藏的形象,而并未使我看到它或识辨它。这个形象便是阿道夫叔公在贡布雷的那间小房,它也散发同样的潮气。然而对这样一个无足轻重的形象的回忆何以使我如此快乐,我不明白,暂时也不想弄明白。此时,我感到德·诺布瓦先生对我的蔑视的确有理,一来我所认为的作家中的佼佼者在他看来仅仅是“吹笛手”,二来我所感受的真正的激情不是出自某个重要思想,而是出自一种霉味。
\par 一段时间以来,在某些家庭中,每当客人提到香榭丽舍大街这个名字,母亲们便露出不以为然的神气,仿佛站在她们面前的是一位著名的医生,而她们曾多次见他误诊,因此无法再信任他。据说香榭丽舍公园对儿童不吉利,不止一次孩子嗓子疼,出麻疹,许多孩子发烧。妈妈的几位女友见她继续让我去香榭丽舍大惑不解,她们虽然没有对她的母爱表示公开怀疑,但至少对她的轻率感到惋惜。
\par 神经过敏者也许是极少“倾听内心”的人,虽然这和一般的看法相反。他们在自己身上听见许多东西,后来发觉不该大惊小怪,从此便听而不闻。他们的神经系统往往大喊“救命!”仿佛生命垂危,其实仅仅是因为天要下雪或者他们要搬家,久而久之,他们习惯于对警告一概不予理睬,就好比一位奄奄一息的士兵在战斗热情的驱使下,对警告置之不理,继续像健康人一样生活几天。有一天,我带着惯常的种种不适的感觉(我对它们持续的内部循环与对血液循环一样,始终不予理睬),轻快地跑进饭厅,父母已坐在餐桌旁了,于是我也坐下——我像往常一样对自己说,发冷也许并不意味着应该取暖,而是因为受到呵责;不感到饥饿表示天要下雨,而并不表示不需进食——可是,当我咽下第一口美味牛排时,一阵恶心和眩晕使我停下来,这是刚刚开始的病痛的焦躁的回答。我用冷冰冰的无动于衷以掩盖和推迟病兆,但疾病却顽固地拒绝食物,使我无法下咽。这时,在同一瞬间,我想到如果别人发现我病了便不会让我出门,这个念头(像伤员的本能一样)给予我勇气,我蹒跚地回到卧室,量出我高烧四十度,然后收拾打扮一下便去香榭丽舍大街。虽然我的肉体表层有气无力、十分虚弱,但我的思想却笑吟吟地催我奔往和追求与希尔贝特玩捉人游戏的甜蜜快乐。一小时以后,我的身体支持不住了,但仍然感到在她身边的幸福,仍然有力量来享受快乐。
\par 一到家,弗朗索瓦丝便对众人说我“身体不舒服”,肯定是得了“冷热病”。并马上请来了医生。医生宣称,“倾向于”肺充血所引起的“极度的”和“病毒性”的高烧,它仅仅是“一把稻草火”,将转化为更“阴险”、更“潜在”的形式。很久以来我感到窒息,外祖母认为我酒精中毒,可是医生不顾她的反对,劝我在快发病时除了服用疏畅呼吸的咖啡因以外,适当喝点啤酒、香槟酒或白兰地酒。他说酒精所引起的“欣慰现象”会防止哮喘发作。因此,为了向外祖母讨酒,我无法隐瞒,而是不得不尽量显示我呼吸困难。每当我感到即将犯病,而对病情又无法预料时,便忧心忡忡,我的身体——也许因为太虚弱而无力独自承担疾病的秘密,也许因为害怕别人不知我即将发病而要求做某些力所不及的或者危险的事——使我感到,必须将我的不适精确地告诉外祖母,而这种精确性最后变成一种生理性的需要。每当我在自己身上发现一种尚未识辨的症状时,我必须告诉外祖母,否则我的身体会惶惶不安。如果她假装不理睬,那么我的身体会令我坚持到底。有时我走得太远,于是,在那张不再像往日一样能克制自己的、亲爱的面孔上,出现怜惜的表情和痛苦的挛缩。见她如此痛苦,我十分难受,便扑到她怀中,仿佛我的亲吻能够抹去她的痛苦,我的爱能够像我的幸福一样使她欢悦。既然她已确知我如何不适,我便如释重负,我的身体也不再反对我去安慰她。我再三说这种不适并不痛苦,她完全不用可怜我,我向她保证说我是快乐的,我的身体只是想得到它所应该得到的怜惜,只要别人知道它右边疼痛就够了,它并不反对我说这疼痛不算病因而不能构成对我的快乐的障碍,它并不以哲学自炫,哲学与它无缘。在痊愈之前,几乎每天我的窒息都要发作几次。一天晚上,外祖母离开我时我还平安无事,可是她在夜深时又来看我,却见我呼吸急促,她大惊失色地叫道:“啊!我的天,你多受罪呀!”她马上走了出去,大门一阵响动,不久她便拿着刚出去买的白兰地酒进来,因家里没有酒了。很快我便感到轻松。外祖母脸色微红,神情不大自在,目光中流露出疲乏和气馁。
\par “我还是走开,让你轻松轻松吧。”她说,并且突然离开我,但我仍然亲吻了她并且感到她那清新的面颊有点湿润,莫非这是她刚才穿越的黑夜空气所留下的湿气?我无从得知。第二天,一直到天黑她才来到我的卧室,据说她白天不得不出门。我觉得她在对我表示冷淡,但我克制自己不去责备她。
\par 充血的毛病早已痊愈,但我继续感到窒息,这是什么原因呢?于是父母请来了戈达尔教授。对这种情况下被请的医生来说,仅仅有学问是不够的。他面对的症状可能属于三四种不同的疾病,最终要靠他的嗅觉和眼力来判断是哪一种病,虽然表象几乎相同。这种神秘的天赋并不意味着在别的方面具有超群的智力。一个喜欢最拙劣的绘画、最拙劣的音乐、没有任何精神追求的、俗不可耐的人也完全可以具有这个天赋。就我的情况而言,他所观察到的具体症状可能有多种起因:神经性痉挛、刚刚开始的肺结核、哮喘、伴以肾功能不全的肠道毒素性呼吸困难、慢性支气管炎,或者由这其中好几个因素构成的综合症,对付神经性痉挛的办法是别把它当回事,而对付肺结核则必须精细从事,采取过度饮食疗法,而过度饮食对哮喘之类的关节性疾病十分不利,肠道毒素性呼吸困难则极端危险,而肠道毒素性呼吸困难所要求的饮食对肺结核病人来说又是致命的。然而,戈达尔只犹豫片刻便以不容反驳的口气宣布处方:“大泻强泻。几天以内只能喝奶。禁肉。禁酒。”母亲喃喃说我急需滋补,我已经相当神经质了,这种大泻和饮食会使我垮掉的。戈达尔的眼神焦虑不安,仿佛害怕误了火车,我看出来他在自问刚才的话是否过于出自他温顺的天性,他在努力回顾刚才是否忘记戴上冰冷的面具(仿佛人们寻找镜子来看看是否忘了打领带)。他心存疑虑,想稍加弥补,便粗声粗气地说:“我一向不重复处方。给我一支笔。只能喝牛奶。等我们解决了呼吸困难和失眠以后,你可以喝汤,我不反对再吃点土豆泥,不过一直要喝奶,喝奶。这会使你高兴的,既然现在西班牙最时髦,啊莱!啊莱!\footnote{西班牙语,斗牛时高呼的“加油”,按谐音为法语的“喝奶”,此为同音异意的文字游戏。}(他的学生很熟悉这个文字游戏,因为每次当他在医院里嘱咐心脏病人或肝病人以牛奶为主食时,他总是这样说。)然后你可以逐渐恢复正常生活。不过,只要再出现咳嗽和窒息,你就再来一遍:泻药、洗肠、卧床、牛奶。”他冷冷听着母亲最后的反对意见,不予理睬,不屑于解释为什么采取这种疗法便告辞而去。父母认为这种疗法不仅治不了我的病,而且无谓地大伤我的元气,因此不让我试用。当然他们尽量不让教授知道没有按他的话去做,而且,为了万无一失,凡是可能与教授相遇的社交场所,他们一概不去。后来,我的病情日趋严重,他们才决定不折不扣地执行戈达尔的处方。三天以后,我便不再气喘,不再咳嗽,呼吸也通畅了。于是我们明白,戈达尔看出我的主要病因是中毒(虽然他后来说,他认为我也有哮喘,特别是有点“疯癫”)。他冲洗我的肝和肾,使我的支气管畅通无阻,从而使我恢复呼吸、睡眠和精力。于是我们明白这个傻瓜是一位了不起的医生。我终于起床了。但是他们不再让我去香榭丽舍大街玩耍,据说那里空气不好。我认为这只是不让我见到斯万小姐的借口,所以我强迫自己时时刻刻念着希尔贝特的名字,就像是被俘者努力保持母语,以免忘记他们将永远不能重见的祖国。母亲有时用手摸着我的额头说:
\par “怎么,小儿子不再把烦恼告诉妈妈了?”
\par 弗朗索瓦丝每天走近我时说:“瞧瞧先生的气色!您没照镜子吧,像死人!”如果我只是得了感冒,弗朗索瓦丝也会摆出同样哀怜的面孔。这种忧伤更多地由于她的“等级”,而并非由于我的病情。当时我分辨不出弗朗索瓦丝的这种悲观是痛苦还是满足,我暂时认为它具有社会性及职业性。
\par 有一天,邮递员来过以后,母亲将一封信放在我床上。我将信拆开,漫不经心,因为它里面不可能有唯一能使我快乐的签名——希尔贝特的签名,我和她除了在香榭丽舍大街见面以外没有任何来往。在信纸的下方有一个银色印章,里面是一位戴着头盔的骑士以及下面排成圆形的格言Pre viam rectam拉丁文,意即:正直无欺。。信中的字体粗大,每一句话似乎都用了加强号,因为“t”字母上的横道不是划在中间,而是划在上面,等于在上一行对应的字下面划了一道。在信的下方我看到的正是希尔贝特的签名。不过,既然我认为在我收到的信中不可能有她的签名,我不相信我的眼睛,也未感到欣喜。霎时间,这个签名使我周围的一切失去真实性。这个不可思议的签名以令人目眩的速度与我的床、壁炉、墙壁玩四角游戏。我眼前的一切摇晃起来,仿佛我从马背上跌落下来,我在思考莫非存在另一种生活,它与我们所熟悉的生活迥然不同、甚至恰恰相反,但它却是真实的,当它突然向我显现时,我满心犹豫,仿佛雕刻家的《末日审判》中那些站在天堂门口的死而复生的人一样。信里说:“亲爱的朋友:听说你曾得了重病,并且不再来香榭丽舍了。我也不去那里,因为那里有许多病人。我的女友们每星期一和星期五来我家喝茶。妈妈让我告诉你,欢迎你病好以后来,我们可以在家里继续在香榭丽舍大街有趣的谈话。再见,亲爱的朋友,但愿你的父母能允许你常来我家喝茶。谨致问候。希尔贝特。”
\par 在阅读这封信时,我的神经系统以奇妙的敏捷性接收了信息,即我遇见了喜事。然而我的心灵,即我本人——主要的当事人——并不知晓。幸福,通过希尔贝特获得幸福,这是我一直向往的、纯粹属于思想性的事,正如莱奥纳多说绘画是Cosa mentale\footnote{意大利语,意即:思想性的事。莱奥纳多即达·芬奇(1452—1519)。}。满篇是字的信纸不能马上被思想吸收。然而当我读完信以后,我想到它,它便成为我遐想的对象,成为cosa metale,我爱不释手,每隔五分钟就得再读一遍,再亲吻一次。于是,我认识了我的幸福。
\par 生活里充满了这种爱恋者永远可以指望的奇迹。这次奇迹也可能是母亲人为地制造的,她见我最近以来感到生活索然无味,便托人请希尔贝特给我写信。我记起我头几次海水浴。那时我讨厌海水,因为我喘不过气来,母亲为了引起我对潜水的兴趣,便悄悄地让我的游泳老师将异常美丽的贝壳和珊瑚枝放在水底,让我以为是我发现它们的。何况,在生活中,在各种不同的生活情况中,凡涉及爱情的事最好不必试图理解,因为它们时而严峻无情,时而出人意料,仿佛遵循神奇的法则,而非理性的法则。一位亿万富翁——虽然有钱,但人很可爱——被与他同居的、貌不出众的穷女人所抛弃,他在绝望之际,施展金钱的全部威力和人世间一切影响以求她回心转意,但白费力气,在这种情况下,我们最好不要用逻辑来解释他的情妇为什么顽固不化,而应认为他命中注定要受到这个打击,命中注定要死于心病。情人们往往必须与障碍搏斗,他们那由于痛苦而变得极度兴奋的想象力猜测障碍在哪里,而障碍有时仅仅在于他们无法使之回心转意的女人身上的某个特殊个性,在于她的愚蠢,在于他们所不认识的某些人对她所施加的影响或她所感到的恐惧,在于她暂时对生活所要求的乐趣,而这种乐趣是情人本人或情人的财富所无法给予的。总之,情人无法了解这些障碍的性质,因为女人玩弄手腕向他隐瞒,也因为他的判断力受到爱情的蒙骗而无法进行准确评价。这些障碍好比是肿瘤,医生终于使它消退,但并不了解起因。和肿瘤一样,障碍始终神秘莫测,但却是暂时的。不过,一般说来,它们持续的时间比爱情长。既然爱情并非一种无私的激情,那么,在爱情减退以后,情人们也就不再思考为什么那位曾被自己爱过的、贫穷和轻浮的女人竟然长时间地、顽固地拒绝他的供养费。
\par 在爱情问题上,奥秘使我们看不到灾难的起因,也使我们无法理解突如其来的圆满结局(例如希尔贝特的信所带来的结局)。对这种类型的感情而言,任何满足往往只是使痛苦换一个地方,因此只能称为貌似圆满的结局,而并无真正的圆满结局可言。有时,我们得到暂时的喘息,于是在一段时间内便产生了痊愈的幻觉。
\par 弗朗索瓦丝不相信那是希尔贝特的名字,因为字母G十分花哨,倚在后面省略去一点的字母i之上,看上去像字母A,而最后的音节拉得很长,形成锯齿状的花缀。如果一定要对信中所表达的、并使我满心欢喜的这种友好态度寻找逻辑解释的话,那么也许可以说,在某种程度上应归功于这次生病(相反,我原来以为它会使我在斯万一家的思想中永远失宠)。在这以前不久,布洛克曾来看我,当时戈达尔教授正在我的卧室里(我们采用了他的饮食治疗法,便又将他请了回来)。看完病以后,戈达尔没有走,被父母挽留下来吃饭,这时布洛克走进我的卧室。我们正在聊天,布洛克说他头天晚上曾和一位女士共餐,此人与斯万夫人过从甚密。他听说斯万夫人很喜欢我,我很想说他一定弄错了,而且告诉他我并未结识斯万夫人,从未和她说过话,以澄清事实,正如我当初为了问心无愧,为了不被斯万夫人当做说谎者而对德·诺布瓦先生讲的那番话一样,然而我没有勇气纠正布洛克的错误,我明白他是故意的,他之所以臆造斯万夫人所不可能说的话正是为了表明他曾和斯万夫人的女友共同进餐(他认为这很体面,但这是虚构的)。当初,德·诺布瓦先生听说我不认识斯万夫人并且希望认识她,便拿定主意在她面前绝口不提我,而戈达尔则相反,他从布洛克的话中得知斯万夫人熟悉我并赞赏我,便打定主意下次见到她时(他是她的私人医生)要告诉她我是一个讨人喜欢的孩子,我们常有来往。这些话对我毫无益处,却能为他脸上增光,正是出于双重原因,他决定一有机会见到奥黛特时便将谈到我。
\par 于是我结识了那套房子。斯万夫人所用的香水的气味一直弥漫在楼梯上,但芳香更主要来自希尔贝特的生活所散发的特殊而痛苦的魅力。无情的看门人变成慈悲为怀的欧墨尼德斯\footnote{欧墨尼德斯,希腊悲剧《俄瑞斯忒斯》中的复仇神,后变成慈悲神。}。当我问他能否上楼时,他总是欣然地掀掀帽子,表示答应我的祈求。从外面看,窗户好似一种明亮、冷淡和浮浅的目光(正如斯万夫妇的眼神)将我与并非为我准备的室内珍宝隔开。在风和日丽的季节,我和希尔贝特整个下午待在她的房间里,有时我亲手开窗换换空气。每逢她母亲的接待日,我们甚至可以俯在窗口观看客人们到来。他们下车时往往仰起头向我招招手,把我当做女主人的某位侄子。在这种时刻,希尔贝特的发辫碰着我的脸颊。这些十分纤细(既自然又超自然)的、富有艺术性曲线的发丝,在我看来,简直是举世无双的、用天堂的青草做成的作品。最小一段发辫都值得我当天国之草供奉起来。但是我不敢有此奢望,我只想得到一张照片,它会比达·芬奇所画的小花的复制照片珍贵百倍!为了得到这样一张照片,我对斯万家的朋友、甚至对摄影师卑躬屈膝,但我并未弄到手,反而招惹了一些讨厌的人。
\par 希尔贝特的父母曾长期不允许我和她见面,而现在——我走进那阴暗的候见厅,在那里时时可能与他们相遇;如果与往日人们在凡尔赛宫觐见国王相比,这种等待更为可怕,更为急切。我在那里撞上了一个像圣经中的烛台\footnote{指圣经启示录中七个金烛台(代表七个教会)。}一般的、有七个分枝的巨大衣帽架,接着便糊里糊涂地向坐在木箱上的身穿灰色长袍的仆人致敬,因为在阴暗中我把他当做了斯万夫人——每当我去时,他们两人中的一位从那里过,便微笑着(而无丝毫不快)和我握手,并且说:“您近来可好?(他们说这句话时,从不将字母t作联诵,所以,你们可以想象,我一回家便快活地做这种取消联诵的练习)希尔贝特知道您来了吗?好,你们自己玩吧。”
\par 希尔贝特为女友们所举行的茶会长期以来似乎是使我们不断分离的、不可逾越的障碍,此刻却成为我们相聚的机会。她常常写便条通知我(因我们仍然是新交),而每次的信纸都不一样。有一次,信纸上印着一只蓝色鬈毛狗,下面有一段英文写的幽默文字,后随一个惊叹号;另一次信纸上印着一个船锚,或者是G.S.这两个字母,它们拉得很长,形成长方形占据信纸的整个上部。还有一次,在信纸一角用金色字体印着希尔贝特这个名字,仿佛是她的签名,然后是一个花缀,顶上印着一把打开的黑伞。另一次,这个名字被围在形似中国帽子的花式字体之间,所有的字母都用大写,但你一个字母也认不出来。然而,希尔贝特所拥有的信纸虽然品种繁多,但必有穷尽之时。因此过了几个星期以后,我又见到她第一封信所用的信纸,上面有一个失去光泽的银色印章、戴头盔的骑士及下方的警句。当时我以为信纸是根据某种习俗、按照不同的日期挑选的,现在看来她这样做是好记住哪些信纸她已用过,免得对通讯者——至少对她愿意讨好的人——寄去同样的信纸,即使不得不重复,也得尽量推迟重复。希尔贝特请来喝茶的女友,由于上课时间各不相同,这些人刚到,那些人就告辞,我在楼梯上就听见候见室里传出的隐约的话语声,它在我(一想到即将参加的庄严场面,我便激动万分)踏上这一层楼以前便猛然割断了我和往昔生活之间的联系,使我将走进温暖的房间该摘下围巾、看钟点,免得误了回家之类的事忘得精光。楼梯全部是木制的,在当时仿亨利二世风格的某些房屋里常见,而亨利二世风格曾是奥黛特长期追求、但不久即将抛弃的理想。楼梯口有一个牌子写着:“下楼时禁止乘电梯。”在我眼中,这楼梯如此奇妙,以致我对父母说它是斯万先生从远方运来的古物。我如此酷爱真实,即使我知道这个信息是假的,我也会毫不犹豫地告诉父母,因为只有这样才能使他们像我一样尊敬斯万家这座显贵的楼梯。这就好比在一位不知名医的天才为何物的愚昧者面前,最好不要承认这位名医治不了鼻炎。况且,我没有任何观察力,往往说不出眼前物品的称呼或类型,只知道它们既然与斯万一家有关,便不同寻常,因此,我并不认为在谈这个楼梯的艺术价值和遥远的产地时我一定在撒谎。不一定是撒谎,但很可能是撒谎,因为父亲打断我时,我脸上发红。他说:“我知道那些房子,我去看过一所,它们的结构都一样,只不过斯万家住的是好几层楼,这都是贝利埃\footnote{贝利埃(1843—1911),法国工程师。}盖的。”他还说他曾想租一套,后来放弃了,因为设计不太合理,门厅太暗。这是他的话。但是,我的本能告诉我应该为斯万家的魅力和我自己的幸福而牺牲思想,因此,我对父亲的话充耳不闻,我遵从内心的命令,将这个毁灭性思想(即斯万家住的不过是我们原先也可能住进的不足为奇的房子罢了)义无反顾地抛得远远的,正如虔诚的信徒摒弃勒南\footnote{勒南(1823—1892),法国作家,曾著《基督教发源史》,其中《耶稣传》为第一册。}所写的《耶稣传》一样。
\par 每次去喝茶时,我一级一级地爬上楼梯,来到散发着斯万夫人香水气味的地区。我已失去思维和记忆,仅仅成为条件反射的工具。我仿佛已经看见那威严的巧克力蛋糕,以及它四周那一圈盛小点心的盘子及带图案的灰色缎纹小餐巾,这都是斯万家所特有的规矩。但是这固定不变的一切,有如康德的必然世界,似乎取决于一个最高的自由行动,因为当我们都在希尔贝特的小客厅时,她突然看看钟,说道:
\par “呀,我的午餐开始消失了,晚餐得等到八点钟。我很想吃点什么。你们看怎么样?”
\par 于是她领我们走进客厅,它像伦勃朗画的亚洲庙宇内殿一样阴暗,那里有一个模仿建筑物结构的大蛋糕,它威严、温和、亲切,仿佛出于偶然、随便地耸立在那里,只等希尔贝特心血来潮去摘下它的巧克力雉堞,拆除那黄褐色的陡峭壁垒,这些陡坡是在烤炉内制造的,仿佛是大流士\footnote{大流士,古波斯国王,在位期为公元前521—485年,以显赫战功与大兴土木闻名。}宫殿中的支柱。希尔贝特不仅根据自己的饥饿程度来决定是否应该摧毁这个如尼尼微\footnote{尼尼微,古代小亚细亚王国,后被摧毁。}一般的蛋糕,她还问我饿不饿,一面从坍塌的建筑内取出嵌着鲜红果实的、闪着光泽的、具有东方风格的一大堵墙递给我。她甚至问我我父母什么时候用晚餐,仿佛我还有时间概念,仿佛我那失魂落魄的慌乱并未使饥饿的感觉、晚餐的概念、家庭的形象彻底地从我那空虚的记忆和瘫痪的肠胃中消失似的。不幸的是这种瘫痪只是暂时的。我麻木地吃蛋糕,过一会儿就该进行消化了。不过为时尚早。这时,希尔贝特递给我“我的茶”,我不停地喝着,其实一杯茶就足以使我在二十四小时内失眠。因此母亲常说:“真麻烦,这孩子,每次从斯万家回来就生病。”然而,当我在斯万家时,我明白自己喝的是茶吗?即使我明白,我也会照样喝,因为就算我在刹那间恢复了对现在的辨别能力,我也恢复不了对过去的回忆和对将来的预见。我的想象力无法达到遥远的时间——只有到那时我才能产生睡觉的念头和睡眠的需要。
\par 希尔贝特的女友们并不都处于这种无法作出理智决定的兴奋状态之中。有几位居然不喝茶!希尔贝特用当时十分流行的话说:“当然啦,我的茶不成功!”她将餐桌旁的椅子摆乱,好冲淡庄严的气氛,说道:“我们好像在庆祝婚礼似的,老天爷,这些仆人真蠢!”
\par 她侧身坐在斜靠餐桌的一张X形椅脚的椅子上啃蛋糕。片刻以后,斯万夫人送走客人——她的接待日和希尔贝特的茶会往往是同一天——便快步走了进来。她有时穿着蓝丝绒,经常穿的是饰有白色花边的黑缎裙衣。她表示诧异(仿佛女儿没有经她同意便可能有这么多小点心)地说:“噫,你们吃得多香呀,看见你们吃蛋糕,连我也馋了。”
\par “好呀,妈妈,我们请您也来。”希尔贝特回答说。
\par “哦,不行,宝贝,我的客人会怎么说呢。那儿还有特龙贝夫人、戈达尔夫人、邦当夫人,你知道,亲爱的邦当夫人从来不作短暂的访问,而她刚刚来。这些好人们看见我不回去会怎么说呢?等她们走了,要是没有新客人,我就来和你们聊天(这对我有趣得多)。我想我有权利稍稍安静一下,我已经接待了四十五位客人,而其中竟有四十二人谈到谢罗姆\footnote{谢罗姆(1824—1904),法国画家。}的画!”接着她又对我说:“您哪天来和希尔贝特喝茶,她会做您喜欢的茶,您在小工作室\footnote{原文英语,斯万夫人说话爱夹几个英文字。}里常喝的那种茶。”她一面说,一面走开去招待她的客人。她似乎认为我也意识到我走进这个神秘的世界是寻找什么习惯(即使我喝茶,那能算是有喝茶的习惯吗?至于“工作室”,我不知道自己有没有)?她又说:“您什么时候再来?明天?我们给您做toast(烤面包),味道和哥伦贝糕点店的一样。您不来?您真坏。”她自从有了沙龙,便处处模仿维尔迪兰夫人,说话带着娇嗔。不过我既未见识过toast,也未见识过哥伦贝糕点店,所以,她最后的那点许诺并未使我动心。奇怪的是,当她夸奖我家的nurse(保姆),我最初竟不知道这是指谁,其实大家都用这个词,也许如今在贡布雷仍然通用。我不懂英语,但我不久就明白她是指弗朗索瓦丝。在香榭丽舍大街,我曾担心弗朗索瓦丝给人留下不好的印象,但是我从斯万夫人口中得知,正是由于希尔贝特讲了那么多有关我的nurse的事,斯万夫妇才对我产生好感。“可以感觉到她对您忠心耿耿,她多么好。”(我立即完全改变了对弗朗索瓦丝的看法。由于反作用,我不再认为身穿雨衣头戴羽饰的家庭教师是非有不可的了。)斯万夫人禁不住议论了几句布拉当夫人,说她确实为人善良,但是她的来访令人畏惧,于是我明白她们之间的关系并不如我想象的那样对我有利,它丝毫不能改善我在斯万家中的地位。
\par 如果说我已经带着尊敬和欢乐的战栗探索这个出人意外地向我敞开大门(昔日是关闭的)的仙境的话,那么我的身份仅仅是希尔贝特的朋友。接纳我的王国本身又处于更为神秘的王国之中:斯万夫妇在那里过着超自然的生活。他们在候见厅里与我对面相遇时,与我握握手,然后又走向那个神秘的王国。但是,不久以后我也进入圣殿内部了。例如当希尔贝特不在家而斯万先生或夫人碰巧在家时,他们问谁在按门铃,听见是我便让仆人请我进去谈一谈,希望我在这方面或那方面,这件事或那件事上对他们的女儿施加影响。我回忆起以前写给斯万的那封信,它如此全面、如此具有说服力,而他竟认为不值一复。我不禁感慨起来:思想、推理、心,都没有能力导致任何交谈,没有能力解决任何困难,而生活,在你根本不知是怎么回事的情况下,却轻而易举地解决了困难。我得到希尔贝特的朋友这个新身份,有能力对她产生好影响,因此我享受优待,就好比我与国王的儿子同学,在学校中又一直名列榜首,由于这种偶然性我便可以常去王宫,并且在御座大厅谒见国王。斯万和蔼可亲地让我走进他的书房,仿佛他并不急于处理那许多光荣与体面的工作。他留了我一个小时。我过于激动,因此对他的话根本听不懂,只好结结巴巴地回答,时而胆怯地保持沉默,时而鼓起一瞬即逝的勇气,前言不搭后语地应付。他指给我看他认为会使我感兴趣的艺术品和书籍,虽然我毫不怀疑它们比卢浮宫和国立图书馆的收藏品要精美得多,但是我却看不见它们。如果他的膳食总管此刻让我将表、领带别针、高帮皮鞋都给他,并签署文件承认他为继承人的话,我也会欣然同意的,因为,用一针见血的民间俗语来说:我晕头转向(民间俗语与著名史诗一样,没有留下作者姓名,但与沃尔夫\footnote{沃尔夫(1759—1824),德国哲学家,认为史诗《伊利亚特》和《奥德赛》是各时期的史诗汇合而成。}的理论相反,它确实有过作者,那是些随时可以见到的、富有创造性的谦逊的人,正是他们发明了诸如“往一张脸上贴名字”\footnote{即记起某人的名字。}之类的说法,而他们自己的姓名却不泄露)。访问在继续,我惊奇的是在这神奇的房子里度过的时光竟然使我一无所获,没有得到任何圆满结果。我之所以失望并不是因为他给我看的杰作有任何缺陷,也不是因为我无法用漫不经心的眼光去端详它们,而是因为我坐在斯万书房中所体验的神奇感觉并非由于事物本身的内在美,而是由于附属于这些事物——它们可能是世上最丑的——之上的特殊感情,忧愁和甜蜜的感情。多年以来我便将感情寄托于这间书房,至今它仍浸透书房的每个角落。与此相仿的是另一件事。一位穿短裤的跟班对我说夫人要见见我,于是我便穿过蜿蜒曲折的走廊小道(那里充满从远处梳洗间不断飘来的珍贵的香气),去到斯万夫人的卧室,三位美丽而庄严的女人,她的第一、第二、第三侍女正微笑着为她梳妆打扮。我在那里停留片刻,自惭形秽,又对她感恩戴德,而这些感受与那一大堆镜子、银刷以及出自她的友人、著名艺术家之手的帕多瓦的圣安托万\footnote{圣安托万(1195—1231),葡萄牙传教士。}雕像或画像毫无关系。
\par 斯万夫人回到她的客人那里去,但我们仍听见她谈笑风生,因为即使她面前只有两个人,她也像面对众多“同伴”那样提高嗓门谈话,就像往日在小集团中“女主人”“引导谈话”时那样。人们喜欢——至少在一段时间内——使用新近从别人那里学来的表达法,斯万夫人也不例外,她时而使用丈夫不得不介绍她认识的高雅人士的语言(她模仿他们的矫揉造作,即在修饰人物的形容词前取消冠词或指示代词),时而又使用很俗的语言(例如她一位女友的口头禅“小事一桩”),而且尽量用于她喜欢讲述的故事中(这是她在“小集团”中养成的习惯),然后又说:“我很喜欢这个故事。”“啊!你得承认这故事很美吧!”而这种语言是她通过丈夫从她所不认识的盖尔芒特那里学到的。
\par 斯万夫人离开了饭厅,她那位刚到家的丈夫又来到我们面前。“希尔贝特,你母亲是一个人在那里吧?”“不,她还有客人,爸爸。”“怎么,还有客人,已经七点钟了!真可怕,可怜她一定累得半死。真可恶(odieux这个字我在家里也常常听见,但O发长音而斯万夫妇则发成短音)。”接着他转身对我说:“您看看,从下午两点钟起一直到现在!加米尔说在四五点钟之间,来了足足十二位客人,不,不是十二位,他说的大概是十四位,不,是十二位,我也糊涂了。我刚进来的时候,看见门口停着那么多车,我忘了是她的接待日,还以为家里在举行什么婚礼呢。我在书房里待了一会儿,门铃响个不停,闹得我真头疼。她那里客人还多吗?”“不,只两位。”“是谁?”“戈达尔夫人和邦当夫人。”“啊,公共工程部办公室主任的妻子。”“我知道她丈夫是某个部的职员,但不知道他到底干什么。”希尔贝特用孩子的口吻说。
\par “怎么,小傻瓜,你这话像两岁孩子说的。你说什么?部里的职员?他可是办公室主任,是那个单位的头头。我的天,我怎么糊涂了,跟你一样心不在焉,他不是办公室主任,他是秘书长。”
\par “我可不知道。那么说秘书长是很重要的人物了?”希尔贝特回答。她从不放弃任何机会对父母所炫耀的一切表示冷漠(她也许认为,假装不把如此显贵的朋友放在眼里会使这种关系更引人注目)。



\paragraph*{3}

\par “怎么,是不是很重要!”斯万惊呼说。他使用的不是使我疑惑茫然的语气,而是明确清楚的语言:“部长之下就是他!他甚至比部长还重要,因为凡事都要由他经办。而且据说他很有才干,是出类拔萃的第一流人才。他得过荣誉勋位四级勋章。他很有趣味,而且一表人才。”
\par 他的妻子不顾众人反对嫁给了他,因为他是“充满魅力”的人。他蓄着柔软光滑的淡黄色胡须,五官端正,说话时带鼻音,呼吸浊重,戴一只假眼,这一切足以构成罕见而微妙的整体。
\par “我告诉您,”斯万先生对我说,“这些人进入当今的政府的确是件有趣的事,他们是邦当—谢尼家族中相当典型的、教权主义的、思想狭隘的、反动的资产阶级。你那可怜的祖父对老头谢尼很熟悉,至少听说过,见过面。这老头当时很有钱,可是给车夫的小费只是一个苏。还有那位布雷奥谢尼男爵。总联合公司\footnote{此处指1876年成立的企业,1882年破产倒闭。}的股票暴跌使他们倾家荡产,您那时还太小,不知道这些事。后来,当然啦,他们竭尽全力重振家业。”
\par “他有一位外甥女,她总来我们学校上课,比我低一班,有名的‘阿尔贝蒂娜’。她将来一定很fast(放荡),现在模样有点古怪。”
\par “我女儿什么人都认识,真奇怪。”
\par “我知道她,并不相识。我只是看见她走过时,这儿有人喊阿尔贝蒂娜,那儿也有人喊阿尔贝蒂娜。不过,我认识邦当夫人,对她也没有好感。”
\par “你这就完全错了。邦当夫人很讨人喜欢,她漂亮、聪明,而且颇有风趣。我这就去向她问好,打听她丈夫对战争会不会爆发,狄奥多西国王可靠不可靠的看法。他深知诸神的隐秘,对这些事肯定了解的,对吧?”
\par 斯万以前可不是以这种口吻说话的。但是难道你没见过头脑简单的公主(她与随身男仆私奔,十年以后又想回到上流社会,但感到没人愿意与她来往)自发地像讨厌的老太婆一样说话吗?听见别人谈论一位闻名一时的公爵夫人时,她便急忙说“她昨天还来看过我哩”,或者“我现在是深居简出了”。因此我们要了解风俗,根本不需要观察,根据心理规律来推断便足够了。
\par 斯万夫妇也属于这种很少有客人来访的反常人物。稍稍有点身份的某人的来访、邀请,甚至简单一句话,对他们来说,都是应该广为宣传的大事。奥黛特举行了一次比较成功的晚宴,不巧的是维尔迪兰夫妇正在伦敦,但这个消息居然通过他们一位共同的朋友而以电报的形式传到海峡彼岸的维尔迪兰夫妇那里。就连奥黛特收到的恭维信或电报,斯万夫妇也一定让众人分享快乐。他们告诉朋友们,并让大家传阅。因此,斯万的沙龙很像是张贴着电讯新闻的海边旅馆。
\par 此外,有些人不仅像我一样认识社交生活以外的旧斯万,还认识社交生活中,特别是盖尔芒特圈子中(在那里,除了殿下和公爵夫人以外,其他人必须具有头等情趣和魅力,即使是杰出的人物,如果被认为庸俗或令人讨厌,也会被排斥出来)的旧斯万,他们要是看到斯万在谈到朋友时不再像以前那样含蓄,择友时也不再如此苛求,准会大吃一惊。像邦当夫人如此平庸、如此乖戾的人竟然不使他讨厌?他竟然说她可爱?对盖尔芒特小圈子的回忆似乎应该阻止他这样做,可实际上却促使他这样做。和四分之三的社交圈子不同,盖尔芒特小圈子是具有鉴赏能力的,甚至高雅的鉴赏力,但也有附庸风雅之习气,而它往往使鉴赏力暂时无法发挥。如果涉及的是某位并非为小集团所不可缺少的人物,例如外交部长(有点自命不凡的共和派)或某位饶舌的法兰西学院院士,那么,他会受到鉴赏力的一致否定。斯万很同情德·盖尔芒特夫人,为她不得不与这类人在某大使馆同桌吃饭。任何一位高雅之士也比他们强一千倍,所谓高雅之士是指盖尔芒特圈里的人,他一无所长,只是具有盖尔芒特精神,属于同一宗派。然而,如果某位大公夫人或王族血统公主来德·盖尔芒特夫人家吃饭的话,她会成为这宗派的一员,尽管她并无这个权利,尽管她根本不具备盖尔芒特精神。上流社会的人异常天真。既然这位贵族女士并非因可爱而被接待,而她又已经被接待了,于是人们便极力说她可爱。当殿下离去以后,斯万为盖尔芒特夫人解围说:“她毕竟不坏,甚至还不缺乏幽默感。当然,我想她并不掌握《纯粹理性的批判》,但她并不叫人讨厌。”
\par “我完全同意您的看法,”公爵夫人回答说,“她刚才稍有胆怯,将来会讨人喜欢的。”“比起那位给您列举二十本书的XJ夫人(饶舌的学院院士的夫人,颇有才华的女士)来,她叫人高兴得多。”“根本没法比。”谈论这些事,诚诚恳恳地谈论这些事,这种能力是斯万从公爵夫人那里学到的,并且保持至今,又用于他本人所接待的客人身上。他尽力去识辨他们身上的品质,而当我们怀着善意的偏见而不是带着挑剔的厌恶情绪去观察人时,人人都具有这些品质。斯万强调邦当夫人的优点正如往日强调帕尔玛公主的优点一样。如果某些贵人进入盖尔芒特小集团不是出于优待,如果人们认真考虑的果真只是情趣和魅力,那帕尔玛公主早被开除了。斯万从前也表现出这种兴趣(只是现在他持久地加以发挥而已),那就是以自己的社交地位去换取在某种情况下对自己更为合适的另一种地位。有种人在观察事物时,没有能力对乍一看来似乎不可分的事物进行分解,因此相信地位与人是连成一体的。其实同一个人,在生活的不同时期,会处于不同等级的社会阶层之中,而这等级并不一定越来越高。每当我们在生活的另一时期与某一阶层来往(或重新来往)并感到备受疼爱时,自然而然地我们便攀附于这个阶层,并在那些人中扎了根。
\par 至于邦当夫人,既然斯万一再提到她,我想他不会反对我将邦当夫人对斯万夫人的拜访告诉我父母。斯万夫人一步一步地结识了谁,父母对此颇感兴趣,但毫无赞赏之意。母亲听见特龙贝夫人的名字时说:
\par “啊!这可是位新成员,她会领些别人去的。”
\par 接着,妈妈似乎将斯万夫人广为交友的那种简便、迅速和猛烈的方式比作殖民战争说道:
\par “现在特龙贝归顺了。邻近的部落不久也会投降。”
\par 有一次她在街上遇见了斯万夫人,回家便对我们说:
\par “斯万夫人处于战争状态。她大概在对马塞诸赛人、僧伽罗人、特龙贝人发动胜利的攻势吧。”
\par 我告诉她在那个拼凑的、人为的环境中我都看见了哪些新来者(她们本属不同的社会圈子,被煞费苦心地吸引到这里来),母亲立刻猜出她们的来处,仿佛这是高价购买的战利品:
\par “这是去某某家征战的缴获品。”
\par 斯万夫人居然有兴趣吸收戈达尔夫人这位不甚高雅的小市民,父亲不禁愕然。他说:“当然,教授是有地位的人,但我仍然不明白她是怎么想的。”可是,母亲却很明白。她知道,当一个女人走进与原先的生活截然不同的圈子时,会感到愉快,如果她不能让旧友们知道如今的新交是多么体面的人物,这种乐趣会大为减色。要做到这一点就必须让一位见证人钻进美好的新圈子,仿佛一只嗡嗡叫的、见异思迁的昆虫钻进花丛,然后,见证人在每次拜访以后便散布(至少人们希望如此)消息,暗暗播下羡慕和赞赏的种子。戈达尔夫人正适合于这种角色,她是特殊类型的客人,妈妈(她继承外祖父的某种气质)称之为“异乡人,去告诉斯巴达”斯巴达国王莱翁里达斯及三百士兵为阻挡波斯人进攻而全部战死(公元前80年)。在昔日战场的岩石上刻着这句话:“异乡人,去告诉斯巴达,我们为它而死!”型的客人。此外——除了另一个多年以后才为人所知的理由以外——斯万夫人在“接待日”邀请这位和蔼的、稳重的、谦虚的女友,至少不必担心她是叛徒或竞争对手。斯万夫人知道,这位戴着羽饰、拿着名片夹的积极的工蜂,一个下午便能拜访为数众多的市民花萼。斯万夫人了解她的扩散能力,并且,根据对或然率的计算,她有把握让维尔迪兰家的某位常客第三天就得知巴黎地方长官常去斯万夫人家留下名片,或者让维尔迪兰先生本人知道赛马会主席勒奥·德·普雷萨尼先生常带领她和斯万参加狄奥多西国王的盛会。她认为维尔迪兰夫妇只会获悉这两件对她很光彩的事,仅仅这两件事,因为我们所臆想和追求的光荣往往具有很少几种特殊表现形式,这应归咎于我们的精神缺陷——它没有能力同时想象我们所期望(大致期望)于光荣的一切同步的表现形式。
\par 斯万夫人只是在所谓“官界”中获得成功。高雅女士不与她来往,但这并不是因为她那里有共和派名流。在我年幼时,凡属于保守社会的一切均成为社交风尚,因此,一个有名望的沙龙是决不接待共和分子的。对这种沙龙的人来说,永远不可能接待“机会主义者”,更不用说可怕的“激进分子”了,而这种不可能性将像油灯和公共马车一样永世长存。然而,社会好似一个万花筒,它有时转动,将曾被认为一成不变的因素连续进行新的排列,从而构成新的图景。在我初领圣体的那年以前,高雅的犹太女士便已出入社交场合从而使正统派的女士们吃惊。万花筒中的新布局产生于哲学家称作的标准所发生的变化。后来,在我开始拜访斯万夫人家以后不久,德雷福斯事件产生了一个新标准,于是万花筒再一次将其中彩色的菱形小块翻倒过来。凡属犹太人的一切都落到万花筒的底部,连高雅女士也不例外,而取而代之的是无名的民族主义者。当时,在巴黎最负盛名的沙龙是一位极端天主教徒——奥地利亲王的沙龙。如果发生的不是德雷福斯事件,而是对德战争,那么,万花筒会朝相反的方向转动,犹太人会表现爱国热忱而使众人吃惊,他们会保持自己的地位,那样一来,就再没有人愿意去拜访奥地利亲王,甚至没有人承认去拜访过。虽然如此,每当社会暂时处于静止状态时,生活于其中的人总是认为不可能再发生任何变化,正如他们看到电话问世,便认为不可能再出现飞机,与此同时,新闻界的哲学家们对前一时期进行抨击,他们不但批评前一时期中人们的乐趣,斥之为腐朽已极,甚至还抨击艺术家和哲学家的作品,斥之为毫无价值,仿佛它们与附庸风雅、轻浮浅薄的各种表现形式密不可分。唯一不变的似乎是每次人们都说“法国发生了一点变化”。我初去斯万夫人家时,德雷福斯事件尚未爆发,某些犹太显贵还很有权势,而其中最大的是鲁弗斯·以色列爵士,他的妻子以色列夫人是斯万的姨母。她本人并没有外甥那样高雅的社会交往,外甥也并不喜欢她,从未认真与她联络感情,虽然他很可能是她的继承人。然而,在斯万的亲戚当中,只有这位姨母意识到斯万的社交地位,而其他人在这方面与我们一样(长期地)一无所知。在家族中,当一个成员跻身于上流社会时——他以为这是独一无二的现象,但在十年以后,他会看到在和他同时成长的青年中,以不同的方式和理由完成这个现象者大有人在——他在四周画出一圈黑暗区域terra incognita\footnote{拉丁文:未知地域。},居住其中的人对它了如指掌,而未得其门而入者虽然从它旁边走过,却不觉察它的存在,还以为是一片黑暗,一片虚无。既然没有任何通讯社将斯万的社会交往通知他的亲戚,因此,他们在饭桌上(当然在可怕的婚事以前)谈到斯万时,往往露出屈尊的微笑,讲述他们如何“高尚地”利用星期日去探望“夏尔表亲”,而且把他看做心怀嫉妒的穷亲戚,借用巴尔扎克小说的标题,风趣地称他为“傻表亲”\footnote{小说《贝姨》法文为Cousine Bette,Bete与Bette同音。}。鲁弗斯·以色列夫人与众人不同,她很明白与斯万慷慨交往的是些什么人,而且十分眼红。她丈夫的家族与罗特希尔德家族一样有钱,而且好几代以来便为奥尔良王公们经营事务。以色列夫人既然腰缠万贯,当然很有影响,并且利用自己的影响来劝阻她认识的人接待奥黛特,只有一个人偷偷地违背了她,那就是德·马桑特伯爵夫人。那天奥黛特去拜访德·马桑特夫人,不巧以色列夫人几乎同时到来。德·马桑特夫人如坐针毡。这种人什么都做得出来,所以她竟然背信弃义地不和奥黛特说一句话,奥黛特自然不再将入侵向前推进了,何况这个阶层决非她希望被接纳的阶层。圣日耳曼区对奥黛特丝毫不感兴趣,仍旧将她看做与有产者完全不同的、毫无修养的轻佻女人(有产者精通家谱中的每个细节,而且,既然现实生活并未向他们提供贵族亲友,他们便如饥似渴地阅读回忆录)。另一方面,斯万似乎继续是情人,在他看来,这位往日情妇的一切特点似乎仍然可爱或者无伤大雅,因为我常常听见他妻子说一些难登大雅之堂的话,而他却无意纠正(也许是因为对她尚有柔情,也许是对此掉以轻心,或者懒于帮她提高修养)。这也可能是另一种形式的单纯。在贡布雷,我们曾长期被他的单纯所蒙蔽,而且就在现在,虽然他继续结交体面人物(至少为他自己着想),却不愿他们在他妻子的沙龙的谈话中占有重要地位,何况对他来说,他们的重要性确实大为减少,因为他生活的重心已经转移。总之,奥黛特在社交方面十分无知。当人们先提到德·盖尔芒特公爵夫人,后提到她表亲德·盖尔芒特公主时,她竟然说:“噫,这些人是王公,那么说他们晋升了。”如果有谁在谈到夏尔特尔公爵时用“亲王”一词,她马上纠正说:“是公爵,他是夏尔特尔公爵,不是亲王。”关于巴黎伯爵的儿子德·奥尔良公爵,她说:“真古怪,儿子的爵位比父亲高。”作为英国迷,她又接着说:“这些royalties(王族)真叫人糊涂。”有人问她盖尔芒特家族是哪省人,她回答说:“埃纳省。”
\par 斯万在奥黛特面前是盲目的,他既看不见她教养中的缺陷,也看不见她智力上的平庸。不仅如此,每当奥黛特讲述什么愚蠢的故事时,斯万总是殷勤地、快活地、甚至赞赏地(其中可能掺杂着残存的欲念)聆听,而如果斯万本人说出一句高雅的、甚至深刻的话时,奥黛特往往兴趣索然、心不在焉、极不耐烦,有时甚至厉声反驳。人们因而得出结论说,精华受制于平庸在不少家庭中是司空见惯的,因为,反过来,也有许多杰出女性竟被对她们的睿智横加指责的蠢人所蛊惑,并且被极度慷慨的爱情所左右而对蠢人的俗不可耐的玩笑赞叹不已。说到当时妨碍奥黛特进入日耳曼区的理由,应该指出社交界的万花筒的最近一次转动是由一系列丑闻引起的。人们原来放心大胆地与某些女人交往,而她们竟被揭露是妓女,是英国间谍。在一段时间内,人们首先(至少认为如此)要求他人的是牢靠和稳定……奥黛特代表的正是人们刚刚与之决裂又立刻拾起的东西(因为人们不可能在一夜之间彻底改变,他们在新制度下寻找旧制度的继续),当然它必须换一种形式,以掩人耳目,制造与危机前的社交界有所不同的假象。但奥黛特与那个社交界的替罪羊太相似了。其实,上流社会的人是高度近视眼。他们与原来认识的犹太女士断绝来往,正考虑如何填补空白,却看见一位仿佛被一夜风暴刮来的新女人,她也是犹太人,但由于新颖,便不像在她以前的女人那样使人们联想起他们认为应该憎恶的东西。她不要求人们崇敬他们的上帝。人们便接纳了她。诚然,在我初访奥黛特家时,反犹太主义问题尚未提出,但是奥黛特与当时人们唯恐避之不及的东西十分相似。
\par 至于斯万,他仍然常去拜访旧日的、也就是属于最上层社会的朋友。当他谈到刚刚拜访过什么人时,我注意到在旧日的朋友中,他是有所取舍的,而选择的标准仍然是作为收藏家的半艺术半历史的鉴赏力。某位家道中落的贵妇引起他的兴趣,因她曾是李斯特的情妇,或者因为巴尔扎克曾将一本小说献给她的外祖母(正如他买一幅画是因为夏多布里昂描写过它)。这使我怀疑我们在贡布雷时莫非是从一个谬误过渡到另一个谬误,即最先认为斯万是一位从不涉足社交的资产者,后来又认为他是巴黎顶顶时髦的人物。成为巴黎伯爵的朋友,这不能说明任何问题。“王公的朋友”被排外倾向的沙龙拒之门外的,不是大有人在吗?王公们自知为王公,便不追求时髦,而且自认高居于没有王族血统者之上,大贵族和资产者统统在他们之下,并且(从高处看)几乎处在同一水平上。
\par 此外,斯万在目前的社交圈子中(他重视过去所留下的、至今仍然可以见到的名字)所寻求的不仅仅是文人和艺术家的乐趣,将不同的成分交混起来,将不同的类型聚合起来,从而搭配成社会花束,这也是他的消遣(不那么高雅)。这些有趣的(或者斯万认为有趣的)社会实验在他妻子的每位女友身上并不产生——至少不是经常地——相同的反应。“我打算同时邀请戈达尔夫妇和旺多姆公爵夫人。”他笑着对邦当夫人说,好像一位贪吃的美食家想换换调味汁的成分,用圭亚那胡椒来替代丁子香花蕾。然而,这个似乎会使戈达尔感到有趣的计划却使邦当夫人大为恼火。她最近被斯万夫妇介绍认识旺多姆公爵夫人,认为这事既使人高兴又理所当然,而对戈达尔夫妇讲述它,加以吹嘘,这构成她的愉快中饶有兴味的一部分因素。邦当夫人希望,在她以后,她那圈子里再没有任何人被介绍给公爵夫人,正好比被授勋者一得到勋章便立刻希望将十字勋章的水龙头关上。她暗暗诅咒斯万的低级鉴赏力。他为了实现一种无聊的、古怪的审美观,竟能在一瞬间将她对戈达尔夫妇谈论旺多姆公爵夫人时所散布的迷雾吹得一干二净。她怎敢对丈夫说教授夫妇也即将分享这个愉快(她曾吹嘘说它是独一无二的)呢?要是戈达尔夫妇明白这种邀请不是出自主人的诚心,而是为了解闷,那就好了!其实,邦当夫妇的被邀请难道不也如此吗?不过,斯万从贵族那里学到了永恒的堂璜作风,他有本领使两位不足道的女人同时认为自己是真正的被爱者,因此,当他对邦当夫人提起旺多姆公爵夫人时,那口气仿佛邦当夫人和公爵夫人同桌进餐自然是不在话下的事。“是的,我们打算邀请公主和戈达尔夫妇,”斯万夫人在几星期后说道,“我丈夫认为这种集合可能产生有趣的东西。”如果说斯万夫人保留了“小核心”中维尔迪兰夫人所喜爱的某些习惯——例如高声说话好让所有的信徒听见——的话,那么她也使用盖尔芒特圈子所喜爱的某些语言(例如“集合”一词),她与盖尔芒特圈子并不接近,但却在远处、在不知不觉中受它吸引,正如大海被月亮吸引一样。“是的,戈达尔夫妇和旺多姆公爵夫人,您不觉得这很有趣吗?”斯万问道。“我看这会很糟,您会招来麻烦的,可别玩火。”邦当夫人气冲冲地回答。她和她丈夫,还有阿格里让特亲王都受到邀请,而对这次宴会,邦当夫人和戈达尔各有各的说法,依问话人而定。有些人分别问邦当夫人和戈达尔,那天吃饭的除了旺多姆公主外,还有哪些客人,得到的回答都是漫不经心的两句话:“只是阿格里让特亲王,这完全是熟朋友之间的便餐。”但另一些人可能更知情(有一次有人甚至问戈达尔:“邦当夫妇不是也在场吗?”“哦,我忘了。”戈达尔红着脸回答说,并从此将这个问话的笨蛋列入多嘴饶舌者之流)。对于这些人,邦当夫妇和戈达尔夫妇不谋而合地采取了大致相同的说法,只是将名字对换一下而已。戈达尔说:“唉,只有主人,旺多姆公爵夫妇(自负地微微一笑),戈达尔教授夫妇,此外,对了,莫名其妙,还有邦当夫妇,他们可是有点煞风景。”邦当夫人讲的也完全一样,不同的是,邦当夫妇的名字位于旺多姆公爵夫人和阿格里让特亲王之间,并且受到得意洋洋的夸张,而她最后责怪所谓不请自来并且大煞风景的秃子,就是戈达尔夫妇。
\par 斯万往往在晚饭前不久才从访问中归来。晚上六点钟,这时刻在往日曾使他痛苦,而如今却不然,他不再猜测奥黛特大概在做什么,是接待客人还是外出,他对这些都不在意。他有时回忆起多年以前,他有一次曾试图透过信封看奥黛特给福什维尔写了什么。但这个回忆并不愉快,他不愿加深羞愧感,只是撇了一下嘴角,必要时甚至摇摇头,意思是:“这对我有什么关系呢?”从前他常常坚持一个假定,即奥黛特的生活是无邪的,只是他本人的嫉妒、猜测才使它蒙受耻辱罢了,但是现在,他认为这个假定(有益的假定,它减轻他在爱情病中的痛苦,因为它使他相信这痛苦是虚构的)是不正确的,而他的嫉妒心却看对了。如果说奥黛特对他的爱超过他的想象的话,那么,她对他的欺骗更超过他的想象。从前,当他痛苦万分时,曾发誓说有朝一日他不再爱奥黛特,不再害怕使她恼怒,不再害怕让她相信他热恋她时,他将满足夙愿——本着单纯的对真理的追求,并为了解释历史的疑点,与她一起澄清事实,弄清那天(即她写信给福什维尔,说来探望她的是一位叔叔)他按门铃敲窗子而她不开门时,她是否正和福什维尔睡觉。斯万从前等待嫉妒心的消失,好着手澄清这个饶有兴趣的问题。然而,如今他不再嫉妒了,这个问题在他眼中也失去了一切趣味。当然并不是立刻。他对奥黛特已经不再嫉妒,但是,那天下午他敲拉彼鲁兹街那座小房子的门而无人回答的情景却继续刺激他的嫉妒心。在这一点上,嫉妒心与某些疾病相似:疾病的病灶和传染源不是某人,而是某个地点,某座房屋,嫉妒的对象似乎也不是奥黛特本人,而是斯万敲击奥黛特住所的每扇门窗的那已逝往日中的一天、一个时刻。可以说,只有那一天和那个时刻保留了斯万往日曾有过的爱情品格中的最后残片,而他也只能在那里找到它们。长期以来,他不在乎奥黛特是否曾欺骗他,是否仍然在欺骗他。但是,在几年里他一直寻找奥黛特从前的仆人,因为他仍然有一种痛苦的好奇心,想知道在如此遥远的那一天,在六点钟时,奥黛特是否在和福什维尔睡觉。后来连这种好奇心也消失了,但他的调查却未中止。他继续设法弄清这件不再使他感兴趣的事,因为他的旧我,虽然极度衰弱,仍然在机械地运转,而过去的焦虑已烟消云散。他甚至无法想象自己曾经感到如此强烈的焦虑,当时他以为永生也摆脱不了焦虑,以为只有他所爱的女人的死亡(本书下文中将有一个残酷的反证,说明死亡丝毫不能减弱嫉妒的痛苦)才能打通他那完全堵塞的生活道路。
\par 然而,有朝一日将奥黛特生活中使斯万痛苦的事弄个水落石出,这并不是斯万的唯一愿望。他还保留了另一个愿望,即当他不再爱奥黛特、不再害怕她时,他要为这些痛苦进行报复,而眼前恰恰出现了实现这第二个愿望的机会。斯万爱上了另一个女人。他没有任何理由嫉妒,却仍然嫉妒,因为他无力更新恋爱方式,他将往日与奥黛特的恋爱方式应用在另一个女人身上。她不必有任何不忠行为,只要由于某个原因离开他,比方说,参加晚会,而且似乎玩得很开心,这就足以使斯万妒火中烧,这就足以唤醒他身上那古老的焦虑——他的爱情的可悲而矛盾的赘疣。焦虑使斯万与真实的她保持距离,他必须努力才够得着她(了解这个年轻女人对他的真实感情,她每天的隐秘欲望和内心秘密)。焦虑在斯万和他所爱的女人中间放上了旧日的冥顽不化的猜疑,猜疑的根源在奥黛特或者比奥黛特更早的某个女人身上,正是由于它,年老的情人只能通过“挑起嫉妒心的女人”这个古老的集体幻影来认识他今日的情妇,而且将新爱情也武断地置于这个幻影之中。然而,斯万经常谴责这种嫉妒心理,谴责它使自己相信某些实属虚幻的不忠行为,但是他记起当初也曾采取同样的观点替奥黛特辩解,而且是做错了。因此,当他和他所爱的年轻女人不在一起时,她的所作所为,在他眼中,便不再是清白无邪的。他曾起誓说,万一哪一天他不再爱这位当时未想到会与他结婚的女人时,他将毫不留情地对她冷若冰霜(真正的冷若冰霜!),好为他长期受辱的自尊心进行报复,他现在可以毫无风险地(即使奥黛特把他的话当真,取消他从前梦寐以求的和她单独谈话,他也毫不在乎)进行报复了,但他却无意报复。爱情既已消逝,表示不再爱的愿望也随之消失。当他为奥黛特痛苦时,他多么盼望有一天让她看看他爱上了别的女人,而现在他可以做到这一点,却小心翼翼地不让妻子知道自己另有新欢。
\par 从前,每到喝茶的钟点,我便闷闷不乐地看见希尔贝特离开我,提前回家,而现在,我也参加这些茶会。从前,当她和她母亲出门散步或看日场演出时,我便独自一人痴痴待在香榭丽舍的草坪边或木马旁,因为她来不了,而现在呢,斯万夫妇允许我和他们一起出门,他们的马车里有我的座位。有时他们甚至问我愿意去哪里,去看戏还是看希尔贝特一位同伴的舞蹈课,参加斯万夫人女友家的社交聚会(斯万夫人称为“小会”)还是去参观圣德尼的国王墓。
\par 每逢和斯万一家出门的日子,我便去他们家吃午饭,斯万夫人管它叫lunch(午饭)。他们邀请我十二点半去,那时我父母在十一点一刻吃午饭,所以等他们离开餐桌后,我才朝斯万家的奢华街区走去。在这个街区里,行人向来稀少,何况在这个钟点谁都回了家。即使在严冬,如果天气晴朗,我便在马路上来回溜达,一直等到十二点二十七分。我一会儿扯扯从夏费商店买的那条精美领带的领带结,一会儿看看脚上那双高帮漆皮皮鞋是否弄脏了,我远远看见斯万家小花园里的光秃秃的树在阳光下像白霜一样晶莹闪光。当然,小花园里只有两株树。在这个反常的钟点,景物也焕然一新。与自然所给予的乐趣(习惯的改变,甚至饥饿使它更为强烈)相交织的是即将与斯万夫人同桌进餐的激动,它并不削弱乐趣,而是控制它、奴役它,使之成为社交生活的陪衬。我似乎发现了往日在这个钟点所感觉不到的晴空、寒冷、冬日的阳光,它们好像是奶油鸡蛋的前奏曲,好像是斯万夫人之家这座神秘殿堂表层上的时间光泽、浅红的淡淡冷色,而在殿堂内部却有那么多温暖、芳香和鲜花。
\par 十二点半,我终于下决心走进这座房子。它像圣诞节的大靴子一样将给我带来神奇的快乐(斯万夫人和希尔贝特都不知道圣诞节在法文里怎么说,所以总是用Christmas来代替,Christmas布丁啊,收到什么Christmas礼品啊,在Christmas期间要去外地什么地方等等,我感到不是滋味,回到家中也说Christmas,认为说圣诞节有失体面,而父亲认为这种语言滑稽可笑)。
\par 我最初只遇见一位跟班,他领我穿过好几间大客厅来到一间很小的客厅,那里没有人,从窗口射进来的下午的蓝光使它沉浸在梦幻之中。只有兰花、玫瑰花和紫罗兰陪伴我——它们像人一样待在你身边,但并不认识你。它们是有生命的,而这种特性使它们的沉默产生强烈的效果。它们畏惧寒冷,接受炽热炉火的温暖。那被珍贵地放在水晶挡板后面的炉火不时地将危险的红宝石散落在白色大理石的火盆中。
\par 我已坐了下来,但听见开门声便赶紧站了起来,进来的是第二位仆人,跟着又是第三位仆人,而他们这种使我无谓激动的频繁往来仅仅是为了鸡毛蒜皮的事:往火中添一点煤或往花瓶里加一点水。他们走后,门又关上(斯万夫人最后总会将它打开的),我又独自一人。确实,魔术师的洞穴也不如这间小客厅那样使我眼花缭乱,炉火在我眼前千变万化,好似克林索\footnote{瓦格纳歌剧《帕西法尔》中的魔术师。此处指第二幕开场的魔室。}的实验室。又响起一阵脚步声,我没有站起来,大概又是仆人吧,不是,是斯万先生。“怎么?您一个人在这里真是没办法,我那可怜的妻子从来不知道钟点。一点差十分了。她每天都迟到。您一会儿看见她不慌不忙地进来,她还以为自己提前到哩。”斯万仍然患神经炎,而且变得可笑,这样一个不遵守时间的妻子(从布洛尼林园回来必晚,在裁缝店逗留必久,吃饭必迟到)虽然使他为肠胃担心,但却满足了他的自尊心。
\par 他领我参观新近的收藏品,并且向我解释它们的价值,可是我过于兴奋,又由于在这个钟点我还破例腹中空空,我心神不定,脑子里一片空白。虽然我还能够说话,但什么也听不进去了。何况,就斯万所拥有的收藏品而言,只要它们存在于他家,只要它们属于午餐前的美妙时刻,这对我就绰绰有余了。即使那里有《蒙娜丽莎》,它也不会比斯万夫人的便袍或嗅盐瓶更使我愉快。
\par 我继续等侍,独自一人,或者和斯万一起,希尔贝特还常常来和我们做伴。斯万夫人既然以如此威严的仆人为先导,她的出现一定不同凡响。我屏息静听每一个声响。真正的教堂、风暴中的海涛、舞蹈家的跳跃往往比人们的想象要逊色。穿制服的仆人酷似戏剧中的配角,他们的连续出场为王后的最后显现作准备,同时也削弱显现的效果;在这些仆人之后是悄悄进来的斯万夫人,她身穿水獭皮小大衣,冻得发红的鼻子上盖着面纱,与我的想象力在我等候期间所慷慨臆造的形象何等不相似!
\par 如果她整个上午都没有外出,那么她走进客厅时身穿一件浅色双绉晨衣,对我来说,它比一切衣袍都更雅致大方。
\par 有时,斯万夫妇决定整个下午待在家里。吃完午饭天色已不早,这一天(我原以为它会和别的日子完全不同)的阳光正斜照在小花园的墙上。仆人们端来大大小小的、各式各样的灯,它们各自在蜗形脚桌、独脚圆桌、墙角柜或小桌这些固定祭坛上燃烧,仿佛在进行莫名其妙的祭祀。尽管如此,谈话平淡乏味,我败兴而返,像自童年起每次做完午夜弥撒以后那样大失所望。
\par 然而这仅仅是思想上的失望。我在那座房子里是十分喜悦的,因为,如果希尔贝特尚未和我们在一起,那么她即将进来,而且即将将她的话语、她那专注而微笑的目光(正如我第一次在贡布雷所见到的那样)给予我(而且达数小时之久!)。当我看到她消失在通往宽大房间的内部楼梯上时,我至多稍稍感到嫉妒。我只能留在客厅里(就像一位女演员的恋人,他只能待在正厅前座,不安地臆想在后台、在演员休息室正发生什么事),我向斯万了解房屋的另一部分,我的问题被掩饰得很巧妙,但声调中仍流露出不安。他告诉我希尔贝特去的是衣被间,并自告奋勇要带我去看看,而且说以后希尔贝特去那里,他一定要她带我去。斯万的最后这句话使我如释重负,霎时间消除了那段使我们所爱的女人显得如此遥远的、可怕的内心距离。此刻,我对他的感情油然而生,似乎比我对希尔贝特的柔情更深。因为,他作为自己女儿的主人,将她给予我,而她本人却有时拒绝我。我对她的直接影响比不上我通过斯万而施于她的间接影响。此外,我爱的是她,每当我看见她时,我不禁感到心慌意乱,不禁渴望更多的东西,而这种情绪恰恰使我们在所爱的人面前失去了爱的感觉。
\par 我们往往不待在家中,而是出门走一走。在换衣出门以前,偶尔,斯万夫人在钢琴前坐下,她从粉红色或白色的,总之色彩鲜艳的双绉丝便袍的袖中,伸出那双娇美的手,张开手指抚弹琴键,表现出那种存在于她的目光中却不存在于她心中的忧郁。正是在这样的一天,她偶然为我弹奏凡德伊奏鸣曲,即斯万十分喜爱的那个小乐段。当我们头一次聆听稍微复杂的乐曲时,往往什么也没听出来。然而,等我后来听过两三遍凡德伊奏鸣曲以后,我感到对它很熟悉。看来,第一次听懂的说法是有道理的。如果第一遍没有真正听出什么东西,那么第二、第三遍仅仅是第一遍的重复,不可能在第十遍有新的感悟。这样看来,第一遍所缺乏的也许是记忆,而绝不是理解,因为我们的记忆,与我们聆听时它所面临的复杂感受相比较,是极为微小、极为短暂的,好比一个人在睡眠中想到种种事情但立即忘在脑后,又好比一位老年痴呆症患者将别人一分钟前对他说的话忘得一干二净。这些复杂丰富的感受,我们的记忆力不可能立即向我们提供回忆。回忆是在记忆力中逐步形成的。当我们听过两三遍作品以后,我们就像中学生(他们入睡前还反复复习,觉得尚未掌握)一样,第二天早上倒背如流。只是,我以前从未听过这支奏鸣曲,因此,斯万和他妻子所熟悉的那个乐段与我清晰的感知相距遥远,仿佛是记不起来的名字。人们尽力回忆,但找到的是一片虚空,但是,一个小时以后,当人们不再去想时,最初寻而未得的那个音节却自动跳了出来。真正的稀世之作是难以立即被人们记住的,何况,就每个作品内部来说(例如凡德伊奏鸣曲之于我),人们最先感知的是最次要的部分。我错误地认为,既然斯万夫人已为我弹奏了那十分著名的乐段(在这一点上我和某些傻子一样,他们既然看过威尼斯圣马可教堂的圆顶的照片,便以为再没有什么新奇了),奏鸣曲不会给我任何新启示(因此在长时间中我不注意聆听它)。不仅如此,即使我从头到尾再听一遍,奏鸣曲的整体在我眼前仍然影影绰绰,就像是一座由于距离太远或浓雾迷漫而若隐若现的建筑物。因此,认识作品如同认识在时间中实现的事物一样,这个过程是令人忧郁的。当凡德伊奏鸣曲中最隐蔽的东西向我显露时,我最初所注意并喜爱的东西,在我的感觉所无法左右的习惯的支配下,开始逃走,离开我。既然我只能在相继的时间中喜爱奏鸣曲所给予我的一切,它便像生活一样,我永远也无法全部掌握它。然而,伟大的杰作并不像生活那样令人失望,它最初给予我们的并不是精华。在凡德伊奏鸣曲中,最先被人发现的美也是最快使人厌倦的美,而原因大概是这种美与人们已知的美最接近。然而当这种美远去以后,我们爱上某个片段,对它新颖的结构迷惑不解,我们无法识辨它,无法触及它一丝一毫。我们每日从它身边走过而毫不觉察,它自我保存得十分妥帖。在它本身的美的魔力下,它变得不可见,始终不可知,一直到最后它才走向我们,而我们最后离开的也是它。我们对它的爱比对其他一切的爱都长久,因为我们花了更长的时间才爱上它。一个人理解比较深刻的作品所需要的时间(如同我理解这个奏鸣曲),与公众爱上新的传世之作所需的多少年甚至多少世纪相比,仅仅是缩影和象征。因此,天才为了躲避世人的忽视,对自己说,既然同时代人缺乏必要的时间距离,那么为后代写的作品就只能被后代读懂(仿佛图画一样,站得太近就无法欣赏)。但是实际上,预防错误判断的一切怯懦行动都徒劳无益,因为错误判断是无法避免的。一部天才作品很难立刻受到赞扬,因为它的创作者卓越非凡、与众不同。但作品本身能够孕育出作者的知音(难能可贵的),而且人数越来越多。贝多芬的四重奏(第十二、十三、十四、十五)用了五十年之久才使它的听众诞生和壮大,它像任何杰作一样,使艺术家的价值——至少使知识界——实现跃进,因为,在作品诞生之初,有能力赞赏它的人凤毛麟角,而如今在知识界中却大有人在。所谓后代,其实就是作品的后代。作品本身(为了简明起见,此处不包括这种天才:它们在同一时期不是为自己,而是为其他天才培养未来的更佳公众)必须创造自己的后代。如果作品被封存起来,只是在后代面前才显现的话,那么,对作品来说,这个后代将不是后代,而是同代人,仅仅晚生活五十年罢了。因此,如果艺术家希望作品自辟道路的话,他必须——这正是凡德伊所做的——在有足够深度的地方抛出它,朝着遥远的真正未来抛过去。这个未来的时间是一部杰作的真正远景,蹩脚的鉴赏家的错误在于忽视这未来的时间,而高明的鉴赏家有时带着一种危险的苛求来考虑它。当然,如果从使远处事物显得朦胧不清的视觉出发,人们可能认为迄今为止的一切绘画或音乐革命毕竟都遵循某些规则,而我们眼前的一切,如印象主义、对不谐调效果的追求、中间阶次的绝对化、立体主义、未来主义,都粗暴地有别于前者,这是因为我们在看待以前的事物时,没有想到它们经过长期的同化已经在我们眼中成为虽然各不相同、但根本上一致的材料(其中雨果与莫里哀十分相近)。试想一下,如果不考虑未来的时间及它所带来的变化,那么,我们在少年时代所亲耳听到的对我们成年时期的占卜会显得多么荒诞。占卜并不都准确,而既然在一部艺术作品的美的总数中必须加进时间因素,那么,判断就必然带上某种风险,因此也像预言一样失去真正的意义,因为,预言的不能实现并不意味着预卜家智力平庸,同样,使可能性成为现实,或者将它排除在现实之外,这并非天才的必然天职。一个人可以有天才,但却不相信铁路或飞机的发展,或者说,一个人可以是大心理学家,但却不相信情妇或朋友的不忠(而最平庸的人也会估计到他们的不忠)。
\par 虽然我没有听懂奏鸣曲,我却对斯万夫人的演奏心醉神迷。她的弹奏,正如她的晨衣、她的楼梯上的芳香、她的大衣、她的菊花一样,属于一个特殊的、神秘的整体,它比起可以对天才进行理性分析的世界来,要高出千倍。斯万对我说:“这个凡德伊奏鸣曲很美吧?当树影暗下来,小提琴的琶音使凉气泻落在大地的时刻,这支曲子很悦耳。月光的静止作用表达得淋漓尽致,这是主要部分。我妻子正采用光线疗法,月光能使树叶静止不动,那么光线能作用于肌肉也没有什么奇怪的了。这一点是乐段中最精彩的,即得了瘫痪症的布洛尼林园。要是在海边就更妙,海浪在喃喃回答,我们对浪声听得更真切,因为其他一切都凝定不动。在巴黎却不然,我们充其量注意到那些建筑物上奇特的光线、那片仿佛被既无颜色又无危险的大火照亮的天空,那隐隐约约的闹市生活。然而在凡德伊的这个乐段,以及整个奏鸣曲中,没有这些,只有布洛尼林园,在回音中有一个清晰的声音在说:‘几乎能读报了。’”斯万的这番话原可能将我对奏鸣曲的体会引入歧途,因为音乐不能绝对排斥别人对我们的诱导,然而,我从其他的话语中得知他正是在夜间茂密的树叶下(许多傍晚,在巴黎附近的许多餐馆中)聆听这个小乐段的。因此乐句带给他的不是他曾经常常要求的深邃含意,而是它四周那整齐的、缠绕的、着上颜色的叶丛(乐句使他渴望再见到叶丛,乐句仿佛是叶丛的内在灵魂),而是为他保留的整个春天,因为他从前焦躁而忧郁,没有闲情逸致来享受春天(正如为病人保留他吃不下的美食一样)。凡德伊的奏鸣曲使他重温布洛尼林园中的某些夜晚曾对他产生的魅力,而奥黛特对这种魅力却全然无知,虽然她当时和小乐段一起与他做伴。她仅仅在他身旁(不像凡德伊的主题那样在他身上),因此,即使她的理解力增加千倍,她也根本看不见我们所有人的身上所无法表露的东西(至少在长时间中我认为这个规律无一例外)。“这毕竟很美吧?”斯万说,“声音竟可以反射,像水,像镜子。还有,凡德伊的乐句让我看见从前所未注意的东西。至于我当时的烦恼,当时的爱情,它没有丝毫暗示,它采用的是另一种价值系统。”“夏尔,你这样说对我似乎不太礼貌吧。”“不礼貌!你们女人可真了不起!我只是想告诉这位年轻人,音乐所显示的——至少对我而言——绝不是‘意志本身’和‘与无限共同感应’,而是,比方说,动物园的棕榈温室中身穿礼服的维尔迪兰老爹。我虽然身在客厅,但这段小乐句却一次又一次地领我到阿尔默农维尔与他一同进餐。老天爷,至少这比和康布尔梅夫人同去要有趣得多。”斯万夫人笑了起来说:“人家都说夏尔使这位夫人着了迷。”她的声调使我想起在这以前不久,她谈到弗美尔(她居然知道这位画家,我十分惊讶)时曾说:“我可以告诉你,先生在追求我时对这位画家很感兴趣。对吧,亲爱的夏尔?”此时,斯万内心很得意,但是说:“别瞎议论康布尔梅夫人了。”“我不过在重复别人的话罢了。再说,她好像很聪明,虽然我不认识她。她很pushing(有开拓性),这对聪明女人来说是难得的。所有的人都说她迷上了你,这样说也没有什么坏处呀?”斯万像聋子那样一言不发,这是认可也是自鸣得意的表示。
\par “既然我弹奏的曲子使你想起动物园,”斯万夫人假装愠怒地逗笑说,“我们不妨将动物园作为待会儿出去散步的目的地,要是这小伙子喜欢的话。天气多么好,你可以重温那些珍贵的感受了。说到动物园,你知道,这个年轻人原先以为我们很喜欢布拉当夫人呢,其实我尽量避着她。人们把她当做我们的朋友,这是很不体面的。你想想,从来不说人坏话的、好心肠的戈达尔先生居然也说她令人恶心。”“讨厌的女人!她只有一个优点,就是像萨沃纳罗拉,巴多洛梅奥修士\footnote{巴多洛梅奥修士(1472—1517),意大利画家。}画中的萨沃纳罗拉\footnote{萨沃纳罗拉(1452—1498),意大利教士,是前者的老师,后被开除教籍并处死。}。”斯万喜欢在绘画中寻找与人的相似处,这种癖好是经得起反驳的,因为我们所称作的个体的表情其实属于普遍性的东西,并且在不同时期都可能出现(当人们恋爱并且希望相信个体的独一无二的现实时,这一点他们是难以接受的)。本诺佐·戈佐里\footnote{本诺佐·戈佐里(1420—1498),意大利画家。}将梅第奇家族画进朝拜耶稣诞生的博士的行列之中已属年代谬误,更有甚者,斯万认为在这行列中还有一大群斯万的(而并非戈佐里的)同代人的肖像,也就是说,不仅有距耶稣诞生一千五百年以后的人,还有距画家本人四个世纪以后的人。照斯万的说法,巴黎的当代名人无一不在画上的行列之中,就好比在萨杜\footnote{萨杜(1831—1908),法国剧作家。}所写的一出戏中,所有的巴黎名流、名医、政治家、律师,出于对作者和女主角的友谊,也出于时髦,每晚轮流登台跑龙套,并以此为乐。“可是她和动物园有什么关系呢?”“关系可密切啦!”“怎么,她的屁股也像猴子一样是天蓝色?”“夏尔,真不成体统!不,我刚才想到僧伽罗人对她说的话。你讲给他听吧,真是妙语惊人。”“一件蠢事。你知道布拉当夫人说话时,喜欢用一种她认为有礼的、其实是保护者的口吻。”“我们在泰晤士河畔的芳邻们管这叫patronizing(以保护者自居)。”奥黛特插嘴说。“她不久前去动物园,那里有黑人,我妻子说是僧伽罗人,当然对人种学她比我在行。”“算了,夏尔,别嘲笑我。”“这哪是嘲笑呢。总而言之,布拉当夫人对一位黑人说:‘你好,黑种!’”“其实这没什么。”“那位黑人不喜欢这个词,他生气地对布拉当夫人说:‘我是黑种,你是骚种!’”“可真逗!我爱听这段小插曲,挺‘妙’吧?布拉当那个老婆子当时就愣住了。‘我是黑种,你是骚种!’”
\par 我表示很愿意去看看那些僧伽罗人(其中一人曾称呼布拉当夫人为骚种),其实我对他们毫无兴趣。但是我想,洋槐道是去动物园的必经之路,我曾在那里欣赏过斯万夫人,我盼望那位黑白混血的朋友戈克兰\footnote{戈克兰(1841—1909),曾是法兰西喜剧院的著名演员。}(我从来没有机会在他面前向斯万夫人打招呼)看见我和斯万夫人并排坐在马车里在洋槐道上驶过。
\par 希尔贝特走出客厅去换衣服,斯万先生和夫人趁她不在的片刻高兴地向我揭示女儿身上难能可贵的品德。我所观察到的一切似乎都证明他们言之有理。正如她母亲所说的,我注意到她对朋友、仆人、穷人一概给予细致入微的、深思熟虑的关心,努力使他们高兴,唯恐使他们不快,而这往往通过小事(她却付出极大努力)表现出来。她曾经为香榭丽舍大街的那位女小贩缝了件什么东西,而且立刻冒着大雪给她送去。“你不知道她的心地有多好,但毫不外露。”她父亲说。希尔贝特年龄虽小,看上去却比父亲更懂事。每当斯万谈到他妻子的显赫朋友时,希尔贝特转过头去一言不发,但神情中并无责怪之意,因为她觉得对父亲进行最轻微的批评也是不能容忍的。有一天,我们谈起凡德伊小姐,她对我说:“我永远也不想认识她,原因之一在于据说她对父亲不好,让他伤心。这一点,你我都无法理解,对吧?你爸爸要是死了,你会痛不欲生,我爸爸要是死了,我也会痛不欲生,这是很自然的。怎么能够忘记你从一开始就爱着的人呢?”
\par 有一次她在斯万面前特别撒娇。斯万走开以后我和她谈起这一点。“是的,可怜的爸爸,这几天是他父亲去世的忌日。你能理解他的心情吧!你是能理解的,在这些事情上,我们的感觉是一样的。所以,我尽量比平时少淘气。”“可他并不觉得你淘气,他觉得你很完美。”“可怜的爸爸,这是因为他太好了。”
\par 希尔贝特的父母不仅对我夸奖她的品德——这同一个希尔贝特,甚至在我真正看见她以前,曾在教堂前,在法兰西岛的景色中显现过;后来我在去梅塞格利丝的陡坡小路上,看见她站在玫瑰荆棘篱笆前,她唤醒的不再是我的梦想,而是我的回忆。我问斯万夫人,在希尔贝特的同伴中,她最喜欢的是谁。我尽力使语气冷淡,仿佛一位朋友仅仅对主人家孩子的爱好感到好奇而已。斯万夫人回答说:
\par “您对她的心思应该了解得比我多,您是她最喜爱的,英国人叫作crack(佼佼者)。”
\par 当现实折过来严丝合缝地贴在我们长期的梦想上时,它盖住了梦想,与它混为一体,如同两个同样的图形重叠起来合而为一一样。其实,我们愿意让自己的欢乐保持其全部意义,我们愿意就在触摸这些愿望的同时——为了确信这的确是它们——让它们依旧保持不可触及的特征。但是,思想失去了活动空间,它甚至无力恢复最初状态以便与新状态作比较;我们所完成了的认识,我们对出乎意料的最初时刻的回忆,我们所听见的话语,它们一齐堵住了我们的意识,使我们更多地使用记忆力而不是想象力。它们反作用于我们的过去——以致我们在看待过去时不能不受它们影响——它们甚至作用于我们尚未定形的未来。好几年以来,我一直认为拜访斯万夫人是我永远可望而不可即的朦胧的空想,然而在她家待上一刻钟以后,从前那段未相识的时期便变得朦胧而渺茫,仿佛是被实现了的可能性所摧毁的另一种可能性。我如何还能幻想饭厅是一个不可思议的地方呢?我在精神上每走一步都遇见我刚才吃下的美式龙虾所不断发射的、永不消失的光线,它甚至照射我最遥远的过去。斯万在自己身上一定看到同样的现象,可以说,他接待我的这套住宅是一个汇合点、重叠点,其中不仅有我的想象力所创造的理想住宅,还有斯万的嫉妒爱情(它和我的梦想一样富有想象力)经常向他描绘的住宅——他曾幻想与奥黛特所共有的、他和福什维尔去她那里喝橘子汁那天晚上他感到高不可攀的住宅。我们用餐的这间饭厅的布局已经容纳了那出人意外的天堂,那时他曾想象有一天当他对他们俩的膳食总管说“夫人准备好了吗?”时,他一定激动万分,而现在,他的语气却流露出轻微的不耐烦,并夹杂着自尊心的某种满足。我和斯万一样也无法体验我的幸福。连希尔贝特也颇有感触:“当初谁会想到,你默默注视着玩捉人游戏的小姑娘会成为你随时可来看望的好朋友呢?”她谈到的这种变化,从外部来看我当然不得不承认,但我内心并不掌握它,因为它是两种状态组成,而我无法同时想到它们又让它们各自保持特点。
\par 然而,这个住宅既然是斯万的意志所强烈渴望的,肯定对他仍然具有吸引力,如果从我的角度来判断的话(因为它对我并未失去一切奥秘)。长久以来,在我的臆想中,斯万家被笼罩在一种奇特魔力之中,如今我走了进去,但并未将魔力全部逐出。我使魔力退缩,使之被我这个陌生人,我这个贱民——斯万小姐正优雅地递过一把美妙的、敌视的、愤慨的椅子请我坐下——所控制。至今,在我记忆中,我还能感到当时在我周围的魔力。莫非是因为在斯万先生和夫人请我吃饭然后带我和希尔贝特一同外出的那些日子里,当我独自一人等候在那里时,铭刻在我脑中的念头(即斯万夫人、她丈夫和希尔贝特即将出现)通过我的目光刻印在地毯、安乐椅、蜗形脚桌、屏风和图画上了?莫非是自此以后,这些物品和斯万家庭一同生活在我的记忆中,并且最终具有他们的某些特点?莫非是因为既然我知道他们生活在这些物品中间,我便将物品一律看做是他们的私人生活和习惯的象征(我曾长期被排除在他们的习惯之外,因此,即使我受到优待而分享这些习惯时,它们对我来说仍旧是陌生的)?总之,每当我想到这间曾被斯万认为十分不协调(他的批评并不意味着对妻子的鉴赏力进行挑剔)的客厅时——因为它仍保留他俩初识时她的住宅的整体风格,即半温室半画室的风格,但其中许多如今被她认为“不伦不类”的、“过时”的中国货却已去掉,取而代之的是一大堆蒙着路易十六或古式绸罩的小家具(还包括斯万从奥尔良码头的府邸带来的艺术珍品)——它在我的记忆中却毫不杂乱,而是和谐统一,发出特殊的魅力,而这种效果是年代久远的最完好的家具,或者带上某人烙印的最有生气的家具永远望尘莫及的。我们看见某些物品,相信它们有独立的生命,因此我们便赋予它们灵魂,它们保留这个灵魂,并在我们身上发展它。我认为,斯万一家在这套住宅中所度过的时间不同于其他人的时间,这套住宅之与斯万一家每日生活中的时间犹如肉体之与灵魂,它应该体现灵魂的特殊性,而我这种种想法都分散于、混杂于家具的位置、地毯的厚薄、窗子的方向、仆人的服饰等等之中——不论在何处,这些想法都同样令我惶惑及难以捉摸。饭后我们来到客厅的大窗前\footnote{法文baie,可作大窗或海湾解。},在阳光下喝咖啡,这时斯万夫人问我咖啡里要几块糖,并推给我一个带丝套的小凳,它散发出希尔贝特的名字曾施加于我的——先是在玫瑰荆棘下,后是在月桂花丛旁——痛苦的魔力,以及她父母一度表示的敌意(小凳似乎理解并有同感),所以我觉得配不上它,又觉得将脚放在那毫无防卫的软垫上未免是懦弱的行为。独立的灵魂使小凳在暗中与下午两点钟的光线相连。这里的光线与别处的光线是不同的。在我们这个海湾中,它使金色波浪在我们脚前嬉戏,在波浪之中露出发蓝的长椅和朦胧的挂毯,犹如魔岛一般。就连挂在壁炉上方的鲁本斯的画也与斯万先生的系带高帮皮鞋及斗篷大衣一样,具有同一类型的并且同样强烈的魔力。我曾经想穿他那样的斗篷大衣,奥黛特却叫丈夫去换一件更讲究的大衣,好和我一同上街。她也去换衣服,虽然我再三说哪件“外出”服也远远比不上她吃饭时穿的,而且即将换下的那件十分漂亮的双绉便袍或丝便袍,它的颜色不断变化,深玫瑰色、樱桃色、蒂波洛蒂波洛(1696—1770),意大利画家,以色彩明快见长。粉红色、白色、淡紫色、绿色、红色、净面或带花纹的黄色。我说她应该穿着便袍出门,她笑了,也许嘲笑我无知,也许对我的恭维感到高兴。她抱歉地说便袍穿起来最舒服,所以她有那么许多便袍,接着她便离开我们去换上一套令人肃然起敬的、雍容华贵的服装,有时还让我为她挑选我喜欢的一件。
\par 到了动物园,我们下车,我走在斯万夫人旁边,洋洋得意!她漫步走着,悠然自得,大衣在空中飘动,我用赞赏的目光注视她,她卖弄风情地深深一笑,作为对我的回报。如果有希尔贝特的朋友——男孩或女孩——远远向我们打招呼,那么,在他们眼中,我成了当初被我羡慕至极的希尔贝特的朋友——他认识她的家庭并参与她生活中的另一部分,即香榭丽舍大街以外的那一部分。
\par 在布洛尼林园或动物园的小径上,我们往往和斯万的朋友、某位贵妇相遇,她远远地向我们打招呼,斯万却没有看见,这时斯万夫人便说:“夏尔,你没看见蒙莫朗西夫人吗?”于是斯万带着熟朋友的友好微笑,用他所特有的文雅风度,举帽向她深深致意。有时,那位贵妇停下来,高兴地向斯万夫人打招呼,这个举动不会导致任何后果,因为人们知道斯万夫人在丈夫的影响下已经习惯于谨慎从事,不会对这一礼节大加吹嘘的。斯万夫人已学会上流社会的派头,因此,不论那位贵妇如何雍容高贵,斯万夫人绝不甘拜下风。她在丈夫遇见的女友旁站立片刻,从容自如地将希尔贝特和我介绍给她,殷勤之中既大方又镇静,以致很难说在斯万的妻子和那位过路的贵族女人之间,究竟谁是贵妇。那天我们去看僧伽罗人,回家时迎面看见一位女士,她后面有两位太太相随,仿佛是跟班。这位女士年纪不小,但风韵犹存,身穿深色大衣,头戴小帽,两根帽带系在颔下。“啊!这一位会使您感兴趣。”斯万对我说。老妇人离我们只三步远,温柔动人地对我们微笑。斯万摘下帽子,斯万夫人行屈膝礼,并且想亲吻那位酷似温特哈特\footnote{温特哈特(1805—1873),德国画家,擅长画贵族人物肖像。}肖像人物的女士的手,女士扶起她,并亲吻她。“瞧您,请戴上帽子吧。”她用稍稍不快的浊重声音对斯万说,仿佛是位亲密的朋友。“来,我把您介绍给公主殿下。”斯万夫人对我说。斯万夫人和殿下谈论天气和动物园新添的动物,这时斯万把我拉到一旁说:“这是马蒂尔德公主。您知道,她是福楼拜、圣勃夫、仲马的朋友。您想想,她是拿破仑一世的侄女,拿破仑第三和俄国皇帝曾经向她求婚。挺有意思吧?您去和她说说话。不过我可不愿意陪她站一个钟头。”接着他又对公主说:“那天我遇见泰纳,他说公主和他闹翻了。”“他的行为像头猪,”她用粗嗓门说(在她口中,“猪”这个字与贞德同时代的主教的名字\footnote{即皮埃尔·戈雄。戈雄与Cochon(猪)仅一音之差。}同音),“自从他写了那篇关于皇帝的文章,我给他留下一张名片,写着‘特来告辞。’”我像翻开巴拉蒂娜公主即后来的奥尔良公爵夫人的通讯集一样感到惊异。的确,马蒂尔德公主充满了纯粹法国式的感情,她那直率而生硬的方式使人想起旧日的德意志,而这种直率大概来自她那位符腾堡的母亲。然而,只要她像意大利人那样娇弱地一笑,她那稍嫌粗野的、几乎是男性的直率便变得柔软了,而这一切都裹在她那身第二帝国式的装束里。她之所以采用这身装束大概仅仅为了保持她曾经喜爱的款式,但她也似乎有意避免历史色彩的差错,有意使期待她重现旧时代的人得到满足。我低声让斯万问她是否认识缪塞。“很少交往,先生,”她佯作恼怒地说,她称斯万为先生确实是在开玩笑,因为她和他很熟,“我曾请他吃饭。说好七点钟,可七点半他还没有来,于是我们就开饭了。八点钟他才来,向我问好,坐下来,一言不发,吃完饭就走了,自始至终没有说话。他醉得半死。我大失所望,从此再没有请他。”斯万和我站得离她们稍远一点,斯万对我说:“但愿这场接见别拖得太长了,我的脚掌发疼。真不明白我妻子为什么无话找话,等一会儿她会抱怨说累死了,我可忍受不了这种站立。”斯万夫人正将从邦当夫人那里听来的消息告诉公主,说政府终于意识到自己的态度未免失礼,因此决定在沙皇尼古拉后天参观荣军院之际,邀请公主上观礼台。然而,公主——每当她必须行动时——毕竟是拿破仑的侄女,虽然表面上看不出来,虽然和她交往的主要是艺术家和文学家,她说:“是的,夫人,我今早收到请帖并立即退还给部长,他此刻应该收到了。我对他说,我去荣军院根本不需要被邀请。如果政府希望我去,那么,我的位置不在站台上,而在存放皇帝棺椁的墓穴里。我不需要请帖。我有钥匙。我想去就去。政府只需告诉我希望不希望我去。不过,如果我去,一定去墓穴,否则就不去。”正在这时,一位年轻人向斯万夫人和我打招呼,并向她问好,但没有站住。这是布洛克,我不知道斯万夫人也认识他,我向她打听,于是她告诉我她是经邦当夫人介绍认识他的,他在部里秘书处任职(我原先不知道)。她并不经常见到他——或者她认为“布洛克”这个名字不够“帅”,所以不提——她说他叫莫勒尔先生。我告诉她弄错了,他叫布洛克。公主扯了扯垂曳在身后的拖裙。斯万夫人赞赏地看着它。“这是俄国沙皇送给我的皮货,”公主说,“我刚去拜访他,所以穿去让他看看这也可以做大衣。”“听说路易亲王参加了俄国军队,他不在公主身边,公主会感到忧愁的。”斯万夫人说,对丈夫不耐烦的表情毫不觉察。“这对他有好处。我对他说过:虽然家族中有过一位军人,你也可以照样当军人。”公主的回答唐突而直率地影射拿破仑一世。斯万忍无可忍,说道:“夫人,现在由我扮演殿下吧。请您允许我们告辞。我妻子刚生过病,我不愿意让她站立太久。”斯万夫人行屈膝礼。公主对我们大家露出一个神圣的微笑——它仿佛被她从往昔、从她青春时代的风韵和贡比涅宫堡的晚会中召唤而出,而且完美无缺地、甜蜜地盖在那张片刻前还忿忿不快的面孔上——然后走开去,身后跟着那两位女伴;她们刚才仿佛是译员、保姆或病人看护,在我们谈话时插进一些毫无意义的句子和徒劳无益的解释。“这个星期里,您挑一天去她府上写个名字”,斯万夫人对我说,“对这些英国人所称作的皇族,还不能使用名片,不过,您留下名字的话,她会邀请您的。”
\par 冬末春初,我们在散步之前,有时去参观正在举办的小展览会。斯万,作为杰出的收藏家,备受展览会上画商们的敬重。在那些寒气未消的日子里,展览厅唤醒了我想去南方和威尼斯的古老愿望,因为在大厅中,早到的春天和炎热的阳光使玫瑰色的阿尔比伊山闪着淡紫色反光,使大运河发出晶莹透明的深绿色。如果天气不好,我们就去音乐厅或剧场,然后去一家“茶室”吃点心。每当斯万夫人想告诉我什么事而又不愿意邻座或服侍我们的侍者听懂的时候,她便对我说英语,仿佛只有我们两人懂英语,其实人人都会英语,只有我还没有学会,我不得不提醒斯万夫人,让她别再议论喝茶的人或端茶的人,虽然我一个字也听不懂,但我猜到它绝非赞扬,而这番议论一字不漏地传进被议论者的耳朵。
\par 有一次,在看日场演出的问题上,希尔贝特的态度使我吃惊。那天正是她曾提过的她祖父逝世的忌日。她和我原来准备和她的家庭教师一道去听歌剧片段音乐会。她摆出无所谓的神态(不管我们要做什么,她总是表情冷淡,她说只要我高兴,只要她父母高兴,她做什么都无所谓),但是已经换好衣服准备去听音乐会。午饭前,她母亲将我们拉到一边,对她说这个日子去听音乐会会使父亲不高兴的。我觉得这话有理,希尔贝特无动于衷,但无法掩饰自己的愤怒,她脸色发白,一言不发。丈夫回来时,斯万夫人将他叫到客厅另一头低声耳语。于是他叫希尔贝特和他单独到隔壁房间去。我们听见哇啦哇啦的声音。我不敢相信一向顺从、温柔、文静的希尔贝特竟然在这样一个日子,为了这样一件小事而和父亲顶撞。最后斯万走了出来,一面对她说:
\par “我刚才说的你知道了。你自己看着办吧。”
\par 饭桌上,希尔贝特始终板着脸。饭后我们去她房间,突然,她毫不犹豫(仿佛一分钟也没有犹豫过)地惊呼道:“都两点钟了!你知道,音乐会两点半开始。”她催家庭教师赶紧动身。
\par “可是,”我对她说,“你父亲会不高兴吧?”
\par “绝对不会的。”
\par “不过,他恐怕认为这个日子不大合适吧。”
\par “别人怎么想和我有什么相干?在感情问题上管别人的闲事,真荒唐。我们是为自己感受,不是为人家感受的。小姐很少有娱乐的机会,这次兴高采烈地去听音乐会,我不能仅仅为了使人家高兴而让她扫兴。”
\par 她拿起帽子。
\par “可是,希尔贝特,”我抓住她的胳膊说,“这不是为了使人家高兴,是为了使你父亲高兴。”
\par “希望你别来教训我。”她一面用力挣脱我,一面厉声喊道。
\par 斯万夫妇除了带我去动物园或音乐厅以外,对我另有更为宝贵的厚待,即不将我排除在他们与贝戈特的友情之外,而当初正是这种友情使他们在我眼中具有魔力。我甚至在结识希尔贝特以前就认为,她与这位神圣长者的亲密关系会使她成为我最钟爱的女友,如果她对我的蔑视不致使我的希望(希望她有朝一日带我和贝戈特一同参观他所喜爱的城市)破灭的话。
\par 有一天,斯万夫人请我参加一个盛大宴会。我不知道同桌的客人是谁。我到达时,在门厅里遇到的一件事使我胆怯和惶惑。斯万夫人总是采用本季节中被认为最时髦的,但很快就因过时而被摒弃的礼节(例如,多年以前她曾有过hansom cab(双轮双座马车),或者曾在吃饭请帖上印着这是与某某大小名人的会见)。这些礼仪毫不神秘,不需传授便能入门。奥黛特采用了当时从英国进口的小小发明,让丈夫叫人印了一些名片,在夏尔·斯万的名字前冠以Mr(先生)。我首次拜访斯万夫人以后,她曾来我家留下这样一张“纸片”(用她的话说),在这以前从来没有人给我留过名片,因此我无比得意、无比激动、无比感激,兴奋之余,我倾囊中所有订了一个十分漂亮的茶花花篮送给斯万夫人。我恳求父亲去她家留张名片,并且首先赶紧在名字前印上“Mr”,但他对这两项请求置若罔闻,我大为失望,不过几天以后我思索也许他这样做是对的。“Mr”尽管只是摆设,但含义一目了然,而吃饭那一天我见到的另一个礼仪却令人费解。我正要从候见室走进客厅时,膳食总管递给我一个写着我名字的细长信封。我在惊奇之中向他道谢,看看信封,不知该如何处置,就好比外国人面对中国宴席上分发的那些小工具一样不知如何是好。信封口是封着的,立刻拆开未免显得冒失,于是我带着心领神会的表情将它塞进衣袋。几天以前,斯万夫人写信邀我去她家和“几位熟人”一同吃饭,那天客人竟达十六位之多,而且我根本不知道其中还有贝戈特。斯万夫人先后向好几位客人为我“道名”(这是她的说法),突然,在我的名字以后,她不动声色地说出(仿佛我们仅仅是萍水相逢的客人)那位温柔的白发歌手的名字。“贝戈特”像射向我的枪弹,使我震惊,但是,为了表示沉着,我本能地向他鞠躬。在我面前答礼的是个相貌年轻的人,个子不高,身体粗壮、近视眼、长着一个蜗牛壳似的往上翘的红鼻子、黑色的山羊胡。他站在我面前,仿佛是位魔术师:他穿着礼服在枪击的硝烟中安然无恙,而从枪口飞出的竟是一只鸽子。我颓丧至极,因为刚才被炸为齑粉的不仅仅是那位瘦弱的老者(他已荡然无存),还有那些巨著中的美,我曾使它栖息在我特别为它营造(如殿堂一样)的衰弱而神圣的躯体之中,而我面前这位翘鼻子和黑胡须的矮男人,他那粗壮的身体(充满了血管、骨骼、神经结)上哪会有美的栖息之处呢?我曾用贝戈特作品中的透明美来塑造贝戈特,缓慢地、细细地、像钟乳石一样一滴一滴地塑造他,可是顷刻之间,这个贝戈特毫无意义,因为我必须保留他那个翘鼻子和黑胡子,这就好比我们在做算题时不看清全部数据,不考虑总数应该是什么而求题解一样,毫无意义。鼻子和胡子是无法避免的因素,它们使我十分为难,使我不得不重新塑造贝戈特这个人物,它们似乎意味着、产生着、不断分泌着某种入世和自满的精神,而这是不协调的,因为它与他那些为我所熟悉的、充满了平和而神圣的智慧的作品中的气质毫无共同之处。从作品出发,我永远也到达不了那个翘鼻子。而从这个似乎毫不在意的、我行我素的、随兴所致的鼻子出发,我走上与贝戈特的作品完全相反的方向,我的精神状态仿佛像一位匆匆忙忙的工程师——当人们向他打招呼时,他不等别人问好,便理所当然地回答:“谢谢,您呢?”如果别人说很高兴与他认识,他便采用他认为行之有效的、聪明的、时髦的省略句:“彼此彼此。”以避免在毫无意义的寒暄上浪费宝贵时间。名字显然是位随兴所致的画家,它为人物地点所作的速写异想天开,因此当我们面对的不是想象的世界,而是可见世界时(它并非真实世界,因为我们的感官和想象力一样,不擅长于重现真实;看见的世界和想象的世界大不相同,我们对现实的略图也和看见的大相径庭),我们往往大吃一惊。就贝戈特而言,使我更窘迫的不是我对他的名字的先入之见,而是我对他的作品的了解。我不得不将蓄山羊胡子的男人系在这些作品上,仿佛系在气球上,忧心忡忡地唯恐气球无法升空。然而,我热爱的那些书,看来确实是他的作品,因为当斯万夫人按规矩对他说我钦佩他的某部作品时,他对这番为他而发的、而非为其他客人而发的赞词处之泰然,似乎毫不认为这是误会。他为这些宾客而身着礼服,礼服下是那个贪馋地等待进餐的身体,他的注意力集中于某些更为重要的现实,因此当我们提到他的作品时,他微微一笑,仿佛它们不过是他旧日生活的片断,仿佛我们提到的不过是他当年在化装舞会上扮作吉斯公爵这件区区小事。在这个微笑中,他的作品的价值在我眼前一落千丈(并且波及美、宇宙、生命的全部价值),而成为蓄山羊胡子的男人的拙劣消遣而已。我想他曾辛勤笔耕,其实,如果他生活在盛产珠母的小岛,那么,他不会笔耕,而会经营珍珠买卖。他的创作不再像以前一样是命中注定的。于是我怀疑独特性是否真能证明伟大作家是其特有王国中的神,抑或这一切纯属虚构,实际上作品之间的差异来自劳动,而非来自不同个性之间的根本性本质区别。
\par 此时我们入席就坐。我的盘子旁边放着一株用银纸裹着茎部的石竹花。它不像刚才在候见厅拿到的那个信封(而且我早已忘在脑后)使我如此困惑。这个礼仪虽说对我很新颖,但似乎不难理解,因为我看见所有的客人从餐具旁拿起同样的石竹花,插进礼服的扣眼中。我也如法炮制,神情自然,仿佛一位无神论者来到教堂,他不知弥撒是怎么回事,但是众人站起来他便跟着站起来,众人下跪他也跟着下跪。另一个陌生的,但转瞬即逝的礼仪令我很不愉快。在我的餐盘的另一边,有一个更小的盘子,里面装着黑糊糊的东西(我当时不知这是鱼子酱),我不知道应该拿它怎么办,但我决心不碰它。
\par 贝戈特坐得离我不远,他的话语我听得十分清楚,我忽然理解德·诺布瓦先生为什么对他有那个印象。他的确有一个古怪的器官。最能改变声音的物质品质的,莫过于其中所包含的思想了。思想影响二合元音的强度、唇音的力度,以及声调。他的说话方式似乎和写作方式完全不同,就连他说的内容与写的内容也完全不同。他的声音来自一个面具,但它却不能使我们立刻认出面具后面那张我们在他的文笔中所亲眼见到的面孔。很久以后,我才发现他谈话中的某些片断(他所习惯的讲话方式只有在德·诺布瓦先生眼中才显得矫揉造作、令人不快)与他作品的某些部分完全对应,而作品中的形式变得如此富有诗意、富有音乐性。他认为自己的话语具有一种与词意无关的造型美。既然人的语言与心灵相通但又不像文体一样表达心灵,贝戈特的话语似乎是颠三倒四的,他拖长某些字,而且,如果他追求的是单独一个形象,他便将字串联在一起,形成一个单调得令人厌倦的连读音。因此,一种自命不凡的、夸张而单调的讲话方式正是他谈吐的美学品质的标志,正是他在作品中创造一系列和谐形象的能力在话语中的体现形式。我之所以煞费力气才意识到这一点,是因为他当时说的话,正由于它来自贝戈特本人,所以看上去不像是贝戈特的话。这些丰富而精确的思想,是许多专栏作家引为自诩的“贝戈特风格”中所缺乏的。这种不相似可能根源于事实的另一个侧面——在谈话中只能隐约看见它,好比隔着墨镜看画,即当你读一页贝戈特的作品时,你感到那是任何平庸的模仿者在任何时候都写不出来的,虽然他们在报纸书刊中用“贝戈特式”的形象和思想来大大美化自己的文字。文体上的这种区别在于“贝戈特风格”首先是挖掘,这位伟大作家运用天才,将隐藏在每件事物之中的宝贵而真实的因素挖掘出来,挖掘——而非“贝戈特风格”——才是这位温柔歌手的创作目的。事实上,既然他是贝戈特,那么,不论他愿意与否,他都在实践这种风格。从这个意义上说,他作品中每一点新的美正是他从事物中所挖掘出来的每一点贝戈特。然而,如果说每一点美都与其他的美相关且易于识别的话,它仍然是具有特殊性,对它的挖掘也具有特殊性。美既然是新的,便有别于人们所谓的贝戈特风格,这种风格其实不过是贝戈特已经发现并撰写的各个贝戈特的泛泛综合罢了,它绝不可能帮助平庸者去预料在别处会发现什么。对一切伟大作家来说都是这样,他们的文字的美,如同尚未结识的女人的美一样,是无法预料的。这种美的创造,它附在他们所想到的——想到的不是自己——但尚未表达的某件外界事物之上。当今的回忆录作家,如果想模仿圣西门\footnote{圣西门(1675—1755),法国作家。维拉尔是他回忆录中的一位权贵,法国元帅。}而又不愿太露痕迹,可以像维拉尔画像中头一段那样所写:“这是一位身材高大的棕发男子……面貌生动、开朗、富有表情,”但是谁能担保他找到第二段开头的那句话“而且确实有点疯狂”呢?真正的多样性寓于丰富的、真实的、意想不到的因素之中,寓于那些已经缀满春天花朵的篱笆上出人意外地探出身来的蓝色的花枝之中,而对多样性(可以推广至其他所有的文体特点)的纯粹的形式模仿不过是空虚和呆板——与多样化最不相容的特点——罢了。只有那些对大师作品的多样性毫不理解的人,才会对模仿者产生多样性的幻觉或回忆。


\paragraph*{4}

\par 贝戈特的话语,如果不是与他那正在发挥作用的、正在运转的思想紧密相连(这种紧密联系不可能立即被耳朵捕捉),那么它也许会令人倾倒。反言之,正因为贝戈特将思想精确地应用于他所喜爱的现实,因此他的语言才具有某种实在的、营养过于丰富的东西,从而使那些只期望他谈论“形式的永恒洪流”和“美的神秘战栗”的人大失所望。他作品中那些永远珍贵而新颖的品质,在谈话中转化为一种十分微妙的观察事物的方式。他忽略一切已知的侧面,仿佛从细枝末节着眼,陷于谬误之中,自相矛盾,因此他的思想看上去极其混乱,其实,我们所说的清晰思想只是其混乱程度与我们相同的思想罢了。此外,新颖有一个先决条件,即排除我们所习惯的、并且视作现实化身的陈词滥调,因此,任何新颖的谈话,如同一切具有独创性的绘画音乐一样,最初出现时总是过于雕琢,令人厌烦。新颖的谈话建立在我们所不习惯的修辞手段之上,说话者似乎只是采用隐喻这一手段,听者不免感到厌倦,感到缺乏真实性(其实,从前古老的语言形式也曾是难以理解的形象,如果听者尚未认识它们所描绘的世界的话。不过,长期以来,人们把这个世界当做真实的,因而信赖它)。因此,当贝戈特说戈达尔是一个寻找平衡的浮沉子时(这个比喻今天看来很简单),当他说布里肖“在发式上费的苦心超过斯万夫人,因为他有双重考虑:形象和声誉,他的发式必须使他既像狮子又像哲学家”时,听者很快就厌烦,他们希望能抓住所谓更具体的东西,其实就是更通常的东西。我眼前这个面具所发出的难以辨认的话语,的确应该属于我所敬佩的作家,当然它不可能像拼图游戏中的七巧板一样塞到书中,它具有另一种性质,要求转换;由于这种转换,有一天当我自言自语地重复我所听见的贝戈特的词句时,我突然发现它具有和他的文体相同的结构,在这个我原以为截然不同的口头语言中,我认出并确切看到他文体中的各个因素。
\par 从次要的角度看,他说话时常用某些字、某些形容词,而且每每予以强调。他发这些音时,采取一种特殊的、过于精细和强烈的方式(突出所有的音节,拖长最后音节,例如总是用visage来代替figure在法语中,这两个字都为“面孔”。,并且在visage中加上许多的v,s,g,它们仿佛从他此刻张开的手中爆炸出来),这种发音方式与他在文字中赋予这些他所喜爱的字眼的突出地位十分吻合。在这些字眼前面是空白,字眼按句子总韵脚作一定的排列,因此,人们必须充分发挥它们的“长度”,否则会使节拍错乱。然而,在贝戈特的语言中找不到在他或其他某些作家作品中的那种往往使字眼改变外形的光线,这大概是因为他的语言来自最深层,它的光线照射不到我们的话语;因为当我们在谈话中向别人敞开心扉时,在某种意义上,我们却向自己关闭。从这一点来看,他的作品比话语具有更多的音调变化,更多的语气。这语气独立于文体美之外,与作者最深沉的个性密不可分,因此他本人可能并不察觉。当贝戈特在作品中畅叙心怀时,正是这个语调使他所写的、当时往往无足轻重的字眼获得了节奏。这些语调在作品中并未标明,也没有任何记号,然而,它们却自动地附在词句之上(词句只能以这种方式来诵读),它们是作者身上最短暂而又最深刻的东西,而且它们将成为作者本质的见证,以说明作者的温柔(尽管他往往出言不逊)和温情(尽管好色)。
\par 贝戈特谈话中所显示的某些处于微弱状态的特点并非他所独有。我后来结识了他的兄弟姐妹,发现这些特点在他们身上更为突出。在快活的句子里,最后几个字总是包含某种突然的、沙哑的声音,而忧愁的句子总是以衰弱的、奄奄一息的声音作为结尾。斯万在这位大师年轻时便认识他,因此告诉我他当时常听见贝戈特和兄弟姐妹们发出这种可以说是家传的声调,时而是强烈欢乐的呼喊,时而是缓慢忧郁的低语,而且当他们一同在大厅玩耍时,在那时而震耳欲聋时而有气无力的合唱中,贝戈特的那一部分唱得最好。人们脱口而出的声音,不论多么独特,也是短暂的,与人同时消失,但贝戈特的家传发音则不然。如果说,即使就《工匠歌手》\footnote{即瓦格纳的《纽伦堡的工匠歌手》。}而言,艺术家靠聆听鸟鸣来创作音乐就难以令人理解的话,那么,贝戈特也同样令人惊奇,因为他将自己拖长发音的方式转换并固定在文字之中,或是作为重复的欢叫声,或是作为缓慢而忧愁的叹息。在他的著作中,句尾的铿锵之声一再重复、延续,像歌剧序曲中的最后音符一样欲罢不能,只好一再重复,直到乐队指挥放下指挥棒。后来我发觉,这种句尾与贝戈特家族铜管乐般的发音相吻合。不过对贝戈特来说,自从他将铜管乐声转换到作品之中,他便不知不觉地不再在谈话中使用。从他开始写作的那一天起——更不用说我结识他的时候——他的声音中永远失去了铜管乐。
\par 这些年轻的贝戈特——未来的作家及其兄弟姐妹——并不比其他更为文雅、更富才智的青年优秀。在后者眼中,贝戈特这家人嘈杂喧闹,甚至有点庸俗,他们那令人不快的玩笑标志着他们的“派头”——既自命不凡又愚蠢可笑的派头。然而,天才,甚至最大的天才,主要不是来自比他人优越的智力因素和交际修养,而是来自对它们进行改造和转换的能力。如果用电灯泡来给液体加热,我们并不需要最强的灯泡,而是需要一个不再照明的、电能可以转换的、具有热度而非亮度的灯泡。为了在空中漫游,我们需要的不是最强的发动机,而是能将平面速度转换为上升力的另一种发动机(它不再在地面上跑,而是以垂直线取代原先的水平线)。与此相仿,天才作品的创作者并不是谈吐惊人、博学多才、生活在最高雅的气氛之中的人,而是那些突然间不再为自己而生存,而且将自己的个性变成一面镜子的人;镜子反映出他们的生活,尽管从社交角度,甚至在某种意义上从思想角度来看,这生活平庸无奇,但天才寓于反射力中,而非寓于被反射物的本质之中。年轻的贝戈特能够向他的读者阶层展示他童年时生活过的、趣味平庸的沙龙,以及他和兄弟们的枯燥无味的谈话。此刻,他比他家的朋友上升得更高,虽然这些人更机智也更文雅。他们可以坐上漂亮的罗尔斯罗伊斯牌汽车回家,一面对贝戈特家的庸俗趣味嗤之以鼻,而他呢,他那简单的发动机终于“起飞”,他从上空俯视他们。
\par 他的言谈的其他特点是他与同时代的某些作家(而不是与他的家庭成员)所共有。某些比他年轻的作家开始否认他,声称与他没有任何思想共性,而他们在无意之中却显示了这种共性,因为他们使用了他一再重复的副词和介词,他们采用了与他一样的句子结构,与他一样的减弱和放慢的口吻(这是对上一代人口若悬河的语言的反作用)。这些年轻人也许不认识贝戈特(我们将看到其中几位的确不认识),但他的想法已经被灌注到他们身上,并在那里促使句法和语调起变化,而这些变化与思想独特性具有必然联系。这种关系在下文中还需作进一步解释。如果说贝戈特在文体上并未师承任何人的话,他在谈吐上却师承了一位老同学,此人是出色的健谈家,对贝戈特颇有影响,因此贝戈特说起话来不知不觉地模仿他,但此人的才华不如贝戈特,从未写出真正优秀的作品。如果以谈吐不凡为标准,那么贝戈特只能归于弟子门生、转手作家一流,然而,在朋友谈吐的影响下,他却是具有独特性和创造性的作家。贝戈特一直想与喜好抽象概念和陈词滥调的上一代人有所区别,所以当他赞赏一本书时,他强调和引用的往往是某个有形象的场面,某个并无理性含义的图景。“啊!好!”“妙!一位戴橘红色披巾的小姑娘,啊!好!”或者“啊!对,有一段关于军团穿过城市的描写,啊!对,很好!”从文体来看,他与时代不完全合拍(而且他完全属于他的国家,因为他讨厌托尔斯泰、乔治·艾略特、易卜生和陀思妥耶夫斯基)。他在夸奖某某文体时,常用“温和”一词。“是的,我喜欢夏多布里昂的《阿达拉》胜过《朗塞传》,我觉得前者更温和。”他说这话时很像一位医生:病人抱怨说牛奶使他的胃不舒服,医生回答说:“牛奶可是温和的。”贝戈特的文笔中确实有某种和谐,它很像古人在演说家身上所赞赏的和谐,而这种性质的褒词在今天难以理解,因为我们习惯于现代语言,而现代语言追求的不是这种效果。
\par 当人们赞美他的某些篇章时,他露出羞怯的微笑说:“我觉得它比较真实、比较准确,大概有点用处吧。”但这仅仅是谦虚,正好比一位女人听到别人赞赏她的衣服或她的女儿时说:“它很舒服。”或“她脾气好。”然而,建筑师的本能在贝戈特身上根深蒂固,因此他不可能不知道,只有欢乐,作品所赋予他的——首先赋予他,其次才赋予别人——欢乐才是他的建筑既有用又符合真实的确凿证据。可是,多年以后,他才华枯竭,每每写出自己不满意的作品,但他没有理所应当地将他们抹去,而是执意发表,为此他对自己说:“无论如何,它还是相当准确的,对我的国家不会没有一点用处。”从前他在崇拜者面前这样说是出于狡黠的谦虚,后来他在内心深处这样说是出于自尊心所感到的不安。这同样的话语,在从前是贝戈特为最初作品的价值辩护的多余理由,在后来却似乎是他为最后的平庸作品所进行的毫无效果的自我安慰。
\par 他具有严格的鉴赏力,他写的东西必须符合他的要求:“这很温和。”因此,多年里他被看做是少产的、矫揉造作的、只有雕虫小技的艺术家,其实这严格的鉴赏力正是他力量的奥秘,因为习惯既培养作家的风格也培养人的性格。如果作家在思想表达方面一再地满足于某种乐趣,那么,便为自己的才能划定了永久边界,同样,如果人常常顺从享乐、懒惰、畏惧、痛苦等等情绪,那么他便在自己的性格上亲自勾画出(最后无法修改)自己恶习的图像和德行的限度。
\par 我后来发现了作家和人的许多相通之处,但是,最初在斯万夫人家,我不相信站在我面前的就是贝戈特,就是众多神圣作品的作者,我之所以如此,并非毫无道理,因为贝戈特本人(这个词的真正含义)也不“相信”。他不相信这一点,所以才对与他相差万里的交际人物(虽然他并不附庸风雅)、文人记者大献殷勤。当然,他现在从别人的赞赏中得知自己有天才,而社会地位和官职与天才相比一文不值。他得知自己有天才,但他并不相信,因为他继续对平庸的作家装出毕恭毕敬的样子,为的是不久能当上法兰西学院院士,其实法兰西学院或圣日耳曼区与产生贝戈特作品的“永恒精神”毫不相干,正好比与因果规律、上帝的概念毫不相干一样。这一点他也知道,正如一位有偷窃癖的人明知偷窃不好,但无能为力一样。这位有山羊胡和翘鼻子的男人像偷窃刀叉的绅士一样施展伎俩,以接近他所盼望的院士宝座,以接近掌握多张选票的某位公爵夫人,但他努力不让自己的花招被谴责此类目的的人所识破。他只获得了一半成功。和我们说话的时而是真正的贝戈特,时而是自私自利、野心勃勃的贝戈特,他为了抬高自己的身价,大谈特谈有权有势、出身高贵或家财万贯的人,而当初那位真正的贝戈特却在作品中如此完美地描写了穷人那如泉水一般清澈的魅力。
\par 至于德·诺布瓦先生所谈到的其他恶习,例如近乎乱伦的爱(据说还夹杂着金钱诈骗),它们显然与贝戈特的最新小说的倾向背道而驰。这些小说充满了对善良的追求,执著而痛苦的追求,主人公的任何一点欢乐都夹杂着阴影,就连读者也感到焦虑,而在这焦虑之中,最美满的生活也似乎无法忍受。尽管如此,即使贝戈特的恶习是确有其事,也不能说他的文学是欺骗,不能说他丰富的敏感性只是逢场作戏。在病理学中,某些现象表面上相似,起因却各不相等,有的是因为血压、分泌等等过高过多,有的却因为不足,同样,恶习的起因可以是过度敏感,也可以是缺乏敏感。也许在真正的堕落生活中,道德问题的提出才具有令人焦虑的强度,而艺术家对这个问题的答案并不是从个人生活出发,而是属于一般性的文学性的答案——对他来说这才是真正的生活。教会的大圣师们往往在洁身自好的同时,接触人类的一切罪恶,并从中获得自己个人的神圣性。大艺术家也一样,他们往往在行恶的同时,利用自己的恶习来绘制对我们众人的道德标准。作家生活环境中的恶习(或者仅仅是弱点笑柄),轻率乏味的谈话,女儿令人反感的轻浮行径,妻子的不忠,以及作家本人的错误,这些都是作家在抨击中最经常谴责的东西,但他们并不因此而改变家庭生活的排场或者家中所充斥的庸俗情调。这种矛盾在从前不像在贝戈特时代这样令人吃惊,因为,一方面,社会的日益堕落使道德观念越来越净化,另一方面,公众比以前任何时候都更想了解作家的私生活。有几个晚上,在剧场中,人们相互指着这位我在贡布雷时如此敬佩的作家,他坐在包厢深处,他的伴侣们的身份就足以为他最近作品中的观点作注脚——或是对这观点的可笑或尖锐的讽刺,或是对它的无耻否定。这些人或那些人对我说的话并不能使我对贝戈特的善良或邪恶知道得更多。某位好友提出证据,说他冷酷无情,某位陌生人又举一事为例(令人感动,因为贝戈特显然不愿声张),说明他很重感情。虽然他对妻子无情无义,但是,当他在乡村小店中借宿一夜时,他却守候在试图投水自尽的穷女人身旁,而且,当他不得不离开时,他给店主留下不少钱,让店主别把可怜的女人赶走,要照顾她。也许,随着大作家和蓄山羊胡的人在贝戈特身上的此涨彼落,他的个人生活越来越淹没在他所想象的各种人生的浪潮之中。他不必再履行实际义务。因为它已被想象各种人生这项义务所取代。同时,既然他想象别人的感情时如同自己的切身感受,所以,当形势要求他和一位不幸的人(至少暂时不幸)打交道时,他的观点不再是自己的,而是那位受苦者的;既然他从那个观点出发,于是,凡不顾他人痛苦、一心只打自己小算盘的人的语言便受到他的憎恶,因此,他在周围引起了理所当然的怨恨和永不磨灭的感激。
\par 这个人内心深处真正喜欢的只是某些形象,只是用文字来构图和描绘(如同小盒底的袖珍画)。如果别人送他一点小东西,而这小东西能启发他编织形象的话,那么,他一谢再谢,但他对于一个昂贵的礼品却毫无感激之意。如果他出庭申辩,他斟酌字句时不会考虑它们对法官会产生什么效果,而会不由自主地强调形象——法官肯定没有看到的形象。
\par 在希尔贝特家初次与贝戈特相遇的那天,我对他说不久前看了拉贝玛的《淮德拉》。他告诉我有一个场面,拉贝玛静立着、手臂平举——正好是受到热烈鼓掌的那一幕——这是古典杰作在她高超技巧中的巧妙再现,而她大概从未见过这些杰作,例如奥林匹斯圣殿中楣间饰上的那一位赫斯珀里得斯\footnote{法文复数的赫斯珀里得斯是希腊神话人物阿特拉斯(天的托持者)的三个女儿。},以及古代埃雷克塞伊翁寺殿\footnote{埃雷克塞伊翁是希腊雅典古卫城上的寺殿,上有著名的女像柱。}上美丽的贞女。
\par “这可能是直感,不过我想她肯定去博物馆的。‘判明’这一点将很有意义(‘判明’是贝戈特的常用词,有些年轻人虽然从未见过他,但也借用他的词汇,通过所谓远距离启示而模仿他说话)。”
\par “您是指女像柱吧?”斯万问道。
\par “不,不,”贝戈特说,“当然,她向奥侬娜承认爱情时,那姿势很像凯拉米科斯的赫盖索方碑上的图\footnote{凯拉米科斯,雅典城古区,该区墓园中有好几座公元前四世纪的墓碑,其中有赫盖索方碑,碑上一女奴向女主人献珠宝盒。},但除此以外,她再现的是一种更为古老的艺术。我刚才提到古老的埃雷克塞伊翁寺的卡里阿蒂德群像,我承认它与拉辛的艺术没有丝毫相似之处,不过,《淮德拉》内容那么丰富……再添一点又何妨……啊!再说,六世纪的小淮德拉的确很美,挺直的手臂,大理石雕像般的鬈发,不错,她想出这些来真了不起。比起今年许多‘古典’作品来,这出戏里的古典味要浓得多。”
\par 贝戈特曾在一本书中对这些古老的雕像进行著名的朝谒,因此,他此刻的话在我听来清楚明了,使我更有理由对拉贝玛的演技感兴趣。我努力回忆,回忆我所记得的她平举手臂的场面,我还一面想:“这就是奥林匹斯的赫斯珀里得斯,这就是雅典古卫城中美丽祈祷者雕像的一位姐妹,这就是高贵艺术。”然而,要想使拉贝玛的姿势被这些思想所美化,贝戈特本该在演出以前向我提供思想。如果那样的话,当女演员的姿势确确实实出现在我眼前时(也就是说,当正在进行的事物仍然具有全部真实性时,)我就可以从中提取古雕塑的概念。而现在,对于这出戏中的拉贝玛,我所保留的只是无法再更改的回忆,它是一个单薄的图像,缺乏现在时所具有的深度,无法被人挖掘,无法向人提供新东西。我们无法对这个图像追加新解释,因为这种解释得不到客观现实的核对和认可。斯万夫人为了加入谈话,便问我希尔贝特是否让我读了贝戈特论《淮德拉》的文章。“我有一个十分淘气的女儿。”她补充说。贝戈特谦虚地一笑,辩解说那篇文章没什么价值。“哪里的话,这本小册子,妙极了!妙极了!”斯万夫人说,以显示自己是好主妇,让人相信她读过这本书,她不但喜欢恭维贝戈特,还喜欢赞扬他的某些作品,启发他。她的确以自己想象不到的方式给他以启发。总之,斯万夫人沙龙的高雅气氛与贝戈特作品的某个侧面,这两者之间存在着密切联系,对今天的老人来说,它们可以互作注解。
\par 我随兴所致地谈了谈观感,贝戈特并不同意,但任我讲下去。我告诉他我喜欢淮德拉举起手臂时的绿色灯光。“啊!布景师听您这样说会很高兴的,他是位了不起的艺术家,我要把您的看法告诉他,他为这个灯光设计正十分自豪呢。至于我嘛,说实话,我不大喜欢这种灯光,它使一切都蒙在海蓝色的雾气之中,小淮德拉站在那里就像水族馆缸底上的珊瑚枝。您会说这可以突出戏的宇宙性,确实如此。不过,如果剧情发生在海神的宫殿,那么,这种布景就更合适了。是的,当然,我知道这出戏里有海神的报复。不,我并不要求人们仅仅想到波尔罗亚尔,但是,拉辛讲的毕竟不是海神的爱情呀。话说回来。这是我朋友的主意,效果强烈,而且归根到底,相当漂亮。总之,您喜欢它,您理解它,对吧,我们对这一点的想法从根本上是一致的,他的主意有点荒诞,对吧,但毕竟别出心裁。”当贝戈特的意见与我相反时,他决不像德·诺布瓦先生所可能做的那样,使我无言以对,沉默不语,但这并不是说贝戈特不如大使有见解,恰恰相反。强大的思想往往使反驳者也从其中获得力量。这思想本身就是思想的永恒价值的一部分,它攀附、嫁接在它所驳斥的人的精神上,而后者利用某些毗邻的思想夺回少许优势,从而对最初的思想进行补充和修正,因此,最后结论可以算是两位争论者的共同作品。只有那些严格说来不算思想的思想,那些毫无根基、在对手的精神中找不到任何支撑点,任何毗邻关系的思想,才会使对手无言以对,因为他面对的是纯粹的空虚。德·诺布瓦先生的论点(关于艺术)是无法反驳的,因为它是空幻的。
\par 既然贝戈特不排斥我的不同看法,我便告诉他德·诺布瓦先生曾对我嗤之以鼻。“这是个头脑简单的老头,”他说,“他啄您几下是因为他总以为面前是松糕或墨鱼。”斯万问我道:“怎么,您认识诺布瓦?”“啊,他像雨点一样令人厌烦,”他妻子插嘴说,她十分信赖贝戈特的判断力,而且也可能害怕德·诺布瓦先生在我们面前说她的坏话,“饭后我想和他谈谈,可是,不知是由于年龄还是由于消化问题,他显得很迟钝,我看早该给他注射兴奋剂!”贝戈特接着她说:“对,没错,他往往不得不保持沉默,以免不到散场就把他储存的、将衬衣前胸和白背心撑得鼓鼓的蠢话说光了。”“我看贝戈特和我妻子未免太苛刻,”斯万说,他在家中充当通情达理的角色,“当然,诺布瓦不会引起您很大兴趣,但是从另一个角度来看(斯万喜欢收集‘生活’中的美),他这个人相当古怪,是个古怪的情人,”他等希尔贝特确实听不见时才接着说,“他曾在罗马任秘书,那时他在巴黎有位情妇,他爱得发疯,千方百计每星期回来两次,仅仅和她待上两小时。那女人既美丽又聪明,不过现在已经是老太太了。这期间他又有过许多情妇。要是我待在罗马,而我爱的女人住在巴黎,那我准会发疯。对于神经质的人来说,他们必须屈尊‘下爱’(老百姓的说法),因为这样一来,他们所爱的女人就会考虑利害关系而迁就他们。”斯万突然发现我可以将这句格言应用于他和奥黛特的关系,便对我十分反感,因为,即使当优秀人物似乎和你一同翱翔于生活之上时,他们身上的自尊心仍然气度狭窄。斯万仅仅在不安的眼神中流露出这种反感,嘴上什么也没说。这毫不奇怪。据说(这种说法是捏造的,但其内容每日在巴黎生活中重复)拉辛对路易十六提到斯卡隆斯卡隆(1610—1660),法国作家,他死后,路易十四秘密与他的遗孀结婚。时,这位世上最强大的国王当晚没有对诗人说什么,然而第二天拉辛便失宠了。
\par 理论要求得到充分的表述,因此,斯万在这片刻的不快并擦拭镜片以后,对思想进行补充,而在我后来的回忆中,他这番话仿佛是预先警告,只是我当时毫无察觉罢了。他说:“然而,这种爱情的危险在于:女人的屈服可以暂时缓和男人的嫉妒,但同时也使这种嫉妒更为苛刻。男人甚至会使情妇像囚犯一样生活:无论白天黑夜都在灯光监视之下以防逃跑。而且这往往以悲剧告终。”
\par 我又回到德·诺布瓦话题上。“您可别相信他,他好讲人坏话。”斯万夫人说,那口气似乎说明德·诺布瓦先生讲过她的坏话,因为斯万用责备的眼光瞧着她,仿佛不要她往下讲。
\par 希尔贝特已经两次被催促去更衣,准备出门,但她一直待在那里听我们谈话。她坐在母亲和父亲之间,而且撒娇地靠在父亲肩上。乍一看来,她和斯万夫人毫不相似,斯万夫人是褐色头发,而少女是红色头发、金色皮肤。但是片刻以后,你会在希尔贝特身上认出她母亲的面貌——例如被那位无形的、为好几代人捉刀的雕刻师所准确无误地猛然削直的鼻子——表情和动作。如果拿另一种艺术作比喻,可以说她是斯万夫人的画像,但并不十分相似,画家出于对色彩的一时爱好,仿佛让斯万夫人在摆姿势时半装扮成赴“化装”宴会的威尼斯女人。不仅假发是金黄色的,一切深色元素都从她的身体上被排除了,而肉体既已脱去了褐色网纱,便显得更为赤裸,它仅仅被内心太阳所发射的光线所覆盖,因此,这种化装不仅是表面的,它已嵌入肉身。希尔贝特仿佛是神话传奇动物或是装扮的神话人物。她那橙黄色的皮肤来自父亲,大自然当初在创造她时,似乎只需考虑如何一片一片地重现斯万夫人,而全部材料均来自斯万先生的皮肤。大自然将皮肤使用得完美无缺,好比木匠师傅想方设法让木材的纹理节疤露出来。在希尔贝特的面孔上,在那个惟妙惟肖的奥黛特的鼻子旁边,隆起的皮肤一丝不苟地重现了斯万先生那两颗美人痣。坐在斯万夫人旁边的是她的新品种,就好比在紫丁香花旁边的是白丁香花。但是不能认为在这两种相似之间有一条绝对清晰的分界线。有时,当希尔贝特微笑时,我们看见她那张酷似母亲的面孔上有着酷似父亲的椭圆形双颊,老天爷似乎有意将它们放在一起,以考察这种混合的效果。椭圆形越来越清晰,像胚胎一样逐渐成形,它斜着延伸膨胀鼓起,片刻以后又消失。希尔贝特的目光中有父亲的和善坦率的眼神。她给我那个玛瑙弹子并且说:“拿着作为我们友情的纪念吧!”这时我看到这种眼神。可是,如果你对希尔贝特提问题,问她干了什么事,那么,你就会在这同一双眼睛中感到窘迫、犹豫、躲闪、忧愁,而那正是昔日奥黛特的眼神——斯万问她曾去什么地方而她撒谎。这种谎言当初曾使他这位情人伤心绝望,而如今他是位谨慎的丈夫,他不追究谎言,而是立刻改变话题。在香榭丽舍大街,我常常在希尔贝特身上看见这种眼神而深感不安,而在大多数情况下,我的不安毫无根据,因为她身上的这种眼神——至少就它而言——只是来自她母亲的纯粹生理性的遗迹,没有任何含义。当希尔贝特上完课,或者当她不得不回家做功课时,她的瞳孔闪动,就像奥黛特昔日害怕让人知道她白天曾接待情人或者急于去幽会时的眼神一样。就这样,我看见斯万先生和夫人的两种天性在这位梅吕西娜\footnote{梅吕西娜,中世纪传奇中的人物,被罚每星期六变为半蛇半女。}的身体上波动、回涌、此起彼落。
\par 谁都知道,一个孩子可以既像父亲又像母亲,但是他所继承的优点和缺点在配搭上却甚为奇特,以致父亲或母亲身上那似乎无法分开的两个优点,到了孩子身上只剩下一个,而且还伴之以双亲中另一位身上的缺点,而且此一缺点与彼一优点看上去有如水火互不相容。精神优点伴之以无法相容的生理缺点,这甚至是子女与父母相似的一个规律。在两姐妹中,一位像父亲一样仪表堂堂,但同时也像母亲一样才智平庸,另一位充满了来自父亲的智慧,但却套上母亲的外壳,母亲的大鼻子、干瘪的胸部,甚至声音,都好比是天赋抛弃了原先的优美外表而另换上的衣服。因此,两姐妹中任何一位都可以理直气壮地说她最像父亲或母亲。希尔贝特是独生女,但至少有两个希尔贝特。父亲和母亲的两种特性不仅仅在她身上杂交,而且还争夺她,不过这样说不够确切,使人误以为有第三个希尔贝特以此争夺为苦,其实不然,希尔贝特轮流地是这一个她或者是那一个她,而在同一时间里她只能是其中的一个,也就是说,当她是不好的希尔贝特时,她也不会痛苦,既然那个好希尔贝特暂时隐退,又怎能看见这种堕落呢?因此,两个希尔贝特中那个不好的希尔贝特便可以放心大胆地从事格调不高的娱乐。当另一个希尔贝特用父亲的胸襟说话时,她目光远大,你很乐于和她一道从事美好而有益的事业,你这样对她说,可是,当你们即将签约时,她母亲的气质又占了上风,回答你的是它,于是你失望、气馁,几乎困惑不解、仿佛面前是另一个人,因为此时此刻的希尔贝特正在怡然自得地发表平庸的思想,并伴之以狡猾的冷笑。有时,这两个希尔贝特相距万里,以致你不得不自问(虽属徒劳)你到底做了什么错事才使她完全翻脸。她曾要求和你约会,但她没有来,事后也没有道歉,而且,不论是什么原因使她改变主意,她事后的表现判若两人,以致你以为自己被相似的外表所欺骗(如同《孪生兄弟》\footnote{古罗马喜剧作家普劳图斯的剧作。}的主要情节),你面前这个人并非当初如此热切要求和你见面的人。她有时表示愠怒,这说明她于心有愧又不愿意解释。
\par “好了,快去吧,不然我们又得等你了。”母亲对她说。
\par “在亲爱的爸爸身边有多舒服呀,我还想待一会儿。”希尔贝特回答说,一面将头钻在父亲的胳膊下,父亲用手指温柔地抚摸她那头金发。
\par 斯万属于这种男人,他们长期生活在爱情幻想中,他们曾给予许多女人舒适的条件,使她们更为幸福,但却未得到她们任何感激或温情的表示,可是,他们认为在子女身上有一种与姓名嵌镶在一起的感情,这感情将使他们虽死犹生。当夏尔·斯万不再存在时,斯万小姐,或者娘家姓斯万的某某夫人仍然存在,而且仍然爱着她死去的父亲。甚至爱得过分,斯万这样想,因为他回答希尔贝特说:“你是个好女儿。”声音激动不安——当我们想到将来,在我们死后某人会继续深深爱我们,此刻我们便感到不安。斯万为了掩饰自己的激动,便加入我们关于拉贝玛的谈话。他采用一种超脱的、感到厌倦的语调,仿佛想与他说的话保持一定距离。他提醒我注意女演员对奥侬娜说:“你早就知道!”时的声调是多么巧妙,多么惊人地准确。他说得有理。这个声调至少具有明确易懂的涵义,它完全可以满足我那寻找赞赏拉贝玛的确切论据的愿望,然而,正因为它一目了然,它无法满足我的愿望。如此巧妙的声调,伴之以如此明确的意图和含义。它本身便可以独立存在,任何一位聪明的女演员都能学会它。这当然是高招,但是任何人在充分设想以后便能占有它。当然,拉贝玛的功劳在于发现了它,但是此处能用“发现”一词吗?既然就它而言,发现与接受并无区别,既然从本质上讲它并不来自你的天性,既然旁人完全能够复制它!
\par “天呀,您的在场使谈话升级了!”斯万对我说,仿佛向贝戈特表示歉意。斯万在盖尔芒特社交圈中养成了把大艺术家当做好友接待的习惯,只注意请他们品尝他们所喜欢的茶,请他们玩游戏,或者,如果在乡下,请他们从事他们所喜爱的运动。“看来我们确实在谈论艺术了。”斯万又说。“这挺好嘛,我喜欢这样。”斯万夫人说,一面用感激的眼光看我,她也许出于好心,也许由于仍然像往日一样对智力性谈话感兴趣。后来,贝戈特便和别人,特别是和希尔贝特交谈去了。我已经对他谈出了全部感想,而且毫无拘束(连我自己也吃惊),因为多年以来(在无数孤独和阅读的时刻,贝戈特似乎成为我身上最好的一部分),在与他的关系中,我已经习惯于诚恳、坦率、信任,所以,他不像初次谈话的人那样使我胆怯。然而,出于同样的理由,我担心自己给他留下了不好的印象,因为我所假定的他对我思想的藐视不是自今日始,而是从久远的过去,从我在贡布雷花园中最初阅读他作品的时候就开始了。我也许应该提醒自己,既然我一方面对贝戈特的作品大为赞赏,另一方面又在剧院中感到莫名其妙的失望,而且都同样的真诚,同样的身不由己,那么,这两种驱使我的本能运动相互之间不应有很大区别,而是遵循同一规律;我在贝戈特书中所喜爱的思想不可能与我的失望(我无力说明这种失望)毫不相干,或者绝对对立,因为我的智力是一个整体,而且也许世上只存在唯一一种智力,每个人不过是它的参与者,每个人从自己具有个别性的身体深处向它投以目光,就好比在剧场中,每个人有自己的座位,但舞台却只有一个。当然,我所喜欢探索的思想并不一定是贝戈特在作品中所经常钻研的思想,但是,如果他和我共有一个唯一的智力的话,那么,当他听我表达思想时,他会回忆它、珍爱它、对它微笑,因为,不论我作出任何假定,他心灵的眼睛永远保留着与进入他作品的那部分智力(我曾以此为根据来臆想他的全部精神世界)不同的另一部分智力。神甫的心灵经验最为丰富,他们最能原谅他们本人所不会犯的罪孽,同样,天才具有最丰富的智力经验,最能理解与他们本人作品的基本思想最为对立的思想。这一切我本应该提醒自己,虽然这种想法并不令人十分愉快,因为出类拔萃者的善意所得到的后果往往是平庸者的不理解和敌意。大作家的和蔼(至少在作品中可以找到)所给予人的快乐远远不如女人的敌意(人们爱上她不是因为她聪明,而是因为她使人没法不爱)所给予人的快乐。我本应该提醒自己这一切,但我没有对自己说,我深信自己在贝戈特面前显得愚蠢,这时希尔贝特凑到我耳边低声说:
\par “我高兴极了,你赢得了我的好友贝戈特的赞赏。他对妈妈说他觉得你很聪明。”
\par “我们去哪里?”我问希贝尔特。
\par “啊!去哪里都行,我嘛,你知道,去这里或那里……”
\par 自从在她祖父忌日发生的那件事以后,我怀疑她的性格并非如我的想象;她那种对一切都无所谓的态度,那种克制,那种沉静,那种始终不渝的温柔顺从,大概掩饰着十分炽热的欲望,只是受到她自尊心的约束罢了。只有当欲望偶然受到挫折时,她才猛然反击从而有所流露。
\par 贝戈特和我父母住在同一街区,因此我们一同走。在车上,他提起我的健康:“我们的朋友刚才告诉我说您曾经身体不适。我感到遗憾。不过,虽然如此,我也不过分遗憾,因为我看得出来您有智力乐趣,而对您和所有体验这种乐趣的人来说,这可能是最重要的。”
\par 唉!我当时觉得他这番话对我多么不合适,我对任何高明的推理都无动于衷。只有当我在信步闲逛时,当我感到舒适时我才幸福。我清楚感到我对生活的欲望纯粹是物质性的,我可以轻而易举地将智力抛在一边。我分辨不出乐趣的不同的来源、不同的深度、不同的持久性,因此,当我回答贝戈特时,我自认为喜欢的是这样一种生活:和盖尔芒特公爵夫人来往,像在香榭丽舍大街那间旧日税卡里一样感到能唤醒贡布雷回忆的凉气,而在这个我不敢向他吐露的生活理想里,智力乐趣无立锥之地。
\par “不,先生,智力乐趣对我毫无意义,我寻找的不是它,我甚至不知道我是否体验过它。”
\par “您真这么想?”他回答说,“那好,您听我说,真的,您最喜欢的肯定是它,我看得很清楚,我确信。”
\par 当然他没有说服我,但是我感到快活些、开朗些了。德·诺布瓦先生的那番话曾使我认为我那些充满遐想、热情及自信的时刻是纯粹主观的,缺乏真实性。而贝戈特似乎理解我,他的想法正相反,认为我应该抛弃的是怀疑及自我厌恶情绪。他对德·诺布瓦先生的评价使后者对我的判决(我曾认为无法驳回)黯然失色。
\par “您在好好治病吗?”贝戈特问我,“谁给您看病?”我说戈达尔大夫来过,而且还要来。他说:“他对您可不合适。我不知道他的医道如何,不过我在斯万夫人家见过他。这是个傻瓜,就算傻瓜也能当好大夫(我很难相信),但他毕竟不能给艺术家和聪明人看病。像您这样的人需要特殊的医生,甚至可以说需要特殊的食谱、特殊的药品。戈达尔会使您厌烦,而厌烦就是使他的治疗无效。对您的治疗和对任何其他人的治疗应该有所不同。聪明人的疾病四分之三是来自他们的智力,他们需要的医生至少应该了解他们的病。您怎能期望戈达尔治好您的病呢?他能估计酱汁不易消化,胃功能会发生障碍,但是他想不到莎士比亚作品会产生什么效果……因此,他的估计应用到您身上便是谬误,平衡遭到破坏,小浮沉子又浮了上来。他会发现您胃扩张,其实他不用检查就知道,他眼中早就有这个,您也看得见,他的单片镜里就有反映。”这种说话方式使我感到很累,迂腐的常识使我想:“戈达尔教授的眼镜里根本没有反映胃扩张,就如同德·诺布瓦先生的白背心下没藏着蠢话一样。”贝戈特又说:“我向您推荐迪—布尔邦大夫,这是位很聪明的人。”“想必是您的热情崇拜者吧。”我回答说。贝戈特显然知道这一点,于是我推论说同类相聚,真正的“陌生朋友”是很少见的。贝戈特对戈达尔的评论令我吃惊,与我的想法也绝然相反。我根本不在乎我的医生是否讨厌,我所期待于他的,是他借助一种我不知其奥妙的技艺对我的内脏进行试探,从而就我的健康发表毋庸置疑的旨喻。我并不要求他运用才智(这方面我可能胜过他)来试图理解我的才智;在我的想象中,智力本身并无价值,仅仅是达到外部真理的手段。聪明人所需要的治疗居然应该有别于傻瓜们的治疗,我对此深表怀疑,而且我完全准备接受傻瓜型的治疗。“有个人需要好大夫,就是我们的朋友斯万。”贝戈特说。我问难道斯万病了,他回答说:“是的,他娶了一个妓女。拒绝接待她的女士们,和她睡过觉的男人们,每天让斯万强咽下多少条蛇呀!它们使他的嘴都变了形。您什么时候可以稍加注意,他回家看到有那些客人在座时,那眉头皱得多么紧。”贝戈特在生人面前如此恶言中伤长期与他过从甚密的老友,而当着斯万夫妇的面他却轻声细语,对我来说这都是新鲜事,因为他一再对斯万说的那些甜言蜜语,是我的姑祖母无论如何也说不出口的。姑祖母这个人即使对所爱的人也常常说些使人不愉快的话,可是,她决不背着他们说些见不得人的话。贡布雷的交际圈与上流社会截然不同。斯万的圈子已经是向上流社会的过渡,向上流社会中反复无常的浪涛的过渡,它还不是大海,但已是环礁湖了。“这一切可别外传。”贝戈特在我家门口和我分手时说。要是在几年以后,我会这样回答:“我不会说出去的。”这是交际界的俗套话,是对诽谤者的假保证。那一天我也应该对贝戈特这样回答,因为当你作为社会人物活动时,你讲的全部话语不可能都由你自己来创造,不过当时我还没有学会这句俗套话。此外,姑祖母如遇到类似情况,会说:“你既然不愿我说出去,那何必告诉我呢?”她是位不好交际、好争爱斗的人。我不是这种人,所以我点点头,什么也没说。
\par 我所钦佩的某些文人花了好几年工夫,煞费苦心地与贝戈特建立了联系(始终是在书房内部的、暗中的文学交往),而我却一下子,而且不费吹灰之力便与这位名作家交上了朋友。众人在排队,但只能买到坏票,而你,你从谢绝公众的暗门走了进去,并买到最好的座位。斯万为我们打开这扇暗门,大概也在情理之中,就好比国王邀请子女的朋友们去皇家包厢或登上皇家游艇。希尔贝特的父母也同样对女儿的朋友开放他们所拥有的珍贵物品,并且,尤为珍贵的是,将他看做家庭的知己。但是当时我认为(也许有道理),斯万的友好表示是间接针对我父母的。还在贡布雷时期,我仿佛听说过,他见我崇拜贝戈特,便自告奋勇要带我去他家吃饭,父母却不同意,说我太小,太神经质,不能“出门”。我父母在某些人(恰恰是我认为最卓越的人)眼中的形象完全不同于我对他们的看法,当初那位粉衣女士对父亲未免过奖,现在我希望父母对斯万表示感谢,因为我刚刚得到的礼物是无价之宝。慷慨而彬彬有礼的斯万将礼物送给我,或者说送给他们,而似乎并不意识到它价值连城,就好比是卢伊尼\footnote{卢伊尼(1480—1532),意大利画家,达·芬奇的弟子。}壁画中那位迷人的、金发钩鼻的朝拜王一样。人们从前说斯万和画中人十分相似。
\par 回家时,我来不及脱大衣便对父母宣布斯万对我的这番优待,希望在他们心中唤起与我相同的激情,促使他们对斯万夫妇作出重要而关键性的“答谢”,然而,很不幸,他们似乎不太欣赏这种优待。“斯万介绍你认识贝戈特了?多么了不起的朋友!多么迷人的交往!这算登峰造极了!”父亲讽刺地大声说。不巧的是,我接着说贝戈特丝毫不欣赏德·诺布瓦先生。
\par “那还用说,”父亲说,“这恰好证明他是个假装聪明、不怀好意的人。我可怜的儿子,我看你连常识也没有了,居然和会断送你前程的人们为伍,我真难过。”
\par 我对斯万家的拜访原来就已经使父母很不高兴。与贝戈特的相识,在他们看来,仿佛是第一个错误——他们的软弱让步(祖父会称之为“缺乏远见”)——的必然恶果。我感到,只要我再补充说这位对德·诺布瓦先生不抱好感的坏人认为我很聪明,那么,父母就会暴跳如雷。当父亲认为某人,例如我的一位同学误入歧途——好比此时此刻的我——时,如果他看到这位迷途者受到他所不齿的人的赞许,会更坚信自己的严厉判断是正确的,更认为对方恶劣。我似乎听见他在大喊:“当然啦,这是一路货!”这句话使我万分恐惧,它仿佛宣布某些变化、某些十分模糊、十分庞大的变化将闯入我那安宁的生活之中。然而,即使我不说出贝戈特对我的评价,我也无法擦去父母已经得到的印象,因此,破罐子破摔。何况我认为他们极不公道,坚持错误。我不再希望,甚至可以说我不再想法让他们回到公正的立场上来。然而,当我开口时,我感到贝戈特对我的赏识会使他们惊慌失措——因为此人将聪明人当做蠢人,此人被高雅的绅士嗤之以鼻,此人对我的夸奖(我所羡慕的)会使我走上邪路——因此,我羞愧地、低声地最后带上一句:“他对斯万夫妇说他认为我很聪明。”一条狗中了毒在田野上胡乱啃草,而这种草恰恰为它解了毒,我也一样,在不知不觉中我说出了世上唯一能克服父母对贝戈特的偏见的话——而我所能做的最好论证,所能说的一切赞同都无法消除这种偏见。顷刻之间,形势突变。
\par “啊!……他说你很聪明?”母亲说,“我很高兴,因为他是个颇有才气的人。”
\par “真的!这是他说的?”父亲接着说,“……我丝毫不否定他的文学才能,这是有口皆碑的。可惜他生活不太检点,诺布瓦老头暗示过。”父亲这样说,他并不意识到我刚才出口的那句话具有神妙的至高威力,贝戈特的堕落习性和拙劣判断力在这威力面前败下阵来。
\par “啊!亲爱的,”母亲插嘴说,“有什么证据肯定这是真的呢?人们总爱瞎议论。再说,德·诺布瓦先生虽然为人和气,但并不永远与人为善,特别是对待和他不对路的人。”
\par “这倒也是,我也有所察觉。”父亲说。
\par “再说,既然贝戈特欣赏我可爱的乖儿子,许多地方我们应该原谅他。”母亲一面说,一面用手指抚摸我的头发,梦幻的眼光久久地凝视我。
\par 在贝戈特的这个裁决以前,母亲早就对我说过,有朋友来时,我也可以邀请希尔贝特来吃午后点心。但是我不敢邀请她,这有两层原因,一是希尔贝特家从来只喝茶,而我们家却相反,除了茶以外,母亲坚持要朱古力,我害怕希尔贝特会认为这十分粗俗,从而极度蔑视我们。另一个原因就是我始终无法解决的礼节问题。每次我去斯万夫人家,她总是问我:
\par “令堂大人可好?”
\par 我向母亲提过,希尔贝特来她能不能也这样问,因为这一点好比是路易十四宫中“殿下”的称呼,至关重要。但是妈妈根本听不进我的话。
\par “不行,我不认识斯万夫人呀。”
\par “可她也不认识你。”
\par “我没说她认识我。不过我们不一定对一切事情采取同样的做法。我要用另一种方式来款待希尔贝特,和斯万夫人对你的接待方式不同。”
\par 我并不信服,所以宁可不邀请希尔贝特。
\par 我离开父母去换衣服,在掏衣袋时突然发现斯万家的膳食总管在领我进客厅时递给我的那个信封。我现在身边无人,便拆开来看,里面有一张卡片,上面写着我应该将胳臂伸给哪位女士,并领她去餐桌就坐。
\par 就在这个时期,布洛克使我的世界观完全改变了,他向我展开了新的幸福的可能性(后来变成痛苦的可能性),因为他告诉我女人最爱的莫过于交媾了——与我去梅塞格利丝散步时的想法相反。在这次开导以后,他又给我第二次开导(其价值我在很久以后才有所体会):他领我头一次去妓院。以前他曾对我讲那里有许多美女供人占有,但她们在我的脑海中面目模糊,后来我去了妓院,才对她们具有了确切印象。如果说我对布洛克——由于他的“福音”,即幸福和对美的占有并非可望不可即,甘心放弃实属愚蠢——充满感激的话(如同感激某位乐天派医生或哲学家使我们盼望人世间的长寿,盼望一个并非与人世完全隔绝的冥间),那么几年以后我所光顾的妓院对我大有益处,因为它们对我提供幸福的标本,使我往女性美上添加一个我们无法臆造的因素,它绝非仅仅是从前的美的综合,而是神妙的现在,我们所无法虚构的现在;它只能来自现实,超于我们智力的一切逻辑创造之上,这就是:个体魅力。我应该将这些妓院与另一些起源较近但效用相似的恩人们归为一类,这些恩人即带插图的绘画史、交响音乐会及《艺术城市画册》,因为在它们以前,我们只能通过别的画家、音乐家、城市来毫无激情地想象曼坦纳、瓦格纳和西埃内的魅力。不过,布洛克带领我去而他本人长久不去的那家妓院规格较低,人员平庸而且很少更新,因此我无法满足旧的好奇心,也产生不了新的好奇心。客人所点要的女人,妓院老鸨一概佯称不认识,而她提出的又尽是客人不想要的女人。她在我面前极力夸奖某一位,笑着说包我满意(仿佛这是稀有珍品和美味佳肴似的):“她是犹太人。您不感兴趣?”(可能由于这个原因,她叫她拉谢尔。)她愚蠢地、假惺惺地激动起来,想以此打动我,最后发出一种近乎肉欲快感的喘息声:“你想想吧,小伙子,一个犹太女人,您肯定要神魂颠倒的,呃!”这位拉谢尔,我曾见过她一面,但她没有看见我。此人一头棕发、不算漂亮,但看上去不蠢,她用舌尖舔嘴唇,放肆地向被介绍给她的嫖客微笑。我听见她和他们谈了起来。在她那张窄窄的小脸两侧是鬈曲的黑发,它们极不规则,仿佛是中国水墨画中的几条影线。老鸨一而再、再而三地推荐她,夸奖她聪明过人,并受过良好教育,我每次都答应一定专程来找拉谢尔(我给她起了个绰号:“从上帝身边来的拉谢尔”\footnote{这是法国作曲家阿莱维(1799—1862)的著名歌剧《犹太女人》第四幕中著名乐段的开始。})。然而,第一天晚上,我就曾听见拉谢尔临走时对老鸨说:
\par “那么说定了,明天我有空,要是有人来,您可别忘了叫我。”
\par 这些话使我在她身上看到的不是个体,而是某一类型的女人,其共同习惯是晚上来看看能否赚一两个路易,她的区别只在于换个说法罢了:“如果您需要我,或者如果您需要什么人。”
\par 老鸨没有看过阿莱维的歌剧,不明白我为什么老说“从上帝身边来的拉谢尔”,但是,不理解这个玩笑并不等于不觉得它可笑,因此她每次都开怀大笑地对我说:
\par “怎么,今晚还不是您和‘从上帝身边来的拉谢尔’结合的时辰?您是怎么说来着,‘从上帝身边来的拉谢尔’,啊,这可真妙!我要给你们俩配对。瞧着吧,您不会后悔的。”
\par 有一次我差点下了决心,但她“正在接客”,另一次她又在接待一位“理发师”,此人是位老先生,他和女人在一起时,只是往她们散开的头发上倒油,然后进行梳理。我等得不耐烦,几位常来妓院的身份卑微的女人(她们自称女工,但始终无工作)走过来给我沏药茶,并和我长谈,她们那半裸或全裸的身体使严肃的话题变得简明有趣。我后来不再去这家妓院。在这以前,我看到老鸨需要家具,我想对她表示友好,便从莱奥妮姨母留给我的家具中挑了几件——特别是一张长沙发——送给她。原先我根本看不见它们,因为家里没有地方放,父母不让人把它们搬进来,于是它们只能堆在库房里。然而我在妓院又见到了它们,我看见那些女人在使用它们,于是,昔日充溢在贡布雷的那间姨母卧室的种种魔力再次显现,但却在磨难之中,因为我迫使它们手无寸铁地承受残酷的接触!我的痛苦甚于听任一位死去的女人遭人蹂躏。我不再去那位鸨母那里,我感到家具有生命,它们在哀求我,就像波斯神话故事一样:神话里的物品表面上似乎没有生命,但内部却隐藏着备受折磨、祈求解脱的灵魂。此外,由于记忆力向我提供的回忆往往不遵守时序,而仿佛是左右颠倒的反光,因此,我在很久以后才想起多年以前我曾在这同一张长沙发上头一次和一位表妹品尝爱情的乐趣,当时我不知道我们去哪里好,她便想出了这个相当冒险的主意:利用莱奥妮不在场的时机。
\par 其他许多家具,特别是莱奥妮姨母那套古老而漂亮的银餐具,我都不顾父母的反对将它们卖了,为的是换钱,好给斯万夫人送更多的鲜花。她在接受巨大的兰花花篮时对我说:“我要是令尊,一定给您找位指定监护人。”然而当时我怎会想到有一天我将特别怀念这套银器,怎会想到在对希尔贝特的父母献殷勤这个乐趣(它可能完全消失)之上我将有其他乐趣呢?同样,我决定不去驻外使馆,正是为了希尔贝特,正是为了不离开她。人往往在某种暂时情绪下作出最后决定。我很难想象希尔贝特身上那种奇异的物质,那种在她父母身上和住宅中闪烁从而使我对其他一切无动于衷的物质,会脱离她而转移到别人身上。这个物质确实未变,但后来在我身上产生了绝对不同的效果,因为,同一种疾病有不同的阶段,当心脏的耐力随着年龄而减弱时,它再无法承受有损健康的美味食品。
\par 父母希望贝戈特在我身上所发现的智慧能化为杰出的成就。在我还不认识斯万夫妇时,我以为我无心写作是因为我不能自由地和希尔贝特见面,是因为我焦灼不安。可是当他们向我敞开家门时,我在书桌前刚刚坐下便又起身向他们家跑去。我从他们家归来,独自一人,但这只是表象,我的思想仍无法抗拒话语的水流,因为在刚才几个小时里,我机械地听任自己被它裹挟。我独自一人,但继续臆造可能使斯万夫妇高兴的话语,而且,为了使游戏更有趣,我扮演在场的对话者,我对自己提出虚构的问题,目的是使我的高见成为巧妙的回答。这个练习虽然在静默中进行,但它却是谈话,而不是沉思。我的孤独是一种精神沙龙,在这个沙龙中,控制我话语的不是我本人,而是想象的对话者;我表述的不是我认为真实的思想,而是轻手拈来的、缺乏由表及里的反思的思想,因此我感到一种纯粹被动的乐趣,好比因消化不良而待着不动时所感到的被动乐趣。
\par 如果我不是作长期写作打算的话,那我也许会急于动笔。既然我这个打算确定无疑,既然再过二十四小时(明天是一个空白的框框,我还没有进去,所以框中的一切安排得井然有序),我的良好愿望便能轻易地付诸实现,那又何必挑一个写作情绪不佳的晚上来动笔呢?当然,遗憾的是,随后的几天也并非写作的吉日。既然已经等待了好几年,再多等三天又有何妨。我深信到了第三天,我一定能写出好几页,所以我对父母绝口不提我的打算。我宁愿再忍耐几个小时,然后将创作中的作品拿去给外祖母看,以安慰她,使她信服。可惜的是,第二天仍然不是我热切盼望的广阔的、行动的一天。当这一天结束时,我的懒惰,我与内心障碍的艰苦斗争仅仅又多持续了二十四小时,几天以后,我的计划仍是纸上谈兵,我也就不再期望它能立即实现,而且也再没有勇气将这件事作为先决条件了。于是我又开始很晚睡觉,我不必再抱着明晨动笔的确切幻想早早躺下。在重新振作以前,我需要休息几天。有一天(唯一的一次),外祖母鼓起勇气,用失望的温柔口气责怪说:“怎么,你这项写作,没有下文?”我怨恨她居然看不出我一旦决定决不更改。她的话使我将付诸实行的时间又往后推,而且也许推迟很久,这是因为她对我的不公正使我烦恼,而我也不愿意在烦恼的情绪下动手写作。她意识到她的怀疑盲目地干扰了我的意图,向我道歉,并亲吻我说:“对不起,我再什么也不说了。”而且,为了不让我泄气,她说等我身体好了,写作会自然而然地开始。
\par “何况,”我心里想,“去斯万家消磨时光,我这不是和贝戈特一样吗?”我父母几乎认为,既然我和名作家同在一沙龙,那么,在那里度过的时光一定能大大促进天才,虽然我十分懒惰。不从本人内部发挥天才,而从别人那里接受天才,何其荒谬!这就好比是一个根本不讲卫生、暴食暴饮的人仅仅依靠和医生经常共餐而居然保持健康!然而,这种幻想(它欺骗我和我父母)的最大受害者是斯万夫人。当我对她说我来不了,我必须留在家里工作时,她那副神气仿佛认为我装腔作势,既愚蠢又自命不凡。
\par “可是贝戈特要来的。难道您认为他的作品不好?不久以后会更好的,”她接着说,“他给报纸写的文章更尖锐,更精炼,不像他的书那样有点嗦。我已经安排好,请他以后给《费加罗报》写社论,这才是the right man in the right place(最恰当的人在最恰当的位置上)。”
\par 她又说:“来吧,他最清楚您该怎么做。”
\par 她正是为我的事业着想才叮嘱我第二天无论如何要去和贝戈特同桌吃饭(正好比志愿兵和上校见面),她似乎认为文学佳作是“通过交往”而产生的。
\par 这样一来,无论是斯万夫妇,还是我父母——他们在不同时刻似乎应该阻止我——都再没有对我轻松的生活提出异议,这种生活使我能够尽情地,如果不是平静地,至少是陶醉地和希尔贝特相见。在爱情中无平静可言,因为人们永远得寸进尺。从前我无法去她家,便把去她家当做高不可攀的幸福,哪里会想到在她家中将出现新的烦恼因素。当她父母不再执意反对,当问题终于得到解决时,烦恼又以新的形式出现。从这个意义上讲,可以说每天都开始一种新友谊。夜间归来,我总想到某些对我们的友谊至关重要的事,我必须和希尔贝特谈,这些事无穷无尽也永不相同。但我毕竟感到幸福,而且这幸福不再受任何威胁。其实不然,威胁终于出现了,而且,遗憾的是,它来自我认为万无一失的方面,即希尔贝特和我。那些使我感到宽慰的事,那个我所认为的幸福,原本应该引起我的不安。我们在恋爱中往往处于一种反常状态,觉得最简单最常见的小事具有最大的严重性。我们之所以感到幸福,是因为在我们心中有某种不稳定的东西,我们不断努力去维持它,而且,只要它未转移,我们几乎不再觉察。确实,爱情包含持久的痛苦,只不过它被欢乐所冲淡,成为潜在的、被推迟的痛苦,但它随时可能剧烈地爆发出来(如果人们不是如愿以偿,那么这痛苦早就爆发了)。
\par 有好几次我感到希尔贝特不愿我去得太勤。的确,她父母越来越深信我对她产生良好影响,我想和她见面时只需让他们邀请我就行了,因此我想道:“这样一来,我的爱情再不会有任何危险。既然他们站在我一边,他们对希尔贝特又很有权威,我又有什么可担心的呢?”然而,当她父亲在某种程度上违背她的心愿而邀请我时,她流露出不耐烦的情绪,这些表示使我产生疑问:我原先所认为的幸福的保障莫非恰恰是使幸福中断的秘密原因?
\par 我最后一次去看希尔贝特时,下着雨。她被邀参加舞蹈训练,但她和那家人不熟,不能带我去。那天我比往常服用了更多的咖啡因以抵御潮湿。斯万夫人大概因为天气不好,或者因为对聚会的那家人有成见,所以在女儿出门时很生气地唤住了她:“希尔贝特!”并且指指我,表示我是来看她的,她应该留在家里陪我。斯万夫人出于对我好意而发出——或者喊出——“希尔贝特”,但是希尔贝特一面放下衣物一面耸耸肩,我立刻意识到这位母亲在无意中加快了我和女友逐渐分手的过程,而在此以前,这个过程也许还可以阻止。“没有必要天天去跳舞。”奥黛特对女儿说,那副明哲的神气大概是她以前从斯万那里学来的。接着她又恢复奥黛特的常态,和女儿讲起英语来,立即,仿佛有一堵墙将希尔贝特的一部分遮盖起来,仿佛有一个邪恶的精灵将我的女友从我身边裹胁而去。对于我们所熟悉的语言,我们可以用透明的思想来替代不透明的声音,但是我们所不熟悉的语言却像一座门窗紧闭的宫殿,我们所爱的女人可以在那里与人调情,而我们被拒之门外,绝望已极却无能为力,什么也看不见,什么也阻止不了。这场英语谈话中常出现某些法语专有名词,它们仿佛是线索,使我更为不安。要是在一个月前,我会一笑了之,然而此刻,虽然她们一动不动地在咫尺之内谈话,我却感到这是残酷无情的劫持,剩下我孤苦伶仃。最后,斯万夫人总算走开了。这一天,也许因为希尔贝特埋怨我身不由己地阻碍她去跳舞,也许因为我故意比往日冷淡(我猜到她生我的气),她脸上没有一丝欢乐,干涩木然,闷闷不乐,仿佛整个下午都在怀念我的来访使她未能跳成的四步舞,仿佛整个下午都在责怪所有的人,当然首先是我,责怪我们竟不理解她如此钟情于波士顿舞的奥妙原因。她仅仅时不时地和我交换几句话,天气如何啦,雨愈下愈大啦,座钟走快了啦,中间还夹着沉默和单音节字。我作绝望挣扎,执意要糟蹋这些原本应该献给友谊和幸福的时刻。我们所说的一切都是那么生硬,那么空洞而荒谬,这一点倒使我得到安慰,因为希尔贝特不会将我平庸的思想和冷漠的语气当真的。尽管我说的是:“从前这个钟仿佛走得慢。”她理解我的意思是:“你真坏!”在这个雨天,我顽强奋斗,延长这些没一丝阳光的话语,但一切努力均属枉然,我知道我的冷漠并非如佯装那般凝固不变,希尔贝特一定感觉到,既然我已说了三遍“白天变短了”,如果我再贸然重复第四遍,那我一定难以自制,会泪如雨下。她现在的模样,眼中和脸上毫无笑意,忧愁的眼神和阴郁的脸色充满令人懊丧的单调。这张脸几乎变得丑陋,就像那单调枯燥的海滩,海水已经退得很远,它在那固定不变的封闭的地平线之内的闪光千篇一律,令人厌烦。最后,我看到希尔贝特仍然不像我好几个小时以来所期望的那样回心转意,便对她说她不够意思。“你才不够意思呢。”她回答说。“我怎么了?”我自问有什么地方做得不对,一无所获,便又问她。“当然啦,你认为自己很好!”说完后她笑了很久。于是我感到,我无法达到她的笑声所表达的另一层思想,另一层更难以捉摸的思想,这是多么痛苦的事。她的笑似乎意味着:“不,不,我根本不信你的话。我知道你爱我,不过我无所谓,我不把你放在眼里。”然而我又提醒自己,笑毕竟不是一种明确的语言,我怎能肯定自己理解正确呢,何况希尔贝特的话还是富有感情的。“我什么地方不好?告诉我,我一定按你的话去做。”“不,没必要,我没法和你解释。”刹那间,我害怕她以为我不爱她,这是另一种同样强烈的痛苦,它要求另一种逻辑。“你要是知道使我多伤心,那你会告诉我了。”如果她怀疑我的爱情,那么我的伤心会使她高兴,但此刻却相反,她很生气。我意识到自己判断错误,决心不再相信她的话,随她说:“我一直爱你,有一天你会明白的。”(罪人们往往说他们的清白无辜将大白于天下,然而,出于神秘的原因,这一天永远不会是他们受审的那一天)。我鼓起勇气,突然决定不再和她见面,但暂时不告诉她,因为她不会相信这话的。
\par 你所爱的人可能给你带来辛酸的悲伤,即使当你被与她(他)无关的忧虑、事务、欢乐缠住而无暇顾及也罢。但是,如果这悲伤——例如我这次的悲伤——诞生于我们浸沉在与她见面的幸福之中时,那么,在我们那充满阳光的、稳定而宁静的心灵中便会产生急剧的低压,从而在我们身上掀起狂烈风暴,使我们没有信心与它抗争到底。此刻在我心中升起的风暴无比凶猛,我告辞出来,晕头转向,遍体鳞伤,同时感到只有再回去,随便找一个借口再回到希尔贝特身边去,我才能喘过气来。但是她会说:“又是他!看来我对他可以为所欲为了。他总会回来的,走的时候越痛苦,回来时就越顺从。”我的思想以无法抗拒的力量将我拉回到她身边。当我到家时,这些变幻不定的风向,这种内心罗盘失调的现象依然存在,于是我动笔给希尔贝特写了些前后矛盾的信。
\par 我即将经历艰难的处境,人在一生中往往会多次面临此种处境,而每一次,即在不同的年龄,人们所采取的态度也不相同,尽管他们的性格或天性并无改变(我们的天性创造了爱情,创造了我们所爱的女人,甚至她们的错误)。此时,我们的生命分裂为二,仿佛全部分放在相对的天平盘上。一个盘里是我们的愿望,即我们不要使我们所爱但不理解的人不高兴,但不能过于谦卑,巧妙地稍稍冷落她们,别让她们感到她们是须臾不可缺少的人,因为这种感觉会使她们离开我们。另一个天平盘里是痛苦(并非确定的、部分的痛苦),它与前一种状态相反,只有当我们不再试图讨好这个女人,不再让她相信她对我们可有可无,从而再去接近她时,这种痛苦才有所缓解。如果我们从装着自尊心的天平盘上拿去被年龄耗损的一部分毅力,往装着悲伤的天平盘里加进我们逐渐获得的、并任其发展的生理痛苦,那么天平所显示的将不是我们二十岁时的勇敢决定,而是我们年近半百时的决定——它十分沉重、缺乏平衡力,令人难以承受。何况,处境在不断重复中有所变化,我们在中年或晚年时,可能乐于将某些习惯与爱情混为一谈(这对爱情是致命的),而青年时代却不承认这些习惯,它受到其他许多义务的约束,不能随意支配自己。
\par 我给希尔贝特刚写了一封信来发泄怒火,但也故意安排了几句貌似偶然的话,女友可以抓住这些救命圈与我和解;但片刻以后,风向变了,我写下一些温情脉脉的句子,使用某些甜蜜而悲伤的短语,例如“永不再”之类。使用者认为这些词句感人肺腑,而那位读信的女人则会认为枯燥乏味,或者她觉得这统统是假话,将“永不再”解释为“今晚如果你需要我”;或者她相信这是真话,因此意味着永远分手(和我们所不爱的人分手何足为惜)。既然我们正在恋爱,我们便不可能像将来不再恋爱时那样行事,我们无法想象那女人真正的心理状态,因为,虽然明知她冷漠无情,但我们仍然遐想她以爱恋者的口吻说话(我们这样做是为了用美丽的幻想欺骗自己,或是为了解脱沉重的悲伤)。我们面对所爱的女人的思想举止,犹如古代最早的科学家面对大自然现象(科学尚未建立,未知事物尚未被解释),茫然失措,甚至更糟。我们看不到因果关系,看不到这个现象和那个现象之间的联系,我们眼中的世界像梦幻一般缥缈不定。当然,我试图克服这种紊乱,试图寻找原因。我甚至试图做到“客观”,认真考虑希尔贝特在我眼中的地位,我在她眼中的地位,以及她在别人眼中的地位,它们是多么悬殊!如果我看不到这种悬殊性,那么我就会把女友简单的殷勤看做炽热爱情的流露,把我自己滑稽可笑、有失体面的行为看做对美貌的简单优雅的倾爱。但是我也害怕走到另一个极端,以致把希尔贝特的不准时赴约和恶劣情绪看做是无法改变的敌意。我试图在这两种同样歪曲真相的观点中找出正确反映事物的第三种观点,我为此而作的种种计算稍稍缓和了我的痛苦。我决定第二天去斯万家(也许是服从于这些计算的结果,也许是我使计算表达了我的心愿),我很高兴,就像一个人本不愿旅行,并为此烦恼多时,最后来到车站才下决心取消旅行,于是高高兴兴回到家中解开行装。在人们犹豫不决时,采取某种决定的念头(除非不采取任何决定,从而使念头丧失生命力)像一粒富有生命力的种子,勾画出完成行动后所产生的激情的种种轮廓,因此,我对自己说,不再与她见面仅仅是想法而已,我却像实有其事那样感到痛苦,何其荒唐!再说,既然我最终会回到她身边,又何必作如此痛苦的决定和允诺呢?
\par 然而,这种友好关系的恢复仅仅持续了片刻,即我去斯万家的路上。它的破灭并不是因为膳食总管(他很喜欢我)对我说希尔贝特不在家(当晚我从遇见她的人口中得知她确实不在家),而是他的说话方式:“先生,小姐不在家,我向您担保她确实不在家。先生如果想打听清楚,我可以去叫小姐的随身女仆。先生尽可相信我会尽一切努力使先生高兴的。小姐要是在家,我会立刻领先生去见她。”这番话的唯一重要意义在于它的自发性,因为它对矫饰的言语所掩盖的难以想象的现实进行了X光透视(至少是粗略的)。这番话证明,在希尔贝特身边的人眼中,我是个纠缠者。这些话刚从他口中说出来,便在我心中激起仇恨,当然,我乐于将他,而不是将希尔贝特,当做仇恨的对象。我将对她的全部愤怒集中倾泻在他身上,这样一来,我的爱情摆脱了愤怒,单独存留下来。然而,这番话也表明短期内我不应去找希尔贝特。她会写信向我道歉的。尽管如此,我不会马上去看她,我要向她证明没有她我照样可以活下去。再说,等我收到希尔贝特的信后,我能更轻易地忍受与她暂不见面之苦,因为只要我想见她便一定能见到。为了承受这故意设计的分离而不至过于痛苦,我的心必须摆脱可怕的疑虑,例如莫非我们从此绝交,莫非她与别人订婚走了,被劫走了。接下来的几天和新年那个星期十分相似,因为当时我不得不在没有希尔贝特的情况下继续生活。不过,当时我很清楚,那个星期一结束,她便会回到香榭丽舍大街,我便会像以前一样见到她,另一方面,只要新年假不结束,我去香榭丽舍大街也没有用。因此,在那个已经遥远的、愁闷的星期中,我平静地忍受忧愁,既无恐惧也不抱希望。但现在却不然,这后一种感情,即希望,几乎像恐惧一样,使我痛苦得难以忍受。
\par 当天晚上我没有收到希尔贝特的信,我归咎于她的疏忽和忙碌,深信第二天清晨的信件中肯定有她的来信。我每天都期待早上的信件,我的心在剧烈跳动,而当我收到的是别人的来信,而不是希尔贝特的来信时,我垂头丧气。有时我一封信也没有,这倒不见得更糟,因为另一个女人对我的友好表示会使希尔贝特的冷漠更为无情。我接着便寄望于下午的信件。即使在邮局送信的钟点以外,我也不出门,因为她很可能让人送信来。终于,天色已晚,邮递员或斯万家的仆人都不会登门了,于是我便将平静下来的希望转寄于第二天上午。我之所以这样做,是因为我认为我的痛苦不会持久,我必须不断地予以姑且说更新吧。悲伤依旧如前,但它不再像以前那样一成不变地延长最初的激情,而是每日多次地重新开始,激情的更新如此频繁,以至于它最后——它是纯粹物质的、暂时的状态——稳定在那里,因此,前一期待所引起的惶惑还未平静下来,第二次期待便已出现,我每天无时无刻不处在焦虑之中(忍受一个小时也非易事)。这次的痛苦,比起从前那个新年假日来,要严峻百倍,因为这一次我并非完完全全接受痛苦,而是时时盼望结束痛苦。
\par 最后我毕竟接受了痛苦,我明白这是决定性的,我将永远放弃希尔贝特,这也是为我的爱情着想,因为我决不愿意她在回忆中仍然蔑视我。从此刻起,当她给我订约会时,我甚至往往允诺,免得她认为我在为爱情赌气,但到最后一刻钟,我写信对她说我不能赴约,并一再表示遗憾,仿佛我在和某位我不想见的人打交道。我觉得,这些一般用于泛泛之交的表示歉意的客套话,比起对所爱的女人佯装的冷淡口气来,更能使希尔贝特相信我的冷漠。我不用言词,而用不断重复的行动,以便更好地说明我无意和她见面;等我真正做到这一点,她也许会重新对我感兴趣。可惜,这是空想。不再和她见面以便重新唤起她和我见面的兴趣,这种办法等于永远失去她,因为,首先,当这个兴趣重新苏醒时,为了使它持久,我便不能立刻顺从它,其次,到那时最严酷的时刻已成过去,因为我最需要她的是此时此刻。我真想警告她,很快,这种分离的痛苦将大大减弱,我将不会像此时此刻那样,为了结束痛苦而想到投降、和解,重新和她相见。将来,等到希尔贝特恢复对我的兴趣,而我也可以毫无危险地向她表达我的兴趣时,这种兴趣经不起如此漫长的分离的考验,将不复存在。希尔贝特对我来说将成为可有可无的人。我很清楚这一点,但我没法对她讲。如果我告诉她长久不见面我不会再爱她,那么她会以为我的目的仅仅是让她赶快召唤我。在此期间,我总是挑希尔贝特不在家,她和女友外出不回家吃饭的日子去拜访斯万夫人(对我来说她又成为往日的她,当时我很少看见她女儿,少女不来香榭丽舍大街时,我便去槐树大街散步),好让希尔贝特明白,我之所以不见她,并非被别的事缠身,也并非身体欠佳,而是不愿意见面,尽管我作了相反的表白。这种办法使我比较顺利地坚持了分离。既然我能听见别人谈到希尔贝特,她肯定也听见人们谈到我,而且她会明白我并不依恋她。像所有处于痛苦中的人一样,我觉得自己的处境虽然不妙,但并不是最糟的,因为我可以随意进出希尔贝特的家(虽然我决不会利用这项特权)。如果痛苦过于剧烈,我可以使它中止。所以我的痛苦每天都是暂时的,这样说还不够,每小时中有多少次(但此刻已无决裂的最初几个星期里那种令人窒息的、焦虑的期待——在我回到斯万家以前),我对自己朗诵有一天希尔贝特将寄给我,或者亲自送来的那封信!这个时时浮现在眼前的、想象的幸福,帮助我忍受了真正的幸福的毁灭。不爱我们的女人犹如“失踪者”,尽管我们知道再无任何希望,我们却仍然期待,等待稍稍一点儿动静,稍稍一点儿声响。好比母亲虽然明知作危险勘察的儿子已葬身大海,但仍时时想象他会奇迹般得救,而且即将身强体壮地走进门来。这种等待,根据回忆的强弱及器官的抗力,或者使母亲在多年以后承认这个事实,逐渐将儿子遗忘并生活下去,或者使母亲死去。另一方面,一想到我的悲伤有利于我的爱情,我便稍稍得到宽慰。我探望斯万夫人而不和希尔贝特见面,这种访问每次都是残酷的,但是我感到它会改善希尔贝特对我的看法。
\par 每次去看斯万夫人以前,我总要打听清楚她女儿是不是确实不在家,我这样做不仅仅是因为我决心与她断交,也因为我仍希望和解,这个希望重叠在断交的意图之上(希望和意图很少是绝对的,至少并不总是绝对的,因为人的心灵有一条规律,它受突然涌现的不同回忆所左右,这规律即间断性),并且使我意识不到这个意图的残酷性。我很清楚希望极为渺茫。我像一个穷人,如果他在啃干面包时心想等一会儿也许有位陌生人会将全部家财赠给他,那么他就不会那么伤心落泪了。为了使现实变得可以忍受,我们往往不得不在心中保留某个小小的荒唐念头。因此,如果不和希尔贝特相遇,我的希望会更完好无损——虽然与此同时,我们的分离更成为现实。如果我在她母亲家与她迎面相遇,我们也许交换几句无法弥补的话,那会使决裂成为永恒,使我的希望破灭,另一方面,它所产生的新焦虑会唤醒我的爱情,使我难以听天由命。
\par 很久以前,早在我和她女儿决裂以前,斯万夫人就曾对我说:“您来看希尔贝特,这很好,不过希望您有时也来看看我,但不要在我的舒弗莱里日\footnote{舒弗莱里,奥芬巴赫轻歌剧中的主人公,此处指正式接待日。}来,客人很多,会使您厌烦,挑别的日子来,辰光稍晚我总在家。”因此,我的拜访仿佛仅仅是满足她很久以前表达的愿望。我在时辰很晚、夜幕降临、我父母即将吃晚饭时出门去斯万夫人家,我知道在访问中不会遇见希尔贝特,但我一心想的仅仅是她。那时的巴黎不像今天这样灯火辉煌,即使市中心的马路也无电灯,室内的电灯也少见,而在这个当时被认为偏僻的街区里,底层或比底层略高的中二层(斯万夫人通常接待客人的房间就在这里)的客厅射出明亮的灯光照亮街道,使路人抬眼观看。他自然将这灯光,将这灯光的明显而隐晦的起因与大门口那几辆华丽马车联系起来。当他看到一辆马车起动时,便颇有感触地认为奥秘的起因发生了变化,其实只是车夫怕马匹着凉,因此让马匹来回溜达,这种走动给人留下深刻印象,因为胶皮车轮静寂无声,它使马蹄声显得更清脆、更鲜明。
\par 在那些年代里,不论在哪条街上,只要住房离人行道不是太高,从街上就能看见室内的“冬季花园”(如今只能在斯达尔\footnote{斯达尔是法国文人及出版商(1814—1886)。}新年礼品丛书的凹版照片中见到),这种花园与如今路易十六式客厅的装饰——极少鲜花,长颈水晶玻璃瓶中只插着单独一枝玫瑰花或日本蝴蝶花,再多一枝也插不进——恰恰相反,它拥有大量的、当时流行一时的室内装饰性植物,而且在安排上毫无讲究,它体现的不是女主人如何冷静地采用毫无生气的装饰,而是她如何热切爱着活生生的植物。它更使人想到当时流行于公馆中的便携式微型花房。元月一日凌晨,人们将这种花房放在灯下——孩子们没有耐心等到天亮——放在新年礼品中间,而它是最美的礼品,因为人们可以用它培育植物,从而忘记光秃秃的冬天。冬季花园不仅和这种花房相似,还和花房旁边的那本精美书本上的花房图画相似,那幅画也是新年礼物,但不是赠给孩子们,而是赠给书中女主人公莉莉小姐的,它使孩子们如此着迷,以至他们现在虽已老迈,但仍然认为那些幸运年代的冬天是最美好的季节。过路人踮起脚往往就能看见在这冬季花园的深处,在各式各样的乔木的内侧(从街上看进去,亮着灯的窗子仿佛是儿童花房——图画或实物——的玻璃罩),一位身着礼服、纽扣上插着一枝栀子花或石竹花的男人,正站在一位坐着的女士面前,两人的轮廓影影绰绰,如同一块黄玉中的两个凹雕,客厅充满了茶炊——当时是新进口货——的雾气,这种茶炊雾气今天仍然有,但人们习以为常,不再理会。斯万夫人很重视这种“茶”,她认为对男人说“您每天晚一点来,我总在家,您来喝茶”这句话既新颖又有魅力,她暂时用英国口音,并伴之以温柔甜蜜的微笑,因此对方十分认真,神情严肃地向她鞠躬,仿佛此事至关重要,奇异不凡,人们应该肃然起敬,决不可掉以轻心。斯万夫人客厅里的鲜花不仅具有装饰性,除了上述原因以外,还有一个与时代无关,仅与奥黛特旧日生活有关的原因。她曾经是交际花,大部分时间和情人在一起,也就是说在她家中,因此她要安排好自己的家。在体面女人家里所看到的,并且被体面女人认为重要的东西,对交际花来说就更为重要。她每天的高峰时刻不是穿衣去给别人观赏,而是脱衣和男人幽会。她无论穿便袍还是穿睡衣,都必须像出门打扮一样风度翩翩。别的女人将珠宝炫耀于外,而她却将它藏于内室。这种类型的生活,要求并且使人习惯于一种隐秘的、几乎可以说是漫不经心的奢侈。斯万夫人的这种奢侈也扩及花草。在她的安乐椅旁总有一个硕大的水晶玻璃盆,里面全都是帕尔马蝴蝶花或是花瓣散落在水中的雏菊花。花盆似乎向来访者证明这是她所喜好的消遣——正如她喜欢独自喝茶一样,可惜被不速之客打断了。这种消遣甚至比喝茶更亲密,更神秘。因此,当来客看到展示在她身旁的鲜花时,会情不自禁地想向她道歉,仿佛他翻看了奥黛特尚未合上的书的标题,而标题会泄露她读的是什么,也就是说她此刻想的是什么。何况鲜花比书籍更有生命。人们走进客厅拜访她,发现她并非单独一人而惶惑不安;人们和她一同回家,看到客厅并非空寂而惶惑不安。这些鲜花在客厅中占有神秘的地位,它们与人所不知的女主人的生活密切相关。它们不是为来访者准备的,而是仿佛被奥黛特遗忘在那里。它们以前和现在都与奥黛特密谈,因此,人们害怕打扰它们,同时目不转睛地盯着那如稀释水彩般的、淡紫色的帕尔马蝴蝶花,徒劳地试图窥见其中的奥妙。从十月底起,奥黛特尽量按时回家喝茶,当时它仍然称作five o'clock tea(五点钟的茶),因为奥黛特听说(并喜欢向别人重复)维尔迪兰夫人办沙龙正是为了告诉别人她这个钟点一定在家。奥黛特也想办一个沙龙,与维尔迪兰沙龙同一类型,但是更自由,用她的话说,senza rigore\footnote{意大利文:无拘束。}。因此,她仿佛是德·莱斯比纳斯小姐,从小集团中的迪·德方\footnote{德·莱斯比纳斯,迪·德方都是十八世纪著名沙龙的女主人。}夫人那里夺来最讨人喜欢的男人,特别是斯万,好另立门户。按某种说法,在她的分裂活动和隐居生活中,斯万一直追随她,然而,尽管她能轻易地使不了解往事的新交相信她的话,她自己却并不信服。然而,当我们喜欢某些角色时,我们一再在众人面前扮演,又一再私下排练,因此想到的往往是它们虚幻的见证,而将真实几乎遗忘殆尽。斯万夫人整天在家时,穿着双绉丝便袍,它如初雪一般洁白纯净,有时穿着百褶薄纱长袍,上面洒满了粉色和白色的花瓣。今天,人们可能认为这身装束与冬天不相称,其实不然。这些轻盈的丝绸和柔和的色彩使她(那时的客厅挂有门帘,十分闷热,描写沙龙生活的小说家当时最高的褒词便是“舒舒服服地垫得厚厚的”)像她身边那些仿佛冬去春来裸露出肉红色的玫瑰花一样显得娇弱畏寒。地毯使脚步声难以觉察,女主人又隐坐在客厅一角,毫不觉察你的到来,因此,当你来到她面前时,她仍在埋头看书,这增加了浪漫性,增加了魅力——仿佛突然发现奥秘,至今我们记忆犹新。斯万夫人穿的便袍当时已不时新,大概只有她还仍然穿着它们,因此仿佛是小说中的人物(只有亨利·格雷维\footnote{亨利·格雷维,法国女小说家(1842—1902),作品情节曲折,以俄罗斯为背景。}的小说中才见过这种便袍)。此刻是初冬,奥黛特客厅里硕大的菊花万紫千红,这是斯万从前未在她的寓所见过的。我赞赏它们——当我闷闷不乐地拜访斯万夫人时,我的失意使这位希尔贝特的母亲具有浓厚的神秘诗意,因为她第二天会对女儿说:“你的朋友来看我了”——可能是由于那些菊花或是和路易十五式丝椅垫一样呈浅粉色,或是和她的双绉睡袍一样雪白,或是和她的茶炊具一样呈铜红色,它们给客厅的布置又加上一层装饰,这层装饰也同样艳丽高雅,但却具有生命,而且只能持续几天。使我尤为感动的是,与十一月黄昏薄雾中的夕阳所放射的绚丽的红色或深褐色相比,菊花的颜色并非转瞬即逝,它持续的时间更长。我看见阳光在空中暗淡下去,我跨进斯万夫人家,发现阳光再现,转移到菊花那火焰般的色彩上。这些菊花仿佛是高超的彩色画家从瞬息万变的大气和阳光中猎取来装点住宅的光彩一样,它们敦促我抛开深沉的忧郁,利用喝茶的这个小时去贪婪地享受十一月份短暂的乐趣(这乐趣闪烁在我身旁那亲切而神秘的菊花光辉之中)。可惜,我所听见的谈话并不能使我达到这光辉,谈话与光辉毫无共同之处。时光不早,但是斯万夫人温柔地对戈达尔夫人说:“啊不,还早呢,别瞧钟,还不到时间,钟也不准。您有什么事要急着走呢?”同时又朝并未放下小皮夹的教授夫人递去一小块馅饼。
\par “要从这里出去可不容易。”邦当夫人对斯万夫人说。这句话表达了戈达尔夫人的感想,她惊奇地大声说:“可不是,我的小脑瓜里也总是这么想的。”她的话得到赛马俱乐部先生们的赞成。当斯万夫人将他们介绍给这位毫不可爱、平庸无奇的矮女人时,他们仿佛受宠若惊,一再致敬,而戈达尔夫人对奥黛特显赫的朋友也十分谨慎,用她的话说,“严阵以待。”(她喜欢用高雅的字句来表述最简单的事物)“您瞧瞧,连着三个礼拜三您都失约。”斯万夫人对戈达尔夫人说。“可不是,奥黛特,有多少个世纪、多长的日子我们没见面了。我这不是认罪了吗?不过,您知道,”她用一种过分腼腆和含糊的神气说(虽然是医生的夫人,她谈起风湿病或肾绞痛来也不直截了当),“我遇到不少小麻烦。各人都有难念的经嘛!我的男仆中出了一场风波,其实我并不比别的女人更看重权威,但是,我不得不辞退膳食总管,以示警戒,他也正想找一个更赚钱的工作。他这一走几乎引起内阁全体辞职,连我的贴身侍女也不愿意留下,那场面可以和荷马媲美。不过,我终于掌稳了舵,这个教训使我获益匪浅。瞧,我用这些仆人的琐事来使您厌烦。您也知道,不得已进行人员调整,这是多么伤脑筋的事。您那位漂亮女儿不在家?”她问道。“不,我那位漂亮女儿在女友家吃饭。”斯万夫人回答,同时转身对我说:“我以为她给您写过信,让您明天来看她哩。”接着又对教授夫人说:“您的婴儿怎么样?”我长长地舒了一口气。斯万夫人的话向我证明,只要我愿意我就可以和希尔贝特见面,而这正是我前来寻找的安慰,正因为如此,我这段时期的访问成为必不可少的。“没有,我今晚给她写几个字。再说,希尔贝特和我再不能见面了。”我说话的语气仿佛将这分离归结为某个神秘原因,这样一来,我可以保持爱情的幻想,我谈到希尔贝特和她谈到我时的温柔口吻使这幻想不至于破灭。“您知道她十分爱您。您明天真的不来?”斯万夫人说。一阵喜悦突然使我飞了起来,我心里想:“为什么不来呢?既然是她母亲亲自请我?”但我立刻堕入忧愁之中。我担心希尔贝特看到我时会认为我最近的冷淡是伪装的,因此我宁愿继续不见面。在个别交谈中,邦当夫人抱怨说她讨厌政治家的夫人们,并且装腔作势地说所有的人都可厌和可笑,她为她丈夫的地位感到遗憾。
\par “这么说,您可以一口气接待五十位医生夫人?”她对戈达尔夫人说,因为后者对谁都和蔼可亲,认真履行义务。“啊,您是有美德的人。我嘛,在部里,当然我必须接待。哎!那些官太太,您知道,真没办法,我没法不对她们伸舌头。我的外甥女阿尔贝蒂娜也和我一样。您不知道这小姑娘有多冒失。上星期我的接待日那天,来了一位财政部次长的夫人,她说她对烹调一窍不通。我那位外甥女露出最美妙的微笑回答说:‘可是,夫人,您肯定知道烹调是怎么回事,因为令尊大人刷过盘子。’”
\par “啊!我真喜欢这故事,妙极了!”斯万夫人说,接着又向戈达尔夫人建议道:“医生出诊的日子,您至少能享受一下可爱的家,和花草书本及您喜欢的东西做伴吧。”
\par “就这样,她直截了当地给了那位女士两下,砰,砰,她可不含糊。事先一点风也不透,这个小坏蛋,像猴子一样机灵。您是幸运者,您能克制自己,我特别羡慕那些善于掩饰思想的人。”
\par “我并不需要这样做,夫人,我这人很随和。”戈达尔夫人轻声说,“首先,我没有您这样的特权地位。”她略略提高声音。每当她在谈话中塞进微妙的殷勤和灵巧的恭维,以博得好感并有益于丈夫的事业时,她总是这样略略抬高声音以增强效果的,“其次,我对教授是鞠躬尽瘁的。”
\par “不过,夫人,问题不在于愿意不愿意,而在于能够不能够。您大概不属于神经质的人。而我,一看见国防部部长夫人装模作样,我就禁不住模仿她。我这脾气真糟糕。”
\par “啊!对了,”戈达尔夫人说,“听说她有抽搐的毛病。我丈夫还认识一位地位很高的人,当然,这些先生私下议论起来……”
\par “对了,夫人,正像那位驼背的礼宾司司长。他每次来,不到五分钟我必定要碰碰他的驼背。我丈夫说我会让他丢了差事,有什么办法呢,让他的部见鬼去吧!对,让他的部见鬼去吧!我该把这句话印在信纸上作为座右铭。我这样说一定使您听着刺耳吧,您是位和气的人,而我,我承认,我喜欢小小的恶作剧,不然生活就太单调了。”
\par 她一个劲地谈论丈夫的部,仿佛它曾是奥林匹斯似的。为了转移话题,斯万夫人转身对戈达尔夫人说:
\par “您看上去真漂亮。是勒德弗商店做的?”
\par “不,您知道,我是罗德尼兹商店的信徒,再说,这是改的。”
\par “是吗,挺有派头!”
\par “您猜多少钱?……不,第一位数不对。”
\par “怎么,这么便宜,简直是白给的。人家告诉我的比这要贵三倍。”
\par “人们就是这样写历史的。”医生的妻子回答说。接着她指着斯万夫人送她的围脖缎带说道:“您瞧,奥黛特,您还认得吗?”
\par 门帘掀开了一半,伸进一个脑袋,他毕恭毕敬、彬彬有礼,戏谑地假装唯恐打扰众人,这是斯万。“奥黛特,阿格里让特亲王正在我的书房,他问能不能来看看你。我该怎样回答他呢?”“我很乐意。”奥黛特显然满意地说,但脸色平静。这很自然,因为她曾接待过高雅人士(即使在她当交际花的时期)。斯万将这个批准令带去给亲王。如果不是在这个空隙里维尔迪兰夫人走了进来,他就要领着亲王回到妻子身边。
\par 斯万和奥黛特结婚时,曾要求她不再和那个小集团来往(他这样做当然有许多理由,而且,即使没有理由,他也会这样做,因为忘恩负义是一条规律,它容不得例外,它更证明了这一点:所有牵线搭桥的中间人不是缺乏远见就是毫无私心)。他只允许奥黛特和维尔迪兰夫人每年互访两次。“女主人”的某些信徒十分气愤,认为这未免太过分,为她鸣不平,因为多年以来,奥黛特,甚至斯万,一直被她视为上宾。小集团中诚然有虚情假意的兄弟,他们不去维尔迪兰夫人家,而是偷偷地赴奥黛特的约会,而且,万一事情泄露,他们便借口说想见见贝戈特(尽管“女主人”说贝戈特不去斯万家,又说他毫无才华可言,但她仍然想方设法——用她的话说——吸引他),但小集团中也有“过激分子”,他们对妥善的个别处理方式(它往往使当事人避免采取极端态度来对待某人)一窍不通,而是盼望维尔迪兰夫人与奥黛特一刀两断(这个愿望当然落空),使奥黛特从此再不能得意洋洋地笑着说:“自从分裂出来,我们很少去‘女主人’家。我丈夫还是单身汉时,去她家比较容易,可是结婚以后就不那么容易了……说老实话,斯万先生受不了维尔迪兰大妈,所以他也不愿意我和她经常来往。而我呢,作为忠实的妻子……”斯万陪同妻子出席维尔迪兰家的晚会,但是当维尔迪兰来看奥黛特时,他往往回避。因此,如果“女主人”在座,他就让阿格里让特亲王一个人进去。奥黛特单独将亲王介绍给维尔迪兰夫人,她不愿意维尔迪兰夫人在这里听见默默无闻的姓氏,而愿意让她看到许多陌生面孔,从而自认为置身于贵族名流之中。奥黛特的这番算计十分奏效,维尔迪兰夫人当晚便带着鄙夷的神气对丈夫说:“她的朋友们真可爱,的确是反动势力的精华!”
\par 奥黛特对维尔迪兰夫人也抱着相反的幻觉。这个沙龙当时并未具有后来的雏形,维尔迪兰夫人甚至还不到孵化期——在此期间停止大聚会,因为新近赢得的、为数可观的名流会被众多无名小卒所淹没,因此宁可等待,等到被吸引来的十位体面人物繁殖七十倍!如同奥黛特即将做的那样,维尔迪兰夫人也将“上流社会”作为目标,但她的进攻范围仍然狭窄,而且与奥黛特的进攻区相距甚远(奥黛特有可能达到同样目标,有可能进行突破),因此,奥黛特对“女主人”所拟定的战略计划一无所知。当人们对奥黛特说维尔迪兰夫人是赶时髦的女人时,奥黛特笑了起来,真心诚意地说:“恰恰相反。首先她不具备赶时髦的条件。她谁也不认识。其次,说句公道话,她觉得现在就很好。不,她喜欢的是星期三的聚会,愉快的谈话。”她暗暗羡慕维尔迪兰夫人作为“女主人”所强调的艺术(奥黛特在这所杰出学校中也学到了这门艺术),那就是(对女主人而言),善于“聚集”,善于“组织”、“发挥”、“隐退”的艺术,充当“桥梁”的艺术,虽然这些艺术仅仅是为空虚涂上色彩,对空虚进行雕琢,确切地说是虚无的艺术。
\par 斯万夫人的女友们看到维尔迪兰夫人来访十分诧异,因为在她们的想象中,维尔迪兰夫人与她高朋满座(永远是小集团)的客厅是无法分开的,而此刻她们惊奇地看到,在这位作为客人的“女主人”身上,在她那张安乐椅上,竟重现、凝聚、浓缩了整个小集团。她裹在一件和这间客厅墙上挂的白色皮毛同样毛茸茸的皮大衣里,仿佛是客厅中的客厅。胆怯的女客唯恐打扰主人,起身告辞,并且用复数人称说:“奥黛特,我们先走了。”就仿佛人们在探视刚能行走的病人时采用复数人称说话,以暗示别让病人过度疲劳。人们羡慕戈达尔夫人,因为“女主人”称呼她的名字。“我带您一起走?”维尔迪兰夫人问戈达尔夫人,她怎能忍受一位信徒不追随她而独自留下呢?“这位夫人已经好意要我坐她的车了。”戈达尔夫人回答,她不愿意让人以为她为了讨好有名气的人而将答应乘邦当夫人的三色标志马车一事忘在脑后:“我真谢谢你们这些朋友。你们要我乘你们的车,对我这个没车夫的人来说,真是运气。”“特别是,”“女主人”回答说(她不敢说得太多,因为她对邦当夫人略有了解,而且刚刚邀请她参加每星期三的聚会),“您住得离克雷西夫人那么远。啊,我的天,我永远也不习惯说斯万夫人。”对小集团这些才智平庸者来说,佯装不习惯称斯万夫人,这也是一种玩笑。维尔迪兰夫人又说:“我一向习惯于称克雷西夫人,刚才差一点又说漏嘴。”其实她在对奥黛特说话时故意说错,而决非差一点说漏嘴了。“奥黛特,您住的地方这样偏僻,不害怕吗?晚上回家我会提心吊胆的。再说,这里又潮湿,对您丈夫的湿疹十分不利。总不至于有耗子吧?”“没有!多可怕呀!”“那就好,这是别人对我说的。我很高兴这是谣传,我这人特别害怕老鼠,都不敢来看您了。再见,亲爱的,回头见,您知道我多么高兴见到您。您不会摆弄菊花。”她一面往外走一面说,斯万夫人起身送她。“这是日本菊花,您得照日本方式插花。”当“女主人”走了以后,戈达尔夫人大声说:“我可不同意维尔迪兰夫人的看法,虽然在一切问题上我都把她当做戒律和先知。奥黛特,只有您能找到这么漂亮的菊花,用时新的说法,漂亮应用阳性形容词。”斯万夫人轻声回答说:“亲爱的维尔迪兰夫人对别人的花有时不够友好。”戈达尔夫人为了打断对“女主人”的批评,便问道:“您去哪家花店?勒梅特尔?那天在勒梅特尔花店前有一株很大的粉色灌木,于是我便做了一件大蠢事。”但她不好意思说出那株灌木的精确价格,只是说“不易上火”的教授也暴跳如雷,说她瞎花钱。“不,不,除了德巴克以外,我没有固定的花店。”戈达尔夫人说:“我也一样,不过我承认我偶尔对它不忠,去拉肖姆花店。”“哈!您抛弃德巴克花店而去拉肖姆花店,我可要去告密了。”奥黛特回答说,尽量显得风趣,好引导谈话。她在自己家中比在小集团中要轻松自如得多,她又笑着补充说:“再说,拉肖姆花店的价格惊人,未免太贵了,我觉得实在不像话。”
\par 邦当夫人曾不止一百次地说过她不愿意去维尔迪兰家,此刻却因受到星期三聚会的邀请而兴奋不已,而且盘算着如何才能尽量多去几次。首先,她不知道维尔迪兰夫人是容不得任何一次缺席的。其次,邦当夫人属于那种人们不乐于与之交往的女人,这种女人被邀请参加“系列”聚会时往往不是干脆地赴约(她们不像那些稍稍有空便愿意出门的人那样使主人高兴),而是相反地强制自己不去参加第一次和第三次晚会,希望自己的缺席会引起注意;她们只出席第二次和第四次晚会,但如果别人告诉她们第三次晚会将十分精彩,那么她们便将秩序颠倒一下,借口说“很可惜,上一次她们没有空”。邦当夫人既然是这种人,便盘算在复活节前还有几个星斯三,她怎样才能多去一次,而无强加于人之嫌。她想在和戈达尔夫人一同回家的路上得到稍许启示。“啊!邦当夫人,您站起来了,这种逃跑的信号可真不好。您上星期四没有来,应该给我补偿……来,再坐下,就一会儿。晚饭以前,您总不会再拜访别人吧!真的,您不想尝尝?”斯万夫人一面递过点心,一面说:“您知道,这些小玩意味道不坏,虽然看上去不怎么样,您尝尝,您一定会喜欢的。”戈达尔夫人说:“不,看上去就好吃。奥黛特,您家里的食品可真丰富。我不用问是在哪里买的,我知道您总是去雷巴特商店。我得承认,我不像您那样专一,我常去布内博内商店买小点心和糖果,那里的冰淇淋可实在不好,而雷巴特商店对冰冻食品,不论是冷冻甜点还是果汁冰糕,都很拿手,我丈夫说,nec plus cultra\footnote{拉丁文:世界的尽头;好得不能更好了。}。”“不过,这些点心是自己家里做的,您真的不要?”邦当夫人说:“不,要不我就吃不下饭了。不过我再坐片刻,您知道,和您这样聪明的女人谈天是件快事。”“您会觉得我多管闲事,奥黛特,不过我很想知道您对特龙贝夫人那顶大帽子的评价。当然大帽子是目前流行的款式,但是,是不是稍稍过分了?刚才她那顶帽子比起前几天她来我家戴的帽子,还是小巫见大巫哩。”“哪里,我可不聪明,”奥黛特带着理当如此的神气说,“其实我这人很轻信,人家说什么我都相信,常常为一点小事伤心发愁。”她影射的是最初因嫁给斯万这样的人而痛苦不安,斯万有自己的生活并和别的女人来往。阿格里让特亲王听见她说“我可不聪明”,立刻认为应该加以否定,但却缺乏敏捷的反应能力。“您胡说什么呀!”邦当夫人高声说。“您还不聪明?”亲王赶紧抓住这根救命稻草说:“这是什么话?大概耳朵在骗我吧?”奥黛特说:“真的,我不骗你们,我确实是小市民,容易大惊小怪,满脑子偏见,坐井观天,十分无知。”接着她打听夏吕斯男爵的近况:“您见到亲爱的男爵了吗?”“您算无知?”邦当夫人惊呼道,“那么,那些官员,那些只会谈论衣着服饰的殿下夫人又算什么呢!……对了,夫人,就在上个礼拜,我和公共教育部部长夫人谈到《洛亨格林》\footnote{《洛亨格林》是瓦格纳的三幕歌剧。}。她说:‘啊,《洛亨格林》,对了,这是牧羊女游乐场上一次的表演,据说逗人笑得直不起腰。’我听了真想给她一记耳光,您瞧瞧,夫人,有什么办法,这种话怎不叫人发火。我是个倔人,您是知道的,”接着她又转脸对我说,“您说呢,先生,我的话有理吧?”“依我说,”戈达尔夫人说,“这情有可原,我们常常被突然的问题弄得措手不及,所以答非所问,这一点我略有体会,因为维尔迪兰夫人经常这样让我们出洋相。”“谈到维尔迪兰夫人,”邦当夫人问戈达尔夫人:“您知道下星期三她家有哪些客人?……我记起来了,对,我们接受了邀请,下星期三去她家。您是不是先到我家吃晚饭?然后我们一同去她家。我独自去有点胆怯,也不知为什么,这位尊贵的女士一直使我害怕。”“我可以告诉您,”戈达尔夫人说,“使您害怕的是她的嗓音,这没办法,哪会人人都有斯万夫人那样好听的声音呢?不过,‘女主人’这话很对,只要你开口说话,冰雪立刻融化,维尔迪兰夫人确实很好客,当然我理解您此刻的心情,第一次去陌生地方总是不太自在的。”“您也来和我们一道吃饭吧,”邦当夫人对斯万夫人说,“饭后我们一同去维尔迪兰家,玩维尔迪兰游戏,到那里以后我们三人待在一边自己交谈,‘女主人’会对我瞪眼睛,从此不再邀请我,不过我不在乎。那会使我大大开心咧。”她这番话似乎不太真实,因为她接着又问:“您知道下星期三她家会有哪些客人?聚会都干些什么?客人总不至于太多吧?”“我肯定不会去,”奥黛特说,“我们只能在最后那个星期三露露面。如果您愿意等到那时……”然而,邦当夫人对这个延期的建议似乎毫无兴趣。
\par 一个沙龙的才智价值往往与风雅成反比,然而,既然斯万认为邦当夫人讨人喜欢,那就是说一个人沉沦而被迫与另一类人为伍时,他对他们不再苛求,对他们的才智及其他不再挑剔。如果这一点是真的,那么,个人和民族一样,在失去独立性的同时也失去自己的文化修养,甚至语言。这种容忍态度的后果之一,便是从某个年龄开始,人们越来越喜欢听别人赞扬和鼓励自己的才智和气质,例如,大艺术家不再和具有独特性的天才交往,而只和学生来往,后者和他的唯一共同语言是他的教条,他们对他唯命是从、顶礼膜拜,又例如,在聚会中某位唯爱情至上的、卓越的男士或女士会认为,那位虽然才智平庸,但话语之间对风流韵事表示理解和赞同的人才是最聪明的人,因为他的话使情人或情妇的情欲本能得到愉快。再以斯万为例。邦当夫人说,有些沙龙只接待公爵夫人们,真是岂有此理!此时,作为奥黛特的丈夫的斯万便点头称是,要是往日在维尔迪兰家中,他会对邦当夫人不以为然,而此刻却说她是个好女人,既富有风趣,又不附庸风雅。他也乐于给她讲一些有趣的事,使她“乐得直不起腰”,她没听说过这些事,但一点就“通”,她喜欢讨人欢心,喜欢取乐。
\par “这么说,医生不像您那么酷爱花?”斯万夫人问戈达尔夫人,“啊!您知道,我丈夫是圣人,中庸之道。不过他倒是有一个嗜好。”邦当夫人眼中闪着狡黠、欢乐和好奇,问道:“什么嗜好,夫人?”戈达尔夫人简单明了地说:“看书。”“这种嗜好可没什么让妻子担心的。”邦当夫人惊呼道,一面克制邪恶的微笑,“您知道,医生完全钻到书里去了!”“那好呀,您不用担心害怕……”“哪里,我担心……他的眼睛。我得回去了,奥黛特,下次再来敲府上的门。说到视力,您听说维尔迪兰夫人要在新买的房子里装电灯吗?这消息不是我的私人密探告诉我的,是从另一条渠道,电工米尔德那里听说的。您瞧我对消息来源毫不隐瞒。连卧室也要装电灯,配上灯罩使光线柔和,多么美妙的奢侈!我们的同代人总是追求新玩意,哪怕是世上独一无二的玩意。我一位朋友的嫂嫂在家里装了电话,不用出门就能向供应商订货。我承认我略施小技让她同意我哪天去对着电话机谈话。电话对我很有诱惑力,不过我宁肯去朋友家打电话,而不愿自己装电话。新鲜劲一过,电话会完完全全成为累赘的。好了,奥黛特,我走了,别再挽留邦当夫人,她要送我回家,我必须走,您这下子可让我闯祸了:我丈夫比我先到家!”
\par 我也一样应该告辞回家了,虽然还没有品尝菊花这些鲜艳斑斓的外壳所蕴藏的冬天的乐趣。乐趣尚未来到,而斯万夫人似乎不再等待什么了。她任仆人收拾茶具,仿佛在宣布:“关门了!”她终于开口说:“真的,您也要走?那好吧,再见。”即使我留下来,也就未必能体会到这陌生的乐趣,而原因不仅仅在于我的忧郁,也就是说这种乐趣并不存在于迅速导致告辞时刻的那条时间的老路上,而是存在于我所不知的一条小路上,我本该拐弯进去才对。不过,我的拜访至少已经达到目的,希尔贝特会知道她不在家时我来看过她父母,还会知道,用戈达尔夫人的话说,我“一上来,从一开始就征服了维尔迪兰夫人”(医生夫人从未见过维尔迪兰夫人如此“殷勤讨好”,还说“你们大概天生有缘分”)。希尔贝特将知道我曾恰如其分地、怀着深情谈起她,她将知道我们不见面我仍然能生活下去,而她最近对我的嫌恶,在我看来,正是因为她认为我没有这个能力。我曾对斯万夫人说我不能再见希尔贝特。我这样说,仿佛我决心永远不再见她。我要给她写的信也表达同样的意思。但是,为了给自己鼓气,我要求自己作最后的、短暂几天的努力。我对自己说:“我这是最后一次拒绝她的约会。我将接受下一次约会。”为了减少这种分离的痛苦,我不把它看做是永久分离,虽然我感到它将是永久的。
\par 这一年的元旦对我十分痛苦。当您不幸时,无论是有意义的日子还是纪念日,一切都会令你痛苦。然而,如果你失去了亲爱者,那么,痛苦仅仅来源于强烈的今昔对比,而我的痛苦则不然,它夹杂着未表明的希望:希尔贝特其实只盼着我主动和解,见我没有采取主动,她便利用元旦给我写信:“到底是怎么回事?我爱上你了,你来吧,我们可以开诚布公地谈谈,见不到你我简直无法生活。”从旧年的岁末起,我就认为这样一封信完全可能,也许并非如此,但是我对它的渴望和需要足以使我认为它完全可能。士兵在被打死以前,小偷在被抓获以前,或者一般来说,人在死前,都相信自己还有一段可以无限延长的时间,它好比是护身符,使个人——有时是民族——避免对危险的恐惧(而并非避免危险),实际上使他们不相信确实存在危险,因此,在某些情况下,他们不需要勇气便能面对危险。这同一类型的毫无根据的信念支持着恋人,使他寄希望于和解,寄希望于来信。其实,只要我不再盼望信,我就不会再等待了。尽管你知道你还爱着的女人对你无动于衷,你却仍然赋予她一系列想法——即使是冷淡的想法——赋予她表达这些想法的意图,赋予她复杂的内心生活(你在她的内心中时时引起反感,但时时引起注意)。对希尔贝特在元旦这一天的感觉,我在后来几年的元旦日都有切身体会,那时,我根本不理睬她对我是专注还是沉默,是热情还是冷淡,我不会想,甚至不可能想到去寻求对我不复存在的问题的答案。我们恋爱时,爱情如此庞大以致我们自己容纳不了,它向被爱者辐射,触及她的表层,被截阻,被迫返回到起点,我们本人感情的这种回弹被我们误认为对方的感情,回弹比发射更令我们着迷,因为我们看不出这爱情来自我们本人。
\par 元旦一小时一小时地过去,希尔贝特的信没有来。那几天我收到几张迟发的或者被繁忙的邮局延误的贺年卡,所以在元月三号和四号,我仍然盼望她的信,不过希望越来越微弱。后来几天里,我哭了许多次。这是因为,我放弃希尔贝特并不如我想象的那样出自真心诚意,我一直盼望在新年收到她的信,眼前这个希望破灭了,而我又来不及准备另一个希望,我像服完了一小瓶吗啡而手头又没有第二瓶吗啡的病人一样痛苦异常。但是也可以有另一种解释,而这两种解释并不相互排斥,因为同一种感情有时包括相反的因素,那就是在我的内心中,对希尔贝特来信所抱的希望曾使她的形象离我更近,当初我急于见她,我如何见到她,她如何待我,凡此种种所引起的激情曾再次涌上心头。立即和解的可能性否定了顺从——其巨大力量往往不被我们察觉。人们对神经衰弱的病人说,只要他们躺在床上不看信不读报,他们便会逐渐安静下来,然而病人却不相信,认为这种生活方式只会更刺激他们的神经,同样,恋人们从相反的心理状态来观察“放弃”,在未真正付诸实行以前,他们也不会相信“放弃”会具有裨益身心的威力。
\par 由于我心跳过速,人们叫我减少咖啡因的剂量,我减量以后,剧烈心跳果然停止,于是我开始怀疑:与希尔贝特近乎绝交时我所感到的焦虑莫非是由咖啡因所引起的?而每当这种焦虑重现时,我总以为是因为我看不见希尔贝特,或者(偶尔与她相遇)看见她冷冷的面孔而感到痛苦。不过,如果说这药才是痛苦的根源,而我的想象力进行了错误解释的话(这也不必大惊小怪,因为情人们最沉重的精神痛苦往往是由和他们同居的女人的生理习惯所引起的),那么它仿佛是使特里斯坦和依索尔德\footnote{特里斯坦和依索尔德是十二世纪法国民间传奇中的两个人物,他俩因误喝药酒永生相爱,并受迫害。}饮后长久相爱的药酒。咖啡因的减量虽然立即使我身体好转,但并未消除我的忧郁。如果说这带毒性的药没有创造忧郁,至少它曾使忧郁更为尖锐。
\par 快到一月中旬,我对新年来信的希望破灭,失望所引起的附加的痛苦稍稍有所缓解,然而,“节日”前的悲伤又卷土重来。它之所以十分残酷,是因为我就是这个悲伤的制造者,有意识的、自愿的、无情的、有耐心的制造者。希尔贝特和我的关系是我唯一珍惜的东西,而我却不遗余力地破坏它,用长期不来往的办法逐渐制造我的冷漠(并非她的冷漠,但实际上是一回事)。我不断地、竭尽全力地使我身上爱恋希尔贝特的那个我进行残酷的慢性自杀,而我清楚地意识到我此刻的行为及将来的后果。我不仅知道再过一段时间我将不再爱希尔贝特,还知道她将为此感到遗憾,她会想方设法和我见面,但都和今天一样不能如愿以偿,并不是因为我太爱她,而是因为我肯定会爱上另一个女人,我将长时间地渴望她,等待她,不肯腾出一秒钟来和希尔贝特见面,因为希尔贝特对我将毫无意义。毫无疑问,就在此刻(我已决心不见她,除非她正式要求解释,或者表白全部爱情,而这是决不会发生的),我已失去希尔贝特,但我却更爱她(我比去年更强烈地感到她对我是多么重要,去年的每天下午,我都能如愿以偿地和她在一起,以为我们的友谊不受任何威胁)。毫无疑问,此刻我憎恶这个念头:有一天我会对另一个女人产生同样的感情。这念头从我这里夺去的不仅仅是希尔贝特,还有我的爱情和痛苦,而我是在爱情和痛苦之中,在眼泪中努力确定希尔贝特的意义的,现在却必须承认这爱情和痛苦并非她所专有,它们迟早会献给另一个女人。因此——这至少是我当时的想法——我们永远超然于具体对象之外,当我们恋爱时,我们感到爱情上并未刻着具体对象的名字,它在将来,在过去,都可能为另一个女人(而不是这个女人)诞生:而当我们不恋爱时,我们以明哲的态度对待爱情中的矛盾,我们随兴所至地高谈阔论,但我们并不体验爱情,因此并不认识它,因为对爱情的认识具有间歇性,感情一出现,认识即消亡。我将不再爱希尔贝特,我的痛苦让我隐约窥见我的想象力所看不到的未来,当然,此刻还来得及向希尔贝特发出警告,告诉她这个未来正逐渐成形,告诉她它的来临是迫近的,甚至无法避免的——如果她希尔贝特不来协助我对那尚在萌芽状态的未来的冷漠进行摧毁的话。多少次我想象给希尔贝特写信,或者跑去对她说:“请注意,我已作出决定。此刻是我最后一次努力。这是我们最后一次见面。很快我就不再爱你了!”可这又何必呢?我有什么权利责备希尔贝特无动于衷呢?我自己不是对除她以外的一切无动于衷,而并不引咎自责吗?最后一次!对我来说,这是天大的事,因为我爱希尔贝特。但是对她来说,这就好像是友人在移居国外以前写信要求来访一样,而我们往往予以拒绝(仿佛拒绝爱我们的讨厌女人),因为我们在盼望快乐。我们每天所支配的时间具有弹性,我们所体验的热情使它膨胀,我们所引起的热情使它收缩,而习惯将它填满。
\par 此外,即使我对希尔贝特讲,她也听不懂。我们说话时,总以为听话者是我们自己的耳朵,自己的脑子。我的话语仿佛穿过暴雨的活动水帘才到达希尔贝特那里,拐弯抹角,面目全非,仅仅是可笑的声音,而再无任何含义。人们借话语所表达的真理并不具有不可抗拒的确凿性,它不能立即使人信服,必须经过一段时间真理才能在话语中完全成形。例如,在论战中,某人不顾种种论据证据,将对立面的理论斥为叛逆,但是后来他却皈依了这个最初被他憎恶的信念,而原先徒劳传播这个信念的人却不再相信它。又例如一部杰作,对于高声朗诵的崇拜者来说,它当然是传世之作,无需证明,而听者却认为它毫无意义或者平庸无奇,但后来听者也承认这是杰作,可惜为时太晚,作者已无法知道。同样,在爱情上,不论你做什么,障碍决不会被绝望者从外部摧毁;只有当你对它们不再感兴趣时,它们才会从另一方面,被不爱你的女人的内心力量所推倒;昔日你试图推倒但总不成功,如今它却突然坍塌,但对你已毫无意义。如果我将自己未来的冷漠及其防止办法告诉希尔贝特,她会以为我这样做表明我对她的爱情和需求超过她的估计,因此她更讨厌和我见面。确实,正是爱情使我比她更清楚地预见到这个爱情的结束,因为我连续处于前后矛盾的精神状态。我本来可以通过写信或见面对希尔贝特发出这个警告,因为这段时间说明我并非须臾离不了她,并且向她证明没有她我也能活下去。不巧的是,某些人,不知出于好意还是恶意,向她说起我,而那口气使她认为是我央求他们这样做的。每当我得知戈达尔、我母亲,甚至诺布瓦先生用笨拙的话语破坏我刚刚作出的牺牲,践踏我的克制态度所获得的结果时(他们使她误认为我不再保持克制),我感到双倍的气恼。首先,我那用心良苦又卓有成效的回避必须从头开始,因为那些讨厌的人在我背后破坏了我的努力,使我前功尽弃。不仅如此,我和希尔贝特见面的愉快也会减色,因为她不再认为我在体面地顺从,而认为我暗中活动,以谋求她不屑于赏赐的会晤。我诅咒人们这种无聊至极的闲言碎语,他们往往在关键时刻深深地伤害我们,而并无使坏或帮忙之意。他们什么也不想,为说话而说话。有时是因为我们未能对他们保持沉默,而他们的嘴又不紧(和我们一样)。当然,在摧毁爱情的这项残酷工程中,他们的作用远远比不上两个人——这两人往往在一切即将圆满解决时使一切付之东流,其中一人出于过度的善意,另一人出于过度的恶意,而我们并不像怨恨不识时务的戈达尔之流一样怨恨这两个人,因为第二位是我们所爱的人,第一位是我们自己。
\par 每次拜访斯万夫人,她总邀请我和女儿一道喝午茶,而且叫我直接给她女儿回信,因此,我常常给希尔贝特写信,在信中我没有选用我认为最有说服力的词句,而仅为我的眼泪寻找最温柔的河床,因为遗憾和欲望一样,并不试图自我分析,只要求自我满足。当一个人恋爱时,他的时间不是用来弄明白他的爱情是怎么回事,而是用来促成明天的约会。当他放弃爱情时,他不试图理解自己的悲伤,而是试图向引起这种悲伤的女人献上他认为最动人的话语。他说的是他认为有必要讲的,而对方不会理解的话,他在为自己说话。我写道:“我原先以为这决不可能,唉!看来这并非十分困难。”我还说:“也许我再不见你了。”我的话避免冷淡(她会认为那是矫揉造作),但当我写下这些话时,我在流泪,因为我感到它们表达的不是我可能相信的事,而是实际上即将发生的事。下一次她托人要求和我见面时,我也会像这次一样鼓足勇气不让步,这样一来,经过一次又一次的拒绝,我会逐渐达到因长久不见面而不想见面的状态。我流泪,但是我有勇气(而且感到愉快)牺牲和她相会的幸福,以求有朝一日吸引她,然而,到了那一天,吸引不吸引她对我来说已无关紧要了。我假定——尽管不太可能——此刻她在爱我,正如我最后那次拜访她时她说的那样,我假定她的厌倦情绪不是出于对我的厌烦,而是出于嫉妒的敏感性,出于和我相似的虚假的冷漠,这种假定仅仅使我的决定不那么残酷。我想象在几年以后,当我们彼此相忘时,我回顾往事,对她说我此刻写的信没有一个字是真的,她会回答:“怎么,你当时爱着我?你知道我多么盼望这封信,多么盼望和你见面,这封信使我哭得多伤心!”我从她母亲家一回来便动手写信,虽然我想到我可能正在制造误会,但这个想法,由于它带来的忧愁,也由于它带来的愉快(我想象希尔贝特爱着我),促使我把信写下去。
\par 当斯万夫人的“茶会”结束,客人们告辞时,我脑子里想的是如何给她女儿写信,而戈达尔夫人想的却完全是另一种事情。她“巡视”一番,毫无例外地向斯万夫人赞扬客厅的新家具,醒目的新“添置品”,在其中发现奥黛特在拉彼鲁兹街的前寓所里某几件东西(虽然为数极少),特别是她的吉祥物——宝石雕成的动物。
\par 斯万夫人从一位受她敬重的朋友那里学到了“过时”一词,它打开了新的眼界,因为它所指的恰恰是几年以前她认为“时髦”的东西,因此这些东西便统统隐退,与曾作为菊花支撑的金色格子架、许多希鲁商店的糖果盒,以及印有花饰的信纸堆在一起(还不算装饰壁炉板的硬纸钱币,早在她认识斯万以前,一位颇有修养的男人就劝她将它们收起来)。此外,在这些暗色墙壁(与斯万夫人稍后的白色客厅完全不同)的房间中,在这种艺术气质的紊乱和画室般的杂乱中,远东风格在十八世纪风格的进逼下节节败退,斯万夫人为了使我更“舒服”而拍打的椅凳上绣的是路易十五式的花束,而不再是中国龙。她经常待在房间里,她说:“我很喜欢这间房,常常使用它。我不能生活在怀有敌意的、陈腐的东西中间。在这里我才能工作。”(她并未说明是画画还是写书;当时那些不愿无所事事,想有点作为的女人开始对写书感兴趣。)她的周围都是萨克森瓷器(她说这个字时带英国音,她喜欢这种瓷器,甚至不论谈到什么都说:这真漂亮,就像萨克森瓷器上的花)。她爱惜它们,甚过往日的瓷雕像和瓷花盆,唯恐无知的仆人碰坏它们。他们那无知的手常使她惶惶不安,使她大发雷霆,而斯万这位如此温顺和彬彬有礼的主人,竟目睹妻子吵吵嚷嚷而毫无反感。清醒地看到缺点,这丝毫无损于爱情,而是相反,使缺点更为可爱。如今,奥黛特在接待熟朋友时不再穿日本睡袍了,而是穿色彩鲜艳的绉丝浴袍,她用手抚摸胸前那花纹图案中的泡沫,她浸泡在其中,悠然自得,随心嬉戏,她的皮肤如此清凉,呼吸如此深沉,仿佛丝袍在她眼中并非像布景一样的装饰品,而是满足她对容貌和卫生的苛求的,如tub(澡盆)和footing(散步)一样的必需品。她常说她宁可没有面包,也不能没有艺术和清洁,她常说,如果《蒙娜丽莎》被烧毁,那会比“大量”朋友被烧死使她更为悲痛。这些理论在她的朋友们看来似乎荒谬绝伦,但却使她显得出众,因而引起比利时大臣每周一次的来访。如果以她为太阳的这个小世界的人们得知她在别处,例如在维尔迪兰家,被认为是蠢女人的话,一定会大惊失色。由于头脑灵活,斯万夫人更喜欢和男人来往,而不大喜欢和女人来往。当她评论女人时,总是从风流女人的角度出发,挑剔她们身上不受男人欣赏的地方,体型粗笨哪,面色难看哪,尽写错字哪,腿上汗毛太重哪,气味难闻哪,眉毛是假的哪,不一而足。相反,对曾宽厚待她的某个女人,她便不那么尖刻,特别是当这女人生活不幸时。她巧妙地为这女人辩护说:“人们对她未免太不公平了。我敢保证她是个好人。”
\par 如果戈达尔夫人以及克雷西夫人旧日的朋友长时间没见到奥黛特,那么他们一定很难认出奥黛特客厅的摆设,甚至很难认出奥黛特本人。她看上去比以前年轻许多!当然,这一方面是因为她发胖了,既然身体更健康,显得那么神色安详、精神饱满、容光焕发。另一方面是由于她的新发型,光滑平整的头发增加了面部的宽度,玫瑰色的粉使脸更有神采,昔日那棱角过于鲜明的眼睑和侧面现在似乎柔和多了。这种变化的另一个原因如下:奥黛特到了中年,终于发现或者说发明她自己的独特面貌,某种永恒的“性格”,某种“美的类型”,于是她在那不协调的面部轮廓上——它曾被飘忽不定、软弱无能的肉体所左右,最轻微的疲劳使它在刹那之间长了好几岁,仿佛是暂时的衰老,因此,长久以来,它根据她的心情和面色而向她提供一个零散的、易变的、无定形的、迷人的脸——贴上这个固定的脸式,仿佛是永不衰退的青春。
\par 斯万的房间里没有别人给他妻子拍的那些漂亮照片,尽管她在照片上的穿戴各不相同,但那神秘和胜利的表情仍能使人们认出她那洋洋得意的身影和面庞。他房间里只有一幅十分简单的老式照片,它摄于奥黛特贴上固定脸式以前,因此她的青春和美貌似乎尚未存在,尚未被她发现。然而,斯万忠实于另一种观念,或者说他恢复原有的观念,他在这位处于走动和静止之间的、脸色疲惫、目光沉思的瘦弱少妇身上所欣赏的是波堤切利式的美。确实,他仍然喜欢在妻子身上看到波堤切利的画中人。奥黛特却相反,她不是极力突出,而是弥补和掩饰她身上那些她所不喜欢的东西,它们在艺术家看来可能正是她的“性格”,而她作为女人,认为这是缺点,甚至不愿意别人提起这位画家。斯万有一条精美的、蓝色和粉红色的东方披巾,当初他买下来是因为《圣母赞歌》\footnote{波堤切利的作品。}中的圣母也戴这样一条披巾,但是斯万夫人从不肯戴它。只有一次她听任丈夫为她订做一套衣服,上面饰满了雏菊、矢车菊、勿忘草、风铃草,和《春》\footnote{波堤切利的壁画。}一模一样。有时,傍晚时分她感到疲乏,斯万便低声叫我看她那双沉思的手,它们那无意识的姿势就像圣母在圣书上写字(那里已经写着《圣母赞歌》)以前往天使端着的墨水瓶里蘸墨水的姿势一样灵巧而稍稍不安。但是斯万接着说:“您千万别告诉她,她要知道了准会改变姿势。”
\par 除了斯万情不自禁地试图在奥黛特身上发现波堤切利的忧郁节奏以外,在其他时刻,奥黛特的身体是一个统一体,它全部被“线条”圈住,线条勾画出这个女人的轮廓,而将旧款式的崎岖线路、矫饰的凸角和凹角、网络以及分散杂乱的小玩意统统删去,而且,凡当身体在理想线条内侧或外侧显出错误和不必要的弯曲时,这条线便大胆纠正大自然的错误,并且在整整一段路程上,弥补肉体和织物的缺陷。那些衬垫、其丑无比的“腰垫”已经消失,带垂尾的上衣也无影无踪,以前,这种上衣盖过裙子,并且由僵硬的鲸须撑着,一直给奥黛特一个假腹部,使她仿佛是一堆七拼八凑的、零散的构件。如今,流苏的垂直线和褶裥饰边的弧线已被身体的曲线所取代,身体使丝绸起伏。仿佛美人鱼在拍水击浪,贝克林纱也具有了人性,身体从过时款式那长长的、混沌和模糊的包膜中挣脱出来,成为有机的、活生生的形式。然而,斯万夫人喜欢并善于在新款式中保留旧款式的某些痕迹。有时,我晚上无心工作,又知道希尔贝特和女友们看戏去了,便临时决定去拜访她父母。斯万夫人通常身着漂亮的便服,裙子是一种好看的深色(深红色或橘红色),它不是流行色,因而似乎另有含义,裙子上斜绣着一条宽宽的、镂空的黑丝带,使人想到旧日的镶褶。在我和她女儿绝交以前,有一天,春寒料峭,斯万夫人邀我去动物园。她走热了便或多或少地敞开外衣,露出衬衣的齿状饰边,仿佛是她几年以前常穿而如今不再穿的背心上轻微的齿形贴边。她的领带——她忠实于“苏格兰花呢”,但是颜色柔和得多(红色变为粉红色,蓝色变为淡紫色),以致人们几乎以为这是最流行的闪色塔夫绸——以特有的方式系在颔下,人们看不出它在哪里打结,并不由自主地回忆起如今不再流行的帽“带”。如果她再“坚持”一段时间,那么,年轻人在试图解释她的服饰时会说:“斯万夫人本人就是整整一个时代,对吧?”优美的文体在于将各种不同形式重叠起来,暗藏在其中的传统使它更臻优美,斯万夫人的服饰也一样。对背心及圆结的朦胧回忆,加上立即被克制的“划船服”\footnote{划船式的短上衣。}趋向,甚至加上对“跟我来,年轻人”\footnote{此处指女帽上的花结,飘带披在身后。}的遥远而模糊的影射,这一切使古老的形式一一重现(不完全的重现)在眼前的具体形式之中,那些古老形式是不可能让裁缝或妇女服装商真正制作出来的,但它却牵动人们的思绪。因此,斯万夫人蒙上一层高贵色彩,而这也许是因为这些装饰既然毫无用处,那么它应该有一种比实利更高的目的,也许是因为它是过去岁月留下的痕迹或者这个女人所特有的衣着上的个性,总之,这种高贵色彩使她千姿百态的装束神态如一。人们感到她的穿着不仅仅是为了身体的舒适或装饰。她的衣着仿佛是整个文明的精致而精神化的体系,将她团团裹住。
\par 一般来说,每逢她母亲的接待日,希尔贝特往往请朋友来喝茶,有时却不然,她不在家,我便趁机赴斯万夫人的“午后茶会”。她总是穿得漂漂亮亮的,塔夫绸、双绉、丝绒、绫罗绸缎,她的衣着不像平日居家的便服那样随便,而是精心配色,仿佛准备外出。在这样一个下午,她那居家的闲散中又增添了某种灵敏与活跃。衣服的式样既大胆又简单,与她的身段动作十分贴合,而衣袖仿佛具有象征性,因日子不同而改换颜色。蓝丝绒表达的是突然的决心,白塔夫绸表达的是愉快的心情,而为了显示伸臂动作中所包含的雍容高贵的审慎,她采取了闪烁着巨大牺牲的微笑的形式——黑色双绉。与此同时,既无实际效益又无明显理由的“装饰”给色彩艳丽的袍衣增添了几分超脱、几分沉思、几分奥秘,而这与她一向的忧郁,至少与她的黑眼圈和手指节所蕴含的忧郁是完全一致的。蓝宝石吉祥物、珐琅质的四瓣小叶三叶草、银质纪念章、金颈饰、绿松石护身符、红宝石细链、黄玉栗子,在这大量的珠宝首饰下面,袍衣本身具有彩色图案,它越过镶贴部分而贯彻始终,还有一排虚设的、无法解开的、小小的缎子纽扣,以及富有微妙暗示的、既精致又含蓄的饰带;衣服上的这一切,和珠宝首饰一样,似乎——此外不可能有任何理由——泄露了某种意图,构成爱情的保证,保守隐情、遵守迷信,似乎是对痊愈、誓愿、爱情或双仁核游戏的纪念。有时,蓝丝绒胸衣上隐隐约约出现亨利二世式样的缝叉,黑缎袍上有轻微隆起处,它或是在靠近肩头的袖子上,使人想起一八三〇年的“灯笼袖”,或是在裙子上,使人想起路易十五的“裙环”。袍衣因而显得微妙,仿佛是化装服,它让对往日的朦胧回忆渗入到眼前生活之中,从而赋予斯万夫人某种历史人物或小说人物的魅力。如果我向她提到这一点,她便说:“我不像许多女友一样玩高尔夫球。我没有任何理由像她们那样穿毛线衫。”
\par 斯万夫人送客回来,或者端起点心请客人品尝而从我身边经过时,趁混乱之际将我拉到一边说:“希尔贝特特别叫我请您后天来吃饭。我原先不知道能不能见到您。您要是不来我正要给您写信呢!”我继续反抗,这种反抗对我来说越来越不费劲,因为,虽然你仍然喜爱对你有害的毒品,但是既然你在一段时间内由于某种必要性而不再服用,你就不能不珍视这种恬静(你以前曾失去),这种既无激动又无痛苦的状态。你对自己说永不再见你所爱的女人,如果这话不完全属实,那么,你说愿意再见她也不全是真话。人们之所以能忍受和所爱的人分离,正是因为他们相信这只是短暂的分离,他们想到的是重聚的那一天,然而,另一方面,他们深深感到,会见可能导致嫉妒,它比每日对团聚(即将实现但却一再延期!)的遐想更痛苦,因此,即将与所爱的女人相见的消息会引起不愉快的激动。人们一天天地拖延,他们并非不希望结束分离所引起的难以容忍的焦虑,但他们害怕那毫无出路的激情东山再起。人们喜欢回忆而不喜欢这种会见,回忆是驯良的,人们可以随心所欲地往回忆中加进幻想,因此那位在现实生活中不爱你的女人却可以在你的幻想中对你倾诉衷肠!人们逐渐将愿望掺进回忆,使回忆变得十分甜蜜。既然它比会见更令人愉快,会见便被一再推迟,因为在会见中你再无法使对方说出你爱听的话,你必须忍受对方新的冷淡和意外的粗暴。当我们不再恋爱时,我们都知道,不如意的爱情要比遗忘或模糊的回忆痛苦得多。尽管我没向自己承认,但我盼望的正是这种遗忘所带来的安详的平静。
\par 此外,这种精神超脱和孤独疗法所引起的痛苦,由于另一种原因而日益减弱。此疗法在治愈爱情这个固执念头以前,先使它削弱。我的爱情仍然炽烈,坚持要在希尔贝特眼中赢回我的全部威望。我认为既然我有意不和希尔贝特见面,那么我的威望似乎应该与日俱增,因此,那些接踵而至的、连续不断的、无限期的日子(如果没有讨厌鬼干预的话),每天都是赢得的、而非输掉的一天。也许赢得毫无意义,既然不久以后我就会被宣布痊愈。顺从,作为一种习惯方式,使某些力量无限增长。在和希尔贝特闹僵的第一个晚上,我承受悲哀的力量十分微弱,如今它却变得无法估量的强大。不过,维持现状的倾向偶尔被突然冲动所打断,而我们毫不在意地听任冲动的支配,因为我们知道在多少天、多少月里我们曾经做到、并仍将做到放弃它。在积蓄的钱袋即将装满时,人们突然将它倒空。当人们已经适应某种疗法时,却不等它生效而突然中断,有一天,斯万夫人像往常一样对我说希尔贝特见到我会多么愉快,这话仿佛将我长久以来已经放弃的幸福又置于我伸手可及的地方,我震惊地意识到,要品尝这种快乐,当时还不算太晚,于是我急切地等待第二天,我要在晚饭前出其不意地去看希尔贝特。
\par 这整整一天,我耐心等待,因为我正在策划一件事。既然往事一笔勾销,既然我们重归于好,我要以情人的身份和她见面。我每天将送给她世上最美的鲜花。如果斯万夫人(尽管她无权当过分严厉的母亲)不允许我送花,那么我每隔一段时间就将送些更为珍贵的礼品。父母给我的钱是不够买礼品的,所以我想到了那个中国古瓷瓶,它是莱奥妮姨母给我的礼物,母亲每天都预言弗朗索瓦丝会来对她说:“它都散架了。”既然如此,卖掉它岂不更好?那样一来,我就有条件使希尔贝特高兴了。它大概可以卖到足足一千法郎吧。我让仆人把它包了起来。由于习惯,我一向不注意这个瓷瓶,它的易手至少产生这样一个效果——让我认识它。我带上它出门,我将斯万的地址告诉车夫,让他从香榭丽舍大街走,因为那条街的拐角上有一家我父亲常去的大的中国古玩店。使我万分惊奇的是,店主立刻出价一万法郎,而不是一千法郎,我兴高采烈地接下这一沓钞票,整整一年我都有钱每天买玫瑰花和丁香花送给希尔贝特了。我走出商店坐上马车,由于斯万家离布洛尼林园很近,车夫没有走往常那条路,而是顺着香榭丽舍大街走。当车驶过贝里街的拐角时,在暮色中,我隐约看见在斯万家附近,希尔贝特正朝相反的方向走去,她步履坚定,但走得很慢,正和身旁一位青年男子交谈,那人的面孔我看不见。我在车上直起身来,想让车夫停车,但又迟疑。这时,两位散步者已走远了,他们那悠闲的步伐所勾画出的两条柔和对称的线很快就消失在香榭丽舍的阴影之中。我随即到达希尔贝特家门前。斯万夫人接待我说:“啊!她会后悔的。不知怎么回事她不在家。刚才她上课时感到很热,对我说她想和女友出去换换空气。”“我在香榭丽舍大街上看见的可能是她。”“不会吧。总之,别对她父亲讲,他不喜欢她在这个钟点出门。good evening(晚安)。”我告辞,叫车夫从原路返回,但没有找到那两位散步人。他们到哪里去了?黄昏中,他们神情诡秘地在谈什么呢?
\par 我回家,绝望地想着那意想不到的一万法郎,它们本该使我有能力时时让希尔贝特高兴,而现在,我却决心不再见她。在中国古玩店的停留曾使我充满喜悦,因为我期望从今以后女友见到我时会感到满意和感激。但是,如果没有这次停留,如果马车没有经过香榭丽舍大街,那么我就不会遇见希尔贝特和那青年男子了。因此,从同一件事上长出了截然对立的枝杈,它此刻产生的不幸使它曾经产生的幸福化为乌有。我这次遭遇和通常发生的事恰恰相反,人们企望欢乐,却缺乏达到欢乐的物质手段。拉布吕耶尔说过:“无万贯家财而恋爱是可悲的。”于是只好一点一点地,努力使对欢乐的期望熄灭。我的情况却相反,物质手段已经具备,然而,就在同时,出于第一个成功的必然后果,至少出于它的偶然后果,欢乐却消失了。这样看来,我们的欢乐就该永远无法实现。当然,一般说来,欢乐的消失并不发生在我们获得实现欢乐的手段的同一天晚上。最常见的情况是我们继续努力、继续抱有希望(在一段时间内),但是幸福永远不会实现。当外界因素被克服时,天性便将斗争从外部转移到内部,逐步使我们变心,使我们期望别的东西,而不再是我们即将占有的东西。如果形势急转直下,我们的心尚来不及改变,那么,天性也绝不放弃对我们的征服,当然它得稍稍推迟,但更为巧妙,同样见效。于是,在最后一刹那,对幸福的占有从我们身边被夺走,或者说,由于天性的邪恶诡计,这种占有本身竟毁灭了幸福。当天性在事件和生活的一切领域中失败时,它便创造最后一种不可能性,即幸福心理的不可能性。幸福现象或是无法实现或是产生最辛酸的心理反应。
\par 我捏着一万法郎,但它们对我毫无用处。我很快就花光了,比每日给希尔贝特送花还要快。每当暮色降临,我心中苦闷,在家里待不住,便去找我不爱的女人,在她们怀中痛哭。连使希尔贝特高兴一下的愿望也消失殆尽。如今去希尔贝特家只会使我增加痛苦。头一天我还认为,重见希尔贝特是世上最美的事,现在我却认为这远远不够,因为当她不在我身边时,她使我担心害怕。一个女人正是这样在不知不觉中,通过她给我们带来的新痛苦而增加她对我们的威力,但同时也增加我们对她的要求。她使我们痛苦,越来越缩小对我们的围困,增加对我们的枷锁,但同时也使我们在原先认为万无一失的枷锁之外增加了对她的束缚。就在头一天,如果我不害怕使希尔贝特厌烦,我会要求少数几次会晤,而现在我不能以此为满足,我会提出其他许多条件,因为,爱情和战争相反,你越是被打败,你提的条件就越苛刻、越严厉,如果你还有能力向对方提条件的话。但是我没有这个能力,所以我首先决定不再去她母亲家。我心中仍想:我早已知道希尔贝特不爱我,我如愿意可以去看她,如不愿意便可逐渐将她忘记。然而,这个想法犹如对某些疾病无效的药物,它对时时出现在我眼前的那两条平行线——希尔贝特和那位年轻男子在香榭丽舍大街上慢步远去——无能为力。这是一种新痛苦,有一天它会耗尽,有一天当这个形象出现在我脑海中时会完全失去它的毒汁,就好比我们摆弄剧毒而毫无危险,就好比我们用少许火药点烟而不用害怕爆炸。此时,我身上正有另一种力量与有害力量——一再重现希尔贝特在暮色中散步的情景——相搏斗。我的想象力朝相反的方向作有效的活动,以粉碎记忆力的反复进攻。在这两股力量中,前一种力量当然继续向我显示香榭丽舍大街上的那两位漫步者,而且还提供取自往日的、另一些令人不快的形象,例如,当希尔贝特的母亲要求她留下陪我时她耸肩的形象。但是第二种力量按照我的希望所编织的蓝图,勾画出未来的图景,它比起如此狭小而可怜的过去来,更令人高兴,更充实。如果说,阴郁不快的希尔贝特在我眼前重现了一分钟的话,那么在多少分钟里我设想的是将来,她会想办法和我言归于好,也许还会促使我们订婚!当然想象力施展于未来的这种力量,毕竟来自过去。随着我对希尔贝特耸肩所感到的恼怒逐渐减弱,我对她的魅力的回忆也会减弱,而正是回忆使我盼望她回到我身边。过去还远远没有死亡。我仍然爱着我自以为憎恶的女人。每当人们夸奖我的发型或气色时,我总希望她也在场。当时不少人表示愿意接待我,我十分不快,一概拒绝,甚至在家中引起争吵,因为我不肯陪父亲出席一个正式宴会,而那里有邦当夫妇及他们的侄女阿尔贝蒂娜——几乎还是个孩子。我们生活中的不同时期就是这样相互重叠的。你为了今天所爱的、而有一天会认为可有可无的东西,而轻蔑地拒绝去会见你今天认为可有可无,而明天将爱上的东西。如果你答应去看它,那么你也许会早些爱上它,它会缩短你目前的痛苦,当然,用另一些痛苦取而代之。我的痛苦在不断变化。我惊奇地发现,在我心中,今天是这种感情,明天又是那种感情,而它们往往和希尔贝特所引起的希望或恐惧有关。这里指的是我身上的希尔贝特。我本该告诫自己,另一个希尔贝特,真正的希尔贝特,也许与这个希尔贝特截然不同,她根本没有我所赋予她的惋惜之情,她大概很少想到我,不仅比我对她的思念要少很多,而且比我臆想中她对我的思念也要少得多(我想象和希尔贝特幽会,探寻她对我的真实感情,幻想她思念我,一直钟情于我)。
\par 在这种时期,悲伤虽然日益减弱,但仍然存在,一种悲伤来自对某人的日日夜夜的思念,另一种来自某些回忆,对某一句恶意的话、对来信中某个动词的回忆。其他形形色色的悲伤,留到下文的爱情中再作描写,在此只声明在上述两种悲伤中,第二种比第一种残酷许多倍,这是因为我们对所爱的人的概念始终活在我们心中,它戴上我们立即归还的光环而无比美丽,它充满频繁产生的甜蜜希望,或者(至少)永久的宁静忧伤(还应该指出,使我们痛苦的某人的形象,与它所引起的日益严重、不断延伸、难以治愈的爱情忧伤极不相称,就好比在某些疾病中,病因与连续发烧及缓慢痊愈极不相称一样)。如果说我们对所爱的人的概念蒙上了往往乐观的精神反光的话,那么,对具体细节的回忆,恶言,充满敌意的信(我从希尔贝特那里只收到一封这样的信)却是另外一回事,可以说我们所爱的人恰恰活在这些零散片断之中,而且具有比在我们对她的整体概念中更为强大的威力。这是因为我们读信时,一目十行,怀着对意外不幸的可怕焦虑,而并非像凝视我们所爱的人那样怀着宁静而忧郁的惋惜。这种悲伤是以另一种方式形成的,它来自外部,沿着最深沉的痛苦这条路一直深入我们的心灵。我们以为女友的形象是古老的、真实的,其实这形象一再被我们更新,而残酷的回忆却早于这个更新的形象,它属于另一个时期,是极端可怕的过去的见证人(少有的见证人)。过去仍然存在,但我们除外,因为我们喜欢抹掉它而代之以美好的黄金时代,代之以重归于好的天堂,而这些回忆,这些信件却将我们拉回到现实,对我们迎头痛击,使我们感到我们日夜等待的那种毫无根据的希望离现实多么遥远。这并不是说这个现实应该永远不变(虽然有时的确不变)。在我们的生活中有过许多女人,我们从不希望与她们相见,而她们当然以沉默来回答我们决非敌意的沉默。既然我们不爱她们,我们便不算计与她们分离了多少年头,这是个反例,但当我们论证分离的效果时却忽略了它,好比相信预感的人忽略预感落空的实例一样。
\par 然而,分离毕竟可以起作用。重新相见的欲望和兴趣最终会在此刻蔑视我们的心中重新燃起。但是需要时间,而我们对时间的要求与心对变化的要求同样苛刻。首先,时间是我们极不愿意给予的东西,因为我们急于结束如此沉重的痛苦。其次,另一颗心需要时间来完成变化,但与此同时,我们的心也会利用时间来进行变化,以致当我们原定的目标即将实现时,它却不再是目标了。目标是可以达到的,幸福是最终可以获得的(当它已不再是幸福时),这个想法本身只包含一部分真理。当我们对幸福变得冷漠时,它降临在我们身上。正是这种冷漠使我们变得不大苛求,使我们认为它如果出现在往日会使我们心满意足(其实当时我们会觉得这幸福并不圆满)。人们对于漠不关心的事不太苛求,也缺乏判断。我们所不再爱恋的人对我们所表示的殷勤,与我们的冷漠相比,似乎绰绰有余,但对我们的爱情而言,却远远不足。甜言蜜语和幽会使我们想到的只是它可能带来的乐趣,我们忘记了当初我们会希望其他一系列的情侣幽会,而正由于这种贪婪的渴望我们会使幽会无法实现。因此,当幸福姗姗来迟、我们再无法享受它,我们不再爱恋时,这个迟到的幸福是否是我们从前苦苦期待的幸福呢?只有一个人知道,当时的我,但它又不复存在,而且,只要它再出现,幸福——无论相同或不相同——便烟消云散。
\par 我等待梦想——我将不再依恋它——的实现,我像当初不太认识希尔贝特时一样任意臆想她的话语和信,她请求我宽恕,她承认除我以外从未爱过任何人,并且要求嫁给我,由于这些想象,一系列不断更新的温柔形象终于在我思想中占据很大地盘,压倒了希尔贝特和青年男子的幻象,因为幻象缺乏补给。要不是做了一个梦,此刻我会再次拜访斯万夫人。我梦见一位朋友,究竟是谁难以确定,他对我背信弃义,并且认为我对他也无情无义,这个梦使我痛苦得猝然惊醒,醒来后痛苦未减,于是我重新想这位朋友,试图回忆这位梦中人是谁,他的西班牙名字已经朦胧不清,我开始释梦,仿佛既是约瑟又是古埃及法老。\footnote{指圣经《圣经·创世记》中法老做了两个梦及圣约瑟释梦这段故事。}我知道在许多梦中,人物的外表是不足信的,因为他们可以伪装,可以交换面孔,正好比无知的考古学者在修复大教堂中被损毁的圣像时,将此像的脑袋放在彼像的身躯上,而且使特性与名称混淆不清,因此,梦中人的特性与姓名可能使我们上当。我们只能根据痛苦的剧烈程度来认出我们所爱的人,而我的痛苦告诉我,梦中使我痛苦的那位忘恩负义的青年男子正是希尔贝特。于是我回忆起最后一次相见的情景。那天她母亲不许她去看舞蹈,她一面古怪地微笑,一面说她不相信我对她真心诚意,她这话也许出自真心,也许是瞎编的。这个回忆使我又联想起另一个回忆。在那以前很久,斯万不相信我是诚恳的人,不相信我能成为希尔贝特的良友。我给他写信也无济于事,希尔贝特将信交还给我,脸上露出同样的难以捉摸的微笑。她并没有立即把信给我。月桂树丛后面的那整个场面,我记忆犹新。一个人痛苦时就具有了道德感。希尔贝特此刻对我的反感似乎是生活对我那天行为的惩罚。惩罚,人们以为在穿过马路时留心车辆,避免危险,就能逃过惩罚。其实还有来自内部的惩罚。事故来自未曾预料的方面,来自内部,来自心灵。我厌恶希尔贝特的话“你要是愿意,咱们就继续搏斗吧”,我想象她和陪她在香榭丽舍大街散步的青年男子单独待在家中的内衣间时,大概也是这样。前一段时间,我以为自己安安稳稳地栖息在幸福之中,如今我放弃了幸福,又以为我至少获得了平静,并能保持下去,这都同样地荒谬。因为,只要我们心中永远藏着另一个人的形象,那么,随时会被摧毁的不仅仅是幸福。当幸福消逝,当我们的痛苦得到平息时,此刻的平静与先前的幸福一样具有欺骗性,并且脆弱不堪。我终于恢复平静,那借助梦境而进入我们身上的,改变我们的精神和欲望的东西也必然逐渐消失,因为任何事物,甚至包括痛苦,也不能持久和永恒。此外,为爱情而痛苦的人,像某些病人一样,是自己的医生。既然他们只能从使他们痛苦的人那里得到安慰,而这痛苦又是那人的挥发物,那么,他们最终只能从痛苦中求得解脱。时刻一到,痛苦本身会向他们揭示良方,因为,随着他们的心灵将痛苦来回摆弄,痛苦便显示出那位被思念者的另一个侧面,这个侧面有时如此可憎,以致人们甚至不愿再见到她,因为在与她欢聚以前先得使她痛苦;这个侧面有时又如此可爱,以致人们将臆想的温柔变作她的优点并以此作为希望的根据。在我身上重新苏醒的痛苦终于平息下来,但我愿意尽量少拜访斯万夫人。这首先是因为,在仍然爱恋但遭遗弃的人身上,作为生活支柱的等待——即使是暗中的等待——自然而然地发生感情变化,尽管表面上一切如初,但第一种情绪已经为第二种相反的情绪所取代。第一种情绪是使我们惶惑不安的痛苦事件的后果或者反映。此时我们恐惧地等待可能发生的事,尤其是当从我们所爱的人那里没有传来任何新信息,我们更渴望有所行动,但我们不知道某个办法的成功率是多少,而在那个办法以后我们再不可能有所作为。然而,正如刚才所说的,等待虽然在继续,但很快便不再被我们所经历的过去的回忆所左右,而是对想象中的未来充满希望。自此刻起,等待几乎成为愉快的事。何况,第一种等待,稍稍持续以后,也使我们习惯于生活在期望之中。我在最后几次幽会中所感到的痛苦仍然存在于我们身上,但已昏昏欲睡。我们并不急于重温痛苦,何况我们并不太清楚此刻我们要求的是什么。我们在自己所爱的女人身上所占的地盘越多(哪怕稍稍多一点),我们就越觉得未被占领的部分对我们多么重要,而且它永远是不可得的,因为新的满足产生了新的需要。
\par 后来,除了上述原因以外,还有一个原因使我完全停止对斯万夫人的访问。这个后来出现的原因不是因为我忘记了希尔贝特,而是我试图尽快忘记她。我的巨大痛苦结束了,但仍然忧伤,这时,对斯万夫人的拜访又如当初那样成为珍贵的镇静剂和消遣。但是既然对希尔贝特的回忆与这些拜访紧密相连,镇静剂的效应无助于我散心。要想散心,我就必须激励自己身上与希尔贝特毫无关联的思想、兴趣和热情以与我的感情(由于和希尔贝特的分离而不再与日俱增)相抗衡。这种与我们所爱的人毫无关联的思绪会占据地盘,它虽然最初很小,但也是从原先占领我们整个心灵的爱情那里夺取过来的。我们必须发展这些思绪,使之壮大,与此同时,感情不断衰退,仅仅成为回忆,这样一来,进入我们精神中的新因素与感情展开争夺,夺得的地盘越来越大,最后整个心灵被夺了过来。我意识到这是消灭爱情的唯一办法,我还年轻,有勇气这样做,有勇气承受最残酷的痛苦,我相信不论付出多大的时间代价,我最终会成功。我在信中对希尔贝特说,我之所以不见她,是由于我们之间的某个神秘的误会,纯粹是莫须有的误会,我这样说是希望希尔贝特要求我解释清楚。然而,即使在极其一般的交往中,当读信人知道对方故意用一句隐晦、虚假、指责的话作为试探时,他高兴地感到自己掌握——而且保留——行动的控制权和主动权,他决不会要求对方解释。在亲密关系中更是如此,爱情口若悬河,而冷漠缺乏好奇心。希尔贝特既然不怀疑有误会,也不打听是什么误会,那么,对我来说,误会便成为真实的,我每封信都提到它。这种虚假的处境和矫饰的冷漠,具有一种魔力,使你不能自拔。我写道“自从我们的心分开以后”,好让希尔贝特回信说:“可它们并未分开呀,咱们谈谈吧。”但我一而再、再而三地重复,最终我自己也相信我们的心确实分开了。我写道:“对我们来说,生活改变了,但它抹杀不了我们曾经有过的感情。”为的是让她说:“可什么也没有改变呀,这感情比任何时候都强烈。”然而,在再三重复下,我也认为生活确实改变了,我们所回忆的感情不复存在,正好比神经过敏者假装生病,久而久之,真正成为病人,如今我每次给希尔贝特写信,都必然提到这个臆想的变化,她在回信中只字不提,无异于默认,于是变化便存在于我们之间。后来希尔贝特不再保持沉默,而采纳我的观点,就好比在正式祝词中,受款待的国家元首和东道国的国家元首几乎说同样的话。每次我在信中写道:“生活纵然将我们分开,但我们对相聚时光的回忆却永存于心。”她肯定在回信中说:“生活纵然将我们分开,却无法使我们忘记那美好时光,它将永远是珍贵的。”(我们很难说明为什么“生活”使我们分开,究竟发生了什么变化)我的痛苦减轻了许多。然而有一天,我在信中说香榭丽舍大街那位我们所熟悉的卖麦芽糖的老妇人死了,我写道:“我想这会使你难过,它唤醒我许多回忆。”刚一写完,我便泪如雨下,因为我发现我谈到爱情时用的是过去时,仿佛它是一位几乎被遗忘的死者,其实,我不自觉地始终认为这爱情仍然活着,至少可以复活。不愿相见的朋友之间的书信最温柔动人。希尔贝特的信像我给陌生人的信一样,温柔文雅,充满表面上的热情,但对我来说,从她那里得到这种表示已极其甜蜜。
\par 此外,逐渐地,拒绝和她见面不再使我难过。既然她不再像往日那般珍贵,我那痛苦的回忆在不停的再现中失去了威力,无法摧毁佛罗伦萨和威尼斯在我眼前日益增长的魅力。此刻我后悔放弃外交职业而选择了一种定居的生活,当初这样做是为了一位姑娘,但我将再也见不到她,并且几乎忘了她。我们为某人而设计我们的生活,但是,当我们终于能够在其中接待她时,她却不来,接着她从我们的视线中消失,而我们成为为她建造的生活中的囚徒。我父母似乎认为威尼斯太远,气候也太热(对我而言),去巴尔贝克可避免旅途劳顿,因此切实可行。不过如此必须离开巴黎,放弃对斯万夫人的拜访。这些拜访虽然并不频繁,但我偶尔可以听斯万夫人谈起女儿。我开始从中感到某种乐趣,而它与希尔贝特毫不相干。
\par 春天临近,天气骤然变冷。在冰冻的大斋期和冷雨夹雪的复活节前一周,斯万夫人怕冷,便常常裹在皮裘里接待客人,双手和双肩抖瑟地缩在硕大的长方形手笼和洁白发亮的皮毛披肩下。手笼和披肩都是白鼬皮的,她从外面回来并不将它们摘下,因此,它们仿佛是比其他白雪更为持久的残留冬雪,无论是热的炉火还是季节的转换都未能使它们融化。然而,在这间我后来不再光顾的客厅里,这几个虽然冰冷但已经绽开鲜花的星期的全部真理已在我眼前显露,而它通过的是另一种令人醉倒的白色,例如“雪球花”——它那高高的、赤裸的茎干像拉斐尔前派画家\footnote{此派绘画藐视约定俗成的规则,其风景画中常有开满白花的灌木。}作品中的直线型小灌木,茎干顶端是既分瓣又合拢的球形花,它像报信天使一样洁白无瑕,并向四周散发柠檬的芳香。当松维尔城堡的这位女主人知道,在四月份,即使天寒地冻,也不可能没有鲜花,她知道春夏秋冬决不像城里人所想象的那样泾渭分明(城里人直到初夏时还仍然以为世上只有将房屋淋得透湿的淫雨)。斯万夫人是否只满足于贡布雷的花匠送来的这些花,而不从“特约”花店买来地中海岸的早春花以弥补这尚嫌不足的春之呼唤呢,我不敢肯定,何况当时我根本不在意。在斯万夫人手笼的晶冰旁,摆着那些雪球花(在女主人思想中,它们可能只是按照贝戈特的建议而组成一部与摆设和服饰相协调的《白色大调交响乐》\footnote{法国诗人戈蒂埃(1811—1872)的一首诗。}),这就足以使我思念乡村,因为它们使我想到《帕西法尔》\footnote{瓦格纳的歌剧,此处指最后部分。}中《耶稣受难节的魔力》的音乐其实就是大自然的奇迹的象征(而如果我们稍稍理智一些,每年都可以亲眼目睹奇迹),因为它们夹杂着另一种花朵的酸酸的、令人心醉的芳香,我不知道那种花的名字,但我在贡布雷散步时频频停下来欣赏,因此,斯万夫人的客厅像当松维尔的小斜坡地那样纯净、那样花满枝头(虽无一片绿叶)、那样充溢着浓郁而纯正的芳香。
\par 然而我不该回忆往事,它很可能使我身上残存的对希尔贝特的爱情持久不灭。因此,尽管这些拜访不再使我感到任何痛苦,我还是一再减少拜访的次数,尽量少见斯万夫人。在我未离开巴黎以前,我最多答应和她散步几次。阳光明媚的日子终于到来,天气转暖。我知道斯万夫人在午饭前必出门一个小时,在林园大道,星形广场及当时称作“穷光蛋俱乐部”(因为他们总是聚在那里观看他们听说过的有钱人)的地区附近散步,因此我请求父母允许我在星期日——因为平时我有事——晚一点吃午饭,先去散步到一点一刻时再吃饭。五月份希尔贝特去乡间友人家了,所以每星期日我都去散步。快到正午时我来到凯旋门,我在林园大道路口等待,眼睛紧盯着斯万夫人即将出现的那条小街,她的家离街口只有几米远。在这个钟点,散步者大都回家了,剩下的人寥寥无几,而且多半衣着入时。突然,在沙土小径上出现了斯万夫人,她姗姗来迟、不慌不忙,充满了生机,仿佛是只在正午开放的最美丽的花朵。她的衣裳向四周撒开,它们永远是不同的颜色,但我记得主要是淡紫色,她全身光耀照人,接着她举起长长的伞柄,撑开一把大阳伞的丝绸伞面,丝绸的颜色和衣服上的落花一样。整整一班人马簇拥着她,其中有斯万,还有五六位早上去探望她或与她相遇的俱乐部的男子。他们这一堆灰色或黑色的人顺从地做着几乎机械性动作,像无生命的框架将奥黛特围在中央。你觉得这个唯一的、目光炯炯有神的女人在注视前方,越过这堆男人而注视前方,她仿佛站在窗前凝神远眺,在自己那裸露的柔和色彩中显得纤弱而无畏,她似乎属于另一个种族、陌生的种族,具有战争威力,因此她一个人就足以应付那众多的随从。她微笑着,对美好的天气,对尚未妨碍她的阳光感到满意,像完成作品以后再无一丝顾虑的创作者一样安详而自信,她确信自己的装束——即使不为某些过路的庸人所欣赏——是高雅中之最高雅的,这是为了她自己,也是为了朋友,当然,她并不过分重视,但也不是无动于衷。她让胸衣和裙子上的小花结在她身前轻轻飘舞,仿佛这是些小生灵,只要它们能跟上她的步伐,她便慷慨地听任它们按自己的节奏尽情嬉戏。她出现时手中的阳伞往往还未撑开,她朝这把淡紫色的阳伞投去幸福和温柔的目光,仿佛这是一束帕尔玛紫罗兰,这目光如此温柔,即使当它不是投向一位朋友,而是投向无生物的物体时,似乎也洋溢着微笑。就这样,她为自己的衣裳保留了,或者说占据了一片高雅的空间,而与她亲热交谈的男人们也不得不尊重这片空间,当然他们像门外汉那样显出某种程度的敬畏,自愧不如,承认这位女友有能力和权利决定自己的衣着,正如承认病人有能力和权利决定吃什么特效药,母亲有能力和权利决定如何教育子女一样。斯万夫人在这么晚的钟点出现,又被那批奉承者簇拥(他们对行人视而不见),人们不免联想到她的住所——她刚刚在那里度过漫长的上午,并即将回去进餐。她从容安详地走着,仿佛在自家花园中散步,这似乎表明她的家近在咫尺,也可以说她身上携带着住所内室的清凉阴影,而正是由于这一切,她的到来使我感觉到户外的空气和热度。再说,我深信,她的衣着,按照她所擅长的礼仪,通过一根必然的、独一无二的纽带,与季节和钟点紧紧相连,因此,她那柔软草帽上的花朵,在裙衣上的小花结,像花园和田野的鲜花一样,自然而然地诞生在五月。为了感受季节带来的新的变化,我的眼光只需抬到她那把阳伞的高度,它张得大大的,仿佛是另一个更近的天空,圆圆的、仁慈的、活动的、蓝色的天空。如果说这些礼仪是至高无上的话,它们却在清晨、春天、阳光前屈尊俯就,并以此为荣(斯万夫人也以此为荣),而清晨、春天、阳光却并不因为受到如此高雅的女士的青睐而感激涕零。她为它们穿上一件鲜艳轻薄的裙衣,宽松的衣领和衣袖使我想到微微发湿的颈部和手腕,总之,她为它们打扮自己,就好比一位高贵夫人愉快地答应去拜访乡村人家,虽然谁都认识她,连最卑俗的人也认识她,她却执意在这一天作村姑打扮。我等斯万夫人一到便向她问好,她让我站住,微笑着说:“good morning”(早上好)。我们一同走了几步。于是我明白她遵守衣着法规是为了自己,仿佛遵守的是最高智慧(而她是掌握这种智慧的大祭司),因为,当她觉得太热时,便将扣着的外衣敞开,或者干脆脱下来交给我,于是我在她的衬衣上发现了上千条缝钮制作的细节,它们幸运地未曾被人觉察,就好比作曲家精心构思而永远不能达到公众耳中的乐队乐谱一样。她那件搭在我臂上的外衣也露出衣袖中的某些精美饰件,我出于乐趣或者出于殷勤而久久地注视它,它和衣服正面一样做工精细,但往往不被人看见,它或者是一条色彩艳丽的带子,或者是一片淡紫色衬缎,它们就像是大教堂中离地八十英尺高处的栏杆内侧所暗藏的哥特式雕塑一样,它们可以和大门廊上的浮雕比美,但是从来没有人见到它们,直到一位艺术家偶然出游到此,登上教堂顶端以俯瞰全村,才在半空中,在两个塔楼之间发现了它们。
\par 斯万夫人在林园大道上散步仿佛在自家花园的小径上散步,人们——他们不知她有“footing”的习惯——之所以有这种印象是因为她是走着来的,后面没有跟着马车。因为从五月份起,人们经常看见她像女神一样娇弱无力而雍容高贵地端坐在有八条弹簧的宽大的敞篷马车里在温暖空气中驶过。她的马是巴黎最健美的,仆役的制服也是巴黎最讲究的。而此刻,斯万夫人却以步代车,而且由于天热步履缓慢,因此看上去似乎出于好奇心,想优雅地藐视礼仪规矩,就好比出席盛大晚会的君主自作主张地突然从包厢来到普通观众的休息室,随从们既赞叹又骇然,但不敢提出任何异议。斯万夫人和群众的关系也是这样。群众感到在他们之间隔着这种由某种财富筑成的壁垒,而它似乎是无法逾越的。当然,圣日耳曼区也有它的壁垒,但是对“穷光蛋”的眼睛和想象力却不大富有刺激性。那里的贵妇人朴实无华,与普通市民相似,平易近人,不像斯万夫人那样使“穷光蛋”自惭形秽,甚至自感一钱不值。当然斯万夫人这样的女人不会对自己那充满珠光宝气的生活感到惊奇,她们甚至不再觉察,因为已经习以为常,也就是说她们认为这一切理所当然、合情合理,并且以这种奢侈习惯作为判断他人的标准,因此,如果说这种女人(她们在本人身上显示的和她们能在他人身上发现的崇高,具有纯粹的物质性,容易被人看见,但难以获取,并且万一消失难以补偿)将路人置于最低贱的地位,那么反过来,她在路人眼前一出现便立刻不容辩驳地显得至高无上。这个特殊的社会阶层当时包括与贵族女人交往的伊斯拉埃尔夫人以及将要与贵族女人交往的斯万夫人,这个中间阶层低于它所奉承的圣日耳曼区,却高于除圣日耳曼区以外的其他一切。这个阶层的特点在于它已脱离富人社会,但却是财富的象征,而这种财富变得柔软,服从于一种艺术目的、艺术思想,好比是具有可塑性的、刻着诗意图案的、会微笑的金币。这个阶层如今可能不复存在,至少失去了原有的性格和魅力。何况当时组成这个阶层的女士们已人老珠黄,失去了旧日统治的先决条件。言归正传,此刻斯万夫人正走在林园大道上,雍容庄重、满脸微笑、和蔼可亲,仿佛从她那高贵财富的顶端,她那芳香扑鼻的成熟夏季的光荣之巅走下来,像伊帕蒂阿一样看到天体在她缓慢的步履下旋转。\footnote{伊帕蒂阿,公元四世纪希腊女哲学家及数学家,以美貌博学著称。此处指法国一诗人关于她的诗句:“……天体仍在她那白色的脚下旋转……”}过路的年轻人也不安地瞧着她,不知能否凭泛泛之交而向她问好(何况他们和斯万仅一面之交,所以怕他认不出他们来)。他们抱着不知后果如何的忐忑心情决定一试,谁知这具有挑衅性和亵渎性的冒失举动是否会损伤那个阶层不可触犯的至高权威,从而招来滔天大祸或者神灵的惩罚呢!然而,这个举动好比给座钟上了发条,引起奥黛特四周那些小人一连贯的答礼,首先是斯万,他举起镶着绿皮的大礼帽,笑容可掬,这笑容是他从圣日耳曼区学来的,但已失去往日所可能有的冷漠,取而代之的(也许因为他在某种程度上充满了奥黛特的偏见)既是厌烦——他得向衣冠不整的人答礼,又是满意——妻子的交游如此广泛。这种复杂的感情使他对身旁衣冠楚楚的朋友说:“又是一位!我发誓,真不知道奥黛特从哪里弄来这么多人!”她朝那位惶恐不安的行人点点头,现在他已经走远了,但心脏仍然突突直跳。接着她转脸对我说:“这么说,结束了?您永远不再来看希尔贝特了?您对我另眼看待,我很高兴,您不完全‘drop’(丢弃)我。我很喜欢看见您。从前我也喜欢您对我女儿产生的影响。我想她也会很遗憾的。总之,我不愿强人所难,否则您就不愿意再和我见面了。”“奥黛特,萨冈在向你打招呼。”斯万提醒妻子说。果然,亲王(仿佛在戏剧或马戏的高潮场面中,或者在古画中)正拨转马头,对着奥黛特摘下帽子深深致意,这个举动富有戏剧性,也可以说富有象征性,它表达了这位大贵人在女人面前毕恭毕敬的骑士风度,哪怕这位女性代表的是他的母亲和姊妹所不屑于交往的女人。斯万夫人浸沉在阳伞所投下的如流体一般透明又蒙上一层清亮光泽的阴影中,迟迟归来的最后一批骑手认出了她,并向她致意。他们在大道的耀眼阳光下飞驰而过,就像在摄影机前一样。这是赛马俱乐部的成员,是公众熟知的人物——安托万·德·卡斯特兰、阿达贝尔·德·蒙莫朗西以及其他许多人——也是斯万夫人熟悉的朋友。既然对诗意感觉的回忆比对心灵痛苦的回忆寿命更长(相对地长寿),我当初为希尔贝特所感到的忧伤如今早已消逝。但每当我仿佛在日晷上看到五月份从中午十二点一刻到一点钟这段时间,我仍然心情愉快,斯万夫人站定在宛如紫藤绿廊的阳伞下,站在斑驳光影中与我谈话的情景又浮现在眼前。


\subsubsection*{第二部\ 地名:地方}


\par 两年以后我与外祖母一起动身去巴尔贝克时,我对希尔贝特已经几乎完全无所谓了。我领受一张新面庞的风韵时,我希望在另一位少女帮助下去领略意大利哥特式大教堂、宫殿和花园的美妙时,常常忧郁地这样想:我们心中的爱,对某一少女的爱,可能并不是什么确有其事的事情。那原因是:虽然愉快的或痛苦的梦绕魂牵混成一体,能够在一定时期内将这种爱与一个女子联系在一起,甚至使我们以为,这种爱定然是由这位女子撩拨起来的;待我们自觉或不知不觉地摆脱了这种梦绕魂牵的情绪时,相反,这种爱似乎就是自发的,从我们自己的内心发出来,又生出来献给另一个女子。不过,这次动身去巴尔贝克以及我在那里小住的最初时日,我的“无所谓”还只是时断时续的。(我们的生活很少按年月顺序,在后续的日子里,有那么多不以年月为顺序的事情插进来。)我常常生活在更遥远的时光里,也就是比我热爱希尔贝特的前夕或前夕的前夕更久远的时光里。这时,再也不能与她相见,便顿时使我痛苦起来,就像事情发生当时一样。虽然曾经爱过她的那个我,已经几乎完全被另一个我所取代,但是从前那个我,会突然又冒出来,而这种时刻的来到,常常是由于一件小小不然的事,而不是什么重大的事情。例如——我现在把在诺曼第的小住提前来说,我指的就是在巴尔贝克的小住——我在海堤上遇到一个陌生人,我听到他说“邮政部司长一家”时,(如果我当时还不知道这家人家对我们的生活会有什么影响的话)我大概会觉得这句话毫无用处;可是对于与希尔贝特长期分离已经肌销神损、忍受巨大痛苦的我,这句话会引起我巨大的痛苦。其实希尔贝特当我的面与她父亲就“邮政部司长”之家谈过一次话,可是我从来就没有再想到这个。对爱情的回忆并不超出记忆的普遍规律,而记忆规律又受到习以为常这个更为普遍的规律之制约。习以为常能使一切都变得淡漠,所以,最能唤起我们对一个人的记忆的,正是我们早已遗忘的事情(因为那是无足轻重的事,我们反而使它保留了自己的全部力量)。所以我们记忆最美好的部分乃在我们身外,存在于带雨点的一丝微风吹拂之中,存在于一间卧房发霉的味道之中,或存在于第一个火苗的气味之中,在凡是我们的头脑没有加以思考,不屑于加以记忆,可是我们自己追寻到了的地方。这是最后库存的往日,也是最美妙的部分,到了我们的泪水似乎已完全枯竭的时候,它仍能叫我们流下热泪。是在我们身外吗?更确切地说,是在我们心中,但是避开了我们自己的目光,存在于或长或短的遗忘之中。唯有借助于这种遗忘,我们才能不时寻找到我们的故我,置身于某些事情面前,就像那个人过去面对这些事情一样,再度感到痛苦,因为这时我们再也不是我们自己,而是那个人,那个人还爱着我们今天已经无所谓的一切。在惯常记忆的强光照射下,往日的形象渐渐黯然失色,模糊起来,什么也没有剩下,我们再也不会寻找到它了。或者更确切地说,如果几个词(如“邮政部司长”之类)没有被小心翼翼地锁在遗忘中,我们就再也不会寻找到它,正如将某一书籍存在国立图书馆一册,不这样,这本书就可能再也找不到了。
\par 但是这种痛苦和这种对希尔贝特的再生之爱,并不比人们梦中的痛苦和再生之爱更持久。这一次,倒是因为在巴尔贝克,旧的习惯势力再也不在这里,不能使这些情感持续下去了。习惯势力的这种效果之所以看上去似乎相互矛盾,这是因为这个习惯势力遵循着好几条规律。在巴黎,借助于习以为常,我对希尔贝特越来越无所谓。我动身去巴尔贝克,改变习惯,即习惯暂时停止,便圆满完成了习以为常的大业。这习以为常使事物变得淡漠,却又将事物固定下来,使事物解体却又使这种解体无限地持续下去。数年来,每一天我都好好歹歹将我的精神状态套在前天精神状态的套子上。到了巴尔贝克,换了一张床。每天早上有人将早点送至床边,这早点也与巴黎的早点不同,这大概就再也支持不住我对希尔贝特的爱所赖以生存的想法了:有时候(这种时候很罕见,确是如此),久居一地会使时日停滞,赢得时间的最好办法便是换换地方。我的巴尔贝克之行正如大病初愈的人第一次出门一样,单等这一时刻来到,便可发现自己已经痊愈了。
\par 从巴黎到巴尔贝克这段路程,如今人们一定会坐汽车走,以为这样会更舒服一些。这么走,在某种意义上,甚至这段旅程会更真实,因为会更亲切地、感受更深切地体会到大地面貌改变的各种渐变。但是归根结底,旅行特有的快乐并不在于能够顺路而下,疲劳时便停下,不是使动身与到达地点之间的差异尽量使人感觉不到,而是使人尽可能深刻感受到;在于完全地、完整地感受这种差异,正如我们的想象一个跳跃便把我们从自己生活的地方带到了一个向往地点的中心时,我们心中所设想的二者之间的差异那样。这一跳跃,在我们看来十分神奇,主要还不是因为穿越了一段空间距离,而是它把大地上两个完全不同的个性联结在一起,把我们从一个名字带到另一个名字那里,在火车站这些特别的地方完成的神秘的过程(比散步好,散步是什么地方想停下来就可以停下来,也就不存在目的地的问题了)将这一跳跃图像化了。火车站几乎不属于城市的组成部分,但是包含着城市人格的真谛,就像在指示牌上,车站上写着城市名一样。
\par 但是,在各种事情上,我们这个时代有一个怪癖,就是愿意在真实的环境中来展示物件,这样也就取消了根本的东西,即将这些物件与真实环境分离开来的精神活动。人们“展示”一幅画,将它置于与其同时代的家具、小摆设和帷幔之中,这是多么乏味的布景!如今,一个家庭妇女头一天还完全无知,一旦到档案馆和图书馆去待上几天,便最善于在当今的公馆里搞这种玩艺!但是人们一面进晚餐一面在这种布景中望着一幅杰作,那幅杰作绝不会给予人心醉神迷的快感。这种快感,只应要求它在博物馆的一间大厅里给予你。这间大厅光秃秃的,没有任何特点,却更能象征艺术家专心思索以进行创作时的内心空间。
\par 人们从车站出发,到遥远的目的地去。可惜车站这美妙的地点也是悲剧性的地点。因为,如果奇迹出现,借助于这种奇迹,还只在我们思想中存在的国度即将成为我们生活其中的国度,就由于这个原因,也必须在走出候车室时,放弃马上就会又回到刚才还待在里面的那个熟悉的房间的念头。一旦下定决心要进入臭气冲天的兽穴——经过那里才能抵达神秘的境界,进入一个四面玻璃窗的偌大的工场,就像我到圣拉扎尔的四面玻璃窗大工场里去找寻开往巴尔贝克的火车一样,就必须放弃回自己家过夜的一切希望。这圣拉扎尔车站,在开了膛破了肚的城市高处,展开广阔无垠而极不和谐的天空,戏剧性的威胁成团成堆地聚集,使天空显得沉重,与曼坦那\footnote{曼坦那(1431—1506),意大利画家,他画过一幅《钉上十字架》,普氏时代已在卢浮宫展出过。}或委罗内塞\footnote{委罗内塞(1528—1588),意大利画家,他画过数幅《钉上十字架》。}笔下那几乎形成巴黎时髦的某些天空十分相像。在这样的天空下,只会完成某一可怕而又庄严的行动,诸如坐火车动身或者竖起十字架。
\par 在巴黎,我躺在自己床上,从鹅毛大雪漫天飞舞中遥望巴尔贝克那波斯式教堂,不出此限时,我的躯体对这次旅行并没有提出任何异议。只有当我的躯体明白了它必须亲自出马,抵达的当晚,人家要把我送到它很陌生的“我的”房间去的时候,异议才开始出现。动身的前一天,我明白了母亲并不陪同我们前往时,它的反抗就更加激烈。我父亲与德·诺布瓦先生动身去西班牙之前一直要留在部里,他宁愿在巴黎郊区租一所房子度假。此外,欣赏巴尔贝克的美景,并不因为必须付出痛苦的代价去换取就使人的欲望大减。相反,这痛苦在我看来,似乎能使我即将去寻求的印象现实化,保证它的真实性。任何所谓相同美丽的景色,任何我得以去观看,而又并不因此就妨碍我回到自己的床上去睡觉的“全景”都无法代替这种印象。我感到喜欢做什么事的人和为此而感到快乐的人并不是同一些人,这已不是第一次了。给我看病的大夫见我动身当天早晨神色痛苦,大为惊异,他对我说:“我向你保证,哪怕我只能找到一周的时间到海滨去乘乘凉,我决不摆架子等人来请我。你马上可以看到赛船竞渡,太好了!”我认为自己和这位大夫一样深深向往着巴尔贝克。对我来说,甚至早在去听贝玛演唱以前,我就已经知道,不论我喜欢什么,这件东西永远只能放在痛苦追求的尽头,而在这痛苦追求的过程中,我首先必须为这个最高的利益牺牲我的快乐,而不是去寻求快乐。
\par 和从前一样,我的外祖母仍然热切希望赋予人们给予我的馈赠以艺术性,自然她对我们动身的想法就不同。为了通过这次旅行对我进行一项部分古典式的“考验”,她本来打算一半乘火车,一半乘马车,来完成当年德·塞维尼夫人从巴黎经过肖内和欧德迈尔桥到东方\footnote{这是一个地名。该城建于1666年。在此两年以前成立了“东印度公司”,这个公司的造船厂造出的第一艘船定名为“东方的太阳”,取其中“东方”定为该城市名。后来该公司消失了,地名照旧。}去所走过的这段旅程\footnote{见塞维尼1689年4月27日、5月2日及8月12日各函,这三个地名分别在这三封信中出现。}。但在父亲的明令禁止之下,外祖母不得不放弃这个计划。我父亲知道,外祖母安排一次外出,以便将出门旅行所能包含的智力方面的好处全部发挥出来时,事先便可预知会有多少次误车,丢失行李,咽喉疼痛以及违章。她想到我们要到海滩去时,不至于因突然来了“该死的一车人”而受阻去不成,会十分高兴。这“该死的一车人”,是外祖母喜爱的塞维尼夫人的叫法见1671年6月28日塞维尼夫人致格里尼昂夫人函。塞维尼夫人在这封信中写道:“令人愉快的来客走了,我多么伤心难过,你是知道的。叫我又受拘束又厌烦的该死的一车人走了,我又多么心花怒放,你也知道。正因为如此,我们认定:比起令人愉快的客人来,更希望来令人讨厌的客人。”。因为勒格朗丹没有为我们给他姐姐写封引见信,我们在巴尔贝克一个人也不认识(这一忽略,我的姨祖母塞莉纳和维多利亚\footnote{在第一卷中,这两位姨祖母叫塞莉纳和弗洛拉。}均很不欣赏。为了突出往日的密切关系,她们至今仍称那个当姑娘时她们就认识的人为“勒内·德·康布尔梅”,而且还保留着那个人送的礼物。这礼品装饰一个房间,也装点谈话,只是当前的现实与这些礼品已经对不上号。我的这两位姨祖母,在勒格朗丹老太太家里,再也不提她女儿的名字,只是一走出他们的家门,便用诸如此类的话来互相道贺:“那个人,你知道的,我提都没提她。我想,他们心里自然明白。”她们以为这样便为我们报了仇,雪了恨)。
\par 所以,我们就要乘一点二十二分的那趟火车从巴黎动身。我花了好长时间在铁路局时刻表上找这趟车以自得其乐,每次这时刻表都使我激动不已,甚至使我产生已经动身那种兴冲冲的幻觉。花的时间那么长,不会不想到我对这趟车已经了如指掌了。我们对列车的想象中,幸福不幸福的决定因素更主要地是关系到它会给我们什么性质的快乐,而不是我们对这趟列车的情况是否了解确切,所以我觉得自己对这趟车已经了解得很细,我一点都不怀疑,当天气变得凉爽起来,我凝望着即将抵达某一车站会出现某种效果时,我将会在车厢里领略到一种特殊的快乐。这列火车,虽然在我心中总是唤起同一些城市的景象,我用列车穿过的下午时光的光线将这些城市镶嵌起来,可是我似乎觉得这列火车与任何其他列车都不相同。正像人们常常对一个从来没有见过、又喜欢想象已经得到他的友情的人常常所做的那样,我最后也赋予一个金发艺术家旅客以特有的不变的面容。他可能带我踏上他的旅途,我可能在圣洛大教堂\footnote{圣洛大教堂,又称圣洛圣母院,始建于十三世纪末、十四世纪初。拉斯金认为该教堂三角楣的尖顶为火焰式建筑之典范。}脚下向他告别,然后他朝着夕阳的方向远去。
\par 我的外祖母好容易下定决心去巴尔贝克,总不能“白去”一趟,所以她将要在一位女友家停留二十四小时。我当天晚上从那人家里再度踏上旅程,以免叨扰,同时也为了第二天白天能去参观巴尔贝克教堂。我们早已获悉,这所教堂距巴尔贝克海滩相当远,从那里再赶到海滩开始我的海水浴治疗,可能就来不及了。我这次旅行中的精彩节目,列在残酷的第一夜之前,这种感觉可能还会叫我好受一些。在那残酷的第一夜里,我要走进一个新住所,而且要同意在那里生活。
\par 但是,首先得离开原来的住所。我母亲正好安排在同一天到圣克卢安顿,她早已采取了一切措施,或者佯装已经采取了全部措施,把我们送到车站以后,就直接去圣克卢,而不需要再回我们自己的家。她怕我不但不去巴尔贝克,反而要跟她回家。她甚至以在那所刚刚租下的房子里有许多事要做,她又以时间很紧为借口,决心不与我们待到火车开动,实际上是为了给我免去这残酷的告别。火车开动之前,她躲在来来去去、准备这准备那之中。再也无法避免分手时,因为精力完全集中在那无能为力而又无比高尚的清醒时刻上,分手也就突然显得无法忍受了。
\par 我生平第一次感觉到,我母亲没有我,不为了我,而过另一种生活也能活。她就要和我父亲一起去住。说不定她觉得我身体不好,神经过敏,把我父亲的生活搞得更复杂,更惨淡了。这次分别使我更加难过,因为我心中暗想:说不定对我母亲来说,这是我引她不断伤心的结果。她没有对我说过我怎样不断使她伤心,但是经过那些事之后,她明白再也无法共同度假了。说不定也是过另外一种生活的初次尝试。随着父亲和她年岁的逐渐增长,为了将来,她要开始心甘情愿地接受这另一种生活。这就是与从前相比我与她见面要少;她对我已经有些形同路人;她成了一个人们看见她独自一人回到一幢房屋的妇人,而我并不在那房屋中;她向看门人询问是否有我的来信。这种情形,甚至在我做过的噩梦中也从未出现过。
\par 车站雇员想把我的箱子拿走,我几乎无法答话。我母亲为了安慰我,使出她认为最有效的手段。她觉得对我的悲伤佯作不见没有用,便轻轻地拿这个开玩笑:
\par “喂,巴尔贝克教堂如果知道人家是这么愁眉苦脸地准备去看它,会说什么呢?拉斯金说的兴高采烈的旅行家\footnote{拉斯金在《亚眠圣经》中,经常提到“旅行家”以及他在路上遇到了艺术品得到无限快乐的情形。普鲁斯特将拉斯金的《亚眠圣经》译成法文,对拉氏著作当然是了如指掌的。但拉斯金并不喜欢乘火车旅行。}是这样的吗?再说,你是否能够适应环境,我会知道的。即使离得很远,我仍将和我的小狼在一起。你明天就能收到妈妈的一封信。”
\par “女儿,”外祖母说道,“我看你和塞维尼夫人一样,一张地图放在眼前,一刻也没有分开。”\footnote{见1671年2月9日塞维尼夫人致女儿函:“一张地图摆在我面前,你过夜的地方,我全知道。”}
\par 然后母亲又设法叫我开心,她问我晚餐时我要点什么菜,她对弗朗索瓦丝佩服得五体投地,称赞她把一顶帽子和一件大衣改得认不出原样来,她从前看见这顶帽子新的时候戴在我姨祖母头上,这件大衣新的时候穿在我姨祖母身上,是曾经引起她厌恶的。那帽子顶上有一只大鸟,大衣上到处是难看的图案和乌黑发亮的点点。可是大衣不能穿了,弗朗索瓦丝叫人把大衣翻个个,将色调很好看的一色里子露在外面。至于那只大鸟,因为坏了,早就把它扔了。在一首民歌里讲到,最有艺术意识的艺术家费尽心血把最精致的装饰装点到农民住宅的门面上,使得这住宅门顶上正合适的地方开出一朵雪白或淡黄的玫瑰来。有时你遇到这么精致的东西,真叫你动心。与此种情形相同,天鹅绒结呀,鸡蛋壳形的丝带呀,这些在夏尔丹或惠斯勒\footnote{夏尔丹和惠斯勒的名字,在这部小说中,这里第一次出现。从普氏的美学观点形成来说,这两位画家极为重要。夏尔丹(1699—1779),是著名法国画家。普氏在1895年左右曾就夏尔丹写过一篇研究文章。后来又将他对于伦勃朗的研究补充进去,一起发表在《驳圣佩甫》一书中。惠斯勒(1834—1903),美国画家,在巴黎和伦敦住过多年。普氏经人介绍,与惠斯勒相识,并见到1891年画家为孟德斯基乌伯爵画的肖像。但是拉斯金很看不起惠斯勒。普氏摆脱了拉斯金的影响,在1905年所写的文章及书信中,对惠斯勒极为推崇。普氏此处所提情形,在惠氏的许多肖像画中均可见到。}的肖像画上会令人兴高采烈的东西,弗朗索瓦丝用无懈可击而又纯朴的审美观将这些东西缀在那顶帽子上,那帽子便变得十分动人了。
\par 这事还得往从前说,谦逊和正直常常赋予我们这位年老的女仆以高贵的面部表情。她是内向而没有卑劣情感的女子,她很懂得“不越礼,保体面”,为这次出门,她穿上了人家不穿而送给她的衣裳,以便跟我们坐在一起既相配,又不致显出非要人家瞧她的样子。弗朗索瓦丝穿着樱桃红而又陈旧的大衣,毛皮围领并不硬扎扎地露出毛来,她那样子使人想到一位年长的大师在《时时刻刻》一书中所绘之安娜·德·布列塔尼\footnote{《安娜·德·布列塔尼的时时刻刻》于1508年出版,为法国画家让·布尔迪松(约1457—1521)的作品。}的某一形象。在那些形象中,一切都安排得那么妥帖,整个画面的情感在各个部分也分布得特别匀称,以致那华丽而又过时的特殊服装跟眼睛、嘴唇和双手一样,都表现出虔诚的严峻来。
\par 说到弗朗索瓦丝,就不能提到思想。她一无所知,这意思是指,一无所知就等于什么也不懂,但内心能直接领会的几条罕见的真理除外。庞大的思维世界对她来说是不存在的。但是,在她清澈的目光面前,在她那鼻子、嘴唇细腻的线条面前,在所有这一切证物面前,人们会像面对一条狗那智慧而善良的目光一样心慌意乱。可是人们明明知道,对于人的全部意念,这狗是一窍不通的。在许多有文化教养的人身上,竟然缺乏这些证物!如果有,对他们来说,那就会意味着绝顶的优秀,杰出品德的高尚表现了。人们确实可以琢磨这样的问题,就是在其他的地位低下的兄弟中,农民中,是否有相当于头脑简单的人群中的上等人这样的人类,更确切地说,是否有由于不公正的命运而注定在头脑简单的人之中生活,被剥夺了知识,但是他们更天然地、更出自本性地接近像大部分受教育的人那样的杰出的人呢?这些人就像耶稣家族分散、迷失、被剥夺了理智的成员,像最有智慧的阶层的亲属仍停留在童年时期一样,对他们来说,要具有才具,只差知识这一着了。这从他们眼睛闪射出来的、不可否认的光芒中看得出来,可是这光芒没有用到任何事物上。
\par 母亲见我强忍泪水,对我说:“雷古鲁斯对大场面可见惯了……\footnote{雷古鲁斯为罗马大将,在与迦太基作战中表现极其英勇。但是普鲁塔克并未为雷古鲁斯作过传,倒是西塞罗和贺拉斯称颂过雷古鲁斯的业绩。}再说,你这样对妈妈可不好,咱们也像外祖母一样引用塞维尼夫人的话吧:‘我将不得不把全部勇气都用上,这种勇气你没有。’”\footnote{此处亦是引用1617年2月9日塞维尼夫人致女儿函的大意,原话是:“你若是愿意真叫我高兴,就把勇气全拿出来,我倒是缺少这种勇气的。”}她又想起,对他人的深情可以转移自私的痛苦,便尽量叫我高兴,对我说,她想,她去圣克卢一路上会顺利,她对自己订下的出租马车很满意,车夫彬彬有礼,马车也很舒适。听到这些琐事,我强作微笑,并且用同意、满意的表情点点头。可是这些事只会叫我去更真实地想象母亲的离去,我揪心地望着她,仿佛她已经与我分离。她戴着为去乡下而买的圆草帽,穿着薄薄的长裙。因为要在酷热之中长途跋涉,她才穿上这件长裙,可是已使她变了样,她已经属于蒙特都在圣克卢别墅了,而我则不会在那个别墅见到她。
\par 为了避免旅行可能造成我气闷发作,医生建议我在动身时稍微多喝些啤酒或白兰地,以便处于他称之为“欣快”的状态,在这种状态中,神经系统短时间不那么脆弱。是不是照医生的建议办,我还拿不定主意。但我至少希望,一旦我下定决心那么做,我的外祖母能承认我自己拥有这种权利和理智。所以我谈起这件事,似乎我的犹豫不决只在我到什么地点去喝酒的问题上,是在冷餐部还是酒吧车厢。我看到外祖母脸上现出责备、甚至根本对此不予考虑的表情。一见这种表情,我突然下定了决心非去喝酒不可,既然口头宣布未获得无异议通过,要证明我是自由的,实施这一行动变成了必不可少。我大叫起来:
\par “怎么?我病得多么厉害,你是知道的!医生对我说的话,你是知道的!可是你倒这么劝我!”
\par 待我向外祖母将我身体不适的情形解释完,她现出那么歉疚、善良的神情,回答我说:“那就快去买啤酒或者白酒吧,既然这对你会有好处。”我听了立刻扑到她的怀里,在她的脸上印满了亲吻。我去酒吧车厢喝了过量的酒,之所以如此,是因为我感到,如果不这样,我的病会剧烈发作,那样她会最难过不过的。到了第一站,我又上车回到我们那个车厢,我对外祖母说,我多么高兴到巴尔贝克去,我感到一切都会顺利,我内心感到会很快习惯与母亲远离,这趟车很舒服,酒吧老板和雇员都那么热情,我真愿意经常来往于这条线上,以便有可能再和他们见面。对于所有这些好消息,我的外祖母却没有表现出我那样的兴高采烈。她有意避开我的目光回答我说:“可能你该想办法睡一会了。”并且将目光转向窗户。我们已经放下了窗帘,可是窗帘遮不住整个玻璃窗框,所以太阳能将在林中空地上小憩的温和而又懒洋洋的光线投射在车厢门打蜡橡木上和靠椅的罩子上(比起铁路局挂在车厢高处的广告来,这似乎是对与大自然浑成一体的生活更有说服力得多的一则广告。车厢里的广告挂得太高,是什么地方的风景,我无法看清那地名)。
\par 外祖母以为我闭上了眼睛,可我看见她透过她那带大圆点的面纱,不时向我投过一瞥,然后又将目光收回,然后再反复下去,就像一个人为了养成习惯,极力在进行困难的操练一般。
\par 于是我与她谈起话来,不过似乎这并不使她开心。不管怎样,对我来说,我自己的声音使我感到快乐,同样,我的身体最令人觉察不到的、最内在的活动使我感到快乐。所以,我尽量使之持续下去,任凭我讲话的每一个抑扬顿挫长时间停留在字眼上,我感觉到我的每一目光都确确实实位于它落下去的地方,并在那里停留得超过惯常的时间。
\par “好了,休息吧!”外祖母对我说,“睡不着的话,就看看书!”
\par 说着她递给我一本塞维尼夫人的著作。我打开书,她自己则沉醉在《博泽让夫人回忆录》\footnote{此书名为作者所虚构,并不存在,很可能来源于布瓦涅伯爵夫人回忆录。普鲁斯特曾就布瓦涅伯爵夫人回忆录写过一篇文章,发表于1907年。}之中。每次旅行时,她非带这两位女作家的书不可。这是她偏爱的两个作者。这时,我有意保持头部不动,一旦取了某种姿势,就保持这种姿势不变,从中感受到很大的快乐。我手擎着塞维尼夫人的著作,并不打开,也不垂下目光去看书,在我的目光前面,只有蓝色的窗帘。我凝望着窗帘,觉得真是美妙无穷,这时如果有谁想叫我将注意力从这上面转移过去,我肯定不予置理。我似乎觉得那窗帘的蓝色并非由于其美,而是由于它生机勃勃,正在把自我出生直到我终于将酒吞下去,那酒也开始起作用为止这期间在我眼前出现过的一切色彩全部隐去,以致与这窗帘的蓝色相比,其余的色彩对我来说全都黯淡无光,毫无意义。那些先天盲人,很晚才给他们实行手术,他们终于看见了颜色,当初他们生活其中的黑暗世界想必就是这样的。一位上了年纪的雇员来查我们的车票。他身着制服上装,金属纽扣闪耀着银色的光芒,又使我着迷。我真想请他在我们身旁坐一坐。可是他到另一车厢去了。于是我怀着眷恋的心情想到铁路工人的生活,他们的全部时间都在铁路上度过,大概没有一天不看见这个上了年纪的雇员吧!凝视蓝窗帘,感觉到我的嘴半张半合所感受到的快乐,程度终于开始降低。我想动一动。我活动活动。我打开外祖母递给我的那本书,能够将注意力固定在我这里那里挑选的页数上了。我一边看书,一边感到对塞维尼夫人越来越佩服。
\par 千万不要为一些纯属表面的特点所蒙蔽,这些地方与时代、与沙龙生活相关。正是这些地方使一些人以为只要他们说了诸如“叫我好了,我的好人儿!”或“我看这位伯爵很有风趣”,或者“翻动割下来的牧草是世界上最美妙的事情”\footnote{此句见于1671年7月22日塞维尼夫人写给库朗日的书信,当时被人认为极有风趣,争相传诵。}这类的话,他们就形成了自己的塞维尼形象。已经有德·西米阿纳夫人\footnote{德·西米阿纳夫人(1674—1737),是塞维尼夫人的外孙女,闺名波林娜阿黛玛尔·德·蒙德依·德·格里尼昂,1695年嫁给路易·德·西米阿纳。她同意出版外祖母的信并亲自参加编纂,但出于某些顾虑,将她母亲的信大部分都毁掉了。她本人的书信于1773年发表。}的先例为证,她因为自己写了诸如“德·拉布里先生健康极佳,先生,听到他死亡的消息,他完全受得住”此句出于1735年3月15日致德·埃里古尔函。或“噢,亲爱的侯爵,您的信多么叫我喜欢!有什么办法能不回信呢?”\footnote{此句出于1734年3月8日致高蒙侯爵函。}或者什么“先生,似乎您欠着我一封回信,我欠您几鼻烟壶的香柠檬。我刚还清了八封信的债,马上又有别的信要来了……这大地从来产量没这么高过。看上去是为讨您喜欢”此句出于1735年2月3日致德·埃里古尔函。此类的句子,就自以为与她的外祖母很相像了。而且她也用这种体例写信谈放血,柠檬等等等等\footnote{谈放血的信为1734年11月17日;谈柠檬的信有二,1735年1月13日和1月15日,这几封信都是写给德·埃里古尔的。},自以为这就是塞维尼夫人的书信。但是我的外祖母是从内在的东西,从作者对家人的热爱,对大自然的热爱来接近塞维尼夫人的,她教我喜欢塞维尼夫人真正的美妙之处,那与上述的例子毫不相关。我即将在巴尔贝克遇到一位画家,他叫埃尔斯蒂尔\footnote{埃尔斯蒂尔的名字第一次在本书中出现。在《斯万之恋》中,这个画家以比施的名字出现。埃尔斯蒂尔的原型基本上是惠斯勒。1898年奥朗多夫书店出版的一本小说《亡人的太阳》中有一位画家,名字也叫尼尔·埃尔斯蒂尔。},对于我的审美观有非常深刻的影响。塞维尼夫人与这位画家是属于同一家族的伟大艺术家,因此她作品中的美此后不久便给我留下更深的印象。我在巴尔贝克意识到,她向我们展示事物的方式与埃尔斯蒂尔是相同的,是按照我们感知的顺序,而不是首先就以其起因来解释事物。那天下午,在那节车厢里,我反复读着出现了月光的那封信时,已经心花怒放了:
\refdocument{
    \par 我无法抗拒这种诱惑,我戴上帽子,穿上颜色鲜艳的上衣,其实并非必须如此。我来到网球场上,那里的空气非常温馨,与我卧房一样。我看到千百种莫名其妙的东西,着白衣黑衣的修道士,数位着灰衣和白衣的修女,散乱各处的内衣,挺直身体紧靠大树躲起来的男子……\footnote{塞维尼夫人1680年6月12日致格里尼昂夫人函片断。}
}
\par 这便是此后不久我称之为《塞维尼夫人书信》中的陀思妥耶夫斯基一面(难道她描写景物和性格的方式不和他一模一样吗?)的东西。
\par 我将外祖母送到她的女友家里,我也在那里待了几个小时。然后,晚上,我又一个人乘上火车,至少我没有感到夜晚降临时光难耐。这是因为我不需要在旅馆房间那样的监狱里度过这一夜,而旅馆房间那睡意蒙眬的模样大概会叫我毫无睡意。包围着我的,是列车各种运动那令人镇静的活动。这各种运动伴着我,如果我没有睡意。它们会主动过来与我聊聊,它们的声响像摇篮曲一样催我入睡。我把这声响像贡布雷教堂的钟声一样搭配起来,一会是这个节奏,一会又是另一种节奏(根据我的想象,首先听到四个叠声的等长的八分音符,然后是一个叠声的八分音符疯狂地冲到一个黑色的八分音符上去)。这声响使我那失眠的离心力动弹不得,对失眠施加了相反的压力,将我保持在平衡之中。我一动不动以及以后我的睡意来临,我都感到与那压力密切相关,那种清新的印象与在大自然和生活的怀抱中有一股强大的力量作警戒,因而得到安息所给予我的印象完全相同,好像我在一瞬间得以化身为某种鱼类在大海中安睡,睡意蒙眬中被水流和浪涛荡来荡去,或者化成一只鹰,仰卧在暴风雨这唯一的支柱上。




\paragraph*{1}

\par 和煮鸡蛋、带插图的报纸、纸牌、船在其中拼命开动却不前进的河流一样,日出也是长途铁路旅行的伴随物。我正在清点前几分钟充斥我的脑际的想法,以便意识到我刚才是不是睡着了(是确实没有把握才叫我提出这个问题自问,可是就是这个“没有把握”正在向我提供一个肯定的回答),就在这时,在窗玻璃里,一小块暗色的小树林上方,我看见了几片有凹边的云朵,那毛茸茸的边缘为玫瑰色;固定成形,死去一般,再也不会改变,有如点染鸟翼羽毛的玫瑰色,那羽翼也就化成了粉红,有如画家随兴所至将之置于画面上的粉画。但是我感到与之相反,这片色彩既不是毫无生气,也不是兴之所至,而是必不可少和蓬勃的生机。瞬间,这色彩后面,光线蓄积起来,堆积起来。这色彩越来越深,天空变成一片肉红。我将双眼紧贴在玻璃上,尽量看清楚些,因为我感觉到这与大自然的深邃存在紧密相关。可是铁路方向改变,列车拐弯了,窗框里的晨景为夜色笼罩的一村庄所代替。小村的屋顶为月白色,在仍然镶满星斗的天空下,脏污的洗衣池\footnote{法国农村多有公共的、露天的供村妇洗衣的地方,称为洗衣池。}有如夜色下不透明的螺钿。我正为失去那片玫瑰色的天空而惋惜,就在这时,我在对面的窗子里再度望见了它,但这一次是红色的。铁路又拐了第二个弯,这片天空又抛弃了对面的窗子。结果我就将时间花在从这一面窗奔向那一面窗之中,为的是将我这美妙的、火红的、三心二意的清晨断断续续的片断连接起来,将画面装裱起来,以便有一个全景和连续的画面。
\par 景色变成地势起伏,更加陡峭,列车停在两座山之间的一个小站上。峡谷之底,急流岸边,只能看见守道口人的一所小屋,它陷进水中,那河水就紧贴窗下流过。如果一个人可以是土地的产物,人们从他身上可以品尝到土地独特的风韵,一个村姑就更是如此。我在梅塞格利丝那边鲁森维尔森林中独自漫步时,是多么希望看见一个村姑出现在我面前啊!我希望的,大概就是这个高个子姑娘。我看见她从这座小屋中走出来,背着一罐牛奶,沿着初升的太阳照亮的小路,向车站走来。在高山峻岭遮断了世界其余部分的山谷中,除了这些只停留一小会的列车,她大概从来没有在别处见到任何人。她沿着车厢走来,向几位已经醒过来的旅客出售牛奶咖啡。晨光映红了她的面庞,她的脸比粉红的天空还要鲜艳。面对着她,我再次感受到生活的欲望。每当我们重又意识到美与幸福的时候,这种生活欲望就在我们心中再次萌生出来。我们总是忘记美和幸福是单独存在的,在我们的头脑中总是用某一约定俗成的类型来代替,而这个类型是我们从讨我们喜欢的各个不同面庞之中、从我们领略过的快乐中找一个平均数而形成的。我们只有抽象的形象,而这些形象是死气沉沉的、沉闷乏味的,因为它们恰巧没有一件新鲜的与我们领略过的事物不同的品性,这正是美与幸福所特有的品性。于是我们对生活作出悲欢的判断,我们还以为这是正确的,因为我们以为已经把美和幸福打到里面去了,实际上我们忽略了这两样东西并且用一些中和物来代替它们,而在这种中和物中连美和幸福的一个原子也没有。一个文人,人们向他谈一部新出的“好书”,他还没听就先生厌倦打起哈欠来,情形就是如此。因为他想象的是所有他读过的好书的综合,而一本好书是与众不同的,无法预见的,并不是由前面的所有杰作的总和构成的,而是由某种东西构成的,完全吸收前面的那一总和又绝不足以叫人找到这种东西,因为正好是在它之外。刚才感到厌倦的那个文人,一旦接触到这部新作,立刻会感到自己对这本书所描写的现实颇有兴趣。这位美丽的姑娘立即使我品味到某种幸福(唯一的,总是与众不同的,只有在这种形式下我们才能品味到幸福的滋味),一种生活在她身边可能会实现的幸福。这位美丽的姑娘也是如此,她与我一个人独处时头脑中描绘出的美貌模式毫无共同之处。但是这里在很大程度上又有一个习惯的短暂中止在起作用。我使卖牛奶的女郎受益于我的全部存在,是渴望品尝强烈享受、站在她对面的我。平时我们总是将我们的存在压缩到最低限度来生活。我们的大部分能力停留在睡眠状态,因为这些能力依凭着习惯,习惯知道要做什么,习惯不需要能力。但是在这旅途的早晨,我生活的老习惯中断了,时间、地点改变了,就使得各种能力必须出来。我的习惯是经常在家,不早起。这个习惯现在不在了,我的各种能力就全都跑过来以代替习惯,而且各种能力之间还要比比谁有干劲,像波涛一样,全都升高到非同寻常的同一水平——从最卑劣到最高尚,从呼吸、食欲、血液循环到感受,到想象。在我叫自己相信这个少女与任何其他女子都不同的时候,我不知道是这些地方优美的田园景色为她增加了魅力,还是她使这些地方产生了魅力。只要我能一小时一小时地将生命与她一起度过,陪伴她一直走到急流那里,奶牛那里,列车旁,一直在她身边,感到她了解我,在她的心里有我的位置,那我会觉得生活该是多么甜蜜!她会教我领略乡村生活和晨曦初现的魅力。我向她招招手,叫她给我送牛奶咖啡来。我需要她注意到我。她没有看见我。我叫她。在她那高大的身躯之上,她的面庞是那样粉红、那样闪着金光,似乎别人是透过灯火照亮的彩绘大玻璃窗在看她。她回过头,朝我这边走来,她的面庞越来越宽阔,有如可以固定在那里的一轮红日,我简直无法将目光从她的面庞上移开。这面庞似乎会向你接近,一直会走到你身边,任凭你贴近观看,那火红与金光会使你头晕目眩。她向我投过机灵的一瞥。就在这时,列车员关上车门,列车开动了。我看见她离开车站,重又踏上小径。现在天已大亮:我正远离黎明而去。不论我的兴奋是由这姑娘激发出来的,抑或相反我置身于她的身旁所领略的大部分快乐是我的激动心情所引起,总而言之,她与我的快乐是那样浑成一体,以致我要与她重见的欲望首先是精神上向往着不要使这种兴奋状态完全消失,不要永远与参与其事的那个人分离,哪怕她自己并不知晓。不仅因为这种状态是多么令人愉快,而且特别是(就像一根绳子拉得更紧会发出一种声响,或一根缀线更快地振动会产生另一种颜色一样)它使我看到的事物产生了另一种色调,它将我作为演员带进了一个陌生而又更加无比有趣的世界。列车加速前进,我仍然依稀望见那个美丽的姑娘,她就像与我熟悉的生活完全不同的另一种生活的一部分,一条带子将我的生活与她隔开。在那另一种生活中,事物唤起的感觉再也不相同,现在从那种生活里走出来,就好像自己要死掉一样。为了能享受到至少感到自己与那种生活相联的温馨,大概只要我住在小站附近,就可以每天早晨向这位村姑买牛奶咖啡了。可叹!我向另外一种生活越来越快地走去,而她将再也不会出现在这种生活里!我设想着种种计划,好让我有一天再乘坐这同一列车,再在这同一车站停留,只有这样我才能勉强接受那另外一种生活。设想这种种计划同时还有一个好处,便是给我们那唯利是图的、活跃的、实用的、机械的、懒惰的、离心的精神状态提供了养料。我们的大脑确是这种状态,因为当需要作出努力,以便普遍地、不图个人利害地去加深我们有过的愉快印象时,我们的大脑往往喜欢避开这种努力。另一方面我们又希望继续想着这甜美的印象,大脑就宁愿从未来的角度对此作出设想,巧妙地为这甜美印象的再生准备时机。这对于理解那美好时刻的精髓丝毫无补,却免了我们费心劳神在自己内心重温一时刻的辛苦,使我们指望再度从外界得到这种愉快印象。
\par 一些城市名,维兹莱还是夏尔特尔,布尔日还是波韦,通过这简略的形式,用来指明其主要教堂。我们常常使用这种局部的含义,如果是我们还不了解的地方,最后就会把整个城市的名字刻在心上。当我们打算把城市的概念加进去的时候,这城市的名字立刻就会像铸模一样,给它印上同一风格的同样的刻纹,也把它变成一种大教堂。不过这一次是在一铁路车站上,我看到了巴尔贝克这个地名,在一家冷餐馆的上方,在蓝色警报器上,几乎是波斯体的白字。我匆匆穿过车站和通往车站的大街,我向人询问海滩在哪,为的是只看教堂和大海。从人们的表情看,他们似乎不明白我问的是什么。我现在就在巴尔贝克老城,巴尔贝克陆地,这里既不是海滨,也不是海港。当然,依照传说,显圣的基督确是渔民们从海里找到的。教堂就在距我几米开外的地方,教堂里有一彩绘玻璃窗叙述的就是发现这位基督的故事。修建教堂大殿和钟楼的石头,也确实是从海浪拍击的峭壁上取来的。正因为如此,我想象的大海,是海水一直冲到彩绘玻璃窗前的。可实际上大海距这里还有五里\footnote{法国古里,一古里约等于四公里。}多路,在巴尔贝克海滨的教堂圆顶旁那个钟楼,我从前在书本上读过,说这钟楼本身就是一座诺曼第峭壁,上面各种籽粒汇聚,群鸟盘旋,所以我一直以为那钟楼底座是接受大海激起千重浪的飞沫的。实际上,钟楼耸立在一座广场上,两条有轨电车线从这里分叉,对面是一家咖啡馆,门口金字招牌上写着“台球”二字。钟楼的背后是一大片住宅,住宅屋顶上没有掺杂一根桅杆。我一面留神咖啡馆,一面留神向其问路的行人,一面又注意着要回去的车站,走进教堂。教堂与其余的一切构成一体,仿佛是一种偶然,是这天下午的产物。那软绵绵的在天空中鼓起来的圆顶好像一颗果实,住宅烟囱沐浴其中的同一阳光,催熟了那粉红、金色而又进口就化的果皮。但是,认出众使徒的雕像——我曾经在特罗卡德罗博物馆看见过铸出的圣像——站在教堂大门口的门洞里,在圣母的两旁列队而立,等待着我,似乎是为着欢迎我时,我就只愿意考虑雕塑的永恒意义了。圣母那仁慈、温和的面孔,短而扁的鼻子,弓着的背,似乎唱着某一天的“哈利路亚”欢迎似的向前走来。但是人们发觉这些圣像的表情是呆滞不动的,正像死人的表情一样。只有人围着他们转时,他们的表情才有所改变。我心中暗想:就是这里,这就是巴尔贝克教堂。这个广场看上去知道自己的荣光,它是世界上唯一拥有巴尔贝克教堂的地方。迄今为止我见过的,是这个大名鼎鼎的教堂、这些使徒、这大门之下圣母的照片,仅仅是拓片。而现在,是真的教堂,真的圣母像,唯一无二的,近在眼前了:这就远远胜过从前了。
\par 说不定也不如从前。好比一个小伙子,到了考试或者决斗的那一天,当他想到他储备的知识和他准备表现出的勇敢时,会感到人们向他提出的问题、他打出去的子弹,都没有什么了不起了。同样,我的头脑中远远超出我眼前的复制品的,是高高耸立在门洞中的圣母形象。各种变故可以构成对复制品的威胁,却无法企及我头脑中的圣母;如果有人将复制品摧毁,我头脑中的圣母却不受任何损伤;她是尽善尽美的,具有世界性意义。现在,我的头脑见到了这个早已为人雕塑过一千次的雕像,对这个雕像外表仅仅是石头,我伸出手臂即可触及,占据着一席之地,还有一张选举布告和我的手杖头作她的对手,都感到惊异。这一席之地与广场连成一片,与主要街道的出口不可分,她无法避开咖啡馆里和电车办公室里人的目光,她脸上受到半抹夕阳的照耀——过一会,几小时之后,便是街灯之光的照耀了——另一半为贴现银号的办公室接受去了;她与那家信贷公司分理处同时被糕点铺灶间的怪味所降服,任凭凡人肆虐;如果我也想在这石头上刻上我的名字,那么她,这著名的圣母像,迄今为止我赋予她以凡人的生命和捕捉不到的美的,巴尔贝克的圣母,独一无二的(可叹,这也意味着只此一家)圣母,就要以她那沾满了与其毗邻的房屋同样的煤炱,向所有前来瞻仰她的崇拜者,显示我用粉笔画下的痕迹和我的名字的各个字母,而无法去掉这些字迹。总而言之,这向往已久的不朽的艺术品,我觉得她和教堂一样,变成了一个小小的石头老太太,我可以量出她的身高,数出她的皱纹了。
\par 时间过得飞快,该回车站了。我要在车站等待外祖母和弗朗索瓦丝到来,然后一起到巴尔贝克海滨去。我忆起从前读过的对巴尔贝克的描写,忆起斯万的话:“精美之至,和锡耶那\footnote{锡耶那为意大利佛罗伦萨附近一古城。}一样美。”我只能用偶然来解释我的失望,是我的精神状态不好,是我很疲劳,是我不会欣赏,我极力这样安慰自己,想到对我来说还有别的完美无缺的城市,说不定很快就能看到,就像在珍珠般的细雨中,在坎佩尔勒雨滴清新的淅沥中穿过沐浴着阿方桥\footnote{坎佩尔勒及阿方桥的联想,请见本书第一部。}那绿色和玫瑰色的霞光一般。就巴尔贝克来说,我一走进这座城市,就好像把一个本应密封的地名打开了一条缝。这里,一列有轨电车,一家咖啡馆,广场上来往的人群,贴现银号的分店,无法抗拒地受到外部压力和大气力量的推动,一下子拥进了这个地名各个音节的内部。这些东西进去以后,这几个音节又关上了大门,现在,它任这些事物镶嵌起波斯式教堂的大门,再也不会将这些事物排除在外了。我在应该把我们送到巴尔贝克海滨的当地小火车里找到了外祖母,可是只有她一个人。她提前打发弗朗索瓦丝前来,以便事先做好一切准备。但是她指点弗朗索瓦丝有误,结果叫弗朗索瓦丝走错了方向。此刻,毋需怀疑,弗朗索瓦丝的火车正向南特飞快奔驰,说不定到了波尔多她才会醒过来。
\par 车厢里充满了日落时分那转瞬即逝的余晖和下午那不肯散去的炎热(可叹,在落日余晖映照下,我从外祖母的整个面庞上看到她因天气炎热而多么疲惫不堪)。我刚一坐下,她就问我:“巴尔贝克怎么样?”因为满怀希望,她的微笑是那样热情爽朗,她以为我一定感受到了极大的快乐。见她如此,我简直不敢立即向她承认我很失望。加之,随着我的身躯越来越接近它应该习惯的地点,我头脑中追寻的印象不像从前那样萦绕我的脑际了。到最后,距旅行的终点还有一个小时路程时,我就极力想象巴尔贝克的旅馆老板是什么模样了。对他来说,此刻我还不存在。我多么希望向他作自我介绍时,有一个比外祖母更有名气的旅伴——外祖母肯定要求他降价。似乎他必然十分傲慢,但轮廓很模糊。
\par 在这段小铁路上,火车不时在一个车站停车,一站又一站,巴尔贝克海滨始终没有到。光是这些车站的站名(安加市,马古维尔多市,古勒夫尔桥,阿朗布市,老圣马尔斯,埃蒙维尔,梅恩市\footnote{这些地名有真有假;有的在这条铁路线上,多数不在这条线上。})我就觉得莫名其妙。在一本书中读到这些地名时,说不定会觉得它们与贡布雷附近的某些地名有关系。但是对一位音乐家的耳朵来说,两个音节,即使由数个相同的音符组成,如果谐音色彩和组合不同,也可能毫无相像之处。同样,这些由沙子、狂风呼啸而又空旷的空间和盐分组成的难听的名字,“城市”一词安在上面安不住,就像“飞鸽”这个词里面的“飞”也安不住一样。没有什么比听到这些名字更会令我想到别的地名,如鲁森市或马丹市。我在饭桌上、在“大厅”里那样经常听到我的外祖母提到这些地名,这些地名早已获得了某种暗中的魅力,说不定其中还混进了果酱的香味,木材燃烧的味道和贝戈特哪一本书书页的气味,对面房屋那赭红的颜色,以至直到今天,这些地名像气泡一样重又从我脑海深处漂上来的时候,虽然它们要穿过一层层,才能达到表层,却仍然保留着自己独特的品性。
\par 有些小站高踞于自己的沙丘上俯瞰着远方的大海,有些小站则位于深绿色、形状令人不快的小山脚下,已经准备睡去——那小山,形状就像刚走进去的一间旅馆房间里的长沙发,山下是一些别墅,再伸展下去便是一个网球场,有时是一家赌场。赌场大门上的旗帜迎着凉爽的海风飒飒作响,场中空荡无人,焦虑不安。初次向我显示自己主人的小站,乃通过其司空见惯的外表来显示——戴着白色遮阳帽的打网球的人,生活在自己的柽柳和玫瑰身边的车站站长,一位戴着扁平草帽的太太。那妇人沿着我永远不会体验得到的生活的日常轨迹,唤回在外久久不归的猎兔狗,然后回到自己的木头小板房里去,屋中已经燃起灯火。这些小站以这些司空见惯、使人非常熟悉的现象,无情地刺伤着我这陌生的目光和人生地不熟的心。
\par 我们走进巴尔贝克大旅社\footnote{普氏1907——1914年夏天到卡布尔度假,他描写的巴尔贝克大旅社便是卡布尔大旅社。}的大厅,面对着仿大理石的偌大楼梯,我的外祖母不顾会增加那些陌生人的敌意和鄙视——我们就要生活在这些陌生人之中——在和旅社经理讲“条件”时,又怎样加重了我的痛苦啊!经理是个普萨式的人物,满脸满嘴都是毛病(挖掉好几个疖子,在脸上留下了伤疤。由于祖籍遥远,童年时期起便在世界各地闯荡而口音混杂,给他的声调留下了毛病),他身穿花花公子的大礼服,闪动着心理学家的目光。“慢车”一到,他一般总是把阔老爷当成满腹牢骚的人,而把住旅馆的吝啬鬼当成阔老爷!他大概忘记了他自己一个月也挣不上五百法郎的薪水,却深深鄙视那些认为五百法郎——或者更确切些,如他所说,是“二十五路易”——“是个数目”的人,总是把这些人当成是贱民的组成部分,而大旅社可不是给这些人预备的。在这家豪华大旅馆里,有些人并不花很贵的房钱却也受到经理的敬重,这也是真的,条件是经理确切知道这些人注意开支是因为吝啬而不是因为穷。吝啬是一种毛病,在各个社会阶层中均可遇到,因此它确实丝毫不会损害威望。有社会地位,这是经理唯一注意的事情。有社会地位,更确切地说,在他看来有说明地位高的标志,例如走进旅社大厅不脱帽啊,穿高尔夫球裤和紧身短上衣啊,从镶金、带红的高级皮革烟盒里往外掏雪茄烟啊之类(可惜,这些优越性,我一样也没有)。他用讲究的字眼去点缀自己的生意经,但意义总是用得相反。
\par 我坐在一张长椅上等待。我听到外祖母拿腔拿调地问他:“房钱……是什么价?……啊!太贵了,我这点钱可不够!”他听外祖母说话时,帽子也不摘下,还吹着口哨,外祖母也不生气。我听着这话,尽量逃进自己内心深处,竭力到一些永不改变的想法中去游荡,不让任何有活力的东西露出我的躯体表面——就像动物的表皮出于抑制作用,当人们伤害它们的时候,它们装死一动不动一样——以便在这个地方不要太难受。我对这种地方还完全不习惯,看到别人对此很习惯就使我更加敏感。我看见一位衣着华丽的妇人,经理对她毕恭毕敬,对跟在她身后的小狗十分亲热;一个衣着讲究、样子可笑的青年,帽子上缀着羽毛,回到旅馆,问“有没有我的信”。所有这些人都将登上那假大理石的台阶视为回家,他们似乎对这一切都很习惯。与此同时,一些大概很不精通“接待”艺术却带有“首席接待”头衔的先生,严厉地向我投以迈诺斯、埃阿刻和拉达芒特\footnote{这是宙斯的三个儿子,他们死后被召至地狱作判官。迈诺斯的名字在《追忆似水年华》中经常出现。}的目光(我将自己赤裸裸的心灵投入这目光之中,就像投入一个再没有任何东西保护我的心灵的未知世界一样)。再远一些,在一扇关着的玻璃门后,有一些人坐在一间阅览室内,要描写这个阅览室,要依次描写我想到这些有权利在那里安安静静阅读的人上人所享的清福,想到如果我的外祖母不顾我会产生这样的印象,命令我走进去的话,她会使我感到多么恐惧,我恐怕必须相继选择但丁笔下赋予天堂和地狱的各种色调了。
\par 过了一会,我那种孤独的印象更加浓重。我向外祖母承认,我感到不舒服,我觉得说不定我们很快就不得不返回巴黎。她没有抗议,说她要出去买些物品,无论我们是走还是留下,反正这些物品都有用(后来我才知道这些东西都是给我买的,因为所有这些我缺的东西,都在弗朗索瓦丝身上);等待外祖母返回时,我到街上信步走走。街上熙熙攘攘,人群使大街保持着与室内同样的炎热,理发店和一家糕点铺子还开着门,常客们在糕点铺子里站在迪盖特鲁安(1673—1736),是圣马洛的海盗。他的塑像也在圣马洛。他在《回忆录》中,讲述了许多历险事情。塑像前吃冰淇淋。这塑像引起我的快乐,那与他的形象出现在一本画报中,也能使在外科医生的候诊室内翻阅画报的病人得到快乐一样。一些人对我相当无所谓,使我感到惊异。旅社经理满可以建议我到城里走走散散心,一个新住所,这种受罪的地方,在某些人眼里也是可以显得是“令人心旷神怡之小住地点”了。旅社的说明书就是这么说的。这说明书可能有些夸大其辞,不过这是面向所有主顾的,他们专门迎合主顾之所好。确实,为了把主顾招到巴尔贝克大旅社来,说明书不仅提到什么“美肴佳馔”、“游艺场花园令人销魂”,还说什么“时装女王陛下驻足,不被视为笨伯之人不会因奸污而不受惩罚,任何有教养的男子可能都不愿意冒此风险。”



\paragraph*{2}

\par 我越是怕外祖母伤心,就越是需要她。她大概很灰心丧气,感到如果这么点累我都受不了,那就没有希望了,任何旅行对我都不会有好处。我下定决心回去等她。经理亲自走来按了一个按钮:一个我还完全陌生的人物,人称“lift”\footnote{英文:电梯。}的(此人被安顿在旅社的最高点,大概是诺曼第教堂灯笼式天窗的地方,好像是玻璃板后面的一幅照片或管风琴演奏者在自己的房间里)开始朝我走下来,动作之轻盈有如家养松鼠,灵巧而又是被束缚之物。然后他又沿着一个柱子滑下来,将我带在他身后朝这商业主殿的圆顶升去。每一层上,通道小楼梯两侧,阴暗的游廊成扇形展开。一个收拾房间的女仆人抱着一个长枕头,从游廊里走过。黄昏的光线使她的面庞模糊不清,我把自己最狂热梦想中的面具贴到她的脸上,但是从她朝我递过来的目光里,我看到的是对我这个一钱不值的人的厌恶。每一层唯一的厕所形成仅有的一排竖着的玻璃窗,从玻璃窗透进的光线照亮了这毫无诗意的半明半暗的地方,神秘得很。在永无尽头的向上走的过程中,为了打消我默默穿过这神秘地方所体验的致命焦虑,我便对那个年轻的管风琴演奏者、我的旅程的匠师、我被俘的伙伴开了腔,他还是继续拉他的乐器音栓和推导管。我为自己占这么大地方,给他惹这么多麻烦而向他表示歉意,问他我是否妨碍他施展艺术才能。在这种地方,为了吹捧名家高手,我不仅表现出好奇,而且还忏悔自己对此十分偏爱。但是他不理我,可能对我的话惊异不止;也可能专心致志于自己的工作,一心想着各种标记;也可能他耳背,对这个地点很尊重;也可能怕出危险;也可能懒得动脑子;也可能这是经理的命令。
\par 一个人,哪怕无足轻重,我们认识他之前和认识他之后,他对我们所采取态度的变化,恐怕没有什么比这个更能赋予我们对外界现实的印象了。我一直是同一个人,下午稍晚时候,乘坐了来巴尔贝克的小火车,一直怀着同一颗心。但是,六点钟的时候,由于无法想象出经理、豪华大旅社、其服务人员是什么模样,我抵达的时刻心中有一种模糊而又带几分恐惧的期待。现在,在这颗心中,则是走南闯北的经理那脸上挖掉的疣子(虽然如他自己所说,“特点是罗马尼亚”\footnote{经理将“祖籍”origine说成了“特点”——originalité。}——因为他总是使用他认为高级的词儿,而又没有发现用得有毛病——实际上他的国籍是摩纳哥),为招呼电梯而按铃的姿势,开电梯的本人,从大旅社这个潘多拉盒子\footnote{潘多拉是希腊神话中的人物,她有一个神秘的盒子。这盒子一打开,世界上所有的灾难、坏事都冒出来。}里冒出来的整个木偶戏剧场沿幕的人物。这一切都无法否认,终身在此。而且,像一切人造的东西一样,没有繁殖能力。我并没有参与这种变化,但至少这种变化向我证明在我的外界发生了什么事情——这事情毫无意义,是自在的——而我则像一个游客,开始游览时,太阳在面前;待他看见太阳到了身后时,便得知时间已经过去了。
\par 我累得骨头都碎了,我发着烧,睡觉必需的物品一点也没有,不然我早就睡下了。至少我想在床上躺一会,可是面对这一大堆强烈的感受,我反正是无法歇息的,又何必呢?这一大堆强烈的感受对我们每个人来说,不等于他的物质躯体的话,至少也等于他的有意识躯体,因为包围着这个躯体的陌生事物,虽然强迫它在一贯保持警觉的防御基础上进行感知,却也能将我的视觉、听觉、所有的感官保持在很受局限、很不舒服的姿势上(即使我把腿伸开),就像拉巴吕红衣主教\footnote{让·拉巴吕(1421—1491),本为路易十一之神师,后来为红衣主教,因为与大胆查理进行秘密谈判,被路易十一关在洛什城堡国家监狱中,在铁笼中度过十一年,后来经教皇西克斯特四世干预,获得释放。}在笼子里的姿势一样,既不能站,也不能坐。在一间卧房里,我们的注意力要求将一些物品放在这里,待习惯了又好像将这些东西搬走了,给我们自己腾出地方来。可是在巴尔贝克的卧室里(仅仅名义上是“我的”卧室),我觉得没有一点空地方,房间里塞满了不认识我的器物。我向它们投去戒备的目光。它们也报我以戒备的目光。它们丝毫不在乎我的存在,现出我打扰了它们正常生活秩序的模样。在家里,一星期当中我只有几秒钟听见我的挂钟走动,那就是我从沉思默想中走出来的时候。旅馆里这只挂钟则一刻不停地用一种陌生的语言连续说着可能使我极为不快的话语,因为宽大的紫色窗帘默默倾听,不作回答,但是那种态度,与人耸耸肩膀用以表示看见一个第三者使他们很恼火极为相似。房间天花板很高,窗帘赋予房间几乎一种历史意义,简直能叫人觉得它很适于暗杀吉斯公爵\footnote{吉斯公爵即亨利一世(1550—1588),他于1588年12月28日被觊觎其王位的亨利三世在三级会议上暗杀。画家保罗·德拉洛什(1797—1856)曾据此画了一幅油画,勒巴吉及加尔麦特于1908年亦据此事拍成电影。},以后又适于库克旅行社的一个导游率领旅游者前来参观\footnote{汤姆斯·库克(1808—1892)于1841年组织了一次“快乐列车”旅行,这便是他那鼎鼎大名的旅行社的起源。他死时将旅行社作为遗产交给了他的长子。},但是决不适于我的睡眠。沿墙有数个玻璃小书橱,它们的存在对我是个折磨。特别是房间中横着一面全身大穿衣镜,这东西搞得我心慌意乱,如果不挪走它,我就觉得自己根本别想放松下来。我不时抬眼望望天花板——在巴黎,我房间中的各种器物不妨碍我的目光,不比我自己的眼球更妨碍,因为它们只不过是我的各种器官的附件,是我自己的一种放大——天花板上方是旅社最顶端的平台,是外祖母特意为我挑选的。库斯草的气味将其攻势一直推进到比我们看得见和听得见的更为幽密的地方,推进到我们感受到各种气味的特点的地方,推进到了我最后的战壕里,几乎推进到了我的内心。我不无厌倦地用惊慌不安的鼻子去嗅,以这种无益的不断反击去对付它的进攻。再也没有地盘,没有房间,没有躯体,只有一味受到将我重重包围的敌人的威胁,热度一直侵入我的骨髓,我孤立无援,我真想死。就在这时,外祖母走了进来。立刻,无限的空间向我受到压抑而要扩张的心敞开了。
\par 她身穿一件高级密织薄纱室内便袍。在家时,每逢我们这些人中有哪一个病了,她就要穿上这件便袍(她说,穿了这件衣服很舒服,她总是将她做的事归之于自私的动机),这件便袍是为了照顾我们,看护我们的,是她的用人服,看护工作服,她的修女服。用人和看护对人的细心照顾,她们的善良,人们体会到的她们的优点,人们对她们的感激,都更增加了她们对人的印象,她们觉得人的外表与内心不同,人自我感到孤独,自己背负着头脑中思想的重负、自己的生活欲望。我知道,我和外祖母在一起时,不论我内心多么忧郁,它都会被更大的怜悯所接受。我的一切,我的烦恼,我的欲望,在外祖母那里都会得到支持。用以支持的东西,便是她保持和扩大我生活的欲望比我自己的这种欲望更强烈;我的想法在她心中延伸,不需要改变方向,因为这些想法从我的头脑里传到她的头脑里并没有改换地点,也没有换人。就像一个人站在穿衣镜前想要打上领带,可是不明白他看见的那一头与他的手动作的方向跟他本人相比并不在一边,或者一条狗在地上追逐着昆虫跳跃着的影子一样。在这世界上,人们总是受到躯体外表的蒙蔽,因为我们不能直接感受到心灵。我也这样上当受骗,一头扎进外祖母的怀里,将我的双唇贴在她的脸上,似乎这样我就能进入她向我敞开的宽阔的胸怀。我这样把嘴紧贴在她的双颊上、她的前额上以后,我从那里吮吸到那样有益、那样富有营养的东西,我半天一动不动,是吃奶孩子的那种认真、放心大胆的贪婪。
\par 然后我百看不厌地注视着她那宽大的脸膛,那轮廓就像一片热烈而又平静的美丽云霞,可以感觉到那后面闪射着柔情之光。一切多少还能接受她的感受的东西,一切还可以说属于她的东西,都因此而立刻变得那样神圣,那样超俗,我情不自禁地用手掌理着她那刚刚灰白的秀发,怀着尊敬、小心翼翼和轻柔,似乎我抚摸的是她的善良。她在难过之中又为使我免去了一种痛苦而感到那样高兴,就这样一动不动过了一会。对我那疲惫不堪的四肢,是那样平静安宁的一瞬,是那样甜蜜。过了一会,我见她想帮我睡下,打算给我脱鞋,我作了一个手势阻止她,开始自己脱衣裳。我的手已经碰到上衣和矮靴的头几个纽扣上,她用乞求的目光拦住我的手。
\par “噢,别这样,”她对我说,“对外祖母来说,这叫她多开心!尤其是你今夜需要什么时,不要忘了敲墙,我的床就靠着你的床,隔栅非常薄。等一会你睡下以后,就敲敲试试,看看咱们是不是能互相听得见。”
\par 果然,那天晚上,我敲了三下。一个星期以后,我不舒服时,有几天我每天早晨都重复这三下,因为外祖母要早早喂我喝牛奶。当我觉得听见她已经醒了以后——为了不叫她等待并且能在喂我牛奶之后马上再度入睡——我鼓起勇气小声敲了三下,胆怯地,轻轻地,但不管怎样却是清清楚楚地,因为我担心如果搞错了,她还在睡,那就会打断她的觉,可我又不愿意她继续侧耳倾听是否是我呼叫,如果她起先没有听清的话。我不敢再敲了。我这边刚一敲三下,立刻就听到另外三击。这三击音调不同,充满平静的威严,为了更加清晰,重复两次,那意思是说:“别着急,我听见啦!过一会就来!”顷刻,外祖母来到。我对她说,我真担心她听不见我的声音,或者她以为那是隔壁的什么人在敲。她笑了:
\par “将我可怜的小狼\footnote{普氏的母亲对自己的两个儿子均称“我的小狼”。}敲击声与别人混淆起来,怎么会呢!就是有一千个人敲,外祖母也辨别得出来呀!你以为世界上还有别人这么傻,这么激动,这样又怕吵醒我又怕人家听不明白他的意思吗?不管怎样,这个小老鼠只要一抓,人家立刻就能认出它来,特别是这个小老鼠跟我的小老鼠一样是独自一人,又叫人可怜的时候!我听见它犹犹豫豫已经有一会了,它在床上折腾,耍各种把戏。”
\par 她把百叶窗打开一半。在旅馆前突的附属建筑上,阳光已经在屋顶上安身,就像早起的盖屋顶工人早早就开始干活,默默地干完活计以免吵醒还在沉睡的城市,而城市一动不动使他显得更加心灵手巧一样。她告诉我几点了,天气会怎样,说我用不着一直走到窗边去,说海上有雾,告诉我面包店是否已经开门,对我叙说听到其声响从街上走过的那辆车是什么样的:这无足轻重的打开窗帘,这可以忽视的、任何人都不在场的清晨“序曲”,只属于我们两个人的一小块生活。白天,当我谈到早晨六点钟的漫天大雾时,我会在弗朗索瓦丝或一些陌生人面前高高兴兴地提起这些,那意图并不在于显示我获得了某种知识,而是要显示我一个人所得到的疼爱。这甜蜜的清晨一刻,由我敲三下、另三下作答这富有节奏的对话开始,像一曲交响乐般展开。柔情和快乐力透隔墙,那墙变成了和谐的、非物质的东西,像天使一般歌唱着。那为人热烈期待的三击回答,重复两次。隔墙善于通过这三击,以天神报喜的轻盈和音乐美的忠诚,将外祖母整个的心灵和就要过来的诺言传送过来。但是抵达巴尔贝克当天那一夜,外祖母离开我以后,我又难过起来,就像在巴黎离家时我已经很难过一样。构成我们眼前生活中精华的事物,对于我们从精神上以我们的接受能力来赋予其未来的模式,而上述事物并不在这未来模式之中,总是以极大的拼死抗拒来对抗。我这种对于在陌生房间里过夜的恐惧——许多人也有这种恐惧——说不定只是上述这种抗拒最普通、最模糊、最机能性、几乎最无意识的表现形式。一想到我的父母有一天可能会死去,我可能为生活所迫不得不远离希尔贝特而生活,或者只是不得不在一个永远再也见不着自己朋友的国度定居,常常使我感到可怕之极,那抗拒就在这恐惧的深处。我自己的死亡,或者像贝戈特向人们许诺的那种在自己著作中永生,我很难想象。我无法将我的回忆、我的缺点、我的性格带到那种虽死犹生中去,这些东西不能接受自己不再存在的概念,也不希望我有一个它们没有位置的虚无或永生。
\par 在巴黎时,有一天我身体特别不适,斯万对我说:“你应该动身到大洋洲那些美妙的海岛上去。那时你就会知道,你再也不会回来了。”\footnote{1888年,英国小说家史蒂文森到大洋洲海岛上去休养,1894年死于萨摩亚群岛。画家高更,到大洋洲去以后,也于1903年死于马克萨斯群岛。}那时我真想回答他说:“那我就再也看不见你的女儿了,那我就要在她从未见过的人和物之间生活了。”然而我的理智却告诉我:“既然你不再为此苦恼,那又有什么关系呢?当斯万先生对你说你将不再回来时,他的意思是你会不想回来;既然你不想回来,这就说明,在那里,你会幸福。”因为我的理智知道,习惯——这种习惯现在即将担负起一项重任,要使我爱上这陌生的住所,爱上改变了位置的大穿衣镜,爱上改变了颜色的窗帘,爱上停摆的挂钟——也担负着使一开始并不讨我们喜欢的伙伴变成亲爱的朋友,赋予面庞另一种形状,使一个人的嗓音变得热情动听,改变心中爱恋对象的任务。自然,对某些地点、某些人新的友情,就是忘记昔日友情的网。但是我的理智正好认为,我可以毫无恐惧地设想一种生活前景。在那种前景中,我将永远与一些人分离,我将忘记他们。这种生活向我的内心作出了忘却的承诺,而忘却只会使绝望更加疯狂,这似乎构成一种安慰。这倒不是说,待习惯了分离之后,我们的心也不会感受到习惯势力那镇痛的效用,而是说,至今这颗心仍在痛苦罢了。惧怕将来我们再也看不见我们喜欢的人,再也不能与他们交谈,正是在这种前景下,我们今天才会得到最难得的快乐。如果我们想,在受到这种剥夺的痛苦之上再加上当前对我们来说似乎更为残酷的事:并不像感受一种痛苦一样感到这种担心,而是对此漠然置之,这种恐惧就不但不会消散,反而会更加增长了。因为,如果是这样,我们的“自我”就变了:不仅我们的父母、我们的情妇、我们的各位朋友的魅力再不存在于我们的四周,而且我们对他们的钟爱,也就完全从我们心中拔除了。而这种钟爱是我们今日内心很重要的一部分。今后我们会喜欢上这种与他们分离的生活,而今日一想到这种生活就叫我们感到恐惧。倘若如此,那便是我们自己真正的死亡。死亡继之以复活,这是真的,但这复活已在与前不同的自我之中,原来的自我中注定要死亡的那些部分是无法上升到热爱这个与前不同的自我的。如今恐惧、抗拒、反抗的,也正是原来的自我中注定要死亡的那些部分——甚至是最羸弱的部分,诸如对一个房间的大小、气氛莫名其妙的眷恋之类。必须看到,这是一种抵抗死亡的潜在的、局部的、确实的、真实的方式,长期地、绝望地、逐日地抵抗那一部分一部分的、连续不断的死亡的方式。这种死亡潜入我们整个生命进程之中,每时每刻从我们身上分离出一片一片的我们自己。正是在这些东西的坏死上,新的细胞增殖起来。对于像我这样一个天生神经过敏的人(也就是说,在这种天性的人身上,中间关节,即神经,不能正常发挥功能,阻挡不住哀叹沿着自己的道路朝意识驶去,而是相反,任凭这哀叹来到,清晰的、疲惫的、无数的、痛苦的哀叹,哀叹自我中那即将消逝的最朴素无华的成份)来说,在这陌生的过高的天花板下我们所感受到的那种焦虑的恐惧,只不过是一种友情发出的抗议。那种对于熟悉而较低的天花板的友情还劫后余生,活在我的心里。说不定这种友情也会消失,另一种友情占据了它的位置(到那时,死亡,然后是一种全新的生活,就会在“习惯”这个名词下,完成它们双重的大业)。但是,直到这友情消亡之前,每天晚上,它还要痛苦,这第一天晚上尤甚。它面对着已经成为现实的前景,再也没有它的位置的前景,在反抗。每当我的目光无法从伤害它的东西上移开,设法停驻在不可企及的天花板上时,它就用哭诉的叫喊来折磨我。
\par 到了第二天早晨怎么样了呢?一个仆役前来将我叫醒,给我送来热水。我洗脸梳头,拼命在我的旅行箱里找我需要的物品,可是徒然,我从里面拽出来的乱七八糟的东西,全都一点用也没有。我已经想到了早餐和散步的快乐,就在这时,从窗户和书柜的每一扇玻璃上,就像从船舱的舷窗上望出去一样,我看到了裸露的大海,无遮无拦,有一半是在自己广阔幅员的阴影中,那是一条纤细而移动的直线所划定的边界。啊,多么快乐!双眼追逐着浪涛,看那浪涛一个接一个地跃起,好像在跳板上跳跃的运动员。多么快乐!我手上拿着僵硬的、上了浆的、上面印着旅馆名字的毛巾,想用这块毛巾擦干身体,可怎么也擦不干。我不时回到窗旁,再向这令人头晕目眩、山岳一般的庞大马戏团再看上一眼,向那此处彼处磨光而又半透明的蓝宝石的波涛白雪般的峰巅再看上一眼。那浪涛,怀着沉着的凶猛和狮子皱眉般的架势,任凭其山坡崩坍,飞滚落下。阳光又用看不见面庞的微笑为这山坡增色。
\par 此后,每天早晨我都置身窗口,就像在驿车里睡了一觉扑到驿车的玻璃窗口去一样,为的是看看我所向往的山脉在夜间是靠近了,还是远去了。在这里,这些大海的丘陵,在狂舞着回到我们身边之前,可能会后退得很远,以至常常要在一片长长的沙土平原后面,我才能在很远的地方依稀望见它们那最早出现的起伏,那远处半透明,雾气笼罩,蓝莹莹的,好似托斯卡纳\footnote{托斯卡纳为意大利中部地区。}文艺复兴前期画家作品景深处的冰川例如乔凡尼的名画《耶稣诞生》、《圣约翰·巴蒂斯特撤至荒原》等。有时,紧挨着我,阳光在这些波涛之上欢笑,那波涛呈嫩绿色,恰似潮湿的土地和光线液体般的流动使高山草地保持着嫩绿一般(在山上,阳光此处彼处展开,有如不均衡地跳跃着欢快地走下山坡的巨人)。此外,海滩与波浪在世界之余部分辟出这个豁口,为的是叫阳光从这里经过,叫阳光在这里积累起来。在这里,从大海过来的方向和我们的肉眼遵循的方向望过去,是阳光在移动着大海的山峦起伏,是阳光确定其位置。光线的千变万化同样会改变一个地点的方位,同样会在我们面前树立起新的目标,使我们产生要达到这目标的欲望,而只有经过千辛万苦长途跋涉才能达到。
\par 清晨,太阳从旅馆后方过来,在我面前展现出阳光普照的沙滩,直到大海最前沿的城堡。太阳似乎将城堡的另一坡也展示给我,并且鼓动我踏着它光芒的转轮,去继续旅行。这旅行是原地不动的,但是透过各个时刻起伏不定的景观中那最美妙的景色,它又是千变万化的。从这第一个清晨开始,太阳总是伸出一根微笑的手指,将远方大海那蔚蓝的峰巅指给我看。这些高峰在任何一张地图上都没有名字。太阳在山脊和雪崩那轰响而又纷乱的表面上尽情游荡累了,最后便来到我的房间里避风,在散乱的床上懒洋洋地躺着,在湿乎乎的洗脸池上,打开的箱子里,摘下它的珍宝。它那辉煌的光焰本身和用得不是地方的奢侈,更加深了杂乱文章的印象。
\par 一个小时以后,在那偌大的餐厅里,我们正吃午饭,从柠檬的皮囊中往两条箬鳎鱼上洒上几滴金水。过了一小会,我们的盘子里就只剩下鱼刺了。鱼刺弯弯,有如一片羽毛;铮然有声,有如一把齐特拉琴。可惜,这时外祖母感觉不到海风那凉爽而富有活力的吹拂,她觉得真是残酷。这是因为门窗虽然透明,却关闭着,像一个橱窗一样,虽然让我们看到整个海滩,却将我们与海滩分隔开来。天空完全进入门窗玻璃之中,以至天空的蔚蓝色似乎是窗子本身的颜色,那雪白的浮云,似乎是玻璃上的毛病。我确信自己是如波德莱尔所说“坐在防波堤上”\footnote{指波德莱尔散文诗《海港》中描述的模糊的回忆。},或“贵妇人小客厅深处”\footnote{出自《恶之花》中《忧郁与理想》。},我自问是不是他所说的“普照大海的阳光”\footnote{出自《恶之花》中之《秋歌》。普氏深爱此诗,在著作及通讯中经常引用。}就是此刻的这种阳光——与落日的余晖很不相同,那是单纯而表面化的,如同一抹金光而又颤动不已——它像黄宝石一般燃烧着大海,使大海发酵,变成一片金黄而又成乳状,好似啤酒;浮着泡沫,好似牛奶。此处彼处,不时又有大块蓝色阴影游来荡去,似乎哪一位神在天空中摆动着一面镜子,将阴影移来移去以自娱。巴尔贝克的这间餐厅,光秃秃,充满绿色的阳光,如同游泳池中的水。几米开外的地方,涨潮的海水和日正中天,如同在天堂前面一样,正竖立起宝石和黄金的不可攻克的游动的堡垒。
\par 可惜这间餐厅与贡布雷那间朝着对面房屋的“大厅”不仅仅外表上不同。在贡布雷,人人都认识我们,所以我不顾及任何人。在行海水浴的生活里,人们是不认识他的邻居的。我年纪还不大,而且一直十分敏感,不会放弃讨人欢喜和占有他们的欲望。一个上流社会的男子对于在餐厅里用餐的人,可能会感到更为高尚的满不在乎。无论是他的这种满不在乎,还是从海堤上经过的青年男女那种满不在乎,我都没有。想到不能和这些青年男女一起去郊游,我心里就很难过。我外祖母对社交形式很鄙视,只顾我的健康,如果她向他们提出要求,要求他们接受我作为散步的伙伴,那对我真是侮辱性的,当然我就要更难过。不论他们回到某一陌生的木头别墅去也好,手执球拍走出别墅到网球场去也好,骑马也好(那马蹄就踩在我的心上),我总是怀着热切的好奇望着他们。在海滩那叫人眼花缭乱的光照中,社会惯常的比例改变了。我在这光照中,透过让这么多光线通过的透明大玻璃海湾,注视着他们的每一个动作。但是照我外祖母看来,这海湾挡住了风,乃是一个缺点。她一想到我损失了一个小时吹海风的益处就受不了,便偷偷打开一扇窗。忽地一下,不仅菜单吹跑了,所有正在用午餐的人的报纸、面纱和遮阳帽也都吹跑了。可外祖母自己,有这天堂好风的支持,在一片责骂声中,依然像布朗迪娜女圣徒在公元177年受到严刑拷打,要她放弃自己的信仰。她始终镇定从容,回答:“我是基督徒。我们的人中间没有犯过任何罪行。”一样镇定,面带笑容。这些责骂使那些瞧不起人、头发给吹乱、怒气冲冲的游客团结一致来对付我们,更增加了我孤独悲哀的印象。
\par 这些游客的相当一部分,由法国这一地区主要省份的杰出人士组成,卡昂法院的主审官啊,瑟堡的首席律师啊,芒市的一位重要公证人啊之类。在那些地方,他们终年成散兵或者像国际象棋中的棋子一样分散着,每到度假时,便从各个点上来到这个旅馆里集合。巴尔贝克这些豪华旅馆的人口,平时一般是富有而且是国际性的,现在又赋予旅馆人口以一种相当突出的地区性了。他们在旅馆里总是保留着那几个房间,与他们那装成贵族妇女模样的妻子一起,构成一个小小的群体。巴黎的一位大律师和一位大夫也加入这一群之中。临走那天,这两位巴黎人对那些人说:
\par “啊,真是,你们不和我们坐同一趟火车,你们真有福气,能到家吃晚饭呢!”
\par “什么?您说有福气?你们住在首都巴黎,大城市,而我住在十万人口的可怜小省城。最近人口统计是十万零二千,这倒是真的。你们有二百五十万人口,你们就要回到柏油马路的巴黎上流社会灯火辉煌的大场面中去。跟你们比,我们这算什么?”
\par 他们用巴黎卷舌“r”音说着这些话,并不含有尖酸刻薄之意,因为他们这外省的阳光似乎也能像人一样到巴黎去了。人家已经数次给卡昂的首席审判官一个上诉法院的席位——但是他们出于对自己城市的热爱,或是喜欢默默无闻,或是喜欢出人头地,或因为他们反动,或为了与别墅的邻居关系好,他们宁愿留在当地。再说,他们当中有好几位也并不立即回到他们的省城去。
\par 在大宇宙之中,巴尔贝克海湾是一个特别的小宇宙,是一篮子四季水果,各种不同的日期和相继而来的月份集之一处,排成一圈。望得见里夫贝尔的日子,是暴风雨的信号。当巴尔贝克天色已经暗下来时,还看得见里夫贝尔房顶上的阳光。不仅如此,当寒冷已征服巴尔贝克时,可以肯定在另一侧海岸上还找得到加出来的两三个月的热天。大旅社的这些常客中,假期开始得晚或持续得久的,当秋季将近,秋雨和浓雾来到时,便吩咐将他们的旅行箱装上一只船,过海到里夫贝尔或科斯特多尔去与夏季会合。
\par 巴尔贝克旅社的这一小群人以提防的神情,注视着每个新来乍到的人。所有的人都一面做出对这个人不感兴趣的样子,一面就此盘问他们的朋友——旅社侍应部领班。每年都是他——埃梅来干这一季,并且服侍他们用餐。这些人的太太,知道埃梅的妻子即将分娩,饭后每人都做一件婴儿用品,同时用她们手握的长柄眼镜对我外祖母和我指指点点,因为我们吃带煮鸡蛋的凉拌菜。这是普普通通的菜,但在阿朗松\footnote{阿朗松是这一地区的重要城市。}的上层社会里没有这么吃的。对一个别人称之为“陛下”的法国人\footnote{此处影射当时的一位有名人物。此人名叫雅克·勒波迪,其父为百万富翁,糖商。他在阿特拉斯山中购得一小块土地,便自封为撒哈拉皇帝,分封贵族称号,将一个女歌星玛格丽特·德里埃立为皇后。他们在美国时,他遵照法老的先例,要娶自己的女儿为妻,“皇后”一怒之下,用手枪将他打死。},他们显露出讥讽加蔑视的态度。这个法国人也确实自称是大洋洲中一个小岛的国王,小岛上只有几个野人居住。他和他那漂亮的情妇住在旅舍里。每当她去洗海水浴,从这里经过时,淘气的孩子们便高喊:“皇后万岁!”因为她大把大把地把五十生丁的硬币朝他们扔过去。首席审判官和首席律师甚至不愿显出看见了她的模样。他们的朋友中若是有谁注视她,他们就认为应该提醒他,说那个女人不过是个女工兼妓女出身。
\par “可是有人向我担保,说他们在奥斯唐德用的是皇家舱室呢!”
\par “那当然啦!二十法郎租的!你自己高兴的话,也可以用这个舱室。而且我确切知道,他曾经要求国王接见,可是国王叫人告诉他,国王不想结识这位木偶剧场上的君主。”
\par “啊,真的吗?真是太有意思了!有的人还真……”
\par 大概这都是真的,不过也是因为他们感到对于大部分人来说,他们只不过是上等资产阶级,他们为自己并不认识这位扔硬币很大方的国王和皇后而十分恼火。公证人,首席审判官和首席律师,在他们称之为奇装滑稽木偶的这两个人经过时,感到那样不快,提高声调表现出他们的愤怒。他们的朋友、旅社侍应部领班对此十分理解。对这两位慷慨大方更甚于货真价实的君主,他一面不得不作出笑脸,可是在记下他们点的菜时,又远远地向他的老主顾会意地挤挤眼睛。有一个他们称为“漂亮先生”的服饰华丽、装腔作势的年轻人,是一个大工业家的儿子,身患肺病,且挥金如土。他每天换一件新礼服,扣眼上插着一朵兰花,午餐时喝香槟酒。然后,面色苍白,毫无表情,唇上挂着冷漠的微笑,到赌场的水晶玻璃赌台上去扔下很大的赌注。人家错误地认为他们这些人不如那个小伙子“帅”,他们也无法解释说他们就比他“帅”。可能也有点由于这种恼火,公证人对首席审判官说“他根本输不起这么大的数目”,首席审判官的老婆则“根据可靠消息来源”,说什么这个“世纪末”小伙子叫他的父母愁煞。
\par 另一方面,首席律师及其朋友们又对一位富有而又有贵族称号的老妇人极尽讽刺挖苦之能事,因为她到任何地方去都要把自己的整个日常生活原封不动地带着走。每次公证人的妻子和首席审判官的妻子在餐厅里吃饭看见她的时候,都用长柄眼镜狂妄地审视她,那种仔细和怀疑的劲头,似乎她是一盘菜。这盘菜名称古怪、外表可疑,经过系统观察,结果是予以否定,作出拒之于千里之外的姿态和恶心的怪相,叫人把那盘菜端走。
\par 无疑,她们做出这种样子,无非是要表现出:如果说有些东西她们没有的话,诸如这位老妇人的某些特权,与她有关系之类,并非她们不能有,而是她们不愿有。久而久之,连她们自己也对此深信不疑,于是就成了对于自己不了解的生活方式没有任何欲望,没有任何好奇心,对讨好新认识的人不抱任何希望。在这些女人身上,这一切都为佯作轻慢、故作快乐所代替。这有一个弊病,就是叫她们在满意的幌子之下故作不快,而且经常不断地自己骗自己,这两条便足以使她们倒霉了。不过,大概这旅社里所有的人的做法都与她们相同,只不过形式不同罢了。这样,不是出于自尊心的话,至少也是出于某些教育原则或思考习惯,便牺牲了参与完全陌生的生活那种其味无穷的妙处。显然,老妇人与外界隔绝、自己生活其中的微型宇宙,并未因气急败坏冷嘲热讽的公证人老婆与首席审判官老婆那一伙人的尖酸刻薄而受到毒化。相反,这个小宇宙散发着高雅而又有点老气横秋的芬芳,这种香气也同样矫揉造作。因为归根结底,老妇人如果能引来并维系住(为此,她本人也要不断更新)新认识的人神秘的好感,她肯定会从中体会到无穷的乐趣。而现在她只是跟她自己那个小宇宙的人来往,总是想着这个小宇宙是大宇宙之精华,对他人的轻蔑也不大知晓,简直可以忽略不计。这样生活虽然令人愉快,却没有上述那种无穷的乐趣。可能她感到,如果她默默无闻地来到巴尔贝克大旅社,穿着她那黑毛料长裙,戴着她那过时的便帽,她一定会使哪位花天酒地的公子哥或者哪位要人发出一阵冷笑的。公子哥可能一面摇摇摆摆跳着舞,一面从牙缝里挤出“穷酸老婆子!”几个字来。要人,像首席审判官一样,在一圈花白连鬓胡子中保持住了红润的面孔和她喜欢的聪明智慧的眼睛,他那一双长柄眼镜的镜片一向眼睛靠近,就表示这奇人怪物出现了。人们知道这头一分钟是短暂的,但也令人畏惧——就像一头扎入水中一样。老妇人事先派遣一个仆人前来,将她的个性和习惯告知旅社。然后自己前来,打断经理的致意,那简短之中腼腆多于傲慢,径直走进自己的房间,说不定就是由于下意识地惧怕这一分钟。房间里,自用的窗帘代替了原来挂在窗上的窗帘、屏风、照片等等,在她与她本应适应的外界之间安置了她自己的生活习惯这扇隔栅,安置得那样好,以至可以说,这不是她本人在旅行,而是她的家在旅行。她依然待在自己家里。
\par 在以她为一方,旅社人员及供应商人为一方之间,她安排下自己的仆人。此后便是她的仆人代她与这里的新人类进行接触,同时在女主人周围维持着惯常的气氛。在她与洗海水浴的人之间,她也道出自己的成见,而不顾忌会得罪一些人,这些人是她的女友根本不肯接待的。通过与女友的通讯,通过回忆,通过内心意识到自己有地位,举止得体,礼节周到,她继续生活在自己的世界里。每天,她下楼乘坐敞篷四轮马车去散步时,贴身女仆带着她的衣物尾随其后,小厮在前,有如在使馆门口值勤的哨兵。在挂着自己所属国家国旗的使馆门前,哨兵置身于异国土地上,为使馆确保其特有的治外法权。
\par 我们抵达那天,老妇人下午没有离开她的房间,我们在餐厅中没有望见她的影子。因为我们新来乍到,开午饭时,旅社经理将我们置于他保护之下,送我们到餐厅去,就像一个军官将新兵带到下士裁缝那里让人给他们发军装一样。不过,过了一小会,我们在餐厅里见到了一位乡绅德·斯代马里亚先生及其女儿德·斯代马里亚小姐,他们属布列塔尼一个默默无闻而又非常古老的世家。经理以为他们晚上才会回来,把他们的桌子给了我们。他们父女就是为了会见居住在这附近的、他们认识的城堡主人而来到巴尔贝克的。除了接受外面的邀请和回访之外,他们在旅社餐厅中度过的时间只限于绝对必需的范围内。狂妄使他们对于坐在他们周围的陌生人没有丝毫近乎人情的好感,没有丝毫兴趣。置身于这些人之中,德·斯代马里亚先生始终保持着冷若冰霜、急如星火、拒人于千里之外、粗暴、脾气很大、心怀恶意的表情。在火车的便餐厅里,置身于从不相识、也不会再次相见的旅客之间,与这些人的关系,除了保卫自己的冷烤鸡和车厢的这一角不受他们侵犯之外,就想不出还有什么别的关系,人的表情就是这样的。
\par 我们刚开始用午餐,就有人来按照德·斯代马里亚先生的吩咐叫我们起身。这位先生刚刚来到,对我们没有丝毫致歉的表示,高声请旅社侍应部领班注意,再不要发生类似的错误,他“不认识的人”占了他的桌子,他觉得很不愉快。
\par 某一个女演员(她因衣着华丽、才思敏捷、有成套的德国瓷器而著名,远远胜过她在奥代翁剧院扮的几个角色)及她的情夫(一个极为富有的年轻人,为了他,她才培养自己的情趣),还有两个在贵族阶层中非常出头露面的男士,他们四个人在生活上自成一伙,非一起出门不可,在巴尔贝克用午饭很晚,所有的人都用完饭他们才来,终日在他们的客厅中玩牌。促使他们这样做的情感中,自然是没有任何恶意的,只不过是他们对于某些幽默的谈话方式的趣味,对某些佳肴美馔的精细口味要求如此罢了。这种趣味和口味使他们从非一起生活、一起吃饭不可之中得到乐趣,如果和不得其中之韵味的一些人共同生活,他们就会受不了。甚至面对着已经上菜的桌子或一张赌桌,他们中的每个人还需要知道,坐在自己对面的客人或搭档头脑中某些知识和在任何事情上他们区别善恶的共同标准是否悬而不用了。许多巴黎人的住宅都用一个所谓真正的“中世纪”或“文艺复兴”时期的蹩脚货装饰着,某些知识使人能够辨别出真伪来。大概在这种时刻,这伙朋友希望到处都沉浸其中的那种特殊生活,就只能通过默默吃饭或打牌当中发出的难得而又滑稽的感叹或者年轻女演员为午饭或玩扑克而穿的迷人的新裙子来表现了。这种生活用他们了解透彻的习惯将他们包围住,也就足以使他们不为周围生活的秘密所侵害。漫长的下午,他们面前的大海,只不过像挂在有钱光棍小客厅墙上的一幅色彩柔和的油画罢了。一个玩牌的人,在出牌的间歇无事可干,才抬起眼睛朝大海望上一眼,看看是否有什么标志着天气晴朗或者几点钟了,并且提醒其他人该吃下午的点心了。晚上他们不在旅馆用晚餐。在旅馆里,电源使餐厅光芒四射,餐厅似乎变成了偌大的美妙的养鱼缸。巴尔贝克的工人、渔民以及小市民的家庭,躲在暗处。你看不见他们,他们却在这养鱼缸的玻璃四壁前拥挤着,想要远远看看这些人在金光摇曳中的奢侈生活。对贫穷的人来说,这些人的生活确与奇异的鱼类和软体动物的生活一样不可思议(玻璃壁是否永远能够保护住绝妙动物的盛筵,夜间贪婪凝望的默默无闻的人是否就不会到养鱼缸里来把这珍奇动物掠走并且将其吃掉,这是一个很重大的社会问题)。在这驻足凝视、黑夜里看不清楚的人群里,说不定有个什么作家,什么人类鱼类学爱好者,他们注视着雌性老魔鬼张开颌骨咬住一块食物又闭上的情景,便按照品种、生性以及后天获得的特性来对这些老魔鬼加以分类以自娱呢!一个塞尔维亚老太婆,口腔的延伸部分和一条大海鱼一样,因为她自童年时代起便生活在圣日耳曼区的淡水里。正是这后天获得的特性使她吃起凉拌菜来,犹如一个拉罗什富科家族中人。\footnote{拉罗什富科家族为法国一古老贵族家庭。}
\par 此刻,人们远远望见那三个身穿无尾常礼服的男子正在等待那位姗姗来迟的女戏子。过了一会,那女人穿着常换常新的长裙和按照她情夫特殊趣味选定的围巾,从她居住的那一层叫了电梯,像从玩具盒子里出来一样走了出来。这四个人觉得豪华大厦这种国际怪物移植到巴尔贝克以后,使奢侈之花盛开,远远胜过高级烹调。他们钻进一辆车,到半里\footnote{法古里。}以外的一家著名小饭馆吃晚饭去了。到了这家小饭馆,他们就食谱编排和烹调技术问题,与厨师进行了无尽无休的讨论。从巴尔贝克出去是一条两旁都是苹果树的路,在漆黑的夜色中,这条路与他们巴黎家中到英国咖啡馆\footnote{这家饭馆因英国人常去而得到这个名字,当时很有名。巴尔扎克笔下,拉斯蒂涅曾在这里用餐。左拉笔下,娜娜也在这里吃过饭。该饭馆位于意大利人街与马里沃街相交处。}或银楼之间相差无几,这段路程对他们来说无非是必须穿过的距离而已。他们抵达漂亮的小饭馆以后,富有的年轻人的朋友们对他有衣着如此华丽的情妇艳羡不已。那女人的围巾在小团体面前展开,有如熏香而轻柔的面纱。但是这围巾也将小团体与外界隔绝开来。
\par 可叹,为了安静休息,我根本无法像这些人那样行事。我关心着旅社房客之中的许多人。有一个男子,额头凹陷,目光在其成见与所受教育之间游移不定,他是本地的大财主,我真希望这个人对我不要视而不见。他不是别人,正是勒格朗丹的姐夫:他有时到巴尔贝克来出访,每个星期天,他妻子和他举办每周一次的花园晚会,常常使旅馆的房客减少一部分,因为这其中常有一两位应邀参加这些节庆活动。其他人为了不要显出自己没有受到邀请的模样,便挑选这一天到远处去郊游。第一天,旅馆对他接待很冷淡,因为他刚从天蓝海滨法国南方地中海海滨从马赛到尼斯一段,景色绝佳,人称“天蓝海滨”。下船来,这里的工作人员还不知道他是谁。他不仅未着白法兰绒衣裤,而且对豪华大厦的生活完全无知,依然按照法国老规矩,走进大厅,看见那里有几位女士时,一进门便脱下了帽子。这一动作使得经理回答他的问话时,甚至没碰自己的帽檐一下,认为他大概是个出身最寒微的人,也就是经理自己称之为“老百姓出身”的人。唯有公证人的妻子感到自己受到这个新来人的吸引,认为他散发出有身份的人佯装俗气的味道。她宣称在他面前,人们感到对方是一位很出类拔萃的人,极有教养,而且在所有在巴尔贝克遇到的人当中,他如鹤立鸡群。她认为,只要她本人不能与他经常来往,那他就是不能与之经常来往的人。说这些话时,用的是对芒市的最上等阶层了如指掌、辨别能力万无一失、对其权威无可辩驳的人的口气。她对勒格朗丹的姐夫作出这样有利的评断,可能是因为此人外表极为平淡,没有任何借势吓人的地方,也可能是因为她从这个举止有如虔诚教徒的乡绅身上认出了自己那一教派——共济会——的征象。
\par 我已经得知——又有什么用!每天在旅馆门前骑马的几个小伙子,他们的父亲是一个新产品商店的老板,满肚子鬼主意。我的父亲永远不会同意与这些人结交。“洗海水浴的生活”使他们长成了大个头,在我眼中,简直是半人半神的骑士雕像。我抱的最大希望,就是他们永远不要将他们的目光停驻在我这个可怜的小男孩身上,这个就是为了到沙滩上去坐坐才离开旅馆餐厅的小男孩。我甚至希望得到曾是大洋洲某荒岛之王的那个冒险家和患肺病的小伙子的好感。我爱设想那个患肺病的小伙子在他那狂妄的外表下掩盖着一颗胆小怕事而又温柔的心,说不定对我一个人能慷慨赠予深情之珍宝。何况(与人们惯常对于旅途中之新交所说的情形相反),看见你跟某些人在一起,在有时再去的海滩上,会在真正的社交生活中给你增加一项无比的系数,在这里,也就只有洗海水浴的友情了。人们对友情倒也不是敬而远之,在巴黎生活中,人们还细心培植它呢!所有这些瞬时的或地方性的名人,他们会对我有什么看法,我很在意。我那爱为人设身处地、重现他们的思想状况的秉性,使我不仅把他们放在他们自己真正的地位上,把他们放在假如在巴黎他们会占据的地位上——那地位大概很低——而且还把他们放在他们自己认为应该处于的地位上。说老实话,在巴尔贝克,他们就是把自己放在了自认为应处的地位上。由于这里缺乏共同的尺度,便赋予他们某种相对的优越感和某种莫名其妙的趣味。可叹,所有这些人的轻蔑,没有一个比德·斯特马里亚先生的轻蔑那样叫我难受。
\par 他的女儿一走进来,我就注意了。我注意到她那苍白而又几乎蓝莹莹的美丽面庞,注意到她那高高的个儿,她与众不同的举止,令我不无道理地忆起她的遗传、她所受的贵族教育的地方,尤其是我知道她的名字,这一切就更加清楚,正像天才音乐家所发现的那些具有表现力的题材,将闪烁的火光、江河的声响和田野的宁静为听众描绘得那样精彩一样。听众如果事先浏览过乐谱,更是早就将自己的想象力引导到了恰当的道路上。“种”,又给德·斯特马里亚小姐的风韵加上了其缘由的概念,使其风韵更可理喻,更加完美。这也使其风韵更加撩人欲望,因为这等于宣布她是可望而不可即的,正像一件物品很叫我们喜欢,而价格昂贵就更增加了它的价值一般。这精选的上等津液组成了面庞,遗传的茎杆又赋予它海外珍果或著名海鲜的香味。
\par 一个偶然事件骤然间给我外祖母和我送来了合适的手段,使我们在大旅社的所有房客眼中,威信立即提高。确实,就在那头一天,那位老妇人从自己家中下得楼来。前有小厮开路,后有贴身女仆小跑跟随,手中拿着忘下的一本书和一条毯子。靠着这些,对人的心灵产生了影响,在所有人心中激起了好奇和崇敬。看得出来,德·斯特马里亚先生比任何人都更无法摆脱这种好奇和崇敬。就在这时,旅馆经理向我外祖母弯下身来,出于客气(就像将波斯国王或拉娜瓦洛娜王后指拉娜瓦洛娜三世(1862—1917),她1883—1897年曾为马达加斯加王后,后被流放到留尼汪及阿尔及利亚。指给一个默默无闻的看热闹的人看一样。显然这个看客不可能与那权势炙手可热的君王有任何关系,但也会觉得曾在几步开外的地方见过他很有意思),向她耳边溜出一句:“德·维尔巴里西斯侯爵夫人。”就在此刻,这位老妇人远远望见了我的外祖母,情不自禁地射出惊喜交加的目光。
\par 在这人生地不熟的地方,对于要接近德·斯特马里亚小姐而无可求助的我,最有魔力的仙女以一个小老太太的形象突然出现,还有什么会比这个更能使我心花怒放,诸位可以想见。实际上,我再也听不见任何人讲话的声音。从美学观点来说,人的数量极其有限,不论到哪里去,都经常会体验到见到熟人的快乐,即使不像斯万那样到前辈大师的画面中去寻找也会遇到。就这样,我们到巴尔贝克小住的头几天,我就遇到勒格朗丹,斯万的门房和斯万太太本人。勒格朗丹成了咖啡店的侍者;斯万的门房成了过路的陌生人,我没有再见过他;斯万太太则成了游泳教练。对于相貌和思想方法上具有某些特点的人,似乎有一种磁现象,将他们彼此吸引到一起,紧紧抓住分不开,以至于大自然这样将一个人引进一个新的机体时,并不会使这个人受到过分的损伤。勒格朗丹变成了咖啡店侍者,但是他的个头、他鼻子的侧影和下巴的一部分都保持完好。斯万太太变成了男性,加上游泳教练的身份,不仅仅她平时的长相跟随着她,甚至某种说话的模样也跟随着她。只是她现在系着红腰带,海上稍有长浪涌来,她便举起小旗,禁止游泳(游泳教练都小心翼翼,难得有人会游泳),对我已经用处不大,正像从前斯万在《摩西生平》那幅壁画中从叶忒罗的女儿的面庞中认出了她,\footnote{见《斯万之恋》中描述的情节:斯万发现奥黛特与波堤切利《摩西壁画》中叶忒罗的女儿西坡拉相像,因而越发觉得奥黛特美丽非凡。}也不可能有什么用处一样。这位德·维尔巴里西斯夫人可是货真价实的,她并没有受到魔法的折磨,魔法一施可就夺去了她的权势。相反,她能够将一种魔法交给我的权势使用,使这权势顿时增加百倍。多亏有了这个,我就像有神鸟的翅膀托着一样,很快穿越了将我与德·斯特马里亚女儿隔开的无限远的社会地位的距离——至少在巴尔贝克是如此。
\par 可惜,如果说这世界上有谁比任何人都更离群索居的话,那就是我的外祖母了。她知道我对舆论看得很重,如果我对哪一个人、哪些人有兴趣,她却不会因此看不起我,也不会理解我。而这些人,她根本就没有注意到他们的存在,她大概一直到离开巴尔贝克也没有记住他们的名字。我不敢向她招认,如果这些人看见她与德·维尔巴里西斯夫人说话,我会非常高兴,因为我感到侯爵夫人在旅馆中很有威信,而且她的友情能在德·斯特马里亚先生眼中提高我们的地位。再说,我外祖母的这位女友在我心目中也根本不代表贵族中的一员:我的思想还没有停驻在她的姓上面时,这个姓氏在我耳边就已那么熟悉,我已经司空见惯了。我还是孩童时,就常听见家里人提起这个姓。她的贵族头衔也只不过在姓氏上加上了一个莫名其妙的特殊玩艺而已,就像一个不常见的名字一样。街名也是如此。在拜伦爵士街\footnote{拜伦爵士街位于巴黎第三区,于这位英国诗人逝世的次年1825年命名。},那么大众化、那么俗气的罗什舒阿街\footnote{以蒙马特尔修道院女院长(1717—1727)玛格丽特·德·罗什舒阿的名字命名,位于巴黎第九区。直到十八世纪时,该区还有许多下等酒馆。到普氏在世时,此区内有了布雷耶尔音乐厅及罗什舒阿通俗戏院(1910年成为现代剧院)。},或在格拉蒙街,格拉蒙街位于巴黎第二区。此处原有格拉蒙家族之大公馆,十八世纪末以此命名街道。发现不了任何比莱翁思雷诺街\footnote{莱翁思雷诺街于1884年命名,位于巴黎第十六区。莱翁思雷诺本为工程师,领导海岸灯塔事宜,著有关于法兰西海岸照明之论文。}或希波里特勒巴街\footnote{希波里特勒巴街于1861年命名,位于巴黎第九区。希波里特勒巴为本区内洛莱特圣母院之建筑师。}更高尚的东西。德·维尔巴里西斯夫人也好,她的表兄麦克马洪也好,并不使我想到一个什么特殊世界的人。对麦克马洪\footnote{麦克马洪,1873—1879年曾任总统。}和也是共和国总统的卡尔诺\footnote{卡尔诺,1837年生,1894年被无政府主义者卡兹里奥在里昂暗杀。}以及拉斯巴耶\footnote{拉斯巴耶(1794—1878),政治家、医生、记者,参加了1830年和1848年革命。},我也不加区分。弗朗索瓦丝买过拉斯巴耶和教皇庇护十一世的照片。
\par 我的外祖母有一个原则,那就是:出门在外,不应该再有什么交往,上海滨不是为了去看望人的,要做这种事在巴黎多少时间都有;这宝贵的时间应该全部在露天,面对海浪来度过,而礼尚往来、客气俗套会使你浪费宝贵的时间。她还以为所有的人都同意她的这个观点,她下令,老朋友在同一旅馆中巧遇,要演一出相互隐姓埋名的戏。她觉得这样更方便一些。听到旅馆经理提到那个姓氏,外祖母只是扭过头去,作出似乎没有看见德·维尔巴里西斯夫人的样子。德·维尔巴里西斯夫人明白我的外祖母并不一定要相认,于是自己也漫无目标地望去。她走远了。我孤独地留在那里,好似一个落水者,一艘船只似乎靠近了他,但是,接着,并没有停下便消逝了。
\par 德·维尔巴里西斯夫人也在这个餐厅中用餐,不过是在另一头。住在旅馆里的人或者来这里拜访的人,她一个也不认识,甚至不认识德·康布尔梅先生。有一天德·康布尔梅先生和妻子接受邀请与首席律师共进午餐,果然我看到他并未向那位老妇人打招呼。首席律师与这位绅士同桌进餐,觉得十分光彩,喜不自禁。他回避往日的朋友,只远远向他们挤挤眼睛,以便(还算是不加声张地)暗示这一历史性重大事件,为的是不要让人理解为这是敦请他们前来。
\par “喂,我想您混得不错,成了个时髦人物啦!”当天晚上首席审判官的老婆对他说。
\par “时髦?为什么?”首席律师问道,故作惊讶地掩饰自己的喜悦,“是因为我请的客人吗?”感到自己再装不下去了,他这样说道,“可是有几位朋友共进午餐,有什么可时髦的呢?他们反正得在哪儿吃饭呀!”
\par “就是,就是时髦!他们就是德·康布尔梅夫妇\footnote{“德”是加在贵族爵位上的一个标记,一般应说“德·康布尔梅侯爵”,不应与爵衔分开,只加“德”字。首席审判官老婆如此说话,表明她对上流社会很不熟悉。}吧,是不是?我确实认出来了。那是一位侯爵夫人。而且是货真价实的。并不是通过娶妻得到头衔的。”
\par “嗨,她是很朴实的一位女子,非常可爱,一点没有客套。我以为你们会来,我直跟你们打招呼……你们来了,我不就给你们介绍了!”他用轻微的讥讽口吻使这个提议的重要性稍微减弱一些,就像阿絮埃吕斯对爱丝苔尔说“要不要把我这列国给你一半?”\footnote{见拉辛名剧《爱丝苔尔》第二幕第七场。}一样。
\par “不,不,不,不,我们还是躲起来,像平平常常的紫罗兰一样的好。”
\par “我再跟你们说一遍,你们不该那样,”首席律师回答道,反正危险已经过去,他胆子壮起来了,“他们还会把你们吃了!咱们玩牌吧?”
\par “太好了,我们都不敢跟您提这个了,你们现在请侯爵夫人吃饭了!”
\par “噢,算了吧,这些人毫无不同寻常之处。喂,我明天晚上要去跟她们吃饭。你愿意不愿意替我去?我这么说是真心诚意的。说老实话,我也一样喜欢待在这里。”
\par “不,不,不!……那人家要把我当反动分子撤职了!”首席审判官大叫大嚷道,因为自己开的这个玩笑笑得眼泪都出来了,“您也一样,人家在菲特尔纳接待您。”他扭过身对公证人说话,加上这么一句。
\par “噢!我每个礼拜天去,一个门进,另一个门出。但是他们可不像在首席律师家那样在我家吃饭。”
\par 德·斯特马里亚先生那一天不在巴尔贝克,真叫首席律师遗憾。但是他很狡诈地对饭店侍应部领班说:
\par “埃梅,你可以告诉德·斯特马里亚先生,他并不是在这间餐厅里吃饭的唯一贵族。今天中午与我一起用午饭的那位先生,你可看见?嗯?小胡子,军人模样?对,那就是德·康布尔梅侯爵!”
\par “真的吗?怪不得呢!”
\par “这应该向他表明,他并不是唯一有贵族头衔的人。捉弄捉弄他好了!煞一煞这些贵族的威风,不是坏事。埃梅,你知道吗,我说的这些话,请你一点也别告诉他。这倒不是为我自己。再说,这些他全知道得一清二楚。”
\par 第二天,德·斯特马里亚先生知道了首席律师为他的一个朋友辩护的事,亲自出马自报家门。
\par “咱们共同的朋友德·康布尔梅夫妇本来正是打算让咱们在一起聚聚的,不巧咱们安排的日程凑不到一块,总之,我也不知道是怎么一回事。”首席律师说道,像所有撒谎的人一样,自以为人家是不会设法弄清某一个无足轻重的细节的。实际上某个细节便足以(如果碰巧你掌握了朴素的事实真相,那真相与这细节相互矛盾)揭示某人的性格,并叫人永远对你存有戒心。
\par 我像往常一样望着德·斯特马里亚小姐。她父亲走开去与首席律师谈话时,就更方便。她的仪态显得异常放肆,又始终特别优美。例如,她双臂支在桌上,将酒杯举到前臂之上,目光冷淡,很快就无精打采,固有的、家传的生硬,她的声音中个人的抑扬顿挫掩盖不住这种冷淡和生硬,从口气里人们可以感觉到这些东西。这使我的外祖母非常不快。那是返祖遗传的傲慢,每当通过某个眼神或某种声调她表达完了自己的思想之后,就要回到那种傲慢的表情上去。这一切必须使注视她的人想到她的家系上去,是这个家系将这种缺乏人情味、缺乏敏锐感受和缺少宽大胸怀传给了她。有时她的目光从眼珠那飞快干涸的背景上瞬息闪过,从这目光中可以感到几乎谦恭的温柔,那是感官享乐占主导地位的滋味赋予世界上最骄傲的女子的温柔。这女子转眼间就只承认一种威望,那就是任何可以使她体会到这些感官享乐滋味的人在她面前的威望,哪怕是一个喜剧演员或者江湖艺人。为了他,说不定她会离开自己的丈夫一整天。有时她的面色现出肉感而且鲜艳的玫瑰色,这玫瑰在她那苍白的双颊上盛开,那面色犹如将肉红色加进了维福纳河中白色睡莲的花蕊。从某些这样的目光和这样的面色中,我似乎感觉到,她说不定会轻易应允,让我前来在她身上寻找她在布列塔尼过的那么富有诗意的生活的味道。也许是太司空见惯了,也许天生与众不同,也许厌恶自家的贫穷或吝啬,她似乎并未给这种生活找到很大的价值,不过,在她的身上就暗暗包含着这种生活。
\par 遗传给她的意志力,储备量甚微,赋予她的表情某种懦弱,大概她从那微量的储备中找不到抵抗力量的源泉。她每次用餐都戴一顶灰色呢帽,从不变样,帽上插着一根已有些过时却又自命不凡的羽毛。在我眼中,这顶呢帽使她变得更加温柔,并不是因为这帽子与她那银白和粉红的面色十分相谐,而是因为这顶帽子使我设想她很贫穷,这就使她与我更加接近。父亲在场,她必须取一种合乎习俗的态度,但是对于她面前的人有何感受,如何对这些人进行分类,她已经有了与其父亲不同的原则。说不定她在我身上并没有注意到地位不够,而是注意到了性别和年龄。如果哪一天德·斯特马里亚先生单独出门,不带着她,特别是如果德·维尔巴里西斯夫人走来坐在我们的餐桌上,使她对我们产生一个概念,我可能会壮起胆子去接近她,说不定我们就能交谈几句,约会几面,关系更紧密了。如果有一个月,她父母不在,她一个人就留在那富有浪漫情调的古堡中了。黄昏时节,在海浪汩汩敲击的橡树下,在那色泽暗淡下去的水面上,欧石南粉红的花朵发出更柔和的闪光,说不定那时我们两人就能单独散步了。我们会一起足迹踏遍这个岛屿。对我来说,这小岛充满了魅力,因为它隐藏着德·特斯马里亚小姐的日常生活,因为它安眠在她双眼的回忆中。当我穿过这些地点,这些地点以那么多的往事包围着她,我似乎感到只有在这里,我才真正地拥有她。这些往日的回忆如一层面纱,我的欲火真想将它掀开。还有大自然在女性与某些人之间投下的回忆(怀着同样的意图,大自然对所有的人,在他们与最强烈的快感之间,放上传宗接代的行为;对昆虫,在花蜜前放上花粉,好让昆虫将花粉带走),以便他们受到这样更能完全占有她的幻觉欺骗之后,不得不首先占有自然景色,她就在这景色之中生活。比起肉欲的快感来,这景色对他们的想象更有用。但是如果没有这种肉欲的快感,这景色是不足以吸引他们的。
\par 可是这时我必须将视线从德·斯特马里亚小姐身上移开了,因为她父亲已向首席律师告辞,并且回来坐在她的对面,搓着双手,好像一个人刚刚得了什么宝物一样。他大概认为结识一位重要人物是一件奇怪而简短的举动,这举动本身就已足够;为了扩展这一举动所包含的全部意义,握一握手,注视一下也就够了,并不需要立即交谈,也不需要事后有什么交往的。至于首席律师嘛,这次会见那初次的激动一过去,他就像平日人们有时听见他谈话那样,对旅馆侍应部领班开了腔:
\par “埃梅,我可不是国王;你去国王身旁服侍吧……喂,这头一道菜小鳟鱼,看上去很好吃,咱们再向埃梅要点。埃梅,你们做的这小鱼,我看完全可以再叫几盘。你再给我们送点来,埃梅,悄悄地。”
\par 他不时反复叫着埃梅的名字,这就使得他请什么人吃饭时,他的客人会对他说:“我看出来,你在这里完全和在家里一样嘛!”从这种想法出发,客人觉得也应该嘴里不断地叫着“埃梅”,这里面既有胆怯,又有俗气,又有愚蠢。某些人认为,一字不差地模仿跟他们在一起的人,是既聪明又漂亮的事,这些人就是又胆怯,又俗气,又愚蠢。他不断地重复这名字,但是面带笑容,因为他既要将他与旅馆侍应部领班的良好关系展现在人们面前,又要将自己高于他的那种优越感表现出来。旅馆侍应部领班也一样,每次他的名字又出来的时候,他都既感动又骄傲地微笑着,表明他既感到受抬举,又完全明白那是开玩笑。
\par 大旅社这间宽大的餐厅,一般是座无虚席的。对我来说,在这里用饭总是很吓人的事。当旅社的业主(或者是合伙人公司选出的总经理,我不太清楚)来到待上几日时,这种情形尤甚。此人并非这一家豪华旅馆的业主,而是七八家旅馆的主人。这些旅馆遍布法国各地,他就在这些旅馆之间往来穿梭,在每一处不时待上一个星期。这时,几乎就在晚餐开始时,每天晚上在餐厅入口处,这个小老头儿就会出现,白头发,红鼻子,不动声色,衣冠整齐,不同寻常。据说,无论是在伦敦,还是在蒙特卡洛,他都以欧洲最大的旅馆主之一而赫赫有名。
\par 有一次,晚餐开始时我出去了一会,回来时从他面前经过。他向我施礼,显然是为了表明我是他的顾客,但是十分冷淡。我无法辨清这种冷淡的原因,是一个人忘不了自己的身份,而表现出的矜持,抑或是对一个无足轻重的顾客的蔑视。反过来,面对那些十分重要的客人,总经理鞠躬时亦同样冷淡,但是腰弯得更深一些,毕恭毕敬,垂下眼皮,好像在葬礼上站在死者父亲面前或圣体面前一样。除了这种冷淡而又难得的敬礼之外,他一动不动,似乎为了表明他那前突而又熠熠闪光的双眼什么都看得见,什么问题都能解决,在“大旅社的晚餐”中,既保证各种细处完美,又保证总体和谐。显然他感到自己比导演高明,比乐队指挥高明,是真正的大元帅。他认为,将凝视提高到最高程度,就足以保证一切就绪,犯下的任何过失也不会导致完全溃败。为了负起自己的责任来,他不仅仅不作任何手势,甚至眼睛也不眨一眨。由于注意力集中,那眼睛几乎都化成了化石。可这眼睛对全部行动一览无余,而且指导着全部行动。我感到甚至我那羹匙的动作都逃不过他的眼睛。一喝完汤,他就溜之大吉了。可是他刚才的检阅,叫我整个晚餐过程都没有胃口。
\par 他的胃口倒极佳,因为他像一个普通人一样,与所有的人同时在餐厅中用午餐。大家都看得出来,他那餐桌只有一点特殊,那就是在他吃饭过程中,另一位经理,平常的那位,一直站在他身旁与他谈话。因为这位经理是总经理的下级,他极力拍总经理的马屁,而且对总经理怕得要命。吃午饭时我的恐惧有所减少,因为总经理这时消失在顾客之中,极力不引人注目,如同一位将军坐在一家饭馆里,饭馆中也有士兵,他要显出不管他们的模样。尽管如此,穿制服的仆役环绕四周,门房向我宣布“他明天早晨走,到迪纳尔去。从那,他到比亚里茨去,然后到戛纳去”时我总算呼吸更自由一些了。
\par 我在旅馆中没有什么交往,而弗朗索瓦丝结交了许多熟人,这就使我在这里的生活不仅很凄凉,而且很不舒服。看上去,似乎她结交的人应该使我们办事方便。实际则正相反。虽然那些无产者很难叫弗朗索瓦丝把他们当熟人待,只有在极为彬彬有礼待她的某些条件下,才能达到这个目的。反过来,他们一旦达到这种地位,那弗朗索瓦丝心中就只有他们了。她的老经验已经教她明白了,对她主人的朋友,可以丝毫不受约束。如果她有要紧的事,就可以把一位前来看望我外祖母的太太打发走。但是对她自己的熟人,就是说那些难得为她那难得的友情所接纳的平民百姓,她的行为可是遵照最细致周到、最绝对的外交礼仪的。
\par 弗朗索瓦丝认识了主管饮料的掌班,认识了一个小小的贴身女仆,她是给一位比利时太太做长裙的。弗朗索瓦丝认识他们以后,午饭后再也不马上上楼为我外祖母准备各种器物,而是在一小时之后,因为主管饮料的掌班要给她弄咖啡或者药茶喝,那个贴身女仆要她去看自己怎样做衣裳。而拒绝他们是不可能的,是属于不可为之事之列。此外,她对那个小贴身女仆特别关心。那人是一个孤儿,几个陌生人将她养大,她就要到那些人家里去过几天。这种情形激起弗朗索瓦丝的怜悯之情,也激起她那善意的蔑视。她自己有家庭,从父母那里继承了一所小房子,她的兄弟在那里养了几头乳牛。她不能将一个无家可归的人视为她的同类。这个小姑娘希望八月十五\footnote{8月15日西方为圣母升天节。}时去看望她的恩人。弗朗索瓦丝情不自禁地反复叨念着:“她真叫我好笑。她说:‘我希望八月十五回家去。’她说‘家’!那根本不是她的老家,而是收养她的人,可她还说‘家’,好像真是她的家似的。可怜的小姑娘!她真穷得可以,都不知道什么叫有个自己的家了。”
\par 弗朗索瓦丝与顾客带来的一些贴身女仆要好,这些人跟她一起在“邮件处”用晚饭。她们看见她那漂亮的花边便帽和苗条的体态,把她当做是一位太太,说不定是贵族太太,因境况不佳或者对我外祖母非常依恋而来给她当个伴。如果弗朗索瓦丝只与这些人要好,一言以蔽之,如果她只与不是旅馆的人要好,那害处还不大,因为她还不会妨碍旅馆的人为我们做事。其实,即使她不认识旅馆的人,这些人在任何情况下也不会对我们有什么用。可是弗朗索瓦丝也与一个饮料掌班、一个厨房里的人、一个管一层楼的女管事交上了朋友。结果是,在我们的日常起居上,弗朗索瓦丝新来乍到,还什么人都不认识时,为一点点小事,她就乱按铃叫人。有时时间不合适,我外祖母和我都不敢按铃,她却敢。我们如果为此对她稍加批评,她便回答说:“花了不少钱嘛,就得这样!”似乎那钱是她付的。而现在,自从她成了厨房里一个大人物的朋友后,我们本以为这对我们住得舒服一些是个好兆头,然而不是这样,如果外祖母或我脚冷,哪怕是正常时间,弗朗索瓦丝也不敢按铃。她说,这样会叫人产生不好的印象,因为这等于逼他们再把锅炉升起来,或者妨碍仆人吃晚饭,他们会不高兴的。最后她还要用上一个固定词组:“事实是……”虽然她自己说时也不大有把握,可是这句话的意思仍很明显,明明白白地是说我们不对。我们也不坚持,生怕她再对我们来上一个固定词组,而且更厉害得多:“有什么了不得!……”结果是:因为弗朗索瓦丝成了烧热水的人的朋友,我们反倒再也没有热水了。
\par 最后,通过我外祖母,我们也认了一个熟人,虽然她是不得已而为之。因为有一天早晨她和德·维尔巴里西斯夫人在一扇门边迎面相遇,不得不上前搭话,事先双方都作出惊讶和犹豫不决的手势,作出后退、怀疑的动作,最后又因礼节和高兴做出抗议的动作,就像莫里哀戏剧的某些场面一样:两个演员相距几步远,但是长时间各自在一边进行独白,忽然,他们你看见了我,我看见了你,最后又两人一起说起话来,对话之后就来了个合唱,两人拥抱在一起。\footnote{普氏可能想到了莫里哀《妇人学堂》的开头。}
\par 德·维尔巴里西斯夫人出于谨慎,过了一会就想离开我的外祖母。可是外祖母相反,更希望一直挽留她到午饭时刻,极力想知道她是怎么搞的,收到信件既比我们早,又能吃到上好的烤肉(德·维尔巴里西斯夫人很贪吃,她很少品尝旅馆里的饭菜。我们是在旅馆里用餐的。我的外祖母总是引用塞维尼夫人的原话,认为旅馆的饭菜是“富丽堂皇到叫人饿死”\footnote{出自塞维尼夫人1689年7月30日致其女儿函,说的是瓦纳主教的华宴。意思是菜肴极为丰盛,但是客人不敢吃,因为全是不好消化的东西。}的)。从此,侯爵夫人养成了习惯,每天在餐厅里等人家给她上菜时,便到我们身旁坐一会,而且不许我们站起身来,不许我们在任何事上为她忙碌,至多在我们吃完午饭,桌上杯盘狼藉的时刻,常常多待一会与她聊聊。
\par 我呢,为了能爱上巴尔贝克,为了保持我置身于地球尽头的想法,我竭力向更远的地方望去,只看见大海,在那里寻找波德莱尔所描写的各种效果,只有上什么大鱼的日子我的目光才低垂下来注视餐桌。这海中魔怪与刀叉相反,与原始时代是同时代之物。那个时代,生命开始在大洋之中涌流,在西梅里安这是古代的一个民族,荷马在《奥德赛》中曾经提到。普鲁斯特在《追忆似水年华》中数次提到。据说这些人生活在天涯海角,永远是黑夜。时代,鱼类那无数椎骨和蓝色、粉红色神经的躯体已经由大自然创造出来,而且是按照一种建筑蓝图,好像一座多彩的海上教堂一样。
\par 一个理发师正在毕恭毕敬地服侍一位军官。一位顾客走进来,理发师见那军官认出了顾客,并与他搭起话来,聊上一会。理发师很高兴,他明白这两位属于同一阶层,去拿肥皂碗时,禁不住微微一笑,因为他知道在他这店里,在使用洗头肥皂这粗俗的活计之上,还可加上社会上的、甚至贵族味道的快乐。埃梅也像这个理发师一样,他看到德·维尔巴里西斯夫人发现了我们是老熟人,去给我们端漱口水时,那种微笑和一位很会适时走开的家庭主妇那既自豪又谦虚又非常不引人注目的微笑一样。也可以说那是一位兴高采烈而又深受感动的父亲,他密切地注视着在他的餐桌上行了订婚礼的子女的幸福,而又不去打扰这种幸福。再说,只要听人道出一个有贵族头衔的人名,埃梅就会显得兴高采烈。这与弗朗索瓦丝正好相反,谁若是在她面前说“某某伯爵”,她的脸色没有不阴沉下来,话语没有不变得干巴巴而又简短的。但这并不说明她钟爱贵族的程度就比埃梅差。
\par 其次,弗朗索瓦丝还有一个本事,那就是她能从别人身上找出其最大的缺点来。她很为此自豪。埃梅属于令人愉快又充满善良纯朴的一类人,弗朗索瓦丝则不然。给埃梅他们讲一件多少带点尖刻味道、但在报纸上没有的、尚未发表的事情时,他们便感到非常高兴,而且形诸于色。弗朗索瓦丝可不愿露出惊异的神色。奥地利大公鲁道夫\footnote{鲁道夫(1858—1889)为奥地利国王弗朗索瓦约瑟夫一世的独生子,1889年,人们在梅耶林的猎宫中找到他与情妇玛丽亚·维茨拉的尸体,不知他们是自杀还是被暗杀。},她从来就没想过有这么个人。若是在她面前说,这位大公并没有像人们认为确有其事那样已经死掉,而是还活着,她也会回答“对”,似乎她早就知道一样。此外,还应相信,她虽然那样谦恭地称我们为主人,我们也几乎完全驯服了她,但是她出身的家庭在自己的村庄里境况富裕,地位独立,享有一定威望,这个家庭的地位一定受到这些贵族的干扰。所以,即使是从我们嘴里她听到一个贵族的姓名,她也没有不强忍怒气的。而埃梅则相反,他自孩童时代起便在贵族家中当仆役,甚至可以说他是靠慈善在这些人家长大的。
\par 因此,对弗朗索瓦丝来说,德·维尔巴里西斯夫人因自己是贵族就需要向人讨饶。至少在法国,这正是那些大老爷和贵妇人的天才之所在,也是他们唯一操心的事。有些仆人,就他们的主人与他人的关系,不断收集些只言片语,从中有时得出错误的推理——就像人对动物的生活得出错误的推理一般。弗朗索瓦丝遵循这个倾向,总是觉得人家“亏待”了我们。再说,和她对我们极度偏爱一样,她从别人使我们不快中得到快乐,这也很容易使她得到这个结论。但是,当她看到,而且决不可能看错,德·维尔巴里西斯夫人对我们和对她本人的百般殷勤照顾以后,她便原谅了这位夫人身为侯爵夫人,而且由于她不停地感谢这位夫人身为侯爵夫人,她喜欢这位夫人胜过我们认识的所有的人。这是因为我们认识的人当中,确实没有哪一个能努力做到这样持续不断地热情备至。每次我外祖母发现德·维尔巴里西斯夫人正看一本书,或者说觉得一位女友赠她的水果漂亮,一小时过后,一位贴身男仆就会上楼来将书或水果送给我们。待我们此后与她相见、向她表示感谢时,她总是作出要给她赠物找一个特殊用途以作为遁辞的模样,只是说:“那书并不是什么杰作,可是报纸到得这么晚,非得有点东西看不可。”或者说:“在海边,弄些可以放心的水果,是比较谨慎的做法。”
\par “可我觉得你们从来不吃牡蛎,”德·维尔巴里西斯夫人对我们说(更增加了我那时的厌恶印象,因为牡蛎的活肉叫我讨厌,更甚于粘乎乎的海蜇,这两样使我觉得巴尔贝克海滩黯然失色),“这一带海边,牡蛎非常鲜!啊,我要吩咐我的贴身女用人,去取我的信时将你们的信也一起取来。怎么,您的女儿每天给您写信?你们能找得出那么多话相互倾诉吗?”
\par 我的外祖母沉默不语。可以相信这是出于蔑视。她在给我妈妈的信中反复地写到塞维尼夫人那句话:“刚刚收到一封信,过一会又想再收到一封,我全靠收信才能呼吸。\footnote{此句见于塞维尼夫人1671年2月18日致女儿函,下面两句却不在此函中。}我的这种感觉,能理解的人微乎其微。”下面的结论是:“我寻求属于这少数之列的人,我回避其他人。”我真担心她会将这个结论应用在维尔巴里西斯夫人身上。她不得不转换话题,对前一天德·维尔巴里西斯夫人叫人给我们送来的水果大加赞扬。那水果也确实精美之至,旅馆经理虽因自己的水果盘深受蔑视而妒意大发,依然对我说:“我跟您一样,比起其他任何餐后小吃来,我更喜欢水果。”我的外祖母对自己的女友说,旅馆里上的水果一般都非常糟糕,因此她对这些水果就更加喜欢。
\par “我可不能像塞维尼夫人那么说,”她补充一句道,“如果我们异想天开想找一个坏水果,则不得不叫人从巴黎弄来。”\footnote{见塞维尼夫人1694年9月9日函,原话是这样的:“如果我们异想天开想找到一个坏甜瓜,可能就不得不叫人从巴黎弄来了,这里是没有的。”}
\par “啊,对,您看塞维尼夫人的《书信集》。我从头一天就看见您手里拿着她的《书信集》(她忘了,她在门边与外祖母相遇之前,在旅馆里从未见过我的外祖母)。她总是操心她的女儿,您不觉得有点过分?她谈女儿谈得太多了,不可能是真心诚意的。她写的东西不够自然。”
\par 外祖母觉得辩论毫无用处。为了避免在无法理解她之所爱的人面前谈论这些事,她干脆把手提包放在《德·博泽让夫人回忆录》上边,把那本书遮住。
\par 弗朗索瓦丝戴着一顶漂亮的便帽,旅社的全体人员对她敬重备至。她下楼“到信件处去吃饭”,她称这个时刻为“中午十二点”。德·维尔巴里西斯夫人如在这时遇到她,便拦住她打听我们的消息。弗朗索瓦丝将侯爵夫人委托的话转达给我们,她模仿着德·维尔巴里西斯夫人的嗓门说道:“她说:‘您一定向他们问好。’”她以为是逐字逐句引用那位夫人的话,可是歪曲的程度,不亚于柏拉图歪曲苏格拉底的话\footnote{柏拉图确实在其《对话录》中经常提及苏格拉底。仔细研究以后,确实苏格拉底的形象与柏拉图给我们描述的不尽符合。},或者圣约翰歪曲耶稣的话。自然弗朗索瓦丝对这种关切十分感动。外祖母担保说,德·维尔巴里西斯夫人从前姿色出众。弗朗索瓦丝可不相信,她认为外祖母出于阶级利益在信口开河,富人反正总是护着富人。确实,那出众的姿色,如今已残留无多。除非比弗朗索瓦丝更具艺术家气质,否则是无法恢复这位夫人那已经毁坏的美貌的。要理解一位老妇人当初会怎样美丽,不仅要注视她,而且要对每个线条进行研究。
\par “我得想着哪一次问问她,是不是我搞错了,她是不是与盖尔芒特家有什么亲戚关系。”外祖母对我说。这话激起我满腔怒火。这两个姓氏,一个是通过亲身体验那低矮而可耻的门进入我的心中,另一个是通过想象那金色的大门进入我的心中。说这两个姓氏之间有共同的宗室,我怎能相信?
\par 人们经常看见卢森堡亲王夫人走过,已经有好几天了。车马华丽,她本人身材高大,红棕头发,美丽非凡,只是鼻子有些过大。她在此地度假,住几个星期。她的敞篷四轮马车停在旅馆门前,一个小厮过来与旅馆经理说话,又回到马车旁,然后送来一些上好的水果(集各种水果于一个篮子之中,正如海湾本身将各个季节都汇集在一处一般),附一张卡片:“卢森堡亲王夫人”,上面用铅笔写了几个字。蓝莹莹的、闪闪发光的、滚圆的李子,跟此刻大海那么圆一样;透明的葡萄挂在枯枝上,好似明媚的秋日;天青石般的梨子。这些水果,送给哪一位隐姓埋名住在这里的王子呢?这不会是送给外祖母的女友的,亲王夫人希望来拜访她。可是第二天晚上,德·维尔巴里西斯夫人就差人给我们送来了新鲜而又金光闪闪的串串葡萄、一些李子和梨。虽然李子已变成了紫色,犹如我们进晚餐时刻的大海;虽然天青色的梨子上,已漂着玫瑰色的云朵,我们还是认出了这些水果来自何处。
\par 过了几天,上午在海滩上有交响乐音乐会演出,散场时我们遇到德·维尔巴里西斯夫人。我坚信自己听到的作品(《洛亨格林》序曲,《坦豪斯尔》均为瓦格纳的歌剧作品,分别于1850年和1845年上演。序曲等)表达了最高的真理,尽量提高自己以达到那作品的境界。为了理解这些作品,我从自身提炼出一切最美好、最深刻的东西,也将一切最美好、最深刻的东西赋予这些作品。



\paragraph*{3}

\par 外祖母和我从音乐会出来,踏上归途回旅馆。我们在海堤上停了一会,与德·维尔巴里西斯夫人交谈几句。德·维尔巴里西斯夫人对我们说,她在旅馆里为我们订了火腿干酪夹心面包片和奶油蛋。就在这时,我望见卢森堡亲王夫人从远处向我们走来。她半拄着一把阳伞,那高大而美丽的身躯现出微微的曲线,画出帝国时代美貌风流的女子珍爱的阿拉伯图案。这些女子双肩下垂,后背上提,臀部凹陷,腿部绷紧,很善于使她们的身躯像一条围巾一样无精打采地飘动。穿过躯体的那条肉眼看不见的柔软而倾斜的茎杆作为骨架,她们的身躯便围绕着这骨架飘动。
\par 卢森堡亲王夫人每天上午出来在海滩上转一圈。那时节,所有的人都洗完了海水浴,上岸准备吃午饭了。她是非到一点半钟才进午餐的,所以,洗海水浴的人早就放弃了那空荡而灼热的海堤之后,她才返回自己的别墅。德·维尔巴里西斯夫人向她介绍我的外祖母,也想介绍我。可是不得不向我询问我的姓名,因为她想不起来了。说不定她根本就不知道我姓什么,或者说,她早就忘记我外祖母将自己的爱女嫁给谁了。我的姓氏似乎给德·维尔巴里西斯夫人留下强烈的印象。这时,卢森堡亲王夫人已向我们伸出了手。当人们向奶妈带着的婴儿微笑时,常常还要加上一个亲吻。她与侯爵夫人说话过程中,不时转过头来带着这种亲吻的雏形,向外祖母和我投过柔和的目光。她希望不要显出自己地位比我们高的样子,但是她肯定没有计算好这段距离。由于计算错误,她的目光充满了善意,以至于我看到她就要像抚摸两头可爱的动物那样用手来抚摸我们。在驯化动物园\footnote{在巴黎布洛尼森林附近。}里,两头可爱的小兽就会越过铁丝网,朝她伸过头去。顿时,这种关于动物和布洛尼森林的想法在我心中固定下来。
\par 那时节,海堤上尽是来往走动、高声叫卖的小贩,卖的是点心、糖、小面包之类。亲王夫人不知道怎样表示她的好意,便拦住了从我们身边经过的第一个小贩。他只剩下一块黑麦面包了,就是人们扔给鸭子吃的那种。亲王夫人买了这块面包,对我说:“这是给你外祖母的。”可是她却把面包递给了我,微微一笑对我说:“你亲自交给她吧!”她大概以为,在我与动物之间如果没有中介,我的快乐就会更加完整了。
\par 又有其他小贩走过来,她将所有的东西都买了来,塞满了我的口袋,有扎好的一包一包,有角帽形小点心,有罗姆酒蛋糕,有大麦糖。她对我说:
\par “你自己吃,也给你外祖母吃吧!”
\par 然后她叫穿红锦锻衣服的小黑人给商贩付钱。那小黑人到处跟随着她,成了海滩上的奇景。此后,她向德·维尔巴里西斯夫人告别,并向我们伸过手来,有意对我们和她的女友一视同仁,当密友对待,而且有意降低自己的身份使我们能够接近她。不过有一次,她似乎将我们的水平在人的阶梯上放得不那么低,因为她与我们的平等,是通过亲王夫人向我外祖母温柔而充满母爱的微微一笑来表示的。人们像向一个大人告别一样向一个淘气孩子道再见时,就是这样微笑的。我的外祖母在进化上产生了美妙的飞跃,她不再是一只鸭子或一只羚羊,而已经成了斯万太太大概会称之为的“baby”\footnote{英语:婴儿。}。最后,亲王夫人离开了我们三个人,到充满阳光的海堤上继续散步去了。她那美丽的腰肢弯曲着,像绕在木棍上的一条蛇一样,缠绕在合拢起来拿在手中、白底蓝花的阳伞上。
\par 这是我遇到的第一位亲王夫人。我说第一位,因为马蒂尔德公主从仪态上说完全不是亲王夫人。这第二位,以后诸位会看到,以其钟情也叫我大吃一惊。第二天,德·维尔巴里西斯夫人对我们说:“她觉得你们很迷人。这个女人很有眼光,心地十分善良。她跟那许多女君主或亲王夫人可不一样。她具有真正的价值。”这时我便明白了,那是一种大老爷的和蔼可亲,自愿在国君与资产阶级之间充当中间人。德·维尔巴里西斯夫人又用坚信不疑的神情加上一句:“我想,她会很高兴再与你们见面。”她非常高兴能对我们这样说。
\par 离开卢森堡亲王夫人之后,当天下午,德·维尔巴里西斯夫人告诉我一件事,叫我更为惊异,而且又不属于和蔼可亲的范围。
\par “你父亲可是部里的司长?”她问我道。“啊!据说你父亲是个美男子。此刻他正在作美妙的旅行。”
\par 几天以前,我们从母亲的一封信中获悉,我父亲和他的旅伴德·诺布瓦先生丢失了行李。
\par “行李找到了,更正确地说,根本就没丢,就是这么回事。”德·维尔巴里西斯夫人对我们说。不知道为什么,对旅行的细节,她似乎比我们知道得更详细。“我想你父亲下个星期要提前回来了,他大概放弃去阿尔及西拉的计划了。不过他想在托莱多\footnote{西班牙城市。}多待一天,因为他对提香的一个弟子\footnote{此弟子即指西班牙画家格雷戈。}十分欣赏。我想不起此人的姓名了,不过在当地那是很有名气的。”
\par 对她所认识的那群人单纯、细微而又模糊的骚动,德·维尔巴里西斯夫人一向是用不动声色的眼镜远远打量的。我自忖,是什么巧合,使得她观看我父亲的那个地方,正好嵌了一块无限放大的镜片,使她那么有立体感地、极为详细地看到了我父亲所有令人愉快的东西,例如使他不得不回家的偶然事件呀,在海关遇到的麻烦呀,对格雷戈\footnote{格雷戈(1541—1614),西班牙画家。}的兴趣呀等等。这块镜片改变了她视野的比例尺,在万头攒动的芸芸众生中唯一使她看到这一个人,就像居斯塔夫·莫罗画朱庇特在一个软弱的下界女子旁边,将他画得超人大小一样。\footnote{大概指的是《朱庇特与塞墨勒》一画,画上,朱庇特将塞墨勒置于自己膝上,塞墨勒犹如其掌中玩物。也有说指的是《朱庇特与欧罗巴》。}
\par 我的外祖母向德·维尔巴里西斯夫人告辞,以便我们能在旅馆前多呼吸一会新鲜空气,一面等待着人家隔着玻璃窗向我们打招呼,说我们的午饭已经备好。这时只听得一阵喧嚣。原来是野蛮人部落国王那年轻的情妇刚刚洗罢海水浴,回来进午餐。
\par “这真是一大害,她应该离开法兰西!”首席律师此时正经过这里,他义愤填膺地大喊大叫。
\par 公证人的老婆却眼睛睁得大大的,死死盯着冒牌女君主。
\par “布朗代太太那样望着这些人,多么叫我着恼,我简直没法告诉你,”首席律师对首席审判官说道,“我真想给她一记耳光!这个女无赖,你这么看她就提高了她的身份,她就盼着人家注意她呢!你叫布朗代提醒提醒她,告诉她这很可笑。我呀,如果你再作出对这些冒牌货加以注意的模样,我再也不跟你们一道出去了!”
\par 卢森堡亲王夫人的马车,在她前来送水果那天,已在旅馆前停过。她的前来,自然也未逃过公证人、首席律师和首席审判官的老婆那一群人的眼睛。这几个女人看见德·维尔巴里西斯夫人这么受到敬重,都迫不及待地想知道她到底配不配,她们已经手忙脚乱了一些时候,想知道她是真正的侯爵夫人还是一个女冒险家。德·维尔巴里西斯夫人穿过大厅时,到处刺探不对头的事的首席审判官老婆从活计上抬起头来,望着那位夫人,那劲头叫她的女友们笑个半死。
\par “噢,我呀,你们知道,”她骄傲地说,“我一开始总是往坏处想。非给我拿出一个女人的出生证和公证人证件,我才会相信这个女人真正结了婚。此外,你们别害怕,我要进行小小的调查。”
\par 于是,每天这些女人都笑着跑来问:
\par “我们是来听新闻的。”
\par 卢森堡亲王夫人前来拜访的那天晚上,首席审判官的老婆把一根手指搁到嘴上。
\par “有新鲜事。”
\par “啊!她真了不起,邦森太太!我从未见过……你说,你说怎么啦?”
\par “咦,一个女人,黄头发,脸上擦的粉有一尺厚,一里开外就能闻到风流味,只有那些小姐才会有这样的车,她刚才来看望那位所谓的侯爵夫人啦!”
\par “哟,哟哟哟哟哟哟哟!嘿,你们看哪!就是我们看见的那位太太,你想起来了吗,首席律师?我们真觉得她不怎么样,可不知道她是来看侯爵夫人的。一个女的,带一个小黑人,是不是?”
\par “就是,就是。”
\par “啊,你们说得够多了。你们不知道她的姓名吗?”
\par “知道,我故意装做走错门了,拿着了她的名片,她的外号叫卢森堡亲王夫人!我多加提防就是有道理嘛!这地方,人很混杂,还有这类天使男爵夫人“天使男爵夫人”是小仲马1855年写的一个剧本《半上流社会》中的女主角。她是一个交际花,试图通过嫁人进入上流社会,但是没有成功。来搞鱼目混珠,真是够惬意的!”
\par 首席律师向首席审判官引证了马杜林·雷尼埃和玛塞特\footnote{马杜林·雷尼埃(1573—1613),著有讽刺作品《玛塞特》,叙述一个浪荡女人晚年成了虔诚的教徒的故事。}的故事。
\par 再说,这一误会,并非像一出轻松的喜剧里那些第二幕形成到最后一幕便解除了的误会一样只是暂时性的。德·卢森堡亲王夫人是英国国王和奥地利国王的外甥女。当她前来接德·维尔巴里西斯夫人一起出去坐马车兜风时,这两人总显得像两大怪一般,属于那种水域难以躲开的怪物。圣日耳曼区的人,在大部分资产阶级人士眼中,有四分之三是输光了赌本的恶棍(再说,个别人有时也确是如此),所以,任何人都不会接待他们的。在这方面,资产阶级是太老实了,因为贵族老爷的毛病决不会妨碍他们自己在凡是资产阶级永远不会受到接待的地方得到垂青,受到接待。而贵族自认为资产阶级了解这一点,所以他们在与己有关的事情上装得天真纯朴,而对他们那些穷困潦倒的朋友则故作诽谤,这就造成了误会。如果一个上流社会的人偶尔与小资产阶级发生关系,因为这个贵族非常富有,恰巧主持最大的一些财团,资产阶级终于会看到,一个贵族当资产阶级成员也很相称。但他还会发誓说,这个人绝不会与一个破了产的赌徒侯爵交往,认为侯爵越是和蔼可亲,他就越没有人缘。待到大宗生意管理委员会主席公爵先生娶了赌徒侯爵先生的女儿作自己的媳妇,资产阶级就更莫名惊诧了。那位侯爵虽是个赌徒,但他的姓氏在法国最为古老。正如一国之君宁愿娶已被废黜的国王之女作自己的儿媳,也不愿娶现任共和国总统之女给自己儿子为妻一样,这说明这两个世界之间彼此的看法都很虚幻,正如巴尔贝克海湾这一端海滩上的居民对位于海湾另一端海滩的看法也很虚幻一样:从里夫贝尔隐约可以望见马古维尔这个“骄傲的公主”。但是就是这一点也是骗人的,因为里夫贝尔的人以为,从马古维尔也能看见里夫贝尔。事实上与此相反,里夫贝尔的灿烂美景,从马古维尔那里,大部分是看不到的。
\par 我突然发烧,请来了巴尔贝克的医生。这位医生认为我不应该整天待在海边风吹日晒,给我开了几个药方。外祖母表面上恭恭敬敬地拿了药方,但我从那表面的恭恭敬敬上立刻看出来,她已坚定地下了决心,不照任何药方去买药。但是她对医生的保健建议很重视,接受了德·维尔巴里西斯夫人的好意,下午带我们坐马车去兜风。这样,上午,直到午饭前,我便在我的房间与外祖母的房间之间窜来窜去。
\par 外祖母的房间与我的房间不一样,不直接面对大海,而且从三个不同角度采光:海堤的一角,一个内院,田野。这房间内的器物也与我的房间不同,有上面绣着金银丝线和粉红花朵的沙发。一走进去便闻到的那种清新芬芳,似乎从那玫瑰色的花朵上散发出来。我更衣出去散步之前,穿过这个房间。这时,从南面进来的光线,与不同时刻进来的光线一样,折断了墙角,在海滩的反光旁,将绚丽多彩的临时祭坛安放在五屉柜上,似乎放上了小径上盛开的鲜花;光线那收拢、颤抖而又温暖的双翼挂在墙壁上,随时准备重新飞起。那光线像洗浴一般,晒热了小院一侧窗旁一方外省地毯,阳光如葡萄藤一般装点着小院,为小院的美丽动人、丰富多彩又加上动态的装饰,好似将沙发上那绣花丝绸一层层剥下,并将其金银丝边一一取下一般。这个房间有如一面棱镜,外面光线的七色在这里分解;有如蜂巢,我就要品尝的白昼的津液在这里溶解,散开,芳香醉人,看得见,摸得着;有如希望之园,溶成怦然跳动的银光和玫瑰花瓣。不过,先于一切的,还是我迫不及待地要知道今天早晨在海滨如涅瑞伊得斯涅瑞伊得斯是涅柔斯和多里斯的五十个女儿之一,在希腊诗人笔下,她“以微笑自娱”,勒贡特·德·利尔则称她是“欢乐的格劳科斯女神”。在希腊神话中,海神格劳科斯本为男性。般游玩的大海是什么模样。我拉开窗帘。每一个模样的大海停驻的时间从未超过一天。第二天,就是另一个大海了,偶尔也与前一日的大海相像。但我从未见过完全相同的大海出现过两次。
\par 有时,大海现出那样罕见的美,我远远见了,惊异万状,更加欢喜。是这一天早晨,而不是另一天早晨,半开的窗扉在我沉迷的眼前展现出格劳科斯女神的丽姿。她那慵懒的秀色,无力的呼吸,像朦胧的蓝宝石那样半透明。透过这蓝雾,我看到了给她点染上颜色的可以称得出来的各种元素在涌流。啊,真是得天独厚!女神露出睡眼蒙的笑容,令肉眼看不见的薄雾使阳光发出千变万化。这看不见的薄雾,无非是在她那半透明的表面周围所保留的一块空间而已。正因为有这一方空间,那表面就变得更为缩小,更为感人,就像雕刻家从整块石头的残存部分上分离下来的那些女神,他又不肯将这整块石头做成粗坯。女神就这样身着单色衣裙,邀我们到那粗糙而又在陆上的道路上去散步。我们坐在维尔巴里西斯夫人的敞篷四轮马车里,从这道路上,整日依稀望见她那慵倦跳动着的仙姿,却永远也到不了她的身边。
\par 为了使我们有充足的时间或到圣马尔斯,或到格特奥尔姆山岩,或到别的什么郊游的地方去,维尔巴里西斯夫人吩咐早早驾车。对于一辆行进缓慢的马车来说,这都是很远的地方,要走上一整天。想到我们要去远足,我十分快乐,哼起一首最近听到的什么曲子,来回踱着,等待维尔巴里西斯夫人穿戴整齐。如果是星期日,那么在旅馆门口的就不只是她的马车了。好几辆租来的街车,不仅等待着应邀前往菲代纳城堡康布尔梅夫人家做客的人,而且也等待着别的人。这些人与其像受惩罚的孩子一样留在这里,宁愿宣称巴尔贝克的星期天简直腻死人,他们一吃完午饭便启程躲到附近的海滩去或去参观什么名胜。当人们询问布朗代太太是否去过康布尔梅夫妇家中时,她甚至常常断然回答说:“没有,我们到贝克瀑布去了。”似乎纯粹是因为这个她才没有到菲代纳去度过一天。这时,首席律师就会大慈大悲地说:
\par “我真羡慕你,我跟你们一样改变主意就好了,那肯定别有情趣。”
\par 马车旁,我等人的门廊前边,一个年轻的穿制服的饭店仆役笔直站在那里,好像一株稀有品种的灌木。他那染色的头发惊人的和谐,较之他那树木的外表更引人注目。大厅相当于前廊,或初学教理者的教堂,或罗曼时代的教堂,不住在旅馆的人也有权经过。那大厅内的这位“外侍”的伙伴,并不比他多干多少活,但是至少还动弹动弹。很可能早晨他们是帮忙打扫的。但是下午他们就站在那里,像那些即使什么事也没有仍然站在台上增加哑角数目的合唱队员一样。叫我心惊胆战的那位总经理“站得高,看得远”,准备明年大大增加这些人的数目。他的这个决定叫这个旅馆的经理心里好生难过,因为他觉得所有这些小伙子无非是“碍事的人”,意思是说他们什么用也没有,还挡道。不过至少在午饭与晚饭之间,在顾客出入之间,他们还能填补情节的空白,就像德·曼特侬夫人的那些学生一样,他们身着年轻的古代以色列人的服装,每当爱丝苔尔或若阿德下场时,便由他们来演幕间插曲。\footnote{影射拉辛的最后两个悲剧《爱丝苔尔》和《阿塔莉》,此二剧应德·曼特侬夫人之请为圣西尔的各位小姐写成,她们在这两个戏的合唱队中扮演角色。}
\par 门外的那个穿制服的仆役,衣着华丽,身体修长瘦削。我就在离他不远的地方等待着侯爵夫人下楼来。他木然不动,而且木然不动上面又加上一层悲悲切切的神色,因为他的兄长都已离开了旅馆去寻找更光辉灿烂的前程去了,他自己在这块异乡土地上感到十分孤独。
\par 维尔巴里西斯夫人终于来到。照应她的车辆,服侍她上车,大概应当属于这个仆役职能的一部分。可是他也知道,一个随身带着仆役的人,是由自己的仆役来侍候的,而且一般来说,这种人在旅馆里给的小费很少,圣日耳曼老区的贵族们就是如此行事。德·维尔巴里西斯夫人同时属于这两种人。于是这株灌木仆役得出结论,他对侯爵夫人不抱任何希望,便任凭旅馆侍应部领班和侯爵夫人的贴身女仆将这位夫人及其衣物安置停当,而他自己仍然在那里忧伤地梦想着自己那些小兄弟令人艳羡的命运,保持着他那植物般的木然不动。
\par 我们启程。绕过铁路车站以后不久,便走上一条乡间小路。小路在迷人的园圃间拐一个弯,又拐一个弯。路两旁均为耕过的土地。很快我便感到这条小路像贡布雷的小路一样熟悉而亲切。耕地中间,不时可见一株苹果树。苹果树上确实已经没有花朵,只有一簇雌蕊。但这已足以令我心醉神迷,因为我又认出了那无法模拟的树叶。那大大的叶子,有如婚礼结束后台阶上的地毯,刚刚被红扑扑的花朵那白缎长裙的拖裾踏过。
\par 翌年五月,在巴黎,有多少次,我在花店里买上一枝苹果树枝,然后在它那花朵前度过一整夜啊!花朵放出同样的乳白色的津液,将其飞沫又撒在叶芽上。似乎卖花商人对我十分慷慨,出于创造性的趣味,亦出于巧妙的对比,又在白色的花冠间,每边都加上了恰如其分的粉红色花苞。我久久凝望着这花朵,吩咐将花放在我的灯顶上,直到黎明给花朵送来了曙光,我常常还在望着它们。在巴尔贝克,黎明大概也同时放出这曙光的吧?我在想象中极力将这花朵带回这条路,让这花朵大量增加,将它铺满已准备好的画布上那准备好的框架。边框便是那些园圃。园圃的图案,我已牢记在心。我是多么希望,也应该,在春天怀着天才美妙的热情,以其各种色彩覆盖住其画稿时,有一天重见这一切啊!
\par 上车之前,我已经构思了大海的画面。我要去寻找这画面,我希望看到“普照大地的阳光”下的这一画面。而在巴尔贝克,在那么多的洗海水浴的人、小棚、游艇构成的俗气的插花地之间,我看到的只是支离破碎的画面,是我的梦幻接受不了的画面。
\par 德·维尔巴里西斯夫人的马车到了一处海滨的高处,当我从树木的枝叶间依稀望见了大海时,这么远,那些将大海移到大自然与历史之外的细节,自然都消逝了。我望着大海的波涛,可以尽情地想象,勒贡特·德·利尔在《俄瑞斯忒斯》\footnote{埃斯库勒斯的三部曲是这个标题;但勒贡特·德·利尔从此汲取灵感写成的悲剧,剧名则叫《复仇三女神》。此剧于1873年1月6日首次在奥代翁剧场上演,剧本于当年出版。}中给我们描绘的正是这样的波涛。那时,英雄赫楞手下那些长发勇士,“犹如食肉飞禽黎明时飞过”,“以十万船桨拍打着轰鸣的浪涛”。\footnote{这是剧中人道尔迪比奥斯说的话。}反过来,我距离大海又不够近了,我似乎感到大海不是有生命的,而是固定不动的,我再也感觉不到在那一片色彩之中大海的勃勃生机,如同一幅画在树叶间展现出的一片色彩。此时大海显得和天空一样单薄,只不过比天空颜色更深罢了。
\par 德·维尔巴里西斯夫人见我喜欢看教堂,便向我许诺说,我们以后要去看这个,要去看那个,尤其要去看克拉克维尔的教堂。她说那个教堂“完全掩映在常春藤之中”,说着作了一个手势,似乎很有兴味地将那不在眼前的教堂正面包在看不见而十分优美的枝叶之中。德·维尔巴里西斯夫人作出这种描写性的小小的动作,时常用很准确的字眼将一处古迹的诱人和特别之处表述出来,总是避免使用技术性的词汇。但她无法掩饰,对她所谈的事情,她是非常清楚的。她在她父亲的一座城堡中长大,那座城堡所在的地区有些教堂与巴尔贝克周围的教堂为同一式样。那座城堡是文艺复兴时期建筑最完美的楷模,而她对建筑竟然没有产生兴趣,她似乎极力在为自己辩解。这座城堡也是一所真正的博物馆。另外,肖邦和李斯特在那里弹过琴,拉马丁在那里朗诵过诗作,整整一个世纪的著名艺术家都在那里,在她家的纪念册上写出感想,写过和谐的乐章,画过速写。因此,德·维尔巴里西斯夫人出于美意、良好的教育、真正的谦逊,或缺乏哲学精神,对她自己掌握的对所有各种艺术的知识,只赋予这种纯物质的来源,最后也就显得似乎将绘画、音乐、文学和哲学均视为在著名的列入文物保护清单的古建筑中长大、受最最贵族式教育熏陶的一位少女的特权了。人们似乎有这样的印象,对她来说,除了她继承下来的画以外,就没有别的画。她戴的一条项链,垂到长裙上,我外祖母很喜欢,她感到十分高兴。在提香为她的一位曾祖母绘制的肖像上,就有这条项链。这条项链从来没有出过这个家族。这样就可以肯定这是真品了。不知怎样买来的画着克里索斯的画,她听都不爱听,事先就确信不疑那肯定是赝品,根本不想看。我们知道她本人也画一些花卉水彩。外祖母曾经听人吹捧过这些作品,就与她谈起这事。德·维尔巴里西斯夫人出于谦虚转了话题,倒也没比对恭维已经司空见惯的相当有名气的艺术家流露出更多的惊讶和快乐。她只是说,这是很令人愉快的消遣,虽然画笔下的花朵并没有什么了不起,至少画花使你生活在自然花朵的世界中。尤其当人们不得不仔细注视以求临摹得很像时,对天然花朵的美,是百看不厌的。但是在巴尔贝克,德·维尔巴里西斯夫人给自己放了假,好让自己的双眼得到休息。
\par 外祖母和我,见她甚至比绝大部分资产阶级都更持“自由派”见解,真是惊讶万分。人们对驱逐耶稣会士感到愤慨,她很迷惑不解。她说一直是这么做的,甚至王政时代,甚至在西班牙,也是如此。她捍卫共和,只在下列情况下才谴责共和国的反教权主义:“我想去望弥撒,人家阻拦我;我不想去,人家非强迫我去。我认为这二者都一样糟糕。”她甚至说出这样的话来:“哟!今日的贵族,这算什么玩艺!”“在我看来,一个人不劳动,简直一钱不值。”说不定就是因为她感觉到人家从她嘴里撷取讽刺挖苦、味道醇厚、难以忘却的东西,她才这么说的。
\par 我们很尊重一些人的聪明才智,采取谨慎而又小心翼翼的不偏不倚态度拒绝谴责保守主义者的想法。德·维尔巴里西斯夫人正属于这种人。我外祖母和我,经常听到她坦率地表达一些很先进的见解——不过,还没有先进到赞同社会主义的地步。社会主义是她的眼中钉,我们几乎认为,在各种事情上,真理的尺度和典范都在她身上了。当她对自己的提香的画,她的城堡的廊柱,路易菲利浦谈话的幽默发表评论时,真是她说什么我们信什么。
\par 但是,那些谈起埃及绘画和伊特鲁立\footnote{亚伊特鲁立亚为意大利古地区名。}铭文来令人着迷的学识渊博的学者,谈起现代作品来可就太平常了。我们不得不自忖,对于他们擅长的那些学问,是否我们估价太高,因为他们对波德莱尔的研究很简单,平平常常,而他们对现代作品的研究就连这种平平常常都显不出来。当我就夏多布里昂、巴尔扎克、维克多·雨果向德·维尔巴里西斯夫人提问时——往昔她的父母全接待过这些人,她自己也隐约见过他们——她嘲笑我对这些人十分佩服。像她刚刚对一些贵族大老爷或一些政治家讲一些挖苦的话一样,也对他们讲上一些挖苦的话。她对这些作家品评很苛刻,说他们正是缺少下列的优秀品质:谦虚,不自我炫耀,满足于一种朴实的艺术,恰到好处而不再多加一笔,避免口若悬河以显得可笑,随机应变,总之,缺少那些判断适度、简单朴素的品格。人们告诉她,一个真正有价值的人会达到具有这些品格的高度。看得出来,她毫不犹豫将一些人放在这些作家之上。也说不定那些人由于具有这些品格,确实能胜过巴尔扎克、雨果、维尼式的人物,或在一间客厅里,一次学会上,一次大臣会议上,能胜过莫\footnote{莱莫莱伯爵(1781—1855),参加过第一帝国政府,后拥护七月王朝,1836—1839年任路易-菲利浦政府的首相。},冯塔那\footnote{冯塔那(1757—1821),曾拥护法国大革命,但又被革命暴力吓破了胆。为重建帝国的倡导人之一。“百日事变”时,他没有响应拿破仑的召唤,因此得到路易十八的青睐,曾任国务大臣。},维特罗尔\footnote{维特罗尔男爵(1774—1854),曾在孔德反革命军队中战斗,后投到帝国一边,但又参与了泰勒朗的阴谋活动,无论是查理第十还是路易-菲利浦都未能使他实现自己的野心,但他始终是狂热的保皇党。},贝索\footnote{贝索(1816—1880),因政治活动成功先后获男爵及公爵称号,1851年拒绝效忠第二帝国。1871年以后,曾被任命为高师校长。},巴斯基埃\footnote{巴斯基埃(1767—1862),恐怖时期被关进监狱,效忠帝国和路易十八,参加过黎希留和德卡兹内阁,被路易菲利浦任命为元老院主席。},勒布伦\footnote{勒布伦(1785—1873),七月王朝时期大为走红,拿破仑第三接纳他进了参议院。写过不少悲剧、诗歌。},萨方迪\footnote{萨方迪伯爵(1795—1856),先后效忠于拿破仑和路易十八、查理第十、路易菲利浦。}或达吕\footnote{达吕(1767—1829),先拥护革命,恐怖时期被捕入狱。曾为拿破仑勇敢作战。1819年成为法兰西元老院成员。}。
\par “这就像司汤达的小说一样。你好像很佩服司汤达,可你如果用这种语气与他谈话,那就会叫他大吃一惊了。我父亲在梅里美先生——至少这一位是个天才人物——家里经常见到司汤达,他常常对我说佩耶(这是他的真名)俗不可耐,但在晚宴上又十分风趣,叫人简直无法相信他会写出那样的书。再说,你大概也看到了,德·巴尔扎克先生对他极度赞美时,他是怎样耸肩膀来回答的。至少在这一点上,他是出身高贵的人。”
\par 所有这些伟人,她都有他们的真迹。她的家庭与这些人有过这样特殊的关系,她以此自夸,似乎认为与像我这样未能与这些人有所交往的年轻人相比,她对这些人的评论更为正确。
\par “我认为我可以谈论他们,因为他们常到我父亲家里来。正如很有风趣的圣勃夫所说,有关这些人,应该相信就近看见过他们而且能够对他们的价值作出更正确的评价的人。”
\par 有时,马车在耕地之间走上一条上坡路,我们对田地感受更真切,上坡路给田地加上了真实的印记。像从前某些大师给自己的画幅添上一朵珍贵的小花一样,也有几株犹豫不决的矢车菊,与贡布雷的矢车菊十分相像,追随着我们的马车。很快,我们的马匹就把这些矢车菊甩在后面了。但是,再走几步,我们又远远看见另一株在等待着我们,早在草丛中、在我们面前竖起了它那蓝色的小星。有几株更大着胆子走过来,立在路边。于是,这些矢车菊,与我遥远的回忆和家养的花朵一起,形成了一片星云。
\par 我们下坡,向海岸走去。这时我们会迎面遇到步行、骑自行车、坐着蹩脚的车子或者坐着马车上坡的姑娘。她们是这美好一天的花朵。但是她们与田间的花朵又不相像,因为每一个姑娘都显示出某种特有的东西,这种特有的东西在另一个姑娘身上是没有的。这就使得这一个姑娘在我们心中激起的欲望,与她的同类在一起,是不能得到满足的。某一个田庄姑娘赶着自家的乳牛,或者半躺在小车上,某一个小铺掌柜的女儿在散步,某一个衣着华丽的小姐坐在敞篷四轮马车的折叠式座席上,对面是她的父母。
\par 我在梅塞格利丝一侧独自散步时,曾怀着幻想,希望有一个村姑经过,我将她拥在自己的怀里。一天,布洛克告诉我,这种幻想并非是什么与我身外的任何事情都丝毫不相符合的想入非非。人们路遇的所有姑娘,村姑也好,小姐也好,都随时准备实现同样的幻梦。这一天,布洛克自然为我开辟了一个新时代,对我来说,改变了生命的价值。可我现在病魔缠身,从不单独外出,我是注定永远也无法与她们做爱了。一个监狱中或医院中生下的孩子,长时期以来,一直认为人的机体只能消化干面包和药,当他忽然获悉桃子、梨子、葡萄并不仅仅是田野的装饰品,而是鲜美、可以消化的食物时,该是多么兴高采烈,欢喜若狂!即使看守他的狱卒或他的看护不许他去采摘这些美丽的果实,对他来说,世界也显得更加美好,生活也显得更宽厚了。我就像这个孩子一样。当我们知道,在我们身外,现实与欲望相符,即使对我们来说,这欲望已无法实现,在我们看来它也更为美好,我们会更加有信心地依傍着它。我们会怀着更大的快乐想到,假设这种欲望得到了满足,那该是怎样的生活!当然要做到这一点,有一个条件,那就是能够暂时从我们的思想中排除那个小小的偶然的特殊的障碍。正是这个障碍,使我们的这个欲望无法得到满足。自从我知道可以亲吻从身旁经过的美丽姑娘的双颊那一天开始,我对她们的内心活动就变得十分好奇起来,这个宇宙对我也显得更有兴味了。
\par 德·维尔巴里西斯夫人的马车飞快奔驰。我刚刚来得及看清迎面走来的那个少女。然而人的美与物的美不一样,我们感到这是一个唯一的少女的美,是意识到了的、有意识的美。她的个性,她那隐约可见的心灵,她那我不了解的意愿,刚刚在她那并不专注的目光深处——转瞬间,这目光成了与为雌蕊准备的花粉完全相仿的神秘物——形成一个大大缩小了的、而又不完整的小小的形象,我就感到从自己的内心涌出一种尚为雏形的欲望,模模糊糊,很小很小,这个欲望就是:在她的思想没有意识到我这个人、我没有妨碍她的欲望向别人奔去、我没有停驻在她的幻想中、抓住她的心之前,不要让这个姑娘走过去!可是我们的马车走远了,那美丽的姑娘已经在我们身后。她对我没有产生任何构成一个人的概念,她的明眸刚刚看到我,就已经把我忘记了。是不是因为我只是对她瞥过一眼,才觉得她如此美貌呢?很可能。疾病或贫困使我们不能游历某一国度;此生所余时日无多,这时日已经黯然失色;首先,不可能在一位女子身边停留,很可能也不会再度与她相逢,这一切都顿时赋予她一种魅力,与上述那个国度,那些时日所具有的魅力相同。这是我们注定要失败的战斗。所以,如果没有习以为常这个因素的话,对于每时每刻都受到死亡威胁的人——也就是所有的人——来说,生活会显得十分甜美。其次,在这样的路遇中,一般来说,过路女郎的风韵与很快交臂而过紧密相关。对我们无法拥有的东西产生欲望,这种欲望导致的想象翻腾起来,不受上述路遇中完全感受到的现实的限制。尽管夜幕降临,马车飞快奔驰,在乡村,在城市,没有哪一个女性的身姿,像古代大理石像一般为将我们带走的快速所摧残;也没有哪一个女性的身姿受到将它吞没的黄昏的摧残。而这黄昏,在每一个路口,从每一家店铺的深处,无不向我们的心射来美神的箭矢。遗憾更挑起我们的想象力,我们的想象又给那转瞬即逝的、残缺不全的过路女子形象添加了许多东西。我们有时真想自忖,在这世界上,美神是否正是添加的这一部分,而不是别的呢?
\par 如果我得以下车,得以与这位迎面相遇的女郎交谈,说不定她皮肤有什么毛病会使我幻想破灭,而从车上,我则没有看清那个毛病(于是,一切要进入她的生活的努力,我都立刻觉得不可能了。美是一系列的假设。我们已经看到向未知展开的道路,丑一拦住路,便把那些假设都缩小了)。说不定她只说一句话,微露笑靥,就能给我提供意料不到的启示,数目字,使我能领会她脸上的表情和她举止的含义,而这一切立刻都会变得平淡无奇。这是可能的。有一阵,我与一个十分严肃的人在一起,尽管我找出千百个借口要把他甩掉,我都无法离开。我感到自己一生中遇到的姑娘,从未像那些日子里遇到的女郎那样撩人心弦!第一次去巴尔贝克以后数年,在巴黎,我与父亲的一位朋友坐马车兜风,夜色朦胧中看见一个女子匆匆行走。我想一个人就活一辈子,因为得体不得体的原因而丢掉这份幸福,未免太不讲道理。我于是没有道歉便跳下了车,开始追踪那个素未谋面的女郎。到了十字路口,我被她拉下两条街。到了第三条街,才又找到她的踪影。最后,在一盏街灯下,我气喘吁吁地与年老的维尔迪兰太太撞了个满怀。原来是她!这个人,是我到处避之不及的!她又惊又喜,大叫道:“啊呀,跑着追我,为的是向我问个好,这可太客气了!”
\par 这一年,在巴尔贝克,每逢这一类的相遇,我就对外祖母和德·维尔巴里西斯夫人说,我头痛得厉害,最好我一个人步行返回。她们不肯叫我下车。这样,在我准备就近看个仔细的美好系列上,就又加上了这个美丽的姑娘(比一处古迹还要难找得多,因为她无名无姓,又是活动的)。不过其中有一个,碰巧又从我眼前经过,当时的情形,我认为是可以如愿以偿与她结识的。
\par 那是一个卖牛奶的女郎,她从田庄来,给旅馆送增购的奶油。我想,她也认出了我,而且她确实也非常专注地望着我,大概这种专注只是由于我对她的专注使她感到惊异而引起。第二天,我整个上午都休息,弗朗索瓦丝近中午时分来拉开窗帘,她交给我一封信,是人家留在旅馆里给我的一封信。我在巴尔贝克一人也不认识。我毫不怀疑这信是那个卖牛奶女郎写的。可惜不是。那只是贝戈特的信。他从这里路过,想看看我,但是得知我在睡觉,就给我留了这封热情的短笺。开电梯的人给这封信写了一个信封,我还以为那是卖牛奶女郎的字迹。
\par 我失望极了。即使想到能得到贝戈特一函确实更为难得,更是一种恭维,也丝毫不能安慰我因此信不是卖牛奶女郎所写而感到的失望。比起我只在德·维尔巴里西斯夫人的马车上远远瞥见的姑娘们来,就是这个姑娘,我也没有多见几次。一个个看见这些姑娘,又一个个失去这些姑娘,使我更加烦躁不安,我觉得那些告诫我们节欲的哲学家们确实很明智(万一他们肯谈到人的欲望的话。因为这是唯一能给人留下焦虑的欲望,适用于未知的意识。设想哲学肯谈论对财富的欲望,那恐怕太荒谬了)。不过我准备对这种不完全的明智作出判断,我心想,这些巧遇使我觉得这个世界更美了。这个世界要叫所有的乡间小路上开起既不寻常又寻常的花朵来,是每日转瞬即逝的珍宝,又是散步中意外的收获。种种偶然的情形可能不会经常重演,正因为偶然才使我无法受益,这又赋予生活以新的情趣。
\par 我希望有一天,我更自由,能够在别的路上找到相同的少女。不过,也许我这样希望的同时,就已经开始歪曲了想生活在一个自认为漂亮的女人身边这种人的欲望所具有的纯个人性质。我认为能够人为地使这种欲望产生,仅从这一点来说,我已经暗暗承认这种欲望的虚幻了。
\par 那天,德·维尔巴里西斯夫人带我们去克拉克维尔,她对我们说过的、爬满常春藤的教堂就在这里。这教堂建在一个小丘上,俯瞰着中世纪的小桥。我的外祖母以为让我一个人参观这一古迹我一定会很开心,就向她的女友建议,她们到糕点铺去尝尝点心。这铺子就在广场上,看得清清楚楚,金色的门面古色古香,犹如一件非常古老的文物的另一部分。我们约定,我随后去那里与她们会齐。她们将我留在一片绿荫前。在这里,要认出一所教堂来,一定要花些力气,才能叫我更确切抓住教堂的概念。确实,当人们以本国语译成外国语或外国语译成本国语的形式强制学生将句子的意义从他们熟悉的形式中剥离出来的时候,往往他们会更具体地抓住句子的意思。与此相同,平时,当我站在叫人一见了就辨认得出来的钟楼面前时,我不大需要教堂的概念。可是今天,我不得不时时借助于这个概念才不至于忘掉这里,这个茂密的常春藤拱腹便是彩色的尖顶大玻璃窗,那里绿叶隆起,是因为那里有一个廊柱的突起部分。这时,微风吹过,好似一抹阳光,颤抖而荡漾的伴流穿过会动的大门,那大门便也颤动起来。叶子如汹涌的波涛,一个挤着一个。花草组成的正面,震颤着,将波澜壮阔的、受到抚慰的、渐渐消失的巨柱统统卷走。
\par 我离开教堂时,在古老的小桥前看见村中的一些少女。大概因为那天是星期日,她们精心梳妆打扮,站在那里,与过路的小伙子搭话。有一个个子很高的姑娘,半坐在桥沿上,双腿悬空,面前有一小缸,里面全是鱼,很可能是她刚刚钓上来的。她穿得没有别的姑娘好,但是似乎有某种权势高出她们一头,因为她们跟她说话,她几乎不理不睬。她的表情更严肃,更有意志力。她肤色深棕,双目柔和,但对周围的一切均投以鄙夷的眼光,鼻子小小,形状优雅而可爱。我的目光落在她的皮肤上,也可以勉强相信我的双唇是跟随我的目光的。但是,我要触及的,并不仅仅是她的躯体,还有活在她躯体中的心。而与心接触只有一种方法,那就是引起她的注意;只有一种进入的方法,那就是在她心中唤起一个想法。
\par 这个美丽的钓鱼女郎,她那内心似乎仍对我关闭着。就在我根据折射的迹象瞥见我自己的影像在她那目光的镜子里飞快地反射出来以后,我仍然怀疑,我是否已经进入她的内心。这折射的迹象对我十分陌生,似乎我进入一条牝鹿的视野。我的双唇从她的双唇上得到快感,这对我还不够,我还要给她的双唇以快感。同样,我希望进入她内心的,在那里停驻的对我的想法,不仅仅给我带来她的注意,而且还有她的钦佩,她的欲望,要迫使她记住我,直到我能与她重见那一天。
\par 我只有一小会时间。我已经感到姑娘们见我如此呆立在那里,已开始笑起来了。我口袋里有五个法郎。我掏出这五个法郎来。为了使她听我说话的可能性更大一些,我把这个硬币在她眼前放了一会,然后才向这个美丽的姑娘解释我委托她办的事:
\par “看来你像是本地人,”我对钓鱼女郎说,“你能热心帮我跑一趟吗?必须到一个点心铺子门口去,据说这店铺在一个广场上,可我不知道在哪,那里有一辆马车在等我。再等一下!……为了不致混淆,你就问这是不是德·维尔巴里西斯侯爵夫人的马车。此外,你要看清楚,这辆马车有两匹马。”
\par 我就是想让她知道这些,以便她对我产生很深的印象。当我道出“侯爵夫人”和“两匹马”这几个字以后,突然感到极大的平静。我感觉到钓鱼女郎会记得我,想与她重逢的欲望也伴随着对于再不能与她重逢的恐惧在消散而部分地消散。我似乎觉得刚才已经用肉眼看不见的嘴唇触及了她的内心,而且我很讨她的欢喜。这样强占她的精神,这种非物质性的占有,也与占有肉体一样,使她去掉一些神秘感……
\par 我们下坡,朝于迪迈尼尔驶去。骤然间,我心中充满了深深的幸福。自贡布雷以来,我并不常常有这种幸福感,这与马丹维尔的钟楼赋予我的幸福颇相类似。但是这一次,这幸福感是不完全的。在我们所循的驴背形马路缩进去的地方,我刚刚隐约看见了三株树木,大概是一条林荫道的入口,构成了我并非第一次见到的图案。我无法辨认出这几株树木是从哪里独立出来的,但是我感到从前对这个地点很熟悉。因此,我的头脑在某一遥远的年代与当前的时刻之间跌跌撞撞,巴尔贝克的周围摇曳不定,我自问是否整个这一次散步就是一场幻觉,是否巴尔贝克是只有我想象中才去过的地方,是否德·维尔巴里西斯夫人就是小说中的一个人物,而这三株老树,是否就是从你正在阅读的书籍上面抬起双眼来时重新找到的现实。它向你描绘出一个环境,人们最后会以为自己确实置身于这个环境之中了。
\par 我凝望着这三株树,我看得清清楚楚。但是我的头脑感觉到它们掩盖着某种东西,我的头脑抓不住,就像有些物件放得太远,我们伸直了胳膊,手指头也只能碰着那物件的封套,而一点没抓住那物件一样。这时,我们稍事休息,再使一个猛劲伸出胳膊去,极力达到更远的地方。但是对我来说,要让我的思想能这样集中起来,使一个猛劲,我必须独自一个人才行。就像我离开父母到盖尔芒特一侧去散步那样。此时此刻,我多么希望能够躲开!
\par 可能我那么做就好了。我辨认出了这种快乐,确实,它要求某种就思维而进行思维活动。与这种活动相比,使你放弃这种活动的那种慵懒舒适看来就很平庸了。这种快乐,其对象只能预感到,我要自己为自己去创造。我只感受过难得的几次,但是每一次我似乎都觉得,这中间发生的事情无关紧要,只要赖之以这每一件事实,我都可以开始一次真正的生活。
\par 有一会,我将手放在眼前,为的是能够闭上眼睛,而又不要为德·维尔巴里西斯夫人所察觉。我坐在那里,什么也不想,然后从我用更大的力气集中起来的思想中,向三株树的方向再往前一跃,或者更准确地说,往我内心的方向一跃。在这个方向的尽头,我在内心看见那三株树。我重又感到在那树后还是那个熟悉而又模糊的物件,而我无法拉到自己身边来。随着马车的前进,我看见这三株树都在靠近。从前,在什么地方,我曾经注视过这三株树呢?在贡布雷周围,没有哪一个地方有这样开始的一条林荫道。三株树使我忆起的名胜,在有一年我与外祖母一起去洗矿泉浴的德国乡间,也没有位置。是否应该相信,它们来自我生活中已经那样遥远的年代,以至于其四周的景色已在我的记忆中完全抹掉,就像在重读一部作品时突然被某几页深深感动,自认为从未读过这几页一样,这几株老树也突然从我幼时那本被遗忘的书中单独游离出来了呢?难道不是正相反,它们只属于梦幻中的景色?我梦幻中的景色总是一样的,至少对我来说,这奇异的景观只不过是我白天做的事晚上在梦中的客观化罢了。白天,我努力思考,要么为了探得一个地方的秘密,预感到在这地方的外表背后有什么秘密,就像我在盖尔芒特一侧经常遇到的情形一样;要么是为了将一个秘密再度引进一个我曾想渴望了解的地方,但是,见识这个地方的那天,我觉得这个地方非常肤浅,就像巴尔贝克一样。这几株老树,难道不是前一夜一个梦中游离出来的一个全新的影像,而那个影像已经那样淡薄,以致我觉得是从更远的地方来的吗?抑或我从未见过这几株树,它们也像某些树木一样,在身后遮掩着我在盖尔芒特一侧见过的茂密的草丛,具有跟某一遥远的过去一样朦胧、一样难以捕捉的意义,以致它们挑起了我要对某一想法寻根问底的欲望,我便认为又辨认出某一回忆来了?抑或它们甚至并不遮掩着什么思想,而是我视力疲劳,叫我一时看花了眼,就像有时在空间会看花眼一样?这一切,我不得而知。
\par 这期间,几株树继续向我走来。也可能这是神话出现,巫神出游或诺尔纳是斯堪的纳维亚神话中的命运之神。出游,要向我宣布什么神示。我想,更可能的,这是往昔的幽灵,我童年时代亲爱的伙伴,已经逝去的朋友,在呼唤我们共同的回忆。它们像鬼影一般,似乎要求我将它们带走,要求我将它们还给人世。从它们那简单幼稚又十分起劲的比比画画当中,我看出一个心爱的人变成了哑人那种无能为力的遗憾。他感到无法将他要说的话告诉我们,而我们也猜不明白他的意思。不久,两条路相交叉,马车便抛弃了这几株树。马车将我带走,使我远离了只有我一个人以为是真实的事物,远离了可能使我真正感到幸福的事物。马车与我的生活十分相像。
\par 我看见那树木绝望地挥动着手臂远去,似乎在对我说:“你今天没有从我们这儿得悉的事情,你永远也不会知道。我们从小路的尽头极力向你攀去,如果你又叫我们堕入这小路的尽头,我们给你带来的你自己的一部分,就要整个永远堕入虚无。”确实,虽然以后我又一次体会到刚才这种快乐和焦虑,虽然有一天晚上——已为时过晚,而且永远不再来——我非常怀念这种快乐和焦虑,可是我到底没明白这些树想给我带来什么,也不知道我从前到底在什么地方见过。待马车再次改变方向,我背对着大树,再也看不见大树的时候,德·维尔巴里西斯夫人问我为什么面带沉思,我当时心里真是十分难过,似乎我刚刚失去了一位朋友,我自己刚刚死去,我背弃了一位死者或者没有认出一位天神来。
\par 该想到归去了。德·维尔巴里西斯夫人对大自然颇有欣赏能力,比我外祖母更为冷静。甚至除了博物馆和贵族住宅之外,她也能辨认出某些古老的事物那纯朴而壮丽的美。她吩咐车夫走通往巴尔贝克的老路。这条路来往的人很少,两旁种着老榆树,叫我们看上去叹为观止。
\par 我们一旦得知有这条老路,以后出去时,总要走这条路,除非去时我们已走过这条路,返回时,为了换换花样,我们才走另一条路,穿过尚特雷纳和冈特卢的树林。林中,无数小鸟就在我们身边相互应答,但是我们看不见小鸟在哪里,使人产生与闭上眼睛完全相同的宁静印象。我就像普罗米修斯被锁链拴在山岩上一样被紧紧拴在我的折叠式座席上,倾听着我的俄刻阿尼得斯\footnote{俄刻阿尼得斯是大洋与忒堤斯的女儿,海洋中的女神,相传有三千个。在埃斯库勒斯的《被缚的普罗米修斯》中,她们构成合唱队,对英雄的痛苦表示无限同情。}。纯属偶然,我望见一只小鸟从一片树叶跳到另一片树叶底下,表面看上去它与这合唱似乎没有多大关系,以至于我觉得从这个跳跃的、吃惊而又没有眼神的小小躯体上,看不出来为何要来这个大合唱。
\par 这条路与人们在法国遇到的许多这一类的路完全相同,上坡很陡,然后下坡很长。当时,我不觉得这条路有什么迷人的地方,只是为返回住所而感到高兴。但是后来,对我来说,这条路变成了一个快乐的因由,它留在我的记忆中,如同一条道路开头的一段。我后来散步时或旅行中经过的所有与此相像的道路,无法延续下去,都立刻与它连接起来,借助于它,能够与我的心即刻相通。马车或汽车一踏上这样的路,似乎是我与德·维尔巴里西斯夫人一起走过的那条路的延续,就像刚刚过去的事情支撑我现在的意识一样,我在巴尔贝克附近出游的那些下午产生的印象便立刻来支撑我的意识(这中间的年代完全消失)。那时,树叶散发着芳香,薄雾在缓缓升起,即将抵达的村庄后面,可在树木之间依稀望见落日的余晖,似乎那里便是我们的下一站,树木葱郁,距离遥远,当晚是到不了的。现在我在另一个地区,在一条相似的路上,我感受的印象,充满了与那时的印象相同的次要感觉:自由呼吸,好奇,懒散,有胃口,欢快,排除一切其他的感受。原来的印象与此刻的印象连接在一起,又得到了加强,更加浓稠,成为一种特殊的快乐类型,几乎是一种生活框架,后来我很难得有机会再次遇到。但是在这个框架之中,唤起回忆便在具体物质感受的现实之中注入了相当大一部分回忆的、想象的、难以捕捉的现实,在我经过的这些地区里,除了一种美感以外,又叫我产生希望从此永远在这里生活这种转瞬即逝而又狂热的欲望。有多少次,只是因为闻到了树叶的芳香,便忆起坐在德·维尔巴里西斯夫人对面的折叠式座席上,与卢森堡亲王夫人擦肩而过时,亲王夫人从自己的马车上向她致意,忆起回到大旅社进晚餐的情景。这一切都如同难以形容的幸福一般出现在我的面前。而这种幸福,无论是现在,还是未来,都不会再次还给我们。人的一生中只能领略一次!
\par 常常,我们未返回,太阳就已落山。我将天上的月亮指给德·维尔巴里西斯夫人看,腼腆地背诵出或夏多布里昂,或维尼,或维克多·雨果的美丽诗句:“它将忧郁的古老秘密撒下来”,\footnote{这是夏多布里昂在《阿达拉》中的诗句。}或“像迪亚娜在泉边那样哭泣”,\footnote{这是维尼《牧羊人之家》中的倒数第二句。}或“暗影如新婚之夜,庄重而崇高。”\footnote{这是维克多·雨果《世纪传说》中《沉睡的布兹》中的诗句。}
\par “你觉得这些诗句很美,是吗?”她问我,“‘天才’,像你所说的那样?我告诉你吧,我看见人家现在把一些事情看得太重,总感到很奇怪。而这些先生的朋友们,虽然一面也充分肯定他们的长处,却也首先拿这些事情开玩笑。从前不像现在这样滥用天才这个词。如今,如果你对哪一个作家说,他只有些才华,他会把这当成是一种污辱。你刚才给我背诵了夏多布里昂先生关于月光的一个长句子,我可反对,我有我的道理,你马上会明白。夏多布里昂先生常到我父亲家里来。单独跟他相处时,他非常令人愉快,因为这时他很纯朴,逗人开心。可是客人一多,他就开始装腔作势,变得十分可笑。在我父亲面前,他宣称是他将辞职书摔到了国王的脸上,并且指导教皇选举会。他忘了,是他亲自托我父亲去向国王求情再次起用他,我父亲也曾亲耳听到他对选举教皇发出那些疯狂的预言。关于这个颇有名气的教皇选举会,应该听听布拉加斯先生的话,他跟夏多布里昂先生可不是一样的人\footnote{教皇列昂十二世于1829年去世。当时夏多布里昂为驻罗马大使,对选举新教皇极为关切。德·布拉加斯当时为驻拿不勒斯大使,对选举新教皇亦极关切。最后是红衣主教卡斯蒂格里奥尼当选,成为教皇庇护八世。}。至于德·夏多布里昂先生关于月光的那几句话嘛,在我们家完全成了一种负担。每次城堡四周月光明亮时,如果有新来乍到的客人,总是建议他晚餐后带德·夏多布里昂先生出去换换空气。待他们回来时,我父亲一定会把客人拉到一边,对他说:
\par “‘德·夏多布里昂先生口若悬河吧?’
\par “‘噢,是的。’
\par “‘他跟您谈月光。’
\par “‘对,您怎么知道呢?’
\par “‘等一下,难道他没有对您说……’于是父亲背出那个句子。
\par “‘对对,可这是怎么个秘密呢?’
\par “‘他甚至还与您谈到罗马乡间的月光。’
\par “‘您简直是巫神嘛!’
\par “我父亲并不是巫神,而是德·夏多布里昂先生不论对谁都上那一盘现成菜。”
\par 听到维尼的名字,她笑起来。
\par “就是那个总说:‘我是阿尔弗莱德·德·维尼伯爵’的人。是伯爵也好,不是伯爵也好,这丝毫无关紧要嘛!”
\par 说不定她认为还是多少有点紧要的,因为她接着这样说下去:
\par “首先,我不敢肯定他就是伯爵。不论怎么说,他出身很寒微,这位先生在他的诗里曾提到他的‘绅士顶饰’。\footnote{引自诗作《思想纯正》。}对于读者来说,这格调多么高雅,多么有趣!这就像缪塞身为巴黎的普通市民而大肆夸张地说什么‘武装我帽子的金雀鹰’\footnote{引自诗作《致阿尔弗莱德·达戴先生》。}一样。一个真正的贵族大老爷从来不说这类的话。不过,至少缪塞作为诗人还是有才华的。可是德·维尼先生,除了他的《圣克马尔斯》以外,别的作品,我从来就一点也看不进去,枯燥无味会叫书从我手里掉下去。莫莱先生既有风趣又很机灵,而德·维尼却没有,莫莱让他进了法兰西学院可把他安排得够好的。怎么,你没有读过他的演说?那可是狡诈和狂妄的杰作!”
\par 她见自己的侄儿们钦佩巴尔扎克大为惊讶,她责备巴尔扎克宣称自己描绘了“他被拒之门外”的社会,对这个社会他讲述了大量不可靠的事情。至于维克多·雨果嘛,她对我们说,她父亲德·布永先生在浪漫主义青年派里面有几个伙伴,借助于他们的帮助,《埃那尼》首演式\footnote{《埃那尼》于1830年2月25日在法兰西剧院首次演出,成为著名的古典派与浪漫派征战战场。}时他进去了。但是他未能坚持到底,他觉得这位聪明但过分夸张的作家的那些诗句太可笑了。他得到伟大诗人的头衔只不过是一笔谈好的生意,是对他针对社会主义者危险的胡言乱语鼓吹出于利害关系加以容忍而给他的报酬。
\par 我们已经远远望见旅馆了。刚到的第一天晚上那充满敌意的灯火,现在变成了具有保护性的柔和灯光,成了家园指示灯。待马车到达大门附近时,门房、青年侍者、开电梯的,表现出殷勤、天真,对我们晚归已隐隐约约感到不安,已聚集在台阶上等待着我们。他们变得很亲切。他们属于那种在我们生命过程中要变多少次的人,正像我们自己也在变一样。但是,在某个时期内,他们是我们司空见惯的事物的镜子,这时,我们从他身上找到了亲切感,感到我们自己得到了忠实的、友好的反映。我们喜欢他们更甚于喜欢某些久未见面的朋友,因为他们身上,更多地包含着我们当前的状况。只有那个穿着制服的仆役例外。白天他风吹日晒,现在为了不再忍受夜间的寒冷,已将他移进室内,并以呢绒裹身。再加上他那橘红色的头皮和双颊上那奇粉的花朵,在玻璃大厅中间,不禁使人想到作防寒保护的一棵温室植物。
\par 我们在仆役帮助下下了车。其实用不着那么多人,但是他们感到这场面很重要,自认为必须在里面扮演一个角色。我饥肠辘辘。为了不推迟用晚餐的时间,我常常不回房间。这房间最后也变成真正属于我了,以致重见那紫色的大窗帘和低矮的书架,就等于与自己单独相逢。物品也和人一样,向我提供了自己的形象。我们一起在大厅里等候,等候着侍应部领班来向我们报告晚餐已备好。这时,又是我们听德·维尔巴里西斯夫人讲话的机会。
\par “我们借您的光了。”外祖母说。
\par “说哪儿去了!我真开心,这真叫我心花怒放。”外祖母的女友带着顽皮的微笑回答,拖着长腔,语调优美动听,与平时的纯朴自然形成鲜明对照。
\par 在这种时刻,她确实很不自然,她想起自己所受的教育,想起一位贵妇人在她高兴与之相处的布尔乔亚面前应该表现出什么样的贵族风度。她并不狂妄,而她身上唯一真正礼节不周的地方,正是她过分客套。因为人们从这种过分的客套中辨认出圣日耳曼区贵妇人职业性的习惯。在她眼中,某些资产阶级总是有不满情绪的人,某些时候,她也注定要装成不满的样子。在与这些人热情相处的账上,她贪婪地利用尽可能的一切机会,将贷方的钱数早早支出去,这样,就使她可以在今后将她不邀请这些人出席的晚宴或盛大晚会记入她的借方。她那个社会阶层的天才从前已经对她发生了一劳永逸的影响,但是她不知道现在情形已经不同,对象已经不同。她希望以后在巴黎经常在她家中见到我们,而特许给她的可以热情待人的时间又很短,所以她那个社会阶层的天才狂热地推动着她,在我们在巴尔贝克逗留期间,经常派人给我们送来玫瑰花和甜瓜,借给我们书籍,与我们坐马车出游以及与我们长谈。正因为如此,正如海滩那令人头晕目眩的美景,旅馆房间里色彩斑斓的灯火和如同大洋深处的光线,将小商贩的儿子奉为亚历山大·德·玛塞多瓦纳一样神奇的骑师一样,德·维尔巴里西斯夫人每日的殷勤相待,加上我外祖母接受这些殷勤相待的那种暂时的、夏季的随和,这一切都作为洗海水浴这一段生活的特征留在我的回忆中。
\par “把你们的外套交给他们,叫他们送上楼去!”
\par 外祖母将外套交给经理。他好像对这种不尊敬感到难过。他对我一向很和蔼热情,我为此心里很不好过。
\par “我看这位先生是不高兴了,”侯爵夫人说,“他肯定自以为是大老爷而不能给您拿披巾。我还记得德·纳穆尔公爵\footnote{这里可能是指路易·夏尔·菲利浦·德·奥尔良,路易菲利浦的次子。}的故事。那时候我还很小,我父亲住在布永公馆最高一层。纳穆尔公爵走进我父亲的房间,胳膊底下夹着一大包东西、信件和报纸。从我家那有漂亮木雕的房门框框里,我觉得眼前出现的是身着蓝色礼服的王子。我以为那是巴加\footnote{巴加(1639—1709),法国雕刻家,同时代人称他为“伟大的凯撒”。有时他也搞木雕。}的手艺,您知道的,那些细木匠有时用很精巧的木棍做成小船,就像用缎带包扎花束一样。
\par “‘给你,西律斯,’他对我父亲说,‘这是你的门房让我交给你的。’他说:‘既然您要到伯爵先生那里去,我就不用上好几层楼了。不过,当心,别把捆信报的绳子弄坏了!’好,现在既然您已经把外衣交给人了,请坐吧,来,坐这。”她拉着外祖母的手对她说。
\par “噢,如果哪里对您都一样,我就不坐这张沙发了!两个人坐太小,我一个人坐又太大,我会不自在的。”
\par “噢,您说这话,倒叫我想起一张沙发,完全是一样的。那是很久以前人家让我坐的一张沙发,但我最后还是没能坐成,因为那是可怜的德·普拉斯兰公爵夫人送给我母亲的。我母亲其实是世界上最单纯的人,可是她还有些老年头的思想,我已经不大理解。她刚开始不愿意让人将她介绍给德·普拉斯兰夫人,因为这位太太做闺女时,不过是塞巴斯蒂安尼小姐\footnote{指这位太太并非贵族家庭出身。}。而这位小姐呢,因为自己已经成了公爵夫人,就认为不应该自己主动叫人介绍给别人。而事实上,”德·维尔巴里西斯夫人又加了一句,忘了她对这些细微的差别并不大懂行,“如果她是德·舒瓦瑟尔夫人,她那种雄心也许还能站得住脚。舒瓦瑟尔家族是最伟大的家族,他们是胖子路易国王的一位妹妹的后代,他们是巴希尼真正的君主。\footnote{舒瓦瑟尔家族在巴希尼扎根可上溯到十世纪末期。他们与于格·德·香巴涅伯爵是亲戚,这位伯爵的妻子是法国国王路易六世(1108—1137,人称胖子路易)的姐妹贡斯唐丝。}我承认,从姻亲和知名方面说,我们家占上风,但若论家族的古老,那几乎是一样的。这个谁先谁后的问题产生了一些很可笑的事端,诸如有一次午宴晚开一个多小时,就是因为有一位贵妇人争了这么长时间才同意让人将她介绍给对方。虽然如此,我母亲和德·普拉斯兰公爵夫人还是成了非常要好的朋友,公爵夫人让我母亲坐一张这种式样的沙发。就像您刚才这样,谁都拒绝坐。
\par “有一天,我母亲听见一辆马车进了公馆的院子。她问一个小仆人是谁来了。
\par “‘是德·拉罗什富科公爵夫人,伯爵夫人。’
\par “‘啊,好的,我就见她。’
\par “过了一刻钟,不见人。
\par “‘喂,怎么回事,德·拉罗什富科公爵夫人呢?她在哪儿?’
\par “‘她在楼梯上喘气呢,伯爵夫人。’小仆人回答道。这个小仆人刚从乡下来到不久。我母亲有个好习惯,就是到乡下去雇人,她常常是看着他们生下来的。这样家里就有非常老实可靠的用人,这也是最高级的奢华。果然,德·拉罗什富科公爵夫人上楼艰难,因为她异常肥硕,以至她走进门来时,我母亲一时焦急不安起来,心想可让她往哪儿坐呢?就在这时,德·普拉斯兰太太送的这件家具在她眼前一闪:
\par “‘请坐。’我母亲说,将沙发向她跟前一推。
\par “公爵夫人于是坐满了这张沙发,一直满到边边上。这位太太,虽然这么……肥,可一直相当令人愉快。
\par “‘她走进来时依然会产生某种戏剧性效果。’我们的一位朋友说。
\par “‘走出去时尤甚。’我母亲回答。她的词儿来得很快,可如今这么说可就不大合适了。
\par “在德·拉罗什富科夫人自己家里,人们在她面前随便开玩笑,她本人首先对自己比例太大说上几句笑话。
\par “‘怎么,您一个人在家吗?’一天,我母亲前去拜访公爵夫人,可是在进门处却受到她丈夫的接待。妻子在里头窗口那里,我母亲没有看见,便这样开口向德·拉罗什富科先生发问,‘德·拉罗什富科夫人不在吗?我怎么看不见她呢!’
\par “‘您真是太客气了!’公爵回答说,他这是作出了我从未见过的最错误的判断,但是倒不乏风趣。”
\par 用毕晚饭,我与外祖母上楼以后,我对她说,德·维尔巴里西斯夫人使我们着迷的那些长处,机灵,周到,谨慎,不炫耀自己,说不定并不那么稀罕,因为最高程度拥有这些优点的人只不过是莫莱·洛梅尼这样的人。虽然没有这些长处会使日常相处不愉快,这倒不妨碍成为夏多布里昂、维尼、雨果、巴尔扎克。一些没有判断能力、爱虚荣的人,像布洛克这样的倒很容易嘲笑他们……一听到布洛克的名字,我的外祖母便大叫起来。于是她大肆吹捧德·维尔巴里西斯夫人。正如人们常说的那样,在爱情上,人各有一好,由人种的利害来主导。为了使生下的孩子构造最正常,要叫胖男人找瘦女人,瘦男人找胖女人。同样,神经过敏,多愁善感,孤僻自傲的病态倾向威胁着我的幸福。而我的幸福顽固地要求外祖母将稳健和有判断能力这样的优点放在首位。这不仅是德·维尔巴里西斯夫人所特有的品质,而且也是我能在其中找到消遣和满足的整个上流社会的品质。这个社会与杜当\footnote{杜当(1800—1872),文学评论家,政治家,据说不擅在大庭广众之下演讲,小圈子集会时则口若悬河。}、德·雷米萨\footnote{雷米萨(1797—1875),1840年曾加入梯也尔内阁任内政大臣。1847年反对基佐,1848年站在共和国一边。1851年路易拿破仑·波拿巴政变后,他被放逐,1859年才回到法国。1871年,梯也尔任命他当外交大臣。}这样的人物思想大放光华的社会很相像,至于博泽让夫人、儒贝\footnote{儒贝(1754—1824),伦理学家。}、塞维尼夫人这样的人自然更不用提了。这种思想比起与之相对的精华来,在生活中注入了更多的幸福和尊严。与之相对的精华则将波德莱尔、埃伦·坡、魏尔兰、兰波这样的人引向痛苦,不受尊敬。我的外祖母可不愿意她的孙子这样。我打断她的话,亲了她一下,然后问她是否注意到德·维尔巴里西斯夫人说的哪句哪句话,那句话表现出她这个人实际上比她自己所承认的更看重自己的出身。我就这样把我的印象全都掏给外祖母,因为只有她的指点,我才知道对某某人应该尊敬到什么程度。每天晚上,我便将白日里根据除她以外的所有这些不存在的人物所作的速写像呈现在她面前。
\par 有一次我对她说:“没有你,我将无法生活。”
\par “不应该这样!”她语气慌乱地回答我说,“心要更硬点。不然,如果我出门在外,你怎么办呢?相反,我出门去了,希望你能很讲道理,高高兴兴。”
\par “你如果出门几天,我能做到很讲道理,可我一定度日如年。”
\par “那我若是出门几个月呢……(一想到这,我的心就揪得紧紧的)几年呢……甚至……”
\par 我们两个人都默默无语。谁也不敢看谁。不过,我为她的焦虑而感到难过,更甚于因自己的焦虑而感到痛苦。我走近窗户,眼睛不望她,一字一顿地对她说:
\par “我是一个多么注重习惯的人,你是知道的。刚刚把我与我最热爱的人分开的头几天,我很难过。可是我慢慢会习惯,虽然我还和从前一样热爱他们,但是我的生活变得平静了,温和了。将我与他们分开几个月,几年,也许我受得了……”
\par 我说到这里,不得不住了嘴,完全向窗外望去。我的外祖母从房间出去了一会。
\par 第二天,我谈起了哲学,用的是完全无动于衷的口气,但是安排得很好,让外祖母注意到我说的话。我说,真是怪,科学上有了最新的发现以后,唯物主义似乎破产了,而更有可能的仍然是灵魂永在以及它们未来的相聚。
\par 德·维尔巴里西斯夫人已预先告诉我们,过不久她就不能这样经常与我们见面了。她有一个侄孙,现在正在附近的东锡埃尔驻防,他正在准备报考索穆尔军校,要到她身边来度几周的假期,到那时她的许多时间都要给她侄孙了。在我们出游过程中,她在我们面前大肆吹嘘这个侄孙绝顶聪明,心地是特别善良。我心里已经设想他会对我产生热情,我将是他的挚友。待他来到之前,他的婶祖母在我外祖母面前透露出:可怜他落到了一个他为之神魂颠倒的坏女人手里,那个女人紧抓住他不放。我早就确信,这种爱情,注定最后要以发疯、杀人和自杀来结束。想到留给我们友谊的时光这样短暂,虽然我还没见过他,这友谊在我心中已经那样伟大,我为这友谊和为等待着他的不幸而大哭一场,好像一个亲爱的人,人家刚刚告诉我们他已身患重病、将不久于人世,我们也为他痛哭一样。
\par 一个酷热的下午,我待在餐厅里。为挡住阳光,已经放下了被太阳晒黄的窗帘,餐厅沉浸在半明半暗之中。透过窗帘的缝隙,碧蓝的大海在闪烁。这时,我看见在海滩与大路的中间,一个小伙子走过,高个,瘦削,颈部外伸,高傲地扬着头,目光敏锐,皮肤和头发像吸收了所有的阳光一样金黄。他的衣料薄而发白,我从来就没想到一位男子敢穿这样的料子。他那瘦削的身材更使人想起餐厅的凉爽以及外面的炎热和大好天气。他健步如飞。他的眼珠与大海同样颜色,一只单片眼镜总是从一侧眼睛上掉下来。每个人都好奇地望着他走过,人们知道这位年轻的圣卢昂布雷侯爵是以衣着华丽而著名的。他最近在一次决斗中为年轻的德·于塞斯侯爵作证人时穿的那身礼服,每一家报纸都描写过。他的头发、眼睛、皮肤、举止所特有的长处,使他在人群中,如同稀有的天蓝色而又熠熠生辉的蛋白石矿脉隐藏在粗糙的物质中一样,立刻显现出来。与这一切相对应的生活,大概与他人生活截然不同吧?因此,在德·维尔巴里西斯夫人所抱怨的那场暧昧关系发生之前,当上流社会最标致的女人们都在相互争夺他的时候,假如他伴着自己追求的著名美人在一处沙滩上出现,那不仅要使这个美人成为明星,而且要引来多少目光注视着他,也注视着她!由于他“时髦”,像幼“狮”般的狂傲,主要还是由于他非同寻常的美,某些人甚至觉得他的神情有些女性化,但并不以此相责,因为他多么健壮,他怎样狂热地追求女性,是尽人皆知的事。德·维尔巴里西斯夫人与我们谈起的,就是这个侄孙。
\par 想到就要在几个星期中与他相识,我真是心花怒放,而且我确信,他会将全部疼爱都倾注在我的身上。他飞快地横穿过旅馆,似乎追逐着他的单只眼镜,那眼镜在他身前像蝴蝶一样飞舞。他从海滩上来,将大厅玻璃窗浸到半身高的大海,为他构成了一个背景。他全身从这个背景上突出出来,就像在某些肖像画上,一些画家在极准确观察当前生活上一点不掺假,为他们的模特儿选择一个合适的环境,马球草坪啊,高尔夫球草坪啊,赛马场啊,游艇甲板啊,认为这样便赋予了这些画幅一种当代等同物,而那些原始的画家则叫人像出现在一处风景的近景上。
\par 一辆两匹马驾的车在旅馆门口等待着他。待他的单眼镜又在阳光普照的路上蹦蹦跳跳玩耍起来时,姿态的优美与动作的娴熟,就像一位伟大的钢琴家在最简单的一触琴键之中找到了办法,表现出他就是比一个二流演奏家高出一头一样,而表面看上去,从这最简单的一触琴键中是不可能表现出这么多东西的。德·维尔巴里西斯夫人的侄孙这时接过车夫递过来的缰绳,坐在车夫身旁,一边将旅馆经理交给他的一封信拆开,一边叫牲口起步。
\par 此后的日子,每当我在旅馆内或旅馆外与他相遇时——他衣领高高,单只眼镜转瞬即逝跳来跳去,似乎是他四肢的重心,他总是围绕着单只眼镜来平衡四肢的动作——我都可以意识到,他根本不想接近我们。我也看到他不和我们打招呼,虽然他不会不知道我们是她婶祖母的朋友!我感到多么失望啊!我忆起德·维尔巴里西斯夫人,还有在她身边之间的德·诺布瓦先生,对我那样和蔼可亲,便想道,可能他们只是一些可笑的贵族,而且统辖贵族阶级的法律中可能有一个秘密条款,允许女子和某些外交家在与凡人的接触中(因为什么原因我不得而知),可以不表现出傲慢。相反,一位年轻的侯爵则必须铁面无情地表现出傲慢来。
\par 我的智慧本来可以告诉我,事实正好与此相反。可是我正经历着可笑的年龄——绝不是什么都不懂,而是十分多产的年龄,这个年龄的特点就是不去向智慧讨教,而且认为人的每一种属性似乎都是他们人格不可分割的一部分。周围全是魔鬼和神,简直不得安宁。那时的一举一动,几乎没有一件是以后希望能够忘掉的。相反,应该遗憾的是,当时使我们做出那一举一动的那种自然,发自内心,以后却没有了。以后看问题更实在了,完全与社会的其他部分相符合了,但是,少年时期是唯一学到东西的时期。
\par 我猜测到的德·圣卢先生的傲慢以及这种傲慢所包含的铁石心肠,从每次他从我们身边走过时那种态度上都得到了证实:身体修长而不能弯曲,头部总是高昂着,目光毫无表情。光说毫无表情还不够,还恶狠狠的,完全没有一般人那种对他人权利的隐隐尊重,即使这些人不认识你的婶祖母。正是这种对他人权利的隐隐尊重使我在一位老妇人面前和在一盏煤气路灯前行为不一样。前几天我还设想他会给我写几封十分讨人喜欢的信,以向我表示好感;一个善于想象的人自称代表民众,正在用令人难忘的演说鼓动民众,待他这样一个人高声道出他的梦幻,想象的欢呼声一旦平息下去,他就和以前一样还是一个大傻瓜,依然平平庸庸,默默无闻,距离议会与民众的热情很远。这位公子那冷冰冰的姿态,与上述那想象的来信相距十万八千里,与上述那议会与民众的热情亦相距十万八千里。
\par 那个秉性傲慢而又心怀恶意的人,那些很说明问题的外表在我们心中产生了极坏的印象,大概是为了尽力消除这种坏印象,德·维尔巴里西斯夫人又与我们谈起她的侄孙(他是德·维尔巴里西斯夫人一个侄女的儿子,比我年龄稍大)心地无限善良。世界上竟有人能够不顾一切事实真相,将好心肠这一优秀品质借给心肠那么硬的人,哪怕他们对组成自己那个圈子的有名气的人彬彬有礼也好!对这一点,我算服了!有一天,我在一条窄路上与他们二人相逢,她没有别的办法,只能将我介绍给他。这一次,德·维尔巴里西斯夫人本人,虽然是间接地,倒给她侄孙天性的基本特点加上了一点肯定成分。我对那些基本特点已经确信无疑。
\par 他似乎没有听见人家在他面前道出某一个人的名字,面部肌肉没有一块动弹一下。他的眼睛里,没有一丝最微弱的人类好感之光闪过,从目光的无知与空虚之中,只流露出一种过分的夸张。若没有这一点,他的眼睛可就与没有生命的镜子完全无异了。然后,那冷酷的眼睛盯住我,似乎向我答礼之前,想了解了解我的情况。那种突然的发动,与其说来自有意的行为,还不如说来自肌肉反射更恰当一些。他在自己和我之间留出尽量大的距离,将整个手臂伸出来,远远地向我伸过手来。
\par 第二天他差人将他的名片送给我,我以为至少要有一场决斗。可是,他只与我谈文学,谈了很久之后,他声称非常希望每天能见到我几个小时。但在这次拜访过程中,他对精神方面的事情并没有表现出热烈的兴趣。他对我表示的好感与前一天的答礼也大相径庭。待我后来见到每当人家向他介绍某个人,他都是这个样子时,我明白了,这不过是他那个家族中某一部分人特有的社交习惯。他母亲十分看重他要非常有教养这一点,要求他的躯体服从这一习惯。他这样施礼,是并不考虑的,并不比想到他的漂亮衣服、他的漂亮头发想得更多。这是从思想上来说什么也说明不了的一件事,纯粹是学来的,而我首先认为说明问题的,正如他的另一个习惯一样:他认识了谁,立刻要人家将他介绍给此人的亲属。这个习惯,在他已经变成本能性的了,所以第二天我们遇到的时候,他一见了我,就朝我冲过来,连好也没问,便要求我向身边的外祖母通报他的名字。那种狂热的速度,似乎这要求是来自某种自己的本能,正像挡住迎面一击那个动作,或热水喷过来赶紧闭上双眼一样,不采取这样的防护措施,再过一秒钟停住不动,就会有生命危险。
\par 这第一轮驱魔咒仪式一旦完成,就像怒气冲冲的女妖剥下她的第一层外衣,用迷人的风韵将自己装饰起来一般,我见过的这个傲慢的尤物变成了我遇到过的最可亲可爱的人,最殷勤体贴的小伙子。
\par “好啦,”我心想,“我对他已经看错了,我受了海市蜃楼的害。可是,我不过胜了第一个马上就要落到第二个手里而已,因为他是一个迷恋贵族阶级的大老爷,他又要极力掩盖自己的真相了。”果然,圣卢所受的全部良好教育,他的全部可爱可亲,不久之后,便叫我见识了另一个人,而与我怀疑的很不相同。
\par 这个外表上是个傲慢的贵族和运动员的小伙子,只对精神方面的事情看得重、有兴趣,特别是对文学和艺术上的时髦表现十分有兴趣,这在他婶祖母看来,似乎是那么可笑。此外,他满脑子都是她婶祖母称之为“社会主义演说”的玩艺,对他自己的阶层充满了深深的蔑视,经常花几小时研究尼采和普鲁东。他是很快便佩服人家钻在一本书里,只关心抽象思维的那种“知识分子”“知识分子”这种用法,在当时还是新词。这种倾向非常抽象地表达出来,使他与我平常操心的事情距离很大,甚至就在他进行这样表述的时候,虽然我觉得很能打动人,可是也叫我有些厌倦。我可以说,我刚刚读了关于著名的德·马桑特伯爵那充满轶事的回忆录之后那些日子里,当我确实知道了这马桑特伯爵就是他的父亲以后,我特别希望对德·马桑特先生过去的生活知道得更准确、更详细一些。想到罗贝·德·圣卢不但不满足于做他父亲的儿子,不但不能将我引进他父亲的一生这部过时的小说中去,反而培养自己去热爱尼采和普鲁东,我真是气得要发疯。在马桑特伯爵身上,一个已经遥远时代那样特别的风雅与充满幻想的精神合二而一了。他的父亲说不定不会赞同我的遗憾。他本人是一个聪明人,越出了他那个花花公子生活的界限。他几乎没有来得及了解他的儿子,但他希望儿子比自己有出息。我相信他可能与家族中其他人相反,会赞赏他的儿子,会为儿子将构成父亲从前可怜的消遣的东西抛在一边去进行严肃的思考而感到高兴。他会不露声色地,怀着他那伟大神师的谦虚精神,去偷偷阅读儿子最喜爱的著作,以估计一下罗贝比他高明多少。
\par 再说,还有一件令人伤心的事,就是虽然德·马桑特先生心胸很开阔,会欣赏与自己那么不同的儿子,但是罗贝·德·圣卢是相信品德与某些艺术形式和生活方式相联系的人,他对自己的父亲怀着虽说充满感情却又有些蔑视的记忆,他的父亲一辈子就是关心打猎、赛马,听瓦格纳的曲子要打哈欠,对奥芬巴赫却非常着迷。圣卢还不够聪明,他不懂得智力价值与附和某种美学模式毫无关系,他对德·马桑特的“智慧”看不起,同布瓦尔迪欧的儿子对布瓦尔迪欧、拉比什的儿子对拉比什可能会看不起一样,因为这些儿子如果是最象征主义文学和最复杂的音乐的信徒,就必然会看不起自己的父亲。
\par “我对父亲了解很少,”罗贝常说,“据说他是一位很杰出的人。他的不幸就在于他生活在那个可悲的时代。出生在圣日耳曼区,生活在‘美女海伦’的时代,这就造成了一生中的灾难。如果他是热衷于‘Ring’\footnote{德文:戒指,此处是指瓦格纳的四部曲《尼布隆根的戒指》。}的小资产者,说不定还能做出完全不同的事情来。人家甚至告诉我,说他很喜爱文学。无法知道究竟,因为他所理解的文学,完全由过时的作品组成。”
\par 对我来说,我觉得圣卢有些严肃,而他则不理解我并不比他更严肃。他判断每一事物,只凭这事物所包含的智慧有多重,某些事物赋予我美妙的想象,他体会不到,而认为这些事物很肤浅。他自认为我比他逊色得多,可是我能够对这些事情感兴趣,他很惊异。
\par 头几天,圣卢就征服了我的外祖母。不仅通过他巧妙地向我们两人表现出无时无刻的好意,而且在好意上又加上自然,他在各种事情上均是如此。自然——大概是因为透过待人接物的艺术,他叫人感觉到自然——这是我外祖母看得最重的优点,无论是在花园里,还是在烹调上,还是在钢琴演奏上,都是如此。在花园里,例如在贡布雷的花园里,她不喜欢有特别整齐的花坛;在烹调上,她讨厌所谓的“拼花样”,那种几乎辨认不出是用什么东西做出来的食品;在钢琴演奏上,她不喜欢过分雕琢,加工过细,她甚至对鲁宾斯坦\footnote{鲁宾斯坦(1829—1894),俄国钢琴家,作曲家。}弹琴音符不清、走调都有一种特殊的好感。这种自然,她甚至从圣卢的衣着上体会出来,是轻松的华丽,无任何“装腔作势”以及“拘泥、刻板”,不僵硬,也不上浆。她更欣赏这个富有的年轻人那股毫不在乎、自由自在的劲,生活在奢华之中却没有“铜钱臭”,不摆阔架子。圣卢依然无法阻止自己的面部透露出某种激情,她甚至从这上面也找到这种自然的动人之处。一般来说,随着童年的逝去,那个年龄的某些生理特点一起消失了。例如他热切地期望着什么,而又没有指望得到,哪怕是一句恭维话,都会使他迸发出那种骤然、火热、有感染力而又外露的快乐,他无法控制,也无法掩饰。快活的怪相无可阻挡地飞上他的面庞,双颊细腻的皮肤透出红晕,双眼映出羞涩和快乐。对这种直爽和天真无邪的优美表露,我外祖母无限感动。这种表情,在圣卢身上,至少在我与他友情甚笃的时代,是不骗人的。
\par 我认识另一个人——这样的人很多——对这个人来说,那种来得快去得快的红晕所表现出的生理上的诚恳,丝毫不排除道德上的表里不一。这种红晕,常常只证明一些足以干出最卑鄙、奸诈行为的人感到高兴的强烈程度,他们甚至在快乐面前不能自持,不得不向别人承认这种快乐。使我外祖母特别酷爱圣卢的原因,自然是他那样毫不拐弯抹角地承认他对我怀着好感。为了表达这种好感,他用的那些词语,我外祖母说,似乎连她自己也找不到,是最准确的,真正动情的,是同时属于“塞维尼和博泽让”的词语。他也毫无拘束地拿我的毛病开玩笑——他挑我的毛病那种细心劲,叫我外祖母觉得好玩——但也像我外祖母一样,是满怀柔情的。相反,他热情地、毫无保留地、毫不冷淡地尽情赞扬我的优点,而他那个年龄的年轻人一般认为,非要借助于保留和冷淡才能显出自己了不起。我稍感不适,他就去叫人来;天气转凉,我自己还没发觉,他已经把毯子盖在了我的腿上;若是感到我很忧郁或者不快活,他便不声不响地安排好,晚上陪我陪得更晚。他表现出那样的细心周到,从我健康的角度来说,更严酷一些对我说不定更有好处。我外祖母觉得这几乎有些过分,但是,作为对我疼爱的表示,她深深地受到感动。
\par 我们两人很快就说好了:我们已经成了永不相弃的挚友。他说“我们的友谊”时,就好像谈一件什么存在于我们身外的重要而甜美的事情一般,而且很快他便将“我们的友谊”称之为他生活中最大的快乐了——对他情妇的爱不计在内。这些话引起我某种感伤,我很为难,不知如何作答,因为和他在一起,和他谈话——肯定,与任何别的人也是如此——我丝毫感觉不到没有人陪伴时反而会感觉到的那种幸福。独自一人的时候,有时我感到有一种感觉从内心深处涌来,是那种给我以甜美的快意的感觉。但是,我一跟什么人在一起,一跟一位朋友谈话,我的思想就来了个一百八十度大转弯,思考朝着谈话对象而去,而不是朝我自己而来了。思考循着这样的反方向而去时,丝毫不能引起我的快乐。我一离开圣卢,便借助于语句,将我与他一起度过的纷乱的每一分钟理出点头绪来。我心里想,我有一个好朋友,一个好朋友是罕见的,我感到周围皆是难以到手的财富,这时我恰恰体会到与对我来说实为自然的快乐相反的东西,与从我内心汲取了什么,并将这个隐藏于半明半暗之中的念头置于光天化日之下而体会的快乐相反。如果我花上两三个小时与罗贝·德·圣卢聊天,他对我对他说的话又很赞赏,我便感到某种后悔、遗憾、厌倦,觉得不如一个人独处及准备好开始工作。但是我心里又想,一个人聪明并不仅仅为了自己,最伟大的人物也期望为人欣赏,我不能将这几个小时视为浪费,在这几个小时的过程中,我在朋友的心目中建立起了自己高大的形象。我很容易地说服了自己,认为应该为此而感到高兴,正因为我不曾体会到这种幸福,我更热切地期望永远不要剥夺我这种幸福。对于我们身外的财富,人们总是比担心所有其他的财富更担心这些财富消失,因为我们的心没有占有这些财富。
\par 我感到自己能够比很多人更好地体现友谊的美德(因为我总是将朋友的利害放在所谓个人利益之上,我对这些个人利益是不在乎的,而其他人对这个极为关切)。但是感到我的心灵与他人心灵之间的差异——我们每个人心灵之间都是有差异的——不但没有扩大,反而会消失,我却无法因此而感到快乐。相反,有时,我的思想从圣卢身上辨别出一个比他本人更普通的一个人——“贵族”,而且就像一种内在的精神指挥着他四肢的动作一样,是这个“贵族”在指挥着他的一举一动。这时候,虽然我在他身旁,实际上我是独自一人,我在他面前好似我面对一处风景,理解了这景色的和谐一样。他只不过是一件物品罢了,我的思考力图加深对这件物品的认识。我总是从他身上找到那个先入为主的、上百岁的人,那个恰巧是罗贝期望自己不是的贵族,这时我感到极度的快乐,但属于智力范畴,而不属于友谊范围。
\par 他身心机敏,赋予他的是无限可亲可爱的风雅;他很随便地请外祖母坐他的马车,并且扶她上车;他怕我着凉,灵巧地从座位上跳下来,将他自己的外套披在我的肩上。从这些举动里,我感觉到的,不仅是伟大的猎手世代相传的灵巧——这个年轻人的祖先世世代代就是猎手,而他却一心要搞智力活动,还有他们对富有的蔑视——在罗贝身上,也有这种对富有的蔑视——但同时他又对富有很有兴味,那只是为了能够更好地欢宴他的友人,正是这种蔑视才使他那样漫不经心地将自己的奢华奉献于友人的脚下。从这些举动里,我更感觉到这些贵族大老爷那种认为自己“高人一头”的自信或幻觉。幸亏如此,他们未能将那种想表现自己“与别人一样”的欲望遗传给圣卢,未能将那种怕显得过分殷勤的恐惧遗传给圣卢。圣卢确实不知这种恐惧为何物,而这种恐惧以其僵硬和笨拙,使最诚挚的平民百姓的和蔼可亲都变成了丑态。
\par 有时我责备自己这样从视自己的朋友为一件艺术品中得到乐趣,也就是说,注视着他这个人各个部分的动作,似乎由一个总思想和谐地加以指引,这每一部分都拴在那个总思想上,而他自己并不知道这个总思想是什么。因此,这个总思想并不能给他自己的品质、给他个人的智慧和道德的价值增加任何一点东西,而他对这些是看得很重的。
\par 然而,在某种程度上,这个总思想倒是他的品质得以存在的条件。正因为他是一个贵族,他的思想活动,他对社会主义的向往,在他身上才具有某种真正纯洁和无私的色彩。这种活动和向往使他去寻找一些野心勃勃、衣衫破旧的年轻大学生,那些人的活动和向往并不具有纯洁和无私的色彩。他认为自己是一个无知而又自私的社会阶层的继承人,坦诚地希望大学生们原谅他这些贵族根底。事实与此相反,正是这些贵族根底对大学生产生诱惑力,正因为如此,他们才找他,同时又对他装出冷淡甚至傲慢的样子。
\par 他就这样弄到要向一些人主动追求的地步。我的父母忠于贡布雷的社会学,见他这样对这些人并不扭头而去,一定会惊诧不已的。
\par 有一天,我和圣卢坐在沙滩上,背靠一顶帆布帐篷。我们听见从帐篷里传出咒骂,嫌巴尔贝克犹太人麇集,把巴尔贝克都弄臭了。
\par “就没法走上几步不碰上一个!”那声音说道。“我并非从什么原则出发,对犹太民族有不共戴天的仇视情绪,可是这里,真是过剩了!就听见:‘喂,亚伯拉罕,chai fu Chakop\footnote{希伯莱语:你这个断子绝孙的。}’这种话。真觉得自己是置身于阿布吉尔街呢!”
\par 如此大发雷霆反对以色列的那个人终于从帐篷里走出来了。我们抬起头来看看这个排犹主义者。他正是我的伙伴布洛克。圣卢立即请我提醒布洛克,说他们在大考时遇见过,布洛克那次大考得到荣誉奖,后来他们在一所民众大学里又遇见过。圣卢的哪位志同道合的朋友在交际场合出了差错,做了可笑的事,圣卢对这个毫不在乎。但是他感到,如果别人发现了,那出了错的人是会脸红的。每逢这时,怕伤害别人的自尊心便使他现出一幅窘态。这种时候常常是圣卢满脸通红,似乎出错的是他。从他的窘态中,我能找到他受耶稣教会教士教育的痕迹,对此我最多偶尔讥笑一下也就罢了。布洛克答应到旅馆去看他那天,情形就是如此。布洛克一面应允,一面又加上一句:
\par “在那种供商队住宿的大旅店伪装时髦地等人,我受不了;茨冈女人又叫我恶心,你对‘laift’\footnote{布洛克出于无知,将“lift”(开电梯的人)读成“laift”。}说,叫她们住嘴,并且立即去通知你!”
\par 从我个人来说,我并不很坚持叫布洛克到旅馆来。他在巴尔贝克并不是独自一人,而是和他的姐妹们在一起,可惜!他的姐妹们在这里又有许多亲戚朋友。这个犹太群体很有特色,并不太令人愉快。巴尔贝克和某些国家,如俄国和罗马尼亚一样,地理课教给我们,在这些地方,犹太居民并不享有与巴黎同等的优惠,也不像在巴黎那样达到了那种程度的同化。布洛克的表姐妹和叔伯们,或者与他信仰同一宗教的男男女女上游乐场时,女的是去“舞厅”,男的则上了岔路到纸牌赌博那边去。他们总是一块去,不与任何其他成分混杂。他们组成一个与自身同质的队伍,与注视他们走过,每年在这里看见他们却从来不和他们打招呼的人是完全不同的一帮。不论是康布尔梅的圈子,首席审判官的山头,还是大小资产者,甚至巴黎某些普普通通的杂粮商人,他们的女儿,美貌,傲慢,嘲笑一切,完全法国式,就像兰斯的雕像一样,都不肯与这群没有教养的丫头混在一块。她们念念不忘“洗海水浴”这种时髦,甚至总作出刚刚钓大虾回来或正在跳探戈的模样。说到男子,虽然无尾礼服光鲜夺目,皮鞋溜光铮亮,但是举止装腔作势,使人想到画家那些所谓“聪明的”讲究:他们要给福音书或《一千零一夜》作插图,考虑到那些事情发生在什么国度里,偏偏把巴尔贝克最大腹便便的“大人物”的模样赋予了圣皮埃尔或阿里巴巴。
\par 布洛克一一将他的姊妹向我作了介绍,粗暴得无以复加地叫这些女孩子住嘴。她们对这个哥哥崇拜备至,将他看成自己的偶像,他每说出一句什么俏皮话,她们都要哄堂大笑。所以,很可能这个阶层也与任何其他阶层一样蕴含着许多吸引人之处、优秀品质和崇高道德。要体会到这些,则必须深入到这个阶层中间去。可是,这个阶层不讨人喜欢,他们感受到排犹主义的气氛,看到排犹主义的表现,他们结成密集的封闭的群体与此对抗,任何人都别想开出一条路打进这个圈子。
\par 说到“laift”,这事还不如那之前几天发生的另一件事叫我惊奇:布洛克问我为何前来巴尔贝克(相反,他似乎觉得他自己来这里是极其自然的事),是不是“指望认识几个美人儿”。我对他说,这趟旅行是我向往已久的一件事,然而比去威尼斯的欲望还差一层。这时,他回答说:“对,当然了,为的是一面装做读约翰·拉斯金爵士的《Stones of Venaice》\footnote{《Stones of Venice》为拉斯金的作品,共三卷,第一卷于1851年,第二、三卷于1853年均在伦敦出版。但直到1874年再版本及1881年的缩写本出版,这部著作才打响。1900年春普氏游览威尼斯的圣马可时,手里就捧着这本书。缩写本于1906年由玛蒂尔德·克雷默译成法文,书名为《威尼斯的石头》。此处布洛克出于无知,将Venice”(威尼斯)说成“Venaice”(威耐斯)。},一面和漂亮太太们一道吃冰淇淋。那位拉斯金是个面色阴沉、令人讨厌的家伙,是世界上最叫人讨厌的绅士之一。\footnote{普氏极喜欢拉斯金的著作,这里,布洛克的话怎样刺激了他,诸位可以想见。}”布洛克显然以为,在英国,不仅所有的男性都是“爵士”,而且字母“i”也总是发“ai”的音。圣卢认为这个发音错误并不严重,因为他从中主要看出我这位新朋友缺乏社交概念。我这位新朋友既没有这些概念,又蔑视这些概念。罗贝生怕哪一天布洛克知道了人说“威尼斯”而不是“威耐斯”,拉斯金并不是爵士以后,会往前想到罗贝一定觉得他无知可笑,反倒觉得自己罪过,似乎自己不够宽宏,实际上他真是宽宏无度。布洛克有一天发现自己的错误时会染上面颊的红晕,罗贝已提前感到它飞上了自己的面颊。他肯定布洛克比他自己把这个错误看得更重。这正是此后不久,有一天布洛克听到我说到“lift”时的感受。他立刻打断我说:“啊,应该说‘lift’。”同时用生硬而又高傲的语气说道:“其实这完全无关紧要。”这句类似反应的话,所有自尊心很强的人,无论是在最重大的场合还是在最微不足道的场合也都这么说。这说明,对于声称无关紧要的那个人来说,即使在微不足道的场合之中,所说的那件事也是非常紧要的。任何一个有些高傲的人,刚刚夺走了他紧紧攀住的最后的希望,拒绝给他帮忙,从他嘴上也会首先冒出这句话来,这时便是令人伤心的话,也是悲剧性的一句话了:“啊,好吧,这完全无关紧要,我另作安排吧!”这完全无关紧要地向他推去的“另作安排”,有时竟会是自杀。
\par 此后布洛克对我说了一些非常热情的话。他肯定希望对我非常客气,可亲。可是,他问我:“你与德·圣卢昂布雷交往甚密,是想把自己抬高到贵族吗?——那贵族阶层与其余的人是差不多的,你太幼稚了。你可能正处在赶时髦的狂热之中。告诉我,你是不是时髦青年?是,对不对?”
\par 他这样说,并不是因为他想对我客气这种愿望突然改变了,而是因为他的缺点正是人们用很不正确的法语称之为的“没有受过良好教育”。他自己对这个缺点无所察觉,更不会认为别人会因此而不快或反感。
\par 在人类中,人人具有的品德,与每个人特有的众多的缺点相比,其比例并不更大。显然,“世界上最普遍的事物”,并不是良知,而是善良。在最遥远偏僻的角落里,人们会惊异地看到善良这朵花自动开放,犹如在幽静的山谷中开放着一朵丽春花。这朵花与世界上其他地方的丽春花无异,但它从未见过其他的丽春花,只见识过有时叫它那孤独的小红帽颤抖不已的狂风。即使这种善良因利害关系而变成瘫痪,表现不出来,它依然存在。每当没有任何自私的动机妨碍它发挥的时候,例如读一本小说或一份报纸的时候,这种善良便会大放光华,向弱者、向正义者、向受迫害的人而去,甚至一个杀人犯,作为长篇连载小说的爱好者,他的心仍然很软,在这种人心中,这善良也是如此放光。
\par 与美德令人佩服的情形相似,缺点的多种多样也令人叹为观止。最完美无缺的人也有某个缺点使人不快或令人着恼。某一个人智力超群,高瞻远瞩,从不说任何人的坏话,但是,你亲自交给他请他转交的最重要信件,他却放在自己口袋里忘了交,后来又叫你误了一次重要的约会,而且也不微笑着向你道歉,因为他一向以自己从不知道时间是几点钟为荣。另外一个人思想细腻,性情温柔,待人接物高雅,关于你本人,从来只说会叫你高兴的话,但是你感觉到他对有些事闭口不谈,将某些事埋在心底,各种各样的事在他心里闷着发酵。他见到你很高兴,他把这高兴看得那么宝贵,宁愿叫你累死,也不离开你。第三位更诚恳一些,但是,当你说自己健康状况不佳而未能前去看望他请他原谅时,他把诚恳推进到非叫你知道,有人见你去戏院了,人家觉得你脸色很好不可。或者非叫你知道他并未完全受益于你为他进行的斡旋,再说已经有另外三个人主动提出为他进行活动,所以他对你也只是稍加感恩而已。在这两种情况下,前面那位朋友可能装做不知道你上戏院去了,装做不知道别人也能给他帮这样的忙。至于这最后一位朋友,他感到需要向什么人反复地说或者揭示出可能最令你反感的事,对自己的直爽感到十分得意,而且拼命对你说:“我就是这样。”
\par 有的人则以他们过于好奇或绝对没有好奇心来叫你着恼。你可以对他们谈到最为轰动的重大事件,而他们完全不知所云。有的人等几个月才给你回信,如果你的信是关于你自己的一件事而与他们无关的话。或者,他们对你说,要来问你什么事。你怕错过了他们的来访一直不敢出门,他们却并不前来,叫你等上几个星期,因为他们没有收到你的回信(而他们的来信根本没有要求你回信),以为他们惹你不高兴了。某些人高兴起来,想来看你,他们只顾自己愿意而不顾你愿意不愿意,口若悬河,不给你留下插嘴的地方,也不管你有什么紧急的事情要做。可是,若是他们感到时间长了,累了,或者心情不好,你就引不出他们一句话来,任凭你怎么使劲,他们用无精打采来对付你,再也不肯回答你的话,甚至不肯用一个字来回答,就像没听见你说的话一样。
\par 我们的每个朋友都有自己的缺点,为了能继续喜欢他,我们不得不寻些东西来自我安慰——想到他的才华,他的善良,他的温柔——或者更正确地说,将我们的好意充分发挥出来,对他们的缺点置若罔闻。可惜,我们这样好心对我们朋友的缺点极力做到视而不见,总是敌不过他的极力放纵,因为他看不见自己的缺点,或者以为别人看不见。讨人嫌这种危险主要来自难以评价不显眼的或未被察觉的事,所以出于谨慎,至少应该从不谈论自己。可以肯定地说,在这个题目上,别人的看法与我们自己的看法永远不会一致。人们参观一幢外表平平的房屋,里面不论是珍宝满室,还是遍地皆是盗贼用的撬门铁棒或死尸,发现了别人真正的生活,那表面天地之下的真天地时,都会感到同样的惊异。借助于每个人对我们说的话,我们对自己形成了一个印象。通过他们在背后就我们发表的言词,我们得知他们对我们和我们的生活怀有怎样完全不同的形象时,我们的惊异不会比上述情形更小。因此,我们每次谈论过自己以后,都可以确信,我们说的那些无害而谨慎的话语,被人表面上彬彬有礼并虚伪地表示赞同听了去以后,会叫他们作出最叫人恼怒或最令人快乐的评论,一言以蔽之,是最不利的评论。至少我们对自己的想法和我们的话语之间不成比例,也很会激怒别人。这样的不成比例,一般总是使人们就自己所说的话显得非常可笑,就像那些冒牌音乐爱好者,虽然作出极其赞赏的样子,但是他们叫我们听到的话语并不能说明他们的赞赏。他们一面用有力的指手画脚和一副赞赏备至的表情来补偿那含糊不清、喃喃低语的不足,同时又感到需要哼一首他们喜爱的曲调。
\par 除了谈自己和谈自己缺点这个坏习惯之外,还要加上另外一个与此结成一体的坏习惯,那就是揭露别人身上的某个缺点,恰恰自己也有这同一缺点。人们总是谈论这些缺点,似乎是一种谈论自己的方式,实际上是用拐弯抹角的方式,把承认自己的快乐与宽恕自己的快乐结合在一起。
\par 此外,似乎我们的注意力总是被吸引到构成我们自己特点的东西上去,与别人身上的其他东西相比,更容易发现这些东西。一个近视眼谈论别人时会说:“他眼睛几乎睁不开。”一个肺结核患者对一个最健壮的人肺部是否完好总有疑问;一个很不爱清洁的人总说别人不洗澡;一个嗅觉不灵敏的人总认为别人身上有味道;一个丈夫,自己老婆作风不正,会到处看到老婆作风不正的丈夫;一个举止轻浮的女人到处都看到举止轻浮的女人;一个追求时髦的青年,到处看到时髦青年。每种毛病,也像每种职业一样,要求一种专门知识,并不断发展这种专门知识。将这些知识卖弄一下,并不令人恼火。性欲倒错的人发现性欲倒错的人,一位裁缝应邀到了社交场合,他还未与你谈话,就已经品评起你的衣料,他那手指已经迫不及待要来捻一捻看质量如何了。如果你与一位牙医谈上一会话,然后问他对你有何真实想法,他就会告诉你,你有几颗坏牙。在他看来,没有比这更重要了。待你也发现了他的坏牙,你会觉得没有比这更可笑的了。
\par 不仅仅我们谈到自己时,以为别人都是盲目的,就是我们做事时,也似乎以为别人是盲目的。对我们每个人来说,都有一个专门的上帝无时不在,他遮掩住我们每个人的缺点,或向我们每个人许诺看不见我们的缺点,犹如对不洗澡的人,对他们耳朵上的一条污垢,臂弯里的汗味,他都闭上眼睛,堵上鼻孔,并且要他们坚信,他们可以带着这些污垢和汗味在人间游荡,不会受到任何处罚,人们什么也发觉不了。佩戴假珍珠或以假珍珠相赠的人,以为别人定会把假珠当成真珠。
\par 布洛克很没有教养,有神经病,追求时髦,属于一个不受尊重的家庭,如同在海底一般承受着无法计算的压力。这压力不仅来自表层上的基督教徒,还有高于他所在的阶层的一层层犹太阶层,每一层都以自己的蔑视压迫着紧挨着自己下面的那一层。要从一个犹太家庭上升到另一个犹太家庭,穿过一层又一层,直到呼吸到自由的空气,布洛克可能要花上数千年的时间。最好是设法从另一个方向上开辟一个出口。
\par 布洛克跟我说什么我正处在赶时髦的狂热之中,要我向他承认我是时髦青年时,我本可以这样回答他:“如果我是,我就不会与你常来常往了。”可我只是对他说,他这样讲话太不客气。于是他想道歉,但是没有教养的人实在有福气,依照他们的方式,便是一面毁掉自己的前言,一面伺机将那些话语变得更加沉重。
\par “请你原谅我,”现在他每次遇到我都这样说,“我曾经叫你难过,曾经折磨你,我是故意使坏。不过——从总体来说,所有的人,从个体来说,你的朋友,都是奇怪的动物——你无法想象,我虽那么无情取笑你,可我心中对你是一片柔情。我想到你时,这种柔情常常令我下泪。”说着,他便叫人听到一声呜咽。
\par 布洛克身上使我惊异的,还有更甚于他举止不适度的地方,那便是他的谈话质量好坏相差很大。这个小伙子十分挑剔,对一些最时髦的作家,他常说:“这个人是个面色阴沉的白痴,那个人完全是个傻瓜。”可有时他能十分开心地讲述一些毫不可笑的传闻轶事,引证某一个完全平庸的人的话,说“那人真是了不起”。评断人的智慧、价值、意义的这一双重天平,总是使我惊异不止,直到我结识他的父亲老布洛克先生那一天,这个谜才算解开。
\par 我真没想到,有一天我们竟然同意去与老布洛克结识。因为小布洛克在圣卢面前说了我的坏话,又在我的面前说了圣卢的坏话。他特别对罗贝说我(一直)追求时髦追求得要死。“对,对,他能结识勒——勒——勒格朗丹先生十分荣幸。”他说。布洛克这样将一个词分开说,既表示讽刺,又表示文学味道。
\par 圣卢从未听说过勒格朗丹这个名字,大吃一惊:“此乃何人?”
\par “噢,这是一个很了不起的人。”布洛克回答,哈哈大笑,同时怕冷似的将两手插进外衣口袋里,确信他此刻正在欣赏一位了不起的外省绅士那独具特色的外表。与这位绅士相比,巴尔贝·多尔维利的外表简直就是小巫见大巫。布洛克不会描绘勒格朗丹先生的形象,便用赋予他好几个“勒”字和像躲在柴捆后面品酒一样品味这个名字的办法来聊以自慰。但是这种主观的享受别人是领略不到的。
\par 他一方面在圣卢面前说我的坏话,另一方面在我面前也没少说圣卢的坏话。到了第二天,我们两人便都知道了这些谗言的详细情形,倒不是我们俩相互学舌,那我们可就太罪过了。但是布洛克会觉得这是非常自然而几乎不可避免的事,以至他在心神不安之中——他认为我们肯定会从这个或那个人嘴里得知我们要知道的事——宁愿先下手。他把圣卢拉到一边,向他招认了自己故意说他坏话的事,又告诉圣卢,他以“誓言监护人、克洛诺斯之子宙斯的名义”起誓,他爱圣卢,愿意为圣卢献出生命,说罢又抹去一滴眼泪。同一天,他又安排好单独见我,向我作了忏悔,宣称他那么做是为了我的利益,因为他认为某种社交关系对我有害,而我“比这个更有价值”。然后像醉汉动情那样抓住我的手,虽然他的酒醉纯属神经质:
\par “相信我好了,”他说,“若是昨天想到你,想到贡布雷,想到我对你无限的柔情,想到你自己甚至回忆不起来的某些下午上课的情形,我不曾哭了一整夜,就叫黑煞神凯尔立即把我捉了去,让我穿过人类厌恶的哈得斯\footnote{亦为克洛诺斯之子,宙斯之兄弟,为冥王。他在魔鬼和煞神帮助下(其中就有凯尔),想尽一切办法将活人拉进他那黑暗的王国中去。谁掉进他的冥府,便再不得永生;也无返回之路。}之门好了!对,一整夜,我向你发誓!可是,我知道,我了解人,我知道你不相信我的话。”
\par 确实,我不相信他的话,我感到这些话是临时编造出来的,是随说随编出来的。他“以凯尔的名义”起誓,也并没有增加很大重量,因为布洛克对古希腊宗教的信仰纯属文学性质。此外,每当他激动起来,同时也希望别人为一件虚构的事实所感动时,他总是说“我向你发誓”的。与其说这是为了叫人相信他说的是实话,不如说那是为了撒谎骗人而制造的歇斯底里官能享受。他对我说的话,我不相信。不过我也不怪他,因为我从母亲和外祖母那里继承了不会怀恨在心的天性,甚至对于比这大得多的罪过也不怀恨。我同时也继承了永不谴责任何人的天性。
\par 再说布洛克也不是绝对的坏孩子,他也能做出非常热情的事情来。自从贡布雷人种,也就是如我外祖母和我母亲这样的绝对完美无缺的人从中产生的人种似乎濒于完全灭绝以来,我只能在未开化的、无动于衷的、忠心耿耿的正直人——他们一开口讲话,那声音便很快表明他们根本不关心你的生活——和另一种人之间进行选择。这后一种人,只要他们在你身边,他们就理解你,钟爱你,感动得下泪,可是过了几个小时又会翻脸不认人,跟你开上一个残酷无情的玩笑。此后,他们还会回到你的身边,仍是那样善于察颜观色、热情可爱,立刻就能与你融成一体。相比之下,我可能还是更喜欢这后一种人,就说不喜欢他们的道德价值吧,至少喜欢与他们相处。
\par “我想你的时候那种难受劲,你是无法想象的,”布洛克又说,“归根结底,这是我身上相当犹太人味道的一面又冒出来了。”他冷嘲热讽地加上一句,同时眯起自己的双眼,好像要在显微镜下为那数量极小极小的“犹太血液”定量一般。一个法国贵族大老爷,在全是基督徒的祖先之中,也可将萨米埃尔·贝尔纳或者再往前数,将圣母玛利亚打进去。他可能也会这么说(实际上他是不会这么说的)。据说,莱维家族就自称是圣母玛利亚的后代。
\par “我相当喜欢这样从我的情感中分出这一部分来,再说这是很小的部分,这部分可能属于我的犹太血统。”他又补充道。他道出这句话,因为他觉得道出自己种族的真相,既聪明又正直。在这同一场合,他又设法莫名其妙地减轻这真相的分量,就像那些下定决心还债,又只有勇气偿还一半的吝啬鬼。拿出勇气来宣布真相,同时又在其中掺上很多歪曲真相的谎言,这种弄虚作假的方法,比一般人想象的更为普遍,甚至一般不这么做的人也是如此:生活中某些紧要关头,特别是关系到恋爱关系的紧要关头,便给他们提供了这样的机会。
\par 布洛克瞒着我在圣卢面前对我抨击谩骂,瞒着圣卢在我面前对圣卢抨击谩骂,这一切均以邀请我们前去做客而结束。若说布洛克开始时没有进行尝试以便单独邀请圣卢,我当然不相信。看上去很可能进行了这样的尝试,但是没有成功,于是有一天布洛克对我和圣卢说:
\par “亲爱的师兄,还有你阿瑞斯\footnote{阿瑞斯是希腊神话中的战神,相当于罗马神话中的马尔斯。}和圣卢昂布雷心爱的骑士,驯马人,既然我在乘飞舟的默尼埃家族\footnote{可能指巧克力商人加斯东·默尼埃一家,他们的游船“亚里安娜”号当时是很著名的。}帐篷附近、飞沫轰鸣的安菲特里特\footnote{安菲特里特是海中女神,波塞顿的妻子。}海岸上与你们相遇,二位是否愿意赏光,这星期当中的哪一天到我那位鼎鼎大名、良心清白的父亲家中用晚餐?”\footnote{此处布洛克模仿荷马的笔调讲话。}
\par 他向我们发出这一邀请,因为他极想与圣卢结成更密切的关系,他希望圣卢能使他进入贵族阶层。如果这个希望是我提出来的,是为我自己提出来的,那布洛克就会觉得是十足的令人厌恶的附庸风雅的表现了。这与他对我本性的一个方面的看法完全符合,至少到现在为止,他不认为这是我本性中的主要方面。但是同样的希望从他那里提出来,他就觉得是他的头脑有良好求知欲望的表现了,他热切希望与某些与己不同的社会阶层交往,说不定从中能找到某些文学上有用的东西。
\par 儿子对老布洛克说,要带一位朋友来吃晚饭,用一种略带讽刺挖苦的心满意足的口气道出这朋友的头衔和名字“德·圣卢昂布雷侯爵”时,布洛克先生感受到强烈的震动。他大叫起来:
\par “德·圣卢昂布雷侯爵!啊!他妈的!”对他来说,使用骂人的话,那是对人最高敬重的表现。
\par 他向儿子投过赞美的一瞥:儿子竟能结交上这样的人!那目光意味着:
\par “他真叫人大吃一惊。这个浪子,他是我的孩子吗?”
\par 这目光使我的伙伴快乐不已,好比每个月给他增加五十法郎零用钱一样。布洛克在家中很不自在,感到父亲将他当成不走正道的人,因为他靠崇拜勒贡特·德·利尔、埃雷地亚\footnote{这是布洛克最佩服的两位蒙巴那斯派诗人。}和其他“游手好闲的人”过活。可是他跟圣卢昂布雷结交上了,后者的父亲曾是苏伊士运河公司董事长啊!(啊!他妈的!)这可是“无可争议”的成果啊!
\par 因为怕把立体镜弄坏了,将立体镜留在了巴黎,现在人们更加感到遗憾。只有布洛克父亲一个人掌握了使用这立体镜的艺术,至少他有权使用。再说他也难得用一次,非常小心翼翼,也就是贵客上门设华宴的日子。所以,观看立体镜表演的人,觉得这是特殊礼遇,是对上宾的优待;而组织表演的主人,则产生了威信,与天才产生的威信相仿佛。即使风景照是布洛克先生本人亲自拍摄的,这个镜是他自己发明的,那威信也不会比这更高。
\par “昨天你没有得到邀请去所罗门家吗?”人们在家中这样谈论。
\par “没有,我没有被慧眼看上!都有什么名堂?”
\par “排场很大,立体镜,全套玩艺。”
\par “啊,如果有立体镜,我很遗憾,据说所罗门将立体镜拿出来示人时,非同寻常。”
\par “有什么办法!”布洛克先生对儿子说道,“不应该同时把什么都给他,这样,他就总是还有点什么东西欲求不得。”
\par 从父爱出发,并且想打动他的儿子,他确实想到要把那仪器弄来。但是“具体时间”不够,或者更正确地说,人们以为时间不够。不过,我们不得不将晚餐的时间推迟,因为圣卢走不开,他在等一位舅舅,这舅舅将来到德·维尔巴里西斯夫人身边过四十八小时。这位舅舅非常热衷于体育锻炼,尤其热衷于长途步行,他要从他在乡间度假的那个城堡,基本上步行而来,在农庄过夜,所以他何时抵达巴尔贝克是说不准的。圣卢不敢动,我这位朋友每天给他的情妇发的电报,甚至都委托我去电报局所在的安加维尔发出。
\par 他们等待的舅舅名叫巴拉麦德,他从自己的祖先西西里亲王那里继承下来这个名字。后来我在阅读历史著作时,遇到这个名字——有人说是真正古老的名字——属于中世纪意大利及法国南部某些城市的某某最高行政长官或某某教会之长\footnote{指红衣主教、大主教和主教。},为文艺复兴时期的漂亮招牌。这个名字一直留在这个家族中,代代相传,从梵蒂冈办公室直传到我朋友的舅舅那里。有的人因为没有钱,无法成立勋章馆,美术馆,便去追求古老的姓名(地名,像一张古老的地图,一张骑士照,一个招牌或一个普通人姓名那样有文献意义又有地方色彩;受洗礼的名字,在美妙的法兰西文字结尾音节中震荡着,叫人听得出来舌头有毛病。某地居民俗气的语调,发音不正确,我们的祖先正是按照这些使拉丁词和撒克逊词发生了持久的变化,这些变化后来又成为语法了不起的立法者),总而言之,借助于这些古老音响的汇集,这些人给自己开起了音乐会,就像那些到处搜罗低音古提琴\footnote{大提琴的前身。}和抒情古提琴以便在古老的乐器上奏出往昔音乐的人一样。当我读到这个名字时,我体会到上述这些人的那种快乐。
\par 圣卢对我说,甚至在最封闭的贵族社会中,他的舅舅巴拉麦德仍然以特别难以接近、蔑视一切、醉心于自己的贵族出身而与众不同。他与自己的弟媳和另外几个精心选择的人在一起,组成了人所称之的“凤凰圈子”。就是在这个小圈子里,他也因傲慢令人恐惧,以至以前发生过社交场上有人想与他结识,前去与他的亲弟弟打交道,亦遭到拒绝的事。
\par “不,不,不要要求我将你们介绍给我哥哥巴拉麦德。我妻子,我们所有的人,都合力去做,也无能为力。不然,你们会撞上他很无礼,我不希望如此。”在赛马俱乐部,他和几位朋友指定了二百名俱乐部成员,他从来不让人将这些成员介绍给他们自己。在德·巴里斯公爵家里,他因衣着华丽、性情高傲而以“亲王”这个绰号为人所周知。
\par 圣卢向我谈了他这位舅舅早已逝去的青年时代。他与自己的两个朋友,也像他那么漂亮,合住一套单身汉小公寓,每天他将一些女人带到公寓里来,因此人称他们是“美惠三女神”。
\par “有一天,一个人——照巴尔扎克的说法,这个人如今是圣日耳曼区最出头露面的一个人,但在那还不走运的最初阶段,流露出莫名其妙的嗜好——他向我的舅舅要求到这套单身公寓里来。刚一到,他就开始求爱,并不是向女人,而是向我的舅舅巴拉麦德。我舅舅装做听不懂,找个借口把那两位朋友带了出去。然后他们一起回来,捉住那个坏蛋,剥掉他的衣服,打得他血迹斑斑,零下十度的大冷天,把他踢到门外。人家发现他时,他已经半死不活,结果法院前来进行调查,那个倒霉鬼\footnote{这个倒霉鬼,便是福古贝。}好不容易才叫法院停止调查。今日,我舅舅大概再也不会干这么残酷处置人的事了。他这个人对上流社会的人那样高傲,可你想象不到,如今他与多少平民百姓有热烈的友情,保护他们,哪怕得到的报答是忘恩负义。一个从前在某一公馆里服侍过他的仆役,他会安插到巴黎去。一个农民,他会叫人教他学会一行手艺。这是他身上相当讨人喜欢的一面,与他那花花公子的一面形成鲜明对照。”
\par 圣卢确实属于上流社会的这种青年,他们所处的地位,使人可以对他们道出这样的词句:“他身上有相当讨人喜欢的东西,讨人喜欢的一面。”这是相当宝贵的种子,很快就会生产出一种待人接物的方式。在这种方式中,他人一钱不值,而“平民百姓”便是一切。一言以蔽之,与平民百姓的骄傲截然相反。
\par “据说,他年轻时,在整个那个社会阶层里,他就是表率,他说了就算,简直难以想象。对他来说,在任何情况下,他认为怎样最令人愉快、最实惠,他便怎样办,但是立刻便有附庸风雅的人来加以仿效。在剧场里,他很渴,叫人将饮料送到他的包厢后头。到了下周,每个包厢后头的小客厅都装满了清凉饮料。有一年夏天阴雨连绵,他有些风湿痛,便定做了一件柔软而暖和的驼绒外套,无非是当旅行毛毯用,上面蓝色和橘红的条条他一动未动。立刻,高级裁缝便见他们的主顾都来定做蓝色长毛带流苏的外套了。他在某一城堡度过一天,如果由于某种原因,他希望免去一次晚宴的庄重性质,为了表示出这种细微差别,他没有带礼服来,穿着下午的上装入席,那么,在乡下着普通上装参加晚宴便成为时髦。为了吃一块点心,他没有使用小勺,而使用了一个叉子或什么他向金银器匠定做的自己发明的餐具,那以后便不许他用别的方法吃了。他想再听一遍贝多芬的某几首四重奏(要说他这些异想天开的想法,他可一点都不愚蠢,而是非常聪明),便请了一些艺术家来,每个礼拜为他和几位朋友演奏。那么这一年,聚集为数不多的人,听室内音乐,便是最为高雅的事。我相信他生活中没有烦闷过。像他从前那么漂亮,女人,他肯定有过不少的!不过我无法准确地告诉你都是谁,因为他这个人守口如瓶。但是我知道,他反正把我那可怜的舅母欺骗得够呛!可这并不妨碍他跟她在一起很愉快,她对他无比钟爱。舅母死后,他哭了好几年。他在巴黎时,仍然几乎每天到墓园去。”
\par 罗贝就这样一面等待着他的舅舅,一面对我谈到他。结果是白等。第二天上午,我回旅馆,独自一个人从游艺场前面经过时,感觉到离我不远有一个人在注视我。我扭过头去,看见一个男子,四十岁左右,很高,相当胖,唇髭很黑。他一面用一根小手杖神经质地拍打着他的裤子,一面用睁得大大的眼睛聚精会神地盯着我。有时,极其灵活的眼珠在两只眼眶里骨碌碌地转。只有站在一个陌生人面前,而这个陌生人又由于某种原因使你产生其他人——例如疯子或暗探——不会产生的一些想法时,人才会有这种眼神。他向我飞送过来绝妙的一瞥,既大胆,又谨慎,既飞快,又深沉,好似逃跑时投出的最后一瞥。他环视一下四周,骤然摆出心不在焉而又高傲的神情,整个人突然一转,扭身去看一张海报。他专心致志看海报,一边哼着一首曲子,并整理垂在他扣眼间的那朵苔蔷薇。他从口袋里取出一个摘记簿,好像是将戏名记在本子上。他掏了两三次怀表,把一顶扁平的黑色草帽向下拉到眼睛上,手又作帽檐状,接长了草帽的帽檐,似乎为了看看是不是有什么人来。他做了一个不满意的动作,通过这个动作,可以叫人看出,他已经等烦了。但是如果真的等什么人,则永远不会做出这样的动作。然后他把帽子推向脑后,露出剪得很短的刷子头。可是两侧都还留着相当长而弯曲的鸽子翅膀\footnote{指鸽子翅膀一般的头发。}。他大声吐出一口气来。人不仅很热,而且希望表现出自己热得受不了时,就是这样吐气的。
\par 我忽然想到,这是个旅馆骗子,他可能前些日子已经注意到了我外祖母和我,正准备敲我们一下,可他刚才发现,就在他觊觎我的时候,让我给撞见了。为了骗我,他可能想通过这种新姿态,极力表现出心不在焉和漠不关心的样子。可是他未免夸张得太剑拔弩张了,以至似乎他的目的不仅是要打消我可能产生的怀疑,报复我不知不觉对他可能进行的侮辱,让我明白他不仅没看见我,而且我是一个太无足轻重的东西,根本不可能引起他的注意。他做出勇夫模样,挺起腰杆,撇起嘴唇,翘起胡子,在眼神里再配上某种毫不在乎、生硬而又几乎侮辱人的东西。结果是他那奇异的表情,叫我一会将他当成偷儿,一会将他当成疯子。
\par 然而他的衣着极其讲究,比起巴尔贝克我看见的所有洗海水浴的人的衣着来,要严肃得多,简洁得多,也叫我的上装放了心,因为那些人的海滨装那刺眼而又俗气的淡颜色常使我的上装受到侮辱。
\par 可是这时我的外祖母来迎我了,我们一起转了一圈。一小时以后,她回旅馆去了一小会,我在旅馆门前等她。这时我看见德·维尔巴里西斯夫人与罗贝·德·圣卢以及在赌场前那样死死盯住我看的那位陌生人一起走了出来。他的目光与我看见他那时一样,闪电一般飞快地从我身上扫过,然后,就像他没有看见我一样,收回到自己的眼前稍下的地方,迟钝、有如中性的目光,假装外表上什么也没有看见,内心什么也看不见。这目光仅仅表示睁圆了眼睛,撑开了睫毛,感觉到四周有睫毛而感到满意。这是某些伪君子的那种虔诚而又沉醉的目光,是某些蠢人的自命不凡的目光。
\par 我看到他已经换了衣服。现在他穿的上装颜色更深,显然这是因为真正的优雅比虚假的优雅距离简朴更近一些。但是,还有别的东西:更靠近些人,人们感受到,这些服装上之所以几乎完全没有别的颜色,并不是因为取消这颜色的人对此无动于衷,而更确切地说,是因为出于某种原因,他禁止自己使用颜色。这些服装显示出来的朴素似乎是属于那种源于对某种规定的服从,而不是源于对颜色没有胃口。在长裤的料子中,有暗绿的丝,与袜子上的条纹非常和谐,那种精细透露出一律着深色这种审美观的强大力量,对这种趣味,出于容忍精神,只作了这唯一的让步。领带上有一个红点,作为胆敢放肆,是难以察觉的。
\par “你好吗?我来向你介绍这是我的侄子德·盖尔芒特男爵。”德·维尔巴里西斯夫人对我说。陌生人并不看着我,咕咕哝哝地说了个含糊不清的“荣幸”,后面紧接着便是“哦,哦,哦”,为的是赋予他的和蔼某种勉强的意味。他蜷起小拇指、大拇指和食指,向我递过中指和无名指来,这两个手指上没有一个戒指。我隔着他的瑞典手套,握住这两个指头。然后他没有对我抬起眼皮,朝德·维尔巴里西斯夫人转过身去。
\par “天哪,我昏了头了吧?”这位夫人笑着说,“我把你叫成德·盖尔芒特男爵了!我向您介绍,这位是夏吕斯男爵。不管怎么说,这错误不太严重,”她又添了一句,“反正你确实姓盖尔芒特嘛!”
\par 这工夫,我外祖母出来了,我们便一起上路。圣卢的舅舅不仅不对我们说一句话给我面子,甚至不瞧我一眼。虽然他打量陌生人(这次短短散步过程中,他向一些无足轻重的出身最寒微的路人投过两三次他那凶狠而又深沉的目光作为试探),反过来,他从来就不注视他认识的人,如果以我的判断为准的话——像一个执行秘密任务的警探将自己的朋友置于职业监视之外一般。我任凭外祖母、德·维尔巴里西斯夫人与他谈天说地,将圣卢拉到后面:
\par “告诉我,我是不是没听清楚?德·维尔巴里西斯夫人对你的舅舅说他从前是盖尔芒特家人。”
\par “是啊,当然啦,他就是巴拉麦德·德·盖尔芒特。”
\par “在贡布雷附近有一座城堡,自称是热纳维埃夫·德·布拉邦特后代,他与那家姓盖尔芒特的,是一家吗?”
\par “绝对没错:我舅舅,没人比他更讲究纹章学了,他会回答你说,我们的‘呐喊’,我们的‘战斗口号’,首先是‘贡布雷人’,后来才变成了‘帕萨王’,”他笑着说,为的是不要显得为这个“呐喊”的特权而洋洋自得,只有几乎可以称王的家族,大的帮派首领才有这种“呐喊”。“这城堡的现主人,便是他的兄弟。”
\par 这位德·维尔巴里西斯夫人就这样与盖尔芒特家族结成了近亲。但是对我来说,她很长时间一直是我小时候送我一盒鸭子叼着的巧克力的太太,那时,她与盖尔芒特一侧要比说她被关在梅塞格利丝一侧更为遥远,在我看起来,还不如贡布雷的眼镜店主人显赫,社会地位高。可她现在突然身价倍增,与此平行的,是我们拥有的其他物品出人意料地贬值。增值也好,贬值也好,都在我们的少年时代和我们少年时代残存之中的各个部分,导入与奥维德的变形一样众多的变化。
\par “是不是在这座城堡里有盖尔芒特世家古代高官的全部胸像?”
\par “对,是个好景。”圣卢冷嘲热讽地说。“咱俩说说,勿告他人:我觉得这些东西无味得很。不过在盖尔芒特有更有意思的东西!那就是加里埃\footnote{加里埃(1849—1906),是肖像画及家庭场景画家。}所绘制的我姨母的肖像,十分动人。与惠斯勒或委拉斯开兹的作品一样美,”圣卢又加了一句,他在新教徒的狂热中,不能总是准确地把握住伟大的标尺。“也有居斯塔夫·莫罗的动人的画。我的姨母是你的朋友、德·维尔巴里西斯夫人的侄女,是这位夫人带大的,她嫁给了自己的表兄,也是我的婶祖母维尔巴里西斯夫人的侄子,就是现在的德·盖尔芒特公爵。”
\par “那你的舅舅又是什么人呢?”
\par “他的贵族头衔是夏吕斯男爵。照规矩,我的外叔祖父去世时,我的舅舅巴拉麦德本应取得德·洛姆亲王的头衔,他的哥哥成为盖尔芒特公爵之前就是这个头衔。这个家族里,人们更名改姓就像换衬衣一样。可是我舅舅对所有这些事都有一些特别的想法。他觉得意大利的公爵,西班牙的什么高级称呼等等都用得太滥,虽然他可以在四五个亲王头衔中进行挑选,但他出于抗议,保留了夏吕斯男爵的头衔,表面上很朴素,实际上这里头包含着许多自傲。他说:‘如今什么人都是亲王,可是毕竟得有点东西使你与众不同。待我想隐姓埋名出门旅行时,我一定取一个亲王头衔。’照他的说法,没有比夏吕斯男爵更古老的头衔了。蒙莫朗西男爵自称是法兰西最古老的男爵,其实不确,因为他们那时只是他们的采邑法兰西岛的男爵。为了向你证明夏吕斯男爵早于蒙莫朗西男爵,我的舅舅会兴致勃勃地给你解释上几个小时。虽然他非常精明,有才干,他仍然觉得这是一个非常生动的谈话题材,”圣卢微微一笑说道,“可是我不像他,你不要叫我谈什么谱系,我真不知道还有什么比这个更叫人昏昏欲睡,比这个更过时的了。确实,人生太短暂了。”
\par 从刚才在赌场附近使我转过身去的那股生硬的目光中,我现在认出了当年在当松维尔,斯万太太召唤希尔贝特时我见过的死死盯住我的目光。
\par “你告诉我,你的舅舅德·夏吕斯先生有过许多情妇,这里头有没有斯万太太?”
\par “噢!绝对没有!他是斯万先生的一位好友,一向给斯万先生许多支持。可是,从来没有人说他是斯万老婆的情夫。如果你流露出相信这个的样子,肯定会在上流社会里引起极大的惊异。”
\par 我没敢回答他说,如果我流露出不相信这个的样子,在贡布雷,人们会感到更加惊异的。
\par 我外祖母被德·夏吕斯先生迷住了。当然,他对一切关于世家和社会地位的问题极为重视,外祖母也发现了。但是人们对此严加指责时,一般总有隐隐的妒意和恼怒在里面,因为看到另外一个人享有自己也想有却无法拥有的优越地位。外祖母则丝毫不带此等的严责。相反,她对自己的命运很满意,丝毫不为自己并不生活在一个更加显赫的社会阶层而感到遗憾,所以她只是运用自己的智慧去观察德·夏吕斯先生的毛病而已。她谈到圣卢的舅父时,怀着达观、微笑、几乎好感的善意。我们用这种善意来报答他,因为他作为我们进行毫无利益关系的观察对象,给我们带来了快乐。何况这一次,这观察对象还是一个人物,外祖母觉得他的自命不凡,不说是合情合理吧,至少也独有特点,这使得他与外祖母一般有机会见到的人相比,显得对照鲜明。
\par 与圣卢嘲笑的许多上流社会的人相反,可以看得出来,德·夏吕斯先生极其聪明、感受力极强。我的外祖母也正是因为这一点而轻易地原谅了他的贵族成见。然而无论是舅舅,还是外甥,都没有因为更杰出的优秀品质而丢掉这种成见。更确切地说,德·夏吕斯先生将二者调和起来了。像德·纳穆尔公爵和德·朗贝尔亲王的后代一样,他拥有档案,家具,壁毯,拉斐尔、委拉斯开兹和布歇为他的祖先绘制的肖像。只要概述一下他对自己家族的回忆,就可以名副其实地说,他是在“参观”一座博物馆和一间无与伦比的图书室。可是相反,他将贵族的全部遗产都置于他的外甥将他贬到的那个地位上。说不定还有另外一个因素,那就是他不像圣卢那样空想,不尚空谈,是更现实的人类观察家,他不愿意忽略他们视为根本的威望因素。虽然他赋予自己的想象以非物质利害的享受成分,但是这个因素对于他那功利主义的活动却可以常常成为一剂极为有效的补药。
\par 这种人与另一种人之间一直是有争论的。另一种人听从内心理想的召唤,内心的理想促使他们舍弃这些好处,去一心寻求实现理想。在这方面,他们与那些放弃自己高超的技巧的画家、作家很相似,与采用现代手法的手艺人很相似,与主动实行普遍裁军的善战人民很相似,与实行民主、废弃严酷法律的极权政府很相似,而现实常常并不能酬答他们高尚的努力。有时和平主义反倒使战争增加,宽容也使犯罪增加。如果从外部效果来判断,只能说圣卢努力做到诚恳和外露是非常了不起的,但也容许人们庆幸德·夏吕斯先生恰恰缺乏这二者。夏吕斯先生叫人将盖尔芒特公馆一大部分精美的木器运到了他外甥家里,而不是像他的外甥那样拿这批家具换了一套时髦款式的家具和一些勒布\footnote{勒布(1849—1928),法国画家,早期自由发展,1877年他与莫奈、毕沙罗、德加结识,深受印象派影响。}和纪约曼\footnote{纪约曼(1841—1927),法国画家,与印象派画家关系密切,自觉与塞尚和毕沙罗最接近,其作品已显示出表现主义与野兽派的某些特点,但总的来说他是自然主义的。}的画。
\par 德·夏吕斯先生的理想非常做作,这也是真的,如果“做作”这个修饰语可以与理想这个词联系起来的话,也就是说,既有社交气又有艺术性。几个姿色倾城又有罕见文化素养的女性,两个世纪以前,她们的祖先就已与君主制度全部的荣光与风雅结为一体。他从这样的几个女性身上找到了出众超群的东西,使他能够和她们在一起才感到快乐。诚然,他对这些女性的钦佩是诚心诚意的,但是她们的名字所唤起的许多历史与艺术上的模糊回忆也起了很大的作用。恰如贺拉斯的一首颂歌说不定比如今的一些诗歌逊色,但是一个文人读起前者来会感到快乐,对后者却无动于衷,对古代的回忆是他感到快乐的原因之一。这些女性中的每一个,与一个漂亮的布尔乔亚女子相比,对他来说,犹如那些古画之于当代一幅画着一条路或一次婚礼的油画。对那些古画,知道它们的历史,从定购这些画的教皇或国王开始,中间又经过什么大人物,这些画,通过馈赠,购买,取得或继承遗产,又唤起我们对某一重大事件的回忆,至少也唤起我们某一有历史意义的联想,因此我们获得的知识便赋予这些作品以一种全新的用处,增强了我们头脑中或我们博学中拥有财富的感觉。如果与德·夏吕斯先生的偏见相似的偏见妨碍这几位贵妇人去与血统不那么纯正的女性为伍,而将她们未起任何变化的崇高完整地奉献到他的祭坛上,就像某一十八世纪建筑的门面,由玫瑰色大理石平滑的廊柱支撑着,新朝代来到并未丝毫改变这门面一样,他是很为此庆幸的。
\par 德·夏吕斯先生赞赏这些女性真正精神崇高,心地高尚,\footnote{在法文中,这里用的“崇高”和“高尚”字眼与“贵族”为同一个词——noblesse。}就这样用模棱两可来搞文字游戏,这模棱两可欺骗了他自己,其中也有这一含混概念、这种将贵族、心地高尚与艺术混为一谈所造成的虚假表象,同时也有夏吕斯先生诱人的一面。对于我外祖母这样的人,这种引诱是非常危险的。一个贵族,只看到自己的营盘,对其余的则不闻不问,他的偏见更荒唐,但也更无害人之心。对我外祖母来说,她似乎觉得这种偏见过于可笑,但是一旦某种东西在超人智慧的外表下出现,她就无还手之力了,以至她以为王子们都出众超群、令人艳羡,因为他们得以有拉布吕耶尔\footnote{拉布吕耶尔1684年被指定为波旁公爵(1668—1710)的历史、地理、法国各机构、哲学教师。}和费纳龙\footnote{国王路易十四于1684年任命费纳龙为其孙子勃艮第公爵(1682—1712)的私人教师。}这样的人作私人教师。
\par 在大旅社门前,三位盖尔芒特家人离开了我们。他们到卢森堡亲王夫人家用午餐去了。就在外祖母向德·维尔巴里西斯夫人道再见,圣卢向外祖母道再见的时候,直到此刻没有与我讲过话的德·夏吕斯先生向后走了几步,来到我身边。
\par “今天晚上晚饭后,我要在维尔巴里西斯婶母房内喝茶,”他对我说,“我希望你能赏光与你外祖母前来。”说完他追侯爵夫人去了。
\par 这天虽是星期天,旅馆门前的出租马车并没有度假季节开始时多。尤其是公证人的妻子,她觉得因为不去康布尔梅家而每次租一辆马车实在太破费,干脆待在自己房间里。
\par “布朗代太太身体不适吗?”人们问公证人,“今天没见她呀!”
\par “她有点头疼,天这么热,又下雷阵雨。有一点事她就要……我想今天晚上你们能看见她。我已经劝她下楼了。这会对她有好处。”
\par 我以为德·夏吕斯先生邀请我们去他婶母那里,是想弥补上午散步时他对我表现出的无礼,我也不怀疑他肯定通知了他的婶母。但是,当我走进德·维尔巴里西斯夫人的客厅,想向她的侄子问好时,我在他周围转来转去,一点搭不上话。他正用尖细的嗓门,针对他们的某个亲戚讲一个相当不怀好意的故事。我无法捕捉他的目光。
\par 我下定决心向他问好,而且声音相当大,为的是提醒他注意我的存在。可是我明白他早已注意了我的存在。因为就在我躬身施礼而从我的双唇还没有发出一个字音的时候,我看到他伸出两根手指叫我握,而眼睛却没有转过来,亦未中断他的谈话。显然,他看见了我,只是不露声色。这时我发现他的双眼从来都不定睛望着谈话对方,而是不停地四面转动,就像某些受惊野兽的眼睛,或者露天小贩的眼睛。这些露天小贩,他们一面大吹特吹,展示他们那违法的商品,一面头虽不转,却眼观六路,窥视着警察会出现在地平线上的各点。
\par 我看出,德·维尔巴里西斯夫人看见我们来了很高兴,但是她似乎没有料到我们会到来。我有点惊异。德·夏吕斯先生对我外祖母说:“啊,你们来了,这个主意真不错。婶婶,这真好,是不是?”
\par 我听到这话,更惊诧莫名。显然他发现他婶母见我们进来大吃一惊,作为惯于定调子的人,他想只要指出他本人感到很高兴,就足以将这惊讶变成快乐了,而且我们前来也确实应该激起快乐的情绪。
\par 这件事他算计对了,因为德·维尔巴里西斯夫人对她侄子看得很重,而且知道要讨他开心是多么困难。她似乎突然发现我外祖母有什么新的优秀品质,不断地殷勤招待她。
\par 我无法理解,德·夏吕斯先生在几小时之内便将当天早上向我发出的邀请忘得一干二净。这邀请虽然很简短,但表面上看是那样有意为之,那样经过考虑,他竟然将这个完全是他自己的主意,称作我外祖母的“好主意”。我那时还是“丁是丁,卯是卯”的,直到后来长大了,才明白:对于一个人的意图到底如何,不是向他本人询问就能得知真相的;宁愿冒产生误会的危险,误会说不定未引人注意就过去了,这种风险远远小于天真地认死理。
\par “先生,”我怀着非要弄个一清二楚的心情对他说,“您可记得,不是您向我要求,请我们今晚来的吗?”
\par 没有一个动作,没有一点声音能透露出德·夏吕斯先生听到了我的问题。看到这种情景,我又重复了一遍我的问题,就像外交家或那些闹了别扭的年轻人一样,他们不厌其烦地要得到对方的澄清,但是毫无用处,对方就是下定决心不予以澄清。德·夏吕斯先生并不给我进一步的答复。我仿佛看见他的双唇上掠过一丝冷笑,那是居高临下品评别人的性格和所受教育的人发出的冷笑。
\par 既然他拒绝给予任何解释,我便尝试自己作出解释,结果我在数种解释之间犹疑不决,哪一种解释都不能算是合情合理。可能他想不起来了,或者是我将他今天上午对我说的话理解错了……更可能的是,由于傲慢,他不愿意显出自己曾极力吸引他蔑视的人的样子,而宁愿将他们到来的主动推到他们自己头上。如果是这样,既然他蔑视我们,那为什么他又非要我们来不可呢,或者更准确地说,他非要我外祖母来不可呢?因为整个晚上,他只跟我外祖母一个人讲话,而没有跟我讲过一次话。他藏身在外祖母和德·维尔巴里西斯夫人身后,好像他在包厢里头一样,他与她们极其热烈地谈着,只是有时将他那洞察一切的双眼,探究的目光,停驻在我的脸上。看他那一本正经和专心致志的劲头,似乎我的脸是一部难以辨识的手稿。
\par 显然,如果没有这双眼睛,德·夏吕斯先生的面庞与许多美男子的面庞会十分相像。圣卢后来与我谈起其他的盖尔芒特家人时,对我说:“当然,我舅舅巴拉麦德那种从头到脚、直到指甲尖的大老爷派头,家族派头,他们是没有的!”他这么说也就肯定了,贵族的家族派头和贵族特点,毫无神秘和新鲜之处,而是由这些成分组成的。我能够毫无困难地分辨出这些因素,而且不感到有什么特别感想,我应该感到我的某一幻想破灭了。
\par 但是这张面孔,薄薄的一层粉赋予它舞台上面孔的某些外表,德·夏吕斯先生将其表情封闭得再严实也没有用。双眼好比一条缝隙,好比一处枪眼,只有这个他无法堵上。别人从与他所占据的不同角度出发,通过这条缝隙和这处枪眼,感到骤然被某种内部装置的交叉反光映住了。看来这内部装置丝毫不能令人放心,甚至对于虽然并非这装置的绝对主人却自身携带着它的那个人也是如此。他本人处于不稳定平衡状态,随时有垮台的危险。这双眼睛的表情谨慎而又时刻惴惴不安,带着全部倦意,对面部造成的后果,便是眼睛周围形成一个下缘很低的大黑眼圈。不论组合、修饰得如何好,都会使你想到这是一个隐姓埋名的人,是一个有钱有势的人身处险境的化装,或者根本不是什么有钱有势的人,而只是一个危险而又悲剧性的人物。当我上午在游乐场附近见到德·夏吕斯先生时,对我来说,一桩秘密已将他的目光变成了谜,而其他男子身上是没有这种秘密的。我真想参透这桩秘密。但是依我现在所知的他的亲属关系,我再也无法相信这是偷儿的目光;依我所听到的他之谈话,我再也无法相信这是疯子的目光。他之所以对我那样冷淡,而对我外祖母那样和蔼可亲,大概并非来自个人的好恶,而是一般说来,他对女人怀着多少好意,谈论女人的缺点时一般也带着极大的宽容,他对男人,尤其是年轻人,就怀着多大的深仇大恨,这种仇恨使人想到某些厌恶女人的男人对女性的仇恨。他们家族中抑或圣卢的亲密好友中有两三个小白脸,圣卢偶然提到他们的名字时,德·夏吕斯先生便说道:
\par “这些坏蛋!”表情凶猛,与他惯常的冷淡形成鲜明对照。我明白了,他特别谴责今日之青年人的,便是他们太女人腔。
\par “这是地地道道的婆婆妈妈!”他常常怀着轻蔑说。
\par 但是与他希望的一个男子应该过的日子相比,还有什么样的生活不会显得女人气呢?他一向认为这种生活劲头不足,男子气概不足(他本人在徒步旅行中,疾走了几小时之后,身上热呼呼地便跳进冰冷的河水中)。他甚至不能容忍一个男子戴戒指。
\par 但这种对大丈夫气概的固有之见并不妨碍他具有非常细腻敏感的长处。
\par 德·维尔巴里西斯夫人请他给我外祖母描绘德·维尼夫人住过的一座城堡,同时加上一句话,说与那个令人厌烦的德·格里尼昂夫人分离,塞维尼夫人那么伤心,她本人觉得这无非是文学上的夸张而已。
\par “相反,我觉得没有比这个更真实的了,”他回答道,“再说,那个时代,这种情感人们是很能理解的。拉封丹笔下莫诺莫塔帕的居民梦中看见自己的朋友有些悲伤,便奔至他的家中。一只鸽子最大的痛苦就是另一只鸽子不在自己身边。\footnote{见拉封丹寓言《两个朋友》和《两只鸽子》。}婶婶,您大概会觉得这也和塞维尼夫人迫不及待要与她女儿单独相聚一样是夸张吧!她离开自己女儿时,说的那些话多好啊!——‘这次分别使我内心痛苦,我像肉体痛苦一样感觉到它。在分别中,人们对时间很大方。\footnote{普氏在这里将塞维尼夫人致格里尼昂夫人的两封信混在一起了。1671年2月18日函为:“这次分别使我内心痛苦,我像感觉到肉体痛苦一样感觉到它。”1689年1月10日函为:“在分别中再不是这样,人们丝毫不考虑这些,有时甚至向前推,人们希望:在渴望中时间过得快。人们对一天天的时光很大方,谁愿意要就送给谁。”}人们在渴望的时间中前进。’”
\par 我外祖母听到别人用与她自己完全相同的方式谈到这些书信,真是心花怒放。一个男子能够对这些书信理解得如此之妙,她惊讶不已。她觉得德·夏吕斯先生真像女性一样情感高尚而细腻。后来我们两人单独在一起谈起他的时候,我们说他肯定受过一位女子深刻的影响,或者他的母亲,或是晚些时候他的女儿,如果他有子女的话。我想起圣卢的情妇,在我看来,她对他产生了极大的影响。我心里想道:“一个情妇。”这种影响使我得以意识到:男人与女人一起生活,这些女子会把男子的情感磨炼得多么细腻!
\par “这位塞维尼夫人,一旦到了自己女儿身边,很可能反倒与她无话可谈了!”德·维尔巴里西斯夫人回答道。
\par “肯定有话可谈的,哪怕是那些她称之为‘只有你和我才能注意到的微不足道的事情’\footnote{这句话在塞维尼夫人的1675年5月29日致女儿的信中。}。而且不管怎么说,塞维尼夫人常在女儿身边。拉布吕耶尔告诉我们,这就足够了:‘在自己热爱的人身边,与他们谈话也好,什么话也不与他们谈也好,全是一样的。’\footnote{这句话只是大意,引自拉布吕耶尔《论性格》第二十二章。}他言之有理,这是唯一的幸福,”德·夏吕斯先生又用忧郁的语气补充道,“这种幸福,可惜,人的生活安排得这样糟糕,以至难得品味到这种幸福。总的说来,塞维尼夫人并不比别人更值得可怜。她的大半辈子是在自己喜欢的人身旁度过的。”
\par “你忘了,咱们说的不是爱情,而是她的女儿。”
\par “但是生活中重要的不是我们所爱的人,”德·夏吕斯先生以权威性的、不容置喙的、几乎是斩钉截铁的口气接着说下去,“而是我们在爱。塞维尼夫人对她的女儿的感情,与其说与公子哥塞维尼和他的情妇们之间的那种庸俗关系相类似,不如说更类似于拉辛在《安德罗玛克》或《淮德拉》之中所描写的那种激情。因爱上帝而爱这种神秘主义,亦是如此。我们围绕着爱情划出的分界线过于狭窄,唯一的原因是我们对生活太无知。”
\par “你很喜欢《安德罗玛克》和《淮德拉》吗?”圣卢问他的舅父,语气微带轻蔑。
\par “拉辛的一出悲剧所包含的真理,比维克多·雨果先生的所有正剧还要多。”德·夏吕斯答道。
\par “这上流社会,不管怎么说,是够吓人的!”圣卢附耳对我说,“喜欢拉辛胜过雨果,不管怎么说,这太过分了!”他舅父的话真叫他心里难过,不过,道出“不管怎么说”和“过分”,他又得到了快乐,对他是一种安慰。
\par 德·夏吕斯先生对于离愁别恨发表的一通感想,使我外祖母后来对我说,德·维尔巴里西斯夫人的侄子对某些作品的理解远远超过她的婶母,而这个侄子头脑中有点什么东西,使他远远超出大部分贵族俱乐部的人。从这些感想中,他不仅仅显露出情感的细腻,这在男人确实罕见,就连他的嗓音也与众不同,他的嗓音与某些女低音相像,这女低音的中音区训练得不够,唱起歌来似乎是一个小伙子和一个女人的二重唱。在他表达这些细腻的思想时,他的嗓音落在高音符上,显出出人意料的温柔,似乎包含着未婚妻、姐妹的合唱,发挥出她们的柔情。可是德·夏吕斯先生是非常讨厌女性化的,如果说在他的嗓音里,似乎庇护着一群少女,他大概会心里很难过。但是这群少女不仅仅局限在对表现情感的文学片断的解释和音调转化上。他谈天时,人们常常可听到她们尖细而又爽朗的笑声,这些住宿生或爱俏的女孩正用风趣而幽默的语言、噘着小嘴向她们身边的男子进攻。
\par 他说,有一幢房屋,从前属于他那个家族,玛丽-安托瓦内特\footnote{法国国王路易十六的妻子,与其丈夫都死在断头台上。}曾经在那幢房子里住过,花园为勒诺特尔设计。现在这幢房屋属于富有的金融家伊斯拉埃尔\footnote{伊斯拉埃尔与“以色列”同音同字,因此有下面之发挥。}家族了,他们将这幢房子买了去。
\par “伊斯拉埃尔是这些人的姓,可我总觉得这是人的分类、人种方面的一个词汇,而不是一个专有名词。不知道怎么回事,也可能这类人没有姓,而只有用他们所属的集体来称谓的。这倒无所谓!可是从前是盖尔芒特家的房屋,现在属于伊斯拉埃尔家族!!!”他大叫起来。“这使人想到布卢瓦城堡中的一个房间,带人参观的城堡看守人到了那里,对我说:‘从前玛丽·斯图亚特在这里祈祷,现在我把扫帚什么的放在这里。’自然,对这所丢人现眼的房子以及离开丈夫出走的我的堂嫂克拉拉·德·希梅\footnote{希梅公馆位于马拉盖河堤十七号,1640年芒萨尔建。五十年以后,勒诺特尔又为其设计了花园。此公馆后来相继属于贝尔特朗·德·拉巴吉尼埃尔,亨利埃特·德·法郎士和德·布永公爵,1823年成为财务总监拜拉波拉的财产。后者被推定女儿嫁给了德·希梅亲王。1884年,这所房屋成为美术学校的一部分。克拉拉·德·希梅亲王夫人于1896年离开自己丈夫与一个小提琴家私奔。},我什么都不想打听!但是我还保存着这所房屋仍然完好无缺时的照片,也保留着亲王夫人的照片,那时她的大眼睛里还只有我的堂兄一个人。当照片不再是真实事物的复制品,向我们显示的是已不再存在的事物时,照片便赢得了某些威望。既然您对这类建筑感兴趣,我可以送给您一张。”他对我外祖母说。
\par 这时,他发现自己口袋中绣花手帕那鲜艳的花边露出来了。他赶快将手帕放进袋中,惊恐的表情犹如一个过分腼腆而又绝非天真无邪的女子在遮掩自己的某些魅力。由于顾忌太多,她觉得显露这些东西不合体统。
\par “请你们设想一下,”他接着说下去,“这些人首先就把勒诺特尔的花园毁了,这简直和撕碎普桑的一幅画一样罪过!就为这个,这些伊斯拉埃尔家的人就该给关进监狱里去。”沉默了一会,他又微笑着加了一句:“当然还有许多事,为那些事,他们也应该进监狱,这是真的!不管怎么样,请你们设想一下,在这些建筑物前面,搞上一个英国式花园会产生什么效果!”
\par “可是那房子与小特里亚侬\footnote{小特里亚侬为凡尔赛王宫的一部分,建筑师为雅克-昂日·加布里埃尔(1698—1782)。在小特里亚侬周围,设计的是英国式框架,建有一些小型房屋,如爱情坛,观景亭、微型剧场及田园房舍等,建筑师为理查·米克(1728—1794)。王后玛丽·安托瓦内特特别喜欢住在这里。}是同一款式,”德·维尔巴里西斯夫人说,“玛丽安托瓦内特不是也叫人在小特里亚侬修了一个英国式花园嘛!”
\par “那英国式花园总是有损加布里埃尔那建筑正面的美观嘛!”德·夏吕斯答道,“显然,如今要将那田园房舍拆毁,几乎是野蛮的罪行!但是不论现代精神是什么,在这个问题上,伊斯拉埃尔太太的一个什么异想天开的念头能与对王后的回忆具有同样的威信,我总归是怀疑的。”
\par 这期间,外祖母已经向我示意,要我上楼睡觉去,虽然圣卢一再挽留。圣卢在德·夏吕斯先生面前暗示说,我常常晚上入睡前感到悲哀,他的舅父一定觉得这未免太缺乏男子气概,真是羞煞我也!我又滞留了一些时候,后来就走了。过了一会,我听到有人敲门。我问是谁。令我惊异的是,我听到的竟是德·夏吕斯先生的声音。他干巴巴地说:
\par “是夏吕斯。先生,我可以进来吗?”他走进来,关上房门以后,仍是那样干巴巴地说下去,“我外甥刚才说,您入睡以前有些烦闷,另外,您又非常欣赏贝戈特的著作。我箱子里有一本贝戈特的书,很可能您没有读过,我就把这本书给您送过来,以帮助您度过这段您觉得不大快活的时光。”
\par 我非常激动地向德·夏吕斯先生表示感谢,并对他说,相反,我怕的是,圣卢对他说我在夜晚来临时感到不适,会使我在他眼中显得比我的实际情形更加愚蠢可笑。
\par “没有的事。”他答道,语气更温和一些。“您可能没有什么个人才能,我对此一无所知。可是有才能的人是何等罕见!不过,至少有一段时间,您有青春年少,这本身就总是很有诱惑力的东西。再说,先生,最大的蠢事,是认为凡是自己没有感受的情感,便都是滑稽可笑的或值得谴责的。我喜欢夜晚,可是您对我说,您害怕夜晚。我喜欢玫瑰花的芬芳,可是我有一位朋友,玫瑰花的香气会使他发烧。您难道会以为我因此就觉得他不如我吗?我尽力理解一切,我避免谴责任何事物。总而言之,不要过分抱怨。我不是说这种忧郁感不难受,我知道人可以为某些事情非常痛苦,而别人却不理解。但是至少您已经把自己的爱寄托在您的外祖母身上,您经常看见她。而且这是一种得到别人允诺的柔情,我的意思是得到回报的柔情。有许多人,他们还不是这样的呢!”
\par 他在房间里踱来踱去,看看这件物品,举起那件东西。我的印象是他有什么事需要对我宣布,但是找不出适当的词句来说。
\par “我在这儿还有另一本贝戈特的书,我叫人给您拿来。”他加了一句,便打铃。
\par 过了一会,来了一个青年侍者。
\par “去把你们的侍应部领班给我找来!这儿只有他办事机灵。”德·夏吕斯先生高傲地说。
\par “先生,您是说埃梅先生吗?”侍者问。
\par “我不知道他的名字。噢,对,我想起来了,我听见人家叫他埃梅。快去,我有急事。”
\par “他马上会来,先生,我刚刚在楼下看见他。”侍者回答,想作出消息灵通的模样。
\par 过了一会,侍者回来了。
\par “先生,埃梅先生已经就寝了。我可以替您去办。”
\par “不,不,你只要叫他起来就行了。”
\par “先生,我没办法,他不在这儿过夜。”
\par “那,算啦,你走吧!”
\par “先生,”待侍者走后,我说,“您太好了,贝戈特的书,有一本对我已经足够了。”
\par “对,看来是这样。”德·夏吕斯先生还在走来走去。
\par 就这样过了几分钟。然后,他又犹豫了一会,又改口好几次。最后,他原地打了一个转,说话的嗓音又变得很粗暴刺耳,对我说了一句“先生,晚安!”就走了。
\par 这天晚上,我听他表达了各种高尚的情感。第二天他要走了。上午,在海滩上,我刚要去洗澡,德·夏吕斯先生走到我身边提醒我说,我一出水就要去找我外祖母,她正等着我。出乎我意料的是,他扭住我的脖子,用庸俗的随便而又嘲弄的口气对我说:

\paragraph*{4}

\par “你对年迈的外祖母才不放在心上呢,是不是,小滑头?”
\par “先生,您说什么,我十分爱她!……”
\par “先生,”他迈开一步,冷冰冰地对我说,“您还年轻,您应该好好利用这青年时代学会两件事:第一,您要避免表达一些过于自然的情感,以免让人听出弦外之音来。第二,别人对您说的话,在您未明白那些话究竟意味着什么之前,不要趾高气昂地去回答。前些时候,如果您采取了这样小心谨慎的态度,您就不会显得聋子模样胡说八道了,同时也就不会在游泳装上绣上船锚这样可笑的事情之外再干别的滑稽可笑的事。我借给您一本贝戈特的书。我现在需要。请您叫那个名字可笑、对他很不合适的侍应部领班,过一个小时,把那书给我送回来。我想,他总不至于这时候还在睡觉吧!您使我感到,昨天晚上对您谈什么青春有诱惑力为时太早了,如果我向您指出青春年少的人的傻气、前后不一和不解人意,也许倒会给您更好地帮点忙。先生,我希望这个小小的冷水澡会比您的海水浴对您更有好处。不过,别站在这儿一动不动,您会着凉的。再见,先生。”
\par 显然他为这些话感到后悔。因为过了一些时候,我收到他寄来的一本书,就是他借给我,我又请人还给他的那本书。不过那本书不是埃梅去还的,他碰巧“出去了”,而是开电梯的人去还的。这本书是高级皮面精装,书面上,又夹镶了一块皮革,半凸起,呈一枝勿忘草形状。
\par 德·夏吕斯先生一走,罗贝和我终于能够去布洛克家进晚餐了。在这次小小的晚会上,我明白了,原来我们的伙伴轻易觉得滑稽可笑的那些故事,正是老布洛克的故事;“完全莫名其妙的”人,正是他的一位朋友,他总是这样评论他。有一部分人,人们在童年时代很佩服他们,例如比家里其他人更聪慧的父亲啊,向我们揭示了玄学、而在我们眼中他本人即受惠于玄学的一位老师啊,成绩比我们好(布洛克就比我成绩好)的一个伙伴啊等等。我们还喜欢缪塞的《上帝的希望》时,他已经看不起写了《上帝的希望》\footnote{《上帝的希望》是缪塞1838年2月写的一首诗,1840年发表在《新诗集》中。}的缪塞了。而当我们喜欢勒贡特老爹\footnote{故事发生时,勒贡特·德·利尔刚逝世不久。}或克洛岱尔时,他又只为
\refdocument{
    \par 在圣·勃莱兹,如祖埃卡模样,
    \par 你是那样、那样轻松自如……\footnote{这首诗的题目为《歌曲》,亦发表在《新诗集》中,为缪塞作。} 
}
\par 这样的诗名所陶醉了。还要再加上:
\refdocument{
    \par 帕多瓦\footnote{帕多瓦为意大利一城市。}是美丽的地方,
    \par 伟大的法学博士\footnote{此句补全为“创造了奇迹”。}
    \par 但我更喜欢玉米粥……
    \par 夜幕降临,托帕黛尔双眸柔情似水,
    \par 身着黑色化装长外衣走过。
    \par 可以走近她身边,毫无危险。
    \par 而且对她说:“我是异乡人,您真美。”\footnote{最后四行原文引文不全,经译者补足。这首诗题目为《致意大利归来的兄弟》,亦发表在《新诗集》中。}
}
\par 从各首《夜诗》中,他只记得这几句:
\refdocument{
    \par 在哈佛尔,面对大西洋,
    \par 在威尼斯,可怕的丽都旅馆,
    \par 苍白的亚德里亚姑娘,
    \par 死在一坟墓的青草上。\footnote{这是《十二月之夜》中的一段,亦为缪塞作。}
}
\par 对于发自内心信任而佩服的某个人,人们满怀钦佩之情收集、引用一些句子,实际上这些句子还不如人们发挥自己的天才写出来的东西。可是对后者,人们却严厉地拒绝接受。一位作家在一本小说中,借口真实,使用了一些“词”,一些人物,在有血有肉的总体中,这些词、这些人物反倒构成死沉的重物,平庸的部分,实际情形亦是如此。圣西蒙笔下的人物肖像,他自己并不欣赏,却非常精彩;而他认为迷人的笔触,他了解的聪敏过人的人,却很一般,抑或变成了无法理解的人。关于戈尼埃尔夫人\footnote{戈尼埃尔夫人(1605—1694),据说非常机敏风趣,她在巴黎的沙龙十分著名。她说的那些笑话,当时在社交界广为流传。}或路易十四,他写的那些文字,本人是不屑于去杜撰的,却如此细腻或如此生动。这种现象值得提出,在许多作家身上也同样存在。对此有各种解释,此刻我们记住下面这一种解释也就足够了:这是因为在“观察”的精神状态中,人们远远低于创作时的水平。
\par 所以,我的伙伴布洛克与他那比儿子落后四十年的老子完全是一个模子塑造出来的,他讲些莫名其妙的轶事,放声大笑。外露的真正的老布洛克也是那样,他一面放声大笑,一面将最后一句话重复两三次以便使听众完全品出那故事的味儿来。他的儿子此时也放声大笑,总是这样在餐桌上对父亲的故事表示敬意。就这样,小布洛克道出最富有智慧的事情,显示出他从自己家中得来的财富。此后,他又第三十遍道出几句俏皮话。这种俏皮话,老布洛克是只在非常隆重的日子才往外拿的(同时还有他的燕尾服),那就是小布洛克带来一个什么人,值得向这个人炫耀一番:他的什么老师啊,门门得奖的一个“同学”啊,或者像那天晚上那样,圣卢和我啊……例如他说:“一位了不起的军事评论家,提出了种种证据,由于某种不可置辩的原因,大作文章地演绎出日俄战争中,日本必败,俄国人必胜。”\footnote{此处事件发生时间有误,因日俄战争发生在1904—1905年。日本战胜,俄国战败。}或者说:“这个人很了不起,他在政界中被认为是一位大金融家,而在金融界中被认为是一位大政治家。”这一类的笑话还可以换成关于罗特希尔德男爵的故事和鲁弗斯·以色列军士的故事。用模棱两可的方式将这些人物搬上舞台,暗示布洛克先生对这些人本人都认识。
\par 我自己也上了当。从老布洛克谈论贝戈特那模样看,我也相信了贝戈特是他的一位老朋友。而实际上,所有的名人,老布洛克都是“并不相识”地认识,即在剧场里,在马路上,远远看见过他们。此外他还想象,以为他自己的面孔、名字、人品对那些人来说并不陌生,那些人看见他的时候,常常不得不控制自己隐隐要与他打招呼的欲望。上流社会的人,因为认识有才华的人,第一流的人,他们接待这些人共进晚餐,却不因此就对他们更了解。但是如果在上流社会中稍微过上几天,这个社会中居民的愚蠢就会使你希望生活在那个“并不相识”地认识人的默默无闻的阶层中,使你想象他们有许多智慧。我在谈到贝戈特时,马上就体会到了这一点。
\par 老布洛克在家中很有名气,但并非他一个人如此。我的伙伴在他姐妹面前更是如此。他把头埋在盘子里,以咕咕哝哝的语气,不断盘问她们,搞得她们笑出眼泪。她们也采用兄弟的那种语言,说得很流利,似乎这种语言实为必须,而且是聪明人所能使用的唯一语言。我们来到时,大姐便对一个妹妹说:“快去向我们谨慎从事的父亲和令人尊敬的母亲禀告。”
\par “母狗们,”小布洛克对她们说,“我来向你们介绍一下,这位是圣卢骑士,他手持锋利的标枪,从东锡埃尔来到石头磨光、雕满奔马的住所度过几日。”他既庸俗又识文断字,他的演说一般总以并非那么有荷马味的玩笑结束:“喂,把你们那别针华丽的无袖长衣\footnote{古希腊和古罗马妇女穿的无袖长衣,用别针在肩上扣住。}裹紧点。哟,这位装腔作势的家伙是什么呀?反正不是我父亲!\footnote{这是乔治·费多的喜剧《马克西姆店中的女人》(1899)中一个人物克莱威特的著名台词。}”于是布洛克家各位小姐哄堂大笑,笑得前仰后合。我对他们的兄弟说,他推荐我读贝戈特的书,给我多少快乐!我对贝戈特的书真是喜欢至极。
\par 老布洛克只是远远见过贝戈特,对贝戈特的生平只是道听途说有些了解。看样子,对贝戈特的著作也是借助于肤浅的文学评论,间接了解。他生活的世界,是“差不多”,在空虚中致意,在虚假中判断。在这个圈子里,不准确,不在行,并不会降低人的自信,相反,只会使之增加。这是自尊心受人欢迎的奇迹,能够有显赫熟人和精深学识的人很少,所以缺乏这二者的人仍可自认为了不起。因为从社会阶梯的视角望去,似乎处于某一地位的人,都觉得自己的地位最好。对那些最伟大的人,他可以指名道姓,虽然不认识却可以诽谤他们,虽然不理解他们,却可以对他们评头品足,予以蔑视,认为他们没有自己地位优越,运气不好,值得可怜。自尊心可以将微薄的个人利益扩大几倍,即使在这样仍不足以保证每人都有一份幸福时,每人所必不可少的幸福,总是要高于给别人的份额,便有嫉妒来补充那差额。确实,当嫉妒用蔑视的语句来表达时,就必须将“我才不愿意认识他呢!”翻译成“我无法与他结识”来理解。这是理智上的意思。但感情上的意思确实是“我才不愿意认识他呢!”明明知道并非真的如此,但是,就这么说,并非只是出于虚假,而是确实如此感觉,这也就足以消除上述那个差距,即幸福上的差距了。
\par 自我中心主义使每一个人将自己看成国王,使他们这样去看待比自己低的那个世界。布洛克先生赋予自己一种奢侈享受,就是当一个无情的国王。每天早晨他喝可可时,从刚刚打开的报纸上看到一篇文章底下署着贝戈特的名字,便满怀蔑视地对他简短开庭审判,宣布对他的判决,赋予自己以舒适的快感,每喝一口滚烫的饮料,便重复一句:“这个贝戈特写的东西简直没法看了!这个畜生真叫人讨厌!这报不能订了!这真是叫人上当受骗!写的什么破玩艺!”说着又吃一块涂了黄油的面包片。
\par 老布洛克这种幻觉式的自觉了不起一直扩展到他自己的感受圈子以外。首先,他的子女将他视为一个出类拔萃的人。子女对自己的父母总是要么倾向于看不起,要么倾向于歌颂、赞扬。对于一个孝顺儿子来说,自己的父亲总是最好的父亲,甚至超出佩服他的一切客观理由之外。而对布洛克先生来说,这些客观理由并不绝对缺少,他受过教育,敏锐,对妻子儿女非常有感情。在家族近亲中,人们跟他在一起非常愉快,因为在“上流社会里”,人们根据十分荒谬的标准和错误却又一成不变的规则来评断人。与其他那些体面华贵的人相反,在资产阶级的这个小圈子里,晚宴、家庭晚会总是围绕着人们宣称令人愉快和好玩的人进行的,而这些人在真正的上流社会里,两个晚上就要垮台。总而言之,在这个不存在贵族阶级又故作了不起模样的阶层里,人们用更加莫名其妙的与众不同来代替贵族的装模作样。在其家庭,甚至直到很远的远亲看来,据说老布洛克的唇髭模样和鼻子上部与某贵族相像,因此人们都称老布洛克为“假奥马尔公爵”\footnote{真奥马尔公爵(1822—1897)为路易菲利浦的第四个儿子。在阿尔及利亚屡建战功。著有《孔德亲王传》,1871年进入法兰西学院。}(在“骑士”俱乐部圈子里,某一个人歪戴着制帽,穿一件紧身的上装,以显示出外国军官的模样,对于他的伙伴来说,难道不是一种人物吗?)。
\par 这种相像是最捉摸不定的,但是可以说这毋宁是一个头衔。人们反复地说:“布洛克?哪一个?奥马尔公爵吗”就像人们说“缪拉公主?哪一个?(那不勒斯)王后\footnote{唯一当过那不勒斯王后的缪拉公主是拿破仑的妹妹卡洛琳娜·波拿巴。她嫁给了缪拉。缪拉1808年被封为那不勒斯王。}吗?”一样。某些其他细小的迹象最后又赋予他那与什么人物相似的眼睛以某种所谓的与从不同。布洛克还没有富到拥有一辆马车的地步,某些日子他从马车公司租一辆两匹马拉的维多利亚式敞篷马车穿过布洛尼森林。他有气无力地斜躺在马车里,两个手指头按在太阳穴上,另外两根手指托住下巴。如果不认识他的人因此认为他是一个装腔作势的家伙,家里人则确信,要论“帅”,所罗门大叔简直可以胜过格拉蒙加德鲁斯\footnote{格拉蒙加德鲁斯(1808—1865),是帝国时代一位将军的儿子,由路易-菲利浦养大。他由于行为不端而逃至东方度过晚年,遗嘱中将其财产传给德·克拉医生和一个风靡一时的女演员。}。他属于那种人:因为他们曾经和《激进报》\footnote{《激进报》创办于1871年,为巴黎一份左翼日报。1881年转入维克多·西蒙及亨利·马莱手中,1885年时发行四万份以上,到1912年时仍发行三万份以上。}主编在巴黎林荫大道\footnote{指巴黎市内巴士底广场与玛特莱广场之间的林荫大道。}一家饭馆中同桌用过饭,所以他们去世的时候,这家报纸的“交际纪事”栏里会称他们为“巴黎人熟悉的面孔”。
\par 布洛克先生对圣卢和我说,贝戈特对于为什么他——布洛克先生,不和贝戈特打招呼知道得清清楚楚,以至每当贝戈特在戏院里或俱乐部里远远看见他时,总是回避他的目光。圣卢面孔绯红。因为他考虑到这个俱乐部大概不是自己父亲曾担任主席的赛马俱乐部。另一方面,这可能是一个相对说来很封闭的圈子,因为布洛克先生说,如今贝戈特要去的话,人家是不会接待他的。所以圣卢诚惶诚恐地生怕“低估了对手”地问道,这个俱乐部是不是王家街的那一处。圣卢家族认为那一处是“不上等的”,他知道有某些犹太人在那里受到接待。
\par “不是,”老布洛克先生回答,一副不在意、骄傲而又羞愧的神情,“是一个小圈子,但是令人愉快得多,叫加纳什俱乐部。那里的人对画廊评头品足相当厉害。”
\par “俱乐部主席不是鲁弗斯·以色列爵士吗?”小布洛克向父亲问道,为的是给他提供个机会,叫他撒个体面的谎,同时他也没有料到,这位金融家在圣卢眼中并不具有在他家里人眼中那样的威信。实际上,加纳什俱乐部根本没有鲁弗斯·以色列爵士,只有他手下的一个雇员。但是这个雇员与自己老板的关系非常好,他可以使用大金融家的名片。布洛克先生要出门旅行,那条铁路的董事长正好是鲁弗斯·以色列爵士,那雇员便送了一张名片给布洛克先生。因此老布洛克常说:“我到俱乐部去,向鲁弗斯·以色列爵士请教一下。”那张名片叫他把列车长搞得晕头转向。
\par 各位布洛克小姐对贝戈特更有兴趣,谈话又回到他身上,而不是继续谈“加纳什”。妹妹以极其严肃的口吻问哥哥:
\par “这位贝戈特确实是令人惊异的一个椰子\footnote{“椰子”指人,用做贬意。但布洛克的妹妹此处并不带有贬意。}吗?他是属于大人物,维利埃\footnote{维利埃·德·利尔·阿达姆(1838—1889),其作品受到巴那斯派诗人的欢迎。}或卡蒂尔\footnote{卡蒂尔·孟戴斯(1841—1909),被认为是巴那斯派的创始人。}那样的椰子一类吗?”她认为,为了说明有才华的人,除了她哥哥使用的那些词语以外,这世界上便没有其他词语。
\par “我在好几次彩排时见过他,”纳西姆·贝尔纳先生说,“他很笨拙,是施莱米尔\footnote{这是祖籍法国的德国作家夏米索(1781—1838)的作品《彼得·施莱米尔》中的主人公,他将自己的影子卖给了魔鬼。在犹太德国土话中,“施莱米尔”的意思是“白痴”。}式的人物。”
\par 对夏米索寓言故事的这种影射倒丝毫不是什么严重的事,但是“施莱米尔”这个形容词是半德语半犹太语的方言组成部分,在自己家里用一用,叫布洛克先生心花怒放,但是在外人面前,他觉得太庸俗,不合适。所以他狠狠瞪了自己的叔父一眼。
\par “他很有才华。”小布洛克说。
\par “啊!”他妹妹表情严肃地说道,似乎是说,如果这样,我说的话是情有可原的了。
\par “所有的作家都有才华。”老布洛克轻蔑地说。
\par “据说他就要自荐进法兰西学院呢!”他儿子说,举起叉子,眯起眼睛,魔鬼般冷嘲热讽的表情。
\par “算了吧!他的学问不够。”老布洛克答道。他对法兰西学院似乎不像他的儿子和女儿那样怀着轻蔑,“他的口径不够。”
\par “再说,学院是一家沙龙,贝戈特没有立足之地。”布洛克太太的叔父宣称。她就要继承他的遗产了。这是个无害而温和的人物。只要听到他的姓贝尔纳,说不定就能唤醒我外祖父的诊断天才,但是这个姓又与他那面孔不够协调。他的面庞似乎是从达里奥斯宫带回来,又经过迪欧拉富瓦\footnote{迪欧拉富瓦夫人(1851—1916)与丈夫一起于1885年参加了苏斯·达里奥斯宫殿的发掘工作。她将一幅壁画复原,壁画表现猎狮的场面,现存卢浮宫。她是乔治·迪欧拉富瓦教授的侄女。}夫人复原的,如果他的名字纳西姆,被某个热切希望给这个苏斯面孔加冕的业余爱好者选中,没有让霍尔萨巴德\footnote{霍尔萨巴德为公元前八世纪末萨尔恭二世国王所建之亚述新帝国之首都。萨尔恭王死时,此城亦被弃。遗址在1843—1855年之间先后为法国考古学家所发掘,卢浮宫现存几件该城的绘画和雕刻,尤为著名的是兽身人面雄牛,高4.2米,有五蹄,正面看侧面看均可。这些雄牛是该城城门的守卫者。}的兽身人面雄牛翅膀在这面孔之上翱翔的话。但是布洛克先生不断地侮辱他的叔父,也许是因为他这个出气筒那和善的面孔叫他来火,也许是因为纳西姆·贝尔特先生已经付清了别墅的款项,受益者希望表现出自己保持着独立,根本不想用什么甜言蜜语去竭力保住自己要从这位阔佬那里继承来的遗产。
\par 使这位阔佬特别不快的,是人们当着旅馆侍应部领班的面那样粗暴地对待他。他咕咕哝哝地道出一句谁也不明白的话,人们只能辨别出“米煞在的话”几个字。米煞在《圣经》中是指上帝的侍者。\footnote{据《圣经·旧约》,米煞是巴比伦王尼布甲尼撒所派管理巴比伦事务的三个人之一。}在他们内部,布洛克家的人使用这个词来指仆人,每次都为此而嘻笑,因为他们确信,无论是基督徒还是那些仆人自己都不明白,这使纳西姆·贝尔纳和布洛克先生更加突出感到他们作为“主人”和“犹太人”的双重特点。但是有客人的时候,这后面一种心满意足的原因便变成了不满的一个原因。所以,布洛克先生听到他的叔父说“米煞”时,觉得他未免过分暴露了他那东方人的一面。这与一个卖身的女人请了自己的几个女朋友和一些像样的人前来做客,如果那些女朋友影射她们自己干的营生或者使用一些难听的字眼时,她会着恼是一样的。所以,叔父的请求根本没有对布洛克先生产生任何效果,布洛克先生大发雷霆,再也无法控制自己。他不失任何时机地辱骂这位可怜的叔父。
\par “当然,有什么平庸而一本正经的蠢话可以说的时候,可以肯定,你是不会错过这种时机的。如果他\footnote{此处的“他”,系指贝戈特。}在这儿,你肯定第一个上去舔他的脚!”布洛克先生大叫起来,而伤心的纳西姆·贝尔纳先生将他那萨尔恭国王的鬈胡子朝盘子低下去。我的伙伴自从也留了胡子以来,与他的叔祖父十分相像,他的胡子也是短而鬈曲,微微发蓝的。
\par “怎么,你是德·马桑特侯爵的儿子?我与他很熟。”纳西姆·贝尔纳先生对圣卢说。
\par 我想,他所说的“熟”,那意思与老布洛克说他认识贝戈特是一个意思,就是说,见过。
\par 但是他又加了一句:“你的父亲是我的一位好朋友。”
\par 这时小布洛克已经满面绯红,他的父亲看样子深深不快,各位布洛克小姐掩口而笑。这是因为纳西姆·贝尔纳先生喜欢吹嘘,已经养成了不断说谎话的习惯。布洛克先生及其子女也有这种爱好。例如,出门旅行,住在旅馆里,纳西姆·贝尔纳先生待所有的人都聚集在餐厅里,正吃午饭的时候,要他的贴身男仆将所有的报纸送到餐厅里来,好叫人看清楚他是带着贴身仆人出门旅行的。老布洛克有条件的话,也会这样做。对于他在旅馆里交上的朋友,这位叔父说自己是参议员,这个嘛,他的侄子可永远不会这么干。他可以肯定人家有一天会知道这个头衔是假冒的,但是这也无济于事,他在当时无法抵制要把这个头衔授予自己的那种需要。
\par 布洛克先生对他叔父的谎言和这些谎言给他惹来的麻烦深以为苦。
\par “你们别在意,他特别好吹牛!”他低声对圣卢说。这么一说,圣卢倒更有兴趣了,因为他对说谎者的心理活动非常想知道个究竟。
\par “雅典娜称伊塔克人是最会说谎的人,他比伊塔克人还要厉害。”我们的伙伴布洛克又补充了一句。\footnote{布洛克在此卖弄自己的学识,他指的是《奥德修斯本记》第十三章,奥德修斯刚到伊塔克,在那里遇到一个牧人盘问他的身份,奥德修斯对牧人存有戒心,就说了谎,然而这牧人正是雅典娜所扮,她责备奥德修斯不说真话。}
\par “啊呀!这可真是!”纳西姆·贝尔纳大叫道,“我怎么会料到和我朋友的儿子一起进晚餐呢!在巴黎,我家里,有一张你父亲的照片,还有多少他的信!他一直叫我‘我的叔父’,从来不知道为什么。他是个风度迷人、神采奕奕的人!我还记得在尼斯,在我家的一次晚宴,那天有萨杜,拉比什,奥吉埃……”
\par “莫里哀,拉辛,高乃依。”老布洛克冷嘲热讽地说下去。他的儿子继续完成这一串列举,又加上了“普鲁塔克,米南遮,\footnote{米南遮(约公元前342—292年)是雅典喜剧家。}迦梨陀娑\footnote{迦梨陀娑(公元前4—5世纪),印度诗人,《沙恭达罗》的作者,此书于19世纪译成法文。}。”
\par 纳西姆·贝尔特先生自尊心受伤,故事戛然而止。这位禁欲主义者自我剥夺了一项极大的快乐,直到晚宴结束,没有再开口说一句话。
\par “戴钢盔的圣卢,”布洛克说,“这鸭子大腿很肥,著名的家禽献祭者又在上面洒满了祭奠的红酒,来,再吃点!”
\par 一般来说,老布洛克先生为儿子一个杰出的伙伴,抛出了关于鲁弗斯·以色列爵士及其他人的故事以后,感到儿子已经感激涕零,便自行撤退,以便不要在“中学生”面前“破坏自己的形象”。不过,如果有什么特别重大的理由,例如他的儿子通过了考试,布洛克先生便会在惯常的轶事系列之上增加一个讽刺性的感想。这个节目,更确切地说,他是保留给自己的私人朋友的。小布洛克见到父亲为自己的朋友表演这个节目,为此而感到极度骄傲。只听得老布洛克说:“政府简直不可原谅,竟然没有征求戈克兰先生\footnote{波努阿贡斯当·戈克兰(1841—1909),为法兰西喜剧院极有威望的演员之一。1897年,他成功地上演了爱德蒙·罗斯当的《西拉诺·德·贝日哈克》一剧。}的意见!戈克兰先生已经告知,他对此极为不满。”(布洛克先生自吹是反动分子,非常看不起戏子。)
\par 老布洛克为了表示自己对儿子的两个“拉巴登丝”\footnote{暗指拉比什的喜剧《鲁西纳街公案》(1857)。该剧叙述拉巴登丝寄宿学校两个同学所碰到的倒霉事。此处“拉巴登丝”成了“老同学”的代名词。}郑重其事到底,吩咐送上香槟酒来,并且马马虎虎地宣布,为了“招待”我们,他已经为一个喜剧剧团当晚在游乐场的演出订了一个楼下前排座。听到这话,各位布洛克小姐和她们的哥哥满面红光,这简直太出乎他们意料了!老布洛克为未能搞到包厢而遗憾。所有的包厢全让人租去了。再说,他经常光顾包厢,坐楼下前排更舒服。只是,如果说儿子的缺点,即他的儿子以为别人看不见的东西是粗俗的话,父亲的缺点则是吝啬。他称之为的香槟酒,是他叫人用一个水瓶给大家斟的一种小汽酒;他称为楼下前排座的,实际上是正厅后座,票价较之便宜一半。他像相信奇迹一般坚信通过神的干预,不论在餐桌上,还是在剧场里(实际上所有的包厢都空着),人们都发现不了差异。
\par 布洛克先生让我们将嘴唇在平酒杯——他的儿子以“坡深且陡的火山口”这个名称来形容这酒杯——内浸了一下之后,又让我们欣赏一幅画。他是那么喜欢这幅画,以至把它随身带到了巴尔贝克。他对我们说,这是一幅鲁本斯的画。圣卢天真地问他画上是否有画家的署名。布洛克先生红着脸说,由于画框大小的缘故,他叫人将署名裁掉了。不过这无关紧要,反正他不想将画卖掉。然后很快就把我们打发走,以便专心致志去阅读《政府公报》。各期报纸充塞房间,他非看不可。据他说,这是“出于他在议会中所处的地位”使然。究竟这地位的确切性质如何,他并未对我们加以说明。
\par “我带一条围巾,”布洛克对我们说,“因为西菲洛斯\footnote{西菲洛斯为希腊神话中的西风神。}和波瑞阿斯\footnote{波瑞阿斯为希腊神话中的北风神。}正在争夺着盛产鱼类的大海,而且散戏以后我们只要耽搁一小会,就等得到紫红手指的厄俄斯\footnote{厄俄斯为晨曦女神,古希腊作家一般称她为“长着玫瑰色手指的女神”。}初放晨曦时归来。对了,”待我们走出门外,他向圣卢问道(我浑身发抖,因为我很快就明白布洛克用这种冷嘲热讽的口气谈论的人正是德·夏吕斯先生),“前天上午我看见你在海滩上跟一个身着深色上装的潇洒幽灵散步,那人是谁?”
\par “是我舅父。”圣卢回答,他被刺伤了。
\par 可惜,布洛克根本看不出应该避免说“蠢话”。他笑得弯了腰:
\par “恭喜恭喜,我本应猜想得到的,他非常‘帅’,又长了一张高贵人家的愚蠢面孔。”
\par “您完全大错特错了,他非常聪明。”圣卢怒气冲天地回击道。
\par “我很遗憾,如果这样,他就不够完整了。再说,我很希望与他相识,就这类人我肯定能描写出合适的机体来。看这个家伙走过去,真叫人心烦。不过我可以对漫画式的一面轻描淡写,对于一个热爱句子的造型美和镢子的艺术家来说,这漫画式的一面从根本上说是相当令人瞧不起的。请您原谅,他真是叫我捧腹大笑了好一阵。我要突出描写您舅父那贵族的一面,总的来说,他给人印象很深,而且继第一阵大笑过后,他依然给人风度翩翩的印象,使人难以忘怀。不过,”这次他是对我开言了,“有一件事,完全属于另一概念范畴,我想问问你。可每次我们在一起时,总有一位神,奥林匹斯山上的幸福居民,使我完全忘记了向你打听这件事。否则我早就打听到了,而且这个消息对我肯定非常有用。我在驯化外国动物的动物园遇见你同一个美人在一起,还有一位先生和一个长头发的小女孩伴着她。这位先生,我想在哪儿见过。可那个美人是谁呢?”
\par 我早就看出斯万太太不记得布洛克的名字,既然她对我说的是另外一个名字,而且她将我的同学视为某一个部的随员。后来我也从未想过要打听打听他是否进过那个部做事。但是,照斯万太太那时对我所说,布洛克曾经请人将自己介绍给她。那布洛克怎么会不知道她的名字呢?我简直惊讶得呆若木鸡,半天回答不上那问话来。
\par “不管怎么样,我恭贺你,”他对我说,“你大概跟她没有搅在一起。在那之前几天,我在环城火车上遇到她。她同意垂青你的奴仆,为他宽衣解带。我从未度过那样美好的时刻。不巧,我们刚要制订各种措施以再次见面时,有一个她认识的人不识时务,在倒数第二站上了车。”
\par 我一言不发,似乎这使布洛克先生感到不快。
\par “我希望借助于你得知她的地址,”他对我说,“并且每周数次到她家去品尝厄洛斯\footnote{厄洛斯是希腊神话中的爱神,即罗马神话中的丘比特。}的快乐,神仙们也珍视这种快乐的。不过我并不坚持,既然你装模作样要为一个职业妓女保密。她在巴黎和日角之间,一连委身于我三次,而且非常风流。哪天晚上,我一定会找到她的。”
\par 这次晚餐之后,我又去看望布洛克。他来访问我,可我出去了。他要求见我时,被弗朗索瓦丝看见。虽然他来过贡布雷,但是不巧,弗朗索瓦丝直到那时从未见过他。所以她只知道一位我认识的“先生”来看过我,她不知道“为何而来”,那个人衣着一般,并没有给她留下很深的印象。弗朗索瓦丝对社会的某些看法我一直是搞不大懂的,可能一部分看法是建立在对一些词义的混淆上。一些名词,她有一次把这个当成那个,从此一直混淆下去。这些事我很清楚,很久以来在这些情况下我已经不再费力气去琢磨,但我还是情不自禁地其实是白费力气地去研究一下,布洛克这个姓对弗朗索瓦丝来说,究竟意味着什么了不起的东西。
\par 我刚对她说,她远远看见的那位青年人是布洛克先生,她便后退了几步。她是那样的惊讶,那样的失望!
\par “怎么?布洛克先生,就这样?!”她惊恐万状地大叫起来,似乎一个如此有威望的人物应该具有一种外表,“叫人立即知道”站在自己面前的是一位地球上的大人物。她就像觉得一个历史人物名不副实一样,用激动而又使人感到全球怀疑主义即将萌芽的口气反复地说:“怎么?布洛克先生就这样!啊!看见他,可真想不到他就是!”她那模样,似乎对我怀恨在心,好像是我什么时候在她面前“过高树立了”布洛克的形象。不过她还是好心地加了一句:“嘿,就算他是布洛克先生吧,我家先生可以说自己和他一样俊。”
\par 她对圣卢喜欢得不得了。过了不久,她也经历了一场性质不同的幻想的破灭,但持续的时间较短:那就是她得知圣卢是共和主义者。例如谈到葡萄牙王后时,她说“阿梅莉,菲利浦的妹妹”\footnote{这里是指德·巴里斯伯爵的女儿阿梅莉·德·波旁奥尔良,她生于1865年,1886年嫁给卡洛斯王子。1889年卡洛斯一世登上王位,她成为葡萄牙王后,至1908年其夫被暗杀。她的哥哥菲利浦是奥尔良公爵,路易菲利浦的侄子。},口气不大恭敬,但对老百姓来说,这是最高的恭敬。虽然如此,弗朗索瓦丝仍是个保王党。但是,一位侯爵,一位使她头晕目眩的侯爵赞成共和国,她似乎觉得太不可思议。她对此很为气恼,就像我送她一个盒子,她以为是金的,对我千谢万谢,后来珠宝商告诉她这个盒子只不过是镶金的,她很气恼一样。她立即收回了自己对圣卢的尊重。不过很快又还给了他,因为她考虑过了:作为圣卢侯爵,他不可能是共和主义者。他是出于利害考虑,只装装样子,因为从现在掌权的政府来说,这样可以给他带来许多好处。从这天起,她对圣卢的冷淡,对我的气恼都停止了。她谈起圣卢时,总是说:“他是个伪君子。”并善意地舒畅地微笑着,叫人完全明白,她又和第一天一样“看重”他,而且原谅他了。
\par 与此相反,圣卢的诚恳和不追求物质利益是绝对的。这种高度的道德纯正从爱情这样的自私情感中无法得到完全满足,另一方面在他自身也没有遇到除了在自身以外便找不到精神食粮的问题,而这个问题在我身上是存在的。正是这种高度的道德纯正使他能够承受友谊,正像我无法承受友情一般。
\par 弗朗索瓦丝说,看上去圣卢对于平民百姓倒没有瞧不起的样子。她这样说又是大错特错了。事实并非如此,只要看看他对自己的车夫如何大发雷霆就可以明白。确实,有时罗贝非常粗暴地斥责他的车夫。这证明,他心中对阶级差异的感觉远远胜过对阶级平等的感受。
\par “可是,”我责备他对这个车夫有些粗暴时,他回答我说,“为什么我要装出和他文质彬彬谈话的样子呢?他难道不是跟我一样的人吗?他难道不是跟我的叔伯或堂兄弟们与我一样亲近吗?你似乎认为我应该对他以礼相待,像对一个下等人那样!你讲话完全像一个贵族!”他又轻蔑地加上一句。
\par 确实,如果说他对哪一个阶级有成见和偏见的话,这个阶级就是贵族阶级。他甚至难以相信一个上流社会的人会出类拔萃,却很轻易地相信一个平民百姓会出众超群。我对他谈起卢森堡亲王夫人,说曾经遇见她与圣卢的姑祖母在一起。
\par “傻瓜一个,”他对我说,“跟所有她的同类一样。说起来,她还算是我的表姐呢!”
\par 对于经常与他来往的人,他抱有某种成见。他难得到交际场合去。他在交际场合所持的那种可鄙的、敌视的态度,又使他的所有近亲对于他和一个女“戏子”保有暧昧关系更加伤心。他们认为这种关系对他简直是致命的,特别是因为这在他身上进一步发展了那种诽谤精神,坏思想,将他“引入歧途”,只等他完全“堕入底层”了。所以,圣日耳曼区的许多轻浮男子谈到罗贝的情妇时,嘴上非常无情。
\par “妓女干她们那一行,”人们说,“和别人一样值钱。可是这个女人,不行!我们绝不宽恕她!她对我们喜欢的一个人,干下了太多的坏事!”
\par 当然,他不是与烟花柳巷有瓜葛的第一个人。但是,别的男人是作为上流社会的人玩玩,他们继续以上流社会的人的身份去考虑政治问题,考虑一切。而圣卢,他的家人觉得他“学坏了”。他家里的人意识不到,对许多上流社会青年来说,如果没有这种经历,他们思想上仍是未开化的,在友谊方面仍是粗糙的,没有温情,没有味道。而他们的情妇常常是他们真正的先生,这种男女关系是他们更高级文化入门的唯一道德学校。在这里,他们可以得知要交上排除利害关系的朋友要花什么代价。甚至在下等民众中(论粗野的话,这下等百姓与上流社会常常是那样相似),女人更敏感,更细腻,更闲来无事,对于某些高雅的东西也迫不及待要了解,对于某些情感美和艺术美也很尊重。她虽然不太理解这些东西,但是她把这些放在金钱与地位之上,而这两样似乎是男人最向往的东西。
\par 不论是像圣卢这样的俱乐部青年成员的情妇,还是一个年轻工人(例如,电工如今已列入真正骑士的行列之中)的情妇,情夫对她无比崇拜,无比尊敬,必定会将这种崇拜与尊敬扩展到她本人欣赏和尊重的事物上去,而对他来说,价值的阶梯便倒了一个个。她的性别本身决定了她很柔弱,会有无法解释的神经混乱。如果是一个男子,甚至是另一个女子,是她的姑母或表姐,这些表现都会使这个健壮的年轻人一笑置之。但是,对自己心爱的人,他不能眼看她受痛苦折磨。像圣卢这样的年轻贵族有了一个情妇,会养成到酒馆与她用晚餐时口袋里带上缬草精的习惯,说不定她会需要;会养成习惯坚决而又不带讽刺意味地叮嘱侍者注意关门不要发出声响,不要在桌子上放置潮湿的苔藓类植物,以免引起女友的不适,而他自己从未感受过这种不适。对他来说,这构成了一个隐秘的世界,她教他学会了相信这个世界确实存在。现在,他用不着自己去感受这种不适的滋味,便可怜起这种病症来。将来即使遇到别人感到这样的不适,他也会产生怜悯之情。
\par 圣卢的情妇——像中世纪最早的基督教教士一样——教他学会了可怜动物,因为她酷爱动物,走到哪里都随身携带着自己的小狗、金丝雀和鹦鹉。圣卢怀着母爱照看这些小动物,而把不善待动物的人看成是野蛮人。另一方面,一个女演员,或者所谓女演员,就像与他一起生活的那个女人那样——她聪慧与否,我完全不知道——使他感到上流社会的女人圈子是多么令人厌倦,使他把必须到那里去参加晚会视为一项苦役,就已经使他免受附庸风雅之苦并治愈了他的轻浮症。多亏了她,上流社会的交往在情夫的生活中地位更小了。反过来,如果他只是一个出入沙龙的男子,肯定是虚荣或利害关系来主导他的交友,正如这些友谊关系必然会打上冷酷的烙印一样。而情妇教会他在友情中注入高尚和细腻的情感。她更欣赏男人的某些细心周到,如果没有她,情夫对此很可能不理解或者加以嘲笑。再加上她那女性的本能,她一直能很快地在圣卢的朋友中间分辨出哪一位朋友对圣卢有真正的感情,并能很快地更喜欢这位朋友。她善于促使圣卢对这位朋友感到感激之情,并向他表示出这种感情,注意到什么事情使这位朋友高兴,什么事情使这位朋友难过。很快,圣卢便开始再不需要她的提醒,便能照应到所有这一切了。他的情妇并不在巴尔贝克,她也从来没有见过我,甚至在信中圣卢可能还没有谈起我,他便主动地将我坐的马车的窗子关好,把使我难受的花拿走。当他临走要向好几个人同时告别时,他能安排好先离开他们一会,以便单独最后跟我在一起,这样来显示那些人与我之间的区别,以表示对我、对别人有所不同。
\par 他的情妇开阔了他的精神,使他看到肉眼看不见的东西,她在他的生活中注入严肃认真,在他的心中注入了高尚的情感。但这一切,圣卢的家庭是看不见的,他们眼泪汪汪地反复说:
\par “这个婊子定会要了他的命,在这以前还要他丢人现眼。”
\par 总之,他从她那里吸取了她能使他得到的一切优良品质,这是确切无疑的。而现在,她成了他不断痛苦的缘由,因为她讨厌他了,而且在折磨他。有一天,她突然开始觉得他愚蠢可笑了,因为她在年轻剧作家的男演员群中的朋友向她保证说圣卢是愚蠢可笑的,她也就人云亦云,那种狂热和毫无保留,正是人们接受来自外界的见解或接受自己完全不了解的风俗习惯时所表现出来的劲头。她像那些喜剧演员一般,心甘情愿地鼓吹什么她与圣卢之间有不可逾越的鸿沟啊,因为他们完全是另外一种人哪,她自己是个智力型的人,而他,不管如何自诩,天生就是智慧的敌人哪等等。她这种看法似乎根深蒂固,而且到情夫最无足轻重的话语中、最细小的举动中去寻找证明。此外,还是这些朋友对她说,本来,为她而难得形成的那个圈子的人对她寄予很大的希望,可现在,她正在摧毁这些希望,说她的情夫最后肯定会感染她,说与他一起生活,她会毁掉自己艺术家的前程等等。待她被这些人说服之后,便在对圣卢的蔑视上又加上了仇恨。如果圣卢非要叫她染上一种致命的疾病,她也不过如此恨他而已。她尽量与他少见面,同时又不断推迟最后决裂的时刻,在我看来,这最后决裂不大可能。圣卢为她作了这样大的牺牲,她要找到也同意作出同样牺牲的第二个男人,看来不那么容易,除非她有倾国倾城之貌(圣卢从来不愿意将她的照片给我看,对我说什么“首先,她并不是什么美人;其次,她又不上照。这都是我自己亲自用我的柯达克\footnote{最早的柯达克相机出现于1888年。此后,“柯达克”很快就成了“相机”的代名词。}为她拍的快速曝光照片,给你看了,会使你对她产生一个错误的概念”)。
\par 我不相信,甚至对于一个轻佻女人,自己根本没有才华,又有出名的狂热欲望,加上一些人强加于你的个人尊重(说不定圣卢的情妇还不属于这种情况),就能成为比赚钱的快乐更有决定意义的动机。圣卢对于自己的情妇脑子里到底是怎么回事并不清楚,对他的不公正的责备也好,永恒相爱的诺言也好,他都认为不完全真诚。可是在某些时候,他又感到,到她能够与他断绝关系时,她会断然实行。因此,大概出于想保住自己爱情的本能,这种本能可能比圣卢本人更明智,他用了很实用的一技。这一技与他心中最伟大而又最盲目的激情融成了一体。那就是他拒绝给她立一份本金,他借了很多钱,以便她应有尽有,但是只是一天一天地交给她。如果她确实想到要离开他,大概也要冷静地等待到“发财”之后。从圣卢给的钱数来看,大概需要不了多长时间。但是无论如何,这又补充了一段时间,可以延长我这位新朋友的幸福——或痛苦。
\par 他们关系的这一戏剧性阶段现在达到最尖锐的程度。对圣卢来说,这是最残酷的阶段,因为她不许他待在巴黎,她一见他就恼,迫使他到离自己驻地不远的巴尔贝克来度假。这个阶段是一天晚上在圣卢的一位姑母家里开始的。那天,姑母家有许多客人,圣卢得到姑母同意,让他的女友前来为客人表演一个象征主义剧本的片断。她曾在一家先锋派剧院里演过一次这个戏,而且圣卢也同意了她自己对这个戏的赞美。
\par 她出现了,手里拿着一大朵百合花\footnote{在中世纪宗教画里,圣母玛利亚几乎总是手持一朵百合花。天使向她宣告她将生一个儿子的时候,她回答道:“我是上帝的奴仆。”},服装是仿效《上帝的奴仆》\footnote{可能指的是但丁·加布里埃尔·罗塞蒂的画《上帝的奴仆》(1850)。}。她说服了罗贝,说这套衣服有真正的“艺术眼光”。在这个贵族俱乐部男子和公爵夫人聚集的人群里,她一上台,迎接她的就是一些人的冷笑。她那念经一般的单调语气,某些莫名其妙的字眼,这些字眼又频繁地出现,将冷笑变成了哄堂大笑。刚开始,人们还强忍不要笑出声来,后来竟是那样不可阻挡,以致可怜的朗诵者无法继续下去。
\par 第二天,圣卢的姑母受到一致谴责,说她竟然让这样荒谬可笑的女戏子在她家中出现。一位著名的公爵毫不掩饰地对这位姑母说,她受到批评,是咎由自取。
\par “见了鬼了,给我们来个这种劲头的节目!如果这个女人有点才华,倒也可以,可是她没有才气,而且永远也不会有一点点!见鬼!巴黎人可不像人们想说的那么愚蠢。上流社会不是光由蠢货组成的。这位年轻小姐显然以为她会叫巴黎大吃一惊。可是巴黎可不那么容易吃惊,毕竟有些事,是无法叫我们忍下去的。”
\par 至于说到那位演员嘛,她走出房门时对圣卢说道:
\par “你把我引到什么人家里来了?都是傻瓜,笨蛋,没有受过教育的小丑!我告诉你吧,在场的男士中,没有一个没向我丢过眼风,跺过脚,我拒绝了他们对我的追求,他们现在便设法进行报复!”
\par 这一席话把罗贝原来对上流社会人等的恶感变成了夹杂着痛苦的深仇大恨,最不该恨的一些忠心耿耿的亲戚,尤其叫他恨得咬牙切齿,因为家里人委派他们去说项,设法说服圣卢的女友与圣卢断绝关系。女友在他面前将这种活动说成是那些亲戚出于对她倾心才这么做的。虽然罗贝立即与这些亲戚断绝了来往,但是当他像现在这样远离女友时,他想,也许这些人以及其他人会利用他的远离卷土重来向那个姑娘求爱,说不定已经得到她的青睐;他谈起那些欺骗自己的朋友,引诱妇女,竭力将女人弄到妓院里去的混世魔王时,满面痛苦和仇恨。
\par “我宰一条狗都比宰了他们还要悔恨,狗毕竟是乖顺、效忠、忠诚的动物。这些人就该上断头台!比起那些因为自己贫穷和富人不义而被逼走上犯罪之路的可怜人来,他们这些人更坏!”
\par 他大部分时间都用来给情妇寄信,发电报。她一面阻止他到巴黎去,一面还在远距离想方设法与他闹别扭。每当发生这种事,我都能从他那变了模样的面孔上得悉。他的情妇从来不告诉他,她到底对他有什么不满。圣卢猜想,她之所以不对他讲,说不定她自己就不知道有什么可以不满的,而只是对他厌倦了。他仍希望得到一些解释,便给她写信:“我什么地方不好,请你告诉我。我随时准备承认自己的错误。”他那么伤心,结果是确信自己做得不对。
\par 她总是叫他无限期地等待答复,而那些答复都是没有意义的。所以我看见圣卢从邮局回来,几乎总是眉头紧皱,又常常是两手空空。整个旅馆的人里面,只有他和弗朗索瓦丝到邮局去取信或亲自送信。他是出自情人的迫不及待,弗朗索瓦丝则是出于对仆人不信任(为打电报,他不得不走还要多得多的路)。
\par 在布洛克家进晚餐之后,过了几天,外祖母兴高采烈地告诉我,圣卢刚才问她,愿不愿意在他离开巴尔贝克之前为她拍几张照。为此,她穿上了自己最漂亮的衣裳,为几顶不同的帽子该戴哪顶而拿不定主意。看到这种情况,我感到有点冒火,真料想不到她竟会有这样的孩子气行为。我甚至自忖是否我看错了外祖母,是否我将她看得太高了,是否她并不像我一向认为的那样对有关自己相貌的一切都很淡然,她是否也有些卖弄风骚,而我一向认为这是与她绝对格格不入的东西。
\par 要照相,特别是看上去我外祖母对此那么心满意足,引起我的不满。可惜的是,我这种情绪流露得相当明显,弗朗索瓦丝注意到了,急急忙忙给我来了一套令人感动的情感说教。我根本不想装出同意那套说教的样子,她这样不知不觉地更增加了我的不满情绪。
\par “噢,先生,可怜的太太,人家给她照个相,她会多么高兴!她还要戴上老弗朗索瓦丝亲自给她整理好的帽子。应该让她去照,先生。”
\par 想起在各方面是我的理想人物的我母亲和外祖母也常常嘲笑弗朗索瓦丝的过敏,我确信我那样嘲笑她并非挖苦。可是外祖母发现了我神色不快,便对我说,如果这次照相会使我不悦,她就不照了。
\par 我没同意,向她保证,我认为没有任何不合适的地方,任她去打扮自己。但我对她说了几句冷嘲热讽、刺人的话,目的是要打掉看上去她为拍照而感到的兴高采烈,我觉得这样也就表现出自己洞察能力很强,也很强硬了。结果是,虽然我不得不看外祖母那漂亮至极的帽子,至少我让那兴高采烈的表情从她脸上消逝了。本来这种表情应该叫我高兴,可是只要我们最喜爱的人还活在人世,就常常发生这样的事情,就是我们觉得那种表情是低下的怪癖的表现,叫人着恼,而没有将那看成是我们多么希望给他们带来的幸福,而那就是幸福的宝贵表现形式。
\par 我的心情不好,主要是由于那个星期外祖母似乎总躲着我。白天也好,晚上也好,我未能有片刻时光单独跟她在一起。下午我回到旅馆,想跟她单独在一起待一会儿时,人家告诉我说,她不在。要么她就是关起门来与弗朗索瓦丝长时间窃窃私语,不许我去打扰。在外面与圣卢一起度过晚上以后,回去的路上,我就想着就要重见外祖母并且亲吻她的那一时刻。我等待着她在隔壁墙上轻轻敲几下,叫我过去向她道晚安。但是我徒劳等待,听不见一点声音。最后我便上床,有点怨恨她,她毫不在乎地剥夺了我看得很重的快乐,这种毫不在乎可是新近才有的。我仍像童年一样,心儿剧烈跳动,一直倾听着墙壁发出声音。墙壁始终一言不发,我流着泪进入梦乡。\footnote{下面开始,可视为《在少女们身旁》的第三部分。第一次出版时,下面打有三个星号。此处只以空两行表示。}
\par 那天,像前几日一样,圣卢不得不到东锡埃尔去。在他还没有最终完全回去之前,很可能直到晚上那里一直需要他,他不在巴尔贝克,我很遗憾。我看见一些少妇,远远望去,觉得她们令人心醉。她们从马车上走下来,有的进了游艺场的舞厅,有的进入冷饮店。我正处在年轻人的那样一个阶段,就是还没有一个具体的爱恋对象,心里还空着。在这样的阶段,就像一个堕入情网的人向往着、寻求着他钟情的女人一样,年轻人到处向往,到处寻求,到处看见美人儿。只要有真实的一笔——远远望见一个女子,或只见背影的一个女子,哪怕分辨出一点点模样——就可以叫我们设想出在我们前头的美人是什么模样,我们想象自己认出了她,心儿在剧烈跳动,脚步也加快了。只要那女子消逝了,我们便一直半信半疑到底是不是她;只有能追上她的时候,才会明白我们是大错特错。
\par 再说,我的身体越来越不舒服,就更受到诱惑,将最简单的享乐更加夸大,因为我很难接触到女性。风雅标致的女郎,因我在任何地方都不能与她们接近,便觉得随处可见。如果是在海滩上,则因为我身体太衰弱。如果是在游艺场或糖果店里,则因为我过于腼腆。不过,如果我很快就要死去,我真希望知道,生命能够提供的最漂亮的少女在现实生活中究竟是怎样造就出来的。不管怎么说,将是我之外的另一个人,抑或竟没有任何人能够享受这种供给(事实上,我意识不到,在我这种好奇的根源上,就有着占有的欲望)。如果圣卢与我在一起,也许我就敢进舞厅了。但我是一个人,我只好呆立在大旅社门口,等待着与外祖母会齐的时刻到来。就在这时,几乎在大堤的尽头,我看见五六个小女孩向前走过来,在大堤上形成一片移动的奇异的印痕。无论是外貌还是举止,她们都与人们在巴尔贝克司空见惯的所有姑娘不同。一群海鸥不知来自何处,正在海滩上不紧不慢地踱着方步,姗姗来迟者飞来飞去,追逐着别的海鸥。鸟儿飞来飞去,目的地似乎与洗海水浴的人一样不明确。鸟儿似乎没有看见洗海水浴的人,同时对于它们那鸟类头脑来说。这目的地又是明确规定了的。只有那群海鸥大概对这些鸟儿已司空见惯了。
\par 这些陌生女孩中,有一个手推着自己的自行车。另有两个,手里拿着高尔夫球“俱乐部”球衣。她们的短打扮与巴尔贝克其他少女截然不同。其他少女中确实也有几位从事体育运动,但并不因此就采用专门装束。
\par 这正是各位先生太太每天到堤上来转一圈的时刻,他们都暴露在对着他们定睛细看的手持长柄眼镜的无情火力之下,似乎他们身上有什么毛病,那长柄眼镜非要将每一细部都审视清楚一般。首席法官的老婆骄傲地坐在音乐亭前那令人生畏的一排椅子中间。他们自己刚刚从演员变成评论家,走来坐下,该他们对面前走过的人评头品足了。所有这些人都沿海堤走着,似乎这海堤如同一只船的甲板一般摇摇晃晃(因为他们不会抬起一条腿时同时晃动手臂,转动眼睛,放平肩膀,用相反方向晃动的动作来平衡他们刚才在另一侧所做的动作,并叫脸上充血),装出什么都没看见的模样,以便叫人相信他们对这几个女孩根本不在意。实际上却在对她们偷偷地凝望,以免撞上她们。走在她们身边或从反方向来的人,相反却撞在她们身上,紧追不舍,因为他们双方都是彼此暗暗注意的对象,虽然双方都用同样的轻蔑来掩盖这种注意。
\par 对人群的喜爱——因此也是对人群的恐惧——在每个人心里都是最强有力的动机之一。或者极力讨别人喜欢,或者叫别人惊奇,或者极力向别人表现出自己很看不起他们。在蛰居者心中,绝对甚至直至生命终结的监禁,其缘由常常是对人群有一种失常的嗜好。这种嗜好会那样压倒任何其他的情感,以致由于外出时无法得到门房、行人、停车的车夫的赞美,他宁愿永远不叫他们看见,于是便放弃了一切必须外出的活动。
\par 这些人中,有几个正在沿着某个思路思考,但是通过手势急促,目光走神,与他们的邻人那考虑周到的摇摇晃晃的步伐不相谐,而暴露了自己的思想活动。我远远看见的几个女孩,在所有这些人中,径直前行,身体完全放松,对其余的人类发自内心的蔑视赋予她们动作自如,毫不犹豫,也不僵硬,准确地做出她们想做的动作,四肢每一部分对其他部分而言都完全独立自主,身体的大部分保持不动。华尔兹舞行家就是这样,那是非常精彩的。虽然她们当中每个人都是一个类型,与他人类型不同,但是这几个人无一例外,全都姿容姣好。不过,说老实话,我看见她们才这么一小会工夫,而且还不敢定睛凝望,我还没有抓住她们之中哪一个的个性。有一个除外,她那笔直的鼻梁,棕色的皮肤与他人形成鲜明对照,与文艺复兴时期某一幅画上朝拜初生耶稣的三王之中,那位阿拉伯人模样的人肤色相近。我对她们的了解,一个,仅仅是通过那一双不大灵活、固执而又带着笑意的眼睛;另外一个,仅仅是通过那粉红的双颊。那粉红中又带着一抹镀铜的色调,不禁使人想起绣球花。甚至就是这些面部特点,我也还无法将任何一种特点分别固定在这一个少女而不是另一个少女身上(这个整体是那样优美动人,最不相同的外貌相邻,各种色彩相聚,又像一首乐曲那样叫人难以捉摸。乐句一个个过去的时候,我无法将一句句分开,一句句辨认出来,待我分辨出来以后,马上又忘记了。按照这个整体行进的顺序),我看到一个白色的椭圆形,黑眼睛、绿眼睛相继出现,我不知道她们是不是就是刚才已经对我产生了魅力的姑娘,我无法将看到的东西归到我从他人中分别出来、辨认出来的哪一个少女身上。在我的视野中,没有分界线(过了一会我才弄清了她们之间的区别),透过她们这一组人,一种和谐的浮动在扩展,是液体美、集体美和动态美的持续转移。
\par 个个挑选得这么漂亮,将这几个朋友聚集在一起的,在生活中,可能并非纯属偶然。估计这几个少女(她们的态度足以揭示出大胆、轻浮和狠心的天性)对任何滑稽可笑的事和任何丑陋都极为敏感,接受不了德或智方面的吸引,便在她们同龄的同伴中,自然而然地聚在一起。对于那些通过腼腆、拘谨、笨拙以及她们大概称之为“讨厌的类型”而透露出沉思或敏感的天性的所有女伴,她们感到厌恶并置之不理。相反,风雅,灵活,体态优美的某种混合,将她们吸引到别的人身旁,她们与这些人结成友谊。她们那具有诱惑力的直爽和与她们一起度过幸福时光的允诺,只有通过这唯一的方式才能表现出来。她们属于什么阶级,我无法准确判断出来,说不定那个阶级正处于其发展的这个阶段,或者由于富有和闲暇,或者由于进行体育运动(这是一个新习俗,甚至在某些民众阶层也已普遍),但是在体育之上尚未加上智育,这个社会阶层有如尚未追求扭曲表现形式的那些和谐而又多产的雕塑学校,自然而然地而且大量地生产出美丽的躯体,优美的大腿,优美的臀部,圣洁而安详的面庞,表情机敏而又富有智谋。我在这里,面对大海看见的,难道不是人体美高尚而又平静的模特儿吗,犹如希腊某海岸上那些暴露在阳光下的雕像?
\par 她们这一群,如闪光的彗星,沿着海堤,向前行进。即使她们认为四周的人群由另一个种族组成,甚至他们的痛苦都不会在她们心中唤起同情,但表面上她们似乎没有看见人群。她们迫使停步的人让路,好像突然有一台机器通过,不能期望机器躲开行人一般。对一位年迈的先生,她们是不承认他的存在,拒绝与他接触的。如果这位先生心怀恐惧或怒气冲天但又匆匆忙忙而又可笑地逃开,她们最多也就相视而笑罢了。对于不属于她们这一群的人,她们没有故作轻蔑,她们内心的轻蔑已经足够。但是她们每遇障碍,都无法不以克服障碍为快,或者冲过去,或者双脚并拢,因为她们个个都充满青春活力,是那样需要发挥出去,以至即使在悲伤或痛苦的时候,也是更服从年龄的需要而不是当日的心情。她们从不放过一次跳跃或打滑的机会,而又不是有意识地这样干,只是打断缓步前行,在缓步前行中撒播上优美的转弯,心血来潮与高度的技巧合二而一,正如肖邦在他最忧郁的乐句中撒播上优美的曲线一般。
\par 一位年迈的银行家,他的老伴正在为他寻找好地方,在好几处都未下定决心。最后,叫他面对海堤坐在一个折叠小凳上,有音乐亭为他遮住海风和烈日。老伴见他坐好了,便离开他去买报纸,准备过一会读给他听,叫他消遣消遣。只不过走开一小会,她也就将他单独留在那里。这一小会从不超过五分钟,对老头来说似乎已经相当长。老太太对自己的老伴既悉心照料,又不表露在外。她经常这样走开五分钟,好让老伴觉得自己还能像所有的人一样生活,而决不需要保护。他头顶上的音乐家表演台,构成了一个天然而又有诱惑力的跳板,那一小群少女中年龄最大的一个毫不犹豫地朝表演台跑过来。她从老头头顶上跳了过去,灵巧的双脚擦着了老头海军帽的边缘。老头吓得面如土色,可是另外几个姑娘觉得实在好玩,特别是绿眼珠、娃娃脸的那一个。她的目光中,表现出对这一行为的钦佩和快活。我似乎从她的眼睛里辨出少许的腼腆,既害羞又假充好汉的那种腼腆,这种表情在别人脸上是没有的。
\par “可怜的老帮子,真叫我心里难受,简直半死模样!”其中一个少女说道,嗓音嘶哑,半嘲讽的语气。
\par 她们又向前走了几步,然后在路中间停步一小会,也不顾挡住了行人的来往,呈形状不规则、完整、奇特而又叽叽喳喳的一个集合体,像起飞前聚在一起的一群小鸟。然后她们沿着高出海面之上的海堤继续漫步下去。
\par 现在,她们那迷人的面庞再不是模糊不清、相互混淆了。以个子最高、从老银行家头顶上跳过去的那个为中心,我已经将她们区分和聚集起来(每个人的名字暂缺,我不知道)。小个的从海平面上分离出来,双颊丰满而粉红,绿眼珠;另一个皮肤为棕色,鼻子笔直,与其他人形成鲜明对照;还有一个,面孔雪白像个鸡蛋,鼻子形成一个弓形小弯,好似鸡雏的嘴,她的面孔与某些年纪很小的人相似;还有一个,大个子,裹着一件斗篷(这件斗篷使她显得那么穷酸,与她那优雅的举止那样不相称,以至来到人们头脑里的解释是:这个少女的父母大概地位相当显赫,但是他们的虚荣心远在巴尔贝克洗海水浴的人之下,也在自己孩子的衣着是否华丽之下,所以让她穿什么衣服在海堤上散步,对他们来说绝对一样,小市民才会认为这衣裳穿着太寒酸);还有一个姑娘,双眸明亮而又含笑,颧骨很高,皮肤无光泽,头戴一顶黑色马球运动员式女帽,压得很低。她推着一辆自行车,臀部扭动得好像骨头都脱了节,使用的行话俚语那么粗野,叫嚷的嗓门那么大,我从她身边经过时(从她那些词语里,我听见一句难听的“混他的日子”),便放弃了刚才她的伙伴的斗篷令我作出的假设,而更倾向于得出结论说,所有这些女孩都属于经常光顾赛车场的那帮小民,大概是自行车运动员们最年轻的情妇。总而言之,我的假设中,没有一个认为她们可能是贞洁的。看上一眼——从她们彼此相视而笑的样子,从双颊无光泽那个姑娘那紧盯不放的目光里——我就明白了,她们不是贞洁的女子。加之,外祖母一直过于谨小慎微地悉心照顾我,以至我不会不相信,不可为之事是不可分的整体,对老年人缺乏尊重的少女,碰到从八十岁老翁头顶上跳过去以外的更有诱惑力的快乐时,决不会骤然间为顾忌之心所阻拦。
\par 现在,她们一个个都有了自己的个性。她们的目光因自我满足和伙伴义气而变得炯炯有神,眼中不时燃起兴致勃勃或狂妄而满不在乎的火光,视对象为自己的女友或路上行人而定。她们相互之间了解相当深入,能够一直一起散步,形成“独立大队”。目光相互传递着话语,意识到彼此相互了解,随着她们那独立而彼此分开的身躯缓缓向前,在这些身躯之间注入了一种联系。这种联系虽然肉眼看不见,却很和谐,好似同一个火热的身影,同一个氛围,使她们的身躯合成了一个整体。这整体的各个部分是同质的,而对这一行列在其中缓缓行进的四周人群,又无动于衷。
\par 我从那个颧骨很高、推自行车的棕色皮肤姑娘身边经过。有一瞬间,我的目光与她那斜睨的笑盈盈的目光相遇。这目光来自将这个小部落的生活封闭其中的非人世界的深处,那世界是无法接近的未知数,我是什么人这个想法,肯定达不到那个世界,在那里也找不到位置。这个头戴运动帽、帽子在脑门上压得很低的姑娘,全神贯注倾听同伴们说话。她双眸中闪现出来的黑色光芒与我相遇的那一刻,她是否看见我?如果她看见了我,我对她又意味着什么?她辨别出我属于哪个世界了吗?这些问题我难以回答,好比借助于望远镜,在相邻的一个星球上,某些奇怪的生物出现在我们面前时,我们很难就此得出结论说,有人类居住在那里,他们看得见我们,看见了我们又会在他们心中唤起什么想法。
\par 如果我们认为,这某某姑娘的双眸只不过是发亮的云母圆片,我们就不会贪婪地要了解她的生活并且将她的生命与我们结为一体了。但是我们感觉到,在这个反光圆体中闪闪发光的东西,并非只源于其物质结构。我们感觉到,这是这个生命对于它了解的人和地点——赛马场的草地,小径上的沙土——所形成的看法的黑色投影。这黑色投影是什么,我们还不了解。这个小贝里,比波斯天堂中的贝里\footnote{在波斯神话中,贝里是天堂的使者,手执象征永生的荷花。普鲁斯特此处可能想到了根据保罗·杜卡斯的诗作而创作的芭蕾舞《贝里》,1912年由俄国芭蕾舞团在巴黎演出,娜塔莉亚·特鲁哈诺娃编导。舞剧中有贝里引诱伊斯康德王子,王子夺走她的荷花,她返回天国的情节。}对我更有诱惑力。她蹬着车穿过田野和树林,可能会把我带到那些地方去。我们感觉到,她那目光也是她就要回去的家、她正在形成的计划或者人们已经为她作出的安排的投影。我们尤其感觉到这就是她本人,怀着她的欲望,她的好感,她的厌恶,她那朦朦胧胧、断断续续的意愿。我知道,如果我不能占有她目光中的东西,我就更不能占有这个骑自行车的少女。因此,使我产生欲望的,是她整个的生命。痛苦的欲望,因为我感到这是无法实现的,也是令人心醉的欲望;直到此刻我的生命已骤然停止,已不再是我的整个生命,而是成了我面前这块空间的一小部分,我迫不及待地要将这空间占据,这空间乃由这些少女的生命组成。是这种欲望赋予我这种自我延伸,自我扩展,这就是幸福。无疑,我们之间没有任何共同的习惯,共同的思想,这使我更难与她们交友,讨得她们欢心。但是,说不定正是由于这种差异,由于意识到我所经历的、拥有的任何因素(成分)都不会进入这些少女的天性构成的行为,我心中才刚刚用对某种生活的渴求代替了心满意足——如干渴的大地那样干渴——迄今为止,我的心灵从未得到过一滴这样的甘露,它会更加贪婪地大口大口地吮吸。
\par 那个目光明亮的推自行车姑娘,似乎发现了我那样凝神望着她,便向那个个子最高的姑娘说了一句什么话。说的什么,我没有听见,只见那个高个子姑娘笑了起来。说老实话,这个棕色皮肤的姑娘,正因为她的皮肤是棕色,并不最讨我喜欢。从在当松维尔那陡峭的小山坡上见过希尔贝特那一日起,一个头发棕红、肤色金黄的少女,一直是我心中不可企及的理想。可是,就说希尔贝特本人吧,我之爱她,难道主要不是因为她戴着贝戈特女友的光环,和贝戈特一起去参观大教堂吗?同样,看见这个棕色皮肤的姑娘望着我(这使我刚开始时抱着希望,以为也许与她接触更容易些),我并不感到高兴,因为她会把我介绍给那个从老头头上跳过去的那个无情的姑娘,介绍给说“可怜的老帮子,真叫我心里难受”的那个残忍的姑娘,然后逐次将我介绍给每一个姑娘,因为她享有这种威望,是她们形影不离的朋友。我作了一个假设:有一天我会成为这几个少女中哪一个的男朋友。这些眼睛里那陌生的目光给我留下深刻的印象。她们自己并不知道,有时对我会产生阳光照在一堵墙上那样的效果。通过奇迹般的炼金术,这些眼睛也许会叫“我是存在的”这个想法以及对我个人的某些友情穿透它们那难以形容的立体。有一天,我本人也可能跻身于她们之中,在她们沿海边行走发挥的理论中占一席之地。我觉得这个假设本身就包含着一个无法解决的矛盾,就像站在阿堤刻时代的剧场前或面对着描绘宗教仪式行列的画幅,我也曾以为我这个观众也能受到诸神的喜爱,在列队行进的诸神中占据一席之地一般。
\par 那么,与这些少女结识的幸福,真是无法实现的吗?自然,在我放弃的这类事当中,这大概已经不是第一桩了。只要回忆一下,即使在巴尔贝克,就有多少陌生女郎,飞驰远去的马车便叫我永远放弃了她们,便已足够了。这一小群女孩,在我心中是那样高尚,仿佛由希腊神话中的处女组成。甚至她们给我带来的快乐,也来自她们有些路上行人飞快离去的味道。我们不认识的人,迫使我们从惯常生活中启碇的人,具有一种转瞬即逝性。这种转瞬即逝性使我们处于一种追逐状态中,再没有任何东西阻拦我们的想象。而在惯常生活中,我们与之经常来往的女子,最后都将她们的缺陷暴露出来。将我们的快乐剥去想象这层皮,等于将快乐压缩至其本身,就空无一物了。诸位已经看到,我并不蔑视拉线的中间人。但是这些少女如果到牵线人那里去自荐,她们便失去了赋予她们丰富多彩和捉摸不定的因素,就不会如此叫我着迷了。对于是否能够企及追求的对象没有把握,能唤起人的想象。必须叫想象创造一个目的,这个目的遮掩住另一个目的;必须叫想象用进入一个人的生活之中这种想法代替感官的快乐,以阻止我们去分辨这种快乐,阻止我们去品尝其真正的味道,阻止我们将其限制在本身范围之内。钓鱼的那些下午时光,在我们与鱼之间,非有翻腾的流水将我们隔开不可。光滑的肉,不明确的形状,在天蓝色透明而又活动的流体中,在我们身边滑来滑去,而我们不大知道该拿这玩艺儿干什么。如果我们第一次是看见那鱼做成了菜端上桌子,就会显得不值得千方百计、拐弯抹角去捉它了。
\par 在这里,社会地位所占比例发生变化,这是海水浴生活的特点。这些少女也占了这个便宜。在我们习惯的阶层中能使我们延伸、放大的一切优势,在这里,都变成了看不见的东西,事实上,也就被取消了。反过来,那些别人认为他们大概并不具有这些优势的人,倒被一个人工的范畴变得高大起来,大步向前了。这个人造的范畴比素未谋面的女郎叫人更自在。那一天,这些少女在我眼中显得那么了不起,而根本无法让她们了解我会有什么了不起的地方。
\par 对这一小帮少女来说,她们漫步海滨只不过是路上女客无数飞逝的一个片断,这种飞逝总是使我心绪纷乱。在这里,这种飞逝又回到那么缓慢的动作上去,几乎接近于停滞不动。更确切地说,在某一个这样慢速的阶段中,人的面庞不再被旋风卷走,而是平静而又清晰,我觉得就更美。但是,正像德·维尔巴里西斯夫人的马车将我飞快拉走时我的体验一样,这并不妨碍我想,如果我停下一会就近观看,某些细部,有麻点的皮肤啊,鼻翼上有个毛病啊,眼神很平庸啊,微笑时作鬼脸啊,身段不美啊,都会在女郎的面孔和身段上代替我原来肯定是凭空想象的细部。只要身段有美丽的曲线,远远望见面色很红润,我就能好心地再加上一直记在心底的或事先想好的动人的肩膀,甜美的顾盼。对一个一眼而过的人这样飞快的猜测可能使我们犯下错误,恰似有时看书太快,刚看见一个音节,还未来得及看清其余的音节,便从我们脑海中已有的字里,安上一个字,其实书上写的根本不是那个字一样。
\par 现在不可能属于这种情形。我已经仔细端详过她们的面庞。每个人的面孔,我不是从各个侧面看的,也极少从正面看,但至少根据两三个不同的特点使我足以对第一眼望去时对线条和肤色所做的各种假设或者进行修正,或者进行了核实和“证明”,足以看到,透过一系列的表情,她们的面孔上还存在着某种永久不变的物质的东西。
\par 因此我可以满有把握地想:无论在巴黎还是在巴尔贝克,在最美好的设想中,甚至在我能够停下脚步与之攀谈的令我目光停驻的行路女子中,都从来没有过像今年这几个女子这样,我根本就不认识她们,但是她们的出现和消失给我留下这样的惆怅,使我想到与她们交友会是多么令人陶醉。无论是在女演员中,村姑中,或在教会学校寄宿的小姐中,我从未见过如此的美貌,如此充满未知未闻,如此无法估计的宝贵,又这样令人难以置信地不可企及。就生活中未品尝过而又可能的幸福而言,她们是那样甜美的样品,且状态极其完好,以至几乎完全出于理智的原因我才灰心丧气,怕的是体验不了美女能够给予我们的最神秘的东西。我要在绝无仅有的条件下,保证不会上当受骗才会体验。她们是人们一直向往的美女,是人们永远不占有也可以自慰,而不会去向自己没有欲望追求的女人要求快乐的美人——正像斯万从前爱上奥黛特以前一直拒绝做的那样——结果是一直到人死了也从不知道那另一种快活是什么滋味。也许从未体验过的快乐事实上并不存在,也许到了跟前,这种快乐的神秘性就烟消云散了,也许这只是欲望的一种投影,一种海市蜃楼。如果是这种情形,那我只能责怪自然规律的无情。如果这种自然规律适用于这些少女,也应该适用于所有的少女,而不适用于不完善的对象。她们是我在所有对象中挑选出来的,我怀着植物学家那种心满意足的心情,很清楚地意识到不可能找到比这些少女更罕见的如此齐全的品种。此刻,她们就在我面前中断了她们那轻巧的篱笆般的流动线。这篱笆就像一丛宾夕法尼亚玫瑰\footnote{“宾夕法尼亚玫瑰”这个名称在某些植物学家的著作中可以见到,用以指美国东部的某一玫瑰品种。这个名称在普鲁斯特那个时代并不流行,只不过表现了普氏学识的渊博而已。},是悬崖上一处花园的装饰品。一艘轮船驶过的整个大洋航线均映在其中,这轮船在蓝色平面上滑行得那样慢,相当于从一个茎到另一条茎。一只懒惰的蝴蝶在花冠深处滞留,船体早已超过这只蝴蝶。可是蝴蝶确有把握能比轮船先到达目的地,那船只正向花朵驶去。蝴蝶可能还要等到轮船的船首与玫瑰花的第一个花瓣之间出现一片蓝色才起飞呢!
\par 我回房间去了,因为我要与罗贝一起去里夫贝尔共进晚餐。外祖母要求我最近几天晚上动身以前在床上躺一小时,小睡片刻,这是巴尔贝克的医生提出的要求。不久,他便把这样的小睡扩展到每一天晚上。
\par 再说,要回房间甚至不需要离开大堤,也不需要从大厅,也就是说从后面进入旅馆。在贡布雷,每星期六午饭提前一小时。现在这里正是盛夏,白天那么长,以至在巴尔贝克大旅社里,根据与此类似的提前规则,人们为晚餐摆放餐具时,太阳还高高挂在天上呢,似乎是吃下午点心的时刻。带滑轮的大玻璃依然开着,与海堤在同一平面上。我只要跨过单薄的木制窗框就到了餐厅里,然后我立刻离开餐厅去乘电梯。
\par 从办公室门前经过时,我向经理送过一个微笑,而且一点也不讨厌地从他脸上收来一笑。自从我到巴尔贝克以来,我那宽容的关切已经渐渐地像备自然课一样将微笑灌输到他的脸上,改造了他的面孔。他的面庞对我熟悉起来,显示出某种很一般的意义,但可以像辨认一个人的笔迹一样看懂,与第一天他的面孔向我显示的那些莫名其妙、无法忍受的方块字已经毫无相像之处。那一天我在面前看见的那个人物,如今已被忘却。或者说,如果我还能回忆起来的话,他与那个无足轻重而文质彬彬的人物那令人厌恶而又略微加以漫画化的形象相比,已经判若两人,无法认同了。
\par 我初来巴尔贝克那天晚上的那种腼腆和忧郁已经消失,我按铃叫电梯。在电梯里,我像沿着脊椎动物的胸腔一样,在开电梯的人身旁向高处升去。现在,他再不是默默无语了,而是向我叨叨:“人比一个月以前少了,开始走了,天凉了。”他这么说,并非因为确实如此,而是因为他在这海滨气候更炎热的一个地方又找了个事情做,他希望我们都赶快走,旅馆好关门,这样他“回到”新岗位之前,可以有几天归他自己支配。“回到”和“新”这两个词并不矛盾,因为对于一个开电梯的人来说,“回到”乃是“进入”这个动词的惯用形式\footnote{在法文中,受教育不多的人常常将“entrer”(进入)与“rentrer”(回到)二动词混为一谈。}。唯一使我感到惊异的是,他竟屈尊使用“岗位”一词,因为他属于希望在语言中抹掉雇佣制度痕迹的现代无产者。此外,过了一小会,他告诉我,在即将“回到”的“岗位”上,他会有一套更漂亮的“工作服”和更好的“待遇”。“制服”和“薪俸”两个词,他已觉得陈旧和不适合了。由于莫名其妙的矛盾,在“老板”口中,词汇不顾一切,仍然比不平等这个概念活得更长久,所以,开电梯的人对我说的话,我总是听不懂。唯一我关心的事,是要知道外祖母是否在旅馆。开电梯的人抢在我的问题之前对我说:“那位太太刚才从你住的地方出去了。”
\par 我又上当了,以为是我的外祖母出去了。
\par “不是,我想那位太太是你们家的雇员。”
\par 从前的市民语言,确实应该废除。但是由于在从前的市民语言中,一个厨娘是不叫“雇员”的,所以我考虑了一会:
\par “他搞错了,我们既不拥有工厂,也没有雇员。”
\par 忽然我想起来了,“雇员”这个词也和咖啡馆的侍者留小胡子一样,给予仆人一种自尊心的满足,刚刚出去的那位太太一定是弗朗索瓦丝(很可能去拜访旅馆里的饮料管理员或者正在观看那位比利时太太的贴身女仆做女红)。
\par 对于开电梯的人来说,光是满足自尊心还不够,因为他在怜悯自己的阶级时说“工人家里”或“小人物家里”,像拉辛说“穷人”\footnote{见拉辛《阿塔莉》第二幕第九场第837到838行。}一样,用的是单数。
\par 我第一天刚到时的那种热情和腼腆早已远去,平时我已不再和开电梯的人说话,现在是他在上下穿过旅馆这个短短过程中,得不到我的回答了。旅馆像一个玩具一样,中间镂空,一层一层地在我们四周展开那分枝一般的走廊。走廊深处,灯光昏暗,越来越弱。通道的门或内部楼梯的台阶都变得细小,灯光使这一切都成了金色的琥珀,像黄昏时刻一样绵软而又神秘。在黄昏中,伦勃朗只需瞬间便勾画出窗棂或井上的轱辘。每一层楼上,一缕金光映在地毯上,展露出落日的余晖和起居室的窗户。
\par 我自忖,刚才我看见的少女是否住在巴尔贝克,她们会是何许人也。欲念这样朝着自己选择的一个小部落人群而去的时候,一切可能与这个小小的部落有关系的人都成了动情的缘由,然后又成了梦幻的缘由。我曾经听见一位太太在海堤上说:“她是小西莫内的一个女友。”那种肯定好事的神情就好像谁在解释说:“他是小拉罗什富科形影不离的伙伴”一样。立刻,从听到这件事的那个人脸上,你可以感到有一种强烈的欲望,巴不得再仔细瞧瞧作为“小西莫内的女友”的那个受到如此厚爱的人。肯定这是一种特权,大概不会赋予随便什么人。贵族阶级是相对的,有些价值不高的小小缝隙,在那里,一个家具商的儿子可以当上风雅王子,并且像一个年轻的威尔士亲王一样统治一个宫廷。自那以后,我经常极力回忆在海滩上西莫内这个名字是怎样对我产生影响的,那时我还辨别不出它的形式,对这个名字也没有把握,至于它意味着什么,指的是这一个人抑或是另一个人,也不肯定。这个名字对于我们下面的故事充满了激动人心的既模糊又新鲜的感觉,每一个字母、每一秒钟,都由于我们不断的重视更深地刻在我们的心上,这个名字变成了(从我对小西莫内的态度来说,只是几年以后才如此)回到我们脑海中(或睡醒时,或昏厥之后)的第一词汇,甚至先于“现在是几点钟”,“我们在什么地方”这些概念,甚至先于“我”这个字,似乎它所指的人就是我们自己,更胜于我们自己,似乎失去知觉一刻以后,先于一切休止的休止,便是没有想到这个词汇的那个过程。
\par 不知为什么,从第一天起,我心里便想,西莫内这个名字大概是这些少女之中哪一个的名字。我不断地琢磨,怎样能够结识西莫内一家。当然是通过她认为地位比她高的人。如果这些人只是市井小民中的小烟花女,要叫她不要产生瞧不起我的看法,大概也不难。不可能有十全十美的故友,只要没有战胜这种蔑视,对于蔑视你的人,就不能完全将你纳入他心中。每次彼此那样不同的女子形象进入我们心中的时候,除非遗忘,或其他形象通过竞争将前一个形象排挤出去,只有当我们将这些外来人变成与我们自己相似的某种东西之后,我们的心灵才会得到安宁。在这方面,我们的心灵与我们的肉体具有同样的反应和活动。我们的肉体不能容忍异体的侵入,除非立刻将入侵者消化或同化。
\par 小西莫内大概是所有姑娘中最俏丽的那个——我似乎觉得,她本可以成为我的情妇的,因为只有她一个人两三次扭头顾盼,似乎意识到了我那死死盯住的目光。我问开电梯的,在巴尔贝克是否认识什么人,姓西莫内。此人不喜欢说他对什么事不知不晓,便回答说,他似乎听人提起过这个姓。到了最后一层,我请他叫人将外地人的最新名单给我送来。
\par 我从电梯里走出来,但没有朝自己的房间走去,而是在走廊里一直向前走去。此刻,虽然管这一层楼的仆役害怕穿堂风,也已将走廊尽头的窗户打开。这扇窗子不向着海,而是朝着小山和山谷,但人们从来也不曾看清楚外面的景色,因为窗上的玻璃不透明,且常常关着。
\par 我在窗前稍事停留,也就是对这个“景”朝拜一下的时间。这一次,倒叫人可以望见比小山更远的地方。旅馆背依这座小山,山上,只在远处有一房舍,但是远景以及落日的余晖在保留了其大小的同时,又用精致的雕刻和丝绒般的首饰匣装饰了它,犹如装饰微型建筑模型一般。好像圣物,只在难得的日子才拿出来供信女善男们瞻仰的金银或珐琅制小寺庙或小教堂。可是这朝拜的时刻已经为时过长,仆役一手拿着一大串钥匙,另一只手触到他那教士无边圆帽上向我敬礼,因为晚上空气清新而凉爽,倒没有将帽子摘掉。他已经走来又把两扇窗板关上了,就像将圣人遗骸盒的两扇门板关上一样,这样也就为我的顶礼膜拜遮住了小型的圣殿和金色的圣物。
\par 我走进自己的卧室。随着季节向前推移,从窗中看到的画面也变了。首先是室内很明亮,只有天气阴霾时,室内才昏暗。这时,在海蓝色的玻璃里,在我窗户的铁框中,镶嵌着大海,就像镶在教堂彩绘玻璃的铅条中一样。大海那圆形的波涛使玻璃变得无边无际。在海弯那整个布满岩石的深深边缘上,大海撒开一些三角,三角上装饰着细腻的笔触勾画出来的不动的飞沫,或皮萨内罗笔下的羽毛\footnote{可能指皮萨内罗(意大利画家及木刻家)所作鸟类草图,保存在卢浮宫中。},雪白的、永不褪色的、奶油般的珐琅色把这些三角固定在那里。在加莱\footnote{加莱(1846—1904),他于1890年创立了一所适用于工业的艺术学校——南锡学校。其玻璃艺术作品在万国博览会上获得极大成功。他的艺术以对大自然的热爱和研究为基础,本人作为有实践经验的植物学家,又将植物题材用于其装饰艺术及玻璃制品中。}的玻璃制品中,这代表着一层白雪。
\par 不久,白昼渐短。我回到房间的时候,淡紫色的天空,似乎被太阳那僵硬的、几何图形的、转瞬即逝的、闪闪发光的面庞打上了烙印(好像代表着什么神奇的符号,神秘的鬼怪),沿着地平线的链条正向大海弯下身去,犹如主祭坛上方的宗教画。落日余晖的各个部分,映在沿墙摆开的桃花心木低矮书橱的玻璃上,我心目中已将它与由它脱胎而来的名画联系在一起,似乎那是昔日某大师为哪一个宗教团体在一个框架上绘制的几组场景,后来在博物馆的大厅中,人们将它一片一片分开陈列,观众只有通过想象才能将它们放到祭坛后部装饰屏组画上原来的位置上去。
\par 几个星期过后,我上楼时,已经日落了。大海上方,天空是一条火红的彩带,与我在贡布雷散步归来准备下楼到厨房用晚饭时在髑髅地\footnote{髑髅地原指《圣经》中耶稣受难的地方。}顶上之所见一模一样。这火红的彩带,是完整的一片,又像肉冻一样可以切开。顷刻大海已经发凉,变成蓝色,好似人称鲻鱼的那种鱼,天空则像我们过一会在里夫贝尔点的鲑鱼一样粉红。这一切,更增加了我就要更衣外出晚宴的快乐心情。沉重的暮霭,烟灰般黑色,有光泽,玛瑙那样坚实,肉眼看得见,紧贴着海洋,吃力地从海上升起。这儿几片,那儿几片,高高低低,一层一层,越来越宽阔。最后,最高的几层向已经变形的根茎弯下身来,一直到脱离了直到此刻支持着它们的重心,似乎就要将已到中天高度的脚手架拖走,将它扔到大海中去。
\par 我从前坐在车厢里有一种印象,觉得需要从困倦和关在一间房里受监禁的状态中解脱出来。见一艘轮船如夜行者一般远去,也使我产生同样的印象。但是,在此刻我自己置身的房间里,我并不感到受监禁。因为一小时以后,我就要离开这里乘马车外出。我扑到床上。我看得见距我相当近的船只。奇怪,人们在夜间也看得见船只在黑暗中移动,好似颜色幽暗、默默无声却没有入睡的天鹅。我似乎觉得自己就在一艘轮船的卧铺上,大海的画图从四面八方将我团团围住。
\par 不过,确实经常只是一些画图而已。我忘记了,在画图的色彩下,海滩正在形成凄惨的空旷地带,夜晚那不安的海风吹遍整个海滩。刚到巴尔贝克时,夜风袭来,我是那样焦灼不安。现在,即使在我的房间里,我的全部心思仍在我目睹从我面前走过的几个少女身上,我的情绪再也不能平静,再也不能停留在事不关己的状态。在我心中,是不会产生真正富有美感的印象了。等待着去里夫贝尔晚宴更使我心浮气躁起来。在这种时刻,我的意念停留在躯体的表面上。我就要给这躯体穿上衣服,以便在那灯火辉煌的饭店中,在打量我的女性目光前,尽量显得讨人喜欢。我无法在事物的色彩后面注入深邃的思想。我的窗下,雨燕和燕子不倦地轻轻地翻飞,像喷泉,像生命的火焰,将高喷的间歇与平面方向上长长的轨迹那不动的白色的线条融和在一起。这种地区性的自然现象将我眼前涌现的景色与现实联系起来。如果没有这一令人着迷的奇迹,说不定我会认为眼前的景色只不过是每日更新的绘画选。人们主观地在我所在的地点展开这个绘画选,而那些绘画作品与这个地点并没有必要的联系。有一次,我觉得那就是日本木版、铜版画展览:在精雕细刻出来的好似月亮一般滚圆的红太阳旁边,有一朵黄色的云,犹如一面湖。湖边,是黑色利剑,有如湖滨树木的侧影。还有一道淡淡的玫瑰色,自从我有了第一个彩笔盒以来,从未见过这样的玫瑰色。这颜色绽开,好似一条江,两岸上似乎有船只搁浅在沙滩上,等待着人们前来将它们拖入水中。我怀着业余爱好者或在两次交际访问之间到画廊转上一转的女人那种蔑视、厌烦而又轻浮的目光,自言自语道:“真奇怪,这落日,与众不同,不过我早已见过和这一样优美、令人惊异不止的落日了。”
\par 晚上,一条船被地平线吸收,又将它变成了流体,显得和地平线完全是一种颜色,宛如一幅印象派的画。船只似乎也与地平线一样,由一种原材料所制成,似乎人们只是在雾的蓝天中勾画出船体和缆绳。缆绳交错,船体显得更加细小,变成了金银制品。有时,大洋几乎占满了我的整面窗户,上方是一抹天空,只有一条线,与海一样地蓝,因此我以为那还是大海,只在光照作用下,才显出不同的颜色。
\par 另一日,大海只在窗子的下部描绘出来,窗子其余的部分布满了浮云。水平方向上,一朵一朵的云你推我搡,结果好像出于艺术家的预谋或专长,那窗玻璃正在介绍“云朵研究”。与此同时,书橱的各块玻璃上显示出相似的云朵,但这是在另一部分地平线上的云朵,而且被光线染上了不同的色彩,似乎向你提供同一题材的反复。这是某些当代画家十分珍爱的反复,总是取自不同的时刻。而现在,由于艺术的固定作用,可以在一个房间里一览无余,呈彩粉画形式,并且压在玻璃板下面。
\par 有时,在海天一色的灰色上,细腻精巧地加上一点粉红。这时,在窗子下方安睡的一只小蝴蝶,就像将双翼落在这幅有惠斯勒\footnote{惠斯勒(1834—1903),美国画家及雕刻家,他在伦敦安家落户,住在切尔西区。他对日本艺术和马奈极为赞赏,尤致力于色彩和谐研究。《灰与粉红色的和谐》是他的一幅画的题目。}风味的、题为《灰与粉红色的和谐》的画下方。这是切尔西大师亲自签名的作品。这粉红色渐渐消失,再没有任何东西可以注目。我呆呆站立片刻,然后拉上窗帘,再次躺下。从床上,我看见窗帘上方还留有一线光亮。这一线光亮也渐渐暗淡下去,越来越细。平日,这个时刻,我已坐在饭桌上。今天,我就这样让这个时刻在窗帘上方逝去,既不忧伤,也不惋惜,因为我知道,今天与别的日子不一样,像黑夜只有几分钟打断白昼的极地的白天一样,今天比平时更长一些。我知道,从这黄昏的蛹壳里,里夫贝尔饭店的万丈光芒正在准备经过美好的变形脱壳而出。
\par 我自言自语:“到时间了。”我在床上伸伸懒腰,起身,梳洗完毕。这样无用的时光,脱去了物质生活的重负,我觉得自有其魅力。别的人在楼下进晚餐,而我在这里,将下午无所事事积蓄起来的精力,只用在洗浴后晾干我的身躯、穿一件无尾常礼服、系领带上。指引这些动作的,已经是期待已久的与某个女子重逢的快乐。那是我上一次在里夫贝尔注意到的一个女子,她似乎对我注视良久。有一会她离席了,也许希望我尾随而去。我怀着快乐的心情给自己加上所有这一切诱饵,以便使自己全心全意、全神贯注地投入一种新生活。这是自由的、无忧无虑的生活,我要让圣卢的冷静来支持我的犹豫不决,并在生物的各个品种和来自各地的物产之中进行选择。这些菜,我的朋友一点,便构成罕见的佳馔,会大大刺激我的食欲或者我的想象。
\par 最后,这样的日子终于来到,我再也不能通过餐厅从海堤回到房间了。餐厅的玻璃窗不再敞开,因为外面夜色已经降临,而且这个玻璃蜂巢灯火通明,将贫苦的人和好奇的人都吸引来了。他们无法进入这灯光通明之中,便像秋风卷下的一片黑糊糊的蜂群一样,扒在玻璃蜂巢那发光而又光滑的四壁上。
\par 有人敲门。是埃梅亲自给我送来了外地人的最新名单。
\par 埃梅走之前,非要告诉我,说德雷福斯罪该万死。\footnote{书中年代为1898年。自1897年10月29日参议员史海尔-凯斯杜埃提出重新审理该案件以来,这件事又成为舆论注意的中心。1898年1月13日,左拉在《震旦报》上发表了《我控诉》一文。埃梅所指的文件可能是亨利上校所准备的文件,据说根据这些文件可以最后确定德雷福斯有罪。后来,亨利上校被确认犯了伪造文件罪,于8月31日自杀。但在本书中,直到《盖尔芒特家那边》第一部分中,人们谈论德雷福斯事件时,亨利上校还活着。}
\par “人们会得知一切的,”他对我说,“不是今年,而是明年。这是与参谋部关系非常密切的一位先生对我说的。”
\par 我问他,是不是在年底以前人们还下不了决心马上揭露一切。
\par “他放下烟卷。”埃梅继续说下去,模拟着那个人的动作,并且像他的顾客那样摇着头,晃着大拇指,那意思是说:“不要要求过高。”
\par “‘不是今年,埃梅’,他敲着我的肩膀对我说,‘今年不可能。到了复活节,\footnote{指第二年4月。}行’。”
\par 然后,埃梅轻轻拍着我的肩膀,对我说:“您看,他怎么说的,我都原样告诉您了。”那意思,要么是这样一个大人物对他那么随便,他很洋洋得意,要么是我更能清楚明白地看到那论据的价值和我们抱希望的根由。
\par 我在外地人名单的第一页上,看到“西莫内及其家属”几个字,禁不住心头一震。我心中仍藏着童年时代便产生的由来已久的梦幻。梦想中,心中有的和所感受的全部柔情融成一片,由一个尽量与我不同的人给我带来。这个人,我现在用西莫内这个名字来称呼她,并且忆起在海堤上看见的充满青春活力的躯体。她们展现成可与古代和乔托的名画相媲美的体育队形,是多么和谐。我用这个名字和对这优美的和谐的回忆,创造出了这个我等待的人。我不知道这几个少女中哪一个是西莫内小姐,也不知道她们当中是否有哪一个真姓这个姓。但是我知道西莫内小姐爱着我,我要靠圣卢设法立即与她结识。可惜在这个条件下,圣卢只得到允许延长假期,他不得不每天回到东锡埃尔去。为了叫他不去尽那个军队义务,我本来以为,除了可以指望他对我的友谊之外,还可以指望人类博物学家的那种好奇心。我经常有这种好奇心,常常我并未见过人家说的那个人什么模样,只要听到人家说,哪家水果铺子里有一位漂亮的收款员,我就想与女性美的这个新变种去结识。我希望在圣卢面前谈及我那几个少女,也在他心中激起这种好奇心。谁知我大错特错。他是那个女演员的情夫,他爱她,因此,这种好奇心早已麻木。即使稍有感觉,他也将它压抑下去,因为他很迷信,以为情妇对自己忠实与否,取决于他自己是否忠实。所以我们动身去里夫贝尔晚宴时,他并没有应允积极地去管我那几个少女的事。
\par 最初,我们抵达里夫贝尔时,太阳刚刚落山,但是天色依然很明亮。饭店的花园里,灯火尚未点燃。白昼的热度下降,好像存放在一个花瓶的底部,沿着这花瓶的边壁,空气形成了透明、暗色而又浓稠的果冻。偌大的一丛蔷薇,贴着墙,在暗淡下来的墙上画出粉红的条纹,宛如人们在缟玛瑙石里看到的树枝状纹路。
\par 过了不久,我们走下马车时,夜色已经降临。或是天气不好,或是希望暂时安静一会而推迟了叫人驾车的时间,总之我们从巴尔贝克启程时,夜色就已经降临。但是这样的日子,我听到海风吹拂也不感到忧伤,我知道这并不意味着要放弃我的计划,并不意味着就要关在一个房间里。我知道我们要在茨冈音乐声中走进饭店的大厅,那里无数的灯火将用金光灿烂的宽宽的烙铁,不费吹灰之力地战胜黑暗和寒冷。于是我高高兴兴地上了马车,坐在圣卢旁边。马车在滂沱大雨中等待着我们。
\par 现在,我每天一坐到桌前开始一项评论研究或阅读一本小说,便感到厌倦。贝戈特说,他坚信,我特别是能体会脑力劳动乐趣的材料,虽然我自己并不持有这种看法。在“我以后能干什么”这个问题上,最近这些时候,贝戈特的话倒使我感到,这种厌倦透露出一点希望。
\par “归根结底,”我心中暗想,“说不定写一本小说时能否体验到快乐,并非是判断一篇文字是否美丽、是否有价值的无懈可击的准则。说不定这只是一种常常附带而来的次要状态,而缺乏这种快乐并不能就预先断言文章不美。也许某些杰作就是打着哈欠写出来的。”
\par 外祖母对我说,如果我身体好,我就会写得很好,而且会怀着快乐的心情去写。这话打消了我的疑虑。可是我家的家庭医生认为,更为谨慎一些的作法,还是提醒我,我的健康状况可能会使我面临什么严重的危险。他给我列出了应该遵循的各种保健措施,以免发生意外。我认为各种快乐应从属于目标。与快乐相比,目标无比重要。这个目标便是要变得身强力壮,足以能够完成可能蕴藏于我自身的大业。自从来到巴尔贝克,我对自己进行周密而经常的控制。喝一杯咖啡会使我彻夜失眠,而睡眠对我第二天不感到疲倦必不可少。那么,谁也别想叫我去碰那杯咖啡。
\par 可是,一到了里夫贝尔,在新的快乐刺激下,我又处于另一种思想状况之中了。例外情况才叫我们进入这种状况之中。这么多天以来耐心织成的、将我们导向明智的网已经撞破,似乎再也不该有什么明日,有什么待以实现的高尚目标了。顷刻间,为了维护这高尚目标而起作用的、整个周密谨慎的保健机制烟消云散。一个跟班小厮问我要不要外套时,圣卢总是对我说:
\par “你会不会冷?最好还是穿着,天气可不太热。”
\par 我总是回答说:“不要,不要。”可能当时我并不感到冷,但是不管怎样,我再也不知道害怕病倒、不要死去以及写作重要这些事为何物了。我把外套交出去。我们在茨冈人奏出的军乐声中进入饭店大厅,在一排排已经上了饭菜的桌子间前进,就像在轻易获得荣誉的道路上前进一样。乐队授予我们军事荣誉和我们配不上的凯旋曲,我们感到音乐的节奏将快乐的奔放灌输到我们身上。我们用庄重而冷冰冰的表情和懒洋洋的举止将这种情绪掩盖起来,以便显出与那些咖啡馆音乐会里服饰华丽、装腔作势的女人不同。她们就着火药味十足的曲调,唱着轻佻、放肆的歌曲,跑着上台,那尚武的举止犹如打了胜仗的将军。
\par 从这一刻起,我便成了另外一个人,再也不是我外祖母的外孙子,只有到出了门的时候,才会想起她,而是成了就要服侍我们就餐的小伙计的临时小弟弟了。
\par 在巴尔贝克我一个星期也达不到的啤酒量,更不用说香槟,现在,我一个小时就喝下这么多,还要加上几滴波尔多酒。我心不在焉而不知其味。在我冷静而清醒的时候,这些饮料的味道意味着明显可以称道而又轻易放弃的快乐。我一个月节省下来的两个“路易”,本来想买一件什么东西,此时再也想不起来要买什么,而赏给了提琴师。在桌子之间撒欢上菜的侍者,有几个跑得飞快,张开的手心里托着一盘菜,似乎这里就是那种看谁不把菜盘掉在地上的比赛的终点。确实,巧克力蛋奶酥没有打翻而抵达目的地,英式炸土豆,虽然疾驰快跑本来会摇动,可是抵达目的地时,仍然在波亚克乳羊肉\footnote{波亚克为法国西南部纪龙德河上一河港,在波尔多附近。波亚克羊肉为法国一名菜。}四周排列整齐如初。我注意到一个侍者,个子非常高,长着一头乌黑的秀发,脸上像扑了粉一样,使人更容易想起某些珍禽而不是人类。他不停地从大厅这头跑到那头,似乎没有目的,叫人想到一只南美大鹦鹉。这些南美大鹦鹉以其艳丽的羽毛色泽和不可理解的骚动不安填满了动物园的大鸟笼。
\par 不久,场面井然有序了,更高雅更平静,至少在我眼中如此。所有这些令人头晕目眩的活动全集中成为安静的和谐。我望着那些圆桌,无数的群体将饭店充满,每一桌有如一个星球,有如从前讽喻画中的行星。在这各不相同的星球之间,有一种无法抵挡的引力在起作用。每桌的就餐者,眼睛都望着别的餐桌,只有某个阔气的东道主例外,他有办法,带来了一位著名的作家。借助于旋转小桌的特点,极力逗引作家说些毫无意义的话,太太们倒听得兴高采烈。这些星球般的餐桌之间的和谐,倒也不妨碍无数侍者不停地运转。因为他们不像就餐者那样坐着,而是站着,所以是在高层地区运转。有的跑着送冷盘,有的换酒,有的添加酒杯。虽然有这些特殊原因,他们在圆桌间不断地奔跑,最后还是揭示出这令人头晕目眩而又有规律的运行的法则。两个其丑无比的女收款员,坐在一大丛鲜花后面,忙于没完没了的算账,好像两个女魔术师,忙于通过天文计算以预见在这个按照中世纪科学设计的天体苍穹中偶尔会发生什么大动荡。
\par 我有些可怜起这所有进餐的人来,因为我感到,对他们来说,这些圆桌并非星球,他们在办事中也从不运用什么分类法,以使我们摆脱其惯有外表形式的束缚,能观察到一些相似之处。他们认为,他们正在与某某人进晚餐,这一餐大概多少钱,他们第二天还要再来。对于年轻侍者服务行列的行进,他们显得完全无动于衷。这些侍者很可能这会儿没有什么紧急的活,正排着队递送面包小篮子呢!有几个年纪特别小,饭店总管经过时打他们几巴掌,把他们打得晕头转向,忧郁的眼睛直勾勾地在那里出神。他们从前曾在巴尔贝克大旅社干过,如果有哪一个巴尔贝克大旅社来的顾客认出了他们,跟他们搭上几句话,亲自吩咐将无法下咽的香槟酒拿走,他们就非常得意,只有这时才得到点安慰。
\par 我听到自己的精力在鼓荡,其中有舒适的成分,但这是独立于能使我们感到舒适的外界物品之外的舒服。身体、注意力的极微小的变化,都足以使我感受到这样的舒适,正像轻轻一压便足以使一只闭着的眼睛感觉到颜色一样。我已经喝了很多波尔图葡萄酒。我之所以还要喝,主要并不是为了享受再加几杯能给我带来的新的舒适感,而是因为前几杯所产生的舒适感。我任凭音乐随着每一节拍牵动着我的快乐,快乐乖乖地来到每一节拍中停息。多亏有了那些化学技术,能大量地生产出自然界中只能偶然遇到的物质。里夫贝尔的这家饭店,与那些化学技术相似,它在同一时刻内汇集了许多女子。从她们那里获得幸福的前景激动着我的心。靠散步或旅行的邂逅相遇,一年之内我也不会遇见这么多人。另一方面,我们听到的音乐——华尔兹,德国轻歌剧,咖啡馆音乐会歌曲交相混杂,这一切对我都是全新的——本身就像是神仙快活的去处,它与另一种快活相重叠,又比那另一种快活更醉人。每一个旋律,都像一位女子一样特别,但却不像女子那样,将流露出来的感官享乐的秘密只留给某个备受青睐的人。它主动向我举荐这种快乐,贪婪地望着我,迈着任性的或淫荡的步伐向我走来,与我攀谈,抚摸我,似乎我骤然间变得更有魅力、更加强壮或更加富有了。我感到这些曲调里有某种很无情的东西。因为这些曲调对一切脱离物质利害的美,一切智慧的辉映,都是格格不入的。对它们来说,只存在肉体的快乐。它们将这种快乐——自己爱慕的女子与另外一个男人去品尝的快乐——作为世界上存在的唯一事物呈现在那个可怜的妒者面前对他来说,这实在是最无情、最找不到出路的地狱。
\par 但是,我低声重复着这曲调的音符,并不给它一个亲吻时,它使我感受到的它所独有的肉欲,对我又变得那样珍贵,我甚至会离开自己的父母追随这旋律到一个奇异的世界中去。它用一行又一行一会充满慵懒一会又充满生命活力的音符,正在肉眼看不见的地方建立起这个奇异的世界。这样的快活并不能赋予得到它的人以更高的价值,因为只有他自己感受得到。每次在生活中,我们没有讨得注意到我们的女子的欢心时,她并不知道那个时刻我们是否拥有这种主观的、内心的极度幸福,因而这也丝毫不能改变她对我们的看法。虽然如此,我仍感到自己更加强壮有力,几乎成了无法抗拒的男子。我似乎觉得,我的爱情再不是什么令人讨厌、别人可以嗤之以鼻的东西,而确实具有这音乐的感人之美、诱人之处。这音乐本身好像一个可爱的去处,我心爱的女子与我在这里相逢,顿时变得亲密无间。
\par 这饭店的常客,不但有半堕入风尘的女子,也有最风雅阶层的人,他们下午五点左右才吃茶点或者在这里设盛大的晚宴。茶点设在一条狭窄的成过道形的玻璃长廊里。长廊从衣帽间到餐厅一面,走向花园的一侧,除了几根石柱以外,长廊与花园之间只有玻璃门窗。这里那里,门窗敞开着。结果是除了许多处穿堂风以外,骤然射进的强光,令人头晕目眩和不稳定的光照几乎使人无法看清用茶点女客的模样。所以,这些女客两张桌子两张桌子地拼在一起,沿着这狭窄的细颈瓶一长条坐在那里的时候,她们喝茶或相互打招呼的每一个动作都闪闪发光,简直可以说那是一个鱼池或鱼篓,捕鱼人将捕来的颜色鲜艳的鱼儿堆积在这里。鱼儿半身露在水外,沐浴着阳光,以其变化不定的光芒在人们的眼前像镜子一样闪动。
\par 过了几个小时,便到了开晚餐的时刻。晚餐自然是在餐厅里开的。那时,虽然外面天色依然明亮,餐厅里已燃起灯火。从餐厅里向前望去,可见花园中的楼宇,在落日余晖的映照下,好似夜间面色苍白的幽灵。楼宇旁有株株千金榆,一抹夕阳正穿过那淡绿的树叶。从进晚餐的灯火辉煌的厅室中望出去,玻璃窗外边,那绿树再不像是在闪闪发光而又潮湿的鱼网之中,正如我们形容下午沿着闪射着蓝光金光的长廊用茶点的那些妇人一样,而是像神光照耀下淡绿色巨大养鱼池中的水草了。
\par 人们离席了。如果说,在进餐过程中,各位宾客把时间都用在望着、辨认着邻近各桌的宾客,也叫附近各桌的宾客叫出自己的名字,而在自己桌子的周围则保持着完美的整体的话,围绕着一个晚上的东道主形成重心的引力,在他们到进茶点的那条走廊上去喝咖啡时,便失去了其强大的力量。常发生这样的事:有人经过时,某桌正在进行的晚餐便放弃了一个或数个微粒子。这个粒子或这数个粒子因为受到对方餐桌极大的吸引,便从自己的餐桌分离出来。而前来向朋友问好的一些先生或太太又顶替了他们的位置,然后又回到原位,说:“我得溜了,回到某某先生那儿去……今天晚上我是他的客人。”这一刻,人们可以说,这分开的两束花交换了其中的几朵。
\par 然后,长廊本身也渐渐空了。常常是,甚至晚餐后,天色还有些亮,这长长的走廊没有点起灯火,沿廊玻璃窗外树木摇曳,倒像是树木丛生、笼罩在黑暗之中的公园小径。偶尔会有一位进餐的女士在阴影中滞留良久。一天晚上我穿过长廊出去,发现美丽的卢森堡亲王夫人正在那里,坐在不相识的一群人中。我脱帽向她致意,但没有停下脚步。她认出了我,微笑着点点头。远远超过这致意的,是从这个动作本身升起向我道出的几句话,如仙乐一般。可能是较长的一句道晚安的话,并非叫我驻足,仅仅是对那点头致意的补充,以构成有声的问好。但是这句话说的是什么,非常含混不清,结果我只听到了声音。这声音那样柔和地拉着长腔,我觉得那样富有音乐美,宛如在树林幽暗的纤细树枝中,一只黄莺啼啭起来。
\par 有时碰巧圣卢遇见了他的哪一伙朋友,决定到附近一处海滩的游乐场去与他们一起消磨时光。如果他与那些人一道走,便将我一个人安顿在马车里。这时,我就吩咐车夫奋力疾驰,以便让这没有任何人帮忙度过的时光不要显得那样漫长,免得我向自己敏感的心灵叙述到里夫贝尔以来自己从别人身上得到哪些变化——用回顾和力图走出已陷入齿轮咬合之中一般的被动地位的形式。狭窄的小路只容一辆马车通过,又是伸手不见五指的夜晚,很有可能与来自相反方向的另一辆马车相撞。悬崖上经常有崩塌的土方石块滚下,路面也不平稳。悬崖陡壁垂向海中,就在眼前。这一切都无法在我心中唤起必需的一点点力量,以将对危险的意识和恐惧拉回到我的理智上来。这是因为,使我们得以创作出一部作品的,并不是要成名成家的欲望,而是勤奋的习惯;帮助我们保护未来的,并不是眼前的欢愉,而是对往昔睿智的思考。帮助我们残废的头脑走正路的,是理智思考和自我控制这一副拐杖。然而,如果我抵达里夫贝尔时,早已把这副拐杖扔得远远的,破例地放松我的神经,处于任凭精神失调、酒精肆虐的状态中,就等于我赋予当前的每一分钟以质量和魅力。其结果是既不能使我更能够,也不能使我更有决心去保护这每一分钟。我听凭自己将这些看得比我剩余的生命贵重一千倍的时候,我的激情就已将这每一分钟与剩余的生命割裂开来了。我像英雄,像醉汉一样将自己关闭在现时之中。我的过去已暂时隐去,在我面前再也映不出自己的影子,我们管这个影子称作自己的前程。我将自己生活的目的,再不放在实现往昔梦幻之上,而放在现时这一分钟的欢愉中,我看不到比这一分钟的欢愉更远的东西。结果是,正是在我感到格外快活的时候,正是在我感到我可以过上幸福生活的时候,正是在我看来我的生命应该更有意义的时候,我摆脱了至今生活能够使我设想到的各种烦恼,我毫不犹豫地将生命交给发生意外事故的偶然。看上去这很矛盾,但这只是表面的矛盾。再说,简而言之,我只不过将轻率集中在一个晚上而已,对其他人来说,这种轻率稀释在他们整个生存过程中。在整个生存过程中,他们每天都并非必要地面临着海上旅行、坐飞机或坐汽车游玩所包藏的危险,他们的死亡会使之肝肠寸断的人正在家中等待着他们归来。或者一本书最近就要出版是他们活着的唯一缘由。这本书还与他们脆弱的大脑联系着。
\par 同样,在里夫贝尔的饭店里,我们逗留的晚上,如果有人怀着杀死我的动机来到,由于我在一个不现实的远景中只看到我的外祖母、我未来的生活和我要写的书,由于我完全融入了邻桌那个女子的香水味、旅馆侍应部领班的彬彬有礼和正在演奏的华尔兹乐曲的婉转与悠扬之中,我完全依附在现时的感觉上,除了与它不要分离,再也不能想得更远,再也没有其他目标,我就会紧紧抱着这感觉死去,我就会任人杀害,不去自卫,一动不动,恰似那被烟草的烟雾熏得麻木的蜜蜂,再也无心去保护自己辛辛苦苦积蓄起来的食物,再也不指望保全自己的蜂巢了。
\par 此外,我还应该说,在我极度振奋的心情下,最严重的事情也变得无足轻重,这使我终于理解了西莫内小姐及其女友们。要与她们结识的大业,现在在我看来似乎轻而易举但又无所谓了,因为只有我现时的感觉极度强烈又有每一细微的变化,甚至只是这种感觉持续下去会使我快乐,对我才有重要意义。其余的一切,父母、工作、游玩、巴尔贝克的少女,都不比不容其停留的、大风中的一抹飞沫更有重量,只是与这种内心的强烈感受相对而言才存在:酩酊大醉将主观唯心主义、纯粹的现象论实现了几个小时。一切都只不过是表象,只是随着我们自己的崇高而存在而已。这并不是说,真正的爱情在这种状态中无法存在——如果我们确实有情,而是我们如同新到一个地方那样清楚地感觉到,有一些莫名其妙的压力改变了这种情感的规模,以致我们对它再也无法同等视之了。这同一爱情,我们还能再次寻找到,但是已经易位,再也不考虑我们自己,满足于现时赋予它的感觉,这种感觉对我们已经足够,因为非现时的东西,我们是不在乎的。可惜的是,如此改变价值观的系数,只在酩酊大醉这个时刻才能发生作用。此时此刻再没有任何重要性,像吹肥皂泡一样一吹就破的人,到了明天,会重又具有他们的重量。又得尽力重新开始现在看来已毫无意义的研究工作了。更严重的是,这种明日数学,与昨日数学一样,我们将再度不可自拔地陷入这些数学题目之中,这便是甚至在这样的时刻也约束我们的数学,事实上对我们自己失去了约束力而已。如果恰巧在我们近旁有一位端庄的女子或充满敌意的女子,前一天还那样难办的那件事——即使我们能讨她喜欢——现在我们却觉得一百万倍地更加轻而易举。实际上绝非如此,因为这只是在我们看来,在我们内心看来如此,只是我们自己变了。就在当时,如果我们来得放肆,她也会对此不满,就和我们到了第二天,要为给了侍者一百法郎小费而对自己不满一样。那道理是一样的:此时已不再酒醉。只不过对我们来说,理智迟来一步而已。
\par 那晚在里夫贝尔的女子,我一个也不认识。她们成了我酩酊大醉的一部分,正如反射是镜子的一部分一样。所以她们显得比西莫内小姐一千倍地合乎我的欲望,而西莫内小姐对我是越来越不存在了。一个金发姑娘,独自一人,神情抑郁,戴一顶插满野花的草帽,出神地望了我好一会,她显得那样讨人喜欢。然后轮到另一个,再后轮到第三个。最后轮到一个肤色有光泽的棕发姑娘。圣卢几乎认识所有这些姑娘,我则不然。
\par 认识现在成为他情妇的这个人之前,圣卢确实在这个花天酒地的有限世界里生活过那么长久。这些晚上到里夫贝尔来用晚餐的女子,几乎没有他不认识的,他本人或者他的某一位朋友至少和她们睡过一夜。其中有不少是纯粹出于偶然,才出现在里夫贝尔饭店。她们来到海滨,有的是来与情夫重聚的,有的则是极力想找一个情夫。如果她们和一个男人在一起,圣卢便不与她们打招呼。她们则比望着自己身边的男人更多地望着圣卢,看那神情,似乎并不认识他,因为谁都知道,除了那个女演员,他现在对任何女人都毫不在意了。在这些女人眼中,这一点又赋予他一种特殊的威望。
\par 有一个女子嘁嘁喳喳耳语般地说:“那是小圣卢。看来他一直爱着那个妓女。真是情意缠绵呢!他真是美男子!她觉得他真是了不起!多么帅!不管怎么说,有些女人就是有运气!而且是多么神气的男人!我原来和德·奥尔良在一起时,跟他很相熟。他们是形影不离的一对!他那时为她花天酒地!可现在,他再不那么干了。他不做对她不忠的事。啊!她可以说自己真有运气!我真不知道,他从她那里能得着什么。肯定他也是个大傻瓜!她那两只脚像船一样大,像美国女人一样长着唇髭,内衣脏得很!她的裤子,我相信一个小女工都不要!你瞧瞧他那一双眼睛,为这样一个男人,往火坑里跳也愿意呀!咦,别说话,他认出我来了,他笑了,啊呀,他从前与我很熟呢!跟他一提我就行。”
\par 她们与他会意地相视,让我撞见。我真希望他把我介绍给这些女子,真希望能够要求与她们一见,她们也慨然应允,即使我无法接受这样的约会也罢。如果不这样,在我的脑海中,她们的面庞便永远缺乏自身独特的那一部分——似乎为面纱所遮掩——这一部分,是每一个女子都不相同的。没有见过时,我们无法想象。只有在向我们投过来的目光中,这一部分才显现出来,那目光对我们的欲望表示赞同,并向我们作出许诺:我们的欲望会得到满足。


\paragraph*{5}

\par 她们的面目,虽然我只局部见到,但对我来说,仍然远远胜过我猜想大概会恪守妇道的那些女子的面孔。那些女人的面孔与这些姑娘毫无相像之处,平淡,无底蕴,平板一块,没有厚度。这些姑娘的面庞之于我,肯定又不同于之于圣卢。对于佯装与他并不相识的那种不动声色,他显然毫不在乎,打招呼那么平平常常,向任何人打招呼都可以如此。透过这毫不在乎或平平常常,他心中忆起,眼前浮现出散乱的头发,痴狂的嘴和半张半闭的眼睛。这整个一幅无声的画,恰似画家为了欺骗大部分观众,用一幅得体的油画将它盖上的那种画幅。我感到自己的生命中不曾有一丝一毫进入这些女子中哪一位的心灵,也不会有任何东西被带到她一生所走的吉凶未卜的道路上去。对我来说,自然这些面庞一直是封闭的。但是,知道这些面庞曾经喜笑颜开过,已经足以使我感到这是一种奖赏。如果她们的面庞不是其下隐藏着爱情回忆的圆形饰物,而只是漂亮的奖章,我是不会给她们找到价值的。
\par 至于罗贝,他坐着时永远无法正襟危坐,他用宫廷宠儿的微笑来遮掩武将的渴求行动。仔细端详他时,我意识到,他那三角脸上精力充沛的骨架与其祖先该是多么分毫不爽。这骨架对一位豪情满怀的弓箭手更合适,而不适合于一位风雅文士。在细腻的皮肤下,显现出大胆的房屋建筑,封建时代的建筑艺术。他的头使人想到古老城堡主塔上那些塔楼。塔楼上毫无用处的雉堞依然可见,但是在内部,已把这些塔楼改成了图书室。
\par 返回巴尔贝克的路上,对于他给我介绍的那些陌生女子中的某一位,我一秒钟不停地又几乎不知不觉地在心中反复说着这句话:“多么甜美的女子!”好像唱叠句一样。自然,更确切地说,这些话是发自神经亢奋状态而不是持久的判断。如果我当时身上有一千法郎,而且到那时还有开门营业的珠宝店,我定会给那个陌生女郎买一个戒指。这是真的。当我们像这样在极为不同的环境中度过生活中的某些时刻时,我们常常对各种人过于慷慨相赠。到了第二天,大概又会觉得这些人毫无趣味。但是对于前一日对他们说过的话,人们感到负有责任,而且希望实践诺言。
\par 这样的晚上,由于迟归,回到我的房间,见到床,我很高兴。房间对我已不再抱有敌意。我初来乍到那天,还以为自己永远也无法在这张床上安歇呢!现在,疲倦已极的四肢要在这里寻求一个支撑。因此,我的大腿,我的臀部,我的肩膀,一个接一个地从各个点上尽量与包着床垫的单子合成一体,似乎我的疲倦有如一位雕刻家,打算取得一个完整的人体模具。
\par 但我无法入睡,我感到清晨即将来临。平静的心情,健康的体魄,都不存在。在忧郁中,我似乎感到这些东西再也不会失而复得。我必须安睡多时才能重新得到这些。即使小憩一会,再过两个小时也要被交响音乐会吵醒。可是我骤然入睡,堕入了梦乡。梦中,回到了青春时代,逝去岁月重返,失去的感情重来,灵魂脱离躯体,到处游动,对亡人的回忆,荒唐生活的幻想,倒退到大自然作为最原始主宰的时代(据说我们在梦中经常看见动物,却忘了我们自己在梦中几乎总是个没有理智的动物,是这种理智对事物放射出确实性之光。相反,我们在梦中对于生活中的景象只是提出一种不可信的看法,每一分钟这看法又被遗忘摧毁,前一个景象在后一个景象面前烟消云散,就像走马灯一样,换了一张片子,下一个景出来,前面一个景烟消云散)。所有这些奥秘,我们以为不了解,实际上我们几乎每天晚上都在初步接触,同时也接触另一个大奥秘,就是消灭与重生。自己往事中某些已经暗淡下去了的地方,又逐个被照亮,里夫贝尔的晚餐难以消化,使这种光亮更加游移不定,这使我成了这样一个人:似乎最高的幸福就是与勒格朗丹相遇,因为我刚才在梦中与他聊天。
\par 其实,就是我自己的生活也完全被一个新布景挡住了视线,恰似舞台上所置的布景。后台换景时,一些演员在前台演出一个逗人开心的节目。我在其中扮演角色的滑稽节目,是东方故事味道。由于所置布景极其接近东方色彩,我在戏中对自己的过去,甚至对自己都一无所知,我只是一个因为犯了过失身遭棒打和受各种惩罚的一个人物。是什么过失,我没有发现,实际上这个过失便是喝了太多的波尔图酒。
\par 我忽然醒来,发现多亏这一大觉,竟没有听到交响乐音乐会的喧闹。时已下午。我用力起身想看看表,想知道是否确实如此。一开始,怎么使劲也毫无成效,头又沉沉落在枕头上,半途而废。这是继困倦以及其他的醉态而来的短暂的下沉,或由饮酒或由大病初愈而引起。何况,甚至就在看时间之前,我也肯定中午已过。昨天晚上,我不过是一个被掏空了心肝的、无重量的人(就像非得先躺下才能坐起来,非得睡醒才能住口一样),我不停地翻腾,说话,再也没有重量,没有重心,我被抛掷出去,似乎可以继续这闷闷不乐的奔跑,一直跑到月亮上去。虽然睡着了,我的眼睛没有看见时间,我的身体却能计算出来。它不是在表面绘制出时间的表盘上量度时间,而是通过逐步称量我的力气恢复了多少。像一个大钟一样,我的身体让力气从头脑向身体的其余部分一级一级走下去,现在这力气已经将其积蓄的充足数量实实在在地堆积到了膝盖以上。如果说,从前,大海是我们生命所系的环境,必须将我们的血液重新投入大海之中才能恢复我们的力气,就遗忘和精神空虚而言,情形也是如此。有时,在几个小时之内,似乎脱离了时间。但是,在这个时间内积聚起来而没有花费的力气,通过其数量衡量了时间,与时钟的重量或沙山塌陷衡量时间一样准确。
\par 何况,从这样的睡眠中醒过来,并不比长时间熬夜后再想睡着更容易,任何事情都有持续下去的倾向。如果说,某些麻醉剂确实会催人入睡,那么长时间睡眠则是更厉害的一种麻醉剂。长时间睡眠之后,要醒过来很困难。我就像一个水手,他清清楚楚看见自己的船只绳缆系在码头上,但是船只仍被海浪摇来摇去。我确实想看看时间,想起床,但是我的身体每时每刻都再次被投进睡眠中。着陆很困难,我又倒在枕头上两三次,然后才站起来,走到我的表跟前,将表上的时间与我那软绵绵的双腿所拥有的丰富的物质所指示的时间加以对照。
\par 最后我终于看清楚了:“下午两点!”我按了铃,可是我立刻又睡着了。从我再次醒来时感到的平静和对已经过了一个漫漫长夜的感觉来看,这次大概睡的时间长得多。然而,我之醒来乃由弗朗索瓦丝走进室内而引起,而她进来又是我按了铃的缘故。所以这次睡着,我自己觉得大概比上一次更长,而且给我带来这样的惬意和忘却,而实际上只持续了半分钟的工夫。
\par 外祖母推开我的房门,我就勒格朗丹家族向她提了一大串问题。
\par 只说我恢复了平静和健康,还远远不够,因为这已经远远超出与前一天相比平静与健康与我距离有多远这样一个简单问题。我一整夜都在与逆流搏斗,然后,不仅仅是我又回到平静与健康身边,而是平静与健康又回到我身上。头空空的,有一天大概会粉碎,头上有几处位置明确,还有些难受。头脑任凭我的思想驰骋。思想再次各就各位,并与生命重逢。可叹的是,时至今日,我的思想还不会好好利用我的生命。
\par 我再一次逃脱了无法入睡的困难,躲过了宇宙洪荒,躲过了歇斯底里发作的覆没。前一天晚上我无法得到安宁时威胁着我的一切,现在,我都不再害怕了。面前展现出新的生活。虽然我已经很舒服,但是仍然像骨头散了架一样。我一动不动,怀着喜悦品味着我的疲倦。疲倦将我双腿、双臂的骨头都拆散了,折断了,现在我感到,这些骨头都集中在我面前,随时准备重新接合起来。只要像寓言中的建筑师那样唱起歌来,我写上就能将骨架重新竖立起来。\footnote{宙斯与安提俄珀之子安菲翁从赫耳墨斯处得到竖琴这个礼物后,一心一意沉醉于音乐,经常与其兄仄忒斯争吵。但二人一致同意去解救他们的母亲(陷吕科斯及狄耳刻之手),并在底比斯称王。他们想在底比斯周围筑起城墙来。仄忒斯背石头时,安菲翁演奏竖琴将石头引到自己身边。拉斯金在作品中数次引用这个神话,认为它象征着各社会阶级之间的和谐。}
\par 突然,我忆起了在里夫贝尔见到的、凝望了我好一会的那个神情忧郁的金发少女。整个晚上,还有许多别的少女看上去很顺眼,而现在,只有她一个人刚刚从我记忆的深处升起。我似乎觉得她注意到了我,预料里夫贝尔的一个侍者会前来给我捎上她的一句话。圣卢不认识她,但是认为她还像样。与她见面,经常与她见面,可能很困难。但是为此我会不惜一切,我心中只想着她。
\par 哲学经常谈到自由的行为和必要的行为。一个行为,由于行动过程中抑制了升力,一旦我们的思想处于休整状态,这个行为便这样使某一回忆再次升起——直到此刻之前,这一回忆已被消遣的压力将它与其他回忆拉平——并叫它奔腾起来,因为它比其他回忆更有魅力。我们当时不知不觉,二十四小时过后我们才发觉。比这种行为为我们所更完整地感受的行为,恐怕没有了。说不定也没有比这更自由的行为,因为它还不具有习惯性的性质。在爱情中,正是这种精神怪癖有助于使某一个人的形象单独复活。
\par 正是我在海边看见那一群美女列队而过的第二天。我向好几位几乎每年都到巴尔贝克来的旅馆房客询问她们的情况。他们未能给我提供什么情况。此后,一张照片给我解释了何以如此。仅仅几年以前,她们还是一群依然孩子气十足、未定型而又甜美无比的小姑娘,人们可以看见她们在帐篷四周,围成一圈坐在黄沙上:她们好似隐隐约约的白色星群,即使你从中分辨出一双比他人更明亮的眼睛,在这看不清的银河星云中,也立即会将她忘掉,并与其他人的眼睛混成一片。现在,她们虽说还刚刚脱离女大十八变的年龄,但确实已经脱离了那个年龄。谁又能认出,她们就是几年前那一群小姑娘呢?
\par 在距今不远的那些年代里,肯定她们并不像前一日在我面前第一次出现时那样,给人一个群体概念。这个群体本身那时尚不够清晰。那时节,这些小毛孩子还太小,还处于成型的基本阶段,个性还不曾在每一张脸上打上自己的烙印。正像个体还浑沌存在的初级器官一样,更确切地说个体是由珊瑚骨构成,而不是由组成珊瑚骨的一个个珊瑚虫构成。那时她们还是你挤我我挤你地挤在一起。有时,一个小孩将身旁的小孩弄倒了,于是,一阵狂笑,似乎这是她们个体生命的唯一体现。人人前仰后合,这些线条尚不清晰、作着鬼脸的面孔混成了一团肉冻,闪闪发光,颤颤巍巍。在她们后来有一天给我看、而我亦保留下来的一张旧照片上,她们这孩子气的群体与日后她们那行列的面孔已经是同样数目。人们感到她们在海滩上已经留下了不同寻常的痕迹,禁不住对她们望上几眼。但是人们还只能通过理性逐个地辨认她们,而任凭女儿十八变,直到这些重新组合的形状逐渐侵占到另一个有个性的人上去,才算是分界线,又必须去认明那另一个有个性的人了。高高的身材与鬈曲的头发并存,又一个人的俏丽面庞很可能就是这照相簿上所显示的从前那个干瘪黄瘦的小毛丫头。这些少女,每个人的容貌特点在短暂的时间里有了那么大的变化,反使得这些特点成了一项模糊的标准。另外一方面,她们之间共同的和似乎群体性的东西,从那时起就是那么突出,在这张照片上,有时连她们最好的朋友也会把这一个认作那一个。要消除疑团,只能通过服装上的某个小玩艺,才可以肯定哪个人穿过这样的衣服,戴过这样的小玩艺,而其他人肯定没有。那个时节与我刚刚在海堤上看见她们那一天相比,差异是多么大,而这两个时间距离又是那么近。那个时节以来,她们仍然像我前一日感觉到的那样放声大笑,但是这种笑已不再是童年时期那种断断续续几乎是自发的笑声了。从前那种痉挛性的放松随时能叫这些脑袋去扎个猛子,犹似维福纳河中的鱼群,散开了,消失了,过了一小会又聚拢成群了。
\par 现在,她们的容貌已经成了自己的主人,个个目光紧紧盯着自己追逐的目标。只有我昨天那样第一次依稀望见,犹犹豫豫又抖抖瑟瑟,才会将这些孢子混淆起来,正像往日的狂笑与陈旧的照片将这些孢子混成一团一样。时至如今,这些孢子都具有了个性,而与那苍白的石珊瑚分离了。
\par 肯定,有许多次,在美丽的少女从我面前经过时,我向自己许下诺言,一定要再与她们见面。一般来说,她们不再出现。何况,记忆很快将她们遗忘,很难再找到她们的面庞。可能我们的眼睛还没有认出她们的时候,已经望见别的少女经过了。这些新出现的少女,我们将来也不会再与她们见面。
\par 另外有些时候,就像这狂傲的一群出现这样,偶然又非把她们再次带到我们眼前不可。这时,我们感到这是美妙的偶然,因为我们将从这偶然上分辨出似乎机体形成、发育之初以组成我们生命的东西。对于占有某些形象,事后我们会认为这是天注定的,而这种偶然将我们对某些形象的忠诚变成了轻而易举、不可避免的事,有时——继某些使人希望中止回忆的间断之后——则是很残酷的事。如果没有这种偶然,我们很可能像很多人一样,刚刚开始,就轻易地遗忘了。
\par 不久,圣卢的逗留已接近尾声。我并没有在海滩上与这些少女重逢。圣卢下午只在巴尔贝克待一小会,时间太短,无法顾及她们,也无法为了我去与她们结识。晚上他更得空一些,仍然常常带我去里夫贝尔。在这些饭馆中,正像在公园里和火车上一样,有些人在普普通通的外表之下隐形,而他们的名字会叫我们大吃一惊。偶然问到他们的名字,我们就会发现,他们根本不是我们以为的无足轻重的小人物,而正是我们久闻大名的某一位大臣或公爵。
\par 在里夫贝尔饭店里,已经有两三次,在圣卢和我看见所有的人开始离席时,有一个人刚刚来到,在一张桌旁落座。此人身材高大,肌肉发达,五官端正,胡子花白,然而沉思的目光总是死死地望着天。一天晚上,我们问老板这位阴阴沉沉的、孤独的、姗姗来迟的用餐者是何等人氏。
\par “怎么,这是鼎鼎大名的画家埃尔斯蒂尔,你们不认识?”他对我们说。
\par 有一次,斯万在我面前提过这个名字。怎么提起来的,我完全忘记了。但是,某一记忆的疏忽,与看书时对某一句子成分疏忽一样,有时不是促进把握不定,反而促进了过早的肯定。
\par “他是斯万的一位朋友,是非常著名、身价极高的艺术家。”我对圣卢说道。
\par 顿时,犹似一个寒战传到他身上和我身上,我们两个人都想到,埃尔斯蒂尔是一位大艺术家,名人。然后,我们又想到,他把我们与其他用餐人混成一团,肯定不会料到,想到他的天才我们多么激动。他对我们的崇拜一无所知,他也不知道我们认识斯万。如果我们没有来洗海水浴,大概我们也不会受到这场折磨了。但是,我们还迟迟停留在无法让热情保持沉默的年龄上,又设身处地想到隐姓埋名似乎令人压抑的生活,于是我们写了一封信,署上我们的名字。在信中,我们向埃尔斯蒂尔披露,坐在他几步开外地方的两个用餐者,是对他的才能极为倾倒的两个业余爱好者,是他的好友斯万的两个朋友。在信中我们要求向他致以敬意。一个侍者担当了将这封信函送交那位名人的任务。
\par 埃尔斯蒂尔虽然已经颇有名气,但是那时节,可能他还没有饭店老板声称的那样有名,稍微过了几年之后,他才大有名气。他是在这家饭店还仅仅是农庄一样时,最早来到这里居住并带来一群艺术家的人(那些艺术家,一俟人们在简单的挡雨披檐下露天吃饭的农庄变成阔气的用餐中心,便全部迁徙到别处去了。埃尔斯蒂尔本人与妻子住在距此不远的地方,只因妻子不在,他此刻才又到饭店来)。一位伟大的天才,即使在他还没有得到承认的时候,也必然会激起某些崇拜现象。不止一个稍事停留的英国女人,极想打听埃尔斯蒂尔生活的情形,农庄的老板从英国女人所提的问题或画家收到国外许多来信中便得以猜度出几分来。这时老板更注意到:埃尔斯蒂尔作画时不喜欢别人打扰;月色皎洁时,他深夜起床,把一个小模特儿带到海边,让她裸体摆出姿势来。待他从埃尔斯蒂尔的一幅画中认出挂在里夫贝尔入口处的木制十字架时,不禁心中暗想,受了那么多累没有白费,游人的赞美也并非没有道理。
\par “就是这个十字架,”他瞠目结舌地反复说,“四块木头全在!啊,他费了多大的劲啊!”
\par 可是,埃尔斯蒂尔送给他的一幅小小的《海上日出》是否价值连城,他倒不知道。
\par 我们看到埃尔斯蒂尔读了我们的信,将信放进自己的口袋,继续吃饭,然后开始要他的衣帽,站起来要走了。可以十分肯定,我们的作法使他不快,我们现在真希望(也真害怕)他还没注意到我们时,就赶快溜掉。我们从来没想到一件事,可在我们看来那是最重要的事,那就是我们对埃尔斯蒂尔的热情,我们不容许别人对这种热情的真诚表示怀疑,我们确实也可以拿等待时那颗悬着的心,愿意为这个伟人去赴汤蹈火来加以证明。但是这种热情,并非如我们自己想象的那样,是佩服,既然我们还从未看见过埃尔斯蒂尔的任何作品。我们情感的对象可能就是“大艺术家”这个空洞的概念,而不是一幅我们不曾见过的作品。充其量这是空洞的佩服,是没有内容的佩服的精神框架、感情骨架,也就是说,这是与童年紧密相连的某种东西,正像在成年人身上再也不存在的某些器官一样。我们还是孩子。然而埃尔斯蒂尔就要走到门口时,突然一拐弯,朝我们走来。我又惊又喜,紧张得无以复加。如果是几年之后,我就不会有这样的感受了。因为随着年龄的增长,人的能力越来越差,而对社交场合司空见惯又使人再也不会产生这样的念头,去挑起这样不同寻常的机会,去感受这样的激动了。
\par 埃尔斯蒂尔坐在我们餐桌旁跟我们谈了几句。我数次与他提到斯万,但是他从未回答我。我开始认为他并不认识斯万。他倒没有因此就不请我到他在巴尔贝克的画室去看他。这个邀请,他并没有对圣卢发出,这是因为我说了几句话,使他认为我很喜欢艺术而赢得的邀请。即使埃尔斯蒂尔与斯万是亲密好友,斯万的推荐恐怕也不会达到这样的效果(因为在人的生活中,无利害关系的情感所占的比例要比人们想的大)。他对我极其和蔼可亲,比圣卢还要过之,正像圣卢的和蔼可亲超过一个小市民的殷勤一样。与一位大艺术家的和蔼可亲相比,贵族大老爷的和蔼可亲,再动人,也有演戏、做作的味道。圣卢千方百计讨人喜欢,而埃尔斯蒂尔喜欢的是给予和献身。他拥有的一切,思想,作品,以及他认为次之又次之的其余东西,都会兴高采烈地送给一个理解他的人。但是他没有自己忍受得了的交际圈子,他在孤独中生活,还带有野性的成份。对此,上流社会的人称之为虚假作态,没有教养;当权者称之为思想有问题;邻舍称之为神经病;家人称之为自私和傲慢。
\par 肯定,最初时,即使在孤独中,他也愉快地想过,对于那些不理解或触犯过他的人,他通过作品与他们交谈,使他们对自己有充分了解。说不定他独自生活,并非出自对他人漠不关心,而是出自对他人之爱,正如我为了有一天能以更可爱的面目重新出现而放弃了希尔贝特一样。说不定他的作品就是为某些人画的,犹似返回他们之中。在这个返回中,人们虽然没有看见他本人,但是会喜欢他,钦佩他,谈论他。不论是病人也好,修道士也好,艺术家也好,英雄人物也好,当我们以当初的心态决定放弃什么的时候,一开始并不总是完全彻底的,后来,由于反作用,才对我们发生影响。如果说他曾经希望为某些人作画的话,那么作画的时候他可是为自己活着,远离他已经漠然视之的社会。孤独的实践使他爱上了孤独,正像我们一开始对任何大事都恐惧万分一般。因为我们知道这大事与更小的事不相容,而我们将小事看得很重。大事并没有剥夺掉我们的小事,而更多的是使我们脱离小事。在没有经历大事之前,我们的全部心思都在想知道我们可以在什么程度上将其与某些小小的快活调和,一旦我们经历了大事,那些小小的快乐便再也不成其为快乐了。
\par 埃尔斯蒂尔并没有与我们交谈很久。我准备那之后两三天内到他的画室去。但是,这个晚上的第二天,我陪外祖母从海堤尽头往卡那维尔悬崖方向去散步,回来走到直通海滩的一条小街拐角处时,我们与一个少女迎面而见。她低着头,像一头被人驱赶而很不情愿回圈的牲口,手里拿着高尔夫球棒,身后跟着一个盛气凌人的男士。此人很可能是他的“英国女家庭教师”,或是他一位女友的“英国女家庭教师”。那人与贺加斯\footnote{贺加斯(1697—1764),英国画家,木刻家,生于伦敦。其作品常具讽刺性,他希望创造出一种性格和风俗画派。其肖像画《杰弗莱一家》画的是律师杰弗莱、其妻及其二子女。也有另一种“版本”,不是律师杰弗莱,而是杰弗莱将军。此处不知指哪一幅。}《杰弗莱一家》中的肖像十分相像,面孔红红的,大概他最喜欢的饮料不是茶,而是杜松子酒。他蓄着花白而浓密的唇髭,没嚼完的嚼烟支出黑黑的一个弯钩,把唇髭又加长了一截。走在他前面的小姑娘,与那一小帮少女中那个戴着马球运动员式的黑色女帽、面颊丰满、面孔呆板却有着含笑的双眸的那个十分相像。此刻回家的这一个也戴着一顶黑色马球帽,但我觉得她比那一个更漂亮,她的鼻子线条更直,下部的鼻翼更宽,肉更多。其实,那一个在我面前显得是一个面色苍白而又傲气十足的姑娘,而这一个则显得是一个被制服了的孩子,面色红润。不过,由于她推着一辆一样的自行车,也戴着一样的鹿皮手套,我得出结论说,所见之差异可能是我所处的位置不同、情景不同所致,因为不大可能在巴尔贝克还有面孔如此相似、短打扮中又集中了同样特点的第二个姑娘。她飞快地往我这边扫了一眼。此后的日子里,当我又在海滩上看见这一小帮人,甚至以后我认识了组成这一帮的所有少女之后,我都从未敢绝对肯定,她们当中的哪一个——甚至在所有的人当中,与她最相像的那个推自行车的姑娘,就是我那天晚上在海滩尽头、街角上看见的那个少女。那个少女与我在这一帮子中注意到的那个,虽然差别不大,但毕竟是有些差别的。
\par 前些日子,我特别想那个高个子姑娘。但从那天下午开始,便是那个持高尔夫球棒、我推想她是西莫内小姐的姑娘重又搅得我六神无主了。她与别人在一起时,常常停下脚步,迫使她的女友们——看上去她们对她很尊重——也中止行进。我现在眼前仍然浮动着她停下脚步,马球帽下闪光的双眸,这身影映在大海在她身后为她构成的屏幕上,她与我之间,隔着透明的碧蓝的空间和自那时以来流逝了的时间。这面庞的第一个影像,在我的记忆中非常单薄,我向往着、追寻着,后来又将它遗忘,然后又找到了它。自那以后,我常常将这面庞映在往昔上,以便面对一个在我房间里的少女时,心中可以这样暗想:“就是她!”
\par 可是,我最想结识的,可能还是那个面色如绣球花、有绿色眸子的姑娘。何况,不论哪一天我更希望见哪一个,即使没有这一个,其余的姑娘也足以使我心情激荡。我的欲望,即使这一次基本扑在这个身上,下一次又基本扑在那个身上,但是仍像第一天我那模糊的视觉一样,我的欲望继续将她们聚集在一起,继续将她们当成一个单独的小世界。一个共同的生命使这个小世界活跃起来,大概她们也企望构成这个单独的小世界吧!如果我成了其中一个的男友,我大概就能进入——就像一个细腻的异教徒或一个小心谨慎的基督徒到了蛮夷之中——一个令人更加年轻的圈子里去。这个圈子洋溢着健康、无意识、肉欲、狠毒、非智性和快乐。
\par 我向外祖母讲述了与埃尔斯蒂尔的匆匆一晤,她为我能从埃尔斯蒂尔的友情中得到各种精神收获而感到高兴,认为我到此刻尚未去拜访埃尔斯蒂尔,既荒谬绝伦,又对人缺乏热情。可是我一心只想着那一小帮子,对于这些少女何时从海堤上经过没有把握,我不敢远离。外祖母对我衣冠楚楚也大为惊讶,因为我突然想起了直到那时一直扔在箱底的礼服。我每天更换一件,不重样,甚至给巴黎写了信,让他们给我寄新帽子和新领带来。
\par 在巴尔贝克这样的海滨休养胜地,如果一位美丽少女,一个卖海鲜、糖果或鲜花的女郎,其面庞在我们的心中用鲜艳的色彩描绘出来,对我们来说每天从清晨开始,便成为在海滩上度过的那些游手好闲而又阳光普照的日子的目标,生活便增加了极大的魅力。这样的日子虽然无事可干,像某些工作日一样轻松,但是给引到了某个方向上,受到了磁铁的吸引,朝某一即将到来的时刻稍微翘起了一点,这就是人们一面买油酥饼、玫瑰花、菊石,一面由于在一个女性面孔上见到了犹如纯洁地撒在一朵花上的鲜艳色彩而兴高采烈的时刻。但是,首先,这些小商贩,人们至少可以与她们讲话,这便免得用想象去建造简单视觉向我们提供的方面以外的其他各方面,去重新创造她们的生命,去夸大她们的魅力,如在一幅肖像画面前那样。特别是,正因为跟她们讲话,便可以得知在什么地方,什么时刻,可以再次见到她们。可是就那一小帮少女而言,对我来说,却绝非如此。她们的习惯,我不知晓。某些日子,不见她们的踪影,不知道她们不出现是何种原因。我便想找出一个规律,是否她们不出现有固定的时间,是否只能每两天看见她们一次,或者是与天气如何有关,抑或是否有些日子就永远也见不到她们。我事先将自己想象成她们的朋友,并且对她们说:“哪天哪天,你们不在吗?”“啊,对,那天是星期六,星期六我们从来是不来的,因为……”我还想寻找一个答案,即:如果知道凄凉的星期六,怎么玩命也没有用,你尽可以在海滩上东奔西窜,坐在点心铺子门前,装做吃奶油糕点,走进稀奇小玩艺儿商店,等待洗海水浴时刻到来,音乐会开始,涨潮来到,日落,夜幕降临,反正看不见心中向往的那一小群人,是否事情就同样简单呢?
\par 那要命的日子,可能一个星期内不只是重来一次。可能不一定非在星期六降临。可能某些气候条件对此也有影响,抑或与气候条件完全无关。对于陌生世界表面上这些不规则的运动,必须收集多少耐心却丝毫不平静的观察的资料,才能肯定自己没有为巧合所捉弄,肯定我们的估计不会错,才能对这激动人心的天文现象归纳出确切的规律来啊!这可是通过痛苦的体验换来的呀!有时我想起与今天相同的那个星期几没有看见她们,心中暗想,她们不会来了,在海滩上滞留毫无用处。可就在这时,我依稀望见了她们。反过来,有一天,我以为有些规律决定着这些星宿要返回了,我算出来这天应是一个黄道吉日,可是她们竟没出来。我会不会看见她们,这还是没有把握的事情中的第一件。还有一件更严重的没有把握的事情,那便是我以后会不会与她们重逢,因为我完全不知道她们是不是要动身到美国去或返回巴黎。这便足以叫我开始爱上她们了。对一个人是可以有口味的。但是要让作为爱情前奏的那种悲哀,感到无法弥补、焦躁不安一发而不可收,则必须有“不可能”这个危险才行。“不可能”这个危险焦躁不安地寻找一个目标去拥抱狂热,说不定目标正在这里,而不在一个人身上。相继谈恋爱过程中不断反复的这种影响,已经在这样起着作用(相继谈恋爱是可以发生的,但是恐怕更多是在大城市生活中。对女工而言,不知道她们哪天放假,生怕她们走出车间时没有看见她们),至少这些影响在我相继谈恋爱时是不断反复的。可能这与爱情密不可分。可能所有构成第一次恋爱特殊的地方又通过回忆,启示,习惯,通过我们生活前后衔接的一个个阶段,补充到后来的恋爱中去,赋予其各个方面以一种普遍性。
\par 在希望能与她们相遇的时刻里,我找到各种借口到海滩去。有一次,我们正在用午餐,我远远望见了她们。可惜我到的时候已经太晚,在海堤上等了很久,等待她们走过。此后我在餐厅里只待一小会,眼睛在蓝色的玻璃窗上搜寻。还没上餐后点心,我便站起身来,怕她们换了另外一个时间,而把她们错过。外祖母叫我与她待在一起的时间超过我认为最有利的时机时,我对她便很恼火,这成了她自己未意识到的坏心眼。我把椅子斜放,以尽量延长视野。如果我偶然瞥见了这群少女中的无论哪一个,既然她们全都属于同一特殊品种,我就像在眼前移动的魔怪般的幻觉中看见了幻梦的影子。这幻梦跟我作对,我又狂热地贪恋着它。这一刻之前,这幻梦还只存在于我的脑海中,此后却又经常在那里滞留了。
\par 我不专爱哪一个,我个个都爱,尽量与她们相遇对我打发日子又构成唯一甜蜜的因素,只有与她们相见才能使我心中升起打破一切障碍的希望。如果我没有看见她们,继这种希望之后而来的,便是狂怒。这种时刻,在我心中,这些少女遮住了外祖母。这时,如果说到什么地方去,她们会在那里,我立刻会高高兴兴奔了去。我自以为考虑别的事情,或什么都不想时,实际上我的心思完全愉快地勾在她们身上。当我甚至自己不知不觉地,更加无意识地想到她们时,对我来说,她们就是大海起伏的碧波,就是大海前列队而过的侧影。如果我到她们所在的哪个城市去,我定希望与大海重逢。对一个人最排他性的爱,总是对其物的爱。
\par 我现在对高尔夫球和网球极有兴趣,而放过了观看一位艺术家——外祖母知道他是最伟大的艺术家之一——作画和听他大发宏论的机会。外祖母为此很瞧不起我,我认为这种瞧不起乃源于某些狭隘的看法。从前我在香榭丽舍大街观察到,从那时起我自己更意识到,我们钟情于一个女子时,只是将我们的心灵状态映射在她的身上;因此,重要的并不是这个女子的价值,而是心态的深度;一个平平常常的少女赋予我们的激情,可以使我们自己心灵深处最隐蔽、最有个人色彩、最遥远的、最根本性的部分上升到我们的意识中来。和一个出类拔萃的人的谈话,甚至满怀钦佩地注视他的作品所能给予我们的愉快,却不能产生这样的效果。
\par 我最后只好服从外祖母。更叫我心烦的是埃尔斯蒂尔住在巴尔贝克最新开辟的一条街上,离海滩相当远。有电车从海滩街经过,白昼的炎热使我不得不乘电车前往。为了想象我是处于西梅里安的古王国之中,玛克王的国度中或波劳斯良德森林遗址中\footnote{在《特里斯丹和依索尔德》这个传说中,公主依索尔德许配给了玛克,他是高尔努阿耶国王。但是在船上,特里斯丹与依索尔德饮了魔酒,双双堕入爱河,他们逃进了波劳斯良德森林。这个森林如今叫班朋森林,位于伊尔维兰省,大部分骑士文学中的爱情故事发生在这里。},我极力不去注视在我面前伸展开去的建筑物那蹩脚的豪华。埃尔斯蒂尔别墅可能是这些建筑物当中最难看而又豪华的了。尽管如此,他还是租了下来,因为在巴尔贝克现存的别墅中,唯有这一栋能提供一间宽敞的画室。
\par 我穿过花园时,也是眼睛望着别处。花园中有一片草地,像巴黎郊区随便哪一位布尔乔亚的家中都拥有的一样,但是更小一些:有一个风流园丁的小雕像,从中可以端详自己的玻璃球,秋海棠作的边饰和一个小小的凉棚。凉棚下,一张铁桌子前,几张摇椅排开。接触到这些充满城市丑陋的东西之后,待我到了画室里,便不再注意覆盖接缝板条那巧克力颜色的条纹了。我感到很高兴,通过我四周的所有作品,我感到有可能将自己的情感升华到充满喜悦的诗意般的认识中去,形式多样,直到那时为止。我还没有把这些作品与现实中的整个情景分离开来。
\par 埃尔斯蒂尔的画室在我眼中,犹如世界上某种创新实验室。在这个实验室里,从我们见到的各种杂乱无章的事物之中,他从这里抽出在沙滩上碾碎自己丁香色泡沫的大海波涛,从那里抽出一个着白色人字纹布上装、臂肘支在船甲板上的青年,将它们画在各个长方形的画布上。这些长方形横七竖八地放在那里。青年的上装和飞沫四溅的浪涛,虽然失去了人们认为存在的内容,波涛再也不能溅湿,上装再也不能给任何人穿,但它们仍然继续存在,并因此而得到新的尊严。
\par 我走进去的时候,创作大师手中正握着画笔完成落日的形状。
\par 四面的窗板几乎完全关闭着,画室相当凉爽,只有一个地方,强烈的阳光在暗色的墙上印上那鲜艳而又转瞬即逝的装饰;只有一个长方形的小窗开着。小窗四周忍冬环绕,朝着一条大街,下面是花园一角。因此画室的绝大部分暗淡无光,空气透明,结成完整的一团,但在阳光将它嵌镶的裂缝处,既潮湿暗淡又闪闪发光,好似一大块水晶岩,其中的一面已经经过雕琢,磨平,此处彼处像一面镜子在闪烁,放出七色光。应我的要求,埃尔斯蒂尔继续作画,我则在这半明半暗中转来转去,在这幅画前停留一会,又在另一幅画前停留一会。
\par 我四周的画都是他的作品,大部分并不属于我最期望看到的类型。这些画,正如在大旅社桌子上扔着的一本英国艺术杂志所说,属于他的第一和第二画法,即神话画法和受日本影响的画法\footnote{日本艺术首次在法国出现是1855年的万国博览会。1867年与1878的万国博览会,日本馆得到极大成功,日本艺术在法国风行。有的学者认为惠斯勒(他于1883年在法国定居)对于日本艺术在法国的发展起了重要影响作用。}。据说,这两种画法,在德·盖尔芒特夫人的收藏中,均得到精彩的体现。当然,他画室中的作品,几乎全是在这里,在巴尔贝克取的海景。但我从中仍能辨别出,每一景的魅力都在于所表现的事物有了某种变化,类似诗歌中人们称之为的暗喻。如果说天父创造了每一事物,同时又给了它们一个名称,埃尔斯蒂尔则重新创造了它们,脱去了其名称,或者赋予它们另一个名称。表示事物的名称总是与理性上的某一概念相呼应,而理性与我们的真正印象是格格不入的,这又使我们不得不把一切与这个概念不相关的东西从事物中排除出去。
\par 在巴尔贝克旅馆里,早晨,弗朗索瓦丝将遮住阳光的毯子拿掉时,晚上,我等待着与圣卢一起外出的时刻到来时,我伫立窗前。由于光线的作用,有时我错把大海颜色更深的那部分当成了遥远的海岸,或者满怀欣喜地凝望着蓝色的流动的一片,不知那是海还是天。很快,我的理性将各个成份重新区分了开来,而我的印象则又取消了这种区别。在巴黎也是如此。有时我在自己房间里听到一场争吵,几乎是骚动,直到我将这声音与其原因联系上为止,例如一辆马车行驶到近前,我才能将那尖厉刺耳的斥骂声从这个声音里排除出去。我的耳朵确实听到了那尖厉刺耳的斥骂,而我的理性知道,车轮不会产生这样的声音。人们一如其本色富有诗意地见到大自然的时刻是罕见的,埃尔斯蒂尔的作品正是由这样的时刻组成。此刻在他身边的各幅海景中,他最常用的比较之一正是海天对比,而取消了二者之间的任何分界线。正是在同一幅画中,心照不宣地、不倦地重复这种对比,才在他的画中引进了形式多种多样的高度和谐。埃尔斯蒂尔的绘画在某些爱好者心中引起热烈反响,其原因正在于此,有时这些人自己反倒没有明确认识到这一点。
\par 最近几天他刚画完一幅画,这幅画表现的是卡尔克迪伊海港,我对这幅画凝望良久。例如在这幅画中,埃尔斯蒂尔就让观众对这种比较有思想准备,他对小城只使用与海洋有关的语汇,而对大海,只使用与城市有关的语汇。要么房屋遮住海港的一部分,要么捻缝的水塘、甚至大海深入陆地成为海湾,在这巴尔贝克一带常有这种情形。从修建了城市的前突尖角那边,房顶上露出桅杆(就像房顶上露出烟囱或教堂的钟楼一样),好似屋顶构成了船只,成了船只的一部分。然而这又具有城市特色,是在陆地上修建起来的。其他沿防波堤停靠的船只更加强了这种印象。船只那样密密麻麻挤在一起,竟然可以站在这只船上与另一只船的人聊天,而分辨不出他们是分开的,也分辨不出小的间隙,这捕鱼的船队还不如克里克贝克的教堂那样好像属于大海。克里克贝克的教堂,远远看上去,四面被水包围,因为人们看不见城市。在阳光和海浪有如尘土飞扬之中,教堂好像从水中钻出来一般,宛如白石或泡沫吹鼓而成,系在富有诗意的彩虹腰带上,构成一幅不真实而又有神秘色彩的图画。在前景的海滩上,画家想到了办法使人们的眼睛习惯于在陆地和大洋之间辨认不出固定的界限,绝对的分界线。几个壮汉正在把船只推向海中,他们既在海浪中奔跑,也在沙滩上奔跑。黄沙被打湿,仿佛成了水,映出船体。就是海水也不是齐平地往上涨,而是循着海岸的曲线上溢。远景更将沙岸撕成条条缕缕,一艘在茫茫大海上行驶的船只,被军舰修造厂快要完工的工程掩住了一半,竟像在城市中航行了。在岩石中捡拾海虾的妇女,因为四周都是水,又由于她们置身于岩石筑成的堡垒后面,地势较低,海滩(在最接近陆地的两端)降到了海平面上,她们倒像在海内岩洞之中了。这海内岩洞上部伸向船只和海浪,本身却在奇迹般分开的波涛翻滚中开辟出来并受到保护。虽然整个画面使人对海港产生海洋进入陆地之中,陆地具有海洋性质,人则成了两栖动物这样的印象,但是大海元素的力量仍然到处迸发出来。在防波堤入口处,岩石旁,大海喧嚣的地方,从水手的辛苦中,从船只倾斜成锐角卧在高耸的船坞、教堂、城市中的房屋前,有人回到城市、有人从城市出海打鱼中,人们感觉到他们艰苦地在水上奔忙,好似骑在马背上一般。这匹马性情暴躁,健跑如飞,但是,如果他们不够机敏和灵巧,那牲口一抖擞,就会将他们掀翻在地。
\par 一群游人兴高采烈地乘坐一只小船出海,小艇摇摇晃晃,像一辆蹩脚的马车。一个天性快活同时又很聚精会神的水手,犹如用缰绳驾驶马匹一样驾驭着小船,张开有力的风帆。每个游客都乖乖地坐在自己的位置上,避免船只一侧过重而倾翻。在阳光灿烂的田野里,在绿荫覆盖的名胜区,人们也是这样奔跑着滚下山坡的。虽然下过暴雨,仍是风和日丽的上午。甚至人们还能感觉到平稳不动的船只享受着阳光和荫凉,在大海那样平静的部分,要保住这完美的平衡需要制服什么样的强大阻力!大海那样平静,比起由于阳光的作用似乎已经蒸发的船体来,水中的倒影似乎更结实,更真实。远景更使船体显得鳞次栉比。或者更正确地说,我们还没有提及大海的其他部分。这些部分之间,差异很大,就和某一部分与出水的教堂以及城市后面的船只之间差异很大一样。这边暴风雨,漆黑一片;稍远一些,色彩鲜艳,而且与天空一样如同涂上了釉彩;那边,阳光、云雾和泡沫使大海那样雪白,那样连成一片,那样具有土地气息,那样具有房屋的假象,人们甚至会以为那是一条石路或一片雪原。可是人们又看到那石路或雪原上有一条船,不免吓了一跳。船只悬在陡坡上,停在旱地里,好像一辆马车刚刚走出涉水而过的地段正在晾干。可是,过了一会,人们又在这结结实实的高原那高低不平的辽阔平面上,看见了一些摇摇晃晃的船只。这时人们才醒悟过来,这还是海,而这各种景象都是真实的。
\par 人说在艺术上无进步无发现可言,只在科学上才有;每个艺术家都得自己重新开始个人的努力,任何别人的力量既帮不了他的忙,也阻碍不了他。虽然这么说不无道理,但是还必须承认,在艺术揭示了某些规律的范畴内,一旦某种技巧将这些规律普及,回头一看,先前的艺术就失去了一些新颖独特之处。自埃尔斯蒂尔开始作画起,我们已经经历了人们称之为自然景色和城市的“精彩”摄影阶段。业余爱好者在这种情况下使用这个形容词到底指的是什么呢?要想说明白,我们就会看到,这个形容词一般是用来指一个熟悉的事物所呈现的奇特形象。这个形象与我们司空见惯的不同,奇特然而又是真实的,因此对我们来说倍加引人入胜,因为这个形象使我们惊异,使我们走出了常规,同时又通过唤起我们一种印象使我们回归到自己。例如,这些“精彩”摄影中的某一帧,体现了远景的一个规律,给我们看的是我们的城市中司空见惯的某一大教堂,却从精心选择的一个点上来拍摄。从那个点上看,它似乎比房屋高出三十倍,而且与江边成突角,实际它与江边距离很远。埃尔斯蒂尔下工夫不是原封不动地——他知道原是什么样的——将事物摆出来,而是按照我们原始视觉赖以构成的光学幻觉将其呈现出来。这种工夫正好使他要阐明某些远景规律,这就更叫人惊异,因为是艺术首先揭示了这些规律。一条江,由于水流的曲折,一海湾,由于表面上看靠近悬崖,似乎成了平原或山中掘出的四面绝对封闭的一湖泊。从巴尔贝克取景,赤日炎炎的一个夏日画的一幅画中,大海凹进来的一块,由于封闭在粉红花岗岩岩壁中,似乎不是大海,而大海从稍远的地方才开始。大洋的连续性只通过一些海鸥暗示出来。海鸥在观众以为是石头的东西上面飞旋,吮吸着波涛的潮湿气息。
\par 这同一张画,还揭示出其他的规律。例如,在高高耸立的悬崖脚下,点点白帆映在蓝色明镜中,宛如沉沉入梦的蝴蝶,极尽小巧之美;又如某些阴影暗与光线之亮的强烈对比等。摄景艺术已使阴影的变化无穷家喻户晓,但是埃尔斯蒂尔对阴影的变化无穷那样感兴趣,从前他竟专心致志地喜欢画真正的海市蜃楼。在海市蜃楼中,顶部有塔楼的古堡显出一座完全圆形的古堡模样,顶部有一塔楼将其延长,底部反方向又有一塔楼,也许是天空格外晴朗赋予映在水中的倒影以石质的坚硬和光泽的缘故,也许是晨雾使石头与影子变得一样烟雾缥缈。同样,远处,大海之外,一排远树之后,另一大海开始,落日将它染成玫瑰色,而这正是天空。阳光,如同一种新的固体被创造出来,推动着它直接照射的船体,后面另一船体则笼罩在阴影之中,犹如将水晶楼梯的一级一级摆在一个表面上。从物质构成说,这表面是平的,但是清晨大海的光照将这表面折断了。一条江从一座城市的桥下流过,从那样一个视角取景,这条江竟然显得完全支离破碎了,这里摆成湖,那里细如网,别处又由于安插了一座树木覆盖山顶的小丘而折断,城中的住户晚上到这山顶的树林中来呼吸夜晚凉爽的空气。这座动荡的城市,其节奏本身,只通过钟楼那不折不弯的垂直来表现。钟楼并不伸向天空,通过沉重的直线,就像在凯旋进行曲中一样表明生活的节奏,似乎在自己的身躯下悬挂着沿着折断、压碎的江流笼罩在薄雾之中的楼房那更模糊的整个一大片(由于埃尔斯蒂尔最初的作品产生于用一个人物点染风景画的时代)。在悬崖上或在山中,道路,这自然景色中半有人情味的部分,也和江河或海洋一样,受到远景的侵蚀。或是山峰,或是瀑布的烟雾,或是大海,使人无法沿着道路持续向前,这道路对于游人是可见的,对我们却并非如此。着过时服装的小小人物,迷失在这荒凉孤寂之中,似乎常常在深渊前停步,他遵循的羊肠小道这里已是尽头。而在再过去三百米高处的松林中,我们看见小道那好客的沙土,白白细细的一条又在游人脚下出现,真是叫我们放了心,眼睛也受到了感动。是山坡环绕着瀑布或海湾,为我们掩住了小路中间衔接的九曲十八弯。
\par 埃尔斯蒂尔下工夫在现实面前脱去智性的一切概念,是非常了不起的。尤其他在作画前要让自己变成一无所知,出于正直而忘掉一切(因为人们所知道的事物并不属于自己),而这正是有高度修养的智慧。我在他面前承认我站在巴尔贝克的教堂前感到很失望时,他对我说:
\par “怎么,那大门使你感到失望吗?这可是民众永远读不明白的历史化了的最美的圣经啊!那圣母像和所有叙述她生平的浮雕,是中世纪为歌颂圣母所展开的长卷赞美诗最美好、最有诗意的体现。除了要细致准确地表现圣经以外,年迈的雕刻家又有怎样崇高的发现,进行了多少深邃的思考,赋予其怎样的优美的诗意啊!天使们运送圣母躯体的裹尸布,太神圣了,他们不敢直接触及(我对他说,在圣安德烈教堂也研究了这个主题。他见过圣安德烈教堂大门的照片,但他向我指出,那些小农民,所有的人都同时在圣母的周围奔跑,与此处的两位几乎意大利式的那么苗条,那么温柔的大天使,不可同日而语),这是多么了不起的想法!将圣母的灵魂摄走以便与圣母的肉体合在一起的那个天使;在圣母与伊丽莎白相遇那一节,\footnote{见《新约全书》路加福音第一章。}伊丽莎白触到玛丽亚的乳房,感到乳房隆起而深感惊异的那个动作;没有亲手摸到之前,怎么也不肯相信无玷始胎的接生婆那包裹着的手臂;圣母为了向圣徒多马证明她已复活而向他掷过去的腰带;还有圣母从自己胸前撕下用以遮掩自己儿子赤裸的身体的那块细麻布——在其子的一侧,教会收集鲜血,那是圣体圣事的饮料;另一侧,是统治已结束的会堂,蒙着双眼,手握折断一半的权杖,王冠从头上落下,同时任凭前朝法版滚落在地;最后审判时节,丈夫帮助自己年轻的妻子从坟墓中走出来,将她的手按在她自己的胸口上,为的是叫她放心,并向她证明那心脏确实在跳动,这不也是相当费心思找到的不错的想法吗?还有那个将太阳和月亮带走的天使,既然十字架的光辉将比星辰的光辉强七倍,太阳和月亮不是毫无用处了吗!还有将手浸在耶稣的洗澡水里,看看水是否够热的那个天使;从云端里降下将花环戴在圣母前额上的那个天使;还有所有从天上耶路撒冷圣殿的栏杆之间俯身向下,看见恶人受罪、好人享福,分别由于恐惧或快乐扬起手臂的那些天使!你看到的这些,就是天上的各个团体,就是神学和象征性的整个伟大诗篇!这简直荒唐,简直神妙至极,比你将在意大利之全部所见好上一千倍!何况意大利的三角楣是天才大为逊色的雕塑家原封不动抄袭来的。你一定明白,所有这些玩艺,无非是一个天才问题。人人都有天才的时代,并不曾有过。这么说,全是胡说八道,那要比黄金时代还厉害。雕了这样的门面的家伙,请你一定相信,他也很厉害,与现在你最崇拜的那些人相比,他的思想也和他们一样深刻。如果我们一起去意大利,我会把这些指给你看。圣母升天节宗教仪式的某些歌词在这里得到非常精巧的表现,就是勒东\footnote{奥狄龙—勒东(1840—1916),从一开始就强调想象在艺术中的作用。他本人既是油画家,又是水彩画家,石板画家,粉画画家。作品中宗教题材占很大比重。新的一代画家如鲍那尔,维亚尔,莫里斯·德尼等将他视为大师。}也无法与之媲美。”
\par 他与我谈到的这个广阔仙界,庞大的神学诗篇,现在我终于明白是这样谱写出来的了。当初我在正门前张开充满渴望的双目时,却没有看见这些。我与他谈起那些高大的圣徒雕像,竖在高高的底座上,似乎形成了一条大道。
\par “这条大道从远古时代开始,最后达到耶稣·基督,”他对我说,“一边是耶稣精神上的祖先,另一边是犹大之王,是耶稣肉体上的祖先。每一世纪都集中在这里了。你视为底座的那东西,如果你看得更仔细一些,你就能叫出蹲在高处的人的名字了。因为在摩西脚下,你会认出金牛来;在亚伯拉罕脚下,你会认出羊来;在约瑟夫脚下,你会认出给皮蒂法尔老婆出主意的恶魔。”
\par 我还对他说,我本来以为会看到一所几乎是波斯式的建筑,这大概也是我感到失望的原因之一。
\par “不,不,”他回答我说,“有许多是真的。某些部分完全是东方式的。有一根柱子是那样准确地重现了一个波斯题材,东方传说无所不在这一点竟然不足以解释这种现象。雕刻家肯定是抄袭了航海家从东方带来的一匣子东西。”果然,他给我看了一根柱子的照片,我从柱头上看见几乎是中国式的龙相互吞噬。但是在巴尔贝克,在建筑物总体中,这一小块雕刻未引起我们注意就过去了,而建筑的总体与“几乎是波斯式的教堂”几个字向我展现的情景并不相似。
\par 在这个画室里,虽然我体会到精神上的快乐,但是这丝毫挡不住我感觉到透明涂料的温热,房间那火星四溅的半明半暗,忍冬环绕的小窗外完全乡下气味的大街上被烈日烧灼的土地那持续的燥热。这一切包围着我们,我们已无法自主。只有远方的树荫才给太阳蒙上一层面纱。看到《卡尔克迪伊海港》这幅画叫我十分快乐。这个夏日使我感到的意识不到的舒适,可能又像一条河流的支流一样,扩大了我的快乐。
\par 我本来以为埃尔斯蒂尔很谦和。可是在一句表示感谢的话里,我用了“荣誉”一词时,我看到他的面孔因悲哀而稍稍变了样,这时我才明白我是大错特错了。认为自己的作品永世长存的人——埃尔斯蒂尔正属于这种情形——惯于将自己的作品置于他们本人已化成尘土的时代之中。所以,“荣誉”这个概念使他们不得不对这个虚无世界进行思考,叫他们悲伤,因为这个概念与死亡的概念密不可分。
\par 想不到无意间使这高傲感伤的乌云升上埃尔斯蒂尔的眉宇,我赶紧改变话题以驱散这片乌云。
\par “人家劝我不要到英国去,”我想到从前在贡布雷与勒格朗丹的谈话,而且希望就这一席谈话得知他的见解,便对他说,“说是这对一个已经爱好幻想的头脑不利。”
\par “哪里!”他回答我说,“一个人的头脑已经倾向于幻想的时候,不应该让它离开梦幻,不应该对它进行限制。一旦你叫自己的头脑离开梦幻,你的头脑就再也不理解自己的梦幻了。你将为千百种表象所捉弄,因为你没有理解那表象的本质。如果说有点幻想是危险的,那么医好这一病症的,绝不是少幻想,而是更多的幻想,整个成为幻想。为了不再为幻想所苦,要完全理解自己的幻想,很重要。将幻想与生活适当分开,大有益处,以至我自忖,是否应该像某些外科医生主张应该将所有儿童的阑尾一律割掉以避免将来罹患阑尾炎那样,早早就预防性地将幻想与生活适当分开。”
\par 埃尔斯蒂尔和我一直走到画室的尽头,站在窗前。窗子在花园后面,朝向一条狭窄的横街,几乎是一条乡间小路。我们来到这里呼吸将近傍晚的清新空气。我认为自己离那一小群少女十分遥远,正是下定决心牺牲一次看见她们的希望,我才终于听从了外祖母的请求来看埃尔斯蒂尔的。你寻找的东西在哪里,你并不知道,而且常常长时期回避由于别的原因每个人都请我们去的地方。但是我们料想不到,正是在这里我们会看见自己日夜思念的人。我毫无目的地望着这条乡间小路。小路从画室外紧擦画室而过,但已不属于埃尔斯蒂尔。
\par 突然,那里出现了一小帮子中那个推自行车的少女。她快步沿着这条小路走来,乌黑的秀发上,戴着她那马球帽,帽子压得很低,下面是她那丰满的面颊和快活而又有些执拗的双眼。我看见她在这条奇迹般幸运、充满柔情许诺的小路上,从树下向埃尔斯蒂尔送过一个友好微笑的问候。这简直是一道彩虹,对我来说,它将我们的地球世界与迄今为止我们认为无法企及的地域连接了起来。她甚至走过来将手伸给画家,但没有停下脚步。我看见她下巴上有一颗美人痣。
\par “先生,您认识这位姑娘吗?”我问埃尔斯蒂尔,我明白他可能把我介绍给她,请她到他家来。于是,这间乡间景色环绕的宁静的画室,充满了更多一层的诗意。好比在一所房子里,一个孩子已经待得很高兴,当他又得知,漂亮的东西和高贵的人非常慷慨大方,要无限增加他们的馈赠,正在为他准备一席精美的茶点时一样。
\par 埃尔斯蒂尔告诉我,她叫阿尔贝蒂娜·西莫内,同时也一一道出她的其他女友的名字。我对这些女孩描述得相当准确,所以他道出她们的名字时毫不犹豫。对她们的社会地位,我想错了,但是与一般在巴尔贝克的判断错误并不属于同一类型。店铺掌柜的儿子骑在马上,我轻易地将他们当成王子。可是这一次,我倒把属于相当富有的小布尔乔亚、工商业界家庭的一些少女给安到一个可疑的阶层里去了。这个社会阶层问题,一开始时我最没有兴趣。对我来说,无论是下层民众,还是盖尔芒特家那样的上层社会,都没有什么神秘。肯定,如果海滨生活那色彩斑斓的空虚没有在我看花了眼的双目前事先赋予她们某种魅力,而且她们再也不会失去这种魅力的话,说她们是大批发商的女儿,我大概也不会与这个概念胜利地抗争到底。现在,我只能对法国布尔乔亚是一个绝妙的最丰富多彩的雕塑作坊表示钦佩了。多少出人意料的类型!从面部特征上,是多么了不起的发明!面部线条上,又是怎样的决断,怎样的新鲜,怎样的质朴!这些迪安娜\footnote{迪安娜为希腊神话中之猎神。}和仙女竟然出自吝啬的老布尔乔亚阶级,我真觉得这些老布尔乔亚也是最高大的塑像了。
\par 这些少女社会地位的变化,我还没来得及察觉,在她们那流里流气的面孔后面,又一个想法已经扎下了根。原来我认为她们是自行车运动员、拳击冠军的情妇,现在又觉得她们很可能与我们认识的某一律师家庭关系非常密切了。这些发现的错误,对一个人观念的改变简直具有化学反应般的瞬时性!
\par 阿尔贝蒂娜·西莫内是什么样的人,我所知甚少。肯定她对于某一天她之于我如何,也毫无所知。甚至我在海滩上早已听人说过的西莫内\footnote{西莫内写作“Simonet”。}这个姓,有人叫我写出来的话,我可能会写成两个“n”,一点也料想不到这个家族对于只有一个“n”看得很重。在社会阶梯上,越往下,时髦玩艺越抓住一些鸡毛蒜皮不放。这些鸡毛蒜皮的事,可能并不比贵族的那些标记更毫无意义,但是,这些玩艺更莫名其妙,更因人而异,更叫人惊诧。可能有过姓Simonnet的人干过坏事,甚至比这还糟。总而言之,据说,别人若是将他们的姓写成两个“n”,这西莫内家的人便要大光其火,犹如受了诽谤一般。蒙莫朗西家族为自己是法兰西最早的男爵而感到自豪,而他们为唯有自己姓只有一个“n”的西莫内、而不是两个“n”的西莫内,大概感到同样自豪。
\par 我问埃尔斯蒂尔,这些少女是否住在巴尔贝克。他回答我说,其中某些姑娘是住在巴尔贝克的。有一个姑娘家的别墅就在海滩的尽头,就是卡那维尔悬崖开始的地方。由于这个姑娘是阿尔贝蒂娜·西莫内的挚友,我更加有理由相信,我和外祖母在一起遇到的那个姑娘正是阿尔贝蒂娜·西莫内。当然,有那么多条与海滩成垂直方向的小街街角都很相似,我也无法准确无误地认出那是哪一条街。人们希望记忆准确无误,但是就在当时,视觉就是模糊的。然而,阿尔贝蒂娜与走进女友家的那个少女是同一个人,这一点实际上可以肯定。虽然如此,此后,棕色头发的高尔夫球运动员在我面前呈现的无数形象,不论此形象与彼形象多么不同,全都重叠在一起。如果我沿着回忆的线索上溯,在这个特征掩护下,就像在一个内部通道中一样,我可以从所有这些形象面前经过,而无法从同一个人中绕出来。反过来,如果我希望一直上溯到我与外祖母在一起那天路遇的那个少女,我必须再走到露天中去。我确信又找到了阿尔贝蒂娜,她与走在自己的女友中间,在散步中经常停下来,高出海平线的那个,是同一个人。但是,所有上述的形象依然与最初的那一个形象相分离,因为我无法在事后赋予她在给我的双眼留下深刻印象那一刻对我而言她不具有的特点。不管概率计算能给我什么保证,在小街与海滩的转角处那样大胆地望了我一眼的,我以为可能会爱上我的那个双颊丰满的姑娘,我从来没有与她重逢过。
\par 我在这一小帮的各个少女之间犹疑不定,她们每个人都保留了一点首先使我心荡神驰的集体魅力。这种犹疑是不是又给上述的原因增加了一条,给我后来,即使在我最热恋阿尔贝蒂娜——是我第二次谈恋爱——的期间,留下一种间歇的而且短暂的不爱她的自由呢?由于先在她的所有女友之间游荡,后来才固定在她身上,我的爱情有时在爱与阿尔贝蒂娜的形象之间保留着某种“游戏”性质,这种游戏,像没有对准的光束一样,使爱情先落在别人身上,然后才回来施加在她的身上。我心中感到不自在与对阿尔贝蒂娜的回忆之间,似乎并没有什么必要的联系,说不定与另一个人的形象也能联系在一起。这种想法在闪电般的一瞬间,使我能够将现实化为乌有,不仅是如我对阿尔贝蒂娜的爱这样的外部现实(我承认我对阿尔贝蒂娜的爱是一种内心状态,在这种心态中,完全从自己心中引出我爱的人的特殊品格,特别性格,使得爱情对我的幸福成为不可或缺的一切),甚至是内心的纯主观的现实。
\par “没有哪一天,她们当中这个人或那个人不从画室前经过并走进来略作拜访的。”埃尔斯蒂尔对我说。如果外祖母叫我来看他的时候我立刻就来,很可能我早就结识阿尔贝蒂娜了。想到这里,埃尔斯蒂尔的话真叫我伤心。
\par 她走远了。从画室里再也看不见她了。我想,她到海堤上会女友们去了。如果我早能和埃尔斯蒂尔一起到海堤上去,也会结识她们了。我编出一百样借口来,好叫他同意跟我到海滩上去转一圈。那个少女在那面小窗的窗框里出现之前,我的心是平静的。现在我失去这种平静。那面小窗,直到那时为止,在忍冬的包围中是那样动人,现在却变得空荡荡了。
\par 埃尔斯蒂尔对我说,他要去跟我走几步,但是他不得不首先画完正在画的那幅画。这叫我感到快乐,快乐中又夹杂着折磨。他画的是花,但不是山楂花,刺玫花,矢车菊,苹果花——我如果要向他订一幅画,我更希望订画这些花的画,而不是一幅人物肖像,以便通过他天才的揭示,得悉我经常在这些花前寻觅而始终不可得的东西到底是什么。埃尔斯蒂尔一面作画,一面与我谈植物,但是我听不进去。光是他一个人已经再也不够,他现在只不过是那些少女与我之间必要的中介。他的威望,不久以前对我来说,还由他的天才赋予。而现在,只有在他即将为我介绍的那一小帮子人眼中,他将这种威望泽被到我,这威望才有价值。
\par 我踱来踱去,巴不得他赶快画完。我抓住一些习作仔细端详。许多习作靠墙翻过去,一个压一个地摞在那里。我就这样碰巧发现了一幅水彩画。这幅画大概是埃尔斯蒂尔绘画生涯中很久以前某个时代的作品,使我特别着迷。一些作品不仅仅技巧高超,而且立意那样不同寻常,那样诱人,我们竟然会将作品魅力的一部分归之于立意,似乎这种魅力,本来在大自然中就已经具有物质存在形式,画家只要去发现,去观察,去描摹出来就行了。这样的作品使人特别着迷。这样的物品能够存在,甚至将画家的表现形式抛开不谈也是美的,这就满足了我们心中天生便具有而后来又被理性所打倒的唯物论,而且为美学的抽象充当砝码。
\par 这幅水彩画,是一位少妇的肖像。她并不美丽,却属于一种独特的类型。她头上戴着一顶包头软帽,与帽檐上饰有樱桃红绸带的瓜皮帽很相似。两只手戴着露指手套,一只手擎着一支点燃的烟卷,另一只手将一顶纯粹为了遮阳用的果园大草帽样的东西举到膝盖那么高。她身旁有一张桌子,桌上有一花瓶,插满了玫瑰花\footnote{以下两处则说花瓶中插的是石竹花。}。这类作品妙就妙在它们是在特殊条件下完成的,而我们一下子弄不清楚。常有这种情形,这幅画即是如此。例如我们不知道一个女性模特儿那奇异的装束是不是化装舞会上的化装,抑或一个老头身着红大衣,看上去他故意穿上这件衣服以迎合画家的异想天开,可是我们不知道这是他的教授袍还是董事袍,还是他的主教披肩。我眼前的这张肖像画,画中人的性格叫人捉摸不透,原因是这是一位昔日的年轻女演员,半反串,叫人不明白。她那短发在瓜皮帽下蓬松隆起。她那丝绒上装没有大翻领,中间是白色的硬胸。这瓜皮帽和上装叫我一时拿不准这时装是何时期之物,这模特儿是男是女。结果是,除了知道我眼前是画家最明快的一幅画以外,我什么也说不准。
\par 这幅画使我感到的快活,又被担心所扰乱,我怕埃尔斯蒂尔又磨磨蹭蹭,叫我错过了那些少女,因为那小窗上的日影已经西斜。这幅水彩画上,没有哪一件东西可以简简单单地加以证实就算了事,之所以画出来,那是因为在这场景中有用。画衣着是因为那女子必须穿衣,画花瓶是因为有花。花瓶的玻璃本身就招人喜爱,似乎灌上了水,石竹花的花茎插在瓶中,犹如浸在与水一样清澈、几乎与水一样液态的物质中。女子的服装以独具一格而又令人感到亲切的魅力笼罩着她,似乎工业产品可以与造物主的奇迹相媲美,这些奇迹就和猫毛、石竹花瓣、鸽子羽毛一样娇嫩,视觉接触时感到那样甜美,画得那样鲜艳。硬胸雪白,细如雪霰,那轻盈的褶皱呈钟形小花状,恰似铃兰的花朵,在房间明亮的折射光中开放。这折射光本身本来很强烈,但是正像花束会在被单上映出缕空的花朵一样,这光线也稍稍减弱了一点。上装的丝绒闪射着珠光,这里那里有什么竖起来,有什么撕碎了,有什么毛茸茸的,使人想到花瓶中散乱的石竹花。但是人们特别感觉到的,是埃尔斯蒂尔对一位年轻女演员的这身化装服饰会表现出什么样的道德败坏完全不在乎,对他来说,她会对某些观众那已经麻木或已经堕落的感官产生什么样的刺激,与她扮演自己角色的天才相比,大概更加重要。因此他反而着力于这些模棱两可的特点,就像着力于某一值得突出、他也极尽所能加以强调的美学成份一样。
\par 循着面部线条看,似乎就要承认其性别是一个有点男孩子气的姑娘了。可是就在这时,那性别又消失了,再过去,重又出现,而暗示给人的,毋宁是这样的想法:这是一个女性化的、有恶习的、想入非非的小伙子。此后性别又逃走了,始终无法捕捉得住。目光中那种耽于幻想的忧郁,与属于花天酒地的阶层和戏剧界的那些细节形成强烈对比,这个特点并不是最不会使人心绪动荡的。此外人们会想,这是假扮的,着这身富有挑逗性的服装似乎主动送给人家去抚摸的这个年轻人,很可能觉得再加上点保留在内心的秘密情感、秘不告人的忧郁这样浪漫主义的表情,会更有刺激性。肖像的下方写着:Miss Sacripant\footnote{塞克里本特小姐。《塞克里本特》为吉尔和杜布拉多于1866年创作的一部喜歌剧。剧中男主角男扮女装出现。},1872年10月。
\par 我忍不住叫起好来。
\par “噢,这算不上什么,是年轻时候匆匆画成的东西,那是给杂耍剧院杂耍剧院始建于1807年,位于蒙马特大街7号,第二帝国时代因上演轻松的喜剧及歌剧而名气大振。后来主要在这里上演通俗喜剧。普氏本人曾于1909年11月27日去该剧场观剧。演出画的服装。是老早以前的事了。”
\par “那模特儿后来怎么样了?”
\par 我的话先是叫他一怔,过了一秒钟,他的脸上现出一副毫不在意、心不在焉的表情。
\par “喂,快把那张画给我,”他对我说,“我听到埃尔斯蒂尔太太的脚步声,她来了。虽然戴甜瓜帽的那个年轻人在我的生活里没有起过任何作用,我向你保证,但是叫我妻子看见这幅水彩画毫无益处。我之所以保存这幅画,不过作为那个时代戏剧一个很好玩的材料罢了。”
\par 可能埃尔斯蒂尔已经很久没见过这幅水彩画了。他向画注视了一下,然后将它藏起来。
\par “我必须只保存头部,”他自言自语地说,“下部画得太糟糕了,那双手简直是商人的手。”
\par 埃尔斯蒂尔夫人来到,更要耽搁我们,我心里真难受。窗户的边很快就成了玫瑰色。我们即使出门去,大概也要一无所获了。再没有看见那些少女的任何可能了。因此,埃尔斯蒂尔太太离开我们是快是慢,也再没有任何意义。她并没有待很久。我觉得她特别令人生厌。小上二十岁,在罗马乡间牵着一头牛,她很可能是个美人儿。但是现在,她的黑发正在变白。她普普通通,却又不朴素自然,因为她认为举止庄重、态度庄严乃为她那雕塑美所必需,而她的年龄已使她的雕塑美失去全部魅力。她的服饰极为朴素。埃尔斯蒂尔每时每刻都用含有敬意的柔情蜜意说:“我的美人加布里埃尔!”似乎只要说这句话,就会使他动情,使他满怀尊敬。听到他这样说,人们很受感动,但也感到惊异。后来,当我见识了埃尔斯蒂尔的神话题材绘画以后,倒也觉得埃尔斯蒂尔太太姿色增加了几分。我明白了,既然他将自己的全部时间、整个的思考工夫,一句话,自己的整个生命都献给了更好地分辨这些线条,更忠实地重现这些线条,那么事实上,他早就将几乎天神般的性格归之于某种理想类型,某种准则了。这种理想类型可归结为某些线条,某些在他的作品中不断反复出现的阿拉伯花纹。这样的理想给予埃尔斯蒂尔的灵感,确实是那样严肃、那样要求很高的迷信,这种信仰竟然从不允许他感到满意。这个理想,就是他本人心中最秘不示人的部分,所以他无法将这理想看得很淡漠,无法从中得到激情,直到他遇到了这个理想的那一天。那一天,他在一个女郎的躯体上,遇到了已在外部实现的这个理想。这就是后来成了埃尔斯蒂尔太太的那个人的躯体。从她身上,他得以感到那理想是崇高的、感人的、神妙的——只有对不是我们自己,我们才能有这种感受。直到那时为止,必须千辛万苦从自身开发的美,顷刻,神秘地化成了肉身,主动献身给他,以结成卓有成效的情感一致的硕果。将双唇按在这美上,啊,心灵会得到怎样的宁静!
\par 那时的埃尔斯蒂尔,已不再处于只期待思维旺盛就可以实现其理想的青春年少时代,他已接近指望通过肉体的满足来促进精神充沛的年龄。我们精神疲劳了,往往倾向于物质至上;活动减少了,往往倾向于被动接受影响。精神的疲劳与活动的减少开始使我们同意这样的观点,那就是可能确有某些得天独厚的躯体、行业、节奏能那样自然而然地实现我们的理想,以致即使没有天才,只要描摹某一肩部动作,某一脖颈的紧张,我们就能创造出一幅杰作来。这是我们喜欢用目光去抚摸美的年龄,这美在我们身外,在我们身边,在一幅挂毯上,在旧货商店里发现的一幅提香所作的美妙画稿中,在与提香画稿同样美丽的情妇身上。我理解了这一切之后,每次见到埃尔斯蒂尔太太,再也不能不感到快乐,她的身躯也失去了沉重的臃肿,因为我用一个想法充满了她的躯体,那就是她是非物质的造物,是埃尔斯蒂尔的自我写照。对我来说,她也是一幅肖像画,对他大概也是如此。对艺术家来说,生活中的材料是不算数的,只是显露其天才的一个机会而已。将埃尔斯蒂尔创作的十幅不同人物肖像画排列在一起去看,人们会清楚感觉到,首先,这些人跟埃尔斯蒂尔全是一家人。天才汹涌澎湃覆盖住生活,只有大脑疲劳了,渐渐失去了平衡时,生活才又占上风。好比一条大江,大潮涨来,江水倒灌之后,才又恢复正常水流。在第一个阶段中,艺术家逐渐摸索出自己意识不到的天才所具有的规律和模式。如果他是小说家,他知道,什么情景能向他提供素材;如果他是画家,他知道什么景物能向他提供素材。这素材本身无关紧要,但对他的探索必不可少,正如一间实验室或一间画室之必不可少一般。他清楚地知道,用柔和光线所产生的效果,用对某一过失改变看法而产生的内疚,用站在树下或半潜入水中美如雕像的一些女郎,他造就了自己的杰作。终于会有那么一天,他的大脑已经衰退,面对他的天才使用的材料,他再也无力进行心智活动,只有心智活动才会产生作品。然而他会继续寻找这些材料,为置身这些材料身旁而兴高采烈,因为这些材料在他身上唤起精神上的快乐,精神上的快乐乃是工作的激发剂。他会将这些材料笼罩在迷信的氛围之中,似乎它们高于一切,似乎艺术作品的很大一部分已寓于其中,它们在某种程度上便蕴含着已经现成的艺术作品。与模特儿经常来往、对模特儿宠爱至极,如此而已,他不会走得更远。他会与一些已经幡然悔悟的杀人犯无止无休地聊下去,这些杀人犯的悔恨和堕落昔日曾构成他小说的题材;他会在薄雾使阳光变得轻柔的国度买上一处乡间住所;他会连续几小时地注视女人洗浴;他会收集好看的衣料。生活美好,在某种程度上是毫无意义的词,尚处于艺术境界之下。我见过斯万就停留在这个阶段上。生活美好是一个阶段,由于创作天才速度减慢,由于对促进创造天才的各种形式怀有偶像崇拜,由于希望少下工夫,像埃尔斯蒂尔这样的人,有一天大概就会渐渐蜕化到这样的阶段上去。
\par 他刚才终于给他的花卉画上了最后一笔。我望了望花卉,又浪费了一会工夫。既然我知道那些少女大概再也不会在海滩上了,我望望花卉浪费时间,也就没什么了不起。即使我认为她们还在海滩上,浪费这几分钟就会使我错过与她们见面的机会,我也还是会看的,因为我心中暗想,埃尔斯蒂尔毕竟对他的花卉比对我与这些少女相见更有兴趣。我外祖母的天性与我完全的自私自利截然相反,但她的天性仍在我的天性中有所反映。与我毫不相干的一个人,我对他一直装做很有感情或恭而敬之的人,在我面临着危险,而他只有点小麻烦时,我只会对他的烦恼深表同情,像什么大事一样,而将自己面临的险境视为小事一桩,因为我感到在他看来,这些事大概是这样的比例。如果实事求是地讲,甚至还有过之,我不仅不为自己所处的险境而悲叹,而且还要迎着这风险走上去;而事关别人的危险,则相反,哪怕自己更有可能为危险击中,也要尽量使别人免遭这种危险。这样做原因很多,说起来并不能为我增加光彩。其中一个原因便是,虽然我一味思考时,觉得自己将生命看得很重。但在我生命过程中,每当我为道德上的忧烦,或仅仅是精神上的不安而受到折磨时(有时这些精神不安是那样孩子气,我竟然不敢明说出来),如果突然出现什么意外情形,给我带来生命危险,这种新的思想负担与其他思想负担相比,是那样轻快,以致我会怀着轻松的感觉甚至是欢乐去迎接这种危险。虽然我是世界上最胆怯的人,但我领略过这样对危险的沉醉。在我理智地思考时,看上去这与我的本性是那样格格不入,那样不可想象。即使在一个完全平静而幸福的阶段,当出了某种危险而且是生命危险时,例如我与另一个人在一起,我仍然不会不将他人置于安全的地位,而为自己选择危险的位置。相当数量的实践体验叫我明白了,我一直会这样做而且会高高兴兴这样去做时,我发现,与我一向自认为和肯定的相反,原来我对别人的看法是非常在乎的,这真叫我感到羞愧。
\par 这种不可告人的自尊,却与虚荣、狂妄毫无关系。因为能使虚荣心与狂妄得到满足的东西,一点也不会使我感到快乐,而且我一直是力戒虚荣、力戒狂妄的。在有的人面前,我做到了完全隐藏起自己小小的长处,一旦他们知道这些小小的长处,对我的看法就会不那么平庸。对这些人,我从来无法剥夺自己的快乐,向他们表明,我更加热心的是从他们前进的道路上移开死亡的威胁而不是从我自己前进的道路上。由于我的动机是自尊而不是品德高尚,在任何情况下,他们的做法与此相反,我都觉得极其自然,我根本不会因此而责怪他们。如果我自己的动机是出于一种义务感,我大概会感到,在这种情况下,无论是他们,还是我自己,都必须这样做,可能就会因他们不这样做而责怪他们了。相反,我觉得他们保护自己的生命是非常明智的,同时也无法阻止自己将自己的生命置于第二位。炸弹即将爆炸,我自己置身在他人之前。后来我发现了,在这些人当中,有许多人的生命更没有价值。自那时以来,我觉得这样做就尤其荒唐甚至罪恶了。
\par 话又说回来,拜访埃尔斯蒂尔那天,距我意识到这种价值差距时间还很远,何况也谈不上有任何危险,只不过作为很有害的自尊心的前奏信号,要求自己对于本人那样热切向往的快乐,不要显得比对人家尚未完成的水彩画家的作品看得更重而已。这幅画终于完成。一走到外面,我立即发现——这个季节白昼是多么长——天色并非我想象的那么晚。
\par 我们到海堤上去。我以为那些少女可能还会从那里经过,使出了多少诡计,才叫埃尔斯蒂尔待在那个地方啊!我将我们身边高耸入云的悬崖指给他看,不断地要求他与我谈这些少女的事,以便叫他忘记时间,叫他留在那里。我似乎感到往海滩的尽头走,截住这一小帮人的可能性更大。
\par “我想跟您一起稍微再靠近一些,去看看这些悬崖,”我对埃尔斯蒂尔说,因为我发现这些少女中有一个常常往那边去。“一边走,您一边跟我谈谈卡尔克迪伊吧!啊,我多想到卡尔克迪伊去啊!”我又加一句,并没有想到,在《卡尔克迪伊港》这幅画中那么强有力表现出来的崭新特点,说不定更多地是来自画家的视觉,而不是来自这片海滩真有什么特别价值。
\par “自从我看了这幅画以后,这个港口和海啸角,可能就是我最想见识的地方了,而海啸角从这里去,又路途遥远。”
\par “即使卡尔克迪伊不是更近一些,我大概还是会更倾向于建议你去卡尔克迪伊。”埃尔斯蒂尔回答我说,“海啸角当然很精彩,不过归根结底不过是诺曼第或布列塔尼的那种大悬崖罢了,你已经见识过。而卡尔克迪伊,低矮的海滩上岩石遍布,完全是另一回事。在法国,我不曾见过与此相似的景色,更使我忆起佛罗里达的某些景观。又奇,又极其有野趣。它位于克利杜和纳奥姆这两个地方似乎为作者所杜撰。之间,这些海域是多么荒凉,你是知道的,海滩曲线优美动人。这里,海滩曲线平平常常。可是那里,那曲线多么优美,多么柔和,我简直无法对你形容!”
\par 夜幕降临,必须归去了。我送埃尔斯蒂尔回别墅,突然,有如梅非斯托非勒斯骤然在浮士德面前显现,在大街的尽头——有如与我的气质截然相反的气质和几乎野性而又残酷无情的生命力不真实而又魔鬼般地具体化了,而我那多病之躯、病态的敏感以及过度使用的头脑正缺少这样的生命力——出现了精灵的几颗斑点,人们绝不会将这些精灵与其他东西相混淆,出现了少女植虫类群体的几颗孢子。她们装做没有看见我,但是毫无疑问,正在对我进行冷嘲热讽的评头品足。我感觉到她们与我们势必相遇,不可避免,也感到埃尔斯蒂尔就要叫我,便像一个泳者看到浪峰即将袭来那样转过身去。我骤然停步,任凭我那位鼎鼎大名的同伴继续向前,我则留在后头。当时我们正走过一家古玩店前,我朝古董商的橱窗俯下身去,似乎这橱窗突然吸引了我。我装做不在想这些少女,而能够想别的事,颇为得意。而且我已经隐约知道,待埃尔斯蒂尔呼唤我以便将我介绍给她们时,我会露出询问的目光。那目光流露出的不是惊异,而是希望装出的惊异——每个人都是蹩脚的演员,或者说,每个人身边的人都是善于根据外表判断性格的人——我甚至会用手指指着胸脯问:“您是叫我吗?”并且一溜小跑奔过去,乖乖地低着头,脸上冷冷地掩藏起烦躁,因为我正在聚精会神欣赏古老的瓷器而被打断,要被介绍给我并不希望认识的人。
\par 这时,我打量着橱窗,等待着埃尔斯蒂尔呼唤我的名字,恰似等待一颗期待已久而又没有杀伤力的子弹打到我身上的时刻到来。确信一定会把我介绍给这些少女,结果不仅是叫我装出对她们毫不在意的样子,而且要感受到毫不在乎。既然结识她们的快乐已经不可避免,这种快乐反而受到压抑,缩小,反而没有与圣卢谈话,与外祖母一起进晚餐,在附近郊游那么令人愉快了。有些人大概对古迹不大感兴趣,后来由于与这些人关系微妙,我不得不错过一些郊游的机会,我非常遗憾。此外,使我即将得到的快乐大大逊色的,不仅是来得这样突兀,而且是这样前后不相连贯。有些规律与流体静力学规律一样准确,使我们头脑中按固定顺序形成的形象保持着层次。可是,事件突然在眼前出现,便打破了这些规律。
\par 埃尔斯蒂尔就要叫我了。而我在海滩上、在房间里所设想的与这些少女的结识,完全不是通过这样的方式。即将发生的,是另外一件大事,我思想毫无准备。从这件大事中,我既认不出我的向往之情,也辨别不出这向往的目标。我几乎后悔与埃尔斯蒂尔一起出来了。特别是,我本来以为会感受到的快乐,现在反倒因为肯定再没有任何障碍可以剥夺这种快乐,而大大缩小了。我下定决心扭过头去,见埃尔斯蒂尔站在距这些少女几步开外的地方正与她们说再见时,根据弹力定律,这种快乐便又整个恢复了其高大的形象。距他最近的那个少女,大大的脸儿,双眸熠熠生辉,面孔好似一块大蛋糕,上面还给天空留了点位置。她的双眸,即使目不转睛,也给人以动态的感觉,正如狂风怒吼的日子,虽然肉眼看不见空气,却能感觉到它在空中流动的速度。有一瞬间,她的目光与我的目光相遇,好似暴风雨日子里天上那风驰电掣的乌云挨近了一块行进速度不那么快的云朵,与这块云朵擦肩而过,触着了它,又超过了它。但是,它们互不相识,各自远去。我们的目光也是如此。有一瞬间,你对着我,我对着你,但是,谁也不知道自己面前的这个天国对将来来说蕴含着什么承诺,什么威胁。只是在她的目光并没有减缓速度正好从我的目光下经过时,那目光轻轻遮上了一层薄雾,有如明朗的月夜,风儿卷走了月亮,一块云彩将月亮遮住时,有一瞬间,月光便被迷雾遮掩,然后很快又显现出来。埃尔斯蒂尔并没有叫我,就已经离开了这些少女。她们从一条街斜穿过去,埃尔斯蒂尔向我走过来。一切都错过了。
\par 我曾经说过,那天,在我眼中,阿尔贝蒂娜与以前不同,而且我似乎觉得她一次一个样。在那个时刻,我感觉到,一个人外表、肥瘦、身长的某些改变,也可能来自这个人与我们之间某些状况的变化。在这方面,起作用最大的因素是信还是不信(那天晚上,我先是坚信就要与阿尔贝蒂娜结识,后来这种坚信又烟消云散。几秒钟之间,在我眼中,先是将她变得无足轻重,继而又变得宝贵无比。几年以后,先是坚信阿尔贝蒂娜会忠实于我,后来这种坚信又消失,也引来相似的变化)。
\par 当然,在贡布雷,根据不同的时间,根据平分我的最敏感之处的两大方式,我进入哪一种,我早已感受过不在母亲身边那种痛苦会缩小抑或是增大。整个下午,母亲就像红日高照时谁也感觉不到的月光。夜幕一降临,便只有她占据我这颗惶惑不安的心了。那时,就连新近的往事也已经消逝得无影无踪。
\par 但是那一天,当我看到埃尔斯蒂尔没有呼唤我,正在离开那些少女时,我又明白了;一种快乐或一种忧伤,在我们眼中,其程度变化不同,也可以不仅仅源于两种状态的转换,而是由于肉眼看不见的信仰移位。例如这种看不见的信仰可以使我们视死如归,因为这种信仰为死亡撒下了脱离实际的光辉。也是这种信仰使我们对赴一次音乐晚会看得很重。可是,一宣布我们就要上断头台,音乐晚会立刻就失去了魅力,笼罩着晚会的信仰便会突然消失了。这种“信”所起的作用,头脑中某些东西对此真是明明白白,那就是意愿。但是,如果理性、感性继续无视这种作用,那么意愿再明白也没有用。理性和感性认为我们想离开一个情妇,只有我们的意愿知道我们的心还系在她身上。在这种时候,理性和感性是值得信赖的。正是因为信仰将理性和感性弄得模糊不清,所以我们要在这些时候才能恢复信仰。但是,只要这种信仰消散,只要理性和感性得知这个情妇已经一去不复返,这时理性和感性完全失去了针对性,就变得控制不住,小小的快乐便扩大到无限。
\par 爱情的虚无也是信仰的变种。爱情早已存在,正在四处游动,它停在哪一个女子的形象上,无非因为这个女子几乎无法企及而已。从这一时刻起,对这个女子想得并不多,脑海中很难现出她的模样,而考虑更多的是用什么办法能够把她搞到手。一连串的忧思滋长起来,这就足以将我们心中的爱固定在她身上,她成了我们几乎还不熟悉的爱的对象。爱情变得偌大无比,那个真正的女子在其中占的地位多么小,我们并不考虑。如果突然间,就像我看见埃尔斯蒂尔停下脚步与少女们说话那个时刻一样,我们停止焦虑,停止不安,由于我们整个的爱就是她,在我们终于将猎物抓在手里时,可能骤然间那爱就烟消云散了,对于这猎物的价值,我们并未足够地考虑过。
\par 我对阿尔贝蒂娜了解什么呢?在海上映出的一两个身影,肯定不如委罗内兹笔下那些女郎的侧影漂亮。如果我服从某些纯美学的缘由,我本会喜欢那些女郎胜过喜欢阿尔贝蒂娜。然而,我能服从别的缘由吗,既然丢掉焦虑不安以后,我只能重新找到这些无声的身影,除此之外我根本就别无其他?
\par 自从我见了阿尔贝蒂娜,每日就她进行过千百种思考,与我称之的“她”,进行着内心的对话。在这些对话里,我叫她提问题,回答,思考,行动。在我心中,每时每刻,无穷无尽的想象的阿尔贝蒂娜一个接一个地出现。在这一长串里,真正的、在海滩上远远望见的阿尔贝蒂娜,只出现在排首,正如“扮演”某一角色的明星,在长系列演出中,只在首演式上出现一般。这个阿尔贝蒂娜只是一个侧影,一切附加上去的成份,全是我的想当然。在爱情上,我们内心产生出的添枝加叶,远远胜过从所爱的人身上来到我们心中的东西——哪怕从数量上来说,也是如此。最最实际的爱情也是如此。有的人不仅能自我培养情绪,还能靠一点点东西活着——即使已经得到过肉欲满足的人当中也有如此的。
\par 我外祖母从前有一位图画教师,他跟一个身份不明的情妇生了一个女儿。孩子出生以后不久,那母亲就死了。图画教师伤心难过得自己也没再活多久。实际上他并未与她正式居家度日,而且与她发生关系也不多。外祖母和贡布雷的几位太太,在她们的老师面前甚至从不愿意提到这个女人。在他生命的最后几个月中,她们想到要给这小姑娘一生的命运提供一个保证,每人出了一份钱,给她搞了个终身年金。首先是外祖母倡议,她的某些女友则颇为勉强,她们认为:这个小姑娘难道就真的那么叫人感兴趣,她到底是不是那个自认为是她父亲的人所生的呢?对于那个小女孩的母亲那种人,人们一向是拿不准的。最终她们还是下定了决心。小女孩前来致谢。她长得其丑无比,与上了年纪的图画教师一模一样。顿时一切怀疑都烟消云散。小姑娘唯一长得好的是头发。一位太太对带小女孩前来的父亲说:“她的头发长得多好!”我的外祖母觉得,既然那戴罪的母亲已死,图画教师也将不久于人世,对于一向讳莫如深的那段往事提上一句已无关紧要,便加了一句:“这大概是随家里。她母亲是不是头发这么好?”
\par “我不知道,”孩子的父亲天真地回答道,“我见她的时候,她总是戴着帽子。”
\par 该追埃尔斯蒂尔去了。我从一面大镜子里看见了自己。除了没有得到被介绍的机会这大灾大难之外,我又发现自己的领带完全歪了,长头发也从帽子里露了出来,显得很难看。但是,不管怎么说,就是这样,她们也遇到了我和埃尔斯蒂尔在一起,不会将我忘记。这已经运气不错。那天,照我外祖母出的主意,我穿了那件漂亮的背心,又拿着我最漂亮的手杖,我差点换上另一件难看的背心。这又是好运气一桩。我们期望的重大事件从来不会正如我们所预料的那样发生,因为缺少我们以为可以指望的那些有利条件;而我们并不希望的其他重大事件却接踵而至,相辅相成。我们是那样担心最坏的事,最后我们竟会认为,就总体而言,偶然对我们还算是帮忙。
\par “若是结识了她们,我该多高兴!”我走到埃尔斯蒂尔跟前,对他说。
\par “那您为什么躲在十里开外呢?”
\par 这就是他说的话。他之所以这样说,并非因为这表达的是他的思想。如果满足我的愿望便是他的愿望,叫我一声,岂不易如反掌?他之所以这样说,可能是因为他曾经听别人说过这一类的话,让人揪住了错的凡夫俗子是常常这么说的。他之所以这样说,还因为即使是伟人,在某些事情上,与凡夫俗子也是一样的,他们也从与那些人相同的俗套里寻找日常的遁词,就像总到同一家面包铺子里去买每日的面包一样。要么,这样的话在某种程度上应该从反面去理解,既然这些字眼的意义与真实情况相反,这种话便是某种反应所产生的必然结果、反面的图像。
\par “她们挺急的。”
\par 我心想,更主要的原因恐怕是:某个人对她们不大热情,她们阻止他去叫这个人。如果不是这样,他决不会不叫我。就这些女孩,我向他提过那么多问题,他明明看出我对她们有兴趣嘛!
\par “我刚才正与你谈卡尔克迪伊,”我就要在他家门口与他分手时,他对我说道,“我曾经画了一张草图,上面可以清楚看到海滩的轮廓。那张油画不算太糟糕,但已不可相提并论。如果你允许,为纪念咱们的友情,我把那张草图送给你,”他接着加了一句,“拒绝给予你向往之物的人,给你点别的东西。”
\par “如果你有的话,我倒很希望有塞克里本特小姐小幅肖像的照片。可是这个名字是怎么回事呢?”
\par “这是那个模特儿在一部莫名其妙的轻歌剧中扮演的角色的名字。”
\par “先生,我一点也不认识她,这你是知道的,可你的样子似乎事实上与此相反。”
\par 埃尔斯蒂尔沉默不语。
\par “那总不是婚前的斯万太太吧!”我说,突然不幸而言中。这种情况是相当少见的,但却足以给预感理论提供某些根据,如果有意将可以把这种理论归之无效的种种错误忘记的话。
\par 那确是奥黛特·德·克雷西的一幅肖像。她不愿保留这幅画像,原因很多。有的原因十分明显,也还有一些别的原因。画像时间较早,此后,奥黛特训练了自己的线条,将自己的面庞和身段化成了如今的这个造物。年复一年,她的理发师,她的裁缝,她自己,在她坐卧的姿势,怎么谈话,怎么微笑,手怎么放,眼神怎么传递,怎么思考上,都得遵从这个造物的大致轮廓。非得是一个餍足了的情郎堕落下去,才会像斯万那样,在他那令人心醉神迷的妻子ne varietur\footnote{拉丁文:永不改变。}的奥黛特不可胜数的照片中,唯独喜爱自己卧房中那张小照。在那张照片上,人们看到的是一个相当丑陋而瘦削的少妇,戴一顶饰有三色堇花的草帽,头发蓬松,形销骨立。
\par 话又说回来,即使这幅画像并非像斯万心爱的小照那样,是在奥黛特的线条系统化,成为一个威严而又令人着迷的新式人物之前画就,而是在那之后画成,只要有埃尔斯蒂尔的眼光,也就足以将这个类型拆散。极高的温度可以将原子结构打散,根据另一种类型将这些原子按照完全相反的序列组合起来。艺术天才也能这样动作。这个女人强加于自己各部分线条的那种矫饰的和谐,每日出门之前,她要在穿衣镜中严加审视,一定要它坚持下去。改变帽子的倾斜度,头发的光滑度,目光的活泼度,以保证这种和谐持续下去。这种和谐,大画家的目光在一秒钟之内就能将它摧毁,而以女子线条的另一种组合取而代之,以使自己心中的某种女性理想美、绘画理想美得到充分的满足。同样,也常有这样的情况,从某一年龄起,一位伟大研究家的目光到处能找到构成某种关系的必要成份,他只对这种关系有兴趣。就像那些工人和赌徒,他们不会犯难,手上来什么就是什么,对随便什么东西,他们都可以说:行,这就行。卢森堡亲王夫人的一位表妹,是最高傲的一位美人。她从前爱上了一种艺术,这种艺术在那个时代还是新东西。她请一位最伟大的自然主义画家为她画像。艺术家的目光顿时找到了他到处寻找的东西。在画布上,出现的不是贵妇人,而是一个跑腿的女店员,身后是成斜坡而下的紫色宽阔背景,使人想到比加尔广场比加尔广场在巴黎蒙马特区,是妓女群集的地方。。一位伟大艺术家所作的女子肖像,不仅根本不去考虑如何满足这位女子的某些要求——例如有的女人已开始苍老,却要穿上小女孩的服装要人家给她拍照,这小女孩的服装叫她显示出仍然少女般的体型,显得似乎是自己女儿的姐姐甚或是自己女儿的女儿,而她的女儿站在她身旁,倒按照这种场合的需要而“打扮得十分难看”——反而将她极力掩饰的短处突出表现出来,例如发烧一般的脸色,甚至是发青发紫的脸色。正因为这些短处“极有个性”,就更对画家有吸引力。即使不走到上面那一步,有这些也就足够了,足以使趣味不高的观众幻想破灭,并粉碎他的理想。那个女子那样自豪地支持着这种理想的骨架,也正是这种理想以其唯一的、不可制服的形式将她置于人类之外,人类之上。而现在,这个女人遭了贬,离开了她稳坐金銮不可侵犯的原型,就只不过是个平平常常的女人而已,对她的出类拔萃,我们已失去任何信心。对这种典型,一般来说,我们是那样下苦工夫,不仅表现出奥黛特式的美貌,而且表现出其个性、特点,以至站在这幅剥去了奥黛特式美貌、个性、特点的画像前,我们不仅要大叫一声:“比她丑多了!”而且要大叫:“一点也不像!”我们几乎不敢相信这就是她。我们没有认出她来。这个人,我们确实感到在什么地方曾经见过。但是这个人,又不是奥黛特。这个人的面庞、体态、神情,我们都非常熟悉。这一切使我们忆起的,不是奥黛特这个女子,她从来不采取这种姿势,她惯常的姿态绝不会勾画出这样莫名其妙而又具有挑逗性的阿拉伯图案。使我们忆起的,倒是别的女子,所有埃尔斯蒂尔画过的女子。虽然这些女子彼此很不相同,但埃尔斯蒂尔总是喜欢叫她们摆出正面姿势,足弓弯弯,露出裙外,宽大的圆草帽提在手中,草帽遮住膝部高度,与正面望上去的另一圆形——面孔成对称呼应。总而言之,一幅天才的肖像画不仅肢解了一位女子的原型——其卖弄风骚及其利己主义的美的概念所决定的类型,而且如果这幅肖像画很古老,还会不满足于如照片那样使原型穿着过时的服饰而老化。在画像上,标志时间的不仅是女子怎样着装,还有艺术家怎样作画。这种作画方法,也就是埃尔斯蒂尔最早的作画方法,那便是提炼出对奥黛特压力最大的出身问题,因为这幅画不仅像奥黛特那时期的照片一样,把她表现为著名风流女郎中的一位后来人,而且这幅画像成了马奈或惠斯勒绘的许多肖像画的同时代作品。马奈或惠斯勒这些作品所依据的模特儿已经消逝得无影无踪,已经属于为人遗忘之物或历史的陈迹了。
\par 我一面送埃尔斯蒂尔回家,一边在他身旁默默咀嚼着这些想法。刚刚对其模特儿身份的发现,将我引至这些思考之中。这第一个发现又导致第二个发现,那就是对艺术家其人的发现,这更加使我心慌意乱。他为奥黛特·德·克雷西画过肖像。这位奇才,这位智者,这位孤独者,这位谈吐惊人并在任何事情上都出手不凡的哲人,是否有可能就是从前维尔迪兰家收留的那个可笑而又恶习不改的画家呢?我问他是否认识维尔迪兰一家,是否凑巧他们那时给他起了一个绰号叫比施先生\footnote{比施意为母鹿。}。
\par 他回答我说是的,并不觉得难堪,似乎这是他一生中已经相当遥远的一段,似乎预料不到他在我心中会唤起极其失望的情绪。他抬起眼来,从我的面部表情上看到了这种情绪。他的面孔现出不满的表情。这时,我们已经差不多走到了他家门口。换一个理智和情感不这么高尚的人,大概就会简简单单地道一声有些干巴巴的再见,此后便避免再与我见面了。埃尔斯蒂尔对我并没有这样做。作为一个真正的导师——从纯创作观点来说,说不定好为人师是他唯一的缺点,因为一个艺术家,为了在精神生活上完全站在真理一边,应该保持孤独,而不要挥霍自我,哪怕是对一些弟子——在任何情况下,为了对年轻人最有裨益,他总是极力去开掘某一情境中所包含的部分真理,哪怕这真理对他或对别人都是相对的。与其说上几句可能会挽救自己的自尊心的话,他宁愿说几句可以对我有教育意义的话。



\paragraph*{6}

\par “一个人,不管多么明智,”他对我说,“在年轻时的某一阶段,没有说过什么话,甚至过着某种生活,事后回忆起来觉得很不愉快,希望将其抹掉,这样的人恐怕是没有的。但是他不该绝对地为此而悔恨,因为,只有经过所有的可笑、丑恶之现形,他才能有把握在可能范围内变成一个贤哲。这一切可笑、丑恶的现形应该是这最后现形的先导。我知道有些年轻人,是杰出人物的子孙,他们的家庭教师从他们中学时代起便教导他们要精神崇高、道德高尚。可能他们的生活中没有任何要遮掩的地方,凡是他们说过的话,都可以发表,签上自己的名字。但是,这是一些精神贫乏的人,是理论说教者软弱无力的后代,他们的明智是消极的,是不能开花结果的。明智不能接受而来,必须自己去走一段路亲自去发现,任何人不能代替我们去走,不能免了我们这趟差,因为明智是对事物的一种观点。你钦佩的世人,你觉得端庄的仪态,并不是家长或家庭教师安排妥当的。这些东西的先导,是完全与此不同的人生开端,受到周围占统治地位的恶或俗的影响。这些代表着一场战斗,一次凯旋。我们在最初某一阶段是什么模样,那形象已模糊不清,无法辨认,不管怎么说,是不讨人喜欢的。这我明白。但是我们不应该否认这个形象,因为它是我们确实经历的见证,按照生活和思想的规律,我们从生活的共同因素中——如果是一个画家,就还从画室生活、艺术小团体中——提炼出来超越这一切的某些东西。”
\par 这时我们早已走到他家门口。没有结识那些少女,我很失望。但是现在终于有了可在生活中再次找到她们的一线希望。她们已不再像从前那样只从天际闪过,我想再不会望见她们从那里出现了。在她们周围,那将我们隔绝的巨大漩涡已不再漂浮。这大漩涡不过是她们可能永远可望而不可即,永远溜掉而在我心中唤起的欲望的表现而已。这种欲望时时在心中活动,游移不定,迫不及待,惴惴不安。我对她们的渴望,现在可以放下歇一歇了,可与其他许多欲望一起储备起来。一旦知道这些欲望可以实现,我便将实现的时刻推迟下去。
\par 我离开埃尔斯蒂尔,又是独自一人了。这时,骤然间,尽管我很失望,仍在头脑中看到了所有这一切巧合。这些巧合的出现,完全出乎我的意料。埃尔斯蒂尔正好与这些少女关系密切。这些少女,就在当天早上,对我仍是一幅以大海为背景的油画上的人物,现在她们看见了我,看见了我与一位大画家过从甚密。这位画家现在也了解我有与她们结识的愿望,一定会助我一臂之力。所有这一切都在我心中唤起无比的快乐。但是这快乐对我仍藏而不露。有的客人来到,也叫人禀报过了。但是他们要等别的客人离开,没有别人在场时才走出来。于是我们看见了他们,我们可以对他们说“我们就来见你”,并且听他们谈话。这种快乐即属于这样的客人之列。有时,在这快乐走进我们心中的时刻与我们自己可以走进这快乐之中的时刻之间,又过去了许多时刻,我们在这个空隙里又见了那么多人,以致我们担心,这快乐大概不等待我们了。但是,它们很耐心,并不厌烦,一旦所有的人都离去,这快乐立即就出现在我们眼前。有时,是我们自己太疲劳了,以致觉得我们头脑衰竭,已经精神不够,无法将这些回忆、这些印象牢记心中了。而对这些回忆、这些印象来说,我们那个脆弱的自我是唯一可以居住的地方,是唯一的成型方式。我们也许会为此感到遗憾,因为只有在现实的灰尘与神奇的沙土混在一起的日子里,在某个平平常常的变故成了传奇的契机的日子里,生活才有趣味。这时,不可企及的世界的整个岬角突然从梦幻的光照中涌现出来,进入我们的生活。我们则像一觉醒来便见到了我们日夜热切向往的人一样,本来以为只有在梦幻中才会见到他们呢!
\par 后来的几天,时间都被圣卢离去的准备工作占去,我无法继续窥视这些少女。现在,很有可能在我希望的时刻与她们结识,这给我的心灵带来了平静。这种平静尤其可贵。我的朋友对外祖母和我那样殷勤倍加,外祖母很想向他表示一下自己的感激心情。我告诉过外祖母,说圣卢对普鲁东极为钦佩。这倒叫她有了一个主意,便吩咐将她从前购买的这位哲学家的许多亲笔书信送来。这些东西到的那天,正是圣卢动身的前夕,他前来旅馆观看。他贪婪地阅读了这些书信,恭恭敬敬地用手抚摸每一页纸,极力将每一个句子牢记在心。然后他起身告辞,请我外祖母原谅待了这么久。就在这时,他听到外祖母回答他道:
\par “用不着,拿走吧,这是给你的。我吩咐人送到这里来,为的就是要送给你。”
\par 他不禁喜形于色,并不比对一种不以意志为转移的身体状况更能控制自己。他满面通红,好像刚刚受了处罚的一个孩子。他一再道谢,并极力(并未做到)控制激荡全身的喜悦心情。我外祖母见他如此这般控制自己,更为感动。可是圣卢一直担心自己没有表达出应有的感激之情,第二天,他乘坐当地的小火车返回他所在的部队驻地时,还将身子探出车窗外,请求我原谅。实际上,他的驻地并不远。他本来想坐马车去。他晚上还要回来,并不是一去不复返时,常常坐马车。但是这一次,必须将许多行李放进车厢。他觉得坐火车走更简单些。在这件事上,他采纳了站长的意见。他征求站长意见时,那站长说,马车或者小火车,“几乎意义不清。”可他以为这句话的意思是“几乎相当”(总而言之,这与弗朗索瓦丝说“这差不多是一回事”所表达的意思差不多)。
\par “好吧,”圣卢作出结论说,“我就坐这九曲十八弯的小铁路火车走吧!”
\par 我若不是病魔缠身,也会坐上小火车,一直把我的朋友送到东锡埃尔的。我们待在巴尔贝克车站的时间里——小火车的司机不紧不慢地等一些姗姗来迟的朋友,他们不来,他是不想开车的,同时他也不紧不慢地喝着清凉饮料——我答应每周至少去看他数次。布洛克也到车站来送行——圣卢很讨厌他。圣卢见他听见自己要我到东锡埃尔去吃午饭,吃晚饭,去住,最后也对他说:
\par “如果你哪天下午凑巧路过东锡埃尔,我又有空,你可到司令部来找我。不过,要说有空嘛,我几乎从来就没空。”口气极为冷淡,使命是纠正发出邀请时那迫不得已的热情,防止布洛克对邀请认真对待。可能罗贝也担心,如果我一个人,我不会去。他以为我与布洛克的交情要胜过我自己之所言,这样就叫我能有一个同路的伙伴,一个带动人。
\par 我真怕这种口气、这种一面邀请一面又劝人家不要来的邀请方式会使布洛克不快,觉得圣卢干脆什么都不说也许还更好些。可是我错了。火车开走以后,我和布洛克一起离开车站,一直走到我们必须分手的两条大街交叉处。一条大街通旅馆,另一条通向布洛克家别墅。整个这段路上,布洛克一直不停地问我,我们哪一天到东锡埃尔去,因为“圣卢对我那么好”,如果不应邀前去,他未免“太感情粗糙”。我很高兴,他竟然没有发现,那邀请是用怎样毫不迫切、勉强算得上彬彬有礼的口气发出的。或许他还没有不高兴到那种程度,还愿意装没有发现。不过我还是为他着想,希望他不要立刻去东锡埃尔,以免成为笑柄。但我又不敢向他表明圣卢远不如他那样迫不及待,也不敢给他出个主意。那主意只会使他不快。他真是太迫不及待了。虽然他这类缺点完全可以由一些杰出的优点来补救,换上更内向的别人,是不会有这些缺点的。但他这样的冒昧,确实叫人恼火。照他说,我们这个星期之内非去东锡埃尔不可(他说“我们”,我想,他有点指望我去,好给他去当借口)。整整这一路,走到绿树掩映的体育场前,走到网球场前,走到市政府前,走到卖海鲜的小贩前,他都停下来,求我定一个日子。我不干。他离开我时,生气了,对我说:“请便吧,先生。不管怎么样,我不得不去,既然他请了我。”
\par 圣卢特别担心对我外祖母感谢得不够。第三天我收到他一封信。在这封信里,他再次委托我向外祖母致谢。这封信是从他驻防的城市寄来的,在信封上邮局盖上了邮戳,上有那个城市的名称。这封信似乎向我飞奔过来,对我说,在路易十六骑兵团军营的四堵墙内,他思念着我。信纸上印着马桑特的家徽,我从上面分辨出一头雄狮高踞于一花环之上,花环下方由一顶法兰西元老帽构成圆形。
\par “旅途顺利,”他在信中告诉我,“一路阅读在车站上购买的一本书。这本书的作者叫阿费德·巴丽纳\footnote{阿费德·巴丽纳是露意丝·塞西尔·万桑(1840—1908)的笔名,她是《辩论报》的撰稿人,著有研究贝尔纳丁·德·圣彼埃尔、缪塞的书籍,也是向法国读者介绍易卜生、斯宾塞和托尔斯泰的人。}(我估计这位作者是俄国人,一个外国人能写得这么好,我觉得真了不起。告诉我,你对此书如何评价吧!大概你很熟悉,你是无书不读的渊博学者)。我现在又回到这粗俗的生活中。唉!我觉得在这里自己简直是被流放。我留在巴尔贝克的一切,在这里是没有的。在这种生活中,我找不到任何温馨的回忆,任何智慧的魅力。你一定会蔑视这样的生活环境,不过这种生活也并非没有任何动人之处。自我上次离开这里以来,我好像觉得一切都变了样。因为在这期间,开始了我生命中最重要的一个时代,也就是我们的友谊所开始的时代。我希望这个时代永远不要结束。我只向一个人谈到这个时代,谈到你,这个人就是我的女友。她出我意料地来到我身边,我们一起度过一个小时。她很希望与你结识,我想你们一定会谈得很融洽,因为她也非常爱好文学。相反,为了回忆咱们的交谈,为了重温我永远不会忘记的那些时刻,我倒躲开我的同伴。他们是些很好的小伙子,但是我对他们说这些,他们可能无法理解。对于与你一起度过的那些时光,第一天,我几乎更喜欢自己单独回忆,不给你写信。可是,你思维细致,性情极为敏感,又怕你收不到我的信胡思乱想。你肯俯就这个粗野的骑兵,但是要把他改造得文雅一些,更细腻一些,更与你相称一些,你可要下大工夫。”
\par 这封信,从充满柔情来说,与我自己凭空想象的他给我写的信基本上很相像。我那时尚未结识圣卢。后来,他第一次的接待非常冷淡,使我从幻想中清醒过来,让我面对冰冷的现实。这冰冷的现实倒没有永远那般一成不变。
\par 我收到此信以后,每当午餐时刻信件送到时,哪一封信如果是他寄来的,我立即会认出来,因为这信总具有一个人不在时所显示出来的第二张面孔。从这张面孔的线条上(笔迹的特点),我们没有任何理由认为我们抓不住一个人的心灵,正像我们从鼻子的线条或声音的抑扬顿挫上能抓住人的内心一样。
\par 现在,撤掉餐桌上的杯盘碗盏时,我心甘情愿地坐在桌旁了。如果不是那群少女可能经过的时刻,我也不只是朝大海那边凝望了。依然斜放着的刀叉那中断了的动作,凌乱的餐巾那鼓起的圆形,阳光又在上面增添了一块黄色的丝绒,半空的酒杯更加显示出其形状上那美妙的下小上阔,在半透明玻璃而又似乎凝聚着目光的杯底,残酒颜色很深却熠熠生辉;移动容器,光照引起液体饮料的嬗变;在已经半空的高脚水果盘里,李子从绿到蓝,从蓝又变成金色;用旧的椅子移来移去,每天两次来到桌布四周落座;桌布铺放停当,好比在祭坛上铺放停当,在这里举行美食庆典一般。桌布上,牡蛎壳底还残留着水晶般清澈的几滴汁,如同石雕的小小圣水缸中的几滴水。自从在埃尔斯蒂尔绘的水彩画上看见了一些这样的东西之后,我极力在现实中重新找到这些东西。我喜欢这些东西,正如我喜欢具有诗情画意的某些东西一样。在我从未设想过有美的地方,从最常用的物件中,从“静物”的深沉生命中,我极力寻找美。
\par 圣卢走了几天之后,我终于促成埃尔斯蒂尔举办一次小小的招待会。招待会上,我将会遇到阿尔贝蒂娜。我走出大旅社时,人们感到我魅力无穷,风度翩翩。这完全是暂时性的(而且由于经过长时间的休息和精心的打扮),未能将这魅力与风度保留下来(也未能将埃尔斯蒂尔的信任保留下来)去征服某一更有意义的他人,我深以为憾。花费那么多心血,就是为了得到与阿尔贝蒂娜相识的快乐,我也深以为憾。自从这一快乐有了保证以后,我的理智就认为这一快乐并不珍贵了。但是在我内心,意愿无时无刻不在分享这一幻觉。意愿是我们不断变幻、接踵而至的个性坚韧不拔、永恒不变的奴仆,他躲在暗处,受人蔑视,不倦地忠诚,不顾我们的自我千变万化,不断地为使我们永不缺少必需之物而辛劳。一次向往已久的旅行即将变为现实的时候,理智和感性开始自忖这次旅行是否确实值得一去。意愿知道,如果这趟旅行无法成行,这些无所事事的主人立刻又会觉得这次旅行一定妙不可言,便任凭这二位主人在车站前无止无休地说下去,更加踌躇不决。但是,他负责买票,并按开车时间将我们安顿在车厢里。正如理智和感性变化无常一样,意愿则是永恒不变的。但是,由于它默默无言,并不道出自己的缘由,看上去它似乎不存在。我们自我的其他部分清清楚楚地辨别出自己没有把握的时候,却不知不觉地遵循着意愿坚定的决心。当我从大穿衣镜中望着毫无用处、不堪一击的各种装饰物时,我的感性和理智便展开了一场辩论,辩论的是结识阿尔贝蒂娜的快乐究竟有什么价值,说不定感性和理智希望将这些东西完好无损地保留起来,为另一场合所用。但是我的意愿不允许应该出门的时刻过去,它将埃尔斯蒂尔的地址交给了车夫。既然抽签已经完毕,我的理智和感性便有了闲工夫感到这很遗憾。如果我的意愿给的是另一个地址,我的理智和感性很可能就上当受骗了。
\par 过了一会,我到了埃尔斯蒂尔家。最初我以为西莫内小姐不在画室内。确实有一位少女坐在那里,身穿丝绸长裙,头上没戴帽子。但是,她那秀发,那鼻子,那面色,我都不认识。我从一个漫步海滩、头戴马球帽的骑自行车少女身上归纳出的那个实体,在这些地方我没有找到。可是,她确是阿尔贝蒂娜。甚至得悉了这一点之后,我也没有多留意她。一个年轻人,走进一处社交聚会时,这个人的自我就已经死亡,他变成了迥然不同的另一个人。整个沙龙是一个新天地,在这个新天地中,人们受着另外一种精神环境规律的制约,将注意力完全集中在跳舞、牌局上以及一些人上,似乎这些人和事对我们永远至关重要,实际上,到了第二天便忘个一干二净。为了向与阿尔贝蒂娜交谈几句这个目的地走去,我不得不走一条根本不是由我开辟出来的路线。这条路首先停在埃尔斯蒂尔面前,然后又经过其他好几群客人。有人向这些客人报出我的名字。此后这条路沿着冷餐台延伸,在那里,有人给我送上草莓饼。我将草莓饼吃掉,一面一动不动地听着开始演奏的一首乐曲。对这个阶段,恰巧我都赋予将我介绍给西莫内小姐同样的重要性。将我介绍给她,无非是这各个阶段中的一段。在那之前几分钟,我已经完全忘记了这是我前来的唯一目的。再说,在实际生活中,我们真正的幸福时刻以及我们遇到大灾大难的时刻,不也是如此吗?在许多他人中间,从我们心爱的人口中,得到了我们等待了一年之久的肯定答复或者要命的答复。但是必须继续与人聊天,各种念头相继涌来,形成了一个表面。灾难已降临到我们头上,这个深而狭的记忆,只能不时地在这个表层之下无声地显露出来。如果不是不幸,而是大幸,则可能只有过了数年之后,我们才忆起,我们感情生活中最重大的事件原来发生在一次社交聚会中,我们就是怀着对这件大事的期待去参加那次社交聚会的。而当时我们根本没有时间对这件事给予长时间的注意,几乎没有时间意识到其重要意义。
\par 埃尔斯蒂尔要我过去,以便将我介绍给坐在稍远些的阿尔贝蒂娜的时候,我先将一个咖啡奶油小糕点吃完,然后很有兴味地请我刚刚认识的一位长者详细给我谈谈某些诺曼第地区集市的情况。这位老先生对我扣眼上的那朵玫瑰花十分欣赏,我想可以把这朵花赠送给他。这并不是说,接踵而来的介绍没有引起我任何快乐,在我眼中此事并不具有什么重要性。要说快乐嘛,自然我只在稍晚些时候才体会到,是我回到旅馆,一人独处,又变成了我本人之后。有些快乐与拍照相似。心爱的人在场时,拿到的只是一张底片,然后回到自己家中,可以使用内部暗室时,才将这底片冲印出来。只要待客,暗房的入口便“关闭”着。
\par 我的快乐体验虽然这样推迟了几个小时,这次介绍的重要性,我倒是立刻就感觉到了。介绍时,尽管我们感到自己忽然得到赏赐,握着了一张“券”,适用于今后的快乐。我们朝思暮想希望得到这张“券”,已经好几个星期。我们也清清楚楚地明白,对我们来说,得到这张“券”不仅仅结束了艰苦的寻找——这只能使我们充满欢乐——而且也结束了某一个人的存在。这个人,我们的想象将他歪曲了,我们惴惴不安,担心他永远不会认识我们,又使他变得格外高大。我们的名字在介绍人口中响亮道出的时候,特别是如果介绍人又像埃尔斯蒂尔那样把我们的名字夹在赞扬之辞之中的时候——这个行圣事的时刻,与鬼怪故事中妖精一声“变”,一个人骤然变成另一个人那个时刻很相似——我们热切希望接近的那个女子骤然消失了:首先,她怎么能仍然如同从前她本人一样,既然——由于陌生女子不得不重视我们的名字,不得不注意我们这个人——在昨日还位于无限远的双眸中(我们以为,我们自己那游移不定、目光分散、伤心失望、漫不经心的双目永远也不会与她相对而视),我们原来寻找的有意识的目光,无法辨认的思绪,顷刻间就被我们自己的形象所神奇而又十分简单地代替了。那形象就好比绘在笑容可掬的一面镜子深处。如果我们本人化成了与我们最不相像的人,这种转化也会极大地改变人家刚把我们介绍给他的那个人,他的形状就更相当模糊。我们可以自忖,他到底是神像、桌子还是脸盆。此处影射拉封丹寓言卷九第六个寓言《雕刻家和朱比特的像》:“一块大理石是这样的漂亮,一个雕刻家去把它买下。他说:‘我的刀要把它刻成什么呢?是刻成神像、桌子还是脸盆?’”但是,陌生女郎就要开口对我们说的几句话,就和那些五分钟之内在我们眼前就能塑成一座胸像的蜡像家一样灵巧。这几句话使这个形状明确了起来,而且赋予这个形状某种决定性的因素,会将前一天我们的欲望和想象力发挥出来作出的全部假设一扫而光。无疑,即使来参加这个招待会之前,阿尔贝蒂娜对我来说已不再完全是那个值得扰乱我们生活的唯一幽灵。我们一无所知、勉强看清模样的一个过路女郎,一直是幽灵。她与邦当太太是亲戚,这已经限制了那些美丽的设想,已经堵住了美丽设想能够传播的一条路。随着我越来越接近这个少女,对她了解越来越多,这种了解反倒要以减法计算了,欲望和想象的每一部分,都为一个价值小得多的看法所代替。确实,这看法之上又加上了一种在生活方面,与财团归还最初股份之后之所予完全相同的东西,财团称之为本金已还股。她的姓,她的亲戚,给我的设想加上了第一个边框。我站在她身边,又在她眼下的面颊上看到了那颗小小的美人痣。她那和蔼可亲的样子又是一个界限。最后,我听到她该用“完全”这个副词时却使用“完美”这个副词,真叫我大吃一惊。她是在谈论两个人,对一个人她说:“这个人完美得疯疯癫癫,但待人依然非常热情。”对另一个人,她说:“这位先生完美得平平常常,完美得令人厌倦。”这样使用“完美得”一词令人不快,但是这表明一个人的教养、文化程度。我还真无法想象一个骑自行车的荡妇、玩高尔夫球饮酒纵乐的缪斯能达到这样的水平。此外,这也不妨碍阿尔贝蒂娜经过这第一次变形之后,在我看来又变了好多次。一个人摆在你眼前所显露出来的优缺点,如果我们从另外一个不同的角度走近它,这些优缺点会以完全不同的形式排列起来。正像在一座城市中,从某一条线来看,其名胜古迹分布得很零乱,而从另一观点来看,它们则错落有致,以其各自的宏伟而交相辉映。刚一开始,我觉得阿尔贝蒂娜的神情非但不是桀骜不驯,反而很胆怯。对于我与她谈到的每一个少女,她都加之以“她风度很差”或“她看上去很怪”这样的形容语。由此判断,我似乎觉得她很像样而不是毫无教养。最后,她面孔上的瞄准点是有一侧太阳穴相当火红,看上去很不舒服。她那奇异的眼神也令人不舒服,直到现在我还一直忘不了这眼神。但这还只是第二眼,肯定还有其他的地方,我会渐渐地走过去。正是这样,并非不经过摸索,只有辨认出了刚开始时观察的错误,才能达到对一个人的正确认识,如果这种认识是可能的话。但是,认识是不可能的。因为当我们对这个人的视角不断校正时,他本人并不是一个静止不动的目标,他自己又变了。我们以为能追上他,但他又移动了位置。我们以为终于将他看清楚了,但是我们捕捉到的仅仅是从前的影像。我们终于将这些影像搞清楚了。但是这时,这些影像已经再也不代表他了。
\par 然而,朝着依稀望见的事物走去,朝着有工夫想象出来的事物走去,这个过程,不管会带来怎样不可避免的失望,对于感官来说,都是唯一健康、有益的过程,能吊住人的胃口。有的人,出于怠惰或腼腆,坐了马车直接到他们认识的朋友家里去。到达之前,也从来不敢在路上看见自己向往的东西就停一停。这些人的生活该是多么单调乏味啊!
\par 我回到住处,一面想着这次招待会,眼前又浮现出我乖乖跟随埃尔斯蒂尔到阿尔贝蒂娜身边之前吃完的那块咖啡奶油小糕点,浮现出我送给那位老先生的那朵玫瑰花。所有这一切,我们不知不觉而由情景选择下来的细节,对我们来说,经过精心而又偶然的安排,构成了首次相逢的画幅。但是,这幅画,我似乎是从另一个角度去看的,是在距我自己很远的地方。我明白了,这幅画不仅仅对我来说是存在的。几个月以后,我与阿尔贝蒂娜谈起我认识她的第一天时,使我大为惊异的是,她也跟我提起奶油小糕点,我送人的花。我认为的一切,当然我不能说这只对我有重要意义,但是,这只是我自己的感受。现在我在阿尔贝蒂娜的思想中也见到了,转化成了另一种说法,我根本想不到这会存在的。
\par 从这第一天起,我一面走回住处,一面便得以看到我刚才转述的那种回忆,这时我明白了,完全是变了一个什么魔术,叫我与一个人谈了一会。魔术师技艺高超,这个人竟然与我在海滨跟踪了那么久的那个少女毫无共同之处,而那个人被这个人所取代了。何况我本来可以事先预料到这一点,因为海滨少女本是我自己杜撰出来的。虽然如此,因为我在与埃尔斯蒂尔的交谈中,已将那个少女与阿尔贝蒂娜认同,我便感到对阿尔贝蒂娜负有一种道德义务,要实践自己向想象中的阿尔贝蒂娜许下的爱情诺言。由别人代理订了婚,就自以为此后必须娶这个插进来的人为妻不可了。此外,一回忆起那得体的风度,“完美地平平常常”的说法以及那火红的太阳穴,就足以平息我的忧虑。这种忧虑至少暂时从我生活中消失了。回忆这些还在我心中唤起另一种欲望。这种欲望虽然很甜美,丝毫不痛苦,与对兄弟姊妹的情感相似,但是时间长了,也会变得危险,叫我随时随地感到需要将这个新认识的人拥在怀中。她那得体的举止,腼腆的表情,出人意料的随和,使我想象力那毫无用处的驰骋停止下来,又产生了动情的感激。然后,由于记忆立即开始取出相互独立的一张张底片,在记忆展现的底片系列中,将底片上显现的各个场景之间的任何关联,任何进展全取消了,最后一张底片不一定就能毁掉前面的各张。面对着我与之交谈过的那个平平常常、令人动情的阿尔贝蒂娜,我又看见大海对面那个神秘的阿尔贝蒂娜。到此刻,全是一些回忆,也就是一些画面,在我看来,此一幅并不比彼一幅更真实。
\par 为了再也不想这介绍相识的第一个晚上,我又极力想再看看眼睛下面、面颊上的那颗小小的美人痣。我想起阿尔贝蒂娜离开埃尔斯蒂尔家的时候,我看见这颗痣是在下巴颏上。总而言之,我看见她时,我注意到她有一颗美人痣,但是我那游移不定的记忆随后又带着这颗痣在阿尔贝蒂娜的面庞上漫游,一会儿放在这儿,一会儿放在那儿。
\par 我感到与我认识的所有少女相比,西莫内小姐与她们几乎无甚差异,颇为失望。但是,正像我对巴尔贝克大教堂深感失望并不妨碍我想去甘贝莱、阿方桥和威尼斯一样,我心中暗想,虽然阿尔贝蒂娜本人并非我所希望的那样,至少可以通过她认识她那一小帮朋友。
\par 开始时,我以为在这件事上我又要遭受挫折。她大概还要在巴尔贝克待很久,我也一样,所以我认为最好不要太千方百计地去见她,而等待时机来临,叫我与她相遇。结果我每天都遇到她,她每次只是满足于老远地回我一个招呼。这真叫人担心:如此下去,这整个夏季里,我每天反复跟她打招呼,却可能事态毫无进展。
\par 过了不久,一天早晨,一场雨过后,天气很凉。海堤上,一个少女向我走来。她戴着一顶无边帽、一副套袖,与我在埃尔斯蒂尔家的聚会上见过的那个少女那样截然不同,以致头脑怎么也转不过弯来,会从她身上认出这二者是同一个人。经过一秒钟的惊异,我的脑子总算转过来了。我想,那一秒钟的惊异,并没有逃过阿尔贝蒂娜的眼睛。另一方面,此时此刻我回忆起曾给我留下深刻印象的“得体举止”,此刻她粗暴的口气和“小帮子”的举止又令我朝相反方向大吃一惊。此外,太阳穴不再成为面孔上的视力中心。也许是因为我处在另一边,也可能是无边帽遮住了太阳穴,也可能是那太阳穴并不总是发炎。
\par “这是什么天啊!”她对我说,“总而言之,说巴尔贝克夏季无尽头,纯粹是胡说八道!怎么,你在这什么也不干哪!从来也没见过你打高尔夫球,去游艺场参加舞会。你也不骑马。你该多烦闷啊!你不觉得一天到晚待在海滩上,人都变傻了吗?啊!你喜欢当蜥蜴\footnote{指晒太阳。}?你倒是有时间。我看出来,你跟我不一样,我对各种运动都酷爱!拉索尼赛马,你没去吧?我们坐火车去的。我明白,坐这样的破车,你不会觉得好玩!我们路上花了两个小时!有那工夫,骑我的破车,已经打上三个来回了!”
\par 因为这铁路弯弯曲曲,圣卢将这条地方性的小铁路自然而然地称之为“九曲十八弯”,我对他已经十分佩服。现在阿尔贝蒂娜轻而易举地说什么“破车”,又叫我吓了一跳。我感觉到她在指称方式上运用自如,我真怕她发现我在这方面是个庸才,并且因此看不起我的无能。不过,到那时为止,那一小帮子用来指这条铁路所用的丰富同义词,尚未在我面前显露出来呢!
\par 阿尔贝蒂娜说话时,头部保持不动,鼻翼紧缩,只活动双唇。结果是带着拖腔,鼻音很重。这种声调的组成部分里,可能有外省遗传,年轻人故意模仿英国人的冷漠和外国女教师上课,以及鼻黏膜充血性肥大等各种因素。这种腔调,待她对人了解更深,自然而然又变得孩子气时,很快就消退了。这声调本来可以叫人觉得很不舒服,可是,又别有风味,令我着迷。每当一连数日与她没有见面时,我就心浮气躁起来,一面还用她说这话时那种鼻音很重的腔调,人站得笔直,头部一动不动,自己反复说:“从来没见过你玩高尔夫球。”这时我便认为没有什么人比她更合我的心意了。
\par 人们一对一对,聚拢,停步,以此装点海堤,交谈几句马上又散开,每人沿自己散步的路线走去。那天早晨,我们也构成了这样的一对。我利用静止不动的时刻仔细观看,终于确切知道了那颗美人痣位于何处。凡德伊的《奏鸣曲》中有一段乐谱令我陶醉,但在我的记忆中,这段乐谱从行板到乐曲游荡不定,直到有一天,我手中握着乐谱,我才找到了这个段落,并在我的记忆中将它固定在自己的位置上,原来是在谐谑曲中。与此相同,我一会忆起那颗美人痣在面颊上,一会又记得是在下巴上。现在,这颗痣永远停留在鼻子下方的上唇上了。有些我们倒背如流的诗句,忽然我们在一个剧本里碰到,太出我们意外了。以上情形也是如此。
\par 这时,阿尔贝蒂娜的女友们显露出她们这一群的身影,双腿动人,身材苗条,彼此又那样各不相同。这一群身影越来越大,依傍着大海,成平行线朝我们走来,仿佛这些沐浴着阳光和海风,既身披霞光又红光满面的处女展开美丽的队形,构成丰富多彩而又富有装饰美的整体,要以其形状的千变万化,自由自在地在大海面前繁衍滋长。我请求阿尔贝蒂娜允许我陪她走上一会。可惜她只向她们挥了挥手打招呼。
\par “对你的朋友们这样不理不睬,她们会埋怨的。”我对她说,心里希望着我们能和她们一起散步。
\par 这时一个五官端正的小伙子,手里拿着球拍,走到我们跟前。他就是那个玩纸牌时其荒唐行为令法院首席审判官的太太气愤不已的人。他态度冷淡地、无动于衷地向阿尔贝蒂娜问好,显然自以为他那高人一等就表现在这种神情中。
\par “奥克达夫,你从高尔夫球场来吗?”她问道,“一切顺利吗?体力好不好?”
\par “噢,真恶心,我晕晕乎乎的。”他回答。
\par “安德烈也在吗?”
\par “在,她打了七十七。”
\par “噢,这是个记录嘛!”
\par “昨天我打八十二呢!\footnote{此段话暴露出作者对高尔夫球游戏的规则知之甚少。}”
\par 此人是一位工业巨富的儿子,据说其父在下届万国博览会\footnote{如果我们肯定普氏此次巴尔贝克之行是在1898年,“下届万国博览会”便是1900年那一届。}的组织工作中要扮演相当重要的角色。这个小伙子以及这些少女十分罕见的几位男性朋友,对于一切有关服装、着装、雪茄、英国饮料、马匹的事所掌握的知识真是极善其详,无所不知,令人骄傲,已达到学者那默默无言的谦虚程度。但是这些知识单独扩展,并未伴随着哪怕一丝一毫精神文化修养,实在叫我吃惊。他对于无尾常礼服或睡衣怎样适宜,丝毫无需犹豫,而想不起在什么情况下是否可以使用某一个词,甚至对于最简单的法语规则也搞不清楚。两种文化如此不调和,在他父亲身上大概也是如此。他的父亲是巴尔贝克房地产主联合会主席,在致选民的一封公开信中,竟有这样的词句:“我本想见见市长与他聊聊这个问题。他不肯听取我的正确的不满。”他不久前吩咐在每一面墙上都贴上这封信。
\par 奥克达夫在游乐场中,在波斯顿牌戏、探戈等各种比赛中都经常得奖。如果他愿意,这会使他在“洗海水浴”这个阶层中结成一门好亲事。在这个阶层中,说少女嫁给她们的“舞伴”,那是本义,而不是引申意义。\footnote{在法文中,“嫁”(épouser)这个词用在引伸意义上是“配合默契”的意思,所以“嫁给她们的舞伴”也可理解为“与她们的舞伴配合默契”。这里说的是真正嫁给某人,所以说“是本来意义”而不是“引伸意义”。}
\par 他一面对阿尔贝蒂娜说“对不起”,一面点燃一支雪茄,那样子似乎是请求对方允许自己一面聊天一面结束一件要紧的工作。因为他从来无法“待在那儿什么事都不干”,虽然他实际上从来什么事都不干。完全无所事事,到最后与辛劳过度会产生同样的效果,无论是在精神上还是在身体和筋骨上,都是如此。奥克达夫那沉思默想的前额遮掩着他从来不动脑筋的事实,尽管神情安详,最后还是使他毫无效益地渴望思考。这种渴望使他深夜难以成眠,正如一位劳累过度的玄学家也会难以入睡一样。
\par 我以为,如果我认识这些少女的朋友,就会有更多的机会见到她们,于是立刻准备要求将我介绍给奥克达夫。奥克达夫嘟哝着“我晕晕乎乎的”走了。他一走,我便对阿尔贝蒂娜谈了上述想法。我希望这样她会牢记在心,下次就会这样做。
\par “可是,”她大叫起来,“我不能将你介绍给一个小白脸!这地方,这种人多得很!他们无法跟你谈话。这一位玩高尔夫球很棒,如此而已。我很清楚,他压根不是你这种人。”
\par “你这样抛下你的女友们,她们该埋怨了。”我对她说,心中希望她会向我提议与她一起去追她们。
\par “不会的,她们根本不需要我。”
\par 我们与布洛克走了个头碰头,他对我机智地意味深长地笑笑。见到阿尔贝蒂娜,他又有些难堪。他不认识阿尔贝蒂娜,或者至少是只闻其名而“未见其人”,他作了一个僵硬的叫人讨厌的动作,将头朝衣领方向低了下去。
\par “这个怪物叫什么名字?”阿尔贝蒂娜问我道,“我不知道为什么他跟我打招呼,既然他并不认识我。所以我没还礼。”
\par 我来不及回答阿尔贝蒂娜的话,布洛克已经直冲我们走过来了。
\par “请你原谅我打断你的话,”他说,“我想告诉你,明天我到东锡埃尔去。我不能再等,再等就不礼貌了,圣卢昂布雷对我不知已经怎么想了呢!我通知你,我坐两点钟的火车去。请你安排。”
\par 我这时一心想着再与阿尔贝蒂娜见面并设法结识她的那些女友。东锡埃尔,她们并不去;我去了,回去时就要错过了她们到海滩上去的时刻。所以我觉得东锡埃尔简直是世界的尽头。我对布洛克说,我不能去。
\par “那好,我自己去。我要引阿鲁埃老爷\footnote{阿鲁埃为伏尔泰之本姓。但这几行诗作者并非伏尔泰,而是高乃依,为其剧本《波利耶克特》中女主角波莉娜之台词。布洛克在这里暴露出他既“学究气”——因为他称伏尔泰为“阿鲁埃老爷”,又很无知——将高乃依的诗句安到伏尔泰头上。}两句可笑的亚历山大体诗,对圣卢说:
\par “你要知道,我的义务不取决于他的义务。
\par “如果他愿意,他不尽义务好了。但我应尽我的义务。
\par “这样以便引诱他的教权主义。”
\par “我承认他是相当漂亮的小伙子,”阿尔贝蒂娜对我说,“可他真叫我讨厌!”
\par 我从未想过布洛克会是美男子。不过他确实是。他的头有些鼓,鼻子有鹰钩,神情非常高雅,又显出对自己的高雅十分自信的样子,他的面部叫人看上去很舒服。但是他不会讨阿尔贝蒂娜喜欢。说不定这是由于阿尔贝蒂娜的缺点所致,由于这一小帮子人生硬,无动于衷,由于她们对凡是小圈子以外的东西全很粗暴的缘故。后来,我给他们作介绍时,阿尔贝蒂娜对布洛克的厌恶有增无减。布洛克属于某一阶层,在那个阶层里,一方面对上流社会任意诽谤,一方面对一个“双手干干净净”的人应该有的良好举止又表示出充分的尊重,结果在二者之间来了个特别的妥协,既有别于上流社会的举止,又不管怎样,总是显出一种特别可憎的交际客套。人们将他介绍给别人时,他弯腰鞠躬,既带几分怀疑地微微一笑,又带着过分夸大的恭敬。如果对方是一位男子,他总是说:“先生,很荣幸。”那嗓音似在嘲笑自己道出的话语,同时又意识到这嗓音属于一个并非粗野的人。这第一秒钟用在一个他既遵守又加以嘲笑的习惯上(就像他一月一日时说“我祝您一年称心如意”一样),然后他露出机敏而狡猾的神情,并“高声道出很微妙的事情”。这些事情常常饱含真理,但是叫阿尔贝蒂娜“受不了”。那第一天,我对她说他叫布洛克时,她便大叫起来:
\par “我可以打赌,他是个犹太鬼。装出彬彬有礼的德行,正是他们那一套。”
\par 此外,布洛克后来大概又以另外的方式叫阿尔贝蒂娜恼火。正如许多知识分子一样,他不会将简单的事情简简单单地说出来。他为每一事物寻找一个讲究的形容词,然后又大而化之。这叫阿尔贝蒂娜十分讨厌,她不大喜欢别人管她的事,也不喜欢她扭伤了脚,安安静静待着的时候,布洛克说的那句话:
\par “她坐在长椅上,但是作为普遍现象,她不停地同时来往于隐隐约约的高尔夫球和普普通通的网球之间。”这无非是“文学手法”而已。但是阿尔贝蒂娜感到这会在她与一些人的相处中造成困难。她拒绝了那些人的邀请,说她动弹不了。正因如此,这便足以叫她讨厌那个说出这些话的小伙子的面孔和嗓音了。
\par 我与阿尔贝蒂娜分手,相互许下诺言要一起出去游玩一次。我与她谈过了话,但是不知道我的话语落在何处,不知道我的话语起什么作用,仿佛不知道我是否将石头扔进了无底的深渊一样。一般来说,倾听我们话语的对象,用他从话语要旨中提炼出的意义来充实这些话语,而这个意义与我们赋予这些话语的意义又很不相同。这是日常生活不断向我们揭示的一个事实。更甚之,如果就在一个人的身旁,而我们对这个人所受的教育觉得无从想象(如阿尔贝蒂娜所受教育之于我),对他的爱好,读的书,做人原则都不了解,我们就不知道,是否我们的话语会在他身上唤起某种感觉,这与要在动物身上唤起某种感觉更为相似,因为对动物,还是可以叫它们明白某些事情的。因此,设法与阿尔贝蒂娜交往深厚起来,在我看来,似乎是与未知数接触,如果不说是与不可能接触的话。这似乎是与驯马一样艰难,与养蜂或栽种蔷薇一样费劲的事。
\par 几小时以前,我还以为阿尔贝蒂娜以后只会对我的招呼远远应答。刚才我们分手时已经作出了一起出游的计划。我在内心里向自己许下诺言,以后再遇到阿尔贝蒂娜时,我要对她更大胆一些。我要对她说什么,甚至(既然我完全得到她大概很轻佻的印象)要向她要求什么快乐,我全都提前订出了计划。但是人的思想,像花草,像细胞,像化学元素一样,是可以受影响的。如果将思想深入环境之中,那么改变思想的环境,便是情境,一个新的环境。当我再次和阿尔贝蒂娜在一起时,由于她的在场这个事实本身,我便与平时不同了,结果我对她说的话与我事先计议中的话完全不是一回事。然后,我回忆起那发炎的太阳穴,我又自问是否阿尔贝蒂娜会更欣赏另一种殷勤,她会明白那是不图什么的殷勤。总而言之,在她的某些目光,某些微笑面前,我感到尴尬。这些目光、微笑既可以意味着作风轻浮,也可以意味着一个天性活泼但秉性正直的少女的快活。脸上同一个表情,语言上同一表达方式,可以具有不同的含义,我简直就像一个学生面对拉丁文翻译练习的重重困难一样犹豫不决。
\par 那一次,我们几乎立刻就遇到了那个高个子的姑娘。她叫安德烈,就是从首席审判官身上跳过去的那个女孩。阿尔贝蒂娜不得不将我介绍给安德烈。她这位女友双眸极为清澈明亮,仿佛在绿荫遮掩的一套房间里,从一扇敞开的门走进面向阳光和阳光普照的大海那绿莹莹的反光的一间卧房一样。
\par 五位男士走过去,自从我来到巴尔贝克,经常看见他们,非常面熟。我心里经常琢磨他们是什么人。
\par “他们不是很阔的人,”阿尔贝蒂娜现出蔑视的神情冷嘲热讽地对我说,“那个染头发的小老头,戴黄手套,长得还可以,是不是?他很会作怪相,他是巴尔贝克的牙科医生,人很正直。那个胖子,是市长。不是那个小矮胖子。那小矮胖子,你大概见过,他是舞蹈教师。他长得怪难看的,对我们很受不了,因为我们在游艺场太吵闹,不是把椅子弄坏了,就是想不用地毯跳舞什么的,所以他从来不让我们得奖,虽然只有我们会跳舞,牙科医生是个正直的人,我本应该跟他们打个招呼好气死那个舞蹈教师。可是不行,因为还有德·圣克瓦先生和他们在一起,这个圣克瓦先生是董事长,出身于贵族家庭,可是为了金钱,这个家庭和共和党站到一边去了。没有哪一个正直的人和他打招呼。由于内阁的关系,他认识我叔叔。但我家其余的人都不理睬他。那个穿风雨衣的瘦子,是乐队指挥。怎么!你不认识他?他弹琴简直是仙乐。你没去听Cavalleria Rusticana\footnote{意大利文《乡村骑士》。这是意大利作曲家玛斯卡尼的作品。普鲁斯特在此突出阿尔贝蒂娜对意大利歌剧的热衷,以显现其趣味不高,因当时法国的高等人物对意大利歌剧一律嗤之以鼻。}。啊!我觉得那真是尽善尽美!他今晚还举行音乐会,可是我们不能去,因为今晚的音乐会是在市政府大厅举行。和游艺场没关系,但在将基督像摘走了的市政大厅,如果我们要去,安德烈的母亲说不定会气得中风的!你会对我说,我的姨父也在政府中任职嘛!可是,那有什么办法?姨母就是姨母。并不因此我就得喜欢她!她从来只有一个愿望,那就是把我甩了。真正给我当母亲,而且具有双倍功德的,倒是一位女友,因为她与我一点亲戚关系也没有,我就像爱母亲一样爱她。以后我给你看她的照片。”
\par 有一阵,高尔夫球冠军和玩巴卡拉纸牌戏的奥克达夫走过来和我们说话。我以为发现了我们之间有一种关联,因为从谈话中我得知,他与维尔迪兰家沾点亲,而且还相当为他们所喜爱。但是他谈起那大名鼎鼎的星期三时,满怀蔑视地加上一句:维尔迪兰先生根本不知道穿无尾常礼服,他还说:在某些杂耍歌舞剧院碰到他,真叫人难堪。在那种地方,可真不喜欢听到一位身穿平时的上装、系着黑领带、乡村公证人模样的先生大喊大叫地对你说:“你好啊,淘气的孩子!”
\par 后来,奥克达夫离开了我们。过了一小会,我们又碰上了安德烈。散步了一程,她一句话也未对我讲。走到她家那木屋别墅前,她便进去了。我要阿尔贝蒂娜注意,她的女友对我是多么冷淡,并且阿尔贝蒂娜好像很难在我和她的女友们之间建立起亲密的关系与埃尔斯蒂尔为了实现我的期望似乎第一天就撞到了敌意上这两件事联系在一起。正在这时,一些少女经过,这是昂布勒萨克家的各位小姐。我向她们打招呼,阿尔贝蒂娜也向她们问好。这种情形,使我对安德烈的离去更感遗憾。
\par 我想,在与阿尔贝蒂娜的关系上,我的地位会即将得到改善。这几位小姐是德·维尔巴里西斯夫人一位亲戚的女儿,这位亲戚也认识德·卢森堡亲王夫人。德·昂布勒萨克夫妇非常富有,在巴尔贝克有一所小小的别墅,但是他们过着最简朴的生活,丈夫总是穿着同一件上装,妻子总是穿一件深色长裙。夫妻二人见了我外祖母总是恭恭敬敬地问候,但并无所图。女儿们,天生丽质,衣着更为华丽,但那是城市的华丽而不是海滨的华丽。她们身着长裙,头戴很大的草帽,与阿尔贝蒂娜相比,那样子属于另一种人类社会。她们是谁,阿尔贝蒂娜知道得清清楚楚。
\par “啊!你认识昂布勒萨克家的小姑娘?嘿,你还真认识一些很棒的人呢!不过,她们很简朴。”她补充一句,似乎这二者是相当矛盾的。“这些姑娘对人很好,但是家教那么严,不许她们去游艺场。这主要是因为我们,我们太不像样子。这些女孩讨你喜欢吗?天哪,说不准。她们完全是小白鹅。说不定她们有她们的魅力。如果你喜欢小白鹅,你算是如愿以偿了。看上去,她们也会讨人喜欢,既然有一个已经与德·圣卢侯爵订了婚。那妹妹也爱上了这个小伙子,这可叫她够难受的。我呀,她们讲话那嘴唇几乎不动弹的样子就够叫我腻味的了。她们的衣着也真可笑。她们穿着丝绸长裙打高尔夫球。小小的年纪,衣裳穿得比一些很会打扮的上了年纪的妇女还要自命不凡。你看埃尔斯蒂尔太太,人家才是衣着华丽的妇女呢!”
\par 我回答说,我似乎觉得埃尔斯蒂尔太太衣服穿得简朴得多。阿尔贝蒂娜听了,大笑起来。
\par “确实,她衣服穿得很简朴,可是她穿得叫人心里快活。为了达到你认为的简朴,她花好多钱呢!”
\par 埃尔斯蒂尔太太的长裙,在一个对于衣服饰物没有踏实而简朴的审美观的人眼中,不会引起注意。我正缺乏这种审美观。照阿尔贝蒂娜的说法,埃尔斯蒂尔具有这种审美观,而且达到了最高的程度。这我倒没有料到。我也没有料到,充塞他画室的那些华丽而又简朴的东西,都是他向往已久的珍贵文物。他密切注视这些物品屡次出售的情形,了解其整个的历史,直到有一天,他攒到了足够的钱,才终于把这些东西买到手。但是在这些事情上,阿尔贝蒂娜与我一样无知,不能教我学会什么东西。而对衣着打扮,出于爱俏姑娘的本能,也可能出于贫苦姑娘的遗憾心情,更能以无利害关系的观点,更有高雅口味在富人身上去欣赏不能用来打扮自己的东西。她谈起埃尔斯蒂尔的讲究真是头头是道。埃尔斯蒂尔是那么挑剔,以致他觉得所有的女人都打扮得很糟糕。他把比例、细微的差别摆在最重要的地位上,不惜出重金叫人给自己的老婆制作阳伞、帽子、大衣。他教阿尔贝蒂娜学会了欣赏这些东西的迷人之处,而一个没有审美能力的人则不会比我更能注意这些。此外,阿尔贝蒂娜也搞过一点绘画,虽然她自己承认没有任何“天分”。她对埃尔斯蒂尔佩服得五体投地。多亏了埃尔斯蒂尔对她之所言以及给她看的东西,她在欣赏绘画上很是在行,这与她对“Cavalleria Rusticana”的热衷形成强烈对比。这是因为,虽然现在还不大看得出来,实际上她非常聪颖。她谈吐中的愚蠢,并不是她自己愚蠢,而是她那个环境和她的年龄所致。埃尔斯蒂尔对她产生了很好的影响,但不过是局部的。在阿尔贝蒂娜身上,不是所有的智慧形式都达到了同一开发水平。对绘画的欣赏能力几乎赶上了对衣着以及华丽高雅的各种形式的欣赏能力,但是对音乐的欣赏能力则没有跟上,远远落在后面。
\par 阿尔贝蒂娜知道昂布勒萨克一家是什么人毫无用处。正像一个人可做大事不一定就能做小事一样,我向这家的各位小姐施礼之后,并未感到阿尔贝蒂娜就比从前更积极准备叫我与她的女友们相识。


\paragraph*{7}

\par “你对她们很看重,你心地真好。不过,不要注意她们,不值得。对于你这样有身份的人来说,这些小丫头能算得上什么呢?至少安德烈倒是聪颖过人的。她是一个善良的小姑娘,虽然完美地想入非非。其他的几个确实愚蠢到家了。”
\par 离开阿尔贝蒂娜,我骤然感到一阵心酸,因为圣卢向我隐瞒了他订婚的事,而且他竟要干出与自己的情妇并未断绝关系就结婚这样的坏事来。
\par 没过几天,我被介绍给了安德烈。她谈了不少时间,我利用这个机会对她说,我很想第二天再与她见面。但她回答我说不行,因为她母亲身体很坏,她不想让母亲一个人留在家中。两天以后,我去看望埃尔斯蒂尔,他对我说安德烈对我极有好感。我回答他说:“是我从第一天起便对她有好感,我要求第二天再与她见面,可是她不能来。”
\par “对,我知道,她对我说了,”埃尔斯蒂尔对我说,“她为此十分遗憾。但是她先答应了人家到十里以外\footnote{法古里,一里约等于四公里。}的地方去野餐,她必须坐四轮大马车去,无法再取消邀请。”
\par 安德烈太不了解我。这种谎言虽然无关紧要,但是,一个竟然干出这种事的人,我是绝不应该继续与之来往的。干得出来第一次,还会干无数次。你每年去看一个朋友,第一次他未能赴约或者说他伤风感冒了。下一次,你会发现他又感冒了。再与他约会,他又没来,原因总是同一个,而他以为这是根据情况临时想出来的、不同的原因。
\par 安德烈对我说她不得不留在母亲身边的那天早晨之后,又一天早晨,我远远看见阿尔贝蒂娜手上牵着一段丝绳,上面吊着个莫名其妙的物件。这使她与乔托笔下的《偶像崇拜》\footnote{这里指的是乔托《善与恶寓意画》,为帕多瓦斯克洛维尼小教堂(又称竞技场圣母院)中之壁画。此画也称《不忠》,表现一个男人(不忠之人)手擎一女人偶像,偶像已将一根绳子绕在他的脖颈上,使他背离了俯身向着他的上帝。1900年5月,普鲁斯特在威尼斯小住时,曾专门到帕多瓦去欣赏乔托的壁画。}那幅画很相像,这物件叫“小鬼”,早已停止不用。面对手里拿着这个玩艺儿的少女肖像,未来的评论家们对于她手里的这个玩艺儿,可以像面对竞技场圣母院\footnote{斯克洛维尼小教堂建于一古竞技场的原址上,因得此名。}那幅寓意图一样,发表长篇大论。我与阿尔贝蒂娜走上几步。过了一会,她们那位看上去较贫困、表情严峻的女友走过来对阿尔贝蒂娜说:“你好,我是不是打扰你们?”她就是第一天安德烈大步擦过那个老先生头顶时,恶意讽刺“可怜的老帮子,真叫我心里难受”的那个小姑娘。
\par 帽子碍事,她把帽子摘了。她那头发,有如丰富多彩而又叫不上名字来的花草,四处散开,精巧优美地贴在前额上。阿尔贝蒂娜大概见她光着头,而心中有气,一言不发,一字不答,保持冷冰冰的沉默。
\par 虽然如此,那个女孩仍留下未走。阿尔贝蒂娜总叫她与我保持一段距离,她一会设法单独和她在一起,一会又设法跟我一起走,将她甩在后面。为了叫阿尔贝蒂娜将我介绍给这个女孩,我不得不当着那女孩的面向阿尔贝蒂娜这么请求。待阿尔贝蒂娜道出我的名字时,刹那间,我看见那女孩的脸上和碧蓝的双眸中闪过一丝热情、爱恋的笑容。她向我伸过手来,而在她说“可怜的老帮子,真叫我心里难受”那句话时,我觉得她的神情是那样冷酷!她的头发闪着金光,而且不只是她的头发。她那粉红的双颊和碧蓝的眼睛,也像清晨朝霞红遍的天空一样,到处闪着金光。
\par 顿时我浑身发热,心中暗想,这是一个爱恋起来很腼腆的姑娘。阿尔贝蒂娜那么粗暴无礼,她依然留下来,为的是我,是出于对我的爱。她终于能够用那含笑而充满善意的眼神向我供认,她既能对我十分温柔,也能对别人十分凶狠,大概心中十分快活。甚至在我还不认识她的时候,她大概早就在海滩上注意到我,从那时起心中就想着我了。她之所以嘲笑那位老先生,说不定就是为了让我好佩服她;说不定后来那些日子她神情抑郁,就是因为她无法与我结识。傍晚,我从旅馆里经常望见她在海滩上散步,很可能她期望着与我相遇。正如过去整个一小帮人在场使她局促一样,现在,阿尔贝蒂娜一人在场。她也感到局促。尽管阿尔贝蒂娜的态度越来越冷漠,她仍然紧跟我们不放,很显然,她指望着留在最后,与我订个约会,找个她能溜出来的时间,而又不让家里和女友知道,在望弥撒之前或玩高尔夫球之后,与我在一个可靠的地点幽会。由于安德烈与她关系不好而且很讨厌她,要与她见面就难上加难。
\par “对她那可怕的伪善、卑鄙,以及对我干的卑鄙勾当,我忍了很久,”安德烈后来对我说,“为了别人,我全都忍下来了。但是,终于有一次,我忍无可忍了。”于是她给我讲了那个女孩掀起的一场轩然大波,这件事确实可能有损安德烈的形象。
\par 但是,希塞尔眉目传情,期望着阿尔贝蒂娜会让我们聚在一起好对我讲的话,始终无法道出,因为阿尔贝蒂娜固执地置身在我们两人中间,继续越来越简短地回答女友的话,后来干脆根本不回答她的话了。最后希塞尔只好放弃了这个位置。我责备阿尔贝蒂娜为何如此别扭。
\par “教训教训她,要她放谨慎些。她不是坏女孩,可是叫人讨厌。用不着她到处管闲事。又没请她来,她干吗死缠着我们?再过五分钟我就要叫她滚蛋了!再说,她头发那个样子,我很讨厌,看上去很不正经。”
\par 阿尔贝蒂娜与我说话时,我凝望着她的双颊,心里琢磨着:她那脸蛋会多么香甜,多么有滋味!——那天,她的面颊不是鲜艳,而是光滑,连成一片的粉红,稍带紫色,如奶油一般,仿佛某些花瓣上带着一层蜡霜的玫瑰花。正如有人对某一品种的花朵极为热衷一样,我对那双颊产生了狂热。
\par “我从前没注意到她。”我回答她说。
\par “你今天倒对她看得很仔细,人家简直要说,你想给她画像呢!”她对我说。明明我此刻仔细凝望的是她本人,可是这也无法叫她情绪平息下来。“不过,我不认为她会讨你喜欢。她一点不会调情。你大概喜欢会调情的姑娘吧,你!无论如何,她再也没有机会耍黏乎,也没有机会叫人甩开她了,她马上就要回巴黎了。”
\par “你那些别的女友也和她一起走吗?”
\par “不,就她一个人。她和Miss\footnote{英文,英国女家庭教师。},因为她要补考。她得闷头用功了,这可怜的孩子。我向你保证,这可不是什么开心的事。可能会撞上一个好题目。偶然性太大了,我的一个女友就碰到过《叙述一下你目击的交通事故》这样的题。嘿,真是好运气!可是我也认识一个姑娘,她要阐述(而且还是笔试)的题目是:《在阿尔赛斯特和菲兰特\footnote{莫里哀喜剧《愤世嫉俗》中的两个人物。}之间,你更喜欢谁做你的朋友?》我若是碰上这个题目可就傻眼了,首先,什么都不说吧,就不该向女孩提这样的问题。女孩应该和别的女孩关系密切,而不应该认为她们应该找男士做朋友(这句话向我表明,接纳我进那小帮子的可能性很少,真叫我浑身颤抖)。不过,不管怎么说,即使向一些年轻人提出这个问题,人家能找出什么话来说呢?有好几位家长都给《高卢人报》\footnote{该报的思想倾向为反动和保皇。自1882年阿尔图尔·梅耶重任该报社长以来,在使君主主义者归附布朗基主义上起了重要作用。阿尔贝蒂娜的这句话,除了告诉我们邦当家里阅读这份报纸以外,还给我们一个信息,就是她的排犹主义思想从何而来,因为《高卢人报》是坚决反对宣布德雷福斯无罪的。}写了信,抱怨这类题目太难了。更不像话的是,在一本得奖最佳学生作业集中,这个题目竟然做了两次,而做法完全相反。一切取决于考官。有一个考官要求回答说菲兰特是个交际老手,溜须拍马,骗子;而另一个考官则要求回答说,不能不赞美阿尔赛斯特,但是由于他太好寻衅,脾气太坏,要作朋友嘛,最好还是挑菲兰特。连老师之间意见都不统一,你怎么能叫可怜的学生搞清楚呢?这还不算,问题是一年比一年难。希塞尔恐怕非得走后门才能过关了。”
\par 我回到旅馆,外祖母不在,我等她很久。待她回来,我央求她让我出去远游一次,条件很好,时间大概是四十八小时。与外祖母吃了午饭,叫了一辆马车,吩咐将我拉到火车站去。希塞尔在车站看见我,大概不会感到惊讶。待我们在东锡埃尔换上了去巴黎的火车,便有带单独过道的车厢。待Miss打盹时,我就可以将希塞尔带到僻静的角落去,与她订我回巴黎以后的约会,我尽量赶快回巴黎。然后根据她向我表示的意愿,说不定我会一直将她送到冈城或埃夫勒,然后再坐下一趟车回来。可是,如果她知道了我在她和她的女友之间曾经长期犹豫不决,又想钟情于她,又想钟情于阿尔贝蒂娜,又想钟情于那个明眸少女,又想钟情于罗斯蒙德,她会怎么想呢!既然我与希塞尔彼此有情,即将结为同心,我对上述的事一定感到悔恨不已。何况我可以信誓旦旦地向她保证,我已经不喜欢阿尔贝蒂娜了。今天早晨我见她对我扭过头远去,为的就是我与希塞尔说话。在她那赌气垂下的头上,脑后的头发与别处不同,颜色更深。头发闪着光,似乎她刚刚出水。这使我想到一只落汤鸡,这样的头发使我从阿尔贝蒂娜身上看到另一种心灵的体现,与迄今为止那略显紫色的面孔和神秘的眼神完全不同。她脑后闪亮的头发,有一阵,我能从她身上看到的,就是这个,我继续看见的也只有这个。有的商店在橱窗里这次陈列着某一个人的这张照片,下次又陈列出她的另外一张照片。我们的记忆与这些商店十分相似。一般来说,在一段时间内只有最新的照片摆在那里供人观看。
\par 车夫扬鞭催马,我倾听着希塞尔对我道出的细言软语,这完全是从她那嫣然一笑和伸过来的手中衍生出来的。这是因为我在生活中处于还没有钟情于人而希望钟情于人的阶段,我不仅心怀肉体美的理想——诸位已经看到,我从每个过路女子身上远远辨认出这种理想美,但这过路女子距离要相当远,以便她那模糊的线条与这种认同不要发生矛盾——而且心里怀着一个精神幽灵。这幽灵随时准备化身为人,那就是即将钟情于我,即将向我道出爱情喜剧台词的那个女郎。这出爱情喜剧,自童年时代起,在我头脑中已全部写就,我似乎感到所有可爱的少女全都一样愿意扮演这出戏,只要她们外形过得去。在这个戏中,不论我召来创造这个角色或重演这个角色的新“星”是谁,剧本,剧情变化,甚至戏文,都保持着不变的形式。
\par 虽然阿尔贝蒂娜并不热心为我们介绍,过了几天,我还是认识了第一天的那一小帮子人。除了希塞尔之外,她们依然齐集在巴尔贝克(由于在车站道口前马车停留时间很长,加上列车时刻表的变化,我没有赶上火车,我抵达车站时,火车已开走五分钟了。再说,这时我已经不再想着希塞尔了)。此外,我也认识她们的两三位女友,是应我的要求,她们给我介绍的。这样,通过一个少女再认识另一个少女,希望与这个新认识的少女一起得到快乐,于是那刚刚认识的一个,便好似通过另一品种的玫瑰而得到的新品种的玫瑰花了。在这一系列的花朵中一个花冠一个花冠地溯源而去,认识了一朵不同的花得到的快乐,又使我转回到通过哪朵花认识了这朵花的那一朵上去,感激的心情中又夹杂着向往和新的希望。过了不久,我就终日与这些少女相伴了。
\par 可叹!在最鲜艳的花朵上,也可以分辨出无法觉察的小斑点来。今日绽成花朵的器官,经过干燥或结实的过程,会变成籽粒。对于一个老练的人,这无法觉察的数点已经勾画出籽粒那不变的、事先已经注定的形状。人们的目光追随着一艘船,如醉如痴。涟漪以其优美的姿态吹皱清晨的海水,似乎一动不动,可以入画,因为大海是那样平静,根本感觉不到海潮的汹涌。那船只犹似涟漪。在注视人的面孔的一瞬间,人的面孔似乎是不变的,因为这面孔演变的进程太慢,我们觉察不到。但是,只要看看这些少女身旁的母亲或姑母,就能衡量出这些线条在一种通常很可怕的内部引力作用下,在不到三十年的时间内会走过多少距离,直到目光无神、面庞已完全落到了地平线以下再也沐浴不着阳光的时刻。即使在那些自认为完全摆脱了自己种族束缚的人身上,犹太爱国主义或基督返祖遗传都是根深蒂固而且无法避免的。我知道,在阿尔贝蒂娜、罗斯蒙德、安德烈那盛开的玫瑰花之下,与上述思想根深蒂固、无法避免一样,隐匿着粗大的鼻子、隆起的嘴、臃肿的身躯。这个,她们自己也不知不晓,将来是要伺机出现的。那时会叫人大吃一惊,但是实际上已在后台随时准备出人意料、定人生死地登场了,正像什么德雷福斯主义、教权主义、民族和封建英雄主义,一俟时机呼唤,便骤然从先于本人个性的本性中跳出来一样。一个人按照本性思考,生活,演变,强壮起来或死亡,自己都无法从因本性而采取的特殊动机中将这本性分辨出来。甚至在精神上,我们也受到自然规律的制约,其程度远远超过我们的想象。我们的思想,像某种隐花植物,某种禾本科植物一样,事先便拥有某些特点,而我们以为这些特点是选择而来的。我们只抓住次要的观念,而意识不到首要的原因(犹太人种,法兰西家庭,等等)。首要的原因必然产生出次要的观念来,到了希望的时刻我们会将这首要的原因表现出来。有的观念我们觉得似乎是思考的结果,有的似乎是不注意卫生而得来。正像蝶花科植物其形状来源于其种子一样,说不定不论我们赖以生存的观念也好,我们因之死去的疾病也好,全是从我们的家庭传下来的。
\par 就像一株花期成熟时间各异的植物,在这巴尔贝克的海滩上,我从那些老妇人身上,看到了坚硬的籽实,柔软的块茎。我的女友们有一天可能就要成为这般物品。但是这又有什么关系?此刻,正是开花时节。所以,德·维尔巴里西斯夫人邀我出去散步时,我总是寻找借口说我不得空闲。我去拜访埃尔斯蒂尔,也只有我新交的女友伴我同行时才去。我甚至无法找出一个下午按照我对圣卢许下的诺言去东锡埃尔看望他。交际聚会,严肃的谈话,甚至友好的闲聊,如果要占去我与这些少女外出的时间,对我产生的效果,简单得就和到了早餐时间,不是带我们去吃饭,而是去看画册一样。我们以为和他们在一起得到乐趣的男子,青年人,老年或中年妇女,对我们来说,只触及到一个不坚固的表平面,因为我们只通过压缩为这个表平面的视觉感受去认识他们。这种视觉感受朝少女奔去时,则是作为其他感官的代表前去的。其他感官将到她们一个个身上去寻找色、香、味的各种优点,将品尝这各家之长,甚至无需借助于双手和双唇。借助于情欲十分擅长的移植艺术和综合天才,各种感官足以在双颊或酥胸的色彩下还原成手的接触,初次品尝和严禁的接触的感受,会赋予这些女郎甜美而坚固的形态。在玫瑰园采美或在葡萄田里用眼睛吞食着一串串葡萄时,也是如此。
\par 坏天气吓不住阿尔贝蒂娜,人们有时见她在瓢泼大雨下仍然身穿雨衣骑着自行车飞奔。虽然如此,如果下雨,我们则到游艺场去度过白天。那些日子,我不去游艺场简直就不行。我对从来不进游艺场的各位德·昂布勒萨克小姐蔑视到了极点。我心甘情愿地帮助我的各位女友耍弄舞蹈教师。我们一般总是受到老板和攫取了领导权的雇员的申斥,因为我这些女友从衣帽间到礼堂去,无法控制自己的激情,非要从所有的椅子上跳过去不可;回来的时候,又非要一溜坡滑下来不可。她们用美妙的手臂动作保持平衡,一面唱着歌,犹如古老年代里的诗人那样将各种艺术形式糅进这青春年少的时光。对于古老年代里的诗人来说,各种文学体裁尚未分开,他们在一首史诗中可以将农谚和神学训示混杂在一起。我说“我这些女友”,就连安德烈也不例外。正因为如此,我第一天时还以为她是充满激情的女孩呢!实际上与此相反,她瘦弱,聪颖,那一年身体极为不适。即使如此,她仍不顾自己的健康状况,为那个年龄的特点所驱使。在这种年龄,不顾一切,快活时将病人与身强力壮的人混为一谈。
\par 这个安德烈,第一天时我觉得她最为冷淡,实际上她比阿尔贝蒂娜文雅、多情、细腻多了,她对阿尔贝蒂娜表现出大姐姐那种抚慰、温存的疼爱。她来到游艺场,坐在我的身边,与阿尔贝蒂娜相反,她懂得拒绝跳一场华尔兹,甚至在我疲倦时,放弃去游艺场,到旅馆里来看我。她表达对我的友谊,对阿尔贝蒂娜的友谊,都有着细微的差别,证明她对内心情感的体会极为聪慧,令人心情舒畅。这种聪颖可能部分源于她的病体。她总是面带快活的微笑原谅阿尔贝蒂娜的孩子气。快活的事对阿尔贝蒂娜产生的不可抗拒的诱惑,她都天真有力地表现出来,她不会像安德烈那样,坚决拒绝,而宁愿与我谈天……
\par 去高尔夫球场吃茶点的时刻即将来临,如果我们大家都在一起,阿尔贝蒂娜自己作好准备,然后朝安德烈走过来,说:
\par “喂,安德烈,你还等什么,为什么还不走?你知道的,我们要去高尔夫球场吃茶点。”
\par “我不去,我留下来和他聊天。”安德烈指着我,这样回答。
\par “可是,迪里欧太太请了你,你是知道的。”阿尔贝蒂娜大叫起来,似乎安德烈打算与我待在一起,只能用她不知道人家邀请了她这一点来解释。
\par “你看,我的小姑娘,别那么傻。”安德烈回答道。
\par 阿尔贝蒂娜并不坚持,生怕人家也劝她留下来。她摇摇头:
\par “你想怎么着就怎么着吧,”她回答,“对一个喜欢慢性自杀的病人,就是这么说的。我可跑了,我想你的表慢了。”说完拔腿就跑。
\par “她叫人着迷,可她也是一大怪。”安德烈说道,对女友微微一笑。这微笑既抚慰她,又对她作出评断。
\par 在爱好消遣娱乐这一点上,阿尔贝蒂娜与少年时期的希尔贝特有些相似。在我们相继爱恋的各个女子之间,总存在某种相似之处,虽然也有所变化。这种相似,与我们气质的固定化有关系,因为这些女子是我们的气质所选择的,而将所有与我们既不相反,也不相辅的女子,也就是专门既满足我们的官能享受又折磨我们的心的女子全部淘汰掉。这些被选中的女子,是我们气质的产物,是我们感性的倒影、反成像、“底片”。因此,一个小说家,在描写他笔下主人公的生活时,可以将他历次的恋爱描绘成几乎完全相似,而并不给人以自我抄袭的印象。相反,给人的印象是他在创造,因为虚假的革新总不如旨在暗示一个崭新真理的重复更有力量。在堕入情网者的性格中,小说家还应该指出变异的迹象,随着进入人生其他纬度上新的地区,这种变异的迹象更加突出。如果对自己笔下的其他人物,他描绘出不同的性格,而对自己心爱的女子,则没有赋予她任何性格,说不定这位小说家就再次表达出了另一条真理:对于无关紧要的人,我们了解他们的性格。但是对一个与我们的生命合而为一的人,很快我们就再不能将她与我们自己分开的人,对于她的动机,我们不断地作出各种令人不安的假设、对这假设又不断作出修改,对这样一个人,我们怎么能够捕捉住她的性格呢?对于我们爱恋的女子,我们的好奇心是从理智之外升腾起来的,其驰骋大大超越这位女子的性格。即使我们想停留在这个问题上,恐怕也做不到。我们惴惴不安调查研究的目标,要比这些性格上的特点更为紧要。这些性格上的特点与表皮上那些小小的菱形十分相似,其变化丰富的组合构成了肌肉花纹般的特点。我们直觉的辐射穿透了这些,带给我们的影像完全不是一张特殊的脸的影像,而代表着一副骨架那阴沉而痛苦的普遍性。
\par 安德烈非常富有,阿尔贝蒂娜则贫穷而又孤苦无依,因此安德烈怀着极度的慷慨让她分享自己的奢华。说到安德烈对希塞尔的感情,则与我所想的不完全一样。果然不久阿尔贝蒂娜拿出她收到的希塞尔的来信,大家便有了这位女大学生的消息。此信是希塞尔专门写来,要将她旅途和抵达的消息告知这一小帮子人,同时也请大家原谅她的怠惰,尚未给其他人写信。安德烈说:
\par “我明天就给她写信。如果等她先来信,可能要等很久,她那么粗心大意。”
\par 本来我以为她与希塞尔龃龉得要死,听到她道出这番话来,我真是大为惊异。
\par 安德烈朝我转过身来,补充了一句:“显然你大概不觉得她如何出类拔萃,可她是一个非常正直的姑娘,我对她非常有感情。”
\par 我由此得出结论,安德烈与人龃龉时间不长。
\par 除了这些下雨的日子,我们应该骑自行车到悬崖上去或到乡间去的时候,提前一个小时,我就要极力打扮得漂漂亮亮的。如果弗朗索瓦丝没有将我的衣物准备好,我就要叽哩咕噜地埋怨。弗朗索瓦丝受到夸奖,自尊心得到满足的时候,她是谦恭、谦虚而又可爱的。但是,哪怕你挑出她一点点错,即使在巴黎,她也要骄傲而气恼地挺起腰板——年迈已开始使她弯腰驼背了。这自尊心是她生活中最大的发条,她满意和快乐的情绪与要她做的事的难度成正比。她在巴尔贝克要做的,都是那样轻而易举的事,以致她几乎总是现出不快的神情。我要去会我的女友,抱怨我的帽子没有刷,或者我的领带没有整理停当时,她那不快的神情会突然增加一百倍,还要加上冷嘲热讽的表情。本来她能做到千辛万苦而并不因此就觉得自己干了什么了不起的事,可现在,只要指出一件上装不在应在的地方,她就不仅要自吹一通她是怎样精心将这件上装“收藏起来,而不是叫它在外面落灰尘”,而且还要对自己的活计照理夸奖一遍,抱怨她在巴尔贝克可不是度假,在这里就找不着第二个人过她这样的日子。
\par “我真不明白怎么能叫自己的东西这么乱,你去瞧瞧,是不是换个别人,在这乱七八糟之中就能找出个头绪来。就连魔鬼自己恐怕也要晕头转向。”
\par 要么她就摆出女王的面孔,火冒三丈地瞪着我,一言不发。可是一关上房门,进了走廊,她的沉默就立即打破了。于是话语响彻走廊,我猜想那是骂人的话,可是又跟剧中人上场以前在边幕上道出的头几句台词一样,叫人听不清楚。何况我这样穿衣打扮准备与女友们外出,即使什么也不缺,弗朗索瓦丝情绪也很好的话,她也要表现出叫人无法忍受的样子。在我感到有一种需要,要对人谈谈这些少女的时候,我在她面前曾就这些女孩说过一些开玩笑的话。现在,她利用这些笑谈,摆出向我透露什么的样子。其实,如果是真的,我肯定比她知道得更清楚。可她说的根本不是那么回事,因为她根本就没有听明白我的话。像所有的人一样,她有自己的性情。在人身上,这种性情永远不会与一条笔直的道路相似,而是以其莫名其妙而又不可避免的弯弯曲曲令人惊异。别人发现不了这些弯路,我们要从这些弯路走过,很困难。每次我走到“帽子不在原处”,“安德烈或阿尔贝蒂娜的名字”这个点的时候,弗朗索瓦丝就要强迫我走上弯弯曲曲、莫名其妙的小路,使我迟迟动不了身。我吩咐给我准备夹chester\footnote{英文:柴郡乳酪。}和生菜的三明治和买点心时,也是这样。这是准备到了吃茶点的时候,我和这些少女在悬崖上吃的。可是弗朗索瓦丝宣称,她们如果不是这么看重物质利害的话,本可以轮流出钱买嘛!外地的贪婪和庸俗这整个返祖现象倒来救了弗朗索瓦丝。在她看来,简直可以说,死去的欧拉莉那分裂的灵魂在我的女友这一小帮子人那迷人的躯体上找到了比在圣埃罗瓦身上更优美的化身。\footnote{见《贡布雷》,女圣徒欧拉莉在勃艮第变成了圣埃罗瓦。}听到这些谴责,我真是火冒三丈,感到撞到了这种地方,从这里开始,这乡间熟悉的小路竟变成无法通行的死胡同。幸亏时间不太长。这乡间熟悉的小路,便是弗朗索瓦丝的性情。后来,上装找到了,三明治准备好了,我便去找阿尔贝蒂娜、安德烈、罗斯蒙德,有时还有别人。于是,我们动身,步行或骑自行车。
\par 如果是从前,也许我更喜欢天气不好时这样去散心。那时,我极力在巴尔贝克重新找到“西梅利安人的故乡”,风和日丽的天气在那时大概是不存在的,美好的时光便是洗海水浴的人在普普通通的夏天这个为云雾笼罩的古老地区。现在,我从前鄙视的、视野中避开的一切,不仅是阳光的变幻,甚至还有竞渡、赛马,我都狂热地追求了。与我过去只希望看见风暴席卷的大海原因是一样的,这些都与美学观念相关。这是因为,我和女友们有时去拜访埃尔斯蒂尔。少女们在场的时候,他更喜欢拿出来给大家看的,是根据驾驶快艇的俏丽女郎画的几幅速写或取材于巴尔贝克附近一个跑马场的一幅草图。我首先腼腆地向埃尔斯蒂尔承认,说我从前不愿意参加那种地方的集会。
\par “你错了,”他对我说,“是那么美,又那么奇!首先,那个特别人物,骑手,多少人的目光定睛望着他!他穿着鲜艳夺目的绸上衣,在遛马场前,神情抑郁,面色发灰,与他紧紧牵住的旋转跳跃的马化成了一体。分析出他那职业性的动作,显示出他构成的闪闪发光的一个亮点,该是多么有趣!在赛马场上,马衣也形成闪闪发光的一个亮点!在赛马场这个光芒四射的广阔天地上,各种事物都发生了怎样的变化!阴影,反光,这么多,光看见这个,简直叫人惊异!女人在赛马场上可以显得多么美!尤其是首场,真叫人心花怒放!在那种类似荷兰有些湿气的光线里,感觉到海水那刺骨的寒气在阳光里上升,这里还有衣着极为华丽的女子。这样的光线大概来自海滨的湿气。我从来没见过在这样的阳光中,坐马车前来或将望远镜按在眼睛上的女子。啊!我是多么希望将这阳光表现出来呀!我看赛马归来,就像发了疯一样,有那样强烈的工作欲望!”
\par 然后他对游艇盛会发出赞美,比对赛马更有甚之。于是我明白了,盛装女子沐浴在海滨赛马场那海蓝色的阳光之中的竞渡,体育比赛,对一个当代艺术家来说,可以是与委罗内塞或卡帕契奥这样的画家那么喜欢描绘的节日同样有趣的题材。
\par “他们作画的城市,”埃尔斯蒂尔对我说,“这些节日有一部分具有航海性质,所以你的比喻就更准确了。只是那个时代登船的美经常存在于其沉重、复杂之中。有水上比武,和此地一样,一般这是为招待某使节举行的,与卡帕契奥在《女圣徒厄休尔的传说》中所表现的相仿。\footnote{这是九幅油画组成的画卷,第一幅《女圣徒厄休尔来到科隆》完成于1490年,五六年之后完成全部,其中有《外交使节》及《情侣话别》等场景。此画卷自1812年起属于威尼斯美术学院画廊,普氏1900年威尼斯之行时欣赏过这些油画。}船体庞大,造得如同建筑物一般,似乎可以水陆两用,有如威尼斯城中小小的威尼斯城。借助于铺着深红锦缎和波斯地毯的可移动船桥,船只停泊了。就在镶嵌着各色大理石的阳台旁,身着樱桃红织锦或绿色花缎的妇女上船了。阳台上方,别的妇女身着黑袖白衩、缀着珍珠或镶着镂空花边的长袍,探身观望。人们再也不知道陆地在哪里终止,大海从哪里开始,什么是宫殿或船只,小帆船,威尼斯式帆桨大木船和彩船\footnote{古代威尼斯大公在耶稣升天节这天所乘的船只。}了。”
\par 对埃尔斯蒂尔为我们描述的这些服饰细节,这些奢华的形象,阿尔贝蒂娜聚精会神、十分起劲地听着。
\par “啊,我真想看看你说的那镂空花边,威尼斯花边,太漂亮了!”她大叫起来,“我真想去威尼斯!”
\par “说不定你不久就可以欣赏到从前那里人们穿在身上的妙不可言的衣料了,”埃尔斯蒂尔对她说,“现在只能从威尼斯画派画家的画幅上见到这些,或者难得在教堂的珍藏中得以一见,有时甚至会有一种衣料拿出来销售。不过,据说有一位威尼斯艺术家,叫福迪尼\footnote{这个福迪尼全名为玛丽亚·福迪尼·德·玛德拉佐(1871—1949),为西班牙画家玛丽亚·福迪尼之子。普氏在《追忆似水年华》中经常提到他。福迪尼在威尼斯自己的寓所中,数年潜心研究,力图复活威尼斯历史上最美的服饰。前文谈到的卡帕契奥的画《女圣徒厄休尔的传说》亦为他的样本之一。在妻子亨利埃特的帮助下,他设计出不少服装,也创作了一些画,制造出了壁毯,帷幔,首饰等等。普氏对他极为佩服。}的,他找到了织这些衣料的窍门。再过几年,妇女们就可以身着锦缎出来散步,尤其是身着锦缎待在家中了,与威尼斯为其贵族妇女设计的用东方图案装饰的锦缎一样精美华丽。不知道我会不会喜欢这个,对于今日之妇女,这种服装是不是太不符合时代,哪怕是为竞渡招徕看客。咱们那些现代化的游船,可与往昔那‘亚得里亚海的女王’威尼斯的时代完全相反。一艘游艇,游艇的内部陈设,艇上人的衣着打扮,最动人的地方便是其海上物品的简易、朴素,我是多么爱大海!我向你们坦率承认,比起委罗内塞,甚至卡帕契奥时代的服装式样来,我更喜欢如今的式样。咱们那游艇美的地方,就在于一色,简单,明亮,漆成灰色,阴天时,显得蓝莹莹的,奶油一般线条模糊——尤其是在中型游艇里,我不喜欢庞然大物般的游艇,船味十足。这就跟帽子一样,得有个尺寸。人活动的舱室必须像个小咖啡馆模样。一艘游船上妇女的打扮,也是一样。最优美动人的,是轻松、雪白和一色的打扮,帆布,上等细麻布,北京棉布,人字斜纹布,在阳光下,在碧蓝的大海上,变得跟白帆一样雪白耀眼。话又说回来,会穿衣服的妇女很少,可有的人真是妙不可言。在赛马场上,莱娅小姐戴一顶小白帽,打一把小小的白阳伞,真是迷人!为了得到这把小阳伞,多少钱我都愿意出!”
\par 这把小阳伞与其他阳伞究竟有何不同,我多么想知道!阿尔贝蒂娜比我更想知道,但那是出于别的缘由,是女人爱俏。正像弗朗索瓦丝谈到蛋奶酥时说“这是耍魔术”一样,原来那差别就是剪裁不同。
\par “小极了,圆极了,像一把中国阳伞!”埃尔斯蒂尔说。
\par 我提到某些妇女的阳伞,埃尔斯蒂尔都说完全不是那样,他觉得我说的那些阳伞都其丑无比。他是一个鉴赏能力高雅而又挑剔的人。四分之三女性打的阳伞,他都觉得难看得吓死人。这些人的阳伞与叫他着迷的一个小巧玲珑的玩艺儿之间小小不然的差别,他就能将这个说成了不得。在我看来,一切奢华都会使人思想贫乏。他与我相反,大肆鼓吹他那种“极力画出与这一样美的东西”的绘画欲望。
\par “你们看,这个小姑娘已经明白那帽子和阳伞是什么样的了。”埃尔斯蒂尔指着阿尔贝蒂娜对我说。阿尔贝蒂娜的双眼闪烁着觊觎的光芒。
\par “我多么希望发财,好买一艘游艇啊!”她对画家说,“内部装修,我一定向你请教。我要作多么美好的海上游!去看看考斯\footnote{考斯是英国怀特岛上一海港,以海水浴场及竞渡而著名。}的竞渡该多美!有一辆汽车怎么样?女子汽车服装式样,你觉得漂亮吗?”
\par “不漂亮,”埃尔斯蒂尔回答说,“不过,将来会漂亮的。再说,时装大师很少,也就一两个:加洛\footnote{加洛姊妹自1895年起在泰布街24号开设服装店,曾经设计出带花边的紧腰女用衫。},虽然花边用得有些太多;杜塞\footnote{杜塞父子服装店设在和平大街17号(1853—1928,也有说是21号的),专营衬衣、高级素色手帕、绣的数字及家徽等。其设计构图简洁,多用黑色。埃尔斯蒂尔对高雅而简洁的美极为爱好。},谢吕伊\footnote{谢吕伊于1902年在旺多姆广场2号开业(有说是21号的),直至1915年的旧金山博览会时仍然代表巴黎时装。},有时还有巴甘\footnote{巴甘夫人于1891年(又一说是1880年左右)开店,店址在旺多姆广场。1900年左右迁至和平大街3号。顾客中有西班牙、比利时、葡萄牙王后,也有半上流社会的妇女。她的专长是缎子与丝绒并用的舞会服装。}。其余的全都吓死人。”
\par “如此说来,穿着加洛店里的服装与着普普通通的裁缝做的衣裳,差别很大喽?”我问阿尔贝蒂娜。
\par “当然大极了,我的小傻瓜!”她回答我说,“噢,对不起。只是,唉!别处三百法郎的东西在他们那就要两千法郎。但是确实不一样,对于完全外行的人来说,看上去好像差不多。”
\par “完全正确。”埃尔斯蒂尔答道,他倒没有说,那差异之大,就和兰斯大教堂的一尊雕像与圣奥古斯丁教堂的一尊雕像\footnote{巴黎圣奥古斯丁教堂建于1860—1871年,建筑师为巴达尔,其风格吸取意大利文艺复兴及拜占庭艺术之长。教堂前有保尔·杜布瓦作圣女贞德雕像,乃为兰斯贞德像之仿制品。}之间一样。
\par “对,说到大教堂嘛,”他专门对着我说,因为我们有一次聊天谈到这个问题,那些姑娘没有参加那次谈话,再说,那也绝不会使她们感兴趣,“那天我对你谈到巴尔贝克教堂就像一座高大的悬崖,是当地的石头垒起的大悬崖。可是,相反,”他指着一幅小彩画对我说,“你看这些悬崖(这是一幅草图,取景于克勒尼埃\footnote{克勒尼埃位于特鲁维尔附近。普氏1905年7月14日致露意莎·德·莫尔南的信中曾谈到这个地方。},距这里很近),你看这些切割得有力而又十分高雅的山岩,又多么会叫人想到一座大教堂!”
\par 果然,简直可以说那是高大的玫瑰色拱墙。但是,这是酷热的一日画的,那山岩似乎碎成了齑粉,炎热似乎使山岩蒸发了。炎热吞饮了一半大海,在整个画布的大小上,几乎化成了气体状态。在这阳光似乎已将现实世界摧毁的日子里,现实世界则集中在几个色彩阴暗而又透明的人身上。由于对比鲜明,这些人使你对生命产生更动人心弦、更接近的印象:那是一些影子。大部分渴求凉爽,逃离了火热的海面,躲在山岩脚下,避开阳光。有些人像海豚一样在水上慢悠悠地游着,紧贴着漫游的船舷。在白花花的水面上,人以其油亮而发蓝的身躯使船体显得更高大。说不定正是这些泳者透露出的渴望凉爽的情形,最使人产生这一天那种炎热的感觉。正是这一点叫我发出感叹,我没有见识过克勒尼埃,多么遗憾!
\par 阿尔贝蒂娜和安德烈打保票说,我肯定去过一百次了。如此说来,有一天看到克勒尼埃就会不知不觉地、意料不到地给我以这种对美的渴求了,虽然并不正好是迄今为止我在巴尔贝克的悬崖中寻求的自然美,更确切地说是建筑美。尤其是我,出门去为的是看暴风雨的王国,在我与德·维尔巴里西斯夫人一起出去散步过程中,我们经常只是远远地从树木的空隙中依稀望见大海。我从来不觉得大海真实,流淌,有生命力,使人足以感到它能掀起万顷波涛。我可能只喜欢看到在冬日裹尸布包裹下一动不动的大洋。我真不大能相信,现在我梦寐以求的,竟是失去了其坚固性与色彩的、只不过成了一团白雾的大海!但是,埃尔斯蒂尔,正像那些在因炎热而变得麻木迟钝的船中堕入遐想的人一样,对这样的大海的魅力,已经深得个中三昧,已经善于将海水那觉察不到的涌动,欢乐的一分钟那脉搏的跳动报道出来,固定在画布上了。人们看到这具有魔力的肖像时,只会想到要走遍世界,去寻回那逝去的时日,寻回它那转瞬即逝的沉睡的美。
\par 对埃尔斯蒂尔进行这些访问之前,看到他那幅海景之前,面对着大海,我总是极力从视野中排除前景中的泳人,张着帆的游艇——那帆颜色太白,好似海滩礼服——即排除一切妨碍我说服自己我是在凝望着自古不变的水流的东西。早在人类出现以前,这水流就已经宣泄着它那神秘的生命了。眼前的这幅海景上,一位少妇身着巴莱日纱埃\footnote{巴莱日纱纬纱为毛,经纱为棉或丝,产于比利牛斯山区中一小村。此小村村名为巴莱日,此种轻而薄的衣料由此得名。}或细麻布的长裙,站在一艘挂着美国国旗的游艇上。她将一条细白麻布长裙和一面国旗这“双重”教权注入我的想象之中。我的想象力立刻酝酿起一个贪得无厌的欲望,要立刻在大海附近看见白细麻布长裙和国旗。风和日丽的日子仿佛给这雾气与暴风雨笼罩的海岸裹上了包罗万象的夏季那平平常常的景观,标志着一个时间的简单休止,相当于人们在音乐中称的休止符。现在,在我看来坏天气则成了某种悲惨的变故,坏天气在美的世界里再也找不到位置了:我热切地希望到现实中去找到使我那样激动的事物,我希望天气晴朗,以便能从悬崖顶上看到与埃尔斯蒂尔的画中同样的蓝色的人影。
\par 从前我设想大自然的生命早于人类的出现,而且与令人厌烦的各种工业的完善设备相抵触。这些工业设备直到今日还叫我一参观万国博览会或进女帽商店就要打哈欠。那时我看大海,只是极力观看没有汽船的地段,以便在头脑中保持千古不变的大海的形象,与大海与陆地分离的年代同时,至少也与希腊最初存在的几个世纪同时。这样我便可以反复吟咏布洛克喜爱的“勒贡特老爹”的诗句,并视为永恒真理:
\refdocument{
    \par 他们出发了,精神抖擞、意气风发之王,
    \par 将英雄赫楞手下的长发勇士,
    \par 带往惊涛骇浪的大海上!\footnote{此诗句源于勒贡特·德·利尔的悲剧《复仇三女神》。}
}
\par 埃尔斯蒂尔对我说过,制帽女工以美妙的动作对已经完工的帽子进行最后的修饰,对蝴蝶结或羽毛再至关重要地抚弄一下,这种动作使他很感兴趣,想在绘画上表现出来,就与表现骑手的动作一样(这叫阿尔贝蒂娜心花怒放)。既然如此,我再也不能看不起制帽女工了。但是,制帽女工,要等我返回巴黎才会见到。赛马和竞渡,则要待我重返巴尔贝克才会见到。直到明年以前,在巴尔贝克已经不再举行赛马和竞渡。就连载着身穿白麻细布衣裙妇女远去的游艇也已经无处寻觅了。
\par 我们常常遇到布洛克的姐妹。自从我在她们父亲家里用过晚餐,见了她们就不得不打招呼。我的女友们不认识她们。
\par “家里不许我和以色列人玩。”阿尔贝蒂娜常说。
\par 她将“以色列”说成“以射列”,这种读音方法,即使你没听见这句话的开头,也足以告诉你,这些信仰虔诚的布尔乔亚家庭小姐对于上帝的选民并不怀有好感,说不定她们还会轻易相信犹太人将信仰基督的小孩宰杀之类的话。
\par “何况你的那些女友举止很不像样。”安德烈对我说,微微一笑,表明她很清楚地知道那些人并非我的女友。
\par “所有与这个部落相关的事都是如此。”阿尔贝蒂娜回答道,用的是经验丰富的人那种格言警句式的口气。
\par 说老实话,布洛克的姐妹,既穿得太多又半裸身体,无精打采,胆大包天,又摆阔,又邋遢,不会叫人产生良好印象。她们有一个表妹,只有十五岁,她对莱亚小姐之倾倒令整个游艺场产生反感。老布洛克先生对莱亚小姐的艺术才能极为赏识,但是他对男性演员的艺术才能却缺乏判断能力。
\par 有的日子,我们到附近的一个农庄餐馆去吃茶点。这里的农庄叫什么埃戈尔·玛丽泰蕾斯,爱尔朗十字架,琐事,加利福尼亚,玛丽安托瓦内特等等。这一小帮子选择的常是玛丽安托瓦内特农庄\footnote{爱尔朗十字架田庄和玛丽安托瓦内特田庄位于卡布尔与特鲁维尔之间。}。
\par 有时我们不到哪个农庄去,而是一直攀登到悬崖之巅。一到,坐在野草上,就将带来的三明治、糕点包打开。我的女友们更喜欢吃三明治,见我只吃一块用糖装饰成峨特体的巧克力点心或一块杏子排,都惊讶不已。这是因为,面对加了chester和生菜叶子的三明治这种崭新而无知的食品,我无话可说。而点心受过教育,水果排又絮絮叨叨。点心里有奶油的平淡,水果排里有水果的鲜味,它们对贡布雷、希尔贝特(不仅是贡布雷的希尔贝特,而且是巴黎的希尔贝特。她吃茶点时,我又寻回了贡布雷和在贡布雷的希尔贝特)所知甚多,使我忆起上面有《一千零一夜》故事的那些盛小炉点心的盘子。\footnote{莱奥妮姨母的盘子每一打一套故事。}弗朗索瓦丝一天又一天地今天将《阿拉丁和神灯》,明天将《阿里巴巴》、《睁眼睡觉的人》和《辛伯达携带全部宝物登上巴索拉船》\footnote{这些均为《一千零一夜》中的名篇。}送给姨母莱奥妮时,这些故事的“臣民”们真叫我的姨母开心透了。我真希望再见见这些碟子,可是外祖母不知道这些碟子后来命运如何了,而且她认为那不过是当地买的十分俗气的碟子罢了。这都无关紧要,反正在那香槟省灰蒙蒙的贡布雷,碟子上的商标依然镶嵌着五光十色的图案,正如黑糊糊的教堂内宝石闪动的彩绘玻璃,正如我的房间里黄昏时节那走马灯上映出的影像,正如在车站和省属铁路的风景照前的印度金纽扣和波斯丁香,正如在那外省老太太的阴暗住宅中我姨母那一套中国古瓷器一样。
\par 我躺在悬崖上,眼前只见一片片草地。草地上方,并不是基督教理论的七重天,而只有两重:一重较深——大海,高处的一重较浅。如果我带去了一件什么小玩艺儿,能讨得女友中这一位或那一位的欢喜,她们会那样骤然喜形于色,一瞬间她们那透明的脸庞便变得火红。她们的嘴压抑不住那欢喜,一定要让那欢喜表现出来,于是便开口大笑。我们品味着这种喜悦。她们聚集在我的周围,彼此的面庞相距不远。将一个个面庞分开的空气勾画出碧蓝的小径,有如园丁希望留些空隙,以便自己能够来回走动而在玫瑰丛中辟出的小径。
\par 带来的食物吃光了,我们就做游戏。直到那时为止,我一直觉得这些游戏枯燥无味,有时甚至与“宝塔站岗”或“看谁先笑”一样幼稚可笑。但是,那个时刻,就是给我一个帝国,我也不会放弃这些游戏。这几位少女的面庞仍然洋溢着青春初绽的光彩,我的年龄则已经超出这个。这光彩在她们面前照亮了一切,恰似某些早期宗教画家那酣畅的画面,金色的背景上最无关紧要的细节也从她们的生命中突出起来。对这些少女中的大部分人来说,她们的面庞本身与黎明时那虚无缥缈的红霞混成一体,真正的个性尚未迸发出来。人们见到的,只是艳丽的色彩,在这色彩之下,还无法分辨出来几年之后的轮廓会是什么样。今日的轮廓中还没有任何成份可算是最后定型,只能算做与家庭中某一位已逝的成员暂时有些相像罢了,造物主已向这位去世的成员尽了此种纪念性的礼节。身体已经固定不变,再没有什么指望了,再不会向你许诺什么令你喜出望外之处。不久就会看到尚未显老的面庞四周头发脱落或者变白,就像在盛夏时节的大树上看到已枯的树叶一样,已经毫无希望。这样的时刻会来得那样飞快,这万道霞光的清晨是这样短促,以致有人竟只走到爱情窦初开的少女的地步。这些少女的身体,像一块宝贵的面团,尚在发育。她们只不过是一撮可塑物质,左右她们的转瞬即逝的印痕随时都在塑造着她们。简直可以说,她们每个人都是直率、完整而又转瞬即逝的表情相继塑造而成的快活、少年老成、撒娇、惊讶的小塑像。一个少女对我们流露出的热情关切,这种可塑性会赋予它极度的丰富多彩和极大的魅力。当然,这种热情关切对一位妇女来说也是必不可少的。不喜欢我们的妇女,或者不让我们看出我们讨她喜欢的妇女,在我们眼中,总有某种令人厌倦的千篇一律之处。
\par 这种关切本身,从一定年龄开始,在因生存竞争而变得线条生硬、变成永远有武士气概或出神入化一般的面孔上,再也不会带来柔和的变化。有的面孔,由于乖乖服从丈夫这种力量的反复作用,似乎已经不是女人的面孔,而是士兵的面孔了。另一张面孔,受到母亲每日心甘情愿为子女作出牺牲的雕凿,成了使徒的面孔。又有一张面孔,经过多年的逆境和风暴成了一只老海狼的面孔,只有身上穿的衣裳能揭示她的性别。当然,我们爱这个女子的时候,对我们来说,一个女子的关切尚能在我们在她身边度过的时光上撒播上新的魅力。但是对我们而言,她不会是相继变化前后不同的女子。她的快活对一张不变的面孔而言,乃是外来之物。而少年时代则在完全固体化之先,因此,人们在少女身旁有一种清新感。观看不断变化的形状,不断形成不稳定的对比,就给人以清新感,使人想到大自然中各主要元素永不间断的重新创造。人们面对大海凝望并沉思的,正是这种永不间断的重新创造。
\par 我为这些女友的“环坐猜物集体游戏”或“猜谜语”所牺牲的,还不仅仅是一次白日交际聚会,与德·维尔巴里西斯夫人的一次散步之类。有好几次,罗贝·德·圣卢叫人告诉我,既然我不到东锡埃尔去看他,他可以请二十四小时的假,到巴尔贝克来看我。每次我都写信给他,叫他千万不要这样做,我的借口是我那天正好不在,我要同外祖母到附近什么地方去走亲戚。他从自己的姑祖母那里得知这是我的什么亲戚,扮演我外祖母角色的到底是何人时,肯定对我看法不好。不过,我不仅牺牲了交际活动的快乐,而且也牺牲了友情的欢乐,去选择终日在花园中徜徉的快乐,大概没有错。有这种可能性的人——他们都是艺术家,这倒是真的,而我早就确信自己永远也成不了艺术家了——也有义务为自己生活。友情对他们来说,是对这种义务的支出,是放弃自我。就连作为友谊表现形式的交谈本身,也是非常肤浅的胡言乱语,令我们一无所获。我们可以闲聊上一辈子,什么也不用说,只要无限重复一分钟的空虚即可,在艺术创作的单独工作中思想则是向纵深前进的,唯有这个方向对我们没有封闭,我们可以朝这个方向继续前进。越来越困难,这是真的,但是可以得到真正的成果。而友谊不仅像谈话一样毫无成效,而且有害。我们当中,成长规律纯属内在的人,他们在自己朋友身旁,停留在自己的表面,而不是向纵深方向继续进行自己发现新大陆的航行,就不会不感到烦闷。这种烦闷的印象,在我们恢复独处时,友好的情谊又劝说我们要加以纠正,劝我们激动地回忆起我们的朋友对我们说了什么话,将这些话当成是宝贵的收获。而我们与可以从外部添加石头的建筑不一样,倒与以自己的汁液滋养下一节枝干和最上层花朵的大树十分相像。我庆幸自己得到像圣卢这样善良、聪颖、人人愿意与之交往的人的喜爱和欣赏,我不是叫自己的智慧去适应自己纷乱的印象——理清这些纷乱的印象,本是我的义务——而是去适应朋友的话语。我自己再次重复这些话(我叫活在我们身上、却与自我不是一个人的那个人给我重复这些话,人总是很高兴把思考的重担卸给他人),极力找到这位朋友的美。这种美与我真正孤独一人时所求索的美完全不同,但是这种美赋予罗贝、我自己、我的生命以更大的价值。我这么做的时候,是在自己骗自己,是中断了成长的过程。如果沿着原来的方向发展下去,我确实可以真正地成长起来,得到幸福。在这样的朋友为我造成的生活里,我仿佛娇气地避开了孤独、高尚地希望为他牺牲自己,实际上却意识不到自己的使命了。
\par 相反,在这些少女身旁,虽然我品尝的快乐是自私的,但是至少它不以谎言为基础。谎言极力要我们相信,我们并不是不可救药地孤独,谎言不许我们承认:我们交谈的时候,谈话的不是我们自己,那时候我们是依照别人的模样塑造自己,而不是塑造一个与他人不同的自我。
\par 这一小群少女与我交换的话语没有什么趣味,话也很少,从我这方面又被长时间的沉默所打断。这并不妨碍她们跟我讲话的时候,我怀着同样快乐的心情倾听她们讲话,正如我无比快乐地凝望她们,从她们每个人的声音发现一幅色彩斑斓的图画一样。我怀着极大的乐趣听着她们叽叽喳喳。钟情能帮助人分辨、区别。在一片树林里,鸟类爱好者立刻分辨得出每一种鸟特有的啼啭,一个平常人则搞不清。喜爱少女的人知道人的嗓音比那还要变化多端。每一种嗓音拥有的音符,都比表现力最丰富的乐器还多。每种嗓音对这些音符的组合方式又和人的个性变化无穷一样无穷无尽。与其中一位女友谈天时,我发现,表现她的个性而独有的那幅原画,既通过她嗓音的抑扬顿挫也通过她面部表情的变化,在我面前巧妙地勾画出来,暴虐地强加于我。我发现这是两出戏,每一出在自己的范畴内,表现同一奇异的现实。
\par 肯定,嗓音的曲线与面部的线条一样,尚未最后固定。嗓音还要变,面庞也要变。正如婴儿有一种唾液腺,分泌的液体帮助他们消化牛奶,而长成大人以后这个唾液腺就再也不存在了一样,在这些少女的吱吱喳喳鸣叫声中,也有长成成年妇女以后就再也没有了的音符。这些少女用双唇,怀着贝里尼\footnote{此处指让蒂·贝里尼(1429—1507)。}音乐小天使\footnote{此处普氏指的是威尼斯圣玛丽亚教堂中围绕在圣母及圣婴身旁的那些音乐小天使。}的认真和热情弹奏着这件更为丰富多彩的乐器,这种认真与热情也是青春特有的质地。这热情自信的音色赋予最简单的事情以动人的魅力。无论是阿尔贝蒂娜以权威的口气道出一些俏皮话,还是安德烈谈起她们学校的作业,都是如此。阿尔贝蒂娜说话时,年纪最小的少女无比钦佩地听着,直到最后就像打喷嚏一样怎么也忍不住地狂笑起来;安德烈谈起她们学校的作业,比她们所做的游戏更孩子气,是稚气十足的一本正经。在古代,诗歌与音乐分别还不大时,是以不同的声调来吟诵诗篇的。她们的话语铿锵有声,有如古代的诗句。
\par 尽管如此,这些少女的嗓音已经明确表现出这些小小的人儿每个人对生活的主见。这些主见是那样具有个人色彩,我们如果说这个“她把什么都当玩笑”,说那个“她从肯定到肯定”,说第三个人“她总是停在充满期待的犹豫之中”,都是用词太泛。以后,这些少女会失去这种嗓音。我们面孔上的线条差不多只是由于习惯而形成的、最后不再变化的动作而已。造物主,如同庞培的灾难,仙女变形一般,将我们固定在习惯性的动作上。同样,我们语调的抑扬顿挫包含着我们的人生哲学,是人对事物随时的思考。
\par 当然,这些线条不仅仅属于这些少女。这些线条是他们父母的。个性浸淫于比本人更普遍的事物之中。在这一点上,父母所提供的,不仅是面部线条和嗓音特点这些习惯性的东西,还有某些谈话姿态,某些惯用语句。这些东西几乎与声调一样自己意识不到,几乎与声调一样深刻,也和声调一样,标志着对生活的一种观点。对这些少女来说,在她们达到某种年龄以前,有些词语,她们的父母还没有交给她们,这是真的。一般来说,要待到她们长成成年妇女之后,才会完全交给她们。那些词语现在还储存着。例如,如果谈到埃尔斯蒂尔一位朋友的画,长发还披在身后的安德烈,就还不能使用她母亲和她已成婚的姐姐常用的那种语汇:“那个男子似乎很迷人。”但是,待到准许去王宫时,这样的时刻就到来了。阿尔贝蒂娜自从第一次领圣体以来,已经像她姑母的一位女友那样常常说“我会觉得那相当可怕”这句话了。人们还送给她一个习惯,那就是将别人对她说的话再重复一遍,以便显出很感兴趣并且极力形成有个人特色的看法的模样。如果有人说某一画家的画很好,或者他的房子很漂亮,她就要说:“啊?!他的画好?啊?!他的房子漂亮?”
\par 总而言之,她们出生的省份所强加给她们的有滋有味的原料要比家庭遗产更普遍。她们的嗓音就从出生的外省得来,她们的声调紧紧咬住这乡音。安德烈干巴巴地拨动一个低音音符时,只能使她那发声乐器的短粗弦发出一个带唱腔的音,与她那南方式的五官端正非常和谐。罗斯蒙德呢,她那面孔和嗓音的北方原料与永不休止的顽皮话相呼应,不论她带着自己那个省的口音说什么,都是如此。我发现,这个省份与决定抑扬顿挫的少女气质之间,进行着美妙的对话。是对话,而不是不和。没有任何不和可以将少女与她的故乡分离开来。她依然是它。此外,地方原料对于使用这些材料的天才所产生的反作用,赋予天才更大的活力。对于建筑师的作品也好,精致木器细木工的作品也好,抑或音乐家的作品也好,这种反作用都不会使他们的作品个人味道减少,反映艺术家个性最微妙的特点也不会不细致,因为艺术家不得不在桑利的粗沙岩或斯特拉斯堡的紫砂上创作,他依从了白蜡树上特有的木节,他在写作中考虑到音响的来源及限制,考虑到笛子或中提琴(或女中音)的可能性。
\par 我意识到这一切,我们的交谈却那样少!我与德·维尔巴里西斯夫人或圣卢在一起的时候,我会通过话语表示快乐,比我更正感受的快乐多得多。我离开他们时,总是身心疲惫。相反,静卧在这些少女当中,我丰富的感受无限地超越我们贫乏而稀少的话语,淹没了我不动的身姿和沉默,溢成幸福的河流。流水潺潺而来,消逝在这些初放的玫瑰花脚下。
\par 一个大病初愈的病人,终日在花园或果园中休息,一股花香或果香对于他那悠闲怠惰生活赖以组成的千万琐事来说,绝不及我的目光在这些少女身上寻找的色与香对我感染之深,她们的甜美最后与我融成一体。葡萄就是这样在阳光下积聚起自己体内的糖分。这些如此简单的游戏,慢慢地继续着,给我的内心带来了轻松,幸福的微笑,隐隐约约的头晕目眩,一直叫我闭上了眼睛,正如那些无所事事,终日躺在海边,吸着盐风,晒黑皮肤的人一样。
\par 偶尔,哪一位少女热心的关怀会在我心上唤起激烈的震颤,在一段时间内移开了对其他人的向往。有一天就是这样:阿尔贝蒂娜说:“谁有一支铅笔?”安德烈给了她铅笔,罗斯蒙德给她纸。阿尔贝蒂娜对她们说:“各位女士,正在书写,严禁观看。”她把纸贴在膝盖上,专心致志地将每个字母工工整整画出来,然后把纸递给我,对我说:“注意,别叫人看见!”我将纸条打开,看到她给我写的是这么几个字:“我很喜欢你。”
\par “咱们别写蠢话了,”她向安德烈和罗斯蒙德转过身去,高声叫道,口气激烈而又庄重,“今天早晨我收到希塞尔给我写的信,我得给你们看看。我真是疯了,这信就在我口袋里,对我们会大有用处!”
\par 希塞尔认为应该将她为得到中学毕业证书所写的作文给她的女友寄来,以便她读给其他女友听听。有两个题目供希塞尔任选,在难度上更超过了阿尔贝蒂娜对出题难的担心。一个题目是:索福克勒斯从冥府致函拉辛,以安慰《阿达莉》上演失败;另一个题目是:《爱丝苔尔》首演之后,塞维妮夫人致函拉法耶特夫人,向她表达为她不在场而深感遗憾的心情。请拟信稿。这两个题目里,第一个最难。希塞尔卖劲得很,大概感动了考官。她选了第一个题目,阐述得非常精彩,结果得了十四分\footnote{法国以二十分为满分。},评分委员会也向她祝贺。若不是她西班牙文考试“考砸了”,说不定她能得到“优秀奖”呢!阿尔贝蒂娜立刻给我们读了希塞尔寄给她的作文考卷,因为阿尔贝蒂娜也要参加同样的考试,她很希望听听安德烈的意见。安德烈在这方面比她们所有的人都厉害,可以给她出些好主意。
\par “她真够走运的,”阿尔贝蒂娜道,“这正是她的法文老师叫她在这做过的一个题目!”
\par 希塞尔写的索福克勒斯致拉辛函,是这样开头的:
\refdocument{
    \par \noindent{亲爱的朋友,}
    \par 至今无缘与您相识,冒昧致函,乞谅。新作《阿达莉》岂不表示您对拙作已进行过充分研究?您不仅通过悲剧中主角或主要人物之口道出诗句,且为合唱队写出了精彩诗句。请允许我毫不阿谀奉承地告知于您,据说在希腊悲剧中这合唱队尚可应付,但在法国,此乃地地道道之创举。何况您的天才如此精雕细刻,如此敏锐,如此迷人,如此细腻,如此高尚,已达炉火纯青地步,本人向您致贺。阿达莉、若阿德等人物,您之对手高乃依均无法超出其右。性格粗犷,情节简单、有力。此悲剧并不以情爱为机关,我向您致以真诚赞美。最有名的格言亦非永远最正确。我向您引证的例子便是;
    \par 对这一激情动人的描绘,
    \par 是打动人心的最可靠之路。\footnote{布瓦洛:《诗艺》,第三章。}
    \par 您表明您的合唱队所洋溢的宗教情感并非无法打动人心。广大观众会晕头转向,真正的行家则会给您以公正评价。谨致衷心祝贺并致崇高敬意。
}
\par 阿尔贝蒂娜朗读过程中,双眸不断闪动,熠熠生辉:“真要叫人相信,她这是从什么地方抄来的,”念完以后,她大叫起来,“我从不相信希塞尔能下出这样的蛋来!还有她引的诗句!她是到什么地方去偷来的呢?”
\par 接着,阿尔贝蒂娜钦佩的对象换了,这是真的,但是她的佩服之情有增无减。在安德烈谈话的整个过程中,她眼睛一直瞪得大大的,赞佩之情不停地叫她“眼睛瞪得要掉下来”。安德烈年龄最长,本事也最大,别人要听听她的意见。她首先带着某种讽刺口吻谈到希塞尔的作业,继之,又用难以掩饰真正严肃的轻佻表情,以自己的方式重写了那封信。
\par “还算是不错,”她对阿尔贝蒂娜说,“不过,如果我处在你的地位,人家给我也出这个题目——这是有可能发生的,因为经常出这道题——我就不这么做。我怎么做呢?首先,如果我是希塞尔,我可不那么一下子就冲动起来,我首先在另外一张纸上列出我的提纲。第一行,提出问题,展开主题;然后,要放在发挥部分的大概意思;最后,评价,文体,结论。这样,从要目一看,就知道思路如何。蒂蒂娜\footnote{阿尔贝蒂娜的爱称。},主题刚一展开,或者你更喜欢,既然这是一封信,可以说一入题,希塞尔就干了蠢事。索福克勒斯给一个十七世纪的人写信,他不应该写‘亲爱的朋友’。”
\par “确实,她本应该叫索福克勒斯说‘亲爱的拉辛’,”阿尔贝蒂娜充满激情地大叫起来,“这样就好多了。”
\par “不对,”安德烈用有点讽刺嘲笑的口吻回答道,“她应该写‘先生’。同样,结尾的地方,她本应找到诸如‘先生(最多是‘亲爱的先生’),恕我直表敬意,臣仆谨拜’这一类的字眼。另一方面,希塞尔说在《阿达莉》中合唱队是创举。她把《爱丝苔尔》忘了,还有两出不太著名的悲剧,今年教师正好分析了这两部悲剧。所以,只要提到这两部悲剧,这是老师喜爱的话题,就可以确有把握考取。这两部戏是罗贝·加尼埃的《犹太女人》和蒙克莱斯基安的《饶命》\footnote{古希腊悲剧诗人的作品,例如索福克勒斯、欧里庇德斯的剧本(剧中均有合唱队),于十六世纪上半叶相继译成法文。1553年,艾提安·若代尔创作了《被俘的克丽欧巴特尔》,开法国带合唱队的悲剧先河。罗贝·加尼埃及蒙克莱斯基安走的是同一路子。这两个剧本与《爱丝苔尔》为同一题材:犹太人的痛苦遭遇。罗贝·加尼埃(1544—1590)于1583年写成《犹太女人》,是一个复仇故事。蒙克莱斯基安(1575—1621)的剧本《饶命》于1601年写成,情节与《爱丝苔尔》十分相近。}。”安德烈道出这两个戏名,掩饰不住善意的比别人高出一头的情感,这种感情通过微微一笑表现出来,且是优美动人的一笑。
\par 阿尔贝蒂娜再也忍不住了:
\par “安德烈,你太棒了,”她大叫起来,“你得把这两个戏名给我写下来。你信不信?我若是碰上这道题,那该多走运!甚至口试碰到了,我也要立刻谈起这两个戏,那一定会给人留下深刻印象。”
\par 此后,每次阿尔贝蒂娜要求安德烈再给她说一遍这两个戏的戏名,好把它记下来的时候,这位学识渊博的朋友都声称已经忘了,从来没有再告诉她。
\par “其次,”安德烈接着说下去,口气里对于比她更幼稚的伙伴有一种难以察觉的蔑视,但仍为自己能叫别人佩服而兴高采烈,而且对自己怎么写这篇作文的重视,超出她希望别人对此予以的重视,“冥府中的索福克勒斯应该很熟悉情况。他应该知道《阿达莉》是在太阳王\footnote{太阳王指路易十四。}和几位得天独厚的朝臣面前演出的,\footnote{《阿达莉》为1691年拉辛应路易十四宠幸的曼特侬夫人之请而写的悲剧,因抨击宗教,宣扬宽大容忍而触怒国王。}并不是给广大观众演出。希塞尔就此而言的行家赞美倒一点不错,不过,似乎还可以再补充一些。索福克勒斯已成不朽,很可以具有预言的天才,宣称依伏尔泰之言,《阿达莉》不仅是拉辛的杰作,而且是人类才智的杰作。\footnote{伏尔泰为自己所写的悲剧《信奉袄教的波斯人》(1769,未上演)著一文,其中有“《阿达莉》可能为人类才智的杰作”一句。}”
\par 阿尔贝蒂娜贪婪地饮啜着这些话语。她的双眸燃烧着火焰。这时,罗斯蒙德提议开始做游戏,她十分气愤地加以拒绝。
\par “最后,”安德烈以同样淡漠、随便、有点嘲讽意味而又相当热情自信的口气说道,“如果希塞尔首先将她要加以发挥的总的观点都从容地记了下来,她说不定会想到我会怎么做,那就是指出索福克勒斯的合唱队所受到宗教的启发与拉辛的合唱队所受宗教之启发二者之间的不同。我要叫索福克勒斯指出,虽然拉辛的合唱队像希腊悲剧合唱队一样充满宗教情感,然而他们所信奉的,并非同样的神。若阿德的神与索福克勒斯的神毫无共同之处。到了发挥部分的结尾,会十分自然地导致这样的结论:‘宗教信仰不同又有什么关系?’索福克勒斯强调这一点可能有些顾虑。他可能担心这样会伤害拉辛的宗教信念,于是在这个问题上他又对拉辛在王家港的各位老师\footnote{此处影射拉辛曾在王家港修道院小学校就读的事。}添上几句,宁愿对自己的对手诗才水平之高加以祝贺了。”
\par 钦佩和聚精会神使阿尔贝蒂娜浑身发热,此刻她已大汗淋漓。安德烈则保持着女性纨绔子弟那种微微含笑的冷淡。
\par “再引几位著名批评家的一些评论,也不坏。”她说。然后我们就又做游戏了。
\par “对,”阿尔贝蒂娜答道,“有人对我说过这个。一般来说,最值得推崇的,便是圣勃夫和梅莱\footnote{居斯塔夫·梅莱(1828—1891),路易大帝的中学法语教师,专门讲授修辞,写过许多文学批评研究文字,主要著作有《高级修辞班及文科中学毕业会考法国古典大师文学研究》(1875)一书。}的论点,是不是?”
\par “你倒不一定错。”安德烈回答。不管阿尔贝蒂娜怎么哀求,她始终拒绝给她写出那两个剧本的名字,“梅莱和圣勃夫坏不了事。但是特别应该引用德都尔\footnote{费利克斯·德都尔(1822—1904)亦为法文教师,他于1859年发表《拉辛的敌人》一书。}和加斯克代福塞\footnote{列翁·加斯克代福塞于1898年发表《拉辛剧作选》,在引言中,他引了安德烈上文提到的伏尔泰的话。}。”
\par 这工夫,我一直想着阿尔贝蒂娜递给我的那张从笔记本上撕下来的小纸:“我很喜欢你。”一个小时以后,踏着回巴尔贝克的小路——对我来说,这路过于陡峭——下山时,我心中暗想,我的罗曼蒂克肯定是和她了。
\par 有一系列的信号,一般来说,通过这些信号我们可以辨别出来我们已经堕入了情网。例如,我吩咐旅馆不要因任何人来访而叫醒我,唯独这几个少女中的哪一位来访除外;等待她们(不论该来的是哪一位)前来时,心房那样剧烈地跳动;这种日子,如果我未能找到理发师为我修面,不得不难堪地出现在阿尔贝蒂娜、罗斯蒙德或安德烈面前,我是多么气恼,等等。以这一系列信号为特征的这种状态,因这一个少女或另一个少女轮流反复出现,与我们称之的爱情不同,大概与植形动物类的生命与人的生命之不同情形相仿。如果可以这么说的话,在植形动物类中,生命、个性分布在不同的器官上。但是博物史告诉我们,这样的动物机体是可以观察的,而我们自己的生命,无论如何已经比植形动物更加进化,就我们从前意想不到而现在应该经历的状态的真相而言,并非更加无法肯定,除非我们后来放弃了这种状态。例如,对于我来说,这种同时将心分到好几个少女身上的恋爱状态。一心数爱,或者更确切地说是数爱一体,因为最常使我觉得甜美无比的,与他人不同的,对我来说开始变得那么宝贵,以致成了我生活中最大的快乐的,希望第二天依然如此的,可以说便是这一组少女的全体,从悬崖上,一片草地上,海风吹拂的数小时的总体中获取的全体少女。阿尔贝蒂娜、罗斯蒙德、安德烈的面庞在那一方草地上流露出千姿百态,那样激发起我的想象能力。我无法道出使这些地点对我变得那样珍贵的是哪一个,我最想爱的是哪一个。一场恋爱开始时,也和结尾时一样,我们并非一味依恋爱的对象,更确切地说,因这爱的对象而起的爱恋欲望(以后则是爱的对象留下的回忆)带着肉欲在可相互置换的魅力区域中游荡——这种魅力有时纯属生理、美食、住所方面——各种魅力之间相当和谐,使这种爱的欲望在哪一种魅力身边都不会感到陌生。此外,在她们面前,我还没有因司空见惯而厌倦,我有能力看到她们,这意思就是,我有能力在每次置身于她们之间时都感受到深深的惊异。
\par 显然这种惊异的部分原因,是此人此时又向我们展示出他本人新的一面。每个人的多面性又是那样庞大,面庞与身体的线条那样丰富,很少现出同样的线条。我们刚刚离开这个人的身边,在我们回忆的绝对简单化之中,正如同记忆选择了给我们印象深刻的某一特点,将这个特点孤立起来,加以夸大一样,我们觉得个子很高的一位女子,在草图中就成了身高异乎寻常;我们似乎觉得金发、皮肤白里透红的一位女子,在草图中就成了纯粹的《粉红与金色之和谐》了\footnote{此题目为杜撰,但画家惠斯勒的作品常有这样的题目,例如《金色与黑色的夜景》,《灰与绿之和谐》,《粉红与银色音符》,《金色与栗色之和谐》等等。据说惠斯勒是埃尔斯蒂尔的原型之一。}。待到这位女子重新出现在我们身旁,所有构成她的平衡的被遗忘了的其他长处,以其纷乱的复杂性向我们袭来时,她的身高降低了,粉红的面颊被淹没了,我们专门前来找寻的东西,被其他的特点代替了。这其他特点,回想起来,第一次时我们也曾注意到,只是不知为何竟没有料到会再度看到这些。我们回忆一下,我们想去迎接一只孔雀,可是找到的是一朵牡丹。此种不可避免的惊异无独有偶。还有另一种惊异,从差异而产生,并非回忆的因袭形式与现实之间差异,而是在上一次我们见到的人与今天从另一角度在我们面前出现、向我们显示了一种新面貌的这个人之间的差异。人的面孔确实与东方某多神教神谱中神的面孔一样,是从不同角度重叠在一起的一连串面庞,凡人是不能同时完全看见的。
\par 但是,我们惊异的原因,大部分特别来自别人在我们面前呈现的是同一个面孔。我们必须下很大工夫才能重新创造出我们的身外之物向我们提供的一切——哪怕是一种水果的味道——我们刚刚得到一个印象,便不知不觉地沿着回忆的斜坡滑了下去,结果是在很短时间内,我们已经不知不觉地距离我们的感受很远了。于是,每一次重新见面都是一种纠正,将我们带回我们真真切切之所见上去。我们已经想不起来了,人们称之为记住某某的,实际上是忘记某某。只要我们还有机会重见,已经遗忘的线条在我们面前出现的那一刻,我们又认出来了,我们不得不纠正在记忆中产生了偏差的线条,就这样,无止无休而又丰富多彩的惊异使我与这些海滨少女每日的约会变得那样有益于身心健康,轻松愉快。这惊异既由许多发现,也由许多模糊的回忆组成。再加上她们在我心中唤起的动荡——这种内心动荡从来就不完全是我所想的那样——更使得对下一次聚会的期望与上一次的期望不再完全相同。从最后一次交谈那尚动人心弦的回忆中,可以明白每次散步,都对我的思想重重打上一闷棍,而且丝毫不是朝着我在自己房间的孤寂中头脑冷静时所能规划出来的方向。当我头脑里像一群蜂一样轰响着使我心潮翻滚、久久在我心中回荡的话语回到旅馆时,早已把这个既定方向忘到九霄云外去了。每个人,我们不再看见他的时候,他就被消灭了。此后他再次重现,便是一次新的创造,与紧挨在前面的那次出现便不同,甚至比前面的哪一次都有所不同。在这些创造中主导一切的变化,至少有两个。当我们回忆起精神抖擞的目光、大胆的表情时,到了下一次,不可避免地会是无精打采的身影,若有所思的神气,这正是我们在上次回忆中所疏忽的地方。到下一次相见时,我们又一定感到惊异,也就是说,几乎只对这些留下深刻印象了。在我们的回忆与新的现实对照时,给我们的失望或惊异打上烙印的,正是这个,似乎对现实进行修改,提醒我们记忆不准确的,正是这个。反过来,上一次所忽略的面庞特点,正因为如此,这一次成了最能抓住人、最真实、最有纠正意味的特点,又将成为思考和回忆的材料。我们希望再度见到的,又是无精打采、圆乎乎的身影,和气而又若有所思的表情了。可是,到了下一次,有洞察力的眼睛、尖尖的鼻子、紧闭的嘴唇所包含的意志方面的内涵又要重新来纠正我们的愿望及其认为与之相符合的对象之间的差距了。当然,此种对初次印象的忠实,而且纯粹是外表方面的印象,每次在我的女友们身边都重新得到修正的这些印象,并不仅仅与她们面部五官有关系,诸位读者已经看到,我对她们的嗓音也同样敏感。说不定她们的嗓音更叫人心慌意乱(因为嗓音不仅仅提供了与面庞同样的特殊而又官能性的表面,它还是不可企及的深渊的组成部分,使人产生无望的亲吻那种头晕目眩)。她们的嗓音犹如一件小小乐器的单音,每种声音都全力以赴,却又只属于它自己。哪个嗓音,我已将它遗忘,当哪一种抑扬顿挫又将它勾画出来,我又辨认出这嗓音时,它的某一深曲线又叫我惊异。就这样,每次相见,我不得不进行校正以便回到完全准确上去,就和调音师、音乐教师或制图员进行的校正一样。
\par 这些少女在我心中传播开各不相同的情感波。每种波都对其他波的扩散进行抵制,各种不同的波便相互抵消,已有一些时候。这种和谐的粘合,一天下午我们玩环坐猜物集体游戏时,终于打破,而倾向到阿尔贝蒂娜一边。那是在悬崖顶上一片小树林中。那天我们大概人数很多,那小帮子又带去一些圈外的人。我的位置在不属于这小帮子的两个少女中间,我满怀艳羡地望着阿尔贝蒂娜旁边的一个小伙子。心想:如果我在他那个位置上,在那可能永不会再来的意料不到的几分钟里,就可以触到我女友的手了。想到只要接触到阿尔贝蒂娜的手,甚至没有想这样必然会导致什么后果,我已经觉得甘美无比。这并不是因为我从未见过比她的手更好看的手。甚至就在她的女友这一小组里,安德烈的手,修长而又细腻得多,似乎过着特殊、乖乖服从那姑娘指挥而又独立的生活。那手常常在她面前伸得长长的,好似高贵的猎兔狗,懒洋洋的,又好似漫长的梦。突然拉拉某一节指骨,都会使那手变得更长,因此埃尔斯蒂尔还为这手画过好几张习作。从一张习作上,可以看到安德烈正在火前烤手。在灯光下,她的双手如同两片秋叶,为半透明的金色。阿尔贝蒂娜的手更肥胖一些,与她握手时,在你的手紧握下,她的手先松弛一下,然后便抵住那握力,给人以一种极为特殊的感觉。阿尔贝蒂娜的手着力时,具有性感的柔和,似乎与她的皮肤那粉红之中稍带紫色调的色泽形成浑然一体。这样的着力似乎使你进入少女体内,进入她的感官深处,如同她那响亮的笑声与鸽子叫或某些叫喊相似一般,不大得体。某些女子,与她们握手是那样令人快乐,人们真要感谢社会文明将shake hand\footnote{英文:握手。}变成了初次接触的青年男女之间可以允许的行为。阿尔贝蒂娜就在这样的女子之列。如果有什么不近人情的施礼习惯以另一种动作代替了握手,我大概就只能每天怀着迫不及待的心情望着她那不可触知的手兴叹了。这种迫不及待要接触她的手的心情,与迫不及待要知道她的面颊是什么味道的心情同样强烈。如果做环坐猜物游戏时我坐在她旁边,我期望的将她的手长时间握在我的手里的那种快乐,并不在这快乐本身:那样,直到如今因腼腆而憋在心中的那么多爱情倾诉和表白,就能通过手的某些着力动作传递出去。她那方面,用不同的着力来回答,可以多么轻而易举地向我表示她接受这种感情!多么好的串通,多么美的感官享乐开端!在这样在她身旁度过的几分钟之内,我的恋爱会比自我与她相识以来有更大的进展!我感到这样的时刻不会长久,很快就要结束,因为肯定不会长时间玩这个小小的游戏。游戏一结束,那就为时太晚了!我简直坐不住了!
\par 我故意叫人把戒指抢走。一到了圈子中间,那戒指往下传时,我佯装没有发觉,却用目光瞟着它,等待着它传到阿尔贝蒂娜身边那个男孩子手里的时刻到来。阿尔贝蒂娜放声大笑,游戏很热闹,也很快活,她满脸粉红。
\par “我们正巧是在树林里。”安德烈指着我们四周的树木对我说,眼中含笑。那笑是只为我一个人的,似乎超越了做游戏的人,好像只有我们两个人有足够的聪明才智,能够相互窥视内心并对游戏作出具有诗意的评论。她甚至心细到像去特里亚侬\footnote{特里亚侬为凡尔赛宫殿的一部分,分大、小特里亚侬。大特里亚侬建于1670年,后来1687年芒萨尔建“大理石特里亚侬”,代替了原来的大特里亚侬。}便不能不在那里举行路易十六式的庆祝活动的人,或者觉得在为之写了曲子的环境里叫人唱那个曲子才有滋味的人一样,虽然并不特别有情绪,还是唱了起来:
\refdocument{
    \par 女士们,白鼬从这里过去了。
    \par 美林白鼬从这里过去了。
}
\par 如果我有闲工夫想到这个,肯定要为从这个艺术处理中找不到优美之处而难过。可那时我的心思完全不在这个上。参加游戏的男男女女,开始对我那么愚蠢、抓不住戒指而感到奇怪了。我望着阿尔贝蒂娜,她那么漂亮,那么毫不在乎,那么快活。她怎么也料想不到,待我终于从别人手里截住戒指时,她就要在我旁边了。必须借助于她丝毫不会起疑的一计,不然她会恼火的。在玩得热火朝天之时,阿尔贝蒂娜的长发已经散开,成了一绺一绺的鬈发,散落在她的双颊上。那头发干干爽爽,金色,更加突出了她那粉红的肤色。
\par “你有与劳拉·迪安娜\footnote{劳拉·迪安娜(1476—1534)为阿尔封索一世的宠姬。有人认为提香的肖像画《正在梳妆的少妇》(陈列于卢浮宫中)画的就是她,但证据并不确凿。此处普氏想的正是这幅画:一位美丽的少妇对镜自赏,手中握着半编成发辫的一部分长发。}、埃莱奥诺·德·居荣埃莱奥诺·德·居荣(1122—1204)也以秀发而出名。但是“受到夏多布里昂如此钟爱”的那位女子与她似没有亲戚关系。此人为德·居斯蒂娜侯爵夫人,她是玛格丽特·德·普罗旺斯的后代。但是埃莱奥诺·德·居荣的孙子娶了玛格丽特·德·普罗旺斯的妹妹,而且她的妹妹名字也叫埃莱奥诺。这可能是普氏搞混的原因。以及她那位受到夏多布里昂如此钟爱的后代一样的发辫。”为了接近她,我常常附在她耳边说。
\par 忽然,戒指传到了阿尔贝蒂娜身边那个男孩的手里。我立刻扑上去,粗暴地掰开他的手,抓住戒指。他只好到圈子中央我原来的位置上去了,而我则取代了他的位置,坐在阿尔贝蒂娜旁边。几分钟以前,我看见这个小伙子的手滑到小绳上,随时都碰到阿尔贝蒂娜的手,我非常羡慕这个小伙子。现在轮到我了。可是我太羞涩,不敢去寻求这样的接触;太激动,体验不到这样接触的滋味。我感觉到的,只有我的心在剧烈而痛苦地跳动。
\par 有一阵,阿尔贝蒂娜会意地将她那丰满而又粉红的面庞朝我凑过来,佯装手中握有戒指的样子,以欺骗白鼬,防止他往戒指正在传递的方向看。我立刻明白了,阿尔贝蒂娜目光中那暗示是指的这个把戏。当我看见纯粹为了游戏的需要而佯作有一桩秘密、有一种默契的目光在她眼中闪烁时,我真是心慌意乱。这秘密,这默契,在她与我之间并不存在。但是从此时起,我觉得这似乎是可能的,而且觉得天堂一般甜美。这个念头激动着我,就在这时,我感到阿尔贝蒂娜的手轻轻压在我的手上,她那抚慰人的手指滑到了我的手指下面。我看到她同时向我眨眨眼睛,极力叫别人觉察不到。顿时,直到此刻我自己尚看不清楚的一系列希望形成了:
\par “她这是利用游戏叫我感觉到她很喜欢我。”我高兴得上了天,想道。就在这时,我听到阿尔贝蒂娜恼火地对我说:
\par “快拿住啊,我递给你递了一个钟头啦!”
\par 我的情绪立刻跌了下来。
\par 我难过得痴痴呆呆,松开了小绳。白鼬瞥见了戒指,朝她扑过来。我不得不再次到圈子中央去,心灰意懒,望着那发疯的圆圈继续在我四周打转。所有的姑娘都与我开玩笑,诘问我。为了应答,我只好笑,可我一点也不想笑。
\par 阿尔贝蒂娜却不停地说:
\par “不想聚精会神就别玩!成心叫别人输,就别玩!安德烈,以后咱们做游戏的日子再不请他了,不然我就不来了。”
\par 安德烈超然于游戏之上,仍在唱着那首《美林》。罗斯蒙德见样学样,也并无坚定信念地接着唱起来。安德烈想转移一下阿尔贝蒂娜的责备,对我说:
\par “你那么想看的克勒尼埃景色,就离这儿几步远。来,我领你从一条美丽的小路一直走过去,让她们这些疯子去装八岁小孩吧!”
\par 安德烈对我极好,于是路上我对她谈到似乎在阿尔贝蒂娜身上特有的、足以叫她爱上我的一切。安德烈回答我说,她也很喜欢阿尔贝蒂娜,觉得她非常动人。不过,似乎我对她女友的恭维并不令她开心。
\par 忽然,在低洼的小路上,我停下了脚步,童年时代温馨的回忆打动了我的心:从那经过修剪、闪闪发光、探到路边的树叶上,我认出了一簇山楂树,可叹自暮春便落了花。我的四周,荡漾着从前玛丽亚月玛丽亚月即三月。、星期日下午、已忘却的信仰和失误的气息。我真想抓住这气息。我停下脚步一秒钟,安德烈怀着动人的预见,让我与树叶交谈片刻。
\par 我向树叶询问开花的情况,这些山楂树的花与天性活泼、冒失、爱俏而又虔诚的少女颇为相似。
\par “这些小姐早已经走了。”树叶对我说。
\par 说不定树叶心里在想,我自称是这些花朵的挚友,可是看上去我对花儿的生活习惯并不怎么了解。是一位挚友,但是已经这么多年没有与她们重逢了,虽然曾经许下了诺言。然而,正像希尔贝特是我与少女的初恋一样,这些花朵也是我与花朵的初恋。
\par “对,我知道,她们六月中旬前后走,”我回答道,“但是见见她们在这里住过的地方,我也很高兴。她们曾经到贡布雷我的卧房里来看我,是我生病的时候我母亲带她们来的。我们总是在玛丽亚月的星期六晚上重逢。她们也能到这里来吗?”
\par “噢,当然啦!再说,人们对于在荒漠圣德尼教堂里见到这些小姐看得很重呢!荒漠圣德尼教堂就是离这儿最近的教区。”
\par “那么,现在要看她们呢?”
\par “噢,明年五月以前是不行了。”
\par “可以肯定她们明年一定会在这里吗?”
\par “每年都准时在这。”
\par “只是我不知道我是不是还找得到这个地方。”
\par “会的!这些小姐性情那么快活,只有唱赞美诗的时候,才中断笑声。你从小径的尽头就能分辨出她们的香味,绝不会错!”
\par 我追上安德烈,重又在她面前赞扬起阿尔贝蒂娜。我那么反复强调,我似乎觉得她不会不在阿尔贝蒂娜面前学舌。可是我后来从来没听阿尔贝蒂娜说她知道这些事。安德烈对别人心事的理解和待人之周到,要胜过阿尔贝蒂娜十分。找到恰如其分的眼神、字句、动作,极为巧妙地叫人开心;一个感想,可能叫人难受,便吞进腹中;牺牲一小时的游戏,甚至一个上午,一次游园聚会(又显出这不是一种牺牲的样子)以留在心情悲伤的男友或女友身边,向他(或她)表示她宁愿陪他(或她)一个人而不喜欢那些轻浮的快乐,这都是她习惯成自然的高尚情怀。当人们进一步了解她时,简直可以说,她的情形犹如那些本来很胆小但是不愿意显出恐惧的小英雄,她们的勇武尤其值得赞扬。简直可以说,这种善良丝毫不存在于她的天性之中,她随时随地表现出来,乃出于精神高尚,感觉敏锐,要表现出是别人的忠诚朋友的良好意愿。
\par 关于我和阿尔贝蒂娜之间的缘分,听着她对我说的动人言辞,似乎她会全力以赴以成全我们。然而,可能出于偶然,可以安排的、能够将我和阿尔贝蒂娜结合在一起的事情,她从来没有干过一桩。我不敢发誓说,为了让阿尔贝蒂娜爱上我,我下的那些工夫在她朋友的心中即使没有引起搞些什么秘密勾当以从中作梗的话,至少在她心中引起了某种愤怒。当然这种愤怒掩饰得很好,而且出于高尚的情操,说不定她自己也在与之作斗争。安德烈的种种善意周到,阿尔贝蒂娜是做不到的。然而安德烈内心深处是否善良,我无法肯定,正如那以后我对阿尔贝蒂娜是否善良也不能肯定一样。
\par 安德烈对阿尔贝蒂娜感情奔放而流于轻浮,总是表现出慈爱的宽容,对她说话,微笑,全是一个女友的话语和微笑。更有甚之,她总是以朋友的身份行事。为了叫这个贫困的朋友享受她自己的奢华,为了使这个穷朋友幸福,我日复一日地看见她比打算得到君主垂青的弄臣还要卖力,而个人从中没有任何好处可捞。别人在她面前怜悯阿尔贝蒂娜的贫困时,她是那样温和,话语忧伤而感人肺腑,真是令人动容。较之对待一个富有的朋友,她更是操上一千倍的心。如果有人提出,阿尔贝蒂娜说不定并不像人们说的那么贫穷,安德烈的眉宇间就会罩上一层难以察觉的乌云。她似乎怏怏不乐。如果别人还要进一步说,归根结底,阿尔贝蒂娜也许并不会像人们想象的那么难找婆家,她就要极力与您说相反的话,几乎恼火地反复说:“可惜,她一定嫁不出去!这我知道,而且这叫我心里够难受的了!”
\par 甚至对我而言,在这帮少女中,她也是唯一在我面前从未传过别人对我说的不好听的话的人。更有甚者,假如是我自己唠叨这些话,她还佯装不相信或者作出解释,使那些话变得不伤人了。这一系列的长处,就叫机灵。有的人,如果我们要去跟谁决斗,他们首先要向我们祝贺,并且补充一句,说没有理由要这样干,这是为了在我们眼中更抬高我们表现出的勇气,我们并不是不得已而为之。机灵就是这些人的特性。有人与这种人正相反,在同样的情况下,他们说:“你肯定很讨厌与人去决斗,可是另一方面你又咽不下这口气,不这么干不行。”\footnote{普氏本人1897年2月6日即在默东森林与让·洛兰决斗过。}在任何事情上总有说好与说坏的。如果我们的朋友在我们面前复述别人说我们的伤人的话,而且为这样做而感到高兴,或至少感到无所谓,便证明他们对我们讲这些话的时候,并不怎么能设身处地,并不怎么爱我们,还要往我们身上针刺、刀割,就像往动物肠膜上针刺、刀割一样。而另外一种朋友,也就是满脑子机灵的朋友,他们听到别人对我们的行动之所言,或者我们的行为使他们产生什么看法,会使我们不快,他们总是对我们加以隐瞒,这种艺术可以证明他们具有高超的遮掩本事。如果他们确实不往坏处想,而且人家说的话叫他们不好受,正像这些话也会叫我们难过的话,这种遮掩是并无不妥之处的。我想,安德烈就属于这种情况,当然我这样说并无绝对把握。
\par 我们早已走出小树林,沿着人迹罕至的崎岖小路前进。安德烈倒一点不转向。
\par “看,”她忽然对我说,“这就是你那了不起的克勒尼埃。你还挺有运气,这正好是埃尔斯蒂尔画的那种天气,那种光线。”
\par 顿时,在我脚下,我辨别出了埃尔斯蒂尔所窥视和撞见的海上仙女,她们躲藏在山岩之间,避过炎热。在可与达·芬奇的一幅画相媲美的暗色透明涂料下,这些美丽动人的影子,在树荫遮掩下,转瞬即逝,灵活敏捷,默默无语,随时准备在阳光一抖动之时便溜到石头下面去,躲藏在石缝间。阳光的威胁一过去,这些影子又飞快回到山岩或海带旁。在悬崖和颜色消褪的大洋那碎成斑斑点点的阳光下,这些影子似乎又在看守着山岩或海带小憩,是一动不动而又轻浮的看门女人,紧贴着水面露出她们那凝脂般的身体和暗色眼珠那专注的目光。可惜我还在为环坐猜物游戏时从希望的顶巅跌落下来而痛苦悲伤,所以我并没有体会到不是这种情绪时我会体会到的那种快乐。
\par 我们又和其他少女会齐,踏上归途。现在我知道我爱的是阿尔贝蒂娜了。可惜,我倒不为让她知道此事而操心。自从在香榭丽舍大街游戏以来,虽然我的爱情相继眷恋的人几乎都一样,我的爱情观却已发生变化。一方面,向我心爱的人倾诉,表白自己的柔情,我似乎觉得这不再是谈恋爱最重要、最必要的一幕了;爱情本身,我似乎也觉得不是外在的现实,而只是主观的快乐了。这种快乐,我感到,唯其阿尔贝蒂娜不知道我会感受到,她才会更加高高兴兴地去做一切必须做的事来维系它。
\par 整个归途中,从别的几位少女身上放射出的光焰吞没了阿尔贝蒂娜的形象,她的形象对我来说并不是唯一的存在。但是,正如白昼时月亮只是形状更具特点、更固定的一小片白云,阳光一旦消失,月亮就显示出其全部巨大威力一样,待我回到旅馆以后,从我心中升起并开始光芒四射的,便只有阿尔贝蒂娜的形象了。我似乎骤然间觉得我的房间变了样。当然,这房间早已不是第一天初来乍到的那个晚上那充满敌意的房间了。我们不断地改变着我们四周的住处,随着司空见惯免去了我们的感受,便将体现我们不自在感觉的那些有害的颜色、空间和气味各种因素都取消了。这个房间虽然对我的情感还起着相当大的作用,显然已不再使我痛苦,而是给我以快乐了。它成了美好时日的酿造池,好像一个游泳池,美好的时日使浸着阳光的一片蔚蓝在泳池半人高的地方如明镜般闪烁,阳光像热量散射一样看不见摸不着而又雪白一片,一度覆盖了水中映出的、飞驶的一艘帆船。这房间也不再是欣赏绘画的傍晚那纯粹具有审美意义的房间。这是我在这里住了这么久以致我已经视而不见了的房间。现在,我又开始对它睁大了眼睛,但是这一次,是从恋爱这个自私自利的角度出发了。我想,这倾斜的漂亮大镜子,镶着玻璃的华丽书柜,如果阿尔贝蒂娜来看我,会使她对我看法不错。我的房间作为我逃往海滩或里夫贝尔之前在这里过上一刻的过渡地点,对我又变成实实在在、十分宝贵、焕然一新了,因为我是以阿尔贝蒂娜的眼睛来观看和欣赏室中的每件家具的。
\par 做环坐猜物游戏以后过了几天,我们外出散步。信步走去,走得太远了,最后在梅恩维尔找到了两辆有两个座位的小“酒桶”车\footnote{轻型马车,车棚低矮。要从后面钻进车内,因而称为“酒桶”车。}。坐上这两辆车能叫我们吃饭时回到家,大家真是高兴极了。我对阿尔贝蒂娜已经爱得很强烈,其效果是,我先后向罗斯蒙德和安德烈提议与我同乘一辆马车,而没有一次提出让阿尔贝蒂娜与我同坐一辆车。后来,我一面优先邀请安德烈或罗斯蒙德,一面用时间、路线、大衣这些次要问题的考虑,让大家做出决定——似乎违背我的心愿——最实在的办法还是我与阿尔贝蒂娜同坐一辆车。对于她来陪我,我装做勉强接受的样子。可惜爱情总是倾向于要把一个人完全吸收进去,只不过通过谈话方式,任何人均无法食用。归途中,阿尔贝蒂娜极尽热情之能事。但是这毫无用处。待我将她送到家,留下我一个人,我感到非常幸福,却比动身时对她更加渴望。我只把刚才一起度过的时光看成是一个序曲,与此后一起度过的时光相比,其本身并无多大重要性。然而它具有初次的魅力,一去不复返。我对阿尔贝蒂娜尚未提出任何要求。她可能已在想象我会要求什么,但她并没有什么把握,可能设想我只倾向于并无明确目的的男女关系。在这种关系中,我的女友大概会找到甜蜜的、富有期待的意外的浪花,这就是浪漫情调。
\par 此后的一个星期中,我并不千方百计要见阿尔贝蒂娜。我佯装做更喜欢安德烈。恋爱开始,人们希望在自己心爱的女子面前,仍保留着她会爱的陌生人形象。但是人们又需要她,又需要更多地接触到她的关注,她的心,更甚于接触她的肉体。在一封信中,人们无意地写上一句恶言恶语,这将迫使那个无动于衷的女人向你要求一份热情。爱情,按照一种必然有效的技艺,对我们来说,就是用双向运动来拧紧齿轮系统,我们在这齿轮咬合之中,再也不能不爱,也再也不能被爱。
\par 别人去参加什么白天的聚会,我把这个时间给了安德烈,我知道她因为高兴,会为我牺牲这次聚会,她甚至会很烦闷地出于高尚情操而为我牺牲这几个小时,为的是不让别人和她自己产生什么想法,认为她将相对说来属社交性质的快活看得太重。于是我安排每天晚上单独和她在一起,倒不是为了叫阿尔贝蒂娜妒意大发,而是为了在她眼中提高我自己的威望,或者至少在告诉阿尔贝蒂娜我爱的是她,而不是安德烈时,不会降低自己的威信。这样的话,我也不对安德烈说,担心她会在阿尔贝蒂娜面前学舌。我与安德烈谈起阿尔贝蒂娜时,故作冷漠。我上了她表面轻信的当,她对我的故作冷漠恐怕不会上当。她佯装相信我对阿尔贝蒂娜无动于衷,佯装希望阿尔贝蒂娜与我完美结合。实际上很可能正相反,她既不相信我对阿尔贝蒂娜无动于衷,也不希望我与阿尔贝蒂娜完美结合。在我对她说我并不将她的女友放在心上时,我的心里只想着一件事,那就是极力与邦当太太搭上关系。邦当太太在巴尔贝克附近小住几天,阿尔贝蒂娜大概很快要去她家过上三天。当然,我不叫安德烈看出这个欲望,我与她谈起阿尔贝蒂娜的家庭时,是毫不在意的神情。安德烈那些明确的回答,倒显不出她对我的诚恳有所怀疑。可是有一天,她对我冒出一句:“我正好看见了阿尔贝蒂娜的姨母。”这是为什么呢?当然,她并没有对我说:“你那些似乎偶然说出的话,我理出个头绪来了,我知道你一心想与阿尔贝蒂娜的姨母拉上关系。”但在安德烈的头脑中,显然有这个想法,她认为向我隐瞒这个想法更好一些,而“正好”这个词似乎就是与这个想法相联系的。有些眼神,有些动作,虽然没有逻辑的、理性的形式,没有直接为听话人的智力而规划的形式,但是这些眼神和动作会叫他理会到其真正的含义,正像人的语言在电话中先转变为电,然后又转化为语言为人所听见一样。这个“正好”就属于这一家族。为了从安德烈的头脑中抹去我对邦当太太感兴趣的想法,我再谈到这位太太时,不仅心不在焉,而且还带有恶意。我说从前曾经见过这类疯女人,但愿以后不再遇到这种事。实际上正好相反,我千方百计要与她见面。
\par 我极力要埃尔斯蒂尔同意在邦当太太面前谈起我,并且要我与她见一次面。但我没有对任何人说我求埃尔斯蒂尔办这件事。埃尔斯蒂尔答应让我与她相识,但对我希望做这件事大惑不解,他认为这位太太是一个可鄙的、专门搞鬼的、既没有趣味又贪图钱财的女人。我想到,如果我见邦当太太,安德烈早晚要知道,所以我想最好还是提醒她一下。
\par “什么事,你越想躲,越躲不开,”我对她说,“世界上再没有比与邦当太太见面更叫我腻味的事了。可是,我逃不过这一关。埃尔斯蒂尔大概要跟她一块请我。”
\par “对这事我一刻也未怀疑过,”安德烈大叫起来,语气酸楚,因不满而张大的失神的眼睛,直勾勾望着什么看不见的东西。安德烈的这些话还构不成对一个念头的条理清楚的表述,这个念头可以概括如下:“我清清楚楚知道你爱阿尔贝蒂娜,你千方百计要接近她的家庭。”而她的话是这个念头不成型的、可以重新拼凑起来的碎屑。我触动了这个想法,让它暴露出来了,安德烈并非有意如此。就像我们刚才说的“正好”一样,这些话只在第二层才有含义。有些话(而不是直接的肯定)使我们对某个人产生敬重或戒心,使我们与这个人格格不入。安德烈的话即属于这一类。
\par 我对安德烈说,我对阿尔贝蒂娜的家庭无所谓,安德烈没有相信我的话,这是因为她以为我爱阿尔贝蒂娜。很可能她为此感到不快。
\par 一般来说我与她的女友约会时,她总是以第三者身份在场。然而也有的日子我得见阿尔贝蒂娜一个人。我在狂热中等待着这样的日子。这些时间渐渐过去,并没有给我带来任何决定性的东西,也没有成为我立即将其作用委托给第二天的那种重大的日子,第二天也不比前一天更起什么作用。日子就这样一天一天地过去,好似后浪推前浪的海浪。
\par 从我们玩环坐猜物游戏那天算起,大约过了一个月,有人对我说,阿尔贝蒂娜第二天早晨要动身到邦当太太家去度过四十八小时。她不得不坐早班车走,所以头天晚上要住在大旅社,这样,第二天早晨她可以从旅馆坐公共马车去赶头班火车,不致打扰她寄居的人家的朋友。我与安德烈谈起这件事。
\par “我一点也不相信,”安德烈回答我说,满脸的不高兴,“再说,这也不会使你有什么进展。我可以肯定,如果阿尔贝蒂娜一个人到旅馆来,她根本就不想见你。这不合乎礼节,”她又加上一句,使用了最近她非常喜欢的一个名词,那意思是“这种事情是做不得的”,“我对你这么说,因为我了解阿尔贝蒂娜的想法。至于我个人,你见她与否,关我什么事?这跟我毫无关系。”
\par 这时奥克塔夫遇上了我们。他轻松地告诉安德烈,他前一天在高尔夫球场上打了多少分,阿尔贝蒂娜打了多少分。阿尔贝蒂娜一面散步,一面像修女摆弄自己的念珠一样摆弄着她的球拍。幸亏有这种游戏,她可以独自一人待上几小时而不会厌烦。她一来和我们聚在一起,那调皮的鼻子尖就出现在我面前,这几天我想到她时,倒把她这调皮的小鼻子尖忘却了。她那深色头发下,前额笔直,与我保留的不准确的形象形成鲜明对照,这已不是第一次了。眉宇间白皙的皮肤,又紧紧吸引住我的目光。阿尔贝蒂娜从回忆的灰尘中走了出来,在我面前重现。
\par 玩高尔夫球使人习惯于独处的乐趣。球拍带来的乐趣肯定也是如此。阿尔贝蒂娜遇上我们以后,一面与我们聊天,一面继续玩球,就像一位妇女,她的女友来看望她,她并不因此就停下手中钩的活计一样。
\par “据说德·维尔巴里西斯太太向你父亲提出了抗议,”她对奥克塔夫说(我从“据说”二字听到了阿尔贝蒂娜特有的一个音符。每次我发现自己已将这些音符遗忘时,同时便想起在这些音符后面,曾依稀见过阿尔贝蒂娜那决断而又法西兰式的面部表情。即使我是盲人,也能从这些音符里和她的鼻子尖上认出她的某些机灵而又有外省味道的特点来。音符和鼻子尖都很有价值,说不定能够相辅相成,而她的嗓音又像未来的电视电话那样在声音里清楚地显现出视觉形象来),“她不只是给你的父亲写了信,同时还给巴尔贝克市长写了信,叫人在海堤上再不要玩马球,因为一个马球落到了她脸上。”
\par “对,我听人说到这个抗议。这很可笑。这里已经没有多少消遣。”
\par 安德烈没有插言,她不认识德·维尔巴里西斯夫人,其实阿尔贝蒂娜和奥克塔夫也不认识德·维尔巴里西斯夫人。
\par “不知道这位太太为何要如此小题大作,”安德烈还是开了口,“德·康布尔梅老太太脸上也挨了一球,她并没有提出抗议嘛!”
\par “我给你解释一下这二者的差别,”奥克塔夫表情严肃地一面搓着一根火柴棍一面答道,“这是因为在我看来,德·康布尔梅太太是一个交际花,而德·维尔巴里西斯夫人则是一个暴发户。你们今天下午去不去打高尔夫球?”说着他便离开了我们。安德烈也走了。
\par 我单独与阿尔贝蒂娜留下来。
\par “你瞧,”她对我说,“现在我照你喜欢的样子弄我的头发了,看看我这绺头发!没有人不嘲笑这个,可是没有一个人知道我这是为了谁。我的姨母肯定也要嘲笑我的。当然我也不会告诉她这是什么原因。”
\par 我从侧面望着阿尔贝蒂娜的双颊。她的双颊常常显得苍白,但是,这样,便得到浅色血液的浇灌,那血液照亮了双颊,使它闪闪发光。某些冬日的清晨也这样闪闪发光,局部被阳光照耀的石头仿佛粉红色的大理石,放射出快乐的光芒。此刻,看到阿尔贝蒂娜的双颊也给予我极大的快乐。不过这快乐导向另一种欲望,不是想去散步,而是想亲吻。
\par 我问她,人家说的那些计划是否属实。
\par “对,”她对我说,“今晚我在你那个旅馆过夜。我有点感冒,甚至晚饭前我就要上床。你可以到我床边来看我吃晚饭,然后咱们玩一会。你想玩什么,咱们就玩什么。如果你明天早晨到车站来,我会非常高兴。不过我怕这会显得莫名其妙,我说的不是安德烈,她很聪明;我说的是别的去车站的人。有人告诉我姨母,又会成为闲话。但是我们可以一起度过今天晚上。这个,我姨母一点也不会知道。我去向安德烈告别。好,一会儿见。早点来,咱们时间好多一点。”她又微微一笑补充一句。
\par 听到这些话语,我又回到爱希尔贝特以前的时代,回到我觉得爱情似乎不仅是一个外在的整体,而且可以实现的那个时代。我在香榭丽舍大街看到的希尔贝特,与我独自一人时在我心中重现的希尔贝特完全不同。骤然间,想象的阿尔贝蒂娜,当我还不认识她的时候,我自认为在海堤上偷偷望着我的阿尔贝蒂娜,见我远去现出不心甘情愿回家神情的阿尔贝蒂娜,化成了真正的阿尔贝蒂娜,我每天见到的阿尔贝蒂娜。我原来还以为她充满资产阶级偏见,对她的姨母特别坦白呢!
\par 我去与外祖母一起用晚餐,感到自己心中有一桩她不了解的秘密。同样,对阿尔贝蒂娜来说,明天她的女友们与她在一起,也不知道在我们之间刚刚发生的事。当邦当太太吻她甥女的额角时,她根本不会知道在她们两人之间还有一个我,甥女头发梳成那个式样,是为了讨我喜欢,而这个目的对所有的人都是秘而不宣的。直到那时为止,我是那样羡慕邦当太太,因为她的亲戚也是她甥女的亲戚;她为什么人戴孝,她甥女也为什么人戴孝;她到什么亲戚家走动,她甥女也要到什么亲戚家走动。碰巧对阿尔贝蒂娜而言,我胜过她姨母本人。在她姨母身边时,她思念的会是我。过一会会发生什么事情,我不大清楚。总而言之,这大旅社,这夜晚,在我看来已不再空荡荡,它们蕴含着我的幸福。
\par 我打铃叫来开电梯的人,以便上楼到阿尔贝蒂娜开的房间去。房间是在山谷一侧。任何细小的动作,例如坐在电梯里的长凳上之类,我都觉得那么甘甜,都与我的心息息相通。电梯借以上升的缆绳,走出电梯后还要迈上的几级台阶,在我眼中,只是我的欢乐物化成了齿轮和阶梯。在这条走廊里,我再走上两三步,就到了那个房间,那玫瑰色的身体宝贵的精华就藏在那房间之中。那个房间,即使会有甜美的事情在其中发生,过后仍会保持常态,对于不晓得内情的过客,这房间仍与其他所有的房间无异。所有这些房间都将其中的物件变成了死不开口的见证,谨慎小心的心腹,神圣不可侵犯的快乐保管员。从楼梯口到阿尔贝蒂娜房间的这几步,任何人再也无法阻止的这几步,我满怀快乐、小心翼翼地走过去,仿佛投身于一个崭新的环境中,似乎我每前进一步,都在缓缓地移动着幸福,同时又有一种从未体验过的强大无比的感觉,感到自己终于进入了本来一直就应该属于我的遗产之中。
\par 然后,我忽然想到,我不该有什么怀疑,她要我待她上床之后前来的。这再明白不过了,我高兴得直跺脚。路上碰见弗朗索瓦丝,差点把她撞倒。我双眸发亮向女友的房间跑去。
\par 我见阿尔贝蒂娜躺在床上。白衬衣展露出她的脖颈,改变了她面庞的比例。也许是床,也许是感冒,也许是晚餐使她的面孔更加充血,更加显得艳如桃李。我想到几小时之前在海堤上我见到的面色,现在终于就要知晓这秀色是什么味道了。她那两条乌黑、鬈曲的长辫,为讨我喜欢,已经完全解开,其中一条从上到下穿过面颊。她微笑着望着我。她身旁,窗户里,皎洁的月光照亮了山谷。见到阿尔贝蒂娜裸露的脖颈和那胜过玫瑰的面颊,叫我那样如醉如痴(也就是说,对我而言,现实世界再不是在大自然之中,而是投入了感觉的激流中,我几乎控制不住),这一见便完全打破了在我体内运行的那个偌大、坚不可摧的生命与相比之下那样弱不禁风的宇宙生命之间的平衡。从窗户上,我依稀望见山谷旁边的大海,梅恩维尔最高几处悬崖那隆起的乳房,月亮尚未升到中天的夜空。比起我双眸四周的绒毛来,我似乎觉得这一切扛起来都更轻一些。我感到上下眼皮之间的绒毛已经膨胀起来,坚固结实,准备在其柔嫩的表面上举起许多其他重物,全世界的高山峻岭。地平线这半球本身再也不足以填满这绒毛天体了。与胀满我胸膛的这深深吸上的一口气相比,造物主所能给我带来的全部生命,在我看来已非常微弱,大海的呼吸在我看来已显得那样短促。我向阿尔贝蒂娜俯下身去,想拥抱她。此刻,就是死亡向我袭来,我也会毫不在乎。更确切地说,我觉得那不可能,因为生命不在我身外,而在我身内。此时如果有一位哲学家,阐述他的思想,说有一天,哪怕是遥远的一天,我也要死去;大自然永恒的力量则仍会存活下去,在这大自然力量神圣的脚下,我只不过是一粒尘埃;我死后,这些圆形的、隆起的悬崖,这大海,这月光,这天空还会在,我对他一定发出怜悯的一笑!这怎么可能呢?世界怎么能比我存在得更久,既然我并没有迷失在世界之中,而是世界锁在我心中,世界远远不能充满我的心房,我感到自己心中还有位置,可以容得下许许多多别的珍宝,我会充满蔑视地将天空、大海和悬崖扔在一个角落里。
\par “快收场,不然我可打铃了!”阿尔贝蒂娜见我向她扑去要亲吻她,大叫起来。
\par 但在我看来,一个少女叫一个小伙子偷偷前来,安排得叫她的姨妈不知不晓,肯定不是为了什么事都不干;善于抓住时机的人,只要有胆量,就能成功。我当时处于那么激动的状态之中,阿尔贝蒂娜那圆圆的面庞,为内心的火焰所照亮,仿佛被通宵点燃的小灯所照亮,对我来说,是那样有立体感,以致在我看来它在模仿地球仪的转动而转动,如同米开朗琪罗的群像为静止不动而又令人头晕目眩的旋风所卷走一般。\footnote{此处系指西斯廷教堂穹顶上米开朗琪罗所绘制之《创世纪》组画。}这个从未品尝过的粉红色果子,闻起来是什么味,吃起来是什么味,我马上就会知晓!就在这时,我听到急促、延续而又刺耳的声响。阿尔贝蒂娜已经使足全身力气拉了铃。
\par 从前我一直认为,我对阿尔贝蒂娜的爱情并不建立在对肉体占有的希冀上。但是,那天晚上的尝试所得到的结果,便是在我看来这种占有已不可能。第一天在海堤上见到她,我就曾怀疑她是放荡的女子,后来又经过中间的各种假设,我似乎已最终确认她是绝对洁白如玉的。一星期以后,她从自己姨母家回来之后,冷冷地对我说:“我原谅你了,甚至为叫你难过而感到后悔。可是,永远不要再做那种事了!”这倒与布洛克对我说的可以把任何女人搞到手完全相反。似乎我见到的不是一个真正有血有肉的少女,而是一个蜡制玩具娃娃。
\par 此后,我那种要进入她的生活之中,要跟随她到她度过童年的国度去,要由她启蒙开始生活的欲望便渐渐与她疏远了。思想上极力想知道她对某件事有何想法的那种迫切心情,也没有比相信我能够亲吻她这种信念活得更长久。对占有的希冀一旦停止向我的幻梦提供食粮,我的幻梦就放弃了她。而我从前一直认为这幻梦是独立于对占有的希冀之外的。从此,这些幻梦又恢复了自由,转移到阿尔贝蒂娜的这位或那位女友身上去,首先是安德烈身上——视某一日我在哪一位女友身上寻到的魅力,尤其是我依稀望见的为她所垂青的可能性与机遇如何而定。不过,即使没有和阿尔贝蒂娜这一段瓜葛,此后的日子里,对于安德烈对我表现出的热心,我大概也不会越来越高兴。我在阿尔贝蒂娜那里碰上的钉子,她没对任何人讲过。有些俏丽女郎,一进入豆蔻年华,总是能比姿色与富有程度超过她们的女子更招人喜爱——在家中,在朋友中,在交际场中都是如此。这当然是由于她们姿色动人,但更重要的是由于她们拥有相当神秘地令人快乐、令人着迷的魅力——其源泉可能在于她们有无穷无尽的生命力,没有受到造物主如此垂青的人则到她们这里来解除干渴。阿尔贝蒂娜便属于这种人。有些少女,尚未到恋爱年龄——到了恋爱年龄就更甚之——人家就向她们索取比她们自己的要求多得多的东西,甚至是她们无法给予的东西。她也属于这种人。阿尔贝蒂娜从童年时代起,跟前就有四五个小伙伴对她佩服得五体投地。其中就有安德烈,而安德烈比她出类拔萃得多,安德烈自己也清楚知道这一点(说不定正是阿尔贝蒂娜这种完全无意间对人产生的吸引力帮了她的忙,成为构成这一小帮子人的根由)。
\par 这种吸引力甚至作用到相当远的地方,一直达到相对而言更引人注目的一些阶层:如果要跳孔雀舞\footnote{十六世纪时在法国和西班牙很盛行的一种舞蹈。},他们宁愿请阿尔贝蒂娜去,而不是请一位出身高贵的少女。结果是,虽然她毫无分文做嫁妆,依靠邦当先生过活,日子过得很清苦,人都说这位邦当先生心术不正,又一心想甩掉她,但是不仅有人邀请她进晚餐,而且有人邀她住在自己家里,这些邀请阿尔贝蒂娜的人在圣卢眼中,大概是没有一丝光彩的,但在罗丝蒙德或安德烈的母亲看来——她们也是很有钱的妇女,但是她们不认识这些人——这些人已经代表着很了不得的势力了。就这样,阿尔贝蒂娜每年都在法兰西银行一位总裁、一个大铁路公司管理委员会主任的家中度过几个星期。金融巨头的妻子接待一些很重要的人物,却从来未告诉过安德烈的母亲哪一天是她的“接待日”。安德烈的母亲觉得这个女人甚是无礼,但是对她家发生的一切事情仍然怀着极大的兴趣。她每年都鼓动安德烈把阿尔贝蒂娜请到他们的别墅中来,因为据她说,向一个自己无钱旅行、自己的姨母又对她不加照管的姑娘提供在海滨小住的机会,这是善举。
\par 安德烈的母亲很可能并非出于这样的动机:希望银行总裁及其妻子得悉她和女儿对阿尔贝蒂娜爱如掌上明珠,因此会对她们母女产生好感。她也更不会指望那么善良而又正直的阿尔贝蒂娜会叫人邀请她,或者至少邀请安德烈去出席金融家的花园晚会。每天晚上进餐时,她一面作出轻蔑和毫不在意的模样,一面津津有味地听着阿尔贝蒂娜向她叙述自己在金融家的城堡中生活时那里发生的事,那里接待的人等等。这些人,她几乎全都目睹或耳闻过。甚至想到阿尔贝蒂娜只是以这种方式认识那些人,也就是说,并不了解这些人(她把这叫作认识“各朝各代”的人),也使安德烈的母亲感到一丝忧伤,她露出高傲和心不在焉的神情,轻蔑地就这些人向阿尔贝蒂娜提出一些问题。若不是她对家中总管说“请你对厨子说,这豌豆没烧烂”这句话,从而肯定了自己的地位,而且重新置身于“现实生活之中”的话,这位夫人对自己的重要地位可能要把握不住并且焦虑不安了。说了这句话以后,这位太太又恢复了平静。她早下定决心非叫安德烈嫁个人不可。这个人自然要出身高贵,同时又要相当富有,以使安德烈也能拥有一个厨子和两名车夫。有地位,实实在在的东西就是这个。但阿尔贝蒂娜在银行总裁的城堡中与某某太太共进晚餐,这位太太甚至邀请她去过下一个冬季,在安德烈母亲眼中,这都不能不叫人对这个少女肃然起敬。这种肃然起敬与她身遭厄运而引起的怜悯之情甚至蔑视,正好交织在一起。由于邦当先生背叛自己原来的旗帜投向内阁一边——据隐隐约约的传闻他是巴拿马分子——这种蔑视就更加变本加厉。但是,这也挡不住安德烈母亲出于热爱真相,对那些似乎认为阿尔贝蒂娜出身下贱的人不屑一顾。
\par “怎么?人家出身再好不过了,人家姓西莫内,只有一个‘n’!”
\par 自然,这一切事情发生在金钱起着那么重要作用的阶层。在这个阶层中,风姿绰约可以叫人对你发出邀请,却不能叫人娶你为妻。阿尔贝蒂娜虽然受到如此特殊的厚爱,这厚爱并不足以补偿她的贫寒。这种厚爱的有益后果,对阿尔贝蒂娜来说,似乎绝不会是一桩“过得去的”婚事。这样的“出风头”,即使不能带来成就婚烟的希望,也已激起某些心怀恶意的母亲的妒羡。她们见银行总裁的妻子,甚至安德烈的母亲,将阿尔贝蒂娜当做“自家孩子”来接待,而她们自己几乎不认识这两位太太,一个个气得要死。于是,她们向她们自己共同的朋友以及这两位太太共同的朋友说,这两位太太如果得知事情真相,一定会怒火满腔。那真相便是阿尔贝蒂娜在这家(“反过来亦然”)讲了在那家的一切发现,人们不慎十分亲密地接待她,便使她有了这些发现。这千百种小小的秘密,当事者见到被揭露出来,是很不舒服的。这些嫉妒心重的妇人道出这些话语,目的便是希望有人去传话,好叫阿尔贝蒂娜与她的保护人之间产生不和。但是像常常发生的那样,托人办这种事,一点也没办成。主使他们干这些事的恶意动机,人们感觉太明显了,结果只会使人更加蔑视打这种主意的女人。安德烈的母亲对阿尔贝蒂娜的看法早已固定,不会改变。她把阿尔贝蒂娜视为一个“可怜的孩子”,天性善良,只会想出各种名堂来叫人喜欢。
\par 阿尔贝蒂娜这样风靡一时,看上去并不包含任何实实在在的结果,倒使安德烈的这个女友形成了某些人的那种特性。这些人一向成为别人追求的目标,从来不需要自己主动送上门(由于相同的原因,这种性格在社会的另一极端,即某些风姿绰约的女性身上,也可以见到),但她们从不把别人对她们的追求拿来夸耀,更确切地说,她们总是把这些隐瞒起来。谈到某某时,她从来不说:“他很想见我。”谈到任何人,都怀着极大的善意,似乎追求别人的是她。一个小伙子几分钟之前与她面对面谈话,因她拒绝与他约会而对她大肆谴责。谈起这个小伙子的时候,她不但不以此当众吹嘘或责怪他,反而称赞他说:“这个小伙子真热情!”她甚至为自己如此讨人喜欢而感到烦恼,因为这样她势必要惹人难过,她的天性却是喜欢叫人高兴。
\par 她喜欢叫人高兴,甚至达到使用某些只求实利的人和某些爬上高位的人所特有的那些谎言的地步。这种不诚恳,其实在很多人身上都以雏形状态存在着,其内容便是不善于以办一件事只叫一个人高兴为满足。例如,如果阿尔贝蒂娜的姨妈希望她的甥女陪她去出席一次并不好玩的白日聚会,阿尔贝蒂娜去了,她本应该以得到叫自己的姨母高兴这种精神收获而感到满足的。但是,当她受到聚会的主人热情接待时,她更喜欢对他们说,她早就想与他们见面,因此选定这个机会并征得姨母同意而前来。这还不够:这次聚会上,有阿尔贝蒂娜的一个女友,正好刚刚失恋。阿尔贝蒂娜还要对她说:“我不愿意让你一个人孤孤单单的,我想到我在你身边,可能你会好过些。如果你希望咱们离开这聚会,到别处去,你说怎样,我就怎样,最重要的,是我希望看到你情绪好一些。”(再说,这也是真话。)
\par 有时,假目的毁了真目的。阿尔贝蒂娜为她的一个女友要去求别人办件事,为此前去看望某夫人,情形就是如此。一到这位善良而又热情的太太家里,这位姑娘不知不觉地遵循自己“一事多用”的原则,觉得如果作出纯粹是因为自己感到见到这位太太会多么高兴才前来的样子,就更热乎一些。这位太太见阿尔贝蒂娜纯粹出于友谊这样长途跋涉而来,真是无比感动。阿尔贝蒂娜见这位太太几乎被感动了,便更加喜欢她。可是问题出在这里:她谎称自己纯粹出于友情动身前来,她那样强烈地感受到友情的快乐,如果她为自己的朋友请求这位太太帮忙,反倒担心会叫这位太太怀疑她的感情了。事实上,她是真心实意的。那位太太会以为,阿尔贝蒂娜是为这件事来的,这倒是实情;但她会得出结论说,阿尔贝蒂娜见了她高兴,并非没有利害得失考虑。这倒不确切。结果是阿尔贝蒂娜没有提出要求帮忙便走了。这与那些对一个女人极其殷勤周到,指望得到她的青睐,但是为了使这种热情保持高尚的性质,便不向女人表白自己的爱情的男人情形相似。
\par 在其他情形中,倒也不能说,她总是为了次要的、事后想出的目的而牺牲真正的目的。但是真正的目的与次要的目标针锋相对,如果阿尔贝蒂娜向那个人道明了一个目的,使之大受感动,而当她也得知另一个目的时,她的快乐立刻会变成最深沉的痛苦。下面的故事讲下去,会叫人更加明白这类矛盾之所在。
\par 我们借一个与此完全不属于同类型的例子,可以说明在生活所呈现的五花八门的情形中,这类矛盾比比皆是。一个丈夫将情妇安顿在自己驻防的城市里。他的妻子留在巴黎,对事情真相有所耳闻,很难过,给丈夫写了几封充满妒意的信。正好情妇不得不到巴黎来一天。情妇求他陪同前往,这位丈夫抵挡不住,于是请准了二十四小时的假。可是,他心眼很好,因自己使妻子难过而感到愧疚,到巴黎以后便去妻子那里,流着真诚的眼泪对她说,读了她的信自己真是心乱如麻,设法逃出一天以便前来安慰她、拥抱她。这样,他就想到了办法,用一次旅行同时向情妇和向妻子证明了爱情。但是,如果他的妻子得知他来巴黎的真正原因,她的快乐肯定会变成痛苦,除非这个忘恩负义的家伙不管怎么说,使她感到的幸福胜于用谎言给她带来的痛苦。
\par 依我看,一贯使用这种“目的多用”体系的人,应首推德·诺布瓦先生。有时他接受在两个发生龃龉的朋友之间进行调停的任务,以获得“最热心的人”这个美名。在前来请他帮忙的人面前,他作出热心相助的姿态还嫌不够,在另一方面前,他还要将自己进行斡旋说成并非因前者的请求而干,而是出于对后者的利害考虑。这样他便轻而易举地说服了对方,事先向对方作出了暗示,说明站在他面前的是“最肯帮忙的人”。这样,他两面讨好,干着用行话称之为“里外光”的事,他的声望不会冒任何风险。实际上他所帮的忙,并不构成什么割让,相反,却构成他的一部分威望结出的硕果。另一方面,他帮的每一个忙,似乎都对双方有益,这就使他“肯帮忙的友人”的名声更增加一分,而且是极有成效的“肯帮忙的友人”,并不是抽刀断水,而是每一次斡旋都有成效。这表明双方当事人对他都感激不尽。这种热心相助中的口是心非,再加上任何人身上都有的种种矛盾,是德·诺布瓦先生性格的一个重要组成部分。在内阁中,他常常一面利用我父亲,一面还叫我父亲相信他是为我父亲效力。我父亲相当幼稚,也就轻易信以为真。
\par 阿尔贝蒂娜比她自己希望的更讨人喜欢,她不需要对自己的情场得意大吹大擂。对于在她床边发生的、我与她之间的那一幕,她始终守口如瓶。如果是一个丑八怪,恐怕要让全世界都知道了。她在这一幕中的态度,我始终不得其解。对于她绝对贞洁这种假设(阿尔贝蒂娜那么粗暴地拒绝让我亲吻,拒绝让我得到她的肉体,我首先归结为这样的假设。但就我对自己女友的善良、基本正直的观念而言,这种绝对贞洁绝非必不可少),我不得不反复揣测多次。这种假设,与我第一天见到阿尔贝蒂娜时作出的假设,是那样截然相反!其次,为了逃脱我,她拉了铃。这个粗暴的动作四周,又环绕着那么多与此截然不同的行动,对我均为热情倍加的行动(抚慰性的,有时是焦虑不安的,警觉性的,嫉妒我偏爱安德烈等等)。为什么她要我前去,在她床边度过晚上的时光?为什么她一直使用柔情的语言?想见一个男友,担心他喜欢你的女友胜过喜欢你,设法讨他欢喜,浪漫地对他说别人不会知道他在你身边度过晚上的时光,可是你又拒绝给他这么简单的快乐。如果对你来说,这不是一种快乐,那么,这种种欲望又以何为依托?无论如何,我不会相信阿尔贝蒂娜的女性贞洁竟会达到这种地步。所以我又自忖,是否她的粗暴之中,有些搔首弄姿的缘由,例如,可能她觉得自己身上有一股难闻的气味,怕我不喜欢;或者是胆怯,例如,她对情爱的真实情形完全无知,以为我的神经衰弱症状也会通过亲吻而得以传染呢?
\par 她肯定因未能叫我快活而悔恨,便送我一支烫金铅笔。有的人为你的热情所感动,但是不同意将你的热情所索取的东西给予你,却同意为你办其他的事,例如批评家本该写文抬举小说家,却邀请小说家在广场上用晚餐;公爵夫人则并不亲自把纨绔子弟带到剧院去,而是哪天晚上自己不占那个包厢时才叫他去!做得越少,且可以什么都不干的人,谨慎小心却推着他们去干出什么事情!阿尔贝蒂娜送我一支烫金铅笔就是这种美德心理的反常行为!我对她说,她送我这支铅笔,叫我很高兴。但是与她来旅馆过夜那天晚上,如果她允许我亲吻她,我会得到的快乐相比,这种高兴便大大逊色了。
\par “那该叫我多么快活!对你又有什么坏处呢?你拒绝了我,我真是奇怪。”
\par “使我奇怪的,”她回答我道,“是你竟觉得这事令人奇怪。真不知道你过去都见识过什么样的姑娘,以致我的行为才会使你感到奇怪。”
\par “叫你不快,我深感歉疚。但是,即使是现在,我也不能对你说,我认为自己错了。我的看法是这些事无关紧要,我不明白,一个能够轻而易举使人快乐的姑娘,竟拒绝这样做。咱们说好了,”我又加上一句,为的是叫她那些道德观念得到一半满足,同时也回忆起她和她的女友们是怎样鞭挞女演员莱亚的女友的,“我的意思并不是说,一个少女可以什么事都干,没有任何不道德的事。你听着,有一天你对我谈到住在巴尔贝克的一个小女孩,谈到她与一个女演员之间的那种关系。我认为这种关系太丢人了,太丢人了,以至于我认为是少女的敌手编造出来的,并非真有此事。我认为那不大可能,不可能。但是任凭一位男友拥抱,甚至更有甚者,既然你说我是你的朋友……”
\par “你是我的朋友,但是在你之前,我也有过别的朋友。我见识过一些小伙子,我向你保证,他们对我有着同样的友情。可是,没有一个人敢这么干。他们知道,如果这么干,头上会挨上两巴掌。再说,可能他们连想也没这么想,大家就是很直截了当地,很友好地,作为好伙伴,握握手。从来没有人说过拥抱的事,可是并没有因此降低友情。好啦,你看重我的友情的话,你就会满意,我肯定相当喜欢你才会饶恕你。不过我可以肯定,你不会把我放在心上。请你承认,讨你喜欢的是安德烈。归根结底,你说得对,她比我热情得多,她又那么叫人心醉神迷!啊,男人们!”
\par 我最近虽然非常失望,阿尔贝蒂娜如此坦率的一番话,倒叫我对她敬重万分,给我留下十分良好的印象。说不定这种印象此后对我产生了巨大而不良的后果,因为从这个印象开始,形成了那种几乎亲切的情感、那种道德的内核,在我对阿尔贝蒂娜的爱情中,这种情感和内核一直持续存在。这种情感可以成为最大痛苦的根源。因为要真正为一个女人而忍受痛苦的折磨,必须首先对她完全信任不可。目前,这个道德、敬重、友情的雏形,在我的心中仍像一块石头一般留在那里。如果它就这样停留下去,不再增长,像第二年,甚至像我初次在巴尔贝克小住的最后几个星期那样保持着其毫无生气的状态,只这一个因素,对我的幸福是丝毫不会起到破坏作用的。有些客人,无论如何,较为谨慎的办法还是将他们赶走,但是人们让他们留在原地,不去招惹他们,他们的弱点,是在一个陌生的心灵中感到孤独,这已经使他们暂时不会伤害人了。上述这种情感在我心中,就好像这样的一位客人。
\par 现在,我的幻梦重又可以自由自在地落在阿尔贝蒂娜的这个或那个女友身上,首先是安德烈身上了。安德烈对我的热情是否会被阿尔贝蒂娜得知,如果我对这一点没有把握,她的热情可能就不会那么叫我感动了。当然,长期以来我佯装偏爱安德烈,交谈习惯,表白柔情的习惯,为我对她现成的爱情提供了材料。迄今为止,只缺一样,那就是加上点诚挚的情感。现在我的心又自由了,可以提供这种诚挚的情感。可是,安德烈聪明过分,神经过敏,过分病态,与我过于相像,我不会爱她。如果说我现在感到阿尔贝蒂娜似乎过于空虚,安德烈则充满了某种我过分熟悉的东西。第一天,在海堤上,我本来以为见到的是自行车运动员的情妇,沉醉于对体育运动的爱好之中。可是安德烈对我说,她之所以从事运动,乃是遵从医嘱,为的是治疗她的神经衰弱和营养紊乱,而她最美好的时光是翻译乔治·艾略特的一本小说。对于安德烈是什么样的人,我从一开始就大错特错了。结果是我很失望,事实上,这种失望对我无关紧要。这个错误属于这样的类型:虽然这样的错误仍可以允许爱情产生,但是,只有在爱情再也无法改变时,这样的错误才会为人所承认,因而也就成为痛苦的根源之一。这种错误——可以与我在安德烈的问题上所犯的错误很不相同,甚至相反——尤其是就安德烈而言,常常是由于相当看重外表,希望如此而实际上并非如此的举止,以致第一次接触便产生了幻想。不论是好人还是坏人,除了他们的外表,装腔作势,模仿他人,希望为人欣赏以外,还要加上言谈、举止的假象。有些厚颜无耻的人,残忍的人,也不比某些善良的人,讲义气的人更能经受得住这种考验。同样,人们常常会发现一个以慈善闻名的人原来是一个虚荣的吝啬鬼,一个老老实实、观念正统的女孩竟是梅萨琳娜\footnote{梅萨琳娜为古罗马皇帝克罗德的第五个妻子,以荒淫、残暴、奢侈而著名。}式的人物。我本来以为安德烈是健康而单纯的姑娘,实际上她只不过是一个寻求健康的人。安德烈认为许多人是健康的,事实并非如此,正如一个肥胖粗大、面孔通红、身穿白色法兰绒上衣的关节病患者并不一定就是大力士一样。因为某人显示出来的健康而爱上了他,而他事实上只不过是一个病人。这种病人只从别人身上得到健康,就像某些星球借其他发光星体的光以及某些物体只容电流通过一样。有些情况下,这种情形对幸福并不是无关紧要的。
\par 这些都无关紧要。像罗斯蒙德和希塞尔一样,安德烈毕竟是阿尔贝蒂娜的女友,甚至胜过罗斯蒙德和希塞尔,她与阿尔贝蒂娜共享生活,效仿她的举止,以至第一天刚开始时,我分辨不出她们这个与那个来。这些少女是一枝枝玫瑰,其主要魅力是散布在海上,她们之间仍然保持着我与她们尚未相识时的那种不可分离性。那时,她们之中不论哪一位出现,都会叫我那样激动,向我宣告那一小群已经不远。现在依然如此,看见其中一个人,便使我感到快乐。这快乐中含有见到其他人随她出现或过一会来与她会齐的快乐的成份。即使其他人这一天不来,还有谈论她们的快乐,知道别人会告诉她们说我在海堤上的快乐。至于这成份究竟占多大比例,我就说不上来了。
\par 这已经不再单纯是初来时期的那种吸引力,而是真正在爱情上的三心二意,在她们每个人之间犹豫不决,显然她们每个人都可以代替另一个人。我最大的悲哀,并不是这些少女中我最喜欢的一个抛弃了我,而是我无法做到立刻喜欢上哪一个。如果能做到,我倒可以将不清不楚地在所有人身上飘荡的全部忧伤和幻想集中在她一个人身上,即会抛弃我的那个人身上。在这种情况下,是不是在她的所有其他女友眼中,我会立刻威信扫地,是不是我会不知不觉地留恋她的所有其他女友,因为在那之前我对她们怀着一种集体性的爱呢?政治家或演员对公众也怀着这种集体性的爱,他们得到公众的厚爱之后,如果被丢在一边,是无法感到欣慰的。我未能得到阿尔贝蒂娜的青睐,现在,哪一个少女晚上离开我时,对我说上一句模棱两可的话,向我飞过一个意义不明的眼神,我便骤然希望从这个少女那里得到这青睐。借助于这么一句话,这么一个眼神,我的冲动会一整天围着她打转。
\par 在她们那机灵活泼的面庞上,线条刚刚开始相对固定,足以叫人辨认出可塑的、飘忽不定的人像来,哪怕此后还要变。正因为如此,这种冲动就更加带着肉欲成份在她们之间游荡。这些少女的面庞虽然彼此那样不同,倒说不定能够一一重叠起来,她们的面庞长、宽方面的差异,远远比不上五官之间的差异。但我们对面庞的认识是非数学性的。首先,这种认识并非从衡量每一部分开始,而是以某一表情、一个总体印象为出发点。以安德烈为例,温和的双眼、细腻的线条好像与细小的鼻子连接在一起,鼻子窄而细,有如画出来的一条简单的曲线,为的是叫分在双眸中的微笑那高尚的意念能在一条线上得以继续。她的秀发中也画出一条同样的细线,轻盈而幽深,有如风儿在沙上犁过而画出的线条。这一点上,她大概受遗传影响,因为安德烈母亲那满头银丝也完全是如此造型,这里形成一块凸起,那里形成一块凹陷,如同随着地形起伏隆起或下陷的白雪。
\par 自然,与安德烈鼻子那秀气的线条相比,罗斯蒙德的鼻子似乎提供了宽大的平面,有如一座高塔耸立在宽大的底座上。一条无比细小的线条能构成极大的差异,面部表情便足以使人相信这差异是多么大——一条无比细小的线条本身就能构成一个绝然特殊的表情,一个人的个性——使这些面庞显得彼此不会雷同的,还不仅仅是无比细小的线条和表情的特点。在我这些女友的面庞之间,面色构成更深刻的区别,那原因倒也不在面色为面庞提供了丰富多彩的美。罗斯蒙德沉浸在撒了琉粉的玫瑰色中,双眼那发绿的光芒又作用于这玫瑰色。安德烈雪白的双颊从她乌黑的秀发中得到那么多庄重高贵之气。她们的肤色是那样不同,以致我站在罗斯蒙德面前与站在安德烈面前领略到的,是先后凝望生长在阳光普照之海滨的一株绣球与夜色朦胧中的一株茶花时所得到的同样的快乐。肤色构成更深刻的区别,更主要地是因为通过颜色这个新因素,线条之间无比细小的差别,无比扩大,平面的比例完全改变了。这个新的因素与配色器一样,是一个大发生器,或者至少可以说,是一个比例改变器。结果是,可能构造差异不大的面庞,视其为火红的头发、粉红的肤色之火或为不反光的苍白光线所照耀而会变长或变宽,成了另外的面庞,如同俄国芭蕾\footnote{俄国芭蕾于1909年首次赴巴黎演出,普鲁斯特非常欣赏。}的道具,如果白天观看,有时就是简单的一张圆纸片。而巴克斯特\footnote{莱昂·巴克斯特(1866—1924),俄国画家,为《火鸟》(1910),《达夫尼斯和克洛埃》(1912)等设计过布景。普氏与他见过面,对他的才华及和蔼可亲有深刻印象。}这样的天才,视其将布景笼罩在肉红色或月光的光线之下,便可在一座宫殿的正面镶上绿松石,或者使一座花园中孟加拉玫瑰柔和地盛开。我们认识面孔也是这样,我们是以画家身份仔细衡量面孔,而不是以土地测量员身份去衡量的。
\par 阿尔贝蒂娜及其女友们,情形均如此。某些日子,她身材纤弱,面色发灰,神态抑郁,紫色的半透明的光线下她的双眸深处,如同大海有时呈现的颜色,她似乎忍受着放逐者之悲哀。另外的时日,她的面孔更加光滑,放着釉彩的表面粘附着欲望,又防止那欲望走得更远。除非我突然从侧面看她,因为她那无光泽的双颊,就像一支白蜡烛,表面上由于半透明而呈现玫瑰色,真叫人想去亲亲那双颊,去触触这为他人所看不见的不同的肤色。还有的时候,幸福使她的双颊沐浴在那样颤动的明亮之中,以致皮肤变成了流体,变得模糊不清,似乎有日光偷偷地闪过,使皮肤呈现出与双眸不同的另一种颜色,而不是另一种质地。有时,完全出你意料,望着她那撒播着棕色小斑点,又只有两处更显蓝色的痕迹飘浮的面孔,似乎为金翅鸟的卵做成。又常常像是用只在两处加工并磨光的乳白色的玛瑙做成。在棕色宝石中,她的双眸闪闪发光,如同一只天蓝色蝴蝶那透明的双翅。肌肉成了明镜,使我们产生比起身体的其他各部分来,更让我们心灵接近的幻想。更常见的情形,是她面色更鲜艳,于是也更生机勃勃。有时在她白皙的脸上,只有鼻子尖是粉红的。她的鼻子很纤巧,好似一头狡猾的小猫的鼻子,你真想跟那小猫玩耍片刻。有时她的双颊是那样光滑,以致目光在那玫瑰色的珐琅质上滑下去,就像在一个小巧玲珑的艺术品小壶那玫瑰色的珐琅上流淌下去一样。她乌黑的秀发构成半开而又多重的壶盖,使这玫瑰色的珐琅显得更加优雅、内在。有时她的双颊达到仙客来花朵那种粉红带紫的程度。有时她充血或发烧,更使人想到她是病态体质,这使我的欲火下降,成为某种更性感的东西,也使她的目光表现出更邪恶、更不健康的东西。这时她的面色呈现某些红得几乎发黑的玫瑰的那种深紫色。
\par 这样的一个个阿尔贝蒂娜,各不相同,就像一个女舞蹈演员,随着舞台灯光的千变万化,她的色彩、身影和性格不断变化,每次出场都各不相同一样。说不定正因为那个时期我在她身上欣赏到的人物是那样变化多端,后来我也养成了习惯,根据我想到的是哪一个阿尔贝蒂娜,我自己也化成另一个人物:或妒火中烧,或毫不在乎,或追求肉欲,或郁郁寡欢,或怒气发作,不仅仅随着复苏的记忆偶然而至,而且根据我理解同一回忆的不同方式所施加的信念强度去重新创造这些人物。应该反复地谈这个问题,谈这些信念。大部分时候,这些信念在我们不知不觉间填满了我们的心灵,对我们的幸福来说,它比我们看到的某个人本身更为重要,因为我们是通过这些信念来看他的,是这种信念赋予这个被注视的人以转瞬即逝的大小。为了表述得更准确,我大概应该给以后想到阿尔贝蒂娜的每一个我起一个不同的名字,更应该给在我面前出现的每一个阿尔贝蒂娜起一个不同的名字。在我眼前出现的阿尔贝蒂娜,从来不是一个模样,正像接踵而至的各不相同的各种大海——为了更方便起见,我简单地叫它大海,阿尔贝蒂娜是另一个海中仙女,她在大海中轮廓更加清晰地显现出来。更有甚者——以同样的方式,而且据说更为有益,在一处叙事中,提到那一天天气如何——我应该一直将天气这名称交给信念,哪一天我看见阿尔贝蒂娜,哪一种信念笼罩着我的心灵,构成这一天的气氛。人的外表,就像各种各样的大海的外表一样,这些都取决于那些肉眼几乎看不见的云团。这些云团以其集中的情形,流动的情形,撒播的情形,逃遁的情形,改变着每样事物的色彩——正像有一天晚上,埃尔斯蒂尔停下脚步与那些少女谈话,而没有将我介绍给她们,他撕破了一片云,这些少女远去的时候,她们的形象在我眼中骤然显得更加美好一般——过了几天,我与她们相识了,那云团又形成了,遮住了她们的光彩,经常横亘在她们与我的双眼之间,这云团是不透明的、温和的,好似维吉尔笔下的琉科忒亚\footnote{琉科忒亚是底比斯王卡德摩斯的女儿,为航海神,在《奥德赛》中,她救奥德修斯一命,免得他淹死。维吉尔在《埃涅阿斯纪》中提到她,说她专门拯救海上遇难的人。}。
\par 自从这些少女的话语在某种程度上向我指出应该用什么方法去观看她们的面部表情以后,对我来说,无疑她们每个人面孔的意义都改变了很多。我用提问题的方式,按照我的意愿挑起她们的话语,使话语千变万化,就像一个做实验的人通过反证来证明他的假设一样。对这些话语我就可以赋予更高的价值。将从远处看显得优美而神秘的人与事移到近处,便足以使我们意识到这些人与事既无神秘也无优美之处。总的说来,这是解决人生问题的一种方式。在许多种方式中,这也是可以选择的一种有益于健康的方法。这种方法可能不值得特别推荐,但是这会使我们得到某种平静用以度日,用以忍受死亡——这种方法会使我们毫不留恋,使我们确信我们已经接触到最杰出的人与事,而这最杰出也并没有什么了不起。
\par 我原来以为,在这些少女的头脑深处,是蔑视贞洁,并且靠对贞洁的蔑视,回忆日常那些短暂的男女私情过活。现在,我认为在她们头脑深处是正直的原则在起作用了。这些原则可能还会动摇,但是迄今为止防止了那些从他们的布尔乔亚阶层中接受这些原则的女孩走上任何歧路。一个人一开始就误入歧途时,甚至在小事上也是如此。假设错误或记忆错误使你到错误的方向上去寻找某一流言蜚语的制造者或丢失物品的地方时,可能会发生这样的事:发现了谬误,但是并没有用真理去代替,而是用另一谬误去代替。我与她们亲切交谈时,从她们脸上确实见到清白无邪这个词,就这些少女的生活方式和与她们相处的行为而言,我确实体验到这个字眼的全部效果。不过,说不定我观察得丢三落四,解字过于仓促有误,在她们脸上并没有写着这个词,正像我第一次看拉贝玛的日场演出,朱尔·费里\footnote{朱尔·费里(1832—1893)1879年任公共教育部部长,从未写过开场小戏。}的名字并没有写在那次的节目单上,而这并没有妨碍我对德·诺布瓦先生说,朱尔·费里很可能为那次演出写了开场小戏。
\par 既然在我们有关一个人的回忆中,凡是对我们每日发生的关系没有立竿见影的用处的事,头脑一律将其排除(甚至而且特别是如果这些关系还染上一点爱情的话,这爱情从未得到满足,在最近的将来还活着),对于这一小群少女中我的任何一个女友来说,我所见到的最后一张面孔,怎么能不是我回忆的唯一面庞呢?头脑任凭往日组成的链条通过,只死死留住这链条的最后一截。制成这一截的金属常常与消逝在黑夜中和我们人生旅途中的各截链条完全不同。我们的头脑只把我们现在所在的国度当做真实的国度。我最初的印象已经那样遥远,在我的记忆中无法找到什么凭证防止其每天变形。在我与这位少女一起聊天,吃茶点,一起游玩所度过的漫长时光里,我竟然不记得,她们与我从前如同在壁画上见过一般、在大海前列队走过的无情而又肉感的处女是同一批人。
\par 地理学家、考古学家会把我们带到卡利普索岛\footnote{卡利普索岛为仙女卡利普索所居住之岛,她在这里接待了奥德修斯并挽留他十年。}去,会挖掘出米诺斯的宫殿\footnote{普氏此处可能指克诺索斯宫殿。据荷马史诗,这克诺索斯宫殿是米诺斯王国的大城市,伟大的宙斯每隔九年前来,对米诺斯讲述心腹之言。1900年,考古学家阿尔图尔·伊文斯(1851—1941)挖掘出了这座宫殿,神话遂让位于现实。}。只是卡利普索不过是一个女子,米诺斯不过是一个毫无神气息的国王。甚至历史告诉我们的作为这些极为真实的人的特性的长处和短处,也常常与我们赋予那些叫同样姓名的想象中的人物的长处和短处很不相同。我初来乍到那几天创造的优美的大海神话,就这样消失了。但是,至少我们在曾认为不可企及而热烈向往的不拘礼节气氛中度过了一些时光,这是不能等闲视之的。
\par 那些我们开始时觉得别扭的人,在与他们相处中,即使最后在他们身边终于会体验到不自然的、做作的快乐,这快乐之中也始终滞留着他们掩盖住了的缺点的那种掺假的味道。在我与阿尔贝蒂娜及其女友这样的关系之中,构成其根源的真正的快乐,则留下一股馨香。这股馨香,任何人工的办法都无法将它赋予强摘下来的水果,或赋予未曾在阳光下成熟的葡萄。在一段时间内,对我来说,她们是仙女。甚至在我不知不觉中,她们在我与她们之间最普普通通的关系之中,加进了某些奇妙的成份,或者说,她们防止这些关系中有任何平庸的成份。我的欲望那样贪婪地寻找双眸的含义,如今这双眸了解了我并对我微笑,但是第一天,这双眸与我的目光相交时,犹如另一宇宙的光芒。我的欲望那样广袤地、细致周到地将色彩与芳香撒播在这些少女那有血有肉的表面上,她们卧在悬崖上,纯朴地向我递过三明治或者玩猜谜游戏,以致常常一个下午,我躺在那里——就像那些画家,他们要在现代生活中寻找古代的雄伟,赋予正在剪脚趾甲的一个女人以《拔刺的人》\footnote{《拔刺的人》是古希腊时代的铜塑,表现一个小伙子正从脚跟上往外拔刺,为罗马博物馆最美的藏品之一。普鲁斯特肯定在卢浮宫见过其复制品。}那样的高尚,或者像鲁本斯一样,将自己认识的一些女人画成女神\footnote{普氏这里可能指表现玛丽·德·美第奇生活的系列画,因为朱诺、密涅瓦和美惠三女神均簇拥着这位王后。也可能是指一些神话人物画,如《向维纳斯献祭》,画上就有画家自己的妻子出现。}以构成古代神话场面——这些类型很不相同的长着棕发和金发的美丽身躯,在草地上散布在我的周围。我望着这些美丽的身躯,说不定它们并没有去除全部平庸的内涵,日常的体验使她们充满了平庸的内涵,然而(我并没有回忆起她们那天仙般的出身)我却像赫拉克勒斯或忒勒玛科斯一样,似乎正在仙女之中嬉戏。
\par 此后,音乐会结束,坏天气来临,我的女友们离开了巴尔贝克,不是所有的人都像燕子那样一起走,却都在一周之内。阿尔贝蒂娜第一个走了,突然走了,她的哪一个女友无论是当时,还是事后,都没有弄明白为什么她忽然回巴黎去了,既没有功课,也没有什么消遣呼唤她到巴黎去。
\par “她一声不吭就走了。”弗朗索瓦丝嘟嘟哝哝地说。其实,说不定她巴不得我们这样。她觉得我们在旅社的雇员面前和经理面前太不谨慎。雇员数目已大大减少,但仍有极少数顾客留在这里,依然留下一些雇员。经理则“侵吞钱款”。
\par 确实,旅馆很快就要关门,几乎所有的人都走光了。可是旅馆从未这样舒适。当然经理并不这样认为。客厅里,人们冻得发抖,客厅门口再没有一个侍者照应。经理沿着各个大厅,在过道上踱着方步。他身穿崭新的礼服,头发理得那么讲究,那枯燥乏味的脸似乎构成了一个混合体,一份肉大概就有三份化妆品。他不断更换领带(这样摆阔要比保证取暖和保留工作人员少花钱,这就像一个人再也无法为一件善举送上一千法郎,但是还能毫无困难地摆出大方的样子,给前来送电报的电报员一百个苏小费)。他那样子像在视察虚无,似乎要借助于个人的良好衣着,赋予这凄凉景象一种临时性质。在这个时令已经不佳的旅馆里,人们对这凄凉景象感受良深。经理宛若君主再现的幽灵,出没于自己昔日宫殿的废墟之中。这条地方性铁路见旅客不足,已停止运行,直到明年春季才会恢复。经理对此特别不满。
\par “这里缺的就是交通手段。”他经常这么说。
\par 虽然出现了赤字,他仍为今后几年进行宏伟的规划。不论如何,当一些漂亮字眼施用于旅馆业,而且又能使这一行业显得宏伟壮丽时,他还能准确地记住一些。
\par “尽管在餐厅里我有一个优秀班子,我的帮手仍然不够,”他常常说,“穿制服的仆役仍有待改善。明年我会聚集什么样的优秀部队,你们会看到的!”巴尔贝克邮政总局服务中止,使他不得不派人去取信,有时用蹩脚马车去送旅客。我经常要求上车,坐在车夫旁边,这样,不论什么天气,我都可以出去走走,就像在贡布雷度过的那个冬天一样。
\par 有时暴雨如注,游艺场早已关闭,外祖母和我只好留在空荡荡的一些房间里,就像狂风呼啸时,待在船舱尽头一样。与远渡重洋一样,每天在这船舱里,我们在他们身边度过了三个月而并不了解的人当中,会有一个朝我们走来。雷恩的首席审判官呀,冈城的首席律师呀,一位美国太太及其女儿呀,与我们搭搭话,想出点什么花样,让时间不要显得那么漫长,或露出点什么本事,教我们一种玩牌的办法呀,请我们喝茶呀,或请我们弹奏些乐曲呀,请我们某个时刻聚一聚呀,一起设法消遣呀,等等。这些消遣的真正奥秘就是自寻快乐,不要声称烦闷得很,只是互相帮助度过这烦闷的时光。这些人终于在我们小住的末尾与我们结成了友谊。第二天,他们相继离去,又使这友情中断了。
\par 我甚至认识了一个有钱的小伙子,他的两个贵族朋友当中的一个,以及又来住几天的女演员。这个小圈子已经只有三个成员,另一个朋友已经返回巴黎。他们要我和他们一起到他们常去的那家饭馆去用晚餐。我没有接受,我想他们相当高兴。不过他们发出邀请时,是极尽和蔼可亲之能事的。虽然实际上这邀请只来自有钱的小伙子,其他几个人只不过是他的客人罢了。由于陪同他的朋友莫理斯·德·福代蒙侯爵出身于名门望族,那个女演员问我愿意不愿意去时,为了抬举我,她本能地说道:“这会叫莫理斯喜出望外。”
\par 待我在大厅中碰到他们三个人的时候,那个有钱的年轻人退后一步,倒是德·福代蒙先生对我说:“您不赏光来和我们一起进晚餐吗?”
\par 总而言之,我没有充分利用巴尔贝克,这倒叫我更想再次前来。我觉得自己在那里待的时间太少。可是我的朋友不这样看,他们给我写信,问我是不是打算永远在巴尔贝克生活下去,是不是他们以后将不得不在信封上写上巴尔贝克这个地名。我的窗子不朝着田野,也不朝着一条街,而是朝着大海这边,每天夜里我听到大海的呼啸。入睡之前,我像一只小船一样,将自己的睡梦托付给大海。我有一种幻觉,便是这与波涛一起构成的喧嚣,大概在我不知不觉中就像睡梦中教的功课一般,具体地向我头脑中灌输了其魅力的概念。
\par 旅馆经理主动提出明年给我更好的房间。我现在对自己的房间已经十分眷恋,走进房间里再也闻不到印须芒草的味道。从前在这个房间里,我的思路是那样难以展开,现在,这思路终于那样准确地占据了整个空间,以致当我应该在巴黎我从前那个天花板很低的房间里过夜时,不得不对自己的思路进行反方向的处理。
\par 确实应该离开巴尔贝克了。在这个没有壁炉和取暖器的旅馆里,寒冷和潮湿已经这样浸人骨髓,不能再待下去了。最后的几周,我几乎立即就忘记了。每当我想到巴尔贝克,几乎不加变化地重现在我眼前的,便是每天早晨的时刻。天气晴朗的季节,因为我下午要同阿尔贝蒂娜及其女友外出,外祖母遵照医嘱,强迫我每天早晨在暗中躺在床上。经理发出命令,不许在我这一层弄出声响,并且亲自照看,要人们服从命令。光线很强,我尽量长时间地让那大紫窗帘拉着。我刚来的第一天晚上,这窗帘曾对我表现出那样大的敌意。为了不让光线透进来,每天晚上,弗朗索瓦丝都把毯子、桌子上的红印花布、从这里那里弄来的料子接到窗帘上去,用别针别住。也只有她能把这窗帘解下来。她无法把各处都拼接得恰到好处,于是这黑暗并不完全彻底,窗帘还是让有如秋牡丹鲜红的叶子一样的东西撒播在地毯上。我忍不住要上去赤脚踏住那些“秋牡丹”。
\par 对着窗户的那面墙,已被局部照亮。墙上,没有任何支撑的一个金色圆柱体垂直地立在那里,像在荒漠中作为希伯莱人前导的光柱一样缓缓移动。\footnote{见《旧约出埃及记》第十三章:日间耶和华在云柱中领他们的路,夜间在火柱中光照他们,使他们日夜都可以行走。日间云柱,夜间火柱,总不离开百姓的面前。}
\par 我再次躺下,静静地只通过想象去品味游戏、洗海水浴、步行的快乐,而且同时品味所有这一切快乐,上午很适宜做这些事。快乐使我的心怦怦跳动,好似一台充分开动的机器。但这台机器不能移动,只能自我转动,将其速度就地传递出去。
\par 我知道那些女友此刻正在海堤上,但我看不见她们,她们正从大海那高高低低的支脉前经过。有时短暂放晴,在大海尽头可以望见里夫贝尔小城。阳光将这座小城精心地分成一个个小块。它犹如一座意大利小镇,栖息在大海蓝莹莹的峰巅上。我看不见女友们(而报贩——弗朗索瓦丝管他们叫“报人”\footnote{此词法文中也为“记者”之意。}——的叫卖声,洗海水浴的人和孩子们玩耍发出的呼喊,如海鸟的鸣叫一般为轻轻撞碎的海浪敲击着节拍。这些声音都传到我这高台上来),我推测得到她们的存在,柔和的涛声一直传进我的耳鼓,我听见她们卷进波涛中发出如同涅瑞伊得斯\footnote{涅瑞伊得斯是涅柔斯和多里斯的女儿,为海中仙女。她们一共有姐妹五十人,但名字却有七十七个,其中著名的有安菲特里特、忒提斯、该拉忒亚等。}的笑声。
\par “我们看了半天,”阿尔贝蒂娜当天晚上对我说,“想看看你是不是会下来。可是你的窗板一直关着,甚至到了音乐会的时间还关着。”
\par 确实,十点钟时,音乐会在我的窗下轰响起来。如果海水涨潮,在乐器间歇之中,一个浪头打来,似乎能将小提琴的节拍卷进自己那水晶涡状物之中,泡沫溅到海底音乐那断断续续的回声上,然后那形成浪花的海水重又流淌下去,流水倾注,永不间断。
\par 还不把我的衣物送来,让我可以穿衣起床。我着起急来,时钟敲响正午十二点,弗朗索瓦丝终于来到。连续几个月,在这个我将之想象为只受暴风雨袭击并笼罩在烟雾之中因而那样向往的巴尔贝克,晴朗的天空是那样明亮,那样宁静,弗朗索瓦丝前来将窗户打开时,我总能毫无谬误地推想,我会找到折到外墙角上的那一方阳光。其颜色永恒不变,作为夏天的标志,则不如毫无生气的假珐琅那样抑郁而动人。弗朗索瓦丝将窗帘上的别针一一取下,拿掉布料,拉开窗帘时,她展露出来的夏日似乎与一具华丽的千年木乃伊一般死气沉沉,他是那样亘古有之。我家这位老女仆只是小心翼翼地为这具木乃伊除去原来身上的衣物,叫它身着金袍、散发着香气出现在人们眼前而已。




\subsection{第三卷\ 盖尔芒特家那边}


\begin{center}
\par 赠挚友莱翁·都德:
\par 谨致衷心的感激和敬意。马塞尔·普鲁斯特
\end{center}






















