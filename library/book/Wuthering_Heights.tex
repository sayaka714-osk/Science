


\section{《Wuthering Heights》}


\par 书名:呼啸山庄:英文版(世界名著精选)
\par 作者:[英]勃朗特
\par 出版社:中国纺织出版社
\par 出版时间:2016-01
\par ISBN:9787518021147


\subsection*{序言}


\par 《呼啸山庄》是外国经典文学必读书目,是一部震撼人心的“奇特的小说”和“神秘莫测的怪书”。英国作家毛姆曾说:“《呼啸山庄》的丑恶与美并存,而且它所表达的力量也是一般小说家难以企及的……我不知道还有哪一部小说,其中爱情的痛苦、迷恋、残酷、执著,曾经如此令人吃惊地描述出来。”
\par 《呼啸山庄》奠定了艾米莉·勃朗特在英国文学史以及世界文学史上的地位。本书故事是以荒凉萧条的约克郡为背景,讲述了希斯克利夫与凯瑟琳·厄恩肖之间无法实现的爱情,并由此在两代人之间产生的爱恨情仇的故事。呼啸山庄主人厄恩肖收养了孤儿希斯克利夫,比起他的儿子欣德利,主人厄恩肖更偏爱希斯克利夫。因此,渴望得到父爱的欣德利对希斯克利夫产生了极大的憎恶,这也造就了日后希斯克利夫对欣德利的报复心理。
\par 与之相反,厄恩肖的女儿凯瑟琳却爱上了希斯克利夫,但是凯瑟琳想要获得能够使希斯克利夫脱离狠毒无情的欣德利的力量与钱财,因此,她选择嫁给了画眉山庄主人的儿子埃德加。对此,希斯克利夫绝望至极。
\par 带着复仇心理的希斯克利夫故意诱惑埃德加的妹妹,并与之结婚。同时,他欺骗欣德利并得到了呼啸山庄,此后又虐待欣德利的儿子哈里顿。凯瑟琳生下女儿凯瑟琳·林顿后就去世了。于是,希斯克利夫将无限的仇恨转移到哈里顿和凯瑟琳·林顿的身上。
\par 整部书充满了强烈的反压迫、求自由的斗争精神,又始终笼罩着离奇、紧张、浪漫的艺术气氛。作品开始曾被人称作是年轻女作家脱离现实的天真幻想,但结合其所描写地区激烈的阶级斗争和英国当时的社会现象,不久又被评论界给予高度肯定,并受到广大读者的热烈欢迎。
\par 本书的特点:
\par 1.本书以希斯克利夫被收养为开始,中间以悲剧发展,最终以圆满结局收场,笔触细腻,讲述希斯克利夫的爱情和复仇。
\par 2.纯英文版,权威版本,原汁原味,本色地呈现原作真实的经典。
\par 《呼啸山庄》是世界上震撼人心的“黑色英雄”小说之一,在英国文学史上有“文学中的斯芬克斯”“人间情爱的宏伟史诗”的美誉。它是“一部没有被时间的尘土遮没了光辉的杰出作品”。





\subsection*{Chapter 1}

\par 1801 — I have just returned from a visit to my landlord — the solitary neighbour that I shall be troubled with. This is certainly a beautiful country! In all England, I do not believe that I could have fixed on a situation so completely removed from the stir of society. A perfect misanthropist's heaven: and Mr. Heathcliff and I are such a suitable pair to divide the desolation between us. A capital fellow! He little imagined how my heart warmed towards him when I beheld his black eyes withdraw so suspiciously under their brows, as I rode up, and when his fingers sheltered themselves, with a jealous resolution, still farther in his waistcoat, as I announced my name.
\par “Mr. Heathcliff!” I said.
\par A nod was the answer.
\par “Mr. Lockwood, your new tenant, sir. I do myself the honour of calling as soon as possible after my arrival, to express the hope that I have not inconvenienced you by my perseverance in soliciting the occupation of Thrushcross Grange: I heard yesterday you had had some thoughts —”
\par “Thrushcross Grange is my own, sir,” he interrupted, wincing. “I should not allow anyone to inconvenience me, if I could hinder it—walk in!”
\par The “walk in” was uttered with closed teeth, and expressed the sentiment, “Go to the Deuce!”, Even the gate over which he leant manifested no sympathizing movement to the words, and I think that circumstance determined me to accept the invitation: I felt interested in a man who seemed more exaggeratedly reserved than myself.
\par When he saw my horse's breast fairly pushing the barrier, he did put out his hand to unchain it, and then sullenly preceded me up the causeway, calling, as we entered the court: “Joseph, take Mr. Lockwood's horse; and bring up some wine.”
\par “Here we have the whole establishment of domestics, I suppose,” was the reflection suggested by this compound order. “No wonder the grass grows up between the flags, and cattle are the only hedge-cutters.”
\par Joseph was an elderly, nay, an old man, very old, perhaps, though hale and sinewy. “The Lord help us!” he soliloquized in an undertone of peevish displeasure, while relieving me of my horse: looking, meantime, in my face so sourly that I charitably conjectured he must have need of divine aid to digest his dinner, and his pious ejaculation had no reference to my unexpected advent.
\par Wuthering Heights is the name of Mr. Heathcliff's dwelling.“Wuthering” being a significant provincial adjective, descriptive of the atmospheric tumult to which its station is exposed in stormy weather. Pure, bracing ventilation they must have up there at all times, indeed; one may guess the power of the north wind blowing over the edge, by the excessive slant of a few stunted firs at the end of the house; and by a range of gaunt thorns all stretching their limbs one way, as if craving alms of the sun. Happily, the architect had foresight to build it strong: the narrow windows are deeply set in the wall, and the corners defended with large jutting stones.
\par Before passing the threshold, I paused to admire a quantity of grotesque carving lavished over the front, and especially about the principal door; above which, among a wilderness of crumbling griffins and shameless little boys, I detected the date “1500”, and the name “Hareton Earnshaw”. I would have made a few comments, and requested a short history of the place from the surly owner; but his attitude at the door appeared to demand my speedy entrance, or complete departure, and I had no desire to aggravate his impatience previous to inspecting the penetralium.
\par One step brought us into the family sitting-room, without any introductory lobby or passage: they call it here “the house” preeminently. It includes kitchen and parlour, generally; but I believe at Wuthering Heights the kitchen is forced to retreat altogether into another quarter: at least I distinguished a chatter of tongues, and a clatter of culinary utensils, deep within; and I observed no signs of roasting, boiling, or baking, about the huge fireplace; nor any glitter of copper saucepans and tin cullenders on the walls. One end, indeed, reflected splendidly both light and heat from ranks of immense pewter dishes, interspersed with silver jugs and tankards, towering row after row, on a vast oak dresser, to the very roof. The latter had never been underdrawn: its entire anatomy lay bare to an inquiring eye, except where a frame of wood laden with oatcakes and clusters of legs of beef, mutton, and ham, concealed it. Above the chimney were sundry villainous old guns, and a couple of horse-pistols: and, by way of ornament, three gaudily painted canisters disposed along its ledge. The floor was of smooth, white stone; the chairs, high-backed, primitive structures, painted green: one or two heavy black ones lurking in the shade. In an arch under the dresser, reposed a huge, liver-coloured bitch pointer, surrounded by a swarm of squealing puppies, and other dogs haunted other recesses.
\par The apartment and furniture would have been nothing extraordinary as belonging to a homely, northern farmer, with a stubborn countenance, and stalwart limbs set out to advantage in knee breeches and gaiters. Such an individual seated in his armchair, his mug of ale frothing on the round table before him, is to be seen in any circuit of five or six miles among these hills, if you go at the right time after dinner. But Mr. Heathcliff forms a singular contrast to his abode and style of living. He is a dark-skinned gipsy in aspect, in dress and manners a gentleman: that is, as much a gentleman as many a country squire: rather slovenly, perhaps, yet not looking amiss with his negligence, because he has an erect and handsome figure; and rather morose. Possibly, some people might suspect him of a degree of under bred pride; I have a sympathetic chord within that tells me it is nothing of the sort: I know, by instinct, his reserve springs from an aversion to showy displays of feeling — to manifestations of mutual kindliness. He'll love and hate equally under cover, and esteem it a species of impertinence to be loved or hated again. No, I'm running on too fast:I bestow my own attributes over liberally on him. Mr. Heathcliff may have entirely dissimilar reasons for keeping his hand out of the way when he meets a would-be acquaintance, to those which actuate me. Let me hope my constitution is almost peculiar: my dear mother used to say I should never have a comfortable home; and only last summer I proved myself perfectly unworthy of one.
\par While enjoying a month of fine weather at the sea coast, I was thrown into the company of a most fascinating creature: a real goddess in my eyes, as long as she took no notice of me. I “never told my love” vocally; still, if looks have language, the merest idiot might have guessed I was over head and ears: she understood me at last, and looked a return — the sweetest of all imaginable looks. And what did I do? I confess it with shame — shrunk icily into myself, like a snail; at every glance retired colder and further; till finally the poor innocent was led to doubt her own senses, and, overwhelmed with confusion at her supposed mistake, persuaded her mamma to decamp.
\par By this curious turn of disposition I have gained the reputation of deliberate heartlessness; how undeserved, I alone can appreciate.
\par I took a seat at the end of the hearthstone opposite that towards which my landlord advanced, and filled up an interval of silence by attempting to caress the canine mother, who had left her nursery, and was sneaking wolfishly to the back of my legs, her lip curled up, and her white teeth watering for a snatch. My caress provoked a long, guttural gnarl.
\par “You'd better let the dog alone,” growled Mr. Heathcliff in unison, checking fiercer demonstrations with a punch of his foot. “She's not accustomed to be spoiled — not kept for a pet.” Then, striding to a side door, he shouted again —
\par “Joseph!”
\par Joseph mumbled indistinctly in the depths of the cellar, but gave no intimation of ascending; so his master dived down to him, leaving me vis-à-vis the ruffianly bitch and a pair of grim shaggy sheep dogs, who shared with her a jealous guardianship over all my movements.
\par Not anxious to come in contact with their fangs, I sat still; but, imagining they would scarcely understand tacit insults, I unfortunately indulged in winking and making faces at the trio, and some turn of my physiognomy so irritated madam, that she suddenly broke into a fury and leapt on my knees. I flung her back, and hastened to interpose the table between us. This proceeding roused the whole hive: half a dozen four-footed fiends, of various sizes and ages, issued from hidden dens to the common centre. I felt my heels and coat-laps peculiar subjects of assault; and parrying off the larger combatants as effectually as I could with the poker, I was constrained to demand, aloud, assistance from some of the household in re-establishing peace.
\par Mr. Heathcliff and his man climbed the cellar steps with vexatious phlegm: I don't think they moved one second faster than usual, though the hearth was an absolute tempest of worrying and yelping.
\par Happily, an inhabitant of the kitchen made more dispatch; a lusty dame, with tucked-up gown, bare arms, and fire-flushed cheeks, rushed into the midst of us flourishing a frying-pan: and used that weapon, and her tongue, to such purpose, that the storm subsided magically, and she only remained, heaving like a sea after a high wind, when her master entered on the scene.
\par “What the devil is the matter?” he asked, eyeing me in a manner that I could ill endure after this inhospitable treatment.
\par “What the devil, indeed!” I muttered. “The herd of possessed swine could have had no worse spirits in them than those animals of yours, sir. You might as well leave a stranger with a brood of tigers!”
\par “They won't meddle with persons who touch nothing,” he remarked, putting the bottle before me, and restoring the displaced table. “The dogs do right to be vigilant. Take a glass of wine?”
\par “No, thank you.”
\par “Not bitten, are you?”
\par “If I had been, I would have set my signet on the biter.”
\par Heathcliff's countenance relaxed into a grin.
\par “Come, come,” he said, “you are flurried, Mr. Lockwood. Here, take a little wine. Guests are so exceedingly rare in this house that I and my dogs, I am willing to own, hardly know how to receive them. Your health, sir!”
\par I bowed and returned the pledge; beginning to perceive that it would be foolish to sit sulking for the misbehaviour of a pack of curs: besides, I felt loath to yield the fellow further amusement at my expense; since the humour took that turn. He — probably swayed by prudential consideration of the folly of offending a good tenant —relaxed a little in the laconic style of chipping off his pronouns and auxiliary verbs, and introduced what he supposed would be a subject of interest to me — a discourse on the advantages and disadvantages of my present place of retirement. I found him very intelligent on the topics we touched; and before I went home, I was encouraged so far as to volunteer another visit tomorrow. He evidently wished no repetition of my intrusion. I shall go, notwithstanding. It is astonishing how sociable I feel myself compared with him.


\subsection*{Chapter 2}

\par Yesterday afternoon set in misty and cold. I had half a mind to spend it by my study fire, instead of wading through heath and mud to Wuthering Heights. On coming up from dinner, however (N.B. I dine between twelve and one o'clock; the housekeeper, a matronly lady, taken as a fixture along with the house, could not, or would not, comprehend my request that I might be served at five), on mounting the stairs with this lazy intention, and stepping into the room, I saw a servant girl on her knees surrounded by brushes and coal-scuttles, and raising an infernal dust as she extinguished the flames with heaps of cinders. This spectacle drove me back immediately; I took my hat, and, after a four-miles'walk, arrived at Heathcliff's garden gate just in time to escape the first feathery flakes of a snow shower.
\par On that bleak hilltop the earth was hard with a black frost, and the air made me shiver through every limb. Being unable to remove the chain, I jumped over, and, running up the flagged causeway bordered with straggling gooseberry bushes, knocked vainly for admittance, till my knuckles tingled and the dogs howled.
\par “Wretched inmates!” I ejaculated mentally, “you deserve perpetual isolation from your species for your churlish inhospitality. At least, I would not keep my doors barred in the day time. I don't care — I will get in!” So resolved, I grasped the latch and shook it vehemently. Vinegar-faced Joseph projected his head from a round window of the barn.
\par “Whet are ye for?” he shouted. “T' maister's dahn i' t' fowld. Go rahnd by th' end ut' laith, if yah went tuh spake tull him.”
\par “Is there nobody inside to open the door?” I hallooed, responsively.
\par “They's nobbut t'missis; and shoo'll nut oppen't an ye mak yer flaysome dins till neeght.”
\par “Why? Cannot you tell her who I am, eh, Joseph?”
\par “Nor-ne me! Aw'll hae noa hend wi't,” muttered the head, vanishing.
\par The snow began to drive thickly. I seized the handle to essay another trial; when a young man without coat, and shouldering a pitchfork, appeared in the yard behind. He hailed me to follow him, and, after marching through a wash-house, and a paved area containing a coal shed, pump, and pigeon cot, we at length arrived in the huge, warm, cheerful apartment, where I was formerly received. It glowed delightfully in the radiance of an immense fire, compounded of coal, peat, and wood; and near the table, laid for a plentiful evening meal, I was pleased to observe the “missis”, an individual whose existence I had never previously suspected. I bowed and waited, thinking she would bid me take a seat. She looked at me, leaning back in her chair, and remained motionless and mute.
\par “Rough weather!” I remarked. “I'm afraid, Mrs. Heathcliff, the door must bear the consequence of your servants' leisure attendance: I had hard work to make them hear me.”
\par She never opened her mouth. I stared — she stared also: at any rate, she kept her eyes on me in a cool, regardless manner, exceedingly embarrassing and disagreeable.
\par “Sit down,” said the young man gruffly. “He'll be in soon.”
\par I obeyed; and hemmed, and called the villain Juno, who deigned, at this second interview, to move the extreme tip of her tail, in token of owning my acquaintance.
\par “A beautiful animal!” I commenced again. “Do you intend parting with the little ones, madam?”
\par “They are not mine,” said the amiable hostess, more repellingly than Heathcliff himself could have replied.
\par “Ah, your favourites are among these?” I continued, turning to an obscure cushion full of something like cats.
\par “A strange choice of favourites!” she observed scornfully.
\par Unluckily, it was a heap of dead rabbits. I hemmed once more, and drew closer to the hearth, repeating my comment on the wildness of the evening.
\par “You should not have come out,” she said, rising and reaching from the chimney-piece two of the painted canisters.
\par Her position before was sheltered from the light; now, I had a distinct view of her whole figure and countenance. She was slender, and apparently scarcely past girlhood: an admirable form, and the most exquisite little face that I have ever had the pleasure of beholding; small features, very fair; flaxen ringlets, or rather golden, hanging loose on her delicate neck; and eyes, had they been agreeable in expression, they would have been irresistible: fortunately for my susceptible heart, the only sentiment they evinced hovered between scorn, and a kind of desperation, singularly unnatural to be detected there.
\par The canisters were almost out of her reach; I made a motion to aid her; she turned upon me as a miser might turn if anyone attempted to assist him in counting his gold.
\par “I don't want your help,” she snapped; “I can get them for myself.”
\par “I beg your pardon!” I hastened to reply.
\par “Were you asked to tea?” she demanded, tying an apron over her neat black frock, and standing with a spoonful of the leaf poised over the pot.
\par “I shall be glad to have a cup,” I answered.
\par “Were you asked?” she repeated.
\par “No,” I said, half smiling. “You are the proper person to ask me.”
\par She flung the tea back, spoon and all, and resumed her chair in a pet; her forehead corrugated, and her red under lip pushed out, like a child's ready to cry.
\par Meanwhile, the young man had slung on to his person a decidedly shabby upper garment, and, erecting himself before the blaze, looked down on me from the corner of his eyes, for all the world as if there were some mortal feud unavenged between us. I began to doubt whether he were a servant or not: his dress and speech were both rude, entirely devoid of the superiority observable in Mr. and Mrs. Heathcliff; his thick brown curls were rough and uncultivated, his whiskers encroached bearishly over his cheeks, and his hands were embrowned like those of a common labourer: still his bearing was free, almost haughty, and he showed none of a domestic's assiduity in attending on the lady of the house. In the absence of clear proofs of his condition, I deemed it best to abstain from noticing his curious conduct; and, five minutes afterwards, the entrance of Heathcliff relieved me, in some measure, from my uncomfortable state.
\par “You see, sir, I am come, according to promise!” I exclaimed, assuming the cheerful; “and I fear I shall be weather-bound for half an hour, if you can afford me shelter during that space.”
\par “Half an hour?” he said, shaking the white flakes from his clothes;“I wonder you should select the thick of a snowstorm to ramble about in. Do you know that you run a risk of being lost in the marshes? People familiar with these moors often miss their road on such evenings; and I can tell you there is no chance of a change at present.”
\par “Perhaps I can get a guide among your lads, and he might stay at the Grange till morning — could you spare me one?”
\par “No, I could not.”
\par “Oh, indeed! Well, then, I must trust to my own sagacity.”
\par “Umph!”
\par “Are you going to mak th' tea?” demanded he of the shabby coat, shifting his ferocious gaze from me to the young lady.
\par “Is he to have any?” she asked, appealing to Heathcliff.
\par “Get it ready, will you?” was the answer, uttered so savagely that I started. The tone in which the words were said revealed a genuine bad nature. I no longer felt inclined to call Heathcliff a capital fellow. When the preparations were finished, he invited me with — “Now, sir, bring forward your chair.” And we all, including the rustic youth, drew round the table: an austere silence prevailing while we discussed our meal.
\par I thought, if I had caused the cloud, it was my duty to make an effort to dispel it. They could not every day sit so grim and taciturn; and it was impossible, however ill-tempered they might be, that the universal scowl they wore was their everyday countenance.
\par “It is strange,” I began, in the interval of swallowing one cup of tea and receiving another — “it is strange how custom can mould our tastes and ideas: many could not imagine the existence of happiness in a life of such complete exile from the world as you spend, Mr. Heathcliff; yet I'll venture to say, that, surrounded by your family, and with your amiable lady as the presiding genius over your home and heart —”
\par “My amiable lady!” he interrupted, with an almost diabolical sneer on his face. “Where is she — my amiable lady?”
\par “Mrs. Heathcliff, your wife, I mean.”
\par “Well, yes — Oh, you would intimate that her spirit has taken the post of ministering angel, and guards the fortunes of Wuthering Heights even when her body is gone. Is that it?”
\par Perceiving myself in a blunder, I attempted to correct it. I might have seen there was too great a disparity between the ages of the parties to make it likely that they were man and wife. One was about forty: a period of mental vigour at which men seldom cherish the delusion of being married for love by girls: that dream is reserved for the solace of our declining years. The other did not look seventeen.
\par Then it flashed upon me — “The clown at my elbow, who is drinking his tea out of a basin and eating his bread with unwashed hands, may be her husband: Heathcliff, junior, of course. Here is the consequence of being buried alive: she has thrown herself away upon that boor from sheer ignorance that better individuals existed! A sad pity —I must beware how I cause her to regret her choice.” The last reflection may seem conceited; it was not. My neighbour struck me as bordering on repulsive; I knew, through experience, that I was tolerably attractive.
\par “Mrs. Heathcliff is my daughter-in-law,” said Heathcliff, corroborating my surmise. He turned, as he spoke, a peculiar look in her direction: a look of hatred; unless he has a most perverse set of facial muscles that will not, like those of other people, interpret the language of his soul.
\par “Ah, certainly — I see now: you are the favoured possessor of the beneficent fairy,” I remarked, turning to my neighbour.
\par This was worse than before: the youth grew crimson, and clenched his fist, with every appearance of a meditated assault. But he seemed to recollect himself presently, and smothered the storm in a brutal curse, muttered on my behalf: which, however, I took care not to notice.
\par “Unhappy in your conjectures, sir,” observed my host; “we neither of us have the privilege of owning your good fairy; her mate is dead. I said she was my daughter-in-law, therefore, she must have married my son.”
\par “And this young man is —”
\par “Not my son, assuredly.”
\par Heathcliff smiled again, as if it were rather too bold a jest to attribute the paternity of that bear to him.
\par “My name is Hareton Earnshaw,” growled the other; “and I'd counsel you to respect it!”
\par “I've shown no disrespect,” was my reply, laughing internally at the dignity with which he announced himself.
\par He fixed his eye on me longer than I cared to return the stare, for fear I might be tempted either to box his ears or render my hilarity audible. I began to feel unmistakably out of place in that pleasant family circle. The dismal spiritual atmosphere overcame, and more than neutralized, the glowing physical comforts round me; and I resolved to be cautious how I ventured under those rafters a third time.
\par The business of eating being concluded, and no one uttering a word of sociable conversation, I approached a window to examine the weather. A sorrowful sight I saw: dark night coming down prematurely, and sky and hills mingled in one bitter whirl of wind and suffocating snow.
\par “I don't think it possible for me to get home now without a guide,” I could not help exclaiming. “The roads will be buried already; and, if they were bare, I could scarcely distinguish a foot in advance.”
\par “Hareton, drive those dozen sheep into the barn porch. They'll be covered if left in the fold all night: and put a plank before them,” said Heathcliff.
\par “How must I do?” I continued, with rising irritation.
\par There was no reply to my question; and on looking round I saw only Joseph bringing in a pail of porridge for the dogs, and Mrs. Heathcliff leaning over the fire, diverting herself with burning a bundle of matches which had fallen from the chimney-piece as she restored the tea canister to its place. The former, when he had deposited his burden, took a critical survey of the room, and in cracked tones, grated out:
\par “Aw wonder how yah can faishion to stand thear i'idleness un war, when all on ‘em's goan out! Bud yah're a nowt, and it's no use talking—yah'll niver mend o'yer ill ways, bud goa raight to t'divil, like yer mother afore ye!”
\par I imagined, for a moment, that this piece of eloquence was addressed to me; and, sufficiently enraged, stepped towards the aged rascal with an intention of kicking him out of the door. Mrs. Heathcliff, however, checked me by her answer.
\par “You scandalous old hypocrite!” she replied. “Are you not afraid of being carried away bodily, whenever you mention the devil's name? I warn you to refrain from provoking me, or I'll ask your abduction as a special favour. Stop! look here, Joseph,” she continued, taking a long, dark book from a shelf; “I'll show you how far I've progressed in the Black Art: I shall soon be competent to make a clear house of it. The red cow didn't die by chance; and your rheumatism can hardly be reckoned among providential visitations!”
\par “Oh, wicked, wicked!” gasped the elder; “may the Lord deliver us from evil!”
\par “No, reprobate! you are a castaway — be off, or I'll hurt you seriously! I'll have you all modelled in wax and clay; and the first who passes the limits I fix, shall — I'll not say what he shall be done to —but, you'll see! Go, I'm looking at you!”
\par The little witch put a mock malignity into her beautiful eyes, and Joseph, trembling with sincere horror, hurried out praying and ejaculating “wicked” as he went. I thought her conduct must be prompted by a species of dreary fun; and, now that we were alone, I endeavoured to interest her in my distress.
\par “Mrs. Heathcliff,” I said earnestly, “you must excuse me for troubling you. I presume, because, with that face, I'm sure you cannot help being good-hearted. Do point out some landmarks by which I may know my way home: I have no more idea how to get there than you would have how to get to London!”
\par “Take the road you came,” she answered, ensconcing herself in a chair, with a candle, and the long book open before her. “It is brief advice, but as sound as I can give.”
\par “Then, if you hear of me being discovered dead in a bog or a pit full of snow, your conscience won't whisper that it is partly your fault?”
\par “How so? I cannot escort you. They wouldn't let me go to the end of the garden wall.”
\par “You! I should be sorry to ask you to cross the threshold, for my convenience, on such a night,” I cried. “I want you to tell me my way, not to show it; or else to persuade Mr. Heathcliff to give me a guide.”
\par “Who? There is himself, Earnshaw, Zillah, Joseph, and I. Which would you have?”
\par “Are there no boys at the farm?”
\par “No, those are all.”
\par “Then, it follows that I am compelled to stay.”
\par “That you may settle with your host. I have nothing to do with it.”
\par “I hope it will be a lesson to you to make no more rash journeys on these hills,” cried Heathcliff's stern voice from the kitchen entrance.“As to staying here, I don't keep accommodations for visitors: you must share a bed with Hareton or Joseph, if you do.”
\par “I can sleep on a chair in this room,” I replied.
\par “No, no! A stranger is a stranger, be he rich or poor: it will not suit me to permit anyone the range of the place while I am off guard!” said the unmannerly wretch.
\par With this insult, my patience was at an end. I uttered an expression of disgust, and pushed past him into the yard, running against Earnshaw in my haste. It was so dark that I could not see the means of exit; and, as I wandered round, I heard another specimen of their civil behaviour amongst each other. At first the young man appeared about to befriend me.
\par “I'll go with him as far as the park,” he said.
\par “You'll go with him to hell!” exclaimed his master, or whatever relation he bore. “And who is to look after the horses, eh?”
\par “A man's life is of more consequence than one evening's neglect of the horses: somebody must go,” murmured Mrs. Heathcliff, more kindly than I expected.
\par “Not at your command!” retorted Hareton. “If you set store on him, you'd better be quiet.”
\par “Then I hope his ghost will haunt you; and I hope Mr. Heathcliff will never get another tenant till the Grange is a ruin!” she answered sharply.
\par “Hearken, hearken, shoo's cursing on ‘em!” muttered Joseph, towards whom I had been steering.
\par He sat within earshot, milking the cows by the light of a lantern, which I seized unceremoniously, and, calling out that I would send it back on the morrow, rushed to the nearest postern.
\par “Maister, maister, he's stealing t' lantern!” shouted the ancient, pursuing my retreat. “Hey, Gnasher! Hey, dog! Hey, Wolf, holld him, holld him!”
\par On opening the little door, two hairy monsters flew at my throat, bearing me down and extinguishing the light; while a mingled guffaw from Heathcliff and Hareton, put the copestone on my rage and humiliation. Fortunately, the beasts seemed more bent on stretching their paws and yawning, and flourishing their tails, than devouring me alive; but they would suffer no resurrection, and I was forced to lie till their malignant master pleased to deliver me: then, hatless and trembling with wrath, I ordered the miscreants to let me out — on their peril to keep me one minute longer — with several incoherent threats of retaliation that, in their indefinite depth of virulency, smacked of King Lear.
\par The vehemence of my agitation brought on a copious bleeding at the nose, and still Heathcliff laughed, and still I scolded. I don't know what would have concluded the scene, had there not been one person at hand rather more rational than myself, and more benevolent than my entertainer. This was Zillah, the stout housewife; who at length issued forth to inquire into the nature of the uproar. She thought that some of them had been laying violent hands on me; and, not daring to attack her master, she turned her vocal artillery against the young scoundrel.
\par “Well, Mr. Earnshaw,” she cried, “I wonder what you'll have agait next! Are we going to murder folk on our very doorstones? I see this house will never do for me — look at t' poor lad, he's fair choking! Wisht, wisht; you mun'n't go on so. Come in, and I'll cure that; there now, hold ye still.”
\par With these words she suddenly splashed a pint of icy water down my neck, and pulled me into the kitchen. Mr. Heathcliff followed, his accidental merriment expiring quickly in his habitual moroseness.
\par I was sick exceedingly, and dizzy and faint; and thus compelled perforce to accept lodgings under his roof. He told Zillah to give me a glass of brandy, and then passed on to the inner room; while she condoled with me on my sorry predicament, and having obeyed his orders, whereby I was somewhat revived, ushered me to bed.






\subsection*{Chapter 3}


\par While leading the way upstairs, she recommended that I should hide the candle, and not make a noise; for her master had an odd notion about the chamber she would put me in, and never let anybody lodge there willingly. I asked the reason. She did not know, she answered: she had only lived there a year or two; and they had so many queer goings on, she could not begin to be curious.
\par Too stupefied to be curious myself, I fastened my door and glanced round for the bed. The whole furniture consisted of a chair, a clothespress, and a large oak case, with squares cut out near the top resembling coach windows. Having approached this structure I looked inside, and perceived it to be a singular sort of old-fashioned couch, very conveniently designed to obviate the necessity for every member of the family having a room to himself. In fact it formed a little closet, and the ledge of a window, which it enclosed, served as a table. I slid back the panelled sides, got in with my light, pulled them together again, and felt secure against the vigilance of Heathcliff, and everyone else.
\par The ledge, where I placed my candle, had a few mildewed books piled up in one corner; and it was covered with writing scratched on the paint. This writing, however, was nothing but a name repeated in all kinds of characters, large and small — Catherine Earnshaw, here and there varied to Catherine Heathcliff, and again to Catherine Linton.
\par In vapid listlessness I leant my head against the window, and continued spelling over Catherine Earnshaw — Heathcliff — Linton, till my eyes closed; but they had not rested five minutes when a glare of white letters started from the dark as vivid as specters — the air swarmed with Catherines; and rousing myself to dispel the obtrusive name, I discovered my candle wick reclining on one of the antique volumes, and perfuming the place with an odour of roasted calfskin.
\par I snuffed it off, and, very ill at ease under the influence of cold and lingering nausea, sat up and spread open the injured tome on my knee. It was a Testament, in lean type, and smelling dreadfully musty: a flyleaf bore the inscription — “Catherine Earnshaw, her book”, and a date some quarter of a century back.
\par I shut it, and took up another, and another, till I had examined all. Catherine's library was select, and its state of dilapidation proved it to have been well used; though not altogether for a legitimate purpose: scarcely one chapter had escaped a pen-and-ink commentary — at least, the appearance of one — covering every morsel of blank that the printer had left. Some were detached sentences; other parts took the form of a regular diary, scrawled in an unformed childish hand. At the top of an extra page (quite a treasure, probably, when first lighted on) I was greatly amused to behold an excellent caricature of my friend Joseph,— rudely, yet powerfully sketched. An immediate interest kindled within me for the unknown Catherine, and I began forthwith to decipher her faded hieroglyphics.
\par “An awful Sunday!” commenced the paragraph beneath. “I wish my father were back again. Hindley is a detestable substitute — his conduct to Heathcliff is atrocious — H. and I are going to rebel — we took our initiatory step this evening.
\par “All day had been flooding with rain; we could not go to church, so Joseph must needs get up a congregation in the garret; and, while Hindley and his wife basked downstairs before a comfortable fire —doing anything but reading their Bibles, I'll answer for it — Heathcliff, myself, and the unhappy plough-boy, were commanded to take our prayer books, and mount. We were ranged in a row, on a sack of corn, groaning and shivering, and hoping that Joseph would shiver too, so that he might give us a short homily for his own sake. A vain idea! The service lasted precisely three hours; and yet my brother had the face to exclaim, when he saw us descending, ‘What, done already?’ On Sunday evenings we used to be permitted to play, if we did not make much noise; now a mere titter is sufficient to send us into corners!
\par “‘You forget you have a master here,’ says the tyrant. ‘I'll demolish the first who puts me out of temper! I insist on perfect sobriety and silence. Oh, boy! Was that you? Frances, darling, pull his hair as you go by: I heard him snap his fingers.’ Frances pulled his hair heartily, and then went and seated herself on her husband's knee; and there they were, like two babies, kissing and talking nonsense by the hour —foolish palaver that we should be ashamed of. We made ourselves as snug as our means allowed in the arch of the dresser. I had just fastened our pinafores together, and hung them up for a curtain, when in comes Joseph on an errand from the stables. He tears down my handiwork, boxes my ears, and croaks —
\par “‘T’ maister nobbut just buried, and Sabbath nut o'ered, und t'sahnd uh t'gospel still i'yer lugs, and yah darr be laiking! Shame on ye! sit ye dahn, ill childer! they's good books eneugh if ye'll read'em! sit ye dahn, and think uh yer sowls!’
\par “Saying this, he compelled us so to square our positions that we might receive from the far-off fire a dull ray to show us the text of the lumber thrust upon us. I could not bear the employment. I took my dingy volume by the scroop, and hurled it into the dog kennel, vowing I hated a good book. Heathcliff kicked his to the same place. Then there was a hubbub!
\par “‘Maister Hindley!’ shouted our chaplain. ‘Maister, coom hither! Miss Cathy's riven th' back off ‘Th’ Helmet uh Salvation, un' Heathcliff's pawsed his fit intuh t' first part uh ‘T' Brooad Way to Destruction!’ It's fair flaysome ut yah let, em goa on this gait. Ech! th' owd man ud uh laced' em properly — but he's goan!”
\par “Hindley hurried up from his paradise on the hearth, and seizing one of us by the collar, and the other by the arm, hurled both into the back kitchen; where, Joseph asseverated, “owd Nick” would fetch us as sure as we were living: and, so comforted, we each sought a separate nook to await his advent. I reached this book, and a pot of ink from a shelf, and pushed the house door ajar to give me light, and I have got the time on with writing for twenty minutes; but my companion is impatient, and proposes that we should appropriate the dairywoman's cloak, and have a scamper on the moors, under its shelter. A pleasant suggestion — and then, if the surly old man come in, he may believe his prophecy verified — we 2cannot be damper, or colder, in the rain than we are here.”
\\
\par I suppose Catherine fulfilled her project, for the next sentence took up another subject: she waxed lachrymose.
\par “How little did I dream that Hindley would ever make me cry so!” she wrote. “My head aches, till I cannot keep it on the pillow; and still I can't give over. Poor Heathcliff! Hindley calls him a vagabond, and won't let him sit with us, nor eat with us any more; and, he says, he and I must not play together, and threatens to turn him out of the house if we break his orders. He has been blaming our father (how dared he?) for treating H. too liberally; and swears he will reduce him to his right place —”
\\ 
\par I began to nod drowsily over the dim page: my eye wandered from manuscript to print, I saw a red ornamented title — “Seventy Times Seven, and the First of the Seventy-First. A pious Discourse delivered by the Reverend Jabes Branderham, in the Chapel of Gimmerden Sough.” And while I was, half consciously, worrying my brain to guess what Jabes Branderham would make of his subject, I sank back in bed, and fell asleep. Alas, for the effects of bad tea and bad temper! What else could it be that made me pass such a terrible night? I don't remember another that I can at all compare with it since I was capable of suffering.
\par I began to dream, almost before I ceased to be sensible of my locality. I thought it was morning; and I had set out on my way home, with Joseph for a guide. The snow lay yards deep in our road; and, as we floundered on, my companion wearied me with constant reproaches that I had not brought a pilgrim's staff: telling me that I could never get into the house without one, and boastfully flourishing a heavyheaded cudgel, which I understood to be so denominated. For a moment I considered it absurd that I should need such a weapon to gain admittance into my own residence. Then a new idea flashed across me. I was not going there: we were journeying to hear the famous Jabes Branderham preach from the text — “Seventy Times Seven”; and either Joseph, the preacher, or I had committed the “First of the SeventyFirst”, and were to be publicly exposed and excommunicated.
\par We came to the chapel. I have passed it really in my walks, twice or thrice; it lies in a hollow, between two hills; an elevated hollow, near a swamp, whose peaty moisture is said to answer all the purposes of embalming on the few corpses deposited there. The roof has been kept whole hitherto; but as the clergyman's stipend is only twenty pounds per annum, and a house with two rooms, threatening speedily to determine into one, no clergyman will undertake the duties of pastor: especially as it is currently reported that his flock would rather let him starve than increase the living by one penny from their own pockets. However, in my dream, Jabes had a full and attentive congregation; and he preached— good God! What a sermon. Divided into four hundred and ninety parts, each fully equal to an ordinary address from the pulpit, and each discussing a separate sin! Where he searched for them, I cannot tell. He had his private manner of interpreting the phrase, and it seemed necessary the brother should sin different sins on every occasion. They were of the most curious character — odd transgressions that I never imagined previously.
\par Oh, how weary I grew. How I writhed, and yawned, and nodded, and revived! How I pinched and pricked myself, and rubbed my eyes, and stood up, and sat down again, and nudged Joseph to inform me if he would ever have done. I was condemned to hear all out: finally, he reached the “First of the Seventy-First”. At that crisis, a sudden inspiration descended on me; I was moved to rise and denounce Jabes Branderham as the sinner of the sin that no Christian need pardon.
\par “Sir,” I exclaimed, “sitting here within these four walls, at one stretch, I have endured and forgiven the four hundred and ninety heads of your discourse. Seventy times seven times have I plucked up my hat and been about to depart — seventy times seven times have you preposterously forced me to resume my seat. The four hundred and ninety-first is too much. Fellow-martyrs, have at him! Drag him down, and crush him to atoms, that the place which knows him may know him no more!”
\par “Thou art the Man!” cries Jabes, after a solemn pause, leaning over his cushion. “Seventy times seven times didst thou gapingly contort thy visage — seventy times seven did I take counsel with my soul —Lo, this is human weakness: this also may be absolved! The First of the Seventy-First is come. Brethren, execute upon him the judgment written. Such honour have all His saints!”
\par With that concluding word, the whole assembly, exalting their pilgrim's staves, rushed round me in a body; and I, having no weapon to raise in self-defence, commenced grappling with Joseph, my nearest and most ferocious assailant, for his. In the confluence of the multitude, several clubs crossed; blows, aimed at me, fell on other sconces. Presently the whole chapel resounded with rappings and counterrappings: every man's hand was against his neighbour; and Branderham, unwilling to remain idle, poured forth his zeal in a shower of loud taps on the boards of the pulpit, which responded so smartly that, at last, to my unspeakable relief, they woke me.
\par And what was it that had suggested the tremendous tumult? What had played Jabes's part in the row? Merely, the branch of a fir tree that touched my lattice, as the blast wailed by, and rattled its dry cones against the panes! I listened doubtingly an instant; detected the disturber, then turned and dozed, and dreamt again: if possible, still more disagreeably than before.
\par This time, I remembered I was lying in the oak closet, and I heard distinctly the gusty wind, and the driving of the snow; I heard, also, the fir bough repeat its teasing sound, and ascribed it to the right cause: but it annoyed me so much, that I resolved to — silence it, if possible; and, I thought, I rose and endeavoured to unhasp the casement. The hook was soldered into the staple: a circumstance observed by me when awake, but forgotten.
\par “I must stop it, nevertheless!” I muttered, knocking my knuckles through the glass, and stretching an arm out to seize the importunate branch; instead of which, my fingers closed on the fingers of a little, icecold hand!
\par The intense horror of nightmare came over me: I tried to draw back my arm, but the hand clung to it, and a most melancholy voice sobbed —
\par “Let me in — let me in!”
\par “Who are you?” I asked, struggling, meanwhile, to disengage myself.
\par “Catherine Linton,” it replied, shiveringly (why did I think of Linton? I had read Earnshaw twenty times for Linton); “I'm come home:I'd lost my way on the moor!”
\par As it spoke, I discerned, obscurely, a child's face looking through the window. Terror made me cruel; and, finding it useless to attempt shaking the creature off, I pulled its wrist on to the broken pane, and rubbed it to and fro till the blood ran down and soaked the bedclothes: still it wailed,“Let me in!” and maintained its tenacious grip, almost maddening me with fear.
\par “How can I?” I said at length. “Let me go, if you want me to let you in!”
\par The fingers relaxed, I snatched mine through the hole, hurriedly piled the books up in a pyramid against it, and stopped my ears to exclude the lamentable prayer.
\par I seemed to keep them closed above a quarter of an hour; yet, the instant I listened again, there was the doleful cry moaning on!
\par “Begone!” I shouted, “I'll never let you in, not if you beg for twenty years.”
\par “It is twenty years,” mourned the voice: “twenty years. I've been a waif for twenty years!”
\par Thereat began a feeble scratching outside, and the pile of books moved as if thrust forward.
\par I tried to jump up; but could not stir a limb; and so yelled aloud, in a frenzy of fright.
\par To my confusion, I discovered the yell was not ideal: hasty footsteps approached my chamber door; somebody pushed it open, with a vigorous hand, and a light glimmered through the squares at the top of the bed. I sat shuddering yet, and wiping the perspiration from my forehead: the intruder appeared to hesitate, and muttered to himself.
\par At last, he said in a half-whisper, plainly not expecting an answer, “Is any one here?”
\par I considered it best to confess my presence, for I knew Heathcliff's accents, and feared he might search further, if I kept quiet. With this intention, I turned and opened the panels. I shall not soon forget the effect my action produced.
\par Heathcliff stood near the entrance, in his shirt and trousers: with a candle dripping over his fingers, and his face as white as the wall behind him. The first creak of the oak startled him like an electric shock; the light leaped from his hold to a distance of some feet, and his agitation was so extreme, that he could hardly pick it up.
\par “It is only your guest, sir,” I called out, desirous to spare him the humiliation of exposing his cowardice further. “I had the misfortune to scream in my sleep, owing to a frightful nightmare. I'm sorry I disturbed you.”
\par “Oh God confound you, Mr. Lockwood! I wish you were at the —“ commenced my host, setting the candle on a chair, because he found it impossible to hold it steady. “And who showed you up into this room?” he continued, crushing his nails into his palms, and grinding his teeth to subdue the maxillary convulsions. “Who was it? I've a good mind to turn them out of the house this moment!”
\par “It was your servant, Zillah,” I replied, flinging myself on to the floor, and rapidly resuming my garments. “I should not care if you did, Mr. Heathcliff; she richly deserves it. I suppose that she wanted to get another proof that the place was haunted, at my expense. Well, it is —swarming with ghosts and goblins! You have reason in shutting it up, I assure you. No one will thank you for a doze in such a den!”
\par “What do you mean?” asked Heathcliff, “and what are you doing? Lie down and finish out the night, since you are here; but, for heaven's sake! Don't repeat that horrid noise; nothing could excuse it, unless you were having your throat cut!”
\par “If the little fiend had got in at the window, she probably would have strangled me!” I returned. “I'm not going to endure the persecutions of your hospitable ancestors again. Was not the Reverend Jabes Branderham akin to you on the mother's side? And that minx, Catherine Linton, or Earnshaw, or however she was called — she must have been a changeling — wicked little soul! She told me she had been walking the earth these twenty years: a just punishment for her mortal transgressions, I've no doubt!”
\par Scarcely were these words uttered, when I recollected the association of Heathcliff's with Catherine's name in the book, — which had completely slipped from my memory, till thus awakened. I blushed at my inconsideration; but, without showing further consciousness of the offence, I hastened to add —
\par “The truth is, sir, I passed the first part of the night in” — Here I stopped afresh — I was about to say “perusing those old volumes”, then it would have revealed my knowledge of their written, as well as their printed, contents: so, correcting myself, I went on, “in spelling over the name scratched on that window-ledge. A monotonous occupation, calculated to set me asleep, like counting, or —”
\par “What can you mean by talking in this way to me?” thundered Heathcliff with savage vehemence. “How — how dare you, under my roof? — God! he's mad to speak so!” And he struck his forehead with rage.
\par I did not know whether to resent this language or pursue my explanation; but he seemed so powerfully affected that I took pity and proceeded with my dreams; affirming I had never heard the appellation of “Catherine Linton” before, but reading it often over produced an impression which personified itself when I had no longer my imagination under control.
\par Heathcliff gradually fell back into the shelter of the bed, as I spoke; finally sitting down almost concealed behind it. I guessed, however, by his irregular and intercepted breathing, that he struggled to vanquish an excess of violent emotion. Not liking to show him that I had heard the conflict, I continued my toilette rather noisily, looking at my watch, and soliloquized on the length of the night: “Not three o'clock yet! I could have taken oath it had been six. Time stagnates here: we must surely have retired to rest at eight!”
\par “Always at nine in winter, and always rise at four,” said my host, suppressing a groan: and, as I fancied, by the motion of his shadow's arm, dashing a tear from his eyes. “Mr. Lockwood,” he added, “you may go into my room: you'll only be in the way, coming downstairs so early; and your childish outcry has sent sleep to the devil for me.”
\par “And for me, too,” I replied. “I'll walk in the yard till daylight, and then I'll be off; and you need not dread a repetition of my intrusion. I'm now quite cured of seeking pleasure in society, be it country or town. A sensible man ought to find sufficient company in himself.”
\par “Delightful company!” muttered Heathcliff. “Take the candle, and go where you please. I shall join you directly. Keep out of the yard, though, the dogs are unchained; and the house — Juno mounts sentinel there, and — nay, you can only ramble about the steps and passages.But, away with you! I'll come in two minutes!”
\par I obeyed, so far as to quit the chamber; when, ignorant where the narrow lobbies led, I stood still, and was witness, involuntarily, to a piece of superstition on the part of my landlord, which belied, oddly, his apparent sense.
\par He got on to the bed, and wrenched open the lattice, bursting, as he pulled at it, into an uncontrollable passion of tears. “Come in! come in!” he sobbed. “Cathy, do come. Oh do — once more! Oh! My heart's darling! hear me this time, Catherine, at last!” The spectre showed a spectre's ordinary caprice: it gave no sign of being; but the snow and wind whirled wildly through, even reaching my station, and blowing out the light.
\par There was such anguish in the gust of grief that accompanied this raving, that my compassion made me overlook its folly, and I drew off, half angry to have listened at all, and vexed at having related my ridiculous nightmare, since it produced that agony; though why, was beyond my comprehension. I descended cautiously to the lower regions, and landed in the back kitchen, where a gleam of fire, raked compactly together, enabled me to rekindle my candle. Nothing was stirring except a bridled, grey cat, which crept from the ashes, and saluted me with a querulous mew.
\par Two benches, shaped in sections of a circle, nearly enclosed the hearth; on one of these I stretched myself, and Grimalkin mounted the other. We were both of us nodding, ere anyone invaded our retreat, and then it was Joseph, shuffling down a wooden ladder that vanished in the roof, through a trap: the ascent to his garret, I suppose.
\par He cast a sinister look at the little flame which I had enticed to play between the ribs, swept the cat from its elevation, and bestowing himself in the vacancy, commenced the operation of stuffing a threeinch pipe with tobacco. My presence in his sanctum was evidently esteemed a piece of impudence too shameful for remark: he silently applied the tube to his lips, folded his arms, and puffed away. I let him enjoy the luxury unannoyed; and after sucking out his last wreath, and heaving a profound sigh, he got up, and departed as solemnly as he came.
\par A more elastic footstep entered next; and now I opened my mouth for a “good morning”, but closed it again, the salutation unachieved; for Hareton Earnshaw was performing his orisons sotto voce, in a series of curses directed against every object he touched, while he rummaged a corner for a spade or shovel to dig through the drifts. He glanced over the back of the bench, dilating his nostrils, and thought as little of exchanging civilities with me as with my companion the cat. I guessed, by his preparations, that egress was allowed, and, leaving my hard couch, made a movement to follow him. He noticed this, and thrust at an inner door with the end of his spade, intimating by an inarticulate sound that there was the place where I must go, if I changed my locality.
\par It opened into the house, where the females were already astir, Zillah urging flakes of flame up the chimney with a colossal bellows; and Mrs. Heathcliff, kneeling on the hearth, reading a book by the aid of the blaze. She held her hand interposed between the furnace heat and her eyes, and seemed absorbed in her occupation; desisting from it only to chide the servant for covering her with sparks, or to push away a dog, now and then, that snoozled its nose over-forwardly into her face. I was surprised to see Heathcliff there also. He stood by the fire, his back towards me, just finishing a stormy scene to poor Zillah; who ever and anon interrupted her labour to pluck up the corner of her apron, and heave an indignant groan.
\par “And you, you worthless —” he broke out as I entered, turning to his daughter-in-law, and employing an epithet as harmless as duck, or sheep, but generally represented by a dash —. “There you are, at your idle tricks again! The rest of them do earn their bread — you live on my charity! Put your trash away, and find something to do. You shall pay me for the plague of having you eternally in my sight — do you hear, damnable jade?”
\par “I'll put my trash away, because you can make me, if I refuse,” answered the young lady, closing her book, and throwing it on a chair.“But I'll not do anything, though you should swear your tongue out, except what I please!”
\par Heathcliff lifted his hand, and the speaker sprang to a safer distance, obviously acquainted with its weight. Having no desire to be entertained by a cat-and-dog combat; I stepped forward briskly, as if eager to partake the warmth of the hearth, and innocent of any knowledge of the interrupted dispute. Each had enough decorum to suspend further hostilities: Heathcliff placed his fist, out of temptation, in his pockets;Mrs. Heathcliff curled her lip, and walked to a seat far off, where she kept her word by playing the part of a statue during the remainder of my stay. That was not long. I declined joining their breakfast, and, at the first gleam of dawn, took an opportunity of escaping into the free air, now clear, and still, and cold as impalpable ice.
\par My landlord hallooed for me to stop, ere I reached the bottom of the garden, and offered to accompany me across the moor. It was well he did, for the whole hill-back was one billowy, white ocean; the swells and falls not indicating corresponding rises and depressions in the ground: many pits, at least, were filled to a level; and entire ranges of mounds, the refuse of the quarries, blotted from the chart which my yesterday's walk left pictured in my mind.
\par I had remarked on one side of the road, at intervals of six or seven yards, a line of upright stones, continued through the whole length of the barren: these were erected, and daubed with lime on purpose to serve as guides in the dark; and also when a fall, like the present, confounded the deep swamps on either hand with the firmer path: but, excepting a dirty dot pointing up here and there, all traces of their existence had vanished: and my companion found it necessary to warn me frequently to steer to the right or left, when I imagined I was following, correctly, the windings of the road.
\par We exchanged little conversation, and he halted at the entrance of Thrushcross park, saying, I could make no error there. Our adieux were limited to a hasty bow, and then I pushed forward, trusting to my own resources; for the porter's lodge is untenanted as yet.
\par The distance from the gate to the Grange is two miles: I believe I managed to make it four; what with losing myself among the trees, and sinking up to the neck in snow: a predicament which only those who have experienced it can appreciate. At any rate, whatever were my wanderings, the clock chimed twelve as I entered the house; and that gave exactly an hour for every mile of the usual way from Wuthering Heights.
\par My human fixture and her satellites rushed to welcome me; exclaiming, tumultuously, they had completely given me up; everybody conjectured that I perished last night; and they were wondering how they must set about the search for my remains. I bid them be quiet, now that they saw me returned, and, benumbed to my very heart, I dragged upstairs; whence, after putting on dry clothes, and pacing to and fro thirty or forty minutes, to restore the animal heat, I am adjourned to my study, feeble as a kitten: almost too much so to enjoy the cheerful fire and smoking coffee which the servant has prepared for my refreshment.




\subsection*{Chapter 4}


\par What vain weathercocks we are! I, who had determined to hold myself independent of all social intercourse, and thanked my stars that, at length, I had lighted on a spot where it was next to impracticable — I, weak wretch, after maintaining till dusk a struggle with low spirits and solitude, was finally compelled to strike my colours; and, under pretence of gaining information concerning the necessities of my establishment, I desired Mrs. Dean, when she brought in supper, to sit down while I ate it; hoping sincerely she would prove a regular gossip, and either rouse me to animation or lull me to sleep by her talk.
\par “You have lived here a considerable time,” I commenced; “did you not say sixteen years?”
\par “Eighteen, sir: I came, when the mistress was married, to wait on her; after she died, the master retained me for his housekeeper.”
\par “Indeed.”
\par There ensued a pause. She was not a gossip, I feared; unless about her own affairs, and those could hardly interest me. However, having studied for an interval, with a fist on either knee, and a cloud of meditation over her ruddy countenance, she ejaculated:
\par “Ah, times are greatly changed since then!”
\par “Yes,” I remarked, “you've seen a good many alterations, I suppose?”
\par “I have, and troubles too,” she said.
\par “Oh, I'll turn the talk on my landlord's family!” I thought to myself. “A good subject to start — and that pretty girl-widow, I should like to know her history: whether she be a native of the country, or, as is more probable, an exotic that the surly indigene will not recognize for kin.” With this intention I asked Mrs. Dean why Heathcliff let Thrushcross Grange, and preferred living in a situation and residence so much inferior. “Is he not rich enough to keep the estate in good order?” I enquired.
\par “Rich, sir!” she returned. “He has, nobody knows what money, and every year it increases. Yes, yes, he's rich enough to live in a finer house than this: but he's very near — close-handed; and, if he had meant to flit to Thrushcross Grange, as soon as he heard of a good tenant he could not have borne to miss the chance of getting a few hundreds more. It is strange people should be so greedy, when they are alone in the world!”
\par “He had a son, it seems?”
\par “Yes, he had one — he is dead.”
\par “And, that young lady, Mrs. Heathcliff, is his widow?”
\par “Yes.”
\par “Where did she come from originally?”
\par “Why, sir, she is my late master's daughter: Catherine Linton was her maiden name. I nursed her, poor thing! I did wish Mr. Heathcliff would remove here, and then we might have been together again.”
\par “What! Catherine Linton?” I exclaimed, astonished. But a minute's reflection convinced me it was not my ghostly Catherine.
\par “Then,” I continued, “my predecessor's name was Linton?”
\par “It was.”
\par “And who is that Earnshaw, Hareton Earnshaw, who lives with Mr. Heathcliff? Are they relations?”
\par “No; he is the late Mrs. Linton's nephew.”
\par “The young lady's cousin, then?”
\par “Yes; and her husband was her cousin also: one on the mother's, the other on the father's side: Heathcliff married Mr. Linton's sister.”
\par “I see the house at Wuthering Heights has “Earnshaw” carved over the front door. Are they an old family?”
\par “Very old, sir; and Hareton is the last of them, as our Miss Cathy is of us — I mean of the Lintons. Have you been to Wuthering Heights? I beg pardon for asking; but I should like to hear how she is!”
\par “Mrs. Heathcliff? She looked very well, and very handsome; yet, I think, not very happy.”
\par “Oh dear, I don't wonder! And how did you like the master?”
\par “A rough fellow, rather, Mrs. Dean. Is not that his character?”
\par “Rough as a saw edge, and hard as whinstone! The less you meddle with him the better.”
\par “He must have had some ups and downs in life to make him such a churl. Do you know anything of his history?”
\par “It's a cuckoo's, sir — I know all about it: except where he was born, and who were his parents, and how he got his money, at first. And Hareton has been cast out like an unfledged dunnock! The unfortunate lad is the only one in all this parish that does not guess how he has been cheated.”
\par “Well, Mrs. Dean, it will be a charitable deed to tell me something of my neighbours: I feel I shall not rest, if I go to bed; so be good enough to sit and chat an hour.”
\par “Oh, certainly, sir! I'll just fetch a little sewing, and then I'll sit as long as you please. But you've caught cold: I saw you shivering, and you must have some gruel to drive it out.”
\par The worthy woman bustled off, and I crouched nearer the fire; my head felt hot, and the rest of me chill: moreover, I was excited, almost to a pitch of foolishness, through my nerves and brain. This caused me to feel, not uncomfortable, but rather fearful (as I am still) of serious effects from the incidents of today and yesterday. She returned presently, bringing a smoking basin and a basket of work; and, having placed the former on the hob, drew in her seat, evidently pleased to find me so companionable.
\par Before I came to live here, she commenced — waiting no further invitation to her story — I was almost always at Wuthering Heights; because my mother had nursed Mr. Hindley Earnshaw, that was Hareton's father, and I got used to playing with the children: I ran errands too, and helped to make hay, and hung about the farm ready for anything that anybody would set me to. One fine summer morning— it was the beginning of harvest, I remember — Mr. Earnshaw, the old master, came downstairs, dressed for a journey; and after he had told Joseph what was to be done during the day, he turned to Hindley, and Cathy, and me — for I sat eating my porridge with them — and he said, speaking to his son, “Now my bonny man, I'm going to Liverpool today, what shall I bring you? You may choose what you like: only let it be little, for I shall walk there and back: sixty miles each way, that is a long spell!”
\par Hindley named a fiddle, and then he asked Miss Cathy; she was hardly six years old, but she could ride any horse in the stable, and she chose a whip.
\par He did not forget me; for he had a kind heart, though he was rather severe sometimes. He promised to bring me a pocketful of apples and pears, and then he kissed his children goodbye and set off.
\par It seemed a long while to us all — the three days of his absence —and often did little Cathy ask when he would be home. Mrs. Earnshaw expected him by supper time on the third evening, and she put the meal off hour after hour; there were no signs of his coming, however, and at last the children got tired of running down to the gate to look. Then it grew dark; she would have had them to bed, but they begged sadly to be allowed to stay up; and, just about eleven o'clock, the door latch was raised quietly and in stepped the master. He threw himself into a chair, laughing and groaning, and bid them all stand off, for he was nearly killed — he would not have such another walk for the three kingdoms.
\par “And at the end of it, to be flighted to death!” he said, opening his greatcoat, which he held bundled up in his arms. “See here, wife! I was never so beaten with anything in my life: but you must e'en take it as a gift of God; though it's as dark almost as if it came from the devil.”
\par We crowded round, and over Miss Cathy's head, I had a peep at a dirty, ragged, black-haired child; big enough both to walk and talk: indeed, its face looked older than Catherine's; yet, when it was set on its feet, it only stared round, and repeated over and over again some gibberish, that nobody could understand. I was frightened, and Mrs. Earnshaw was ready to fling it out of doors: she did fly up, asking how he could fashion to bring that gipsy brat into the house, when they had their own bairns to feed and fend for? What he meant to do with it, and whether he were mad?
\par The master tried to explain the matter; but he was really half dead with fatigue, and all that I could make out, amongst her scolding, was a tale of his seeing it starving, and houseless, and as good as dumb, in the streets of Liverpool; where he picked it up and inquired for its owner. Not a soul knew to whom it belonged, he said; and his money and time being both limited, he thought it better to take it home with him at once, than run into vain expenses there: because he was determined he would not leave it as he found it.
\par Well, the conclusion was that my mistress grumbled herself calm; and Mr. Earnshaw told me to wash it, and give it clean things, and let it sleep with the children.
\par Hindley and Cathy contented themselves with looking and listening till peace was restored: then, both began searching their father's pockets for the presents he had promised them. The former was a boy of fourteen, but when he drew out what had been a fiddle crushed to morsels in the greatcoat, he blubbered aloud; and Cathy, when she learned the master had lost her whip in attending on the stranger, showed her humour by grinning and spitting at the stupid little thing; earning for her pains a sound blow from her father to teach her cleaner manners.
\par They entirely refused to have it in bed with them, or even in their room; and I had no more sense, so I put it on the landing of the stairs, hoping it might be gone on the morrow. By chance, or else attracted by hearing his voice, it crept to Mr. Earnshaw's door, and there he found it on quitting his chamber. Inquiries were made as to how it got there;I was obliged to confess, and in recompense for my cowardice and inhumanity was sent out of the house.
\par This was Heathcliff's first introduction to the family. On coming back a few days afterwards, for I did not consider my banishment perpetual, I found they had christened him “Heathcliff”: it was the name of a son who died in childhood, and it has served him ever since, both for Christian and surname. Miss Cathy and he were now very thick; but Hindley hated him, and to say the truth I did the same; and we plagued and went on with him shamefully: for I wasn't reasonable enough to feel my injustice, and the mistress never put in a word on his behalf when she saw him wronged.
\par He seemed a sullen, patient child; hardened, perhaps, to illtreatment: he would stand Hindley's blows without winking or shedding a tear, and my pinches moved him only to draw in a breath and open his eyes, as if he had hurt himself by accident and nobody was to blame.
\par This endurance made old Earnshaw furious, when he discovered his son persecuting the poor, fatherless child, as he called him. He took to Heathcliff strangely, believing all he said (for that matter, he said precious little, and generally the truth), and petting him up far above Cathy, who was too mischievous and wayward for a favourite.
\par So, from the very beginning, he bred bad feeling in the house; and at Mrs. Earnshaw's death, which happened in less than two years after, the young master had learned to regard his father as an oppressor rather than a friend, and Heathcliff as a usurper of his parent's affections and his privileges; and he grew bitter with brooding over these injuries.
\par I sympathized awhile; but when the children fell ill of the measles, and I had to tend them, and take on me the cares of a woman at once, I changed my ideas. Heathcliff was dangerously sick: and while he lay at the worst he would have me constantly by his pillow: I suppose he felt I did a good deal for him, and he hadn't wit to guess that I was compelled to do it. However, I will say this, he was the quietest child that ever nurse watched over. The difference between him and the others forced me to be less partial. Cathy and her brother harassed me terribly: he was as uncomplaining as a lamb; though hardness, not gentleness, made him give little trouble.
\par He got through, and the doctor affirmed it was in a great measure owing to me, and praised me for my care. I was vain of his commendations, and softened towards the being by whose means I earned them, and thus Hindley lost his last ally: still I couldn't dote on Heathcliff, and I wondered often what my master saw to admire so much in the sullen boy, who never, to my recollection, repaid his indulgence by any sign of gratitude. He was not insolent to his benefactor, he was simply insensible; though knowing perfectly the hold he had on his heart, and conscious he had only to speak and all the house would be obliged to bend to his wishes.
\par As an instance, I remember Mr. Earnshaw once bought a couple of colts at the parish fair, and gave the lads each one. Heathcliff took the handsomest, but it soon fell lame, and when he discovered it, he said to Hindley —
\par “You must exchange horses with me: I don't like mine; and if you won't I shall tell your father of the three thrashings you've given me this week, and show him my arm, which is black to the shoulder.”
\par Hindley put out his tongue and cuffed him over the ears. “You'd better do it at once,” he persisted, escaping to the porch (they were in the stable): “you will have to; and if I speak of these blows, you'll get them again with interest.”
\par “Off, dog!” cried Hindley, threatening him with an iron weight used for weighing potatoes and hay.
\par “Throw it,” he replied, standing still, “and then I'll tell how you boasted that you would turn me out of doors as soon as he died, and see whether he will not turn you out directly.”
\par Hindley threw it, hitting him on the breast, and down he fell, but staggered up immediately, breathless and white; and, had not I prevented it, he would have gone just so to the master, and got full revenge by letting his condition plead for him, intimating who had caused it.
\par “Take my colt, gipsy, then!” said young Earnshaw. “And I pray that he may break your neck: take him, and be damned, you beggarly interloper! And wheedle my father out of all he has: only afterwards show him what you are, imp of Satan. — And take that, I hope he'll kick out your brains!”
\par Heathcliff had gone to loose the beast, and shift it to his own stall; he was passing behind it, when Hindley finished his speech by knocking him under its feet, and without stopping  to examine whether his hopes were fulfilled, ran away as fast as he could. I was surprised to witness how coolly the child gathered himself up, and went on with his intention; exchanging saddles and all, and then sitting down on a bundle of hay to overcome the qualm which the violent blow occasioned, before he entered the house. I persuaded him easily to let me lay the blame of his bruises on the horse: he minded little what tale was told since he had what he wanted. He complained so seldom, indeed, of such stirs as these, that I really thought him not vindictive: I was deceived completely, as you will hear.

\subsection*{Chapter 5}

\par In the course of time, Mr. Earnshaw began to fail. He had been active and healthy, yet his strength left him suddenly; and when he was confined to the chimney-corner he grew grievously irritable. A nothing vexed him; and suspected slights of his authority nearly threw him into fits. This was especially to be remarked if anyone attempted to impose upon, or domineer over, his favourite: he was painfully jealous lest a word should be spoken amiss to him; seeming to have got into his head the notion that, because he liked Heathcliff, all hated, and longed to do him an ill turn. It was a disadvantage to the lad; for the kinder among us did not wish to fret the master, so we humoured his partiality; and that humouring was rich nourishment to the child's pride and black tempers. Still it became in a manner necessary; twice, or thrice, Hindley's manifestation of scorn, while his father was near, roused the old man to a fury: he seized his stick to strike him, and shook with rage that he could not do it.
\par At last, our curate (we had a curate then who made the living answer by teaching the little Lintons and Earnshaws, and farming his bit of land himself), he advised that the young man should be sent to college; and Mr. Earnshaw agreed, though with a heavy spirit, for he said — “Hindley was nought, and would never thrive as where he wandered.”
\par I hoped heartily we should have peace now. It hurt me to think the master should be made uncomfortable by his own good deed. I fancied the discontent of age and disease arose from his family disagreements: as he would have it that it did: really, you know, sir, it was in his sinking frame. We might have got on tolerably, notwithstanding, but for two people, Miss Cathy and Joseph, the servant: you saw him, I dare say, up yonder. He was, and is yet most likely, the wearisomest self-righteous Pharisee that ever ransacked a Bible to rake the promises to himself and fling the curses on his neighbours. By his knack of sermonizing and pious discoursing, he contrived to make a great impression on Mr. Earnshaw; and the more feeble the master became, the more influence he gained.
\par He was relentless in worrying him about his soul's concerns, and about ruling his children rigidly. He encouraged him to regard Hindley as a reprobate; and, night after night, he regularly grumbled out a long string of tales against Heathcliff and Catherine: always minding to flatter Earnshaw's weakness by heaping the heaviest blame on the last.
\par Certainly, she had ways with her such as I never saw a child take up before; and she put all of us past our patience fifty times and oftener in a day: from the hour she came downstairs till the hour she went to bed, we had not a minute's security that she wouldn't be in mischief. Her spirits were always at high-water mark, her tongue always going— singing, laughing, and plaguing everybody who would not do the same. A wild, wicked slip she was — but she had the bonniest eye, the sweetest smile, and lightest foot in the parish. And, after all, I believe she meant no harm; for when once she made you cry in good earnest, it seldom happened that she would not keep you company, and oblige you to be quiet that you might comfort her.
\par She was much too fond of Heathcliff. The greatest punishment we could invent for her was to keep her separate from him: yet she got chided more than any of us on his account. In play, she liked exceedingly to act the little mistress; using her hands freely, and commanding her companions: she did so to me, but I would not bear slapping and ordering; and so I let her know.
\par Now, Mr. Earnshaw did not understand jokes from his children: he had always been strict and grave with them; and Catherine, on her part, had no idea why her father should be crosser and less patient in his ailing condition, than he was in his prime.
\par His peevish reproofs wakened in her a naughty delight to provoke him: she was never so happy as when we were all scolding her at once, and she defying us with her bold, saucy look, and her ready words turning Joseph's religious curses into ridicule, baiting me, and doing just what her father hated most — showing how her pretended insolence, which he thought real, had more power over Heathcliff than his kindness: how the boy would do her bidding in anything, and his only when it suited his own inclination.
\par After behaving as badly as possible all day, she sometimes came fondling to make it up at night.
\par “Nay, Cathy,” the old man would say, “I cannot love thee; thou'rt worse than thy brother. Go, say thy prayers, child, and ask God's pardon. I doubt thy mother and I must rue that we ever reared thee!”
\par That made her cry, at first; and then being repulsed continually hardened her, and she laughed if I told her to say she was sorry for her faults, and beg to be forgiven.
\par But the hour came, at last, that ended Mr. Earnshaw's troubles on earth. He died quietly in his chair one October evening, seated by the fireside.
\par A high wind blustered round the house, and roared in the chimney: it sounded wild and stormy, yet it was not cold, and we were all together—I, a little removed from the hearth, busy at my knitting, and Joseph reading his Bible near the table (for the servants generally sat in the house then, after their work was done). Miss Cathy had been sick, and that made her still; she leant against her father's knee, and Heathcliff was lying on the floor with his head in her lap.
\par I remember the master, before he fell into a doze, stroking her bonny hair it pleased him rarely to see her gentle — and saying — “Why canst thou not always be a good lass, Cathy?” And she turned her face up to his, and laughed, and answered, “Why cannot you always be a good man, father?”
\par But as soon as she saw him vexed again, she kissed his hand, and said she would sing him to sleep. She began singing very low, till his fingers dropped from hers, and his head sank on his breast. Then I told her to hush, and not stir, for fear she should wake him. We all kept as mute as mice a full half-hour, and should have done longer, only Joseph, having finished his chapter, got up and said that he must rouse the master for prayers and bed. He stepped forward, and called him by name, and touched his shoulder; but he would not move, so he took the candle and looked at him. I thought there was something wrong as he set down the light; and seizing the children each by an arm, whispered them to “frame up-stairs and make little din — they might pray alone that evening — he had summut to do.”
\par “I shall bid father good-night first,” said Catherine, putting her arms round his neck, before we could hinder her. The poor thing discovered her loss directly — she screamed out — “Oh, he's dead, Heathcliff! he's dead!” And they both set up a heart-breaking cry.
\par I joined my wail to theirs, loud and bitter; but Joseph asked what we could be thinking of to roar in that way over a saint in heaven. He told me to put on my cloak and run to Gimmerton for the doctor and the parson. I could not guess the use that either would be of, then. However, I went, through wind and rain, and brought one, the doctor, back with me; the other said he would come in the morning.
\par Leaving Joseph to explain matters, I ran to the children's room: their door was ajar, I saw they had never laid down, though it was past midnight; but they were calmer, and did not need me to console them. The little souls were comforting each other with better thoughts than I could have hit on: no parson in the world ever pictured heaven so beautifully as they did, in their innocent talk: and, while I sobbed and listened, I could not help wishing we were all there safe together.


\subsection*{Chapter 6}

\par Mr. Hindley came home to the funeral; and — a thing that amazed us, and set the neighbours gossiping right and left — he brought a wife with him.
\par What she was, and where she was born, he never informed us: probably she had neither money nor name to recommend her, or he would scarcely have kept the union from his father.
\par She was not one that would have disturbed the house much on her own account. Every object she saw, the moment she crossed the threshold, appeared to delight her; and every circumstance that took place about her: except the preparing for the burial, and the presence of the mourners. I thought she was half silly, from her behaviour while that went on: she ran into her chamber, and made me come with her, though I should have been dressing the children; and there she sat shivering and clasping her hands, and asking repeatedly: “Are they gone yet?” Then she began describing with hysterical emotion the effect it produced on her to see black; and started, and trembled, and, at last, fell a-weeping—and when I asked what was the matter, answered, she didn't know; but she felt so afraid of dying! I imagined her as little likely to die as myself. She was rather thin, but young, and fresh-complexioned, and her eyes sparkled as bright as diamonds. I did remark, to be sure, that mounting the stairs made her breathe very quick: that the least sudden noise set her all in a quiver, and that she coughed troublesomely sometimes: but I knew nothing of what these symptoms portended, and had no impulse to sympathize with her. We don't in general take to foreigners here, Mr. Lockwood, unless they take to us first.
\par Young Earnshaw was altered considerably in the three years of his absence. He had grown sparer, and lost his colour, and spoke and dressed quite differently; and, on the very day of his return, he told Joseph and me we must thenceforth quarter ourselves in the backkitchen, and leave the house for him. Indeed, he would have carpeted and papered a small spare room for a parlour; but his wife expressed such pleasure at the white floor and huge glowing fireplace, at the pewter dishes and delf- case, and dog-kennel, and the wide space there was to move about in where they usually sat, that he thought it unnecessary to her comfort, and so dropped the intention.
\par She expressed pleasure, too, at finding a sister among her new acquaintance; and she prattled to Catherine, and kissed her, and ran about with her, and gave her quantities of presents, at the beginning. Her affection tired very soon, however, and when she grew peevish, Hindley became tyrannical. A few words from her, evincing a dislike to Heathcliff, were enough to rouse in him all his old hatred of the boy. He drove him from their company to the servants, deprived him of the instructions of the curate, and insisted that he should labour out of doors instead; compelling him to do so as hard as any other lad on the farm.
\par He bore his degradation pretty well at first, because Cathy taught him what she learnt, and worked or played with him in the fields. They both promised fair to grow up as rude as savages; the young master being entirely negligent how they behaved, and what they did, so they kept clear of him. He would not even have seen after their going to church on Sundays, only Joseph and the curate reprimanded his carelessness when they absented themselves; and that reminded him to order Heathcliff a flogging, and Catherine a fast from dinner or supper.
\par But it was one of their chief amusements to run away to the moors in the morning and remain there all day, and the after punishment grew a mere thing to laugh at. The curate might set as many chapters as he pleased for Catherine to get by heart, and Joseph might thrash Heathcliff till his arm ached; they forgot everything the minute they were together again: at least the minute they had contrived some naughty plan of revenge; and many a time I've cried to myself to watch them growing more reckless daily, and I not daring to speak a syllable, for fear of losing the small power I still retained over the unfriended creatures.
\par One Sunday evening, it chanced that they were banished from the sitting-room, for making a noise, or a light offence of the kind; and when I went to call them to supper, I could discover them nowhere.
\par We searched the house, above and below, and the yard and stables; they were invisible: and at last, Hindley in a passion told us to bolt the doors, and swore nobody should let them in that night.
\par The household went to bed; and I, too anxious to lie down, opened my lattice and put my head out to hearken, though it rained: determined to admit them in spite of the prohibition, should they return.
\par In a while, I distinguished steps coming up the road, and the light of a lantern glimmered through the gate. I threw a shawl over my head and ran to prevent them from waking Mr. Earnshaw by knocking. There was Heathcliff, by himself: it gave me a start to see him alone.
\par “Where is Miss Catherine?” I cried hurriedly. “No accident, I hope?”
\par “At Thrushcross Grange,” he answered; “and I would have been there too, but they had not the manners to ask me to stay.”
\par “Well, you will catch it!” I said: “you'll never be content till you're sent about your business. What in the world led you wandering to Thrushcross Grange?”
\par “Let me get off my wet clothes, and I'll tell you all about it, Nelly,” he replied.
\par I bid him beware of rousing the master, and while he undressed and I waited to put out the candle, he continued — “Cathy and I escaped from the wash-house to have a ramble at liberty, and getting a glimpse of the Grange lights, we thought we would just go and see whether the Lintons passed their Sunday evenings standing shivering in corners, while their father and mother sat eating and drinking, and singing and laughing; and burning their eyes out before the fire. Do you think they do? Or reading sermons, and being catechized by their manservant, and set to learn a column of Scripture names, if they don't answer properly?”
\par “Probably not,” I responded. “They are good children, no doubt, and don't deserve the treatment you receive, for your bad conduct.”
\par “Don't you cant, Nelly,” he said: “Nonsense! We ran from the top of the Heights to the park, without stopping— Catherine completely beaten in the race, because she was barefoot. You'll have to seek for her shoes in the bog tomorrow. We crept through a broken hedge, groped our way up the path, and planted ourselves on a flower plot under the drawing-room window. The light came from thence; they had not put up the shutters, and the curtains were only half closed. Both of us were able to look in by standing on the basement, and clinging to the ledge, and we saw — ah! it was beautiful — a splendid place carpeted with crimson, and crimson covered chairs and tables, and a pure white ceiling bordered by gold, a shower of glass drops hanging in silver chains from the centre, and shimmering with little soft tapers. Old Mr. and Mrs. Linton were not there; Edgar and his sister had it entirely to themselves. Shouldn't they have been happy? We should have thought ourselves in heaven! And now, guess what your good children were doing? Isabella — I believe she is eleven, a year younger than Cathy — lay screaming at the farther end of the room, shrieking as if witches were running red-hot needles into her. Edgar stood on the hearth weeping silently, and in the middle of the table sat a little dog, shaking its paw and yelping; which, from their mutual accusations, we understood they had nearly pulled in two between them. The idiots! That was their pleasure! To quarrel who should hold a heap of warm hair, and each begin to cry because both, after struggling to get it, refused to take it. We laughed outright at the petted things; we did despise them! When would you catch me wishing to have what Catherine wanted, or find us by ourselves, seeking entertainment in yelling, and sobbing, and rolling on the ground, divided by the whole room? I'd not exchange, for a thousand lives, my condition here, for Edgar Linton's at Thrushcross Grange — not if I might have the privilege of flinging Joseph off the highest gable, and painting the house-front with Hindley's blood!”
\par “Hush, hush!” I interrupted. “Still you have not told me, Heathcliff, how Catherine is left behind?”
\par “I told you we laughed,” he answered. “The Lintons heard us, and with one accord, they shot like arrows to the door; there was silence, and then a cry, ‘Oh, mamma, mamma! Oh, papa! Oh, mamma, come here. Oh, papa, oh!’ They really did howl out something in that way. We made frightful noises to terrify them still more, and then we dropped off the ledge, because somebody was drawing the bars, and we felt we had better flee. I had Cathy by the hand, and was urging her on, when all at once she fell down.
\par ‘Run, Heathcliff, run!’ she whispered. ‘They have let the bulldog loose, and he holds me!’
\par The devil had seized her ankle, Nelly: I heard his abominable snorting. She did not yell out — no! She would have scorned to do it, if she had been spitted on the horns of a mad cow. I did, though! I vociferated curses enough to annihilate any fiend in Christendom; and I got a store and thrust it between his jaws, and tried with all my might to cram it down his throat. A beast of a servant came up with a lantern, at last, shouting — ‘Keep fast, Skulker, keep fast!’ He changed his note, however — when he saw Skulker's game. The dog was throttled off; his huge, purple tongue hanging half a foot out of his mouth, and his pendent lips streaming with bloody slaver. The man took Cathy up: she was sick: not from fear, I'm certain, but from pain. He carried her in; I followed, grumbling execrations and vengeance. ‘What prey, Robert?’ hallooed Linton from the entrance.
\par ‘Skulker has caught a little girl, sir,’ he replied; ‘and there's a lad here', he added, making a clutch at me, ‘who looks an out-and-outer! Very like, the robbers were for putting them through the window to open the doors to the gang after all were asleep, that they might murder us at their ease. Hold your tongue, you foul-mouthed thief, you! you shall go to the gallows for this. Mr. Linton, sir, don't lay by your gun.’
\par ‘No, no, Robert,’ said the old fool. ‘The rascals knew that yesterday was my rent day: they thought to have me cleverly. Come in; I'll furnish them a reception. There, John, fasten the chain. Give Skulker some water, Jenny. To beard a magistrate in his stronghold, and on the Sabbath, too! Where will their insolence stop? Oh, my dear Mary, look here! Don't be afraid, it is but a boy — yet the villain scowls so plainly in his face; would it not be a kindness to the country to hang him at once, before he shows his nature in acts as well as features?’ He pulled me under the chandelier, and Mrs. Linton placed her spectacles on her nose and raised her hands in horror. The cowardly children crept nearer also, Isabella lisping— ‘Frightful thing! Put him in the cellar, papa. He's exactly like the son of the fortune-teller that stole my tame pheasant. Isn't he, Edgar?’
\par “While they examined me, Cathy came round; she heard the last speech, and laughed. Edgar Linton, after an inquisitive stare, collected sufficient wit to recognize her. They see us at church, you know, though we seldom meet them elsewhere. ‘That's Miss Earnshaw!’ he whispered to his mother, ‘and look how Skulker has bitten her — how her foot bleeds!’
\par ‘Miss Earnshaw? Nonsense!’ cried the dame; ‘Miss Earnshaw scouring the country with a gipsy! And yet, my dear, the child is in mourning — surely it is — and she may be lamed for life!’
\par “‘What culpable carelessness in her brother!’ exclaimed Mr. Linton, turning from me to Catherine. ‘I've understood from Shielders’”(that was the curate, sir) “‘that he lets her grow up in absolute heathenism. But who is this? Where did she pick up this companion? Oho! I declare he is that strange acquisition my late neighbour made, in his journey to Liverpool — a little Lascar,’ or an American or Spanish castaway.’
\par “‘A wicked boy, at all events,’ remarked the old lady, ‘and quite unfit for a decent house! Did you notice his language, Linton? I'm shocked that my children should have heard it.’
\par “I recommenced cursing — don't be angry, Nelly — and so Robert was ordered to take me off. I refused to go without Cathy; he dragged me into the garden, pushed the lantern into my hand, assured me that Mr. Earnshaw should be informed of my behaviour, and, bidding me march directly, secured the door again. The curtains were still looped up at one corner, and I resumed my station as spy; because, if Catherine had wished to return, I intended shattering their great glass panes to a million of fragments, unless they let her out. She sat on the sofa quietly. Mrs. Linton took off the grey cloak of the dairymaid which we had borrowed for our excursion, shaking her head and expostulating with her, I suppose: she was a young lady, and they made a distinction between her treatment and mine. Then the woman-servant brought a basin of warm water, and washed her feet; and Mr. Linton mixed a tumbler of negus, and Isabella emptied a plateful of cakes into her lap, and Edgar stood gaping at a distance. Afterwards, they dried and combed her beautiful hair, and gave her a pair of enormous slippers, and wheeled her to the fire; and I left her, as merry as she could be, dividing her food between the little dog and Skulker, whose nose she pinched as he ate; and kindling a spark of spirit in the vacant blue eyes of the Lintons — a dim reflection from her own enchanting face. I saw they were full of stupid admiration; she is so immeasurably superior to them— to everybody on earth, is she not, Nelly?”
\par “There will more come of this business than you reckon on,” I answered, covering him up and extinguishing the light. “You are incurable, Heathcliff; and Mr. Hindley will have to proceed to extremities, see if he won't.”
\par My words came truer than I desired. The luckless adventure made Earnshaw furious. And then Mr. Linton, to mend matters, paid us a visit himself on the morrow; and read the young master such a lecture on the road he guided his family, that he was stirred to look about him, in earnest. Heathcliff received no flogging, but he was told that the first word he spoke to Miss Catherine should ensure a dismissal; and Mrs. Earnshaw undertook to keep her sister-in-law in due restraint when she returned home; employing art, not force: with force she would have found it impossible.









\subsection*{Chapter 7}

\par Cathy stayed at Thrushcross Grange five weeks, till Christmas. By that time her ankle was thoroughly cured, and her manners much improved. The mistress visited her often in the interval, and commenced her plan of reform by trying to raise her self-respect with fine clothes and flattery, which she took readily; so that, instead of a wild, hatless little savage jumping into the house, and rushing to squeeze us all breathless, there lighted from a handsome black pony a very dignified person, with brown ringlets falling from the cover of a feathered beaver, and a long cloth habit, which she was obliged to hold up with both hands that she might sail in.
\par Hindley lifted her from her horse, exclaiming delightedly, “Why, Cathy, you are quite a beauty! I should scarcely have known you: you look like a lady now. Isabella Linton is not to be compared with her, is she, Frances?”
\par “Isabella has not her natural advantages,” replied his wife: “but she must mind and not grow wild again here. Ellen, help Miss Catherine off with her things — stay, dear, you will disarrange your curls — let me untie your hat.”
\par I removed the habit, and there shone forth, beneath a grand plaid silk frock, white trousers, and burnished shoes; and, while her eyes sparkled joyfully when the dogs came bounding up to welcome her, she dared hardly touch them lest they should fawn upon her splendid garments.
\par She kissed me gently: I was all flour making the Christmas cake, and it would not have done to give me a hug; and then she looked round for Heathcliff. Mr. and Mrs. Earnshaw watched anxiously their meeting; thinking it would enable them to judge, in some measure, what grounds they had for hoping to succeed in separating the two friends.
\par Heathcliff was hard to discover, at first. If he were careless, and uncared for, before Catherine's absence, he had been ten times more so since. Nobody but I even did him the kindness to call him a dirty boy, and bid him wash himself, once a week; and children of his age seldom have a natural pleasure in soap and water. Therefore, not to mention his clothes, which had seen three months' service in mire and dust, and his thick uncombed hair, the surface of his face and hands was dismally beclouded. He might well skulk behind the settle, on beholding such a bright, graceful damsel enter the house, instead of a rough-headed counterpart of himself, as he expected.
\par “Is Heathcliff not here?” she demanded, pulling off her gloves, and displaying fingers wonderfully whitened with doing nothing and staying indoors.
\par “Heathcliff, you may come forward,” cried Mr. Hindley, enjoying his discomfiture, and gratified to see what a forbidding young blackguard he would be compelled to present himself. “You may come and wish Miss Catherine welcome, like the other servants.”
\par Cathy, catching a glimpse of her friend in his concealment, flew to embrace him; she bestowed seven or eight kisses on his cheek within the second, and then stopped, and drawing back, burst into a laugh, exclaiming, “Why, how very black and cross you look! And how —how funny and grim! But that's because I'm used to Edgar and Isabella Linton. Well, Heathcliff, have you forgotten me?”
\par She had some reason to put the question, for shame and pride threw double gloom over his countenance, and kept him immovable.
\par “Shake hands, Heathcliff,” said Mr. Earnshaw, condescendingly;“once in a way, that is permitted.”
\par “I shall not,” replied the boy, finding his tongue at last; “I shall not stand to be laughed at. I shall not bear it!”
\par And he would have broken from the circle, but Miss Cathy seized him again.
\par “I did not mean to laugh at you,” she said; “I could not hinder myself: Heathcliff, shake hands at least! What are you sulky for? It was only that you looked odd. If you wash your face and brush your hair, it will be all right: but you are so dirty!”
\par She gazed concernedly at the dusky fingers she held in her own, and also at her dress; which she feared had gained no embellishment from its contact with his.
\par “You needn't have touched me!” he answered, following her eye and snatching away his hand. “I shall be as dirty as I please: and I like to be dirty, and I will be dirty.”
\par With that he dashed head foremost out of the room, amid the merriment of the master and mistress, and to the serious disturbance of Catherine; who could not comprehend how her remarks should have produced such an exhibition of bad temper.
\par After playing lady's-maid to the new-comer, and putting my cakes in the oven, and making the house and kitchen cheerful with great fires, befitting Christmas Eve, I prepared to sit down and amuse myself by singing carols, all alone; regardless of Joseph's affirmations that he considered the merry tunes I chose as next door to songs.
\par He had retired to private prayer in his chamber, and Mr. and Mrs. Earnshaw were engaging Missy's attention by sundry gay trifles bought for her to present to the little Lintons, as an acknowledgment of their kindness. They had invited them to spend the morrow at Wuthering Heights, and the invitation had been accepted, on one condition: Mrs. Linton begged that her darlings must be kept carefully apart from that “naughty swearing boy”.
\par Under these circumstances I remained solitary. I smelt the rich scent of the heating spices; and admired the shining kitchen utensils, the polished clock, decked in holly, the silver mugs ranged on a tray ready to be filled with mulled ale for supper; and above all, the speckless purity of my particular care — the scoured and well-swept floor.
\par I gave due inward applause to every object, and then I remembered how old Earnshaw used to come in when all was tidied, and call me a cant lass, and slip a shilling into my hand as a Christmas-box; and from that I went on to think of his fondness for Heathcliff, and his dread lest he should suffer neglect after death had removed him; and that naturally led me to consider the poor lad's situation now, and from singing I changed my mind to crying. It struck me soon, however, there would be more sense in endeavouring to repair some of his wrongs than shedding tears over them: I got up and walked into the court to seek him.
\par He was not far; I found him smoothing the glossy coat of the new pony in the stable, and feeding the other beasts, according to custom.
\par “Make haste, Heathcliff!” I said, “the kitchen is so comfortable; and Joseph is upstairs: make haste, and let me dress you smart before Miss Cathy comes out, and then you can sit together, with the whole hearth to yourselves, and have a long chatter till bedtime.”
\par He proceeded with his task and never turned his head towards me.
\par “Come — are you coming?” I continued. “There's a little cake for each of you, nearly enough; and you'll need half-an-hour's donning.”
\par I waited five minutes, but getting no answer, left him. Catherine supped with her brother and sister-in-law: Joseph and I joined in an unsociable meal, seasoned with reproofs on one side and sauciness on the other. His cake and cheese remained on the table all night for the fairies. He managed to continue work till nine o'clock, and then marched dumb and dour to his chamber.
\par Cathy sat up late, having a world of things to order for the reception of her new friends: she came into the kitchen once to speak to her old one; but he was gone, and she only stayed to ask what was the matter with him, and then went back. In the morning he rose early; and as it was a holiday carried his ill humour on to the moors; not reappearing till the family were departed for church. Fasting and reflection seemed to have brought him to a better spirit. He hung about me for a while, and having screwed up his courage, exclaimed abruptly:
\par “Nelly, make me decent, I'm going to be good.”
\par “High time, Heathcliff,” I said; “you have grieved Catherine: she's sorry she ever came home, I dare say! It looks as if you envied her, because she is more thought of than you.”
\par The notion of envying Catherine was incomprehensible to him, but the notion of grieving her he understood clearly enough.
\par “Did she say she was grieved?” he inquired, looking very serious.
\par “She cried when I told her you were off again this morning.”
\par “Well, I cried last night,” he returned, “and I had more reason to cry than she.”
\par “Yes: you had the reason of going to bed with a proud heart and an empty stomach,” said I. “Proud people breed sad sorrows for themselves. But, if you be ashamed of your touchiness, you must ask pardon, mind, when she comes in. You must go up and offer to kiss her, and say — you know best what to say; only do it heartily, and not as if you thought her converted into a stranger by her grand dress. And now, though I have dinner to get ready, I'll steal time to arrange you so that Edgar Linton shall look quite a doll beside you: and that he does. You are younger, and yet, I'll be bound, you are taller and twice as broad across the shoulders; you could knock him down in a twinkling; don't you feel that you could?”
\par Heathcliff's face brightened a moment; then it was overcast afresh, and he sighed.
\par “But, Nelly, if I knocked him down twenty times, that wouldn't make him less handsome or me more so. I wish I had light hair and a fair skin, and was dressed and behaved as well, and had a chance of being as rich as he will be!”
\par “And cried for mamma at every turn,” I added, “and trembled if a country lad heaved his fist against you, and sat at home all day for a shower of rain. Oh, Heathcliff, you are showing a poor spirit! Come to the glass, and I'll let you see what you should wish. Do you mark those two lines between your eyes; and those thick brows, that instead of rising arched, sink in the middle; and that couple of black fiends, so deeply buried, who never open their windows boldly, but lurk glinting under them, like devil's spies? Wish and learn to smooth away the surly wrinkles, to raise your lids frankly, and change the fiends to confident, innocent angels, suspecting and doubting nothing, and always seeing friends where they are not sure of foes. Don't get the expression of a vicious cur that appears to know the kicks it gets are its desert, and yet hates all the world as well as the kicker, for what it suffers.”
\par “In other words, I must wish for Edgar Linton's great blue eyes and even forehead,” he replied. “I do — and that won't help me to them.”
\par “A good heart will help you to a bonny face, my lad,” I continued,“if you were a regular black; and a bad one will turn the bonniest into something worse than ugly. And now that we've done washing, and combing, and sulking — tell me whether you don't think yourself rather handsome? I'll tell you, I do. You're fit for a prince in disguise. Who knows but your father was Emperor of China, and your mother an Indian queen, each of them able to buy up, with one week's income, Wuthering Heights and Thrushcross Grange together? And you were kidnapped by wicked sailors and brought to England. Were I in your place, I would frame high notions of my birth; and the thoughts of what I was should give me courage and dignity to support the oppressions of a little farmer!”
\par So I chattered on; and Heathcliff gradually lost his frown and began to look quite pleasant, when all at once our conversation was interrupted by a rumbling sound moving up the road and entering the court. He ran to the window and I to the door, just in time to behold the two Lintons descend from the family carriage, smothered in cloaks and furs, and the Earnshaws dismount from their horses: they often rode to church in winter. Catherine took a hand of each of the children, and brought them into the house and set them before the fire, which quickly put colour into their white faces.
\par I urged my companion to hasten now and show his amiable humour, and he willingly obeyed; but ill luck would have it that, as he opened the door leading from the kitchen on one side, Hindley opened it on the other. They met, and the master, irritated at seeing him clean and cheerful; or, perhaps, eager to keep his promise to Mrs. Linton, shoved him back with a sudden thrust, and angrily bade Joseph “keep the fellow out of the room — send him into the garret till dinner is over. He'll be cramming his fingers in the tarts and stealing the fruit, if left alone with them a minute.”
\par “Nay, sir,” I could not avoid answering, “he'll touch nothing, not he: and I suppose he must have his share of the dainties as well as we.”
\par “He shall have his share of my hand, if I catch him downstairs again till dark,” cried Hindley. “Begone, you vagabond! What! You are attempting the coxcomb, are you? Wait till I get hold of those elegant locks — see if I won't pull them a bit longer.”
\par “They are long enough, already,” observed Master Linton, peeping from the doorway; “I wonder they don't make his head ache. It's like a colt's mane over his eyes!”
\par He ventured his remark without any intention to insult; but Heathcliff's violent nature was not prepared to endure the appearance of impertinence from one whom he seemed to hate, even then, as a rival. He seized a tureen of hot apple sauce, the first thing that came under his gripe, and dashed it full against the speaker's face and neck; who instantly commenced a lament that brought Isabella and Catherine hurrying to the place. Mr. Earnshaw snatched up the culprit directly and conveyed him to his chamber; where, doubtless, he administered a rough remedy to cool the fit of passion, for he reappeared red and breathless. I got the dishcloth and rather spitefully scrubbed Edgar's nose and mouth, affirming it served him right for meddling. His sister began weeping to go home, and Cathy stood by confounded, blushing for all.
\par “You should not have spoken to him!” she expostulated with Master Linton. “He was in a bad temper, and now you've spoilt your visit; and he'll be flogged: I hate him to be flogged! I can't eat my dinner. Why did you speak to him, Edgar?”
\par “I didn't,” sobbed the youth, escaping from my hands, and finishing the remainder of the purification with his cambric pocket handkerchief. “I promised mamma that I wouldn't say one word to him, and I didn't.”
\par “Well, don't cry,” replied Catherine, contemptuously, “you're not killed. Don't make more mischief; my brother is coming: be quiet! Give over, Isabella! Has anybody hurt you?”
\par “There, there, children — to your seats!” cried Hindley, bustling in. “That brute of a lad has warmed me nicely. Next time, Master Edgar, take the law into your own fists — it will give you an appetite!”
\par The little party recovered its equanimity at sight of the fragrant feast. They were hungry after their ride, and easily consoled, since no real harm had befallen them.
\par Mr. Earnshaw carved bountiful platefuls, and the mistress made them merry with lively talk. I waited behind her chair, and was pained to behold Catherine, with dry eyes and an indifferent air, commence cutting up the wing of a goose before her. “An unfeeling child,” I thought to myself; “how lightly she dismisses her old playmate's troubles. I could not have imagined her to be so selfish.”
\par She lifted a mouthful to her lips; then she set it down again: her cheeks flushed, and the tears gushed over them. She slipped her fork to the floor, and hastily dived under the cloth to conceal her emotion. I did not call her unfeeling long; for I perceived she was in purgatory through out the day, and wearying to find an opportunity of getting by herself, or paying a visit to Heathcliff, who had been locked up by the master: as I discovered, on endeavouring to introduce to him private mess of victuals.
\par In the evening we had a dance. Cathy begged that he might be liberated then, as Isabella Linton had no partner; her entreaties were vain, and I was appointed to supply the deficiency. We got rid of all gloom in the excitement of the exercise, and our pleasure was increased by the arrival of the Gimmerton band, mustering fifteen strong: a trumpet, a trombone, clarionets, bassoons, French horns, and a bass viol, besides singers. They go the rounds of all the respectable houses, and receive contributions every Christmas, and we esteemed it a firstrate treat to hear them. After the usual carols had been sung, we set them to songs and glees. Mrs. Earnshaw loved the music, and so they gave us plenty.
\par Catherine loved it too; but she said it sounded sweetest at the top of the steps, and she went up in the dark: I followed. They shut the house door below, never noting our absence, it was so full people. She made no stay at the stair's head, but mounted farther, to the garret where Heathcliff was confined, and called him. I stubbornly declined answering for a while; she persevered, and finally persuaded him to hold communion with her through the boards.
\par I let the poor things converse unmolested, till I supposed the songs were going to cease, and the singers to get some refreshment; then I clambered up the ladder to warn her.
\par Instead of finding her outside, I heard her voice within. The little monkey had crept by the skylight of one garret, along the roof, into the skylight of the other, and it was with the utmost difficulty I could coax her out again. When she did come Heathcliff came with her, and she insisted that I should take him into the kitchen, as my fellow-servant had gone to a neighbour's to be removed from the sound of our “devil's psalmody,” as it pleased him to call it. I told them I intended by no means to encourage their tricks; but as the prisoner had never broken his fast since yesterday's dinner, I would wink at his cheating Mr. Hindley that once. He went down; I set him a stool by the fire, and offered him a quantity of good things; but he was sick and could eat little, and my attempts to entertain him were thrown away. He leant his two elbows on his knees, and his chin on his hands, and remained wrapt in dumb meditation. On my inquiring the subject of his thoughts, he answered gravely:
\par “I'm trying to settle how I shall pay Hindley back. I don't care how long I wait, if I can only do it at last. I hope he will not die before I do!”
\par “For shame, Heathcliff!” said I. “It is for God to punish wicked people; we should learn to forgive.”
\par “No, God won't have the satisfaction that I shall,” he returned, “I only wish I knew the best way! Let me alone, and I'll plan it out: while I'm thinking of that I don't feel pain.”
\par “But, Mr. Lockwood, I forget these tales cannot divert you. I'm annoyed how I should dream of chattering on at such a rate; and your gruel cold, and you nodding for bed! I could have told Heathcliff's history, all that you need hear, in half a dozen words.”
\par Thus interrupting herself, the housekeeper rose, and proceeded to lay aside her sewing; but I felt incapable of moving from the hearth, and I was very far from nodding.
\par “Sit still, Mrs. Dean,” I cried, “do sit still, another half-hour! You've done just right to tell the story leisurely. That is the method I like; and you must finish it in the same style. I am interested in every character you have mentioned, more or less.”
\par “The clock is on the stroke of eleven, sir.”
\par “No matter — I'm not accustomed to go to bed in the long hours. One or two is early enough for a person who lies till ten.”
\par “You shouldn't lie till ten. There's the very prime of the morning gone long before that time. A person who has not done one half his day's work by ten o'clock, runs a chance of leaving the other half undone.”
\par “Nevertheless, Mrs. Dean, resume your chair; because to morrow I intend lengthening the night till afternoon. I prognosticate for myself an obstinate cold, at least.”
\par “I hope not, sir. Well, you must allow me to leap over some three years; during that space Mrs. Earnshaw —”
\par “No, no, I'll allow nothing of the sort! Are you acquainted with the mood of mind in which, if you were seated alone, and the cat licking its kitten on the rug before you, you would watch the operation so intently that puss's neglect of one ear would put you seriously out of temper?”
\par “A terribly lazy mood, I should say.”
\par “On the contrary, a tiresomely active one. It is mine, at present; and, therefore, continue minutely. I perceive that people in these regions acquire over people in towns the value that the spider in a dungeon does over a spider in a cottage, to their various occupants; and yet the deepened attraction is not entirely owing to the situation of the lookeron. They do live more in earnest, more in themselves, and less in surface change, and frivolous external things. I could fancy a love for life here almost possible; and I was a fixed unbeliever in any love of a year's standing. One state resembles setting a hungry man down to a single dish, on which he may concentrate his entire appetite and do it justice; the other, introducing him to a table laid out by French cooks: he can perhaps extract as much enjoyment from the whole; but each part is a mere atom in his regard and remembrance.”
\par “Oh! here we are the same as anywhere else, when you get to know us,” observed Mrs. Dean, somewhat puzzled at my speech.
\par “Excuse me,” I responded; “you, my good friend, are a striking evidence against that assertion. Excepting a few provincialisms of slight consequence, you have no marks of the manners which I am habituated to consider as peculiar to your class. I am sure you have thought a great deal more than the generality of servants think. You have been compelled to cultivate your reflective faculties for want of occasions for frittering your life away in silly trifles.”
\par Mrs. Dean laughed.
\par “I certainly esteem myself a steady, reasonable kind of body,” she said, “not exactly from living among the hills and seeing one set of faces, and one series of actions, from year's end to year's end; but I have undergone sharp discipline, which has taught me wisdom; and then, I have read more than you would fancy, Mr. Lockwood. You could not open a book in this library that I have not looked into, and got something out of also: unless it be that range of Greek and Latin and that of French; and those I know one from another: it is as much as you can expect of a poor man's daughter.”
\par “However, if I am to follow my story in true gossip's fashion, I had better go on; and instead of leaping three years, I will be content to pass to the next summer — the summer of 1778, that is, nearly twenty-three years ago.”









\subsection*{Chapter 8}


\par On the morning of a fine June day, my first bonny little nursling, and the last of the ancient Earnshaw stock, was born. We were busy with the hay in a far away field, when the girl that usually brought our breakfasts, came running an hour too soon, across the meadow and up the lane, calling me as she ran.
\par “Oh, such a grand bairn!” she panted out. “The finest lad that ever breathed! But the doctor says missis must go: he says she's been in a consumption these many months. I heard him tell Mr. Hindley: and now she has nothing to keep her, and she'll be dead before winter. You must come home directly. You're to nurse it, Nelly: to feed it with sugar and milk, and take care of it day and night. I wish I were you, because it will be all yours when there is no missis!”
\par “But is she very ill?” I asked, flinging down my rake, and tying my bonnet.
\par “I guess she is; yet she looks bravely,” replied the girl, “and she talks as if she thought of living to see it grow a man. She's out of her head for joy, it's such a beauty! If I were her, I'm certain I should not die: I should get better at the bare sight of it, in spite of Kenneth. I was fairly mad at him. Dame Archer brought the cherub down to master, in the house, and his face just began to light up, then the old croaker steps forward, and says he: ‘Earnshaw, it's a blessing your wife has been spared to leave you this son. When she came, I felt convinced we shouldn't keep her long; and now, I must tell you, the winter will probably finish her. Don't take on, and fret about it too much! it can't be helped. And besides, you should have known better than to choose such a rush of a lass!”
\par “And what did the master answer?” I enquired.
\par “I think he swore: but I didn't mind him, I was straining to see the bairn,” and she began again to describe it rapturously. I, as zealous as herself, hurried eagerly home to admire, on my part; though I was very sad for Hindley's sake. He had room in his heart only for two idols —his wife and himself: he doted on both, and adored one, and I couldn't conceive how he would bear the loss.
\par When we got to Wuthering Heights, there he stood at the front door; and, as I passed in, I asked, “How was the baby?”
\par “Nearly ready to run about, Nell!” he replied, putting on a cheerful smile.
\par “And the mistress?” I ventured to inquire; “the doctor says she's—“
\par “Damn the doctor!” he interrupted, reddening. “Frances is quite right; she'll be perfectly well by this time next week. Are you going upstairs? will you tell her that I'll come, if she'll promise not to talk. I left her because she would not hold her tongue; and she must — tell her Mr. Kenneth says she must be quiet.”
\par I delivered this message to Mrs. Earnshaw; she seemed in flighty spirits, and replied merrily:
\par “I hardly spoke a word, Ellen, and there he has gone out twice, crying. Well, say I promise I won't speak: but that does not bind me not to laugh at him!”
\par Poor soul! Till within a week of her death that gay heart never failed her, and her husband persisted doggedly, nay, furiously, in affirming her health improved every day. When Kenneth warned him that his medicines were useless at that stage of the malady, and he needn't put him to further expense by attending her, he retorted:“I know you need not — she's well — she does not want any more attendance from you! She never was in a consumption. It was a fever; and it is gone: her pulse is as slow as mine now, and her cheek as cool.”
\par He told his wife the same story, and she seemed to believe him; but one night, while leaning on his shoulder, in the act of saying she thought she should be able to get up tomorrow, a fit of coughing took her — a very slight one — he raised her in his arms; she put her two hands about his neck, her face changed, and she was dead.
\par As the girl had anticipated, the child Hareton fell wholly into my hands. Mr. Earnshaw, provided he saw him healthy and never heard him cry, was contented, as far as regarded him. For himself, he grew desperate: his sorrow was of that kind that will not lament. He neither wept nor prayed; he cursed and defied: execrated God and man, and gave himself up to reckless dissipation. The servants could not bear his tyrannical and evil conduct long: Joseph and I were the only two that would stay. I had not the heart to leave my charge; and besides, you know I had been his foster-sister, and excused his behaviour more readily than a stranger would.
\par Joseph remained to hector over tenants and labourers; and because it was his vocation to be where he had plenty of wickedness to reprove.
\par The master's bad ways and bad companions formed a pretty example for Catherine and Heathcliff. His treatment of the latter was enough to make a fiend of a saint. And, truly, it appeared as if the lad were possessed of something diabolical at that period. He delighted to witness Hindley degrading himself past redemption; and became daily more notable for savage sullenness and ferocity.
\par I could not half tell what an infernal house we had. The curate dropped calling, and nobody decent came near us, at last; unless Edgar Linton's visits to Miss Cathy might be an exception. At fifteen she was the queen of the countryside; she had no peer; and she did turn out a haughty, headstrong creature! I own I did not like her, after her infancy was past; and I vexed her frequently by trying to bring down her arrogance: she never took an aversion to me, though. She had a wondrous constancy to old attachments: even Heathcliff kept his hold on her affections unalterably; and young Linton, with all his superiority, found it difficult to make an equally deep impression.
\par He was my late master; that is his portrait over the fireplace. It used to hang on one side, and his wife's on the other; but hers has been removed, or else you might see something of what she was. Can you make that out?
\par 
\par Mrs. Dean raised the candle, and I discerned a soft-featured face, exceedingly resembling the young lady at the Heights, but more pensive and amiable in expression. It formed a sweet picture. The long light hair curled slightly on the temples; the eyes were large and serious; the figure almost too graceful. I did not marvel how Catherine Earnshaw could forget her first friend for such an individual. I marvelled much how he, with a mind to correspond with his person, could fancy my idea of Catherine Earnshaw.
\par “A very agreeable portrait,” I observed to the housekeeper. “Is it like?”
\par “Yes,” she answered; “but he looked better when he was animated; that is his everyday countenance: he wanted spirit in general.”
\par 
\par Catherine had kept up her acquaintance with the Lintons since her five weeks' residence among them; and as she had no temptation to show her rough side in their company, and had the sense to be ashamed of being rude where she experienced such invariable courtesy, she imposed unwittingly on the old lady and gentleman, by her ingenious cordiality; gained the admiration of Isabella, and the heart and soul of her brother: acquisitions that flattered her from the first — for she was full of ambition — and led her to adopt a double character without exactly intending to deceive anyone.
\par In the place where she heard Heathcliff termed a “vulgar young ruffian,” and “worse than a brute”, she took care not to act like him; but at home she had small inclination to practise politeness that would only be laughed at, and restrain an unruly nature when it would bring her neither credit nor praise.
\par Mr. Edgar seldom mustered courage to visit Wuthering Heights openly. He had a terror of Earnshaw's reputation, and shrunk from encountering him; and yet he was always received with our best attempts at civility: the master himself avoided offending him, knowing why he came; and if he could not be gracious, kept out of the way. I rather think his appearance there was distasteful to Catherine: she was not artful, never played the coquette, and had evidently an objection to her two friends meeting at all; for when Heathcliff expressed contempt of Linton in his presence, she could not half coincide, as she did in his absence; and when Linton evinced disgust and antipathy to Heathcliff, she dared not treat his sentiments with indifference, as if depreciation of her playmate were of scarcely any consequence to her. I've had many a laugh at her perplexities and untold troubles, which she vainly strove to hide from my mockery. That sounds ill-natured: but she was so proud, it became really impossible to pity her distresses, till she should be chastened into more humility. She did bring herself, finally, to confess, and confide in me: there was not a soul else that she might fashion into an adviser.
\par Mr. Hindley had gone from home one afternoon, and Heathcliff presumed to give himself a holiday on the strength of it. He had reached the age of sixteen then, I think, and without having bad features, or being deficient in intellect, he contrived to convey an impression of inward and outward repulsiveness that his present aspect retains no traces of.
\par In the first place, he had by that time lost the benefit of his early education: continual hard work, begun soon and concluded late, had extinguished any curiosity he once possessed in pursuit of knowledge, and any love for books or learning. His childhood's sense of superiority, instilled into him by the favours of old Mr. Earnshaw, was faded away. He struggled long to keep up an equality with Catherine in her studies, and yielded with poignant though silent regret: but he yielded completely; and there was no prevailing on him to take a step in the way of moving upward, when he found he must, necessarily, sink beneath his former level. Then personal appearance sympathized with mental deterioration: he acquired a slouching gait, and ignoble look; his naturally reserved disposition was exaggerated into an almost idiotic excess of unsociable moroseness; and he took a grim pleasure, apparently, in exciting the aversion rather than the esteem of his few acquaintance.
\par Catherine and he were constant companions still at his seasons of respite from labour; but he had ceased to express his fondness for her in words, and recoiled with angry suspicion from her girlish caresses, as if conscious there could be no gratification in lavishing such marks of affection on him. On the before-named occasion he came into the house to announce his intention of doing nothing, while I was assisting Miss Cathy to arrange her dress: she had not reckoned on his taking it into his head to be idle; and imagining she would have the whole place to herself, she managed, by some means, to inform Mr. Edgar of her brother's absence, and was then preparing to receive him.
\par “Cathy, are you busy, this afternoon?” asked Heathcliff. “Are you going anywhere?”
\par “No, it is raining,” she answered.
\par “Why have you that silk frock on, then?” he said. “Nobody coming here, I hope?”
\par “Not that I know of,” stammered Miss: “but you should be in the field now, Heathcliff. It is an hour past dinner time: I thought you were gone.”
\par “Hindley does not often free us from his accursed presence,” observed the boy. “I'll not work any more today: I'll stay with you.”
\par “Oh, but Joseph will tell,” she suggested; “you'd better go!”
\par “Joseph is loading lime on the farther side of pennistow Crag; it will take him till dark, and he'll never know.”
\par So saying, he lounged to the fire, and sat down. Catherine reflected an instant, with knitted brows — she found it needful to smooth the way for an intrusion. “Isabella and Edgar Linton talked of calling this afternoon,” she said, at the conclusion of a minute's silence. “As it rains, I hardly expect them; but they may come, and if they do, you run the risk of being scolded for no good.”
\par “Order Ellen to say you are engaged, Cathy,” he persisted; “don't turn me out for those pitiful, silly friends of yours! I'm on the point, sometimes, of complaining that they — but I'll not —”
\par “That they what?” cried Catherine, gazing at him with a troubled countenance. “Oh, Nelly!” she added petulantly, jerking her head away from my hands, “you've combed my hair quite out of curl! That's enough; let me alone. What are you on the point of complaining about, Heathcliff?”
\par “Nothing — only look at the almanac on that wall”; he pointed to a framed sheet hanging near the window, and continued, “The crosses are for the evenings you have spent with the Lintons, the dots for those spent with me. Do you see? I've marked every day.”
\par “Yes — very foolish: as if I took notice!” replied Catherine in a peevish tone. “And where is the sense of that?”
\par “To show that I do take notice,” said Heathcliff.
\par “And should I always be sitting with you?” she demanded, growing more irritated. “What good do I get? What do you talk about? You might be dumb, or a baby, for anything you say to amuse me, or for anything you do, either!”
\par “You never told me before that I talked too little, or that you disliked my company, Cathy!” exclaimed Heathcliff, in much agitation.
\par “It's no company at all, when people know nothing and say nothing,” she muttered.
\par Her companion rose up, but he hadn't time to express his feelings further, for a horse's feet were heard on the flags, and having knocked gently, young Linton entered, his face brilliant with delight at the unexpected summons he had received.
\par Doubtless Catherine marked the difference between her friends, as one came in and the other went out. The contrast resembled what you see in exchanging a bleak, hilly, coal country for a beautiful fertile valley; and his voice and greeting were as opposite as his aspect. He had a sweet, low manner of speaking, and pronounced his words as you do: that's less gruff than we talk here, and softer.
\par “I'm not come too soon, am I?” he said, casting a look at me: I had begun to wipe the plate, and tidy some drawers at the far end in the dresser.
\par “No,” answered Catherine. “What are you doing there, Nelly?”
\par “My work, miss,” I replied. (Mr. Hindley had given me directions to make a third parry in any private visits Linton chose to pay.)
\par She stepped behind me and whispered crossly, “Take yourself and your dusters off; when company are in the house, servants don't commence scouring and cleaning in the room where they are!”
\par “It's a good opportunity, now that the master is away,” I answered aloud: “he hates me to be fidgeting over these things in his presence. I'm sure Mr. Edgar will excuse me.”
\par “I hate you to be fidgeting in my presence,” exclaimed the young lady imperiously, not allowing her guest time to speak: she had failed to recover her equanimity since the little dispute with Heathcliff.
\par “I'm sorry for it, Miss Catherine,” was my response; and I proceeded assiduously with my occupation.
\par She, supposing Edgar could not see her, snatched the cloth from my hand, and pinched me, with a prolonged wrench, very spitefully on the arm. I've said I did not love her, and rather relished mortifying her vanity now and then; besides, she hurt me extremely; so I started up from my knees, and screamed out, “Oh, miss, that's a nasty trick! You have no right to nip me, and I'm not going to bear it.”
\par “I didn't touch you, you lying creature!” cried she, her fingers tingling to repeat the act, and her ears red with rage. She never had power to conceal her passion, it always set her whole complexion in a blaze.
\par “What's that, then?” I retorted, showing a decided purple witness to refute her.
\par She stamped her foot, wavered a moment, and then irresistibly impelled by the naughty spirit within her, slapped me on the cheek a stinging blow that filled both eyes with water.
\par “Catherine, love! Catherine!” interposed Linton, greatly shocked at the double fault of falsehood and violence which his idol had committed.
\par “Leave the room, Ellen!” she repeated, trembling all over.
\par Little Hareton, who followed me everywhere, and was sitting near me on the floor, at seeing my tears commenced crying himself, and sobbed out complaints against “wicked aunt Cathy”, which drew her fury on to his unlucky head: she seized his shoulders, and shook him till the poor child waxed livid, and Edgar thoughtlessly laid hold of her hands to deliver him. In an instant one was wrung free, and the astonished young man felt it applied over his own ear in a way that could not be mistaken for jest. He drew back in consternation. I lifted Hareton in my arms, and walked off to the kitchen with him, leaving the door of communication open, for I was curious to watch how they would settle their disagreement. The insulted visitor moved to the spot where he had laid his hat, pale and with a quivering lip.
\par “That's right!” I said to myself. “Take warning and begone! It's a kindness to let you have a glimpse of her genuine disposition.”
\par “Where are you going?” demanded Catherine, advancing to the door.
\par He swerved aside, and attempted to pass.
\par “You must not go!” she exclaimed energetically.
\par “I must and shall!” he replied in a subdued voice.
\par “No,” she persisted, grasping the handle; “not yet, Edgar Linton: sit down; you shall not leave me in that temper. I should be miserable all night, and I won't be miserable for you!”
\par “Can I stay after you have struck me?” asked Linton.
\par Catherine was mute.
\par “You've made me afraid and ashamed of you,” he continued; “I'll not come here again!”
\par Her eyes began to glisten, and her lids to twinkle.
\par “And you told a deliberate untruth!” he said.
\par “I didn't!” she cried, recovering her speech; “I did nothing deliberately. Well, go, if you please — get away! And now I'll cry — I'll cry myself sick!”
\par She dropped down on her knees by a chair, and set to weeping in serious earnest. Edgar persevered in his resolution as far as the court; there he lingered. I resolved to encourage him.
\par “Miss is dreadfully wayward, sir,” I called out. “As bad as any marred child: you'd better be riding home, or else she will be sick only to grieve us.”
\par The soft thing looked askance through the window: he possessed the power to depart, as much as a cat possesses the power to leave a mouse half killed, or a bird half eaten.
\par Ah, I thought, there will be no saving him: he's doomed, and flies to his fate!
\par And so it was; he turned abruptly, hastened into the house again, shut the door behind him; and when I went in a while after to inform them that Earnshaw had come home rabid drunk, ready to pull the whole place about our ears (his ordinary frame of mind in that condition), I saw the quarrel had merely effected a closer intimacy —had broken the outworks of youthful timidity, and enabled them to forsake the disguise of friendship, and confess themselves lovers.
\par Intelligence of Mr. Hindley's arrival drove Linton speedily to his horse, and Catherine to her chamber. I went to hide little Hareton, and to take the shot out of the master's fowling-piece, which he was fond of playing with in his insane excitement, to the hazard of the lives of any who provoked, or even attracted his notice too much; and I had hit upon the plan of removing it, that he might do less mischief if he did go the length of firing the gun.








\subsection*{Chapter 9}

\par He entered, vociferating oaths dreadful to hear; and caught me in the act of stowing his son away in the kitchen cupboard. Hareton was impressed with a wholesome terror of encountering either his wild beast's fondness or his madman's rage; for in one he ran a chance of being squeezed and kissed to death, and in the other of being flung into the fire, or dashed against the wall; and the poor thing remained perfectly quiet wherever I chose to put him.
\par “There, I've found it out at last!” cried Hindley, pulling me back by the skin of my neck, like a dog. “By heaven and hell, you've sworn between you to murder that child! I know how it is, now, that he is always out of my way. But, with the help of Satan, I shall make you swallow the carving-knife, Nelly! You needn't laugh; for I've just crammed Kenneth, head-downmost, in the Blackhorse marsh; and two is the same as one — and I want to kill some of you: I shall have no rest till I do!”
\par “But I don't like the carving knife, Mr. Hindley,” I answered: “it has been cutting red herrings. I'd rather be shot, if you please.”
\par “You'd rather be damned!” he said; “and so you shall. No law in England can hinder a man from keeping his house decent, and mine's abominable! open your mouth.”
\par He held the knife in his hand, and pushed its point between my teeth: but, for my part, I was never much afraid of his vagaries. I spat out, and affirmed it tasted detestably — I would not take it on any account.
\par “Oh!” said he, releasing me, “I see that hideous little villain is not Hareton: I beg your pardon, Nell. If it be, he deserves flaying alive for not running to welcome me, and for screaming as if I were a goblin. Unnatural cub, come hither! I'll teach thee to impose on a good-hearted, deluded father. Now, don't you think the lad would be handsomer cropped? It makes a dog fiercer, and I love something fierce — get me a scissors — something fierce and trim! Besides, it's infernal affectation— devilish conceit it is, to cherish our ears — we're asses enough without them. Hush, child, hush! Well then, it is my darling! wisht, dry thy eyes — there's a joy; kiss me. What! it won't? Kiss me, Hareton! Damn thee, kiss me! By God, as if 1 would rear such a monster! As sure as I'm living, I'll break the brat's neck.”
\par Poor Hareton was squalling and kicking in his father's arms with all his might, and redoubled his yells when he carried him upstairs and lifted him over the banister. I cried out that he would frighten the child into fits, and ran to rescue him.
\par As I reached them, Hindley leant forward on the rails to listen to a noise below; almost forgetting what he had in his hands.
\par “Who is that?” he asked, hearing someone approaching the stair's foot.
\par I leant forward also, for the purpose of signing to Heathcliff, whose step I recognized, not to come farther; and, at the instant when my eye quitted Hareton, he gave a sudden spring, delivered himself from the careless grasp that held him, and fell.
\par There was scarcely time to experience a thrill of horror before we saw that the little wretch was safe. Heathcliff arrived underneath just at the critical moment; by a natural impulse, he arrested his descent, and setting him on his feet, looked up to discover the author of the accident.
\par A miser who has parted with a lucky lottery ticket for five shillings, and finds next day he has lost in the bargain five thousand pounds, could not show a blanker countenance than he did on beholding the figure of Mr. Earnshaw above. It expressed, plainer than words could do, the intense anguish at having made himself the instrument of thwarting his own revenge. Had it been dark, I dare say, he would have tried to remedy the mistake by smashing Hareton's skull on the steps; but we witnessed his salvation; and I was presently below with my precious charge pressed to my heart. Hindley descended more leisurely, sobered and abashed.
\par “It is your fault, Ellen,” he said, “you should have kept him out of sight: you should have taken him from me! Is he injured anywhere?”
\par “Injured!” I cried angrily; “if he's not killed, he'll be an idiot! Oh! I wonder his mother does not rise from her grave to see how you use him. You're worse than a heathen — treating your own flesh and blood in that manner!”
\par He attempted to touch the child, who, on finding himself with me, sobbed off his terror directly. At the first finger his father laid on him, however, he shrieked again louder than before, and struggled as if he would go into convulsions.
\par “You shall not meddle with him!” I continued. “He hates you —they all hate you — that's the truth! A happy family you have: and a pretty state you're come to!”
\par “I shall come to a prettier, yet, Nelly,” laughed the misguided man, recovering his hardness. “At present, convey yourself and him away. And, hark you, Heathcliff! clear you too, quite from my reach and hearing. I wouldn't murder you tonight; unless, perhaps, I set the house on fire: but that's as my fancy goes.”
\par While saying this he took a pint bottle of brandy from the dresser, and poured some into a tumbler.
\par “Nay, don't!” I entreated. “Mr. Hindley, do take warning. Have mercy on this unfortunate boy, if you care nothing for yourself!”
\par “Anyone will do better for him than I shall,” he answered.
\par “Have mercy on your own soul!” I said, endeavouring to snatch the glass from his hand.
\par “Not I! On the contrary, I shall have great pleasure in sending it to perdition to punish its Maker,” exclaimed the blasphemer. “Here's to its hearty damnation!”
\par He drank the spirits and impatiently bade us go; terminating his command with a sequel of horrid imprecations, too bad to repeat or remember.
\par “It's a pity he cannot kill himself with drink,” observed Heathcliff, muttering an echo of curses back when the door was shut. “He's doing his very utmost; but his constitution defies him. Mr. Kenneth says he would wager his mare, that he'll outlive any man on this side Gimmerton, and go to the grave a hoary sinner; unless some happy chance out of the common course befall him.”
\par I went into the kitchen, and sat down to lull my little lamb to sleep. Heathcliff, as I thought, walked through to the barn. It turned out afterwards that he only got as far as the other side the settle, when he flung himself on a bench by the wall, removed from the fire, and remained silent.
\par I was rocking Hareton on my knee, and humming a song that began:
\par It was far in the night, and the bairnies grat,
\par “The mither beneath the mools heard that —”
\par When Miss Cathy, who had listened to the hubbub from her room, put her head in, and whispered:
\par “Are you alone, Nelly?”
\par “Yes, Miss,” I replied.
\par She entered and approached the hearth. I, supposing she was going to say something, looked up. The expression of her face seemed disturbed and anxious. Her lips were half asunder, as if she meant to speak, and she drew a breath; but it escaped in a sigh instead of a sentence. I resumed my song; not having forgotten her recent behaviour.
\par “Where's Heathcliff?” she said, interrupting me.
\par “About his work in the stable,” was my answer.
\par He did not contradict me; perhaps he had fallen into a doze. There followed another long pause, during which I perceived a drop or two trickle from Catherine's cheek to the flags. Is she sorry for her shameful conduct? I asked myself. That will be a novelty: but she may come to the point as she will — I shan't help her! No, she felt small trouble regarding any subject, save her own concerns.
\par “Oh, dear!” she cried at last. “I'm very unhappy!”
\par “A pity,” observed I. “You're hard to please: so many friends and so few cares, and can't make yourself content!”
\par “Nelly, will you keep a secret for me?” she pursued, kneeling down by me, and lifting her winsome eyes to my face with that sort of look which turns off bad temper, even when one has all the right in the world to indulge it.
\par “Is it worth keeping?” I inquired, less sulkily.
\par “Yes, and it worries me, and I must let it out! I want to know what I should do. Today, Edgar Linton has asked me to marry him, and I've given him an answer. Now, before I tell you whether it was a consent or denial, you tell me which it ought to have been.”
\par “Really, Miss Catherine, how can I know?” I replied. “To be sure, considering the exhibition you performed in his presence this afternoon, I might say it would be wise to refuse him: since he asked you after that, he must either be hopelessly stupid or a venturesome fool.”
\par “If you talk so, I won't tell you any more,” she returned peevishly, rising to her feet. “I accepted him, Nelly. Be quick, and say whether I was wrong!”
\par “You accepted him! then what good is it discussing the matter? You have pledged your word, and cannot retract.”
\par “But, say whether I should have done so — do!” she exclaimed in an irritated tone; chafing her hands together, and frowning.
\par “There are many things to be considered before that question can be answered properly,” I said sententiously. “First and foremost, do you love Mr. Edgar?”
\par “Who can help it? Of course I do,” she answered.
\par Then I put her through the following catechism: for a girl of twenty-two it was not injudicious.
\par “Why do you love him, Miss Cathy?”
\par “Nonsense, I do — that's sufficient.”
\par “By no means; you must say why?”
\par “Well, because he is handsome, and pleasant to be with.”
\par “Bad!” was my commentary.
\par “And because he is young and cheerful.”
\par “Bad, still.”
\par “And because he loves me.”
\par “Indifferent, coming there.”
\par “And he will be rich, and I shall like to be the greatest woman of the neighbourhood, and I shall be proud of having such a husband.”
\par “Worst of all. And now, say how you love him?”
\par “As everybody loves — You're silly, Nelly.”
\par “Not at all — Answer.”
\par “I love the ground under his feet, and the air over his head, and everything he touches, and every word he says. I love all his looks, and all his actions, and him entirely and altogether. There now!”
\par “And why?”
\par “Nay — you are making a jest of it; it is exceedingly ill-natured! It's no jest to me!” said the young lady, scowling, and turning her face to the fire.
\par “I'm very far from jesting, Miss Catherine,” I replied. “You love Mr. Edgar because he is handsome, and young, and cheerful, and rich, and loves you. The last, however, goes for nothing: you would love him without that, probably; and with it you wouldn't, unless he possessed the four former attractions.”
\par “No, to be sure not: I should only pity him — hate him, perhaps, if he were ugly, and a clown.”
\par “But there are several other handsome, rich young men in the world: handsomer, possibly, and richer than he is. What should hinder you from loving them?”
\par “If there be any, they are out of my way! I've seen none like Edgar.”
\par “You may see some; and he won't always be handsome, and young, and may not always be rich.”
\par “He is now; and I have only to do with the present. I wish you would speak rationally.”
\par “Well, that settles it: if you have only to do with the present, marry Mr. Linton.”
\par “I don't want your permission for that — I shall marry him: and yet you have not told me whether I'm right.”
\par “Perfectly right; if people be right to marry only for the present. And now, let us hear what you are unhappy about. Your brother will be pleased; the old lady and gentleman will not object, I think; you will escape from a disorderly, comfortless home into a wealthy, respectable one; and you love Edgar, and Edgar loves you. All seems smooth and easy: where is the obstacle?”
\par “Here! And here!” replied Catherine, striking one hand on her forehead, and the other on her breast: “in whichever place the soul lives. In my soul and in my heart, I'm convinced I'm wrong!”
\par “That's very strange! I cannot make it out.”
\par “It's my secret. But if you will not mock at me, I'll explain it: I can't do it distinctly: but I'll give you a feeling of how I feel.”
\par She seated herself by me again: her countenance grew sadder and graver, and her clasped hands trembled.
\par “Nelly, do you never dream queer dreams?” she said, suddenly, after some minutes' reflection.
\par “Yes, now and then,” I answered.
\par “And so do I. I've dreamt in my life dreams that have stayed with me ever after, and changed my ideas: they've gone through and through me, like wine through water, and altered the colour of my mind. And this is one; I'm going to tell it — but take care not to smile at any part of it.”
\par “Oh! don't, Miss Catherine!” I cried. “We're dismal enough without conjuring up ghosts and visions to perplex us. Come, come, be merry and like yourself! Look at little Hareton! he's dreaming nothing dreary. How sweetly he smiles in his sleep!”
\par “Yes; and how sweetly his father curses in his solitude! You remember him, I dare say, when he was just such another as that chubby thing — nearly as young and innocent. However, Nelly, I shall oblige you to listen — it's not long; and I've no power to be merry tonight.”
\par “I won't hear it, I won't hear it!” I repeated hastily.
\par I was superstitious about dreams then, and am still; and Catherine had an unusual gloom in her aspect, that made me dread something from which I might shape a prophecy, and foresee a fearful catastrophe. She was vexed, but she did not proceed. Apparently taking up another subject, she recommenced in a short time.
\par “If I were in heaven, Nelly, I should be extremely miserable.”
\par “Because you are not fit to go there,” I answered. “All sinners would be miserable in heaven.”
\par “But it is not for that. I dreamt once that I was there.”
\par “I tell you I won't hearken to your dreams, Miss Catherine! I'll go to bed,” I interrupted again.
\par She laughed, and held me down; for I made a motion to leave my chair.
\par “This is nothing,” cried she; “I was only going to say that heaven did not seem to be my home; and I broke my heart with weeping to come back to earth; and the angels were so angry that they flung me out into the middle of the heath on the top of Wuthering Heights; where I woke sobbing for joy. That will do to explain my secret, as well as the other. I've no more business to marry Edgar Linton than I have to be in heaven; and if the wicked man in there had not brought Heathcliff so low, I shouldn't have thought of it. It would degrade me to marry Heathcliff now; so he shall never know how I love him: and that, not because he's handsome, Nelly, but because he's more myself than I am. Whatever our souls are made of, his and mine are the same; and Linton's is as different as a moonbeam from lightning, or frost from fire.”
\par Ere this speech ended I became sensible of Heathcliff's presence. Having noticed a slight movement, I turned my head, and saw him rise from the bench, and steal out noiselessly. He had listened till he heard Catherine say it would degrade her to marry him, and then he stayed to hear no further. My companion, sitting on the ground, was prevented by the back of the settle from remarking his presence or departure; but I started, and bade her hush!
\par “Why?” she asked, gazing nervously round.
\par “Joseph is here,” I answered, catching opportunely the roll of his cart-wheels up the road; “and Heathcliff will come in with him. I'm not sure whether he were not at the door this moment.”
\par “Oh, he couldn't overhear me at the door!” said she. “Give me Hareton, while you get the supper, and when it is ready ask me to sup with you. I want to cheat my uncomfortable conscience, and be convinced that Heathcliff has no notion of these things. He has not, has he? He does not know what being in love is?”
\par “I see no reason that he should not know, as well as you,” I returned; “and if you are his choice, he will be the most unfortunate creature that ever was born! As soon as you become Mrs. Linton, he loses friend, and love, and all! Have you considered how you'll bear the separation, and how he'll be deserted in the world? Because, Miss Catherine — ”
\par “He quite deserted! we separated!” she exclaimed, with an accent of indignation. “Who is to separate us, pray? They'll meet the fate of Milo! Not as long as I live, Ellen — for no mortal creature. Every Linton on the face of the earth might melt into nothing, before I could consent to forsake Heathcliff. Oh, that's not what I intend — that's not what I mean! I shouldn't be Mrs. Linton were such a price demanded! He'll be as much to me as he has been all his lifetime. Edgar must shake off his antipathy, and tolerate him, at least. He will, when he learns my true feelings towards him. Nelly, I see now, you think me a selfish wretch; but did it never strike you that if Heathcliff and I married, we should be beggars? whereas, if I marry Linton, I can aid Heathcliff to rise, and place him out of my brother's power.”
\par “With your husband's money, Miss Catherine?” I asked. “You'll find him not so pliable as you calculate upon: and, though I'm hardly a judge, I think that's the worst motive you've given yet for being the wife of young Linton.”
\par “It is not,” retorted she, “it is the best! The others were the satisfaction of my whims: and for Edgar's sake, too, to satisfy him. This is for the sake of one who comprehends in his person my feelings to Edgar and myself. I cannot express it; but surely you and everybody have a notion that there is or should be an existence of yours beyond you. What were the use of my creation, if I were entirely contained here? My great miseries in this world have been Heathcliff's miseries, and I watched and felt each from the beginning: my great thought in living is himself. If all else perished, and he remained, I should still continue to be; and if all else remained, and he were annihilated, the universe would turn to a mighty stranger: I should not seem a part of it. My love for Linton is like the foliage in the woods: time will change it, I'm well aware, as winter changes the trees. My love for Heathcliff resembles the eternal rocks beneath — a source of little visible delight, but necessary. Nelly, I am Heathcliff! He's always, always in my mind— not as a pleasure, any more than I am always a pleasure to myself— but, as my own being — so, don't talk of our separation again: it is impracticable; and —”
\par She paused, and hid her face in the folds of my gown; but I jerked it forcibly away. I was out of patience with her folly!
\par “If I can make any sense of your nonsense, Miss,” I said, “it only goes to convince me that you are ignorant of the duties you undertake in marrying; or else that you are a wicked, unprincipled girl. But trouble me with no more secrets: I'll not promise to keep them.”
\par “You'll keep that?” she asked eagerly.
\par “No, I'll not promise,” I repeated.
\par She was about to insist, when the entrance of Joseph finished our conversation; and Catherine removed her seat to a corner, and nursed Hareton, while I made the supper. After it was cooked, my fellowservant and I began to quarrel who should carry some to Mr. Hindley; and we didn't settle it till all was nearly cold. Then we came to the agreement that we would let him ask, if he wanted any; for we feared particularly to go into his presence when he had been some time alone.
\par “Und hah isn't that nowt comed in frough th' field, be this time? What is he abaht? girt eedle seeght!” demanded the old man, looking round for Heathcliff.
\par “I'll call him,” I replied. “He's in the barn, I've no doubt.”
\par I went and called, but got no answer. On returning, I whispered to Catherine that he had heard a good part of what she said, I was sure; and told how I saw him quit the kitchen just as she complained of her brother's conduct regarding him. She jumped up in a fine fright, flung Hareton on to the settle, and ran to seek for her friend herself; not taking leisure to consider why she was so flurried, or how her talk would have affected him. She was absent such a while that Joseph proposed we should wait no longer. He cunningly conjectured they were staying away in order to avoid hearing his protracted blessing. They were“ill eneugh for ony fahl manners”, he affirmed. And on their behalf he added that night a special prayer to the usual quarter of an hour's supplication before meat, and would have tacked another to the end of the grace, had not his young mistress broken in upon him with a hurried command that he must run down the road, and wherever Heathcliff had rambled, find and make him reenter directly!
\par “I want to speak to him, and I must, before I go upstairs,” she said.“And the gate is open: he is somewhere out of hearing; for he would not reply, though I shouted at the top of the fold as loud as I could.”
\par Joseph objected at first; she was too much in earnest, however, to suffer contradiction; and at last he placed his hat on his head, and walked grumbling forth. Meantime, Catherine paced up and down the floor, exclaiming —
\par “I wonder where he is — I wonder where he can be? What did I say, Nelly? I've forgotten. Was he vexed at my bad humour this afternoon? Dear! tell me what I've said to grieve him? I do wish he'd come. I do wish he would!”
\par “What a noise for nothing!” I cried, though rather uneasy myself.“What a trifle scares you! It's surely no great cause of alarm that Heathcliff should take a moonlight saunter on the moors, or even lie too sulky to speak to us in the hay-loft. I'll engage he's lurking there. See if I don't ferret him out!”
\par I departed to renew my search; its result was disappointment, and Joseph's quest ended in the same.
\par “Yon lad gets war un war!” observed he on re-entering. “He's left th'yate ut t'full swing, and Miss's pony has trodden dahn two rigs uh'corn, un plottered through, raight o'er into t'meadow! Hahsomdiver, t'maister 'ull play t'devil to-morn, and he'll do weel. He's patience itsseln wi'sich careless, offald craters — patience itsseln he is! Bud he'll not be soa allus — yah's see, all on ye! Yah mum'nt drive him out of his heead for nowt!”
\par “Have you found Heathcliff, you ass?” interrupted Catherine.“Have you been looking for him, as I ordered?”
\par “I sud more likker look for th'horse,” he replied. “It'ud be to more sense. But, I can look for norther horse nur man of a neeght loike this— as black as t'chimbley! Und Hathecliff's noan t'chap to coom at my whistle — happen he'll be less hard o'hearing wi'ye!”
\par It was a very dark evening for summer: the clouds appeared inclined to thunder, and I said we had better all sit down; the approaching rain would be certain to bring him home without further trouble. However, Catherine would not be persuaded into tranquillity.She kept wandering to and fro, from the gate to the door, in a state of agitation which permitted no repose; and at length took up a permanent situation on one side of the wall, near the road: where, heedless of my expostulations and the growling thunder, and the great drops that began to plash around her, she remained, calling at intervals, and then listening, and then crying outright. She beat Hareton, or any child, at a good passionate fit of crying.
\par About midnight, while we still sat up, the storm came rattling over the Heights in full fury. There was a violent wind, as well as thunder, and either one or the other split a tree off at the corner of the building: a huge bough fell across the roof, and knocked down a portion of the east chimney-stack, sending a clatter of stones and soot into the kitchen fire.
\par We thought a bolt had fallen in the middle of us; and Joseph swung on to his knees beseeching the Lord to remember the patriarchs Noah and Lot, and, as in former times, spare the righteous, though He smote the ungodly. I felt some sentiment that it must be a judgment on us also. The Jonah, in my mind, was Mr. Earnshaw; and I shook the handle of his den that I might ascertain if he were yet living. He replied audibly enough, in a fashion which made my companion vociferate, more clamorously than before, that a wide distinction might be drawn between saints like himself and sinners like his master. But the uproar passed away in twenty minutes, leaving us all unharmed; excepting Cathy, who got thoroughly drenched for her obstinacy in refusing to take shelter, and standing bonnetless and shawl-less to catch as much water as she could with her hair and clothes.
\par She came in and lay down on the settle, all soaked as she was, turning her face to the back, and putting her hands before it.
\par “Well, Miss!” I exclaimed, touching her shoulder; “you are not bent on getting your death, are you? Do you know what o'clock it is? Half- past twelve. Come, come to bed! There's no use waiting longer on that foolish boy: he'll be gone to Gimmetton, and he'll stay there now. He guesses we shouldn't wait for him till this late hour: at least, he guesses that only Mr. Hindley would be up; and he'd rather avoid having the door opened by the master.
\par “Nay, nay, he's noan at Gimmerton,” said Joseph. “Aw's niver wonder but he's at t'bothom of a bog-hoile. This visitation worn't for nowt, and I wod hev ye to look out, Miss — yah muh be t'next. Thank Hiven for all! All warks togither for gooid to them as is chozzen, and piked out fro'th'rubbidge! Yah knaw whet t'Scripture ses.” And he began quoting several texts, referring us to the chapters and verses where we might find them.
\par I, having vainly begged the wilful girl to rise and remove her wet things, left him preaching and her shivering, and betook myself to bed with little Hareton, who slept as fast as if everyone had been sleeping round him. I heard Joseph read on a while afterwards; then I distinguished his slow step on the ladder, and then I dropped asleep.
\par Coming down somewhat later than usual, I saw, by the sunbeams piercing the chinks of the shutters, Miss Catherine still seated near the fireplace. The house door was ajar, too; light entered from its unclosed windows; Hindley had come out, and stood on the kitchen hearth, haggard and drowsy.
\par “What ails you, Cathy?” he was saying when I entered: “you look as dismal as a drowned whelp. Why are you so damp and pale, child?”
\par “I've been wet,” she answered reluctantly, “and I'm cold, that's all.”
\par “Oh, she is naughty!” I cried, perceiving the master to be tolerably sober. “She got steeped in the shower of yesterday evening, and there she has sat the night through, and I couldn't prevail on her to stir.”
\par Mr. Earnshaw stared at us in surprise. “The night through,” he repeated. “What kept her up? not fear of the thunder, surely? That was over hours since.”
\par Neither of us wished to mention Heathcliff's absence, as long as we could conceal it; so I replied, I didn't know how she took it into her head to sit up; and she said nothing. The morning was fresh and cool; I threw back the lattice, and presently the room filled with sweet scents from the garden; but Catherine called peevishly to me, “Ellen, shut the window. I'm starving!” And her teeth chattered as she shrunk closer to the almost extinguished embers.
\par “She's ill,” said Hindley, taking her wrist; “I suppose that's the reason she would not go to bed. Damn it! I don't want to be troubled with more sickness here. What took you into the rain!”
\par “Running after t'lads, as usuald!” croaked Joseph, catching an opportunity, from our hesitation, to thrust in his evil tongue. “If Aw war yah, maister, Aw'd just slam t'boards i'their faces all on'em, gentle and simple! Never a day ut yah're off, but yon cat o'Linton comes sneaking hither; and Miss Nelly, shoo's a fine lass! Shoo sits watching for ye i't'kitchen; and as yah're in at one door, he's aht at t'other; and, then, wer grand lady goes a coorting of her side! It's bonny behaviour, lurking amang t'fields, after twelve o't'night, wi'that fahl, flaysome divil of a gipsy, Heathcliff! They think Aw'm blind; but Aw'm noan: nowt ut t'soart! — Aw seed young Linton boath coming and going, and Aw seed yah” (directing his discourse to me), “yah gooid fur nowt, slattenly witch! Nip up and bolt into th'house, t'minute yah heard t'maister's horse-fit clatter up t'road.”
\par “Silence, eavesdropper!” cried Catherine. “None of your insolence before me! Edgar Linton came yesterday by chance, Hindley; and it was I who told him to be off: because I knew you would not like to have met him as you were.”
\par “You lie, Cathy, no doubt,” answered her brother, “and you are a confounded simpleton! But never mind Linton at present: tell me, were you not with Heathcliff last night? Speak the truth, now. You need not be afraid of harming him: though I hate him as much as ever, he did me a good turn a short time since, that will make my conscience tender of breaking his neck. To prevent it, I shall send him about his business, this very morning; and after he's gone, I'd advise you all to look sharp:I shall only have the more humour for you.”
\par “I never saw Heathcliff last night,” answered Catherine, beginning to sob bitterly: “and if you do turn him out of doors, I'll go with him. But, perhaps, you'll never have an opportunity: perhaps he's gone.”Here she burst into uncontrollable grief, and the remainder of her words were inarticulate.
\par Hindley lavished on her a torrent of scornful abuse, and bade her get to her room immediately, or she shouldn't cry for nothing! I obliged her to obey; and I shall never forget what a scene she acted when we reached her chamber: it terrified me. I thought she was going mad, and I begged Joseph to run for the doctor.
\par It proved the commencement of delirium: Mr. Kenneth, as soon as he saw her, pronounced her dangerously ill; she had a fever.
\par He bled her, and he told me to let her live on whey and watergruel, and take care she did not throw herself downstairs or out of the window; and then he left: for he had enough to do in the parish, where two or three miles was the ordinary distance between cottage and cottage.
\par Though I cannot say I made a gentle nurse, and Joseph and the master were no better; and though our patient was as wearisome and headstrong as a patient could be, she weathered it through.
\par Old Mrs. Linton paid us several visits, to be sure, and set things to rights, and scolded and ordered us all; and when Catherine was convalescent, she insisted on conveying her to Thrushcross Grange: for which deliverance we were very grateful. But the poor dame had reason to repent of her kindness: she and her husband both took the fever, and died within a few days of each other.
\par Our young lady returned to us, saucier and more passionate, and haughtier than ever. Heathcliff had never been heard of since the evening of the thunder-storm; and, one day I had the misfortune, when she had provoked me exceedingly, to lay the blame of his disappearance on her: where indeed it belonged, as she well knew. From that period, for several months, she ceased to hold any communication with me, save in the relation of a mere servant. Joseph fell under a ban also: he would speak his mind, and lecture her all the same as if she were a little girl; and she esteemed herself a woman, and our mistress, and thought that her recent illness gave her a claim to be treated with consideration. Then the doctor had said that she would not bear crossing much; she ought to have her own way; and it was nothing less than murder in her eyes for anyone to presume to stand up and contradict her.
\par From Mr. Earnshaw and his companions she kept aloof; and tutored by Kenneth, and serious threats of a fit that often attended her rages, her brother allowed her whatever she pleased to demand, and generally avoided aggravating her fiery temper. He was rather too indulgent in humouring her caprices; not from affection, but from pride: he wished earnestly to see her bring honour to the family by an alliance with the Lintons, and as long as she let him alone she might trample us like slaves, for aught he cared! Edgar Linton, as multitudes have been before and will be after him, was infatuated; and believed himself the happiest man alive on the day he led her to Gimmerton Chapel, three years subsequent to his father's death.
\par Much against my inclination, I was persuaded to leave Wuthering Heights and accompany her here. Little Hareton was nearly five years old, and I had just begun to teach him his letters. We made a sad parting; but Catherine's tears were more powerful than ours. When I refused to go, and when she found her entreaties did not move me, she went lamenting to her husband and brother. The former offered me munificent wages; the latter ordered me to pack up: he wanted no women in the house, he said, now that there was no mistress; and as to Hareton, the curate should take him in hand, by and by. And so I had but one choice left: to do as I was ordered. I told the master he got rid of all decent people only to run to ruin a little faster; I kissed Hareton goodbye; and since then he has been a stranger: and it's very queer to think it, but I've no doubt he has completely forgotten all about Ellen Dean, and that he was ever more than all the world to her, and she to him!
\par 
\par At this point of the housekeeper's story, she chanced to glance towards the timepiece over the chimney; and was in amazement on seeing the minute-hand measure half past one. She would not hear of staying a second longer: in truth, I felt rather disposed to defer the sequel of her narrative myself. And now that she is vanished to her rest, and I have meditated for another hour or two, I shall summon courage to go, also, in spite of aching laziness of head and limbs.


\subsection*{Chapter 10}

\par A charming introduction to a hermit's life! Four weeks' torture, tossing, and sickness! Oh, these bleak winds and bitter northern skies, and impassable roads, and dilatory country surgeons! And, oh, this dearth of the human physiognomy! And, worse than all, the terrible intimation of Kenneth that I need not expect to be out of doors till spring!
\par Mr. Heathcliff has just honoured me with a call. About seven days ago he sent me a brace of grouse — the last of the season. Scoundrel! He is not altogether guiltless in this illness of mine; and that I had a great mind to tell him. But, alas! How could I offend a man who was charitable enough to sit at my bedside a good hour, and talk on some other subject than pills and draughts, blisters and leeches?
\par This is quite an easy interval. I am too weak to read; yet I feel as if I could enjoy something interesting. Why not have up Mrs. Dean to finish her tale? I can recollect its chief incidents as far as she had gone. Yes, I remember her hero had run off, and never been heard of for three years; and the heroine was married. I'll ring; she'll be delighted to find me capable of talking cheerfully.
\par Mrs. Dean came.
\par “It wants twenty minutes, sir, to taking the medicine,” she commenced.
\par “Away, away with it!” I replied; “I desire to have —”
\par “The doctor says you must drop the powders.”
\par “With all my heart! Don't interrupt me. Come and take your seat here. Keep your fingers from that bitter phalanx of vials. Draw your knitting out of your pocket — that will do — now continue the history of Mr. Heathcliff, from where you left off, to the present day. Did he finish his education on the Continent, and come back a gentleman? Or did he get a sizar's place at college, or escape to America, and earn honours by drawing blood from his foster-country? Or make a fortune more promptly on the English highways?”
\par “He may have done a little in all these vocations, Mr. Lockwood; but I couldn't give my word for any. I stated before that I didn't know how he gained his money; neither am I aware of the means he took to raise his mind from the savage ignorance into which it was sunk: but, with your leave, I'll proceed in my own fashion, if you think it will amuse and not weary you. Are you feeling better this morning?”
\par “Much.”
\par “That's good news.”
\par 
\par I got Miss Catherine and myself to Thrushcross Grange; and, to my agreeable disappointment, she behaved infinitely better than I dared to expect. She seemed almost over-fond of Mr. Linton; and even to his sister she showed plenty of affection. They were both very attentive to her comfort, certainly. It was not the thorn bending to the honeysuckles, but the honeysuckles embracing the thorn. There were no mutual concessions; one stood erect, and the others yielded, and who can be illnatured and bad-tempered when they encounter neither opposition nor indifference?
\par I observed that Mr. Edgar had a deep-rooted fear of ruffling her humour. He concealed it from her; but if ever he heard me answer sharply, or saw any other servant grow cloudy at some imperious order of hers, he would show his trouble by a frown of displeasure that never darkened on his own account. He many a time spoke sternly to me about my pertness; and averred that the stab of a knife could not inflict a worse pang than he suffered at seeing his lady vexed.
\par Not to grieve a kind master, I learned to be less touchy; and, for the space of half a year, the gunpowder lay as harmless as sand, because no fire came near to explode it. Catherine had seasons of gloom and silence now and then: they were respected with sympathizing silence by her husband, who ascribed them to an alteration in her constitution, produced by her perilous illness; as she was never subject to depression of spirits before. The return of sunshine was welcomed by answering sunshine from him. I believe I may assert that they were really in possession of deep and growing happiness.
\par It ended. Well, we must be for ourselves in the long run; the mild and generous are only more justly selfish than the domineering; and it ended when circumstances caused each to feel that the one's interest was not the chief consideration in the other's thoughts. On a mellow evening in September, I was coming from the garden with a heavy basket of apples which I had been gathering. It had got dusk, and the moon looked over the high wall of the court, causing undefined shadows to lurk in the corners of the numerous projecting portions of the building. I set my burden on the house-steps by the kitchen door, and lingered to rest, and drew in a few more breaths of the soft, sweet air; my eyes were on the moon, and my back to the entrance, when I heard a voice behind me say —
\par “Nelly, is that you?”
\par It was a deep voice, and foreign in tone; yet there was something in the manner of pronouncing my name which made it sound familiar. I turned about to discover who spoke, fearfully; for the doors were shut, and I had seen nobody on approaching the steps.
\par Something stirred in the porch; and, moving nearer, I distinguished a tall man dressed in dark clothes, with dark face and hair. He leant against the side, and held his fingers on the latch as if intending to open for himself.
\par “Who can it be?” I thought. “Mr. Earnshaw? Oh, no! The voice has no resemblance to his.”
\par “I have waited here an hour,” he resumed, while I continued staring; “and the whole of that time all round has been as still as death. I dared not enter. You do not know me? Look, I'm not a stranger!”
\par A ray fell on his features; the cheeks were sallow, and half covered with black whiskers; the brows lowering, the eyes deep set and singular. I remembered the eyes.
\par “What!” I cried, uncertain whether to regard him as a worldly visitor, and I raised my hands in amazement. “What! You come back? Is it really you? Is it?”
\par “Yes, Heathcliff,” he replied, glancing from me up to the windows, which reflected a score of glittering moons, but showed no lights from within. “Are they at home? Where is she? Nelly, you are not glad! you needn't be so disturbed. Is she here? Speak! I want to have one word with her — your mistress. Go, and say some person from Gimmerton desires to see her.”
\par “How will she take it?” I exclaimed. “What will she do? The surprise bewilders me — it will put her out of her head! And you are Heathcliff! But altered! Nay, there's no comprehending it. Have you been for a soldier?”
\par “Go and carry my message,” he interrupted impatiently. “I'm in hell till you do!”
\par He lifted the latch, and I entered; but when I got to the parlour where Mr. and Mrs. Linton were, I could not persuade myself to proceed. At length, I resolved on making an excuse to ask if they would have the candles lighted, and I opened the door.
\par They sat together in a window whose lattice lay back against the wall, and displayed, beyond the garden trees and the wild green park, the valley of Gimmerton, with a long line of mist winding nearly to its top (for very soon after you pass the chapel, as you may have noticed, the sough that runs from the marshes joins a beck which follows the bend of the glen). Wuthering Heights rose above this silvery vapour; but our old house was invisible; it rather dips down on the other side. Both the room and its occupants, and the scene they gazed on, looked wondrously peaceful. I shrank reluctantly from performing my errand; and was actually going away leaving it unsaid, after having put my question about the candles, when a sense of my folly compelled me to return, and mutter, “A person from Gimmerton wishes to see you, ma'am.”
\par “What does he want?” asked Mrs. Linton.
\par “I did not question him,” I answered.
\par “Well, close the curtains, Nelly,” she said; “and bring up tea. I'll be back again directly.”
\par She quitted the apartment; Mr. Edgar inquired, carelessly, who it was.
\par “Someone mistress does not expect,” I replied. “That Heathcliff —you recollect him, sir, — who used to live at Mr. Earnshaw's.”
\par “What! the gipsy — the ploughboy?” he cried. “Why did you not say so to Catherine?”
\par “Hush! You must not call him by those names, master,” I said.“She'd be sadly grieved to hear you. She was nearly heartbroken when he ran off. I guess his return will make a jubilee to her.”
\par Mr. Linton walked to a window on the other side of the room that overlooked the court. He unfastened it and leant out. I suppose they were below, for he exclaimed quickly —
\par “Don't stand there, love! Bring the person in, if it be anyone particular.”
\par Ere long I heard the click of the latch, and Catherine flew upstairs, breathless and wild; too excited to show gladness: indeed, by her face, you would rather have surmised an awful calamity.
\par “Oh, Edgar, Edgar!” she panted, flinging her arms round his neck. “Oh Edgar, darling! Heathcliff's come back — he is!” And she tightened her embrace to a squeeze.
\par “Well, well,” cried her husband crossly, “don't strangle me for that! He never struck me as such a marvellous treasure. There is no need to be frantic!”
\par “I know you didn't like him,” she answered, repressing a little the intensity of her delight. “Yet, for my sake, you must be friends now. Shall I tell him to come up?”
\par “Here” he said, “into the parlour?”
\par “Where else?” she asked.
\par He looked vexed, and suggested the kitchen as a more suitable place for him. Mrs. Linton eyed him with a droll expression — half angry, half laughing at his fastidiousness.
\par “No,” she added after a while; “I cannot sit in the kitchen. Set two tables here, Ellen: one for your master and Miss Isabella, being gentry; the other for Heathcliff and myself, being of the lower orders. Will that please you, dear? Or must I have a fire lighted elsewhere? If so, give directions. I'll run down and secure my guest. I'm afraid the joy is too great to be real!”
\par She was about to dart off again; but Edgar arrested her.
\par “You bid him step up,” he said, addressing me; “and, Catherine, try to be glad, without being absurd! The whole household need not witness the sight of your welcoming a runaway servant as a brother.”
\par I descended and found Heathcliff waiting under the porch, evidently anticipating an invitation to enter. He followed my guidance without waste of words, and I ushered him into the presence of the master and mistress, whose flushed cheeks betrayed signs of warm talking. But the lady's glowed with another feeling when her friend appeared at the door: she sprang forward, took both his hands, and led him to Linton; and then she seized Linton's reluctant fingers and crushed them into his.
\par Now fully revealed by the fire and candlelight, I was amazed, more than ever, to behold the transformation of Heathcliff. He had grown a tall, athletic, well-formed man; beside whom, my master seemed quite slender and youth-like. His upright carriage suggested the idea of his having been in the army. His countenance was much older in expression and decision of feature than Mr. Linton's; it looked intelligent, and retained no marks of former degradation. A half-civilized ferocity lurked yet in the depressed brows and eyes full of black fire, but it was subdued; and his manner was even dignified: quite divested of roughness, though too stern for grace.
\par My master's surprise equalled or exceeded mine: he remained for a minute at a loss how to address the ploughboy, as he had called him. Heathcliff dropped his slight hand, and stood looking at him coolly till he chose to speak.
\par “Sit down, sir,” he said, at length. “Mrs. Linton, recalling old times, would have me give you a cordial reception; and, of course, I am gratified when anything occurs to please her.”
\par “And I also,” answered Heathcliff, “especially if it be anything in which I have a part. I shall stay an hour or two willingly.”
\par He took a seat opposite Catherine, who kept her gaze fixed on him as if she feared he would vanish were she to remove it. He did not raise his to her often: a quick glance now and then sufficed; but it flashed back, each time more confidently, the undisguised delight he drank from hers. They were too much absorbed in their mutual joy to suffer embarrassment. Not so Mr. Edgar: he grew pale with pure annoyance: a feeling that reached its climax when his lady rose, and stepping across the rug, seized Heathcliff's hands again, and laughed like one beside herself.
\par “I shall think it a dream tomorrow!” she cried. “I shall not be able to believe that I have seen, and touched, and spoken to you once more. And yet, cruel Heathcliff! You don't deserve this welcome. To be absent and silent for three years, and never to think of me!”
\par “A little more than you have thought of me,” he murmured. “I heard of your marriage, Cathy, not long since; and, while waiting in the yard below, I meditated this plan: — just to have one glimpse of your face, a stare of surprise, perhaps, and pretended pleasure; afterwards settle my score with Hindley; and then prevent the law by doing execution on myself. Your welcome has put these ideas out of my mind; but beware of meeting me with another aspect next time! Nay, you'll not drive me off again. You were really sorry for me, were you? Well, there was cause. I've fought through a bitter life since I last heard your voice; and you must forgive me, for I struggled only for you!”
\par “Catherine, unless we are to have cold tea, please to come to the table,” interrupted Linton, striving to preserve his ordinary tone, and a due measure of politeness. “Mr. Heathcliff will have a long walk, wherever he may lodge tonight; and I'm thirsty.”
\par She took her post before the urn; and Miss Isabella came, summoned by the bell; then, having handed their chairs forward, I left the room. The meal hardly endured ten minutes. Catherine's cup was never filled: she could neither eat nor drink. Edgar had made a slop in his saucer, and scarcely swallowed a mouthful. Their guest did not protract his stay that evening above an hour longer. I asked, as he departed, if he went to Gimmerton?
\par “No, to Wuthering Heights,” he answered: “Mr. Earnshaw invited me, when I called this morning.”
\par Mr. Earnshaw invited him! and he called on Mr. Earnshaw! I pondered this sentence painfully, after he was gone. Is he turning out a bit of a hypocrite, and coming into the country to work mischief under a cloak? I mused: I had a presentiment in the bottom of my heart that he had better have remained away.
\par About the middle of the night, I was wakened from my first nap by Mrs. Linton gliding into my chamber, taking a seat on my bedside, and pulling me by the hair to rouse me.
\par “I cannot rest, Ellen,” she said, by way of apology. “And I want some living creature to keep me company in my happiness! Edgar is sulky, because I'm glad of a thing that does not interest him: he refuses to open his mouth, except to utter pettish, silly speeches; and he affirmed I was cruel and selfish for wishing to talk when he was so sick and sleepy. He always contrives to be sick at the least cross! I gave a few sentences of commendation to Heathcliff, and he, either for a headache or a pang of envy, began to cry: so I got up and left him.”
\par “What use is it praising Heathcliff to him?” I answered. “As lads they had an aversion to each other, and Heathcliff would hate just as much to hear him praised: it's human nature. Let Mr. Linton alone about him, unless you would like an open quarrel between them.”
\par “But does it not show great weakness?” pursued she. “I'm not envious: I never feel hurt at the brightness of Isabella's yellow hair and the whiteness of her skin, at her dainty elegance, and the fondness all the family exhibit for her. Even you, Nelly, if we have a dispute sometimes, you back Isabella at once; and I yield like a foolish mother:I call her a darling, and flatter her into a good temper. It pleases her brother to see us cordial, and that pleases me. But they are very much alike: they are spoiled children, and fancy the world was made for their accommodation; and though I humour both, I think a smart chastisement might improve them, all the same.”
\par “You're mistaken, Mrs. Linton,” said I. “They humour you: I know what there would be to do if they did not. You can well afford to indulge their passing whims as long as their business is to anticipate all your desires. You may, however, fall out, at last, over something of equal consequence to both sides; and then those you term weak are very capable of being as obstinate as you.”
\par “And then we shall fight to the death, shan't we, Nelly?” she returned, laughing. “No! I tell you, I have such faith in Linton's love, that I believe I might kill him, and he wouldn't wish to retaliate.”
\par I advised her to value him the more for his affection.
\par “I do,” she answered, “but he needn't resort to whining for trifles. It is childish; and, instead of melting into tears because I said that Heathcliff was now worthy of anyone's regard, and it would honour the first gentleman in the country to be his friend, he ought to have said it for me, and been delighted from sympathy. He must get accustomed to him, and he may as well like him: considering how Heathcliff has reason to object to him, I'm sure he behaved excellently!”
\par “What do you think of his going to Wuthering Heights?” I inquired.“He is reformed in every respect, apparently — quite a Christian offering the right hand of fellowship to his enemies all around!”
\par “He explained it,” she replied. “I wondered as much as you. He said he called to gather information concerning me from you, supposing you resided there still; and Joseph told Hindley, who came out and fell to questioning him of what he had been doing, and how he had been living; and finally, desired him to walk in. There were some persons sitting at cards; Heathcliff joined them; my brother lost some money to him, and, finding him plentifully supplied, he requested that he would come again in the evening: to which he consented. Hindley is too reckless to select his acquaintance prudently: he doesn't trouble himself to reflect on the causes he might have for mistrusting one whom he has basely injured. But Heathcliff affirms his principal reason for resuming a connection with his ancient persecutor is a wish to install himself in quarters at walking distance from the Grange, and an attachment to the house where we lived together; and likewise a hope that I shall have more opportunities of seeing him there than I could have if he settled in Gimmerton. He means to offer liberal payment for permission to lodge at the Heights; and doubtless my brother's covetousness will prompt him to accept the terms: he was always greedy; though what he grasps with one hand he flings away with the other.”
\par “It's a nice place for a young man to fix his dwelling in!” said I.“Have you no fear of the consequences, Mrs. Linton?”
\par “None for my friend,” she replied. “His strong head will keep him from danger; a little for Hindley: but he can't be made morally worse than he is; and I stand between him and bodily harm. The event of this evening has reconciled me to God and humanity! I had risen in angry rebellion against Providence. Oh, I've endured very, very bitter misery, Nelly! If that creature knew how bitter, he'd be ashamed to cloud its removal with idle petulance. It was kindness for him which induced me to bear it alone: had I expressed the agony I frequently felt, he would have been taught to long for its alleviation as ardently as l. However, it's over, and I'll take no revenge on his folly; I can afford to suffer anything hereafter! Should the meanest thing alive slap me on the cheek, I'd not only turn the other, but, I'd ask pardon for provoking it; and, as a proof, I'll go make my peace with Edgar instantly. Good night! I'm an angel!”
\par In this self-complacent conviction she departed; and the success of her fulfilled resolution was obvious on the morrow: Mr. Linton had not only abjured his peevishness (though his spirits seemed still subdued by Catherine's exuberance of vivacity), but he ventured no objection to her taking Isabella with her to Wuthering Heights in the afternoon; and she rewarded him with such a summer of sweetness and affection in return, as made the house a paradise for several days; both master and servants profiting from the perpetual sunshine.
\par Heathcliff — Mr. Heathcliff I should say in future — used the liberty of visiting at Thrushcross Grange cautiously, at first: he seemed estimating how far its owner would bear his intrusion. Catherine, also, deemed it judicious to moderate her expressions of pleasure in receiving him; and he gradually established his right to be expected. He retained a great deal of the reserve for which his boyhood was remarkable; and that served to repress all startling demonstrations of feeling. My master's uneasiness experienced a lull, and further circumstances diverted it into another channel for a space.
\par His new source of trouble sprang from the not-anticipated misfortune of Isabella Linton evincing a sudden and irresistible attraction towards the tolerated guest. She was at that time a charming young lady of eighteen; infantile in manners, though possessed of keen wit, keen feelings, and a keen temper, too, if irritated. Her brother, who loved her tenderly, was appalled at this fantastic preference. Leaving aside the degradation of an alliance with a nameless man, and the possible fact that his property, in default of heirs male, might pass into such a one's power, he had sense to comprehend Heathcliff's disposition: to know that, though his exterior was altered, his mind was unchangeable and unchanged. And he dreaded that mind: it revolted him: he shrank forebodingly from the idea of committing Isabella to his keeping. He would have recoiled still more had he been aware that her attachment rose unsolicited, and was bestowed where it awakened no reciprocation of sentiment; for the minute he discovered its existence, he laid the blame on Heathcliff's deliberate designing.
\par We had all remarked, during some time, that Miss Linton fretted and pined over something. She grew cross and wearisome; snapping  at and teasing Catherine continually, at the imminent risk of exhausting her limited patience. We excused her, to a certain extent, on the plea of ill-health: she was dwindling and fading before our eyes. But one day, when she had been peculiarly wayward, rejecting her breakfast, complaining that the servants did not do what she told them; that the mistress would allow her to be nothing in the house, and Edgar neglected her; that she had caught a cold with the doors being left open, and we let the parlour fire go out on purpose to vex her, with a hundred yet more frivolous accusations, Mrs. Linton peremptorily insisted that she should get to bed; and, having scolded her heartily, threatened to send for the doctor. Mention of Kenneth caused her to exclaim, instantly, that her health was perfect, and it was only Catherine's harshness which made her unhappy.
\par “How can you say I am harsh, you naughty fondling?” cried the mistress, amazed at the unreasonable assertion. “You are surely losing your reason. When have I been harsh, tell me?”
\par “Yesterday,” sobbed Isabella, “and now!”
\par “Yesterday!” said her sister-in-law. “On what occasion?”
\par “In our walk along the moor: you told me to ramble where I pleased, while you sauntered on with Mr. Heathcliff!”
\par “And that's your notion of harshness?” said Catherine, laughing. “It was no hint that your company was superfluous: we didn't care whether you kept with us or not; I merely thought Heathcliffs' talk would have nothing entertaining for your ears.”
\par “Oh no,” wept the young lady; “you wished me away, because you knew I liked to be there!”
\par “Is she sane?” asked Mrs. Linton, appealing to me. “I'll repeat our conversation, word for word, Isabella; and you point out any charm it could have had for you.”
\par “I don't mind the conversation,” she answered: “I wanted to be with —”
\par “Well!” said Catherine, perceiving her hesitate to complete the sentence.
\par “With him: and I won't be always sent off!” she continued, kindling up. “You are a dog in the manger, Cathy, and desire no one to be loved but yourself!”
\par “You are an impertinent little monkey!” exclaimed Mrs. Linton, in surprise. “But I'll not believe this idiocy! It is impossible that you can covet the admiration of Heathcliff — that you consider him an agreeable person! I hope I have misunderstood you, Isabella?”
\par “No, you have not,” said the infatuated girl. “I love him more than ever you loved Edgar; and he might love me, if you would let him!”
\par “I wouldn't be you for a kingdom, then!” Catherine declared emphatically: and she seemed to speak sincerely. “Nelly, help me to convince her of her madness. Tell her what Heathcliff is: an unreclaimed creature, without refinement, without cultivation: an arid wilderness of furze and whinstone. I'd as soon put that little canary into the park on a winter's day, as recommend you to bestow your heart on him! It is deplorable ignorance of his character, child, and nothing else, which makes that dream enter your head. Pray, don't imagine that he conceals depths of benevolence and affection beneath a stern exterior! He's not a rough diamond — a pearl-containing oyster of a rustic: he's a fierce, pitiless, wolfish man. I never say to him, ‘Let this or that enemy alone, because it would be ungenerous or cruel to harm them'; I say, ‘Let them alone, because I should hate them to be wronged': and he'd crush you like a sparrow's egg, Isabella, if he found you a troublesome charge.I know he couldn't love a Linton; and yet he'd be quite capable of marrying your fortune and expectations! Avarice is growing with him a besetting sin. There's my picture: and I'm his friend — so much so, that had he thought seriously to catch you, I should, perhaps, have held my tongue, and let you fall into his trap.”
\par Miss Linton regarded her sister-in-law with indignation.
\par “For shame! for shame!” she repeated angrily, “you are worse than twenty foes, you poisonous friend!”
\par “Ah! you won't believe me, then?” said Catherine. “You think I speak from wicked selfishness?”
\par “I'm certain you do,” retorted Isabella; “and I shudder at you!”
\par “Good!” cried the other. “Try for yourself, if that be your spirit: I have done, and yield the argument to your saucy insolence.”
\par “And I must suffer for her egotism!” she sobbed, as Mrs. Linton left the room. “All, all is against me; she has blighted my single consolation. But she uttered falsehoods, didn't she? Mr. Heathcliff is not a fiend: he has an honourable soul, and a true one, or how could he remember her?”
\par “Banish him from your thoughts, miss,” I said. “He's a bird of bad omen: no mate for you. Mrs. Linton spoke strongly, and yet I can't contradict her. She is better acquainted with his heart than I, or anyone besides; and she would never represent him as worse than he is. Honest people don't hide their deeds. How has he been living? how has he got rich? why is he staying at Wuthering Heights, the house of a man whom he abhors? They say Mr. Earnshaw is worse and worse since he came. They sit up all night together continually, and Hindley has been borrowing money on his land, and does nothing but play and drink: I heard only a week ago — it was Joseph who told me — I met him at Gimmerton: ‘Nelly,’ he said, ‘we's hae a crahnr's'quest enah, at ahr folks. One on ‘em's a'most getten his finger cut off wi'hauding t'other fro'stickin, hisseln loike a cawlf. That's maister, yah knaw,’ at's soa up o'going tuh t'grand ‘sizes. He's noan feard o't'bench o'judges, norther Paul, nur Peter, nur John, nur Matthew, nor noan on ‘em, not he! He fair likes — he langs to set his brazened face agean ‘em! And yon bonny lad Heathcliff, yah mind, he's a rare ‘un! He can girn a laugh as weel's onybody at a raight divil's jest. Does he niver say nowt of his fine living amang us, when he goas to t'Grange? This is t'way on't: — up at sundahn; dice, brandy, cloised shutters, und can'de-light till next day at nooin: then, t'fooil gangs banning un raving to his cham'er, makking dacent fowks dig thur fingers i'thur lugs fur varry shame; un'the knave, why he can caint his brass, un ate, un sleep, un off to his neighbour's to gossip wi't'wife. I'course, he tells Dame Catherine how her fathur's goold runs into his pocket, and her fathur's son gallops down t'broad road, while he flees afore to oppen t'pikes?’ Now, Miss Linton, Joseph is an old rascal, but no liar; and, if his account of Heathcliff's conduct be true, you would never think of desiring such a husband, would you?”
\par “You are leagued with the rest, Ellen!” she replied. “I'll not listen to your slanders. What malevolence you must have to wish to convince me that there is no happiness in the world!”
\par Whether she would have got over this fancy if left to herself or persevered in nursing it perpetually, I cannot say: she had little time to reflect. The day after, there was a justice-meeting at the next town; my master was obliged to attend; and Mr. Heathcliff, aware of his absence, called rather earlier than usual. Catherine and Isabella were sitting in the library, on hostile terms, but silent. The latter alarmed at her recent indiscretion, and the disclosure she had made of her secret feelings in a transient fit of passion; the former, on mature consideration, really offended with her companion; and, if she laughed again at her pertness, inclined to make it no laughing matter to her. She did laugh as she saw Heathcliff pass the window. I was sweeping the hearth, and I noticed a mischievous smile on her lips. Isabella, absorbed in her meditations, or a book, remained till the door opened; and it was too late to attempt an escape, which she would gladly have done had it been practicable.
\par “Come in, that's right!” exclaimed the mistress gaily, pulling a chair to the fire. “Here are two people sadly in need of a third to thaw the ice between them; and you are the very one we should both of us choose. Heathcliff, I'm proud to show you, at last, somebody that dotes on you more than myself. I expect you to feel flattered. Nay, it's not Nelly; don't look at her! My poor little sister-in-law is breaking her heart by mere contemplation of your physical and moral beauty. It lies in your own power to be Edgar's brother! No, no, Isabella, you shan't run off,” she continued, arresting, with feigned playfulness, the confounded girl, who had risen indignantly. “We were quarrelling like cats about you, Heathcliff; and I was fairly beaten in protestations of devotion and admiration: and, moreover, I was informed that if I would but have the manners to stand aside, my rival, as she will have herself to be, would shoot a shaft into your soul that would fix you for ever, and send my image into eternal oblivion!”
\par “Catherine!” said Isabella, calling up her dignity, and disdaining to struggle from the tight grasp that held her. “I'd thank you to adhere to the truth and not slander me, even in joke! Mr. Heathcliff, be kind enough to bid this friend of yours release me: she forgets that you and I are not intimate acquaintances; and what amuses her is painful to me beyond expression.”
\par As the guest answered nothing, but took his seat, and looked thoroughly indifferent what sentiments she cherished concerning him, she turned and whispered an earnest appeal for liberty to her tormentor.
\par “By no means!” cried Mrs. Linton in answer. “I won't be named a dog in the manger again. You shall stay: now then! Heathcliff, why don't you evince satisfaction at my pleasant news? Isabella swears that the love Edgar has for me is nothing to that she entertains for you. I'm sure she made some speech of the kind; did she not, Ellen? And she has fasted ever since the day before yesterday's walk, from sorrow and rage that I dispatched her out of your society under the idea of its being unacceptable.”
\par “I think you belie her,” said Heathcliff, twisting his chair to face them. “She wishes to be out of my society now, at any rate!”
\par And he stared hard at the object of discourse, as one might do at a strange repulsive animal: a centipede from the Indies, for instance, which curiosity leads one to examine in spite of the aversion it raises. The poor thing couldn't bear that: she grew white and red in rapid succession, and, while tears beaded her lashes, bent the strength of her small fingers to loosen the firm clutch of Catherine; and perceiving that as fast as she raised one finger off her arm another closed down, and she could not remove the whole together, she began to make use of her nails; and their sharpness presently ornamented the detainer's with crescents of red.
\par “There's a tigress!” exclaimed Mrs. Linton, setting her free, and shaking her hand with pain. “Begone, for God's sake, and hide your vixen face! How foolish to reveal those talons to him. Can't you fancy the conclusions he'll draw? Look, Heathcliff! they are instruments that will do execution — you must beware of your eyes.”
\par “I'd wrench them off her fingers, if they ever menaced me,” he answered brutally, when the door had closed after her. “But what did you mean by teasing the creature in that manner, Cathy? You were not speaking the truth, were you?”
\par “I assure you I was,” she returned. “She has been pining for your sake several weeks; and raving about you this morning, and pouring forth a deluge of abuse, because I represented your failings in a plain light, for the purpose of mitigating her adoration. But don't notice it further: I wished to punish her sauciness, that's all. I like her too well, my dear Heathcliff, to let you absolutely seize and devour her up.”
\par “And I like her too ill to attempt it,” said he, “except in a very ghoulish fashion. You'd hear of odd things if I lived alone with that mawkish, waxen face: the most ordinary would be painting on its white the colours of the rainbow, and turning the blue eyes black, every day or two: they detestably resemble Linton's.”
\par “Delectably!” observed Catherine. “They are dove's eyes —angel's!”
\par “She's her brother's heir, is she not?” he asked, after a brief silence.
\par “I should be sorry to think so,” returned his companion. “Half a dozen nephews shall erase her title, please Heaven! Abstract your mind from the subject at present: you are too prone to covet your neighbour's goods; remember this neighbour's goods are mine.”
\par “If they were mine, they would be none the less that,” said Heathcliff; “but though Isabella Linton may be silly, she is scarcely mad; and, in short, we'll dismiss the matter, as you advise.”
\par From their tongues they did dismiss it; and Catherine, probably, from her thoughts. The other, I felt certain, recalled it often in the course of the evening. I saw him smile to himself — grin rather — and lapse into ominous musing whenever Mrs. Linton had occasion to be absent from the apartment.
\par I determined to watch his movements. My heart invariably cleaved to the master's, in preference to Catherine's side: with reason I imagined, for he was kind, and trustful, and honourable; and she —she could not be called the opposite, yet she seemed to allow herself such wide latitude, that I had little faith in her principles, and still less sympathy for her feelings. I wanted something to happen which might have the effect of freeing both Wuthering Heights and the Grange of Mr. Heathcliff, quietly; leaving us as we had been prior to his advent. His visits were a continual nightmare to me; and, I suspected, to my master also. His abode at the Heights was an oppression past explaining. I felt that God had forsaken the stray sheep there to its own wicked wanderings, and an evil beast prowled between it and the fold, waiting his time to spring and destroy.










\subsection*{Chapter 11}

\par Sometimes, while meditating on these things in solitude, I've got up in a sudden terror, and put on my bonnet to go see how all was at the farm. I've persuaded my conscience that it was a duty to warn him how people talked regarding his ways; and then I've recollected his confirmed bad habits, and, hopeless of benefiting him, have flinched from re-entering the dismal house, doubting if I could bear to be taken at my word.
\par One time I passed the old gate, going out of my way, on a journey to Gimmerton. It was about the period that my narrative has reached: a bright frosty afternoon; the ground bare, and the road hard and dry. I came to a stone where the highway branches off on to the moor at your left hand; a rough sand pillar, with the letters W.H. cut on its north side, on the east, G., and on the south-west, T.G. It serves as guide-post to the Grange, the Heights, and village. The sun shone yellow on its grey head, reminding me of summer; and I cannot say why, but all at once, a gush of child's sensations flowed into my heart. Hindley and I held it a favourite spot twenty years before. I gazed long at the weather-worn block, and, stooping down, perceived a hole near the bottom still full of snail-shells and pebbles, which we were fond of storing there with more perishable things; and, as fresh as reality, it appeared that I beheld my early playmate seated on the withered turf: his dark, square head bent forward, and his little hand scooping out the earth with a piece of slate.“Poor Hindley!” I exclaimed involuntarily. I started: my bodily eye was cheated into a momentary belief that the child lifted its face and stared straight into mine! It vanished in a twinkling; but immediately I felt an irresistible yearning to be at the Heights. Superstition urged me to comply with this impulse: supposing he should be dead! I thought —or should die soon! — supposing it were a sign of death! The nearer I got to the house the more agitated I grew; and on catching sight of it I trembled in every limb. The apparition had outstripped me: it stood looking through the gate. That was my first idea on observing an elflocked, brown-eyed boy setting his ruddy countenance against the bars. Further reflection suggested this must be Hareton, my Hareton, not altered greatly since I left him, ten months since.
\par “God bless thee, darling!” I cried, forgetting instantaneously my foolish fears. “Hareton, it's Nelly! Nelly, thy nurse.”
\par He retreated out of arm's length, and picked up a large flint.
\par “I am come to see thy father, Hareton,” I added, guessing from the action that Nelly, if she lived in his memory at all, was not recognized as one with me.
\par He raised his missile to hurl it; I commenced a soothing speech, but could not stay his hand: the stone struck my bonnet; and then ensued, from the stammering lips of the little fellow, a string of curses, which, whether he comprehended them or not, were delivered with practised emphasis, and distorted his baby features into a shocking expression of malignity. You may be certain this grieved more than angered me. Fit to cry, I took an orange from my pocket, and offered it to propitiate him. He hesitated, and then snatched it from my hold; as if he fancied I only intended to tempt and disappoint him. I showed another, keeping it out of his reach.
\par “Who has taught you those fine words, my bairn?” I inquired. “The curate?”
\par “Damn the curate, and thee! Gie me that,” he replied.
\par “Tell us where you got your lessons, and you shall have it,” said I.“Who's your master?”
\par “Devil daddy,” was his answer.
\par “And what do you learn from daddy?” I continued.
\par He jumped at the fruit; I raised it higher. “What does he teach you?” I asked.
\par “Naught,” said he, “but to keep out of his gait. Daddy cannot bide me, because I swear at him.”
\par “Ah! and the devil teaches you to swear at daddy?” I observed.
\par “Ah — nay,” he drawled.
\par “Who then?”
\par “Heathcliff.”
\par I asked if he liked Mr. Heathcliff.
\par “Ay!” he answered again.
\par Desiring to have his reasons for liking him, I could only gather the sentences — “I known't: he pays dad back what he gies to me — he curses daddy for cursing me. He says I mun do as I will.”
\par “And the curate does not teach you to read and write then?” I pursued.
\par “No, I was told the curate should have his — teeth dashed down his throat, — if he stepped over the threshold — Heathcliff had promised that!”
\par I put the orange in his hand, and bade him tell his father that a woman called Nelly Dean was waiting to speak with him, by the garden gate. He went up the walk, and entered the house; but, instead of Hindley, Heathcliff appeared on the doorstones; and I turned directly and ran down the road as hard as ever I could race, making no halt till I gained the guide-post, and feeling as scared as if I had raised a goblin. This is not much connected with Miss Isabella's affair: except that it urged me to resolve further on mounting vigilant guard, and doing my utmost to check the spread of such bad influence at the Grange: even though I should wake a domestic storm, by thwarting Mrs. Linton's pleasure.
\par The next time Heathcliff came, my young lady chanced to be feeding some pigeons in the court. She had never spoken a word to her sister-in-law for three days; but she had likewise dropped her fretful complaining, and we found it a great comfort. Heathcliff had not the habit of bestowing a single unnecessary civility on Miss Linton, I knew. Now, as soon as he beheld her, his first precaution was to take a sweeping survey of the house front. I was standing by the kitchen window, but I drew out of sight. He then stepped across the pavement to her, and said something: she seemed embarrassed, and desirous of getting away; to prevent it, he laid his hand on her arm. She averted her face: he apparently put some question which she had no mind to answer. There was another rapid glance at the house, and supposing himself unseen, the scoundrel had the impudence to embrace her.
\par “Judas! Traitor!” I ejaculated. “You are a hypocrite, too, are you? A deliberate deceiver.”
\par “Who is, Nelly?” said Catherine's voice at my elbow: I had been over intent on watching the pair outside to mark her entrance.
\par “Your worthless friend!” I answered warmly: “the sneaking rascal yonder. Ah, he has caught a glimpse of us — he is coming in! I wonder will he have the art to find a plausible excuse for making love to Miss, when he told you he hated her?”
\par Mrs. Linton saw Isabella tear herself free, and run into the garden; and a minute after, Heathcliff opened the door. I couldn't withhold giving some loose to my indignation; but Catherine angrily insisted on silence, and threatened to order me out of the kitchen, if I dared to be so presumptuous as to put in my insolent tongue.
\par “To hear you, people might think you were the mistress!” she cried. “You want setting down in your right place! Heathcliff, what are you about, raising this stir? I said you must let Isabella alone! — I beg you will, unless you are tired of being received here, and wish Linton to draw the bolts against you!”
\par “God forbid that he should try!” answered the black villain. I detested him just then. “God keep him meek and patient! Every day I grow madder after sending him to heaven!”
\par “Hush!” said Catherine, shutting the inner door. “Don't vex me.Why have you disregarded my request? Did she come across you on purpose?”
\par “What is it to you?” he growled. “I have a right to kiss her, if she chooses; and you have no right to object. I'm not your husband: you needn't be jealous of me!”
\par “I'm not jealous of you,” replied the mistress, “I'm jealous for you. Clear your face: you shan't scowl at me! If you like Isabella, you shall marry her. But do you like her? Tell the truth, Heathcliff! There, you won't answer. I'm certain you don't!”
\par “And would Mr. Linton approve of his sister marrying that man?” I inquired.
\par “Mr. Linton should approve,” returned my lady, decisively.
\par “He might spare himself the trouble,” said Heathcliff: “I could do as well without his approbation. And as to you, Catherine, I have a mind to speak a few words now, while we are at it. I want you to be aware that I know you have treated me infernally — infernally! Do you hear? And if you flatter yourself that I don't perceive it, you are a fool; and if you think I can be consoled by sweet words, you are an idiot; and if you fancy I'll suffer unrevenged, I'll convince you of the contrary, in a very little while! Meantime, thank you for telling me your sister-in-law's secret: I swear I'll make the most of it. And stand you aside!”
\par “What new phase of his character is this?” exclaimed Mrs. Linton, in amazement. “I've treated you infernally — and you'll take your revenge! How will you take it, ungrateful brute? How have I treated you infernally?”
\par “I seek no revenge on you,” replied Heathcliff less vehemently.“That's not the plan. The tyrant grinds down his slaves and they don't turn against him; they crush those beneath them. You are welcome to torture me to death for your amusement, only allow me to amuse myself a little in the same style, and refrain from insult as much as you are able. Having levelled my palace, don't erect a hovel and complacently admire your own charity in giving me that for a home. If I imagined you really wished me to marry Isabel, I'd cut my throat!”
\par “Oh, the evil is that I am not jealous, is it?” cried Catherine.“Well, I won't repeat my offer of a wife: it is as bad as offering Satan a lost soul. Your bliss lies, like his, in inflicting misery. You prove it. Edgar is restored from the ill-temper he gave way to at your coming; I begin to be secure and tranquil; and you, restless to know us at peace, appear resolved on exciting a quarrel. Quarrel with Edgar, if you please, Heathcliff, and deceive his sister: you'll hit on exactly the most efficient method of revenging yourself on me.”
\par The conversation ceased. Mrs. Linton sat down by the fire, flushed and gloomy. The spirit which served her was growing intractable: she could neither lay nor control it. He stood on the hearth with folded arms, brooding on his evil thoughts; and in this position I left them to seek the master, who was wondering what kept Catherine below so long.
\par “Ellen,” said he, when I entered, “have you seen your mistress?”
\par “Yes; she's in the kitchen, sir,” I answered. “She's sadly put out by Mr. Heathcliff's behaviour: and, indeed, I do think it's time to arrange his visits on another footing. There's harm in being too soft, and now it's come to this — “ And I related the scene in the court, and, as near as I dared, the whole subsequent dispute. I fancied it could not be very prejudicial to Mrs. Linton; unless she made it so afterwards, by assuming the defensive for her guest. Edgar Linton had difficulty in hearing me to the close. His first words revealed that he did not clear his wife of blame.
\par “This is insufferable!” he exclaimed. “It is disgraceful that she should own him for a friend, and force his company on me! Call me two men out of the hall, Ellen. Catherine shall linger no longer to argue with the low ruffian — I have humoured her enough.”
\par He descended, and bidding the servants wait in the passage, went, followed by me, to the kitchen. Its occupants had recommenced their angry discussion: Mrs. Linton, at least, was scolding with renewed vigour; Heathcliff had moved to the window, and hung his head, somewhat cowed by her violent rating apparently. He saw the master first, and made a hasty motion that she should be silent; which she obeyed, abruptly, on discovering the reason of his intimation.
\par “How is this?” said Linton, addressing her; “what notion of propriety must you have to remain here, after the language which has been held to you by that blackguard? I suppose, because it is his ordinary talk, you think nothing of it; you are habituated to his baseness, and, perhaps, imagine I can get used to it too!”
\par “Have you been listening at the door, Edgar?” asked the mistress, in a tone particularly calculated to provoke her husband, implying both carelessness and contempt of his irritation. Heathcliff, who had raised his eyes at the former speech, gave a sneering laugh at the latter; on purpose, it seemed, to draw Mr. Linton's attention to him. He succeeded; but Edgar did not mean to entertain him with any high flights of passion.
\par “I have been so far forbearing with you, sir,” he said quietly; “not that I was ignorant of your miserable, degraded character, but I felt you were only partly responsible for that; and Catherine wishing to keep up your acquaintance, I acquiesced — foolishly. Your presence is a moral poison that would contaminate the most virtuous: for that cause, and to prevent worse consequences, I shall deny you hereafter admission into this house, and give notice now that I require your instant departure. Three minutes' delay will render it involuntary and ignominious.”
\par Heathcliff measured the height and breadth of the speaker with an eye full of derision.
\par “Cathy, this lamb of yours threatens like a bull!” he said. “It is in danger of splitting its skull against my knuckles. By God! Mr. Linton, I'm mortally sorry that you are not worth knocking down!”
\par My master glanced towards the passage, and signed me to fetch the men: he had no intention of hazarding a personal encounter. I obeyed the hint; but Mrs. Linton, suspecting something, followed; and when I attempted to call them, she pulled me back, slammed the door to, and locked it.
\par “Fair means!” she said, in answer to her husband's look of angry surprise “If you have not courage to attack him, make an apology, or allow yourself to be beaten. It will correct you of feigning more valour than you possess. No, I'll swallow the key before you shall get it! I'm delightfully rewarded for my kindness to each! After constant indulgence of one's weak nature, and the other's bad one, I earn for thanks two samples of blind ingratitude, stupid to absurdity! Edgar, I was defending you and yours; and I wish Heathcliff may flog you sick, for daring to think an evil thought of me!”
\par It did not need the medium of a flogging to produce that effect on the master. He tried to wrest the key from Catherine's grasp, and for safety she flung it into the hottest part of the fire; whereupon Mr. Edgar was taken with a nervous trembling, and his countenance grew deadly pale. For his life he could not avert that access of emotion; mingled anguish and humiliation overcame him completely. He leant on the back of a chair, and covered his face.
\par “Oh, heavens! In old days, this would win you knighthood!” exclaimed Mrs. Linton. “We are vanquished! We are vanquished! Heathcliff would as soon lift a finger at you as a king would march his army against a colony of mice. Cheer up, you shan't be hurt! Your type is not a lamb, it's a sucking leveret.”
\par “I wish you joy of the milk-blooded coward, Cathy!” said her friend. “I compliment you on your taste. And that is the slavering, shivering thing you preferred to me! I would not strike him with my fist, but I'd kick him with my foot, and experience considerable satisfaction. Is he weeping, or is he going to faint for fear?”
\par The fellow approached and gave the chair on which Linton rested a push. He'd better have kept his distance; my master quickly sprang erect, and struck him full on the throat a blow that would have levelled a slighter man. It took his breath for a minute; and while he choked, Mr. Linton walked out by the back door into the yard, and from thence to the front entrance.
\par “There! you've done with coming here,” cried Catherine. “Get away, now; he'll return with a brace of pistols, and half a dozen assistants. If he did overhear us, of course he'd never forgive you. You've played him an ill turn, Heathcliff! But go — make haste! I'd rather see Edgar at bay than you.”
\par “Do you suppose I'm going with that blow burning in my gullet?” he thundered. “By hell, no! I'll crush his ribs in like a rotten hazel nut before I cross the threshold! If I don't floor him now, I shall murder him some time; so, as you value his existence, let me get at him!”
\par “He's not coming,” I interposed, framing a bit of a lie. “There's the coachman and the two gardeners; you'll surely not wait to be thrust into the road by them! Each has a bludgeon; and master will, very likely, be watching from the parlour windows, to see that they fulfil his orders.”
\par The gardeners and coachman were there; but Linton was with them. They had already entered the court. Heathcliff, on second thoughts, resolved to avoid a struggle against the three underlings; he seized the poker, smashed the lock from the inner door, and made his escape as they tramped in.
\par Mrs. Linton, who was very much excited, bade me accompany her upstairs. She did not know my share in contributing to the disturbance, and I was anxious to keep her in ignorance.
\par “I'm nearly distracted, Nelly!” she exclaimed, throwing herself on the sofa. “A thousand smiths' hammers are beating in my head! Tell Isabella to shun me; this uproar is owing to her; and should she or anyone else aggravate my anger at present, I shall get wild. And, Nelly, say to Edgar, if you see him again tonight, that I'm in danger of being seriously ill. I wish it may prove true. He has startled and distressed me shockingly! I want to frighten him. Besides, he might come and begin a string of abuse or complainings; I'm certain I should recriminate, and God knows where we should end! Will you do so, my good Nelly? You are aware that I am in no way blamable in this matter. What possessed him to turn listener? Heathcliff's talk was outrageous, after you left us; but I could soon have diverted him from Isabella, and the rest meant nothing. Now all is dashed wrong by the fool's craving to hear evil of self, that haunts some people like a demon! Had Edgar never gathered our conversation, he would never have been the worse for it. Really, when he opened on me in that unreasonable tone of displeasure after I had scolded Heathcliff till I was hoarse for him, I did not care, hardly, what they did to each other; especially as I felt that, however the scene closed, we should all be driven asunder for nobody knows how long! Well, if I cannot keep Heathcliff for my friend — if Edgar will be mean and jealous, I'll try to break their hearts by breaking my own. That will be a prompt way of finishing all, when I am pushed to extremity! But it's a deed to be reserved for a forlorn hope; I'd not take Linton by surprise with it. To this point he has been discreet in dreading to provoke me; you must represent the peril of quitting that policy, and remind him of my passionate temper, verging, when kindled, on frenzy. I wish you could dismiss that apathy out of your countenance, and look rather more anxious about me.”
\par The stolidity with which I received these instructions was, no doubt, rather exasperating: for they were delivered in perfect sincerity; but I believed a person who could plan the turning of her fits of passion to account, beforehand, might, by exerting her will, manage to control herself tolerably, even while under their influence; and I did not wish to“frighten” her husband, as she said, and multiply his annoyances for the purpose of serving her selfishness. Therefore I said nothing when I met the master coming towards the parlour; but I took the liberty of turning back to listen whether they would resume their quarrel together. He began to speak first.
\par “Remain where you are, Catherine,” he said; without any anger in his voice, but with much sorrowful despondency. “I shall not stay. I am neither come to wrangle nor be reconciled; but I wish just to learn whether, after this evening's events, you intend to continue your intimacy with — ”
\par “Oh, for mercy's sake,” interrupted the mistress, stamping her foot, “for mercy's sake, let us hear no more of it now! Your cold blood cannot be worked into a fever: your veins are full of ice-water; but mine are boiling, and the sight of such chillness makes them dance.”
\par “To get rid of me, answer my question,” persevered Mr. Linton.“You must answer it; and that violence does not alarm me. I have found that you can be as stoical as anyone, when you please. Will you give up Heathcliff hereafter, or will you give up me? It is impossible for you to be my friend and his at the same time; and I absolutely require to know which you choose.”
\par “I require to be let alone!” exclaimed Catherine furiously. “I demand it! Don't you see I can scarcely stand? Edgar, you — you leave me!”
\par She rang the bell till it broke with a twang; I entered leisurely. It was enough to try the temper of a saint, such senseless, wicked rages! There she lay dashing her head against the arm of the sofa, and grinding her teeth, so that you might fancy she would crash them to splinters! Mr. Linton stood looking at her in sudden compunction and fear. He told me to fetch some water. She had no breath for speaking. I brought a glass full; and as she would not drink, I sprinkled it on her face. In a few seconds she stretched herself out stiff, and turned up her eyes, while her cheeks, at once blanched and livid, assumed the aspect of death. Linton looked terrified.
\par “There is nothing in the world the matter,” I whispered. I did not want him to yield, though I could not help being afraid in my heart.
\par “She has blood on her lips!” he said, shuddering.
\par “Never mind!” I answered tartly. And I told him how she had resolved, previous to his coming, on exhibiting a fit of frenzy. I incautiously gave the account aloud, and she heard me; for she started up — her hair flying over her shoulders, her eyes flashing, the muscles of her neck and arms standing out preternaturally. I made up my mind for broken bones, at least; but she only glared about her for an instant, and then rushed from the room. The master directed me to follow; I did, to her chamber door: she hindered me from going farther by securing it against me.
\par As she never offered to descend to breakfast next morning, I went to ask whether she would have some carried up.
\par “No!” she replied peremptorily.
\par The same question was repeated at dinner and tea; and again on the morrow after, and received the same answer.
\par Mr. Linton, on his part, spent his time in the library, and did not inquire concerning his wife's occupations. Isabella and he had had an hour's interview, during which he tried to elicit from her some sentiment of proper horror for Heathcliff's advances: but he could make nothing of her evasive replies, and was obliged to close the examination unsatisfactorily; adding, however, a solemn warning, that if she were so insane as to encourage that worthless suitor, it would dissolve all bonds of relationship between herself and him.









\subsection*{Chapter 12}

\par While Miss Linton moped about the park and garden, always silent, and almost always in tears; and her brother shut himself up among books that he never opened — wearying, I guessed, with a continual vague expectation that Catherine, repenting her conduct, would come of her own accord to ask pardon, and seek a reconciliation— and she fasted pertinaciously, under the idea, probably, that at every meal, Edgar was ready to choke for her absence, and pride alone held him from running to cast himself at her feet: I went about my household duties, convinced that the Grange had but one sensible soul in its walls, and that lodged in my body.
\par I wasted no condolences on Miss, nor any expostulations on my mistress; nor did I pay much attention to the sighs of my master, who yearned to hear his lady's name, since he might not hear her voice.
\par I determined they should come about as they pleased for me; and though it was a tiresomely slow process, I began to rejoice at length in a faint dawn of its progress, as I thought at first.
\par Mrs. Linton, on the third day, unbarred her door, and having finished the water in her pitcher and decanter, desired a renewed supply, and a basin of gruel, for she believed she was dying. That I set down as a speech meant for Edgar's ears; I believed no such thing, so I kept it to myself and brought her some tea and dry toast.
\par She ate and drank eagerly; and sank back on her pillow again, clenching her hands and groaning.
\par “Oh, I will die,” she exclaimed, “since no one cares anything about me. I wish I had not taken that.”
\par Then a good while after I heard her murmur,
\par “No, I'll not die — he'd be glad — he does not love me at all —he would never miss me!”
\par “Did you want anything, ma' am?” I inquired, still preserving my external composure, in spite of her ghastly countenance, and strange exaggerated manner.
\par “What is that apathetic being doing?” she demanded, pushing her thick entangled locks from her wasted face. “Has he fallen into a lethargy, or is he dead?”
\par “Neither,” replied I; “if you mean Mr. Linton. He's tolerably well, I think, though his studies occupy him rather more than they ought: he is continually among his books, since he has no other society.”
\par I should not have spoken so if I had known her true condition, but I could not get rid of the notion that she acted a part of her disorder.
\par “Among his books!” she cried, confounded. “And I dying! I on the brink of the grave! My God! Does he know how I'm altered?” continued she, staring at her reflection in a mirror hanging against the opposite wall. “Is that Catherine Linton! He imagines me in a pet — in play, perhaps. Cannot you inform him that it is frightful earnest? Nelly, if it be not too late, as soon as I learn how he feels, I'll choose between these two; either to starve at once — that would be no punishment unless he had a heart — or to recover, and leave the country. Are you speaking the truth about him now? Take care. Is he actually so utterly indifferent for my life?”
\par “Why, ma' am,” I answered, “the master has no idea of your being deranged; and of course he does not fear that you will let yourself die of hunger.”
\par “You think not? Cannot you tell him I will?” she returned.“Persuade him! Speak of your own mind: say you are certain I will!”
\par “No, you forget, Mrs. Linton,” I suggested, “that you have eaten some food with a relish this evening, and tomorrow you will perceive its good effects.”
\par “If I were only sure it would kill him,” she interrupted, “I'd kill myself directly! These three awful nights, I've never closed my lids—and oh, I've been tormented! I've been haunted, Nelly! But I begin to fancy you don't like me. How strange! I thought, though everybody hated and despised each other, they could not avoid loving me. And they have all turned to enemies in a few hours: they have, I'm positive; the people here. How dreary to meet death, surrounded by their cold faces! Isabella terrified and repelled, afraid to enter the room, it would be so dreadful to watch Catherine go. And Edgar standing solemnly by to see it over; then offering prayers of thanks to God for restoring peace to his house, and going back to his books! What in the name of all that feels has he to do with books, when I am dying?”
\par She could not bear the notion which I had put into her head of Mr. Linton's philosophical resignation. Tossing about, she increased her feverish bewilderment to madness, and tore the pillow with her teeth; then raising herself up all burning, desired that I would open the window. We were in the middle of winter, the wind blew strong from the north-east, and I objected. Both the expressions flitting over her face, and the changes of her moods, began to alarm me terribly; and brought to my recollection her former illness, and the doctor's injunction that she should not be crossed. A minute previously she was violent; now, supported on one arm, and not noticing my refusal to obey her, she seemed to find childish diversion in pulling the feathers from the rents she had just made, and ranging them on the sheet according to their different species: her mind had strayed to other associations.
\par “That's a turkey's,” she murmured to herself; “and this is a wild duck's; and this is a pigeon's. Ah, they put pigeons' feathers in the pillows — no wonder I couldn't die! Let me take care to throw it on the floor when I lie down. And here is a moor-cock's; and this — I should know it among a thousand — it's a lapwing's. Bonny bird; wheeling over our heads in the middle of the moor. It wanted to get to its nest, for the clouds had touched the swells, and it felt rain coming. This feather was picked up from the heath, the bird was not shot: we saw its nest in the winter, full of little skeletons. Heathcliff set a trap over it, and the old ones dare not come. I made him promise he'd never shoot a lapwing after that, and he didn't. Yes, here are more! Did he shoot my lapwings, Nelly? Are they red, any of them! Let me look.”
\par “Give over with that baby-work!” I interrupted, dragging the pillow away, and turning the holes towards the mattress, for she was removing its contents by handfuls. “Lie down and shut your eyes: you're wandering. There's a mess! The down is flying about like snow.”
\par I went here and there collecting it.
\par “I see in you, Nelly,” she continued dreamily, “an aged woman: you have grey hair and bent shoulders. This bed is the fairy cave under peniston Crag, and you are gathering elf-bolts to hurt our heifers; pretending, while I am near, that they are only locks of wool. That's what you'll come to fifty years hence: I know you are not so now. I'm not wandering: you're mistaken, or else I should believe you really were that withered hag, and I should think I was under peniston Crag; and I'm conscious it's night, and there are two candles on the table making the black press shine like jet.”
\par “The black press? Where is that?” I asked. “You are talking in your sleep!”
\par “It's against the wall, as it always is,” she replied. “It does appear odd — I see a face in it!”
\par “There's no press in the room, and never was,” said I, resuming my seat, and looping up the curtain that I might watch her.
\par “Don't you see that face?” she inquired, gazing earnestly at the mirror.
\par And say what I could, I was incapable of making her comprehend it to be her own; so I rose and covered it with a shawl.
\par “It's behind there still!” she pursued anxiously. “And it stirred. Who is it? I hope it will not come out when you are gone! Oh! Nelly, the room is haunted! I'm afraid of being alone!”
\par I took her hand in mine, and bid her be composed: for a succession of shudders convulsed her frame, and she would keep straining her gaze towards the glass.
\par “There's nobody here!” I insisted. “It was yourself, Mrs. Linton: you knew it a while since.”
\par “Myself!” she gasped, “and the clock is striking twelve! It's true, then! That's dreadful!”
\par Her fingers clutched the clothes, and gathered them over her eyes. I attempted to steal to the door with an intention of calling her husband; but I was summoned back by a piercing shriek — the shawl had dropped from the frame.
\par “Why, what is the matter?” cried I. “Who is coward now? Wake up! That is the glass — the mirror, Mrs. Linton; and you see yourself in it, and there am I too, by your side.”
\par Trembling and bewildered, she held me fast, but the horror gradually passed from her countenance; its paleness gave place to a glow of shame.
\par “Oh, dear! I thought I was at home,” she sighed. “I thought I was lying in my chamber at Wuthering Heights. Because I'm weak, my brain got confused, and I screamed unconsciously. Don't say anything; but stay with me. I dread sleeping: my dreams appal me.”
\par “A sound sleep would do you good, ma' am,” I answered; “and I hope this suffering will prevent your trying starving again.”
\par “Oh, if I were but in my own bed in the old house!” she went on bitterly, wringing her hands, “And that wind sounding in the firs by the lattice. Do let me feel it — it comes straight down the moor — do let me have one breath!”
\par To pacify her, I held the casement ajar a few seconds. A cold blast rushed through; I closed it, and returned to my post. She lay still now, her face bathed in tears. Exhaustion of body had entirely subdued her spirit: our fiery Catherine was no better than a wailing child.
\par “How long is it since I shut myself in here?” she asked, suddenly reviving.
\par “It was Monday evening,” I replied, “and this is Thursday night, or rather Friday morning, at present.
\par “What! — Of the same week?” she exclaimed. “Only that brief time?”
\par “Long enough to live on nothing but cold water and ill-temper,” observed I.
\par “Well, it seems a weary number of hours,” she muttered doubtfully:“it must be more. I remember being in the parlour after they had quarrelled, and Edgar being cruelly provoking, and me running into this room desperate. As soon as ever I had barred the door, utter blackness overwhelmed me, and I fell on the floor. I couldn't explain to Edgar how certain I felt of having a fit, or going raging mad, if he persisted in teasing me! I had no command of tongue, or brain, and he did not guess my agony, perhaps: it barely left me sense to try to escape from him and his voice. Before I recovered sufficiently to see and hear, it began to be dawn, and, Nelly, I'll tell you what I thought, and what has kept recurring and recurring till I feared for my reason. I thought as I lay there, with my head against that table leg, and my eyes dimly discerning the grey square of the window, that I was enclosed in the oak-panelled bed at home; and my heart ached with some great grief which, just waking, I could not recollect. I pondered, and worried myself to discover what it could be, and, most strangely, the whole last seven years of my life grew a blank! I did not recall that they had been at all. I was a child; my father was just buried, and my misery arose from the separation that Hindley had ordered between me and Heathcliff. I was laid alone, for the first time; and, rousing from a dismal doze after a night of weeping, I lifted my hand to push the panels aside: it struck the table top! I swept it along the carpet, and then memory burst in: my late anguish was swallowed in a paroxysm of despair. I cannot say why I felt so wildly wretched: it must have been temporary derangement, for there is scarcely cause. But, supposing at twelve years old I had been wrenched from the Heights, and every early association, and my all in all, as Heathcliff was at that time, and been converted at a stroke into Mrs. Linton, the lady of Thrushcross Grange, and the wife of a stranger: an exile, and outcast, thenceforth, from what had been my world. You may fancy a glimpse of the abyss where I grovelled! Shake your head as you will, Nelly, you have helped to unsettle me! You should have spoken to Edgar, indeed you should, and compelled him to leave me quiet! Oh, I'm burning! I wish I were out of doors! I wish I were a girl again, half savage and hardy, and free; and laughing at injuries, not maddening under them! Why am I so changed? Why does my blood rush into a hell of tumult at a few words? I'm sure I should be myself were I once among the heather on those hills. Open the window again wide: fasten it open! Quick, why don't you move?”
\par “Because I won't give you your death of cold,” I answered.
\par “You won't give me a chance of life, you mean,” she said sullenly.“However, I'm not helpless, yet: I'll open it myself.”
\par And sliding from the bed before I could hinder her, she crossed the room, walking very uncertainly, threw it back, and bent out, careless of the frosty air that cut about her shoulders as keen as a knife. I entreated, and finally attempted to force her to retire. But I soon found her delirious strength much surpassed mine (she was delirious, I became convinced by her subsequent actions and ravings). There was no moon, and everything beneath lay in misty darkness: not a light gleamed from any house, far or near — all had been extinguished long ago; and those at Wuthering Heights were never visible — still she asserted she caught their shining.
\par “Look!” she cried eagerly, “that's my room with the candle in it, and the trees swaying before it: and the other candle is in Joseph's garret. Joseph sits up late, doesn't he? He's waiting till I come home that he may lock the gate. Well, he'll wait a while yet. It's a rough journey, and a sad heart to travel it; and we must pass by Gimmerton Kirk, to go that journey! We've braved its ghosts often together, and dared each other to stand among the graves and ask them to come. But, Heathcliff, if I dare you now, will you venture? If you do, I'll keep you. I'll not lie there by myself: they may bury me twelve feet deep, and throw the church down over me, but I won't rest till you are with me. I never will!”
\par She paused, and resumed with a strange smile. “He's considering—he'd rather I'd come to him! Find a way, then! not through that kirkyard. You are slow! Be content, you always followed me!”
\par Perceiving it vain to argue against her insanity, I was planning how I could reach something to wrap about her, without quitting my hold of herself (for I could not trust her alone by the gaping lattice), when, to my consternation, I heard the rattle of the door handle, and Mr. Linton entered. He had only then come from the library; and, in passing through the lobby, had noticed our talking and been attracted by curiosity, or fear, to examine what it signified, at that late hour.
\par “Oh, sir!” I cried, checking the exclamation risen to his lips at the sight which met him, and the bleak atmosphere of the chamber. “My poor mistress is ill, and she quite masters me: I cannot manage her at all; pray, come and persuade her to go to bed. Forget your anger, for she's hard to guide any way but her own.”
\par “Catherine ill?” he said, hastening to us. “Shut the window, Ellen! Catherine! Why — ”
\par He was silent. The haggardness of Mrs. Linton's appearance smote him speechless, and he could only glance from her to me in horrified astonishment.
\par “She's been fretting here,” I continued, “and eating scarcely anything, and never complaining; she would admit none of us till this evening, and so we couldn't inform you of her state as we were not aware of it ourselves; but it is nothing.”
\par I felt I uttered my explanations awkwardly; the master frowned. “It is nothing, is it, Ellen Dean?” he said sternly. “You shall account more clearly for keeping me ignorant of this!” And he took his wife in his arms, and looked at her with anguish.
\par At first she gave him no glance of recognition; he was invisible to her abstracted gaze. The delirium was not fixed, however; having weaned her eyes from contemplating the outer darkness, by degrees she centred her attention on him, and discovered who it was that held her.
\par “Ah! You are come, are you, Edgar Linton?” she said, with angry animation. “You are one of those things that are ever found when least wanted, and when you are wanted, never! I suppose we shall have plenty of lamentations now — I see we shall — but they can't keep me from my narrow home out yonder: my resting-place, where I'm bound before spring is over! There it is: not among the Lintons, mind, under the chapel roof, but in the open air, with a headstone; and you may please yourself, whether you go to them or come to me!”
\par “Catherine, what have you done?” commenced the master. “Am I nothing to you any more? Do you love that wretch Heath — ”
\par “Hush!” cried Mrs. Linton. “Hush, this moment! You mention that name and I end the matter instantly, by a spring from the window! What you touch at present you may have; but my soul will be on that hill top before you lay hands on me again. I don't want you, Edgar: I'm past wanting you. Return to your books. I'm glad you possess a consolation, for all you had in me is gone.”
\par “Her mind wanders, sir,” I interposed. “She has been talking nonsense the whole evening; but let her have quiet, and proper attendance, and she'll rally. Hereafter, we must be cautious how we vex her.”
\par “I desire no further advice from you,” answered Mr. Linton. “You know your mistress's nature, and you encouraged me to harass her. And not to give me one hint of how she has been these three days! It was heartless! Months of sickness could not cause such a change!”
\par I began to defend myself, thinking it too bad to be blamed for another's wicked waywardness.
\par “I knew Mrs. Linton's nature to be headstrong and domineering,” cried I; “but I didn't know that you wished to foster her fierce temper! I didn't know that, to humour her, I should wink at Mr. Heathcliff. I performed the duty of a faithful servant in telling you, and I have got a faithful servant's wages! Well, it will teach me to be careful next time. Next time you may gather intelligence for yourself!”
\par “The next time you bring a tale to me, you shall quit my service, Ellen Dean,” he replied.
\par “You'd rather hear nothing about it, I suppose, then, Mr. Linton?” said I. “Heathcliff has your permission to come a courting to miss, and to drop in at every opportunity your absence offers, on purpose to poison the mistress against you?”
\par Confused as Catherine was, her wits were alert at applying our conversation.
\par “Ah! Nelly has played traitor,” she exclaimed passionately. “Nelly is my hidden enemy. You witch! So you do seek elf-bolts to hurt us! Let me go, I'll make her rue! I'll make her howl a recantation!”
\par A maniac's fury kindled under her brows; she struggled desperately to disengage herself from Linton's arms. I felt no inclination to tarry the event; and, resolving to seek medical aid on my own responsibility, I quitted the chamber.
\par In passing the garden to reach the road, at a place where a bridle hook is driven into the wall, I saw something white moved irregularly, evidently by another agent than the wind. Notwithstanding my hurry, I stayed to examine it, lest ever after I should have the conviction impressed on my imagination that it was a creature of the other world.
\par My surprise and perplexity were great to discover, by touch more than vision, Miss Isabella's springer, Fanny, suspended by a handkerchief, and nearly at its last gasp. I quickly released the animal, and lifted it into the garden. I had seen it follow its mistress upstairs when she went to bed; and wondered much how it could have got out there, and what mischievous person had treated it so. While untying the knot round the hook, it seemed to me that I repeatedly caught the beat of horses'feet galloping at some distance; but there were such a number of things to occupy my reflections that I hardly gave the circumstance a thought: though it was a strange sound, in that place, at two o'clock in the morning.
\par Mr. Kenneth was fortunately just issuing from his house to see a patient in the village as I came up the street; and my account of Catherine Linton's malady induced him to accompany me back immediately. He was a plain rough man; and he made no scruple to speak his doubts of her surviving this second attack; unless she were more submissive to his directions than she had shown herself before.
\par “Nelly Dean,” said he, “I can't help fancying there's an extra cause for this. What has there been to do at the Grange? We've odd reports up here. A stout, hearty lass like Catherine, does not fall ill for a trifle; and that sort of people should not either. It's hard work bringing them through fevers, and such things. How did it begin?”
\par “The master will inform you,” I answered; “but you are acquainted with the Earnshaws' violent dispositions, and Mrs. Linton caps them all. I may say this: it commenced in a quarrel. She was struck during a tempest of passion with a kind of fit. That's her account, at least; for she flew off in the height of it, and locked herself up. Afterwards, she refused to eat, and now she alternately raves and remains in a halfdream; knowing those about her, by having her mind filled with all sorts of strange ideas and illusions.”
\par “Mr. Linton will be sorry?” observed Kenneth, interrogatively.
\par “Sorry? He'll break his heart should anything happen!” I replied.“Don't alarm him more than necessary.”
\par “Well, I told him to beware,” said my companion; “and he must bide the consequences of neglecting my warning! Hasn't he been thick with Mr. Heathcliff, lately?”
\par “Heathcliff frequently visits at the Grange,” answered I, “though more on the strength of the mistress having known him when a boy, than because the master likes his company. At present, he's discharged from the trouble of calling; owing to some presumptuous aspirations after Miss Linton which he manifested. I hardly think he'll be taken in again.”
\par “And does Miss Linton turn a cold shoulder on him?” was the doctor's next question.
\par “I'm not in her confidence,” returned I, reluctant to continue the subject.
\par “No, she's a sly one,” he remarked, shaking his head. “She keeps her own counsel! But she's a real little fool. I have it from good authority, that, last night (and a pretty night it was!) she and Heathcliff were walking in the plantation at the back of your house, above two hours; and he pressed her not to go in again, but just mount his horse and away with him! My informant said she could only put him off by pledging her word of honour to be prepared on their first meeting after that: when it was to be, he didn't hear; but you urge Mr. Linton to look sharp!”
\par This news filled me with fresh fears; I outstripped Kenneth, and ran most of the way back. The little dog was yelping in the garden yet. I spared a minute to open the gate for it, but instead of going to the house door, it coursed up and down snuffing the grass, and would have escaped to the road, had I not seized and conveyed it in with me.
\par On ascending to Isabella's room, my suspicions were confirmed: it was empty. Had I been a few hours sooner, Mrs. Linton's illness might have arrested her rash step. But what could be done now? There was a bare possibility of overtaking them if pursued instantly. I could not pursue them, however; and I dare not rouse the family, and fill the place with confusion; still less unfold the business to my master, absorbed as he was in his present calamity, and having no heart to spare for a second grief!
\par I saw nothing for it but to hold my tongue, and suffer matters to take their course; and Kenneth being arrived, I went with a badly composed countenance to announce him.
\par Catherine lay in a troubled sleep: her husband had succeeded in soothing the access of frenzy: he now hung over her pillow, watching every shade, and every change of her painfully expressive features.
\par The doctor, on examining the case for himself, spoke hopefully to him of its having a favourable termination, if we could only preserve around her perfect and constant tranquillity. To me, he signified the threatening danger was not so much death, as permanent alienation of intellect.
\par I did not close my eyes that night, nor did Mr. Linton: indeed, we never went to bed; and the servants were all up long before the usual hour, moving through the house with stealthy tread, and exchanging whispers as they encountered each other in their vocations. Everyone was active, but Miss Isabella; and they began to remark how sound she slept: her brother, too, asked if she had risen, and seemed impatient for her presence, and hurt that she showed so little anxiety for her sister-inlaw.
\par I trembled lest he should send me to call her; but I was spared the pain of being the first proclaimant of her flight. One of the maids, a thoughtless girl, who had been on an early errand to Gimmerton, came panting upstairs, openmouthed, and dashed into the chamber, crying:“Oh, dear, dear! What mun we have next? Master, master, our young lady — ”
\par “Hold your noise!” cried I hastily, enraged at her clamorous manner.
\par “Speak lower, Mary — What is the matter?” said Mr. Linton. “What ails your young lady?”
\par “She's gone, she's gone! Yon' Heathcliff's run off wi' her!” gasped the girl.
\par “That is not true!” exclaimed Linton, rising in agitation. “It cannot be: how has the idea entered your head? Ellen Dean, go and seek her. It is incredible: it cannot be.”
\par As he spoke he took the servant to the door, and then repeated his demand to know her reasons for such an assertion.
\par “Why, I met on the road a lad that fetches milk here,” she stammered, “and he asked whether we weren't in trouble at the Grange. I thought he meant for missis's sickness, so I answered, yes. Then says he, “They's somebody gone after ‘em, I guess?” I stared. He saw I knew nought about it, and he told how a gentleman and lady had stopped to have a horse's shoe fastened at a blacksmith's shop, two miles out of Gimmerton, not very long after midnight! And how the blacksmith's lass had got up to spy who they were: she knew them both directly. And she noticed the man — Heathcliff it was, she felt certain: nob'dy could mistake him, besides — put a sovereign in her father's hand for payment. The lady had a cloak about her face; but having desired a sup of water, while she drank, it fell back, and she saw her very plain. Heathcliff held both bridles as they rode on, and they set their faces from the village, and went as fast as the rough roads would let them. The lass said nothing to her father, but she told it all over Gimmerton this morning.”
\par I ran and peeped, for form's sake, into Isabella's room; confirming, when I returned, the servant's statement. Mr. Linton had resumed his seat by the bed; on my re-entrance, he raised his eyes, read the meaning of my blank aspect, and dropped them without giving an order, or uttering a word.
\par “Are we to try any measures for overtaking and bringing her back?” I inquired. “How should we do?”
\par “She went of her own accord,” answered the master; “she had a right to go if she pleased. Trouble me no more about her. Hereafter she is only my sister in name: not because I disown her, but because she has disowned me.”
\par And that was all he said on the subject: he did not make a single inquiry further, or mention her in any way, except directing me to send what property she had in the house to her fresh home, wherever it was, when I knew it.








\subsection*{Chapter 13}

\par For two months the fugitives remained absent; in those two months, Mrs. Linton encountered and conquered the worst shock of what was denominated a brain fever. No mother could have nursed an only child more devotedly than Edgar tended her. Day and night he was watching, and patiently enduring all the annoyances that irritable nerves and a shaken reason could inflict; and, though Kenneth remarked that what he saved from the grave would only recompense his care by forming the source of constant future anxiety — in fact, that his health and strength were being sacrificed to preserve a mere ruin of humanity — he knew no limits in gratitude and joy when Catherine's life was declared out of danger; and hour after hour he would sit beside her, tracing the gradual return to bodily health, and flattering his too sanguine hopes with the illusion that her mind would settle back to its right balance also, and she would soon be entirely her former self.
\par The first time she left her chamber was at the commencement of the following March. Mr. Linton had put on her pillow, in the morning, a handful of golden crocuses; her eye, long stranger to any gleam of pleasure, caught them in waking, and shone delighted as she gathered them eagerly together.
\par “These are the earliest flowers at the Heights,” she exclaimed.“They remind me of soft thaw winds, and warm sunshine, and nearly melted snow. Edgar, is there not a south wind, and is not the snow almost gone?”
\par “The snow is quite gone down here, darling,” replied her husband;“and I only see two white spots on the whole range of moors: the sky is blue, and the larks are singing, and the becks and brooks are all brim full. Catherine, last spring at this time, I was longing to have you under this roof, now, I wish you were a mile or two up those hills: the air blows so sweetly, I feel that it would cure you.
\par “I shall never be there but once more,” said the invalid; “and then you'll leave me, and I shall remain for ever. Next spring you'll long again to have me under this roof, and you'll look back and think you were happy today.”
\par Linton lavished on her the kindest caresses, and tried to cheer her by the fondest words; but, vaguely regarding the flowers, she let the tears collect on her lashes and stream down her cheeks unheeding.
\par We knew she was really better, and, therefore, decided that long confinement to a single place produced much of this despondency, and it might be partially removed by a change of scene.
\par The master told me to light a fire in the many-weeks-deserted parlour, and to set an easy-chair in the sunshine by the window; and then he brought her down, and she sat a long while enjoying the genial heat, and, as we expected, revived by the objects round her: which, though familiar, were free from the dreary associations investing her hated sick chamber. By evening, she seemed greatly exhausted; yet no arguments could persuade her to return to that apartment, and I had to arrange the parlour sofa for her bed, till another room could be prepared. To obviate the fatigue of mounting and descending the stairs, we fitted up this, where you lie at present: on the same floor with the parlour; and she was soon strong enough to move from one to the other, leaning on Edgar's arm. Ah, I thought myself she might recover, so waited on as she was. And there was double cause to desire it, for on her existence depended that of another: we cherished the hope that in a little while, Mr. Linton's heart would be gladdened, and his lands secured from a stranger's gripe, by the birth of an heir.
\par I should mention that Isabella sent to her brother, some six weeks from her departure, a short note, announcing her marriage with Heathcliff. It appeared dry and cold; but at the bottom was dotted in with pencil an obscure apology, and an entreaty for kind remembrance and reconciliation, if her proceeding had offended him: asserting that she could not help it then, and being done, she had now no power to repeal it. Linton did not reply to this, I believe; and, in a fortnight more, I got a long letter which I considered odd, coming from the pen of a bride just out of the honeymoon. I'll read it, for I keep it yet. Any relic of the dead is precious, if they were valued living.
\par 
\par DEAR ELLEN, it begins:
\par I came last night to Wuthering Heights, and heard, for the first time, that Catherine has been, and is yet, very ill. I must not write to her, I suppose, and my brother is either too angry or too distressed to answer what I sent him. Still, I must write to somebody, and the only choice left me is you.
\par Inform Edgar that I'd give the world to see his face again — that my heart returned to Thrushcross Grange in twenty-four hours after I left it, and is there at this moment, full of warm feelings for him, and Catherine! I can't follow it, though — (those words are underlined) —they need not expect me, and they may draw what conclusions they please; taking care, however, to lay nothing at the door of my weak will or deficient affection.
\par The remainder of the letter is for yourself alone. I want to ask you two questions: the first is — How did you contrive to preserve the common sympathies of human nature when you resided here? I cannot recognize any sentiment which those around share with me.
\par The second question, I have great interest in; it is this — Is Mr. Heathcliff a man? If so, is he mad? And if not, is he a devil? I shan't tell my reasons for making this inquiry; but I beseech you to explain, if you can, what I have married: that is, when you call to see me; and you must call, Ellen, very soon. Don't write, but come, and bring me something from Edgar.
\par Now, you shall hear how I have been received in my new home, as I am led to imagine the Heights will be. It is to amuse myself that I dwell on such subjects as the lack of external comforts: they never occupy my thoughts, except at the moment when I miss them. I should laugh and dance for joy, if I found their absence was the total of my miseries, and the rest was an unnatural dream!
\par The sun set behind the Grange, as we turned on to the moors; by that, I judged it to be six o'clock; and my companion halted half an hour, to inspect the park, and the gardens, and, probably, the place itself, as well as he could; so it was dark when we dismounted in the paved yard of the farmhouse, and your old fellow-servant, Joseph, issued out to receive us by the light of a dip candle. He did it with a courtesy that redounded to his credit. His first act was to elevate his torch to a level with my face, squint malignantly, project his under-lip, and turn away. Then he took the two horses, and led them into the stables; reappearing for the purpose of locking the outer gate, as if we lived in an ancient castle.
\par Heathcliff stayed to speak to him, and I entered the kitchen — a dingy, untidy hole; I dare say you would not know it, it is so changed since it was in your charge. By the fire stood a ruffianly child, strong in limb and dirty in garb, with a look of Catherine in his eyes and about his mouth.
\par “This is Edgar's legal nephew,” I reflected — “mine in a manner; I must shake hands, and — yes — I must kiss him. It is right to establish a good understanding at the beginning.”
\par I approached, and, attempting to take his chubby fist, said:
\par “How do you do, my dear?”
\par He replied in a jargon I did not comprehend.
\par “Shall you and I be friends, Hareton?” was my next essay at conversation.
\par An oath, and a threat to set Throttler on me if I did not “frame off,” rewarded my perseverance.
\par “Hey, Throttler, lad!” whispered the little wretch, rousing a halfbred bulldog from its lair in a corner. Now, wilt thou be ganging?” he asked authoritatively.
\par Love for my life urged a compliance; I stepped over the threshold to wait till the others should enter. Mr. Heathcliff was nowhere visible; and Joseph, whom I followed to the stables, and requested to accompany me in, after staring and muttering to himself, screwed up his nose, and replied:
\par “Mim! Mim! Mim! Did iver Christian body hear aught like it? Minching un' munching! How can Aw tell whet ye say?”
\par “I say, I wish you to come with me into the house!” I cried, thinking him deaf, yet highly disgusted at his rudeness.
\par “None o'me! I getten summut else to do,” he answered, and continued his work; moving his lantern jaws meanwhile, and surveying my dress and countenance (the former a great deal too fine, but the latter, I'm sure, as sad as he could desire) with sovereign contempt.
\par I walked round the yard, and through a wicket, to another door, at which I took the liberty of knocking, in hopes some more civil servant might show himself. After a short suspense, it was opened by a tall, gaunt man, without neckerchief, and otherwise extremely slovenly; his features were lost in masses of shaggy hair that hung on his shoulders; and his eyes, too, were like a ghostly Catherine's with all their beauty annihilated.
\par “What's your business here?” he demanded grimly. “Who are you?”
\par “My name was Isabella Linton,” I replied. “You've seen me before, sir. I'm lately married to Mr. Heathcliff, and he has brought me here — I suppose by your permission.”
\par “Is he come back, then?” asked the hermit, glaring like a hungry wolf.
\par “Yes — we came just now,” I said; “but he left me by the kitchen door; and when I would have gone in, your little boy played sentinel over the place, and frightened me off by the help of a bulldog.”
\par “It's well the hellish villain has kept his word!” growled my future host, searching the darkness beyond me in expectation of discovering Heathcliff; and then he indulged in a soliloquy of execrations, and threats of what he would have done had the “fiend” deceived him.
\par I repented having tried this second entrance, and was almost inclined to slip away before he finished cursing, but ere I could execute that intention, he ordered me in, and shut and refastened the door.
\par There was a great fire, and that was all the light in the huge apartment, whose floor had grown a uniform grey; and the once brilliant pewter dishes, which used to attract my gaze when I was a girl, partook of a similar obscurity, created by tarnish and dust.
\par I inquired whether I might call the maid, and be conducted to a bedroom? Mr. Earnshaw vouchsafed no answer. He walked up and down, with his hands in his pockets, apparently quite forgetting my presence; and his abstraction was evidently so deep, and his whole aspect so misanthropical, that I shrank from disturbing him again.
\par You'll not be surprised, Ellen, at my feeling particularly cheerless, seated in worse than solitude on that inhospitable hearth, and remembering that four miles distant lay my delightful home, containing the only people I loved on earth; and there might as well be the Atlantic to part us, instead of those four miles: I could not overpass them! I questioned with myself — where must I turn for comfort? And — mind you don't tell Edgar, or Catherine — above every sorrow beside, this rose pre-eminent: despair at finding nobody who could or would be my ally against Heathcliff! I had sought shelter at Wuthering Heights, almost gladly, because I was secured by that arrangement from living alone with him; but he knew the people we were coming amongst, and he did not fear their intermeddling.
\par I sat and thought a doleful time: the clock struck eight, and nine, and still my companion paced to and fro, his head bent on his breast, and perfectly silent, unless a groan or a bitter ejaculation forced itself out at intervals. I listened to detect a woman's voice in the house, and filled the interim with wild regrets and dismal anticipations, which, at last, spoke audibly in irrepressible sighing and weeping. I was not aware how openly I grieved, till Earnshaw halted opposite, in his measured walk, and gave me a stare of newly-awakened surprise. Taking advantage of his recovered attention, I exclaimed:
\par “I'm tired with my journey, and I want to go to bed! Where is the maidservant? Direct me to her, as she won't come to me!”
\par “We have none,” he answered; “you must wait on yourself!”
\par “Where must I sleep, then?” I sobbed: I was beyond regarding selfrespect, weighed down by fatigue and wretchedness.
\par “Joseph will show you Heathcliff's chamber,” said he; “open that door — he's in there.”
\par I was going to obey, but he suddenly arrested me, and added in the strangest tone:
\par “Be so good as to turn your lock, and draw your bolt — don't omit it!”
\par “Well!” I said. “But why, Mr. Earnshaw?” I did not relish the notion of deliberately fastening myself in with Heathcliff.
\par “Look here!” he replied, pulling from his waistcoat a curiously constructed pistol, having a double-edged spring knife attached to the barrel. “That's a great tempter to a desperate man, is it not? I cannot resist going up with this every night, and trying his door. If once I find it open he's done for! I do it invariably, even though the minute before I have been recalling a hundred reasons that should make me refrain: it is some devil that urges me to thwart my own schemes by killing him. You fight against that devil for love as long as you may; when the times comes, not all the angels in heaven shall save him!”
\par I surveyed the weapon inquisitively. A hideous notion struck me: how powerful I should be possessing such an instrument! I took it from his hand, and touched the blade. He looked astonished at the expression my face assumed during a brief second: it was not horror, it was covetousness. He snatched the pistol back, jealously; shut the knife, and returned it to its concealment.
\par “I don't care if you tell him,” said he. “Put him on his guard, and watch for him. You know the terms we are on, I see: his danger does not shock you.”
\par “What has Heathcliff done to you?” I asked. “In what has he wronged you, to warrant this appalling hatred? Wouldn't it be wiser to bid him quit the house?”
\par “No!” thundered Earnshaw, “should he offer to leave me, he's a dead man: persuade him to attempt it, and you are a murderess! Am I to lose all, without a chance of retrieval? Is Hareton to be a beggar? Oh, damnation! I will have it back; and I'll have his gold too; and then his blood; and hell shall have his soul! It will be ten times blacker with that guest than ever it was before!”
\par You've acquainted me, Ellen, with your old master's habits. He is clearly on the verge of madness: he was so last night at least. I shuddered to be near him, and thought on the servant's ill-bred moroseness as comparatively agreeable.
\par He now recommenced his moody walk, and I raised the latch, and escaped into the kitchen.
\par Joseph was bending over the fire, peering into a large pan that swung above it; and a wooden bowl of oatmeal stood on the settle close by. The contents of the pan began to boil, and he turned to plunge his hand into the bowl; I conjectured that this preparation was probably for our supper, and, being hungry, I resolved it should be eatable; so, crying out sharply, “I'll make the porridge!” I removed the vessel out of his reach, and proceeded to take off my hat and riding habit. “Mr. Earnshaw”, I continued, “directs me to wait on myself: I will. I'm not going to act the lady among you, for fear I should starve.”
\par “Gooid Lord!” he muttered, sitting down, and stroking his ribbed stockings from the knee to the ankle. “If they's tuh be fresh ortherings— just when I getten used to two maisters, if I mun hev a mistress set o'er my heead, it's like time to be flitting. I niver did think to say t'day that I mud lave th'owld place — but I doubt it's nigh at hand!”
\par This lamentation drew no notice from me: I went briskly to work, sighing to remember a period when it would have been all merry fun; but compelled speedily to drive off the remembrance. It racked me to recall past happiness, and the greater peril there was of conjuring up its apparition, the quicker the thible ran round, and the faster the handfuls of meal fell into the water.
\par Joseph beheld my style of cookery with growing indignation.
\par “Thear!” he ejaculated, “Hareton, thau willn't sup thy porridge toneeght; they'll be naught but lumps as big as my neive. Thear, agean! I'd fling in bowl un all, if I were ye! There, pale t' guilp off, un' then yeh'll hae done wi't. Bang, bang. It's a marcy t' bothom isn't deaved out!”
\par It was rather a rough mess, I own, when poured into the basins; four had been provided, and a gallon pitcher of new milk was brought from the dairy, which Hareton seized and commenced drinking and spilling from the expansive lip. I expostulated, and desired that he should have his in a mug; affirming that I could not taste the liquid treated so dirtily. The old cynic chose to be vastly offended at this nicety; assuring me, repeatedly, that “the barn was every bit as good” as I, “and every bit as wollsome”, and wondering how I could fashion to be so conceited. Meanwhile, the infant ruffian continued sucking; and glowered at me defyingly, as he slavered into the jug.
\par “I shall have my supper in another room,” I said. “Have you no place you call a parlour?”
\par “Parlour!” he echoed sneeringly, “parlour! Nay, we've noa parlours. If yah dunnut loike wer company, there's maister's; un' if yah dunnut loike maister, there's us.”
\par “Then I shall go upstairs!” I answered; “show me a chamber.”
\par I put my basin on a tray, and went myself to fetch some more milk. With great grumblings, the fellow rose, and preceded me in my ascent: we mounted to the garrets; he opening a door, now and then, to look into the apartments we passed.
\par “Here's a rahm,” he said, at last, flinging back a cranky board on hinges. “It's weel eneugh to ate a few porridge in. There's a pack o'corn i't'corner, thear, meeterly clane; if ye're feared o'muckying yer grand silk cloes, spread yer hankerchir o't'top on't.”
\par The “rahm” was a kind of lumber-hole smelling strong of malt and grain; various sacks of which articles were piled around, leaving a wide, bare space in the middle.
\par “Why, man!” I exclaimed, facing him angrily, “this is not a place to sleep in. I wish to see my bedroom.”
\par “Bed-rume!” he repeated, in a tone of mockery. “Yah's see all t' bed-rumes thear is — yon's mine.”
\par He pointed into the second garret, only differing from the first in being more naked about the walls, and having a large, low, curtainless bed, with an indigo-coloured quilt at one end.
\par “What do I want with yours?” I retorted. “I suppose Mr. Heathcliff does not lodge at the top of the house, does he?”
\par “Oh! It's Maister Hathecliff's ye're wenting!” cried he, as if making a new discovery. “Couldn't ye ha' said soa, at onst? Un then, I mud ha' telled ye, baht all this wark, that that's just one ye cannut sea—he alIas keeps it locked, un nob'dy iver mells on't but hisseln.”
\par “You've a nice house, Joseph,” I could not refrain from observing,“and pleasant inmates; and I think the concentrated essence of all the madness in the world took up its abode in my brain the day I linked my fate with theirs! However, that is not to the present purpose — there are other rooms. For heaven's sake be quick, and let me settle somewhere!”
\par He made no reply to this adjuration; only plodding doggedly down the wooden steps, and halting before an apartment which, from that halt and the superior quality of its furniture, I conjectured to be the best one. There was a carpet: a good one, but the pattern was obliterated by dust; a fireplace hung with cut paper, dropping to pieces; a handsome oak bedstead with ample crimson curtains of rather expensive material and modern make; but they had evidently experienced rough usage: the valances hung in festoons, wrenched from their rings, and the iron rod supporting them was bent in an arc on one side, causing the drapery to trail upon the floor. The chairs were also damaged, many of them severely; and deep indentations deformed the panels of the walls. I was endeavouring to gather resolution for entering and taking possession, when my fool of a guide announced, “This here is t' maister's.” My supper by this time was cold, my appetite gone, and my patience exhausted. I insisted on being provided instantly with a place of refuge, and means of repose.
\par “Whear the divil?” began the religious elder. “The Lord bless us! The Lord forgie us! Whear the hell wold ye gang? Ye marred, wearisome nowt! Ye've seen all but Hareton's bit of a cham'er. There's not another hoile to lig down in i' th' hahse!”
\par I was so vexed, I flung my tray and its contents on the ground; and then seated myself at the stairs-head, hid my face in my hands, and cried.
\par “Ech! ech!” exclaimed Joseph. “Weel done, Miss Cathy! Weel done, Miss Cathy! Hahsiver, t'maister sall just tum'le o'er them brocken pots; un'then we's hear summut; we's hear how it's to be. Gooid-for-naught madling! Ye desarve pining froo this to Churstmas, flinging t'precious gifts O'God under fooit i'yer flaysome rages! But I'm mista'en if ye show yer sperrit lang. Will Hathecliff bide sich bonny ways, think ye? I nobbut wish he may catch ye i'that plisky. I nobbut wish he may.”
\par And so he went on scolding to his den beneath, taking the candle with him; and I remained in the dark. The period of reflection succeeding this silly action compelled me to admit the necessity of smothering my pride and choking my wrath, and bestirring myself to remove its effects. An unexpected aid presently appeared in the shape of Throttler, whom I now recognized as a son of our old Skulker: it had spent its whelphood at the Grange, and was given by my father to Mr.Hindley. I fancy it knew me: it pushed its nose against mine by way of salute, and then hastened to devour the porridge; while I groped from step to step, collecting the shattered earthenware, and drying the spatters of milk from the banister with my pocket handkerchief.
\par Our labours were scarcely over when I heard Earnshaw's tread in the passage; my assistant tucked in his tail, and pressed to the wall; I stole into the nearest doorway. The dog's endeavour to avoid him was unsuccessful; as I guessed by a scutter downstairs, and a prolonged, piteous yelping. I had better luck! He passed on, entered his chamber, and shut the door. Directly after Joseph came up with Hareton, to put him to bed. I had found shelter in Hareton's room, and the old man, on seeing me, said: “They's rahm for boath ye un' yer pride, now, I sud think, i' the hahse. It's empty; ye may hev' it all to yerseln, un' Him as allas maks a third, i' sich ill company!”
\par Gladly did I take advantage of this intimation; and the minute I flung myself into a chair, by the fire, I nodded, and slept.
\par My slumber was deep and sweet, though over far too soon. Mr. Heathcliff awoke me; he had just come in, and demanded, in his loving manner, what I was doing there? I told him the cause of my staying up so late — that he had the key of our room in his pocket. The adjective our gave mortal offence. He swore it was not, nor ever should be, mine; and he'd — but I'll not repeat his language, nor describe his habitual conduct: he is ingenious and unresting in seeking to gain my abhorrence! I sometimes wonder at him with an intensity that deadens my fear: yet, I assure you, a tiger or a venomous serpent could not rouse terror in me equal to that which he wakens. He told me of Catherine's illness, and accused my brother of causing it; promising that I should be Edgar's proxy in suffering, till he could get hold of him.
\par I do hate him — I am wretched — I have been a fool! Beware of uttering one breath of this to anyone at the Grange. I shall expect you every day — don't disappoint me!




\subsection*{Chapter 14}

\par As soon as I had perused this epistle, I went to the master, and informed him that his sister had arrived at the Heights, and sent me a letter expressing her sorrow for Mrs. Linton's situation, and her ardent desire to see him; with a wish that he would transmit to her, as early as possible, some token of forgiveness by me.
\par “Forgiveness!” said Linton. “I have nothing to forgive her, Ellen. You may call at Wuthering Heights this afternoon, if you like, and say that I am not angry, but I'm sorry to have lost her; especially as I can never think she'll be happy. It is out of the question my going to see her, however: we are eternally divided; and should she really wish to oblige me, let her persuade the villain she has married to leave the country.”
\par “And you won't write her a little note, sir?” I asked imploringly.
\par “No,” he answered. “It is needless. My communication with Heathcliff's family shall be as sparing as his with mine. It shall not exist!”
\par Mr. Edgar's coldness depressed me exceedingly; and all the way from the Grange I puzzled my brains how to put more heart into what he said, when I repeated it; and how to soften his refusal of even a few lines to console Isabella.
\par I dare say she had been on the watch for me since morning: I saw her looking through the lattice, as I came up the garden causeway, and I nodded to her; but she drew back, as if afraid of being observed.
\par I entered without knocking. There never was such a dreary, dismal scene as the formerly cheerful house presented! I must confess, that if I had been in the young lady's place, I would, at least, have swept the hearth, and wiped the tables with a duster. But she already partook of the pervading spirit of neglect which encompassed her. Her pretty face was wan and listless; her hair uncurled: some locks hanging lankly down, and some carelessly twisted round her head. Probably she had not touched her dress since yester evening.
\par Hindley was not there. Mr. Heathcliff sat at a table, turning over some papers in his pocket-book; but he rose when I appeared, asked me how I did, quite friendly, and offered me a chair.
\par He was the only thing there that seemed decent: and I thought he never looked better. So much had circumstances altered their positions, that he would certainly have struck a stranger as a born and bred gentleman; and his wife as a thorough little slattern!
\par She came forward eagerly to greet me; and held out one hand to take the expected letter.
\par I shook my head. She wouldn't understand the hint, but followed me to a sideboard, where I went to lay my bonnet, and importuned me in a whisper to give her directly what I had brought. Heathcliff guessed the meaning of her manoeuvres, and said: “If you have got anything for Isabella (as no doubt you have, Nelly), give it to her. You needn't make a secret of it: we have no secrets between us.”
\par “Oh, I have nothing,” I replied, thinking it best to speak the truth at once. “My master bid me tell his sister that she must not expect either a letter or a visit from him at present. He sends his love, ma'am, and his wishes for your happiness, and his pardon for the grief you have occasioned; but he thinks that after this time, his household and the household here should drop intercommunication, as nothing good could come of keeping it up.”
\par Mrs. Heathcliff's lip quivered slightly, and she returned to her seat in the window. Her husband took his stand on the hearthstone, near me, and began to put questions concerning Catherine. I told him as much as I thought proper of her illness, and he extorted from me, by crossexamination, most of the facts connected with its origin. I blamed her, as she deserved, for bringing it all on herself; and ended by hoping that he would follow Mr. Linton's example and avoid future interference with his family, for good or evil.
\par “Mrs. Linton is now just recovering,” I said; “she'll never be like she was, but her life is spared; and if you really have a regard for her, you'll shun crossing her way again. Nay, you'll move out of this country entirely; and that you may not regret it, I'll inform you Catherine Linton is as different now from your old friend Catherine Earnshaw, as that young lady is different from me. Her appearance is changed greatly, her character much more so; and the person who is compelled, of necessity, to be her companion, will only sustain his affection hereafter by the remembrance of what she once was, by common humanity, and a sense of duty!”
\par “That is quite possible,” remarked Heathcliff, forcing himself to seem calm: “quite possible that your master should have nothing but common humanity and a sense of duty to fall back upon. But do you imagine that I shall leave Catherine to his duty and humanity? And can you compare my feelings respecting Catherine to his? Before you leave this house, I must exact a promise from you, that you'll get me an interview with her: consent or refuse, I will see her! What do you say?”
\par “I say, Mr. Heathcliff,” I replied, “you must not: you never shall, through my means. Another encounter between you and the master would kill her altogether.”
\par “With your aid that may be avoided,” he continued, “and should there be danger of such an event — should he be the cause of adding a single trouble more to her existence — why, I think I shall be justified in going to extremes! I wish you had sincerity enough to tell me whether Catherine would suffer greatly from his loss: the fear that she would restrains me. And there you see the distinctions between our feelings: had he been in my place, and I in his, though I hated him with a hatred that turned my life to gall, I never would have raised a hand against him. You may look incredulous, if you please! I never would have banished him from her society as long as she desired his. The moment her regard ceased, I would have torn his heart out, and drunk his blood! But, till then — if you don't believe me, you don't know me — till then, I would have died by inches before I touched a single hair of his head!”
\par “And yet,” I interrupted, “you have no scruples in completely ruining all hopes of her perfect restoration, by thrusting yourself into her remembrance now, when she has nearly forgotten you, and involving her in a new tumult of discord and distress.”
\par “You suppose she has nearly forgotten me?” he said. “Oh, Nelly! You know she has not! You know as well as I do, that for every thought she spends on Linton, she spends a thousand on me! At a most miserable period of my life, I had a notion of the kind; it haunted me on my return to the neighbourhood last summer; but only her own assurance could make me admit the horrible idea again. And then, Linton would be nothing, nor Hindley, nor all the dreams that ever I dreamt. Two words would comprehend my future — death and hell: existence, after losing her, would be hell. Yet I was a fool to fancy for a moment that she valued Edgar Linton's attachment more than mine. If he loved with all the powers of his puny being, he couldn't love as much in eighty years as I could in a day. And Catherine has a heart as deep as I have: the sea could be as readily contained in that horse-trough, as her whole affection be monopolized by him! Tush! He is scarcely a degree dearer to her than her dog, or her horse. It is not in him to be loved like me: how can she love in him what he has not?”
\par “Catherine and Edgar are as fond of each other as any two people can be,” cried Isabella, with sudden vivacity. “No one has a right to talk in that manner, and I won't hear my brother depreciated in silence!”
\par “Your brother is wondrous fond of you too, isn't he?” observed Heathcliff scornfully. “He turns you adrift on the world with surprising alacrity.”
\par “He is not aware of what I suffer,” she replied. “I didn't tell him that.”
\par “You have been telling him something, then: you have written, have you?”
\par “To say that I was married, I did write — you saw the note.”
\par “And nothing since?”
\par “No.”
\par “My young lady is looking sadly the worse for her change of condition,” I remarked. “Somebody's love comes short in her case, obviously: whose, I may guess; but, perhaps, I shouldn't say.”
\par “I should guess it was her own,” said Heathcliff. “She degenerates into a mere slut! She is tired of trying to please me uncommonly early. You'd hardly credit it, but the very morrow of our wedding, she was weeping to go home. However, she'll suit this house so much the better for not being over nice, and I'll take care she does not disgrace me by rambling abroad.”
\par “Well, sir,” returned I, “I hope you'll consider that Mrs. Heathcliff is accustomed to be looked after and waited on; and that she has been brought up like an only daughter, whom everyone was ready to serve. You must let her have a maid to keep things tidy about her, and you must treat her kindly. Whatever be your notion of Mr. Edgar, you cannot doubt that she has a capacity for strong attachments, or she wouldn't have abandoned the elegances, and comforts, and friends of her former home, to fix contentedly, in such a wilderness as this, with you.”
\par “She abandoned them under a delusion,” he answered; “picturing in me a hero of romance, and expecting unlimited indulgences from my chivalrous devotion. I can hardly regard her in the light of a rational creature, so obstinately has she persisted in forming a fabulous notion of my character and acting on the false impressions she cherished. But, at last, I think she begins to know me: I don't perceive the silly smiles and grimaces that provoked me at first; and the senseless incapability of discerning that I was in earnest when I gave her my opinion of her infatuation and herself. It was a marvellous effort of perspicacity to discover that I did not love her. I believed, at one time, no lessons could teach her that! And yet it is poorly learnt; for this morning she announced, as a piece of appalling intelligence, that I had actually succeeded in making her hate me! A positive labour of Hercules, I assure you! If it be achieved, I have cause to return thanks. Can I trust your assertion, Isabella? Are you sure you hate me? If I let you alone for half a day, won't you come sighing and wheedling to me again? I dare say she would rather I had seemed all tenderness before you: it wounds her vanity to have the truth exposed. But I don't care who knows that the passion was wholly on one side; and I never told her a lie about it. She cannot accuse me of showing a bit of deceitful softness. The first thing she saw me do, on coming out of the Grange, was to hang up her little dog; and when she pleaded for it, the first words I uttered were a wish that I had the hanging of every being belonging to her, except one: possibly she took that exception for herself. But no brutality disgusted her: I suppose she has an innate admiration of it, if only her precious person were secure from injury! Now, was it not the depth of absurdity—of genuine idiocy, for that pitiful, slavish, mean-minded brach to dream that I could love her? Tell your master, Nelly, that I never, in all my life, met with such an abject thing as she is. She even disgraces the name of Linton; and I've sometimes relented, from pure lack of invention, in my experiments on what she could endure, and still creep shamefully cringing back! But tell him, also, to set his fraternal and magisterial heart at ease: that I keep strictly within the limits of the law. I have avoided, up to this period, giving her the slightest right to claim a separation; and, what's more, she'd thank nobody for dividing us. If she desired to go, she might: the nuisance of her presence outweighs the gratification to be derived from tormenting her!”
\par “Mr. Heathcliff,” said I, “this is the talk of a madman, and your wife, most likely, is convinced you are mad; and, for that reason, she has borne with you hitherto: but now that you say she may go, she'll doubtless avail herself of the permission. You are not so bewitched, ma'am, are you, as to remain with him of your own accord?”
\par “Take care, Ellen!” answered Isabella, her eyes sparkling irefully; there was no misdoubting by their expression the full success of her partner's endeavours to make himself detested. “Don't put faith in a single word he speaks. He's a lying fiend! a monster, and not a human being! I've been told I might leave him before; and I've made the attempt, but I dare not repeat it! Only, Ellen, promise you'll not mention a syllable of his infamous conversation to my brother or Catherine. Whatever he may pretend, he wishes to provoke Edgar to desperation: he says he has married me on purpose to obtain power over him; and he shan't obtain it — I'll die first! I just hope, I pray, that he may forget his diabolical prudence and kill me! The single pleasure I can imagine is to die or see him dead!”
\par “There — that will do for the present!” said Heathcliff. “If you are called upon in a court of law, you'll remember her language, Nelly! And take a good look at that countenance: she's near the point which would suit me. No; you're not fit to be your own guardian, Isabella, now; and I, being your legal protector, must detain you in my custody, however distasteful the obligation may be. Go upstairs; I have something to say to Ellen Dean in private. That's not the way: upstairs, I tell you! Why, this is the road upstairs, child!”
\par He seized, and thrust her from the room: and returned muttering:
\par “I have no pity! I have no pity! The more the worms writhe, the more I yearn to crush out their entrails! It is a moral teething; and I grind with greater energy, in proportion to the increase of pain.”
\par “Do you understand what the word pity means?” I said, hastening to resume my bonnet. “Did you ever feel a touch of it in your life?”
\par “Put that down!” he interrupted, perceiving my intention to depart.“You are not going yet. Come here now, Nelly: I must either persuade or compel you to aid me in fulfilling my determination to see Catherine, and that without delay. I swear that I meditate no harm: I don't desire to cause any disturbance, or to exasperate or insult Mr. Linton; I only wish to hear from herself how she is, and why she has been ill; and to ask if anything that I could do would be of use to her. Last night, I was in the Grange garden six hours, and I'll return there tonight; and every night I'll haunt the place, and every day, till I find an opportunity of entering. If Edgar Linton meets me, I shall not hesitate to knock him down, and give him enough to insure his quiescence while I stay. If his servants oppose me, I shall threaten them off with these pistols. But wouldn't it be better to prevent my coming in contact with them, or their master? And you could do it so easily. I'd warn you when I came, and then you might let me in unobserved, as soon as she was alone, and watch till I departed, your conscience quite calm: you would be hindering mischief.”
\par I protested against playing that treacherous part in my employer's house: and, besides, I urged the cruelty and selfishness of his destroying Mrs. Linton's tranquillity for his satisfaction.
\par “The commonest occurrence startles her painfully,” I said. “She's all nerves, and she couldn't bear the surprise, I'm positive. Don't persist, sir! Or else, I shall be obliged to inform my master of your designs; and he'll take measures to secure his house and its inmates from any such unwarrantable intrusions!”
\par “In that case, I'll take measures to secure you, woman!” exclaimed Heathcliff; “you shall not leave Wuthering Heights till tomorrow morning. It is a foolish story to assert that Catherine could not bear to see me; and as to surprising her, I don't desire it: you must prepare her ask her if I may come. You say she never mentions my name, and that I am never mentioned to her. To whom should she mention me if I am a forbidden topic in the house? She thinks you are all spies for her husband. Oh, I've no doubt she's in hell among you! I guess by her silence, as much as anything, what she feels. You say she is often restless, and anxious-looking — is that a proof of tranquillity? You talk of her mind being unsettled. How the devil could it be otherwise in her frightful isolation? And that insipid, paltry creature attending her from duty and humanity! From pity and charity! He might as well plant an oak in a flowerpot, and expect it to thrive, as imagine he can restore her to vigour in the soil of his shallow cares! Let us settle it at once: will you stay here, and am I to fight my way to Catherine over Linton and his footmen? Or will you be my friend as you have been hitherto, and do what I request? Decide! Because there is no reason for my lingering another minute, if you persist in your stubborn ill-nature!”
\par Well, Mr. Lockwood, I argued and complained, and flatly refused him fifty times; but in the long run he forced me to an agreement. I engaged to carry a letter from him to my mistress; and should she consent, I promised to let him have intelligence of Linton's next absence from home, when he might come, and get in as he was able: I wouldn't be there, and my fellow-servants should be equally out of the way.
\par Was it right or wrong? I fear it was wrong, though expedient. I thought I prevented another explosion by my compliance; and I thought, too, it might create a favourable crisis in Catherine's mental illness: and then I remembered Mr. Edgar's stern rebuke of my carrying tales; and I tried to smooth away all disquietude on the subject, by affirming, with frequent iteration, that that betrayal of trust, if it merited so harsh an appellation, should be the last.
\par Notwithstanding, my journey homeward was sadder than my journey thither; and many misgivings I had, ere I could prevail on myself to put the missive into Mrs. Linton's hand.
\par But here is Kenneth; I'll go down, and tell him how much better you are. My history is dree, as we say, and will serve to while away another morning.
\par Dree, and dreary! I reflected as the good woman descended to receive the doctor; and not exactly of the kind which I should have chosen to amuse me. But never mind! I'll extract wholesome medicines from Mrs. Dean's bitter herbs; and firstly, let me beware the fascination that lurks in Catherine Heathcliff's brilliant eyes. I should be in a curious taking if I surrendered my heart to that young person, and the daughter turned out a second edition of the mother!






\subsection*{Chapter 15}







\subsection*{Chapter 16}







\subsection*{Chapter 17}







\subsection*{Chapter 18}







\subsection*{Chapter 19}







\subsection*{Chapter 20}







\subsection*{Chapter 21}







\subsection*{Chapter 22}







\subsection*{Chapter 23}







\subsection*{Chapter 24}







\subsection*{Chapter 25}







\subsection*{Chapter 26}







\subsection*{Chapter 27}







\subsection*{Chapter 28}







\subsection*{Chapter 29}







\subsection*{Chapter 30}







\subsection*{Chapter 31}







\subsection*{Chapter 32}







\subsection*{Chapter 33}







\subsection*{Chapter 34}
























