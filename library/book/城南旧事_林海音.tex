

\section{城南旧事}


\par 书名:城南旧事
\par 作者:林海音
\par 出版社:中国青年出版社
\par 出版时间:2012-12
\par ISBN:9787515312354








\subsection{冬阳·童年·骆驼队}

\par 骆驼队来了,停在我家的门前。
\par 它们排列成一长串,沉默地站着,等候人们的安排。天气又干又冷,拉骆驼的摘下了他的毡帽,秃瓢儿上冒着热气,是一股白色的烟,融入干冷的大气中。
\par 爸爸在和他讲价钱。双峰的驼背上,每匹都驮着两麻袋煤。我在想,麻袋里面是“南山高末”呢?还是“乌金墨玉”?我常常看见顺城街煤栈的白墙上,写着这样几个大黑字。但是拉骆驼的说,他们从门头沟来,他们和骆驼,是一步一步走来的。
\par 另外一个拉骆驼的,在招呼骆驼们吃草料。它们把前脚一屈,屁股一撅,就跪了下来。
\par 爸爸已经和他们讲好价钱了。人在卸煤,骆驼在吃草。
\par 我站在骆驼的面前,看它们吃草料咀嚼的样子,那样丑的脸,那样长的牙,那样安静的态度。它们咀嚼的时候,上牙和下牙交错地磨来磨去,大鼻孔里冒着热气,白沫子沾满在胡须上。我看得呆了,自己的牙齿也动了起来。地磨来磨去,大鼻孔里冒着热气,白沫子沾满在胡须上。我看得呆了,自己的牙齿也动了起来。
\par 老师教给我,要学骆驼,沉得住气的动物。看它从不着急,慢慢地走,慢慢地嚼,总会走到的,总会吃饱的。也许它天生是该慢慢的,偶然躲避车子跑两步,姿势就很难看。
\par 骆驼队伍过来时,你会知道,打头儿的那一匹,长脖子底下总系着一个铃铛,走起来,“当、当、当”地响。
\par “为什么要系一个铃铛?”我不懂的事就要问一问。
\par 爸爸告诉我,骆驼很怕狼,因为狼会咬它们,所以人类给它戴上铃铛,狼听见铃铛的声音,知道那是有人类在保护着,就不敢侵犯了。
\par 我的幼稚心灵中却充满了和大人不同的想法,我对爸爸说:
\par “不是的,爸!它们软软的脚掌走在软软的沙漠上,没有一点点声音,你不是说,它们走上三天三夜都不喝一口水,只是不声不响地咀嚼着从胃里反刍出来的食物吗?一定是拉骆驼的人类,耐不住那长途寂寞的旅程,所以才给骆驼戴上了铃铛,增加一些行路的情趣。”
\par 爸爸想了想,笑笑说:
\par “也许,你的想法更美些。”
\par 冬天快过完了,春天就要来,太阳特别的暖和,暖得让人想把棉袄脱下来。可不是么?骆驼也脱掉它的绒袍子啦!它的毛皮一大块一大块地从身上掉下来,垂在肚皮底下。我真想拿剪刀替它们剪一剪,因为太不整齐了。拉骆驼的人也一样,他们身上那件反穿大羊皮,也都脱下来了,搭在骆驼背的小峰上。麻袋空了,“乌金墨玉”都卖了,铃铛在轻松的步伐里响得更清脆。
\par 夏天来了,再不见骆驼的影子,我又问妈:
\par “夏天它们到哪儿去?”
\par “谁?”
\par “骆驼呀!”
\par 妈妈回答不上来了,她说:
\par “总是问,总是问,你这孩子!”
\par 夏天过去,秋天过去,冬天又来了,骆驼队又来了,但是童年却一去不还。冬阳底下学骆驼咀嚼的傻事,我也不会再做了。
\par 可是,我是多么想念童年住在北京城南的那些景色和人物啊!我对自己说,把它们写下来吧,让实际的童年过去,心灵的童年永存下来。
\par 就这样,我写了一本《城南旧事》。
\par 我默默地想,慢慢地写。看见冬阳下的骆驼队走过来,听见缓慢悦耳的铃声,童年重临于我的心头。
\par \rightline{一九六〇年十月}



\subsection{惠安馆}







\subsubsection*{一}

\par 太阳从大玻璃窗透进来,照到大白纸糊的墙上,照到三屉桌上,照到我的小床上来了。我醒了,还躺在床上,看那道太阳光里飞舞着的许多小小的、小小的尘埃。宋妈过来掸窗台,掸桌子,随着鸡毛掸子的舞动,那道阳光里的尘埃加多了,飞舞得更热闹了,我赶忙拉起被来蒙住脸,是怕尘埃把我呛得咳嗽。
\par 宋妈的鸡毛掸子轮到来掸我的小床了,小床上的棱棱角角她都掸到了,掸子把儿碰在床栏上,格格地响,我想骂她,但她倒先说话了:
\par “还没睡够哪!”说着,她把我的被大掀开来,我穿着绒褂裤的身体整个露在被外,立刻就打了两个喷嚏。她强迫我起来,给我穿衣服。印花斜纹布的棉袄棉裤,都是新做的,棉裤筒多可笑,可以直立放在那里,就知道那棉花够多厚了。
\par 妈正坐在炉子边梳头,倾着身子,一大把头发从后脖子顺过来,她就用篦子篦呀篦呀的,炉子上是一瓶玫瑰色的发油,天气冷,油凝住了,总要放在炉子上化一化才能擦。
\par 窗外很明亮,干秃的树枝上落着几只不怕冷的小鸟,我在想,什么时候那树上才能长满叶子呢?这是我们在北京过的第一个冬天。
\par 妈妈还说不好北京话,她正在告诉宋妈,今天买什么菜。妈不会说“买一斤猪肉,不要太肥”。她说:“买一斤租漏,不要太回。”
\par 妈妈梳完了头,用她的油手抹在我的头发上,也给我梳了两条辫子。我看宋妈提着篮子要出去了,连忙喊住她:
\par “宋妈,我跟你去买菜。”
\par 宋妈说:“你不怕惠难馆的疯子?”
\par 宋妈是顺义县的人,她也说不好北京话,她说成“惠难馆”,妈说成“灰娃馆”,爸说成“飞安馆”,我随着胡同里的孩子说“惠安馆”,到底哪一个对,我不知道。
\par 我为什么要怕惠安馆的疯子?她昨天还冲我笑呢!她那一笑真有意思,要不是妈紧紧拉着我的手,我就会走过去看她,跟她说话了。
\par 惠安馆在我们这条胡同的最前一家,三层石台阶上去,就是两扇大黑门凹进去,门上横着一块匾,路过的时候爸爸教我念过:“飞安会馆”。爸说里面住的都是从“飞安”那个地方来的学生,像叔叔一样,在大学里念书。
\par “也在北京大学?”我问爸爸。
\par “北京的大学多着呢,还有清华大学呀!燕京大学呀!”
\par “可以不可以到飞安——不,惠安馆里找叔叔们玩一玩?”
\par “做唔得!做唔得!”我知道,我无论要求什么事,爸终归要拿这句客家话来拒绝我。我想总有一天我要迈上那三层台阶,走进那黑洞洞的大门里去的。
\par 惠安馆的疯子我看见好几次了,每一次只要她站在门口,宋妈或者妈就赶快捏紧我的手,轻轻说:“疯子!”我们便擦着墙边走过去,我如果要回头再张望一下时,她们就用力拉我的胳臂制止我。其实那疯子还不就是一个梳着油松大辫子的大姑娘,像张家李家的大姑娘一样!她总是倚着门墙站着,看来来往往过路的人。
\par 是昨天,我跟着妈妈到骡马市的佛照楼去买东西,妈是去买擦脸的鸭蛋粉,我呢,就是爱吃那里的八珍梅。我们从骡马市大街回来,穿过魏染胡同、西草厂,到了椿树胡同的井窝子,井窝子斜对面就是我们住的这条胡同。刚一进胡同,我就看见惠安馆的疯子了,她穿了一件绛紫色的棉袄,黑绒的毛窝,头上留着一排刘海儿,辫子上扎的是大红绒绳,她正把大辫子甩到前面来,两手玩弄着辫梢,愣愣地看着对面人家院子里的那棵老洋槐。干树枝子上有几只乌鸦,胡同里没什么人。
\par 妈正低头嘴里念叨着,准是在算她今天共买了多少钱的东西,好跟无事不操心的爸爸报账,所以妈没留神已经走到了“灰娃馆”。我跟在妈的后面,一直看疯子,竟忘了走路。这时疯子的眼光从洋槐上落下来,正好看到我,她眼珠不动地盯着我,好像要在我的脸上找什么。她的脸白得发青,鼻子尖有点红,大概是冷风吹冻的,尖尖的下巴,两片薄嘴唇紧紧地闭着。忽然她的嘴唇动了,眼睛也眨了两下,带着笑,好像要说话,弄着辫梢的手也向我伸出来,招我过去呢。不知怎么,我浑身大大地打了一个寒战,跟着,我就随着她的招手和笑意要向她走去。——可是妈回过头来了,突然把我一拉:
\par “怎么啦,你?”
\par “嗯?”我有点迷糊。妈看了疯子一眼,说:
\par “为什么打哆嗦?是不是怕——是不是要溺尿?快回家!”我的手被妈使劲拖拉着。
\par 回到家来,我心里还惦念着疯子的那副模样儿。她的笑不是很有意思吗?如果我跟她说话——我说:“嘿!”她会怎么样呢?我愣愣地想着,懒得吃晚饭,实在也是八珍梅吃多了。但是晚饭后,妈对宋妈说:
\par “英子一定吓着了。”然后给我沏了碗白糖水,叫我喝下去,并且命令我钻被窝睡觉……
\par 这时,我的辫子梳好了,追了宋妈去买菜,她在前面走,我在后面跟着。她的那条恶心的大黑棉裤,那么厚,那么肥,裤脚缚着。别人告诉妈说,北京的老妈子很会偷东西,她们偷了米就一把一把顺着裤腰装进裤兜子,刚好落到缚着的裤脚管里,不会漏出来。我在想,宋妈的肥裤脚里,不知道有没有我家的白米?
\par 经过惠安馆,我向里面看了一下,黑门大开着,门道里有一个煤球炉子,那疯子的妈妈和爸爸正在炉边煮什么。大家都管疯子的爸爸叫“长班老王”,长班就是给会馆看门的,他们住在最临街的一间屋子。宋妈虽然不许我看疯子,但是我知道她自己也很爱看疯子,打听疯子的事,只是不许我听我看就是了。宋妈这时也向惠安馆里看,正好疯子的妈妈抬起头来,她和宋妈两人同时说“吃了吗?您!”爸爸说北京人一天到晚闲着没有事,不管什么时候见面都要问吃了没有。
\par 出了胡同口往南走几步,就是井窝子,这里满地是水,有的地方结成薄薄的冰,独轮的水车来一辆去一辆,他们扭着屁股推车,车子吱吱扭扭地响,好刺耳,我要堵起耳朵啦!井窝子有两个人在向深井里打水,水打上来倒在一个好大的水槽里,推水的人就在大水槽里接了水再送到各家去。井窝子旁住着一个我的朋友——和我一般高的妞儿。我这时停在井窝子旁边不走了,对宋妈说:
\par “宋妈,你去买菜,我等妞儿。”
\par 妞儿,我第一次是在油盐店里看见她的。那天她两只手端了两个碗,拿了一大枚,又买酱,又买醋,又买葱,伙计还逗着说:“妞儿,唱一段才许你走!”妞儿眼里含着泪,手摇晃着,醋都要洒了,我有说不出的气恼,一下蹿到妞儿身旁,叉着腰问他们:
\par “凭什么?”
\par 就这样,我认识了妞儿。
\par 妞儿只有一条辫子,又黄又短,像妈在土地庙给我买的小狗的尾巴。第二次看见妞儿,是我在井窝子旁边看打水。她过来了,一声不响地站在我身边,我们俩相对笑了笑,不知道说什么好。等一会儿,我就忍不住去摸她那条小黄辫子了,她又向我笑了笑,指着后面,低低的声音说:
\par “你就住在那条胡同里?”
\par “嗯。”我说。
\par “第几个门?”
\par 我伸出手指头来算了算:
\par “一,二,三,四,第四个门。到我们家去玩。”
\par 她摇摇头说:“你们胡同里有疯子,妈不叫我去。”
\par “怕什么,她又不吃人。”
\par 她仍然是笑笑的摇摇头。
\par 妞儿一笑,眼底下鼻子两边的肉就会有两个小漩涡,很好看,可是宋妈竟跟油盐店的掌柜说:
\par “这孩子长得俊倒是俊,就是有点薄,眼睛太透亮了,老像水汪着,你看,眼底下有两个泪坑儿。”
\par 我心里可是有说不出的喜欢她,喜欢她那么温和,不像我一急宋妈就骂我的:“又跳?又跳?小暴雷。”那天她跟我在井窝子边站一会儿,就小声地说:“我要回去了,我爹等着我吊嗓子。赶明儿见!”
\par 我在井窝子旁跟妞儿见过几次面了,只要看见红棉袄裤从那边闪过来,我就满心的高兴,可是今天,等了好久都不见她出来,很失望,我的绒褂子口袋里还藏着一小包八珍梅,要给妞儿吃的。我摸摸,发热了,包的纸都破烂了,黏糊糊的,宋妈洗衣服时,我还得挨她一顿骂。
\par 我觉得很没意思,往回家走,我本来想今天见妞儿的话,就告诉她一个好主意,从横胡同穿过到我家,就用不着经过惠安馆,不用怕看见疯子了。
\par 我低头这么想着,走到惠安馆门口了。
\par “嘿!”
\par 吓了我一跳!正是疯子。咬着下嘴唇,笑着看我。她的眼睛真透亮,一笑,眼底下——就像宋妈说的,怎么也有两个泪坑儿呀!我想看清楚她,我是多么久以前就想看清楚她的。我不由得对着她的眼神走上了台阶。太阳照在她的脸上,常常是苍白的颜色,今天透着亮光了。她揣在短棉袄里的手伸出来拉住我的手,那么暖,那么软。我这时看看胡同里,没有一个人走过。真奇怪,我现在怕的不是疯子,倒是怕人家看见我跟疯子拉手了。
\par “几岁了?”她问我。
\par “嗯——六岁。”
\par “六岁!”她很惊奇地叫了一声,低下头来,忽然撩起我的辫子看我的脖子,在找什么。“不是。”她喃喃地自己说话,接着又问我:
\par “看见我们小桂子没有?”
\par “小桂子?”我不懂她在说什么。
\par 这时大门里疯子的妈妈出来了,皱着眉头怪着急地说:“秀贞,可别把人家小姑娘吓着呀!”又转过脸来对我说:
\par “别听她的,胡说呢!回去吧!等回头你妈不放心,嗯——听见没有?”她说着,用手扬了扬,叫我回去。
\par 我抬头看着疯子,知道她的名字叫秀贞了。她拉着我的手,轻摇着,并不放开我。她的笑,增加了我的勇气,我对老的说:
\par “不!”
\par “小南蛮子儿!”秀贞的妈妈也笑了,轻轻地指点着我的脑门儿,这准是一句骂我的话,就像爸爸常用看不起的口气对妈说“他们这些北仔鬼”是一样的吧!
\par “在这儿玩不要紧,你家来了人找,可别赖是我们姑娘招的你。”
\par “我不说的啦!”何必这么嘱咐我?什么该说,什么不该说,我都知道。妈妈打了一只金镯子,藏在她的小首饰箱里,我从来不会告诉爸爸。
\par “来!”秀贞拉着我往里走,我以为要到里面那一层一层很深的院子里去找上大学的叔叔们玩呢,原来她把我带进了他们住的门房。
\par 屋里可不像我家里那么亮,玻璃窗小得很,临窗一个大炕,炕中间摆了一张矮桌,上面堆着活计和针线盒子。秀贞从矮桌上拿起了一件没做完的衣服,朝我身上左比右比,然后高兴地对走进来的她的妈妈说:
\par “妈,您瞧,我怎么说的,刚合适!那么就开领子吧。”说着,她又找了一根绳子绕着我的脖子量,我由她摆布,只管看墙上的那张画,那画的是一个白胖大娃娃,没有穿衣服,手里捧着大元宝,骑在一条大大的红鱼上。
\par 秀贞转到我的面前来,看我仰着头,她也随着我的眼光看那张画,满是那么回事地说:
\par “要看炕上看去,看我们小桂子多胖,那阵儿才八个月,骑着大金鱼,满屋里转,玩得饭都不吃,就这么淘……”
\par “行啦行啦!不——害——臊!”秀贞正说得高兴,我也听得糊里糊涂,长班老王进来了,不耐烦地瞪了秀贞一眼说她。秀贞不理会她爸爸,推着我脱鞋上炕,凑近在画下面,还是只管说:
\par “饭不吃,衣服也不穿,就往外跑,老是急着找她爹去,我说了多少回都不听,我说等我给多做几件衣服穿上再去呀!今年的衬褂倒是先做好了,背心就差缝钮子了。这件棉袄开了领子马上就好。可急的是什么呀!真叫人纳闷儿,到底是怎么档子事儿……”她说着说着不说了,低着头在想那纳闷儿的事,一直发愣。我想,她是在和我玩“过家家儿”吧?她妈不是说她胡说吗?要是过家家儿,我倒是有一套玩意儿,小手表,小算盘,小铃铛,都可以拿来一起玩。所以我就说:
\par “没关系,我把手表送给小桂子,她有了表就有一定时候回家了。”可是——这时我倒想起妈会派宋妈来找我,便又说:“我也要回家了。”
\par 秀贞听我说要走,她也不发愣了,一面随着我下了炕,一面说:“那敢情好,先谢谢你啦!看见小桂子叫她回来,外面冷,就说我不骂她,不用怕。”
\par 我点了点头,答应她,真像有那么一个小桂子,我认识的。
\par 我一边走着一边想,跟秀贞这样玩,真有意思;假装有一个小桂子,还给小桂子做衣服。为什么人家都不许他们的小孩子跟秀贞玩呢?还管她叫疯子?我想着就回头去看,原来秀贞还倚着墙看我呢!我一高兴就连跑带跳地回家来。
\par 宋妈正在跟一个老婆子换洋火,房檐底下堆着字纸篓、旧皮鞋、空瓶子。
\par 我进了屋子就到小床前的柜里找出手表来。小小圆圆的金表,镶着几粒亮亮的钻石,上面的针已经不能走动了,妈妈说要修理,可一直放着,我很喜欢这手表,常常戴在手上玩,就归了我了。我正站在三屉桌前玩弄着,忽然听见窗外宋妈正和老婆子在说什么,我仔细听,宋妈说:
\par “后来呢?”
\par “后来呀,”换洋火的老婆子说,“那学生一去到如今就没回来!临走的时候许下的,回他老家卖田卖地,过一个月就回来明媒正娶她。好嘛!这一等就是六年啦!多傻的姑娘,我眼瞧着她疯的……”
\par “说是怎么着?还生了个孩子?”
\par “是呀!那学生走的时候,姑娘她妈还不知道姑娘有了,等到现形了,这才赶着送回海甸义地去生的。”
\par “义地?”
\par “就是他们惠安义地,惠安人在北京死了就埋在他们惠安义地里。原来王家是给义地看坟的,打姑娘的爷爷就看起,后来又让姑娘她爹来这儿当长班,谁知道出了这么档子事儿。”
\par “他们这家子倒是跟惠难有缘,惠难离咱们这儿多远哪?怎么就一去不回头了呢?”
\par “可远喽!”
\par “那么生下来的孩子呢?”
\par “孩子呀,一落地就裹包裹包,趁着天没亮,送到齐化门城根底下啦!反正不是让野狗吃了,就是让人捡去了呗!”
\par “姑娘打这儿就疯啦?”
\par “可不,打这儿就疯了!可怜她爹妈,这辈子就生下这么个姑娘,唉!”
\par 两个人说到这儿都不言语了,我这时已经站到屋门口倾听。宋妈正数着几包红头洋火,老婆子把破烂纸往她的大筐里塞呀塞呀!鼻子里吸溜着清鼻涕。宋妈又说:
\par “下回给带点刨花来。那——你跟疯子她们是一地儿的人呀?”
\par “老亲喽!我大妈娘家二舅屋里的三姐算是疯子她二妈,现在还在看坟,他们说的还有错儿吗?”
\par 宋妈一眼看见了我,说:
\par “又听事儿,你。”
\par “我知道你们说谁。”我说。
\par “说谁?”
\par “小桂子她妈。”
\par “小桂子她妈?”宋妈哈哈大笑,“你也疯啦?哪儿来的小桂子她妈呀?”
\par 我也哈哈笑了,我知道谁是小桂子她妈呀!

\subsubsection*{二}

\par 天气暖和多了,棉袄早就脱下来,夹袄外面早晚凉就罩上一件薄薄的棉背心,又轻又软。我穿的新布鞋,前头打了一块黑皮子头,老王妈——秀贞她妈,看见我的新鞋说:
\par “这双鞋可结实呦——把我们家的门槛儿踢烂了,你这双鞋也破不了!”
\par 惠安馆我已经来熟了,会馆的大门总是开着一扇,所以我随时可以溜进来。我说溜进来,因为我总是背着家里的人偷着来的,他们只知道我常常是随着宋妈买菜到井窝子找妞儿,一见宋妈进了油盐店,我就回头走,到惠安馆来。
\par 我今天进了惠安馆,秀贞不在屋里。炕桌上摆着一个大玻璃缸,里面是几条小金鱼,游来游去。我问王妈:
\par “秀贞呢?”
\par “跨院里呢!”
\par “我去找她。”我说。
\par “别价,她就来,你这儿等着,看金鱼吧!”
\par 我把鼻子顶着金鱼缸向里看,金鱼一边游一边嘴巴一张一张地在喝水,我的嘴也不由得一张一张地在学鱼喝水。有时候金鱼游到我的面前来,隔着一层玻璃,我和鱼鼻子顶牛儿啦!我就这么看着,两腿跪在炕沿上,都麻了,秀贞还不来。
\par 我翻腿坐在炕沿上,又等了一会儿,还不见秀贞来,我急了,溜出了屋子,往跨院里去找她。那跨院,仿佛一直都是关着的,我从来也没见过谁去那里。我轻轻推开跨院门进去,小小的院子里有一棵不知什么树,已经长了小小的绿叶子了。院角地上是干枯的落叶,有的烂了。秀贞大概正在打扫,但是我进去时看见她一手拿着扫帚倚在树干上,一手掀起了衣襟在擦眼睛,我悄悄走到她跟前,抬头看着她。她也许看见我了,但是没理会我,忽然背转身子去,伏着树干哭起来了,她说:
\par “小桂子,小桂子,你怎么不要妈了呢?”
\par 那声音多么委屈,多么可怜啊!她又哭着说:
\par “我不带你,你怎么认得道儿,远着呢!”
\par 我想起妈妈说过,我们是从很远很远的家乡来的,那里是个岛,四面都是水,我们坐了大轮船,又坐大火车,才到这个北京来。我曾问妈妈什么时候回去,妈说早着呢,来一趟不容易,多住几年。那么秀贞所说的那个远地方,是像我们的岛那么远吗?小桂子怎么能一个人跑了去?我替秀贞难过,也想念我并不认识的小桂子,我的眼泪掉下来了。在模模糊糊的泪光里,我仿佛看见那骑着大金鱼的胖娃娃,是什么也没穿啊!
\par 我含着眼泪,大大地倒抽了一口气,为的是不让我自己哭出来,我揪揪秀贞裤腿叫她:
\par “秀贞!秀贞!”
\par 她停止了哭声,满脸泪蹲下来,搂着我,把头埋在我的前胸擦来擦去,用我的夹袄和软软的背心,擦干了她的泪,然后她仰起头来看看我笑了,我伸出手去调顺她揉乱的刘海儿,不由得说:
\par “我喜欢你,秀贞。”
\par 秀贞没有说什么,吸溜着鼻涕站起来。天气暖和了,她也不穿缚腿棉裤了,现在穿的是一条肥肥的散腿裤。她的腿很瘦吗?怎么风一吹那裤子,显得那么晃荡。她浑身都瘦,刚才蹲下来伏在我的胸前时,我看那块后脊背,平板儿似的。
\par 秀贞拉着我的手说:
\par “屋里去,帮着拾掇拾掇。”
\par 小跨院里只有这么两间小房,门一推吱吱扭扭的一串尖响,那声音不好听,好像有一根刺扎在人心上。从太阳地里走进这阴暗的屋里来,怪凉的。外屋里,整整齐齐地摆着书桌,椅子,书架,上面满是灰土,我心想,应该叫我们宋妈来给掸掸,准保扬起满屋子的灰。爸爸常常对妈说,为什么宋妈不用湿布擦,这样大掸一阵,等一会儿,灰尘不是又落回原来的地方了吗?但是妈妈总请爸爸不要多嘴,她说这是北京规矩。
\par 走进里屋去,房间更小一点,只摆了一张床、一个茶几。床上有一口皮箱,秀贞把箱子打开来,从里面拿出一件大棉袍,我爸爸也有,是男人的。秀贞把大棉袍抱在胸前,自言自语地说:
\par “该翻翻添点棉花了。”
\par 她把大棉袍抱出院子去晒,我也跟了去。她进来,我也跟进来。她叫我和她把箱子抬到院子太阳底下晒,里面只有一双手套、一顶呢帽和几件旧内衣。她很仔细地把这几件零碎衣物摊开来,并且拿起一件条子花纹的褂子对我说:
\par “我瞧这件褂子只能给小桂子做夹袄里子了。”
\par “可不是,”我翻开了我的夹袄里给秀贞看,“这也是用我爸爸的旧衣服改的。”
\par “你也是用你爸爸的?你怎么知道这衣服就是小桂子她爹的?”秀贞微笑着瞪眼问我,她那样子很高兴,她高兴我就高兴,可是我怎么会知道这是小桂子她爹的?她问得我答不出,我斜着头笑了,她逗着我的下巴还是问:
\par “说呀!”
\par 我们俩这时是蹲在箱子旁,我很清爽地看着她的脸,刘海儿被风吹倒在一边,她好像一个什么人,我却想不出。我回答她说:
\par “我猜的。那么——”我又低声地问她,“我管小桂子她爹叫什么呀?”
\par “叫叔叔呀!”
\par “我已经有叔叔了。”
\par “叔叔还嫌多?叫他思康叔叔好了,他排行第三,叫他三叔也行。”
\par “思康三叔,”我嘴里念着,“他几点钟回家?”
\par “他呀,”秀贞忽然站起来,紧皱着眉毛斜起头在想,想了好一会儿才说,“快了。走了有个把月了。”
\par 说着她又走进屋,我再跟进去,弄这弄那,又跟出来,搬这搬那,这样跟出跟进忙得好高兴。秀贞的脸这时粉嘟嘟的了,鼻头两边也抹了灰土,鼻子尖和嘴唇上边渗着小小的汗珠,这样的脸看起来真好看。
\par 秀贞用袖子抹着她鼻子上的汗,对我说:“英子,给我打盆水来会不会?屋里要擦擦。”
\par 我连忙说:
\par “会,会。”
\par 跨院的房子原和门房是在一溜沿的,跨院多了一个门就是了,水缸和盆就放在门房的房檐下。我掀开水缸的盖子,一勺勺地往脸盆里舀水,听见屋里有人和秀贞的妈说话:
\par “姑娘这程子可好点了吗?”
\par “唉!别提了,这程子又闹了,年年开了春就得闹些日子,这两天就是哭一阵子笑一阵子的,可怎么好!真是……”
\par “这路毛病就是春天犯得凶。”
\par 我端了一盆水,连晃连洒,泼了我自己一身水,到了跨院屋里,也就剩不多了。把盆放在椅子上,忽然不知哪儿飘来炒菜香,我闻着这味儿想起了一件事,便对秀贞说:
\par “我要回家了。”
\par 秀贞没听见,只管在抽屉里翻东西。
\par 我是想起回家吃完饭还要到横胡同去等妞儿,昨天约会好了的。
\par 又凉又湿的裤子,贴在我的腿上,一进门妈妈就骂了:
\par “就在井窝子玩一上午?我还以为你掉到井里去了呢!看弄这么一身水!”妈一边给我换衣服,一边又说:“打听打听北京哪个小学好,也该送进学堂了,听说厂甸那个师大附小还不错。”
\par 妈这么说着,我才看见原来爸爸也已经回来了,我弄了一身水,怕爸爸要打骂我,他厉害得很,我缩头看着爸爸,准备挨打的姿势,还好他没注意,吸着烟卷在看报,漫应着说:
\par “还早呢,急什么。”
\par “不送进学堂,她满街跑,我看不住她。”
\par “不听话就打!”爸的口气好像很凶,但是随后却转过脸来向我笑笑,原来是吓我呢!他又说:“英子上学的事,等她叔叔来再对他说,由他去管吧!”
\par 吃完饭我到横胡同去接了妞儿来,天气不冷了,我和妞儿到空闲着的西厢房里玩,那里堆着拆下来的炉子、烟筒,不用的桌椅和床铺。一只破藤箱子里,养了最近买的几只刚孵出来的小油鸡,那柔软的小黄绒毛太好玩了,我和妞儿蹲着玩弄箱里的几只小油鸡。看小鸡啄米吃,总是吃,总是吃,怎么不停啊!
\par 小鸡吃不够,我们可是看够了,盖上藤箱,我们站起来玩别的。拿两个制钱穿在一根细绳子上,手提着,我们玩踢制钱,每一踢,两个制钱打在鞋帮上嗒嗒地响。妞儿踢时腰一扭一扭的,显得那么娇。
\par 这一下午玩得好快乐,如果不是妞儿又到了她吊嗓子的时候,我们不知要玩到多么久。
\par 爸爸今天买来了新的笔和墨,还有一叠红描字纸。晚上,在煤油灯底下,他教我描,先念那上面的字:“一去二三里,烟村四五家,亭台六七座,八九十枝花。”
\par 爸爸说:
\par “你一天要描一张,暑假以后进小学,才考得上。”
\par 早上我去惠安馆找秀贞,下午妞儿到西厢房里来找我,晚上描红字,我这些日子就这么过的。
\par 小油鸡的黄毛上长出短短的翅膀来了,我和妞儿喂米喂水又喂菜,宋妈说不要把小鸡肚子撑坏了,也怕被野猫给叼了去,就用一块大石头压住藤箱盖子,不许我们随便掀开。
\par 妞儿和我玩的时候,嘴里常常哼哼唧唧的,那天一高兴,她竟扭起来了,她扭呀扭呀比来比去,嘴里唱着:“……开哀开门嗯嗯儿,碰见张秀才哀哀……”
\par “你唱什么?这就是吊嗓子吗?”我问。
\par “我唱的是打花鼓。”妞儿说。
\par 她的兴致很好,只管轻轻地唱下去,扭下去,我在一旁看傻了。她忽然对我说:“来!跟我学,我教你。”
\par “我也会唱一种歌,”不知怎么,我想我也应当现一现我的本事,一下子想起了爸爸有一回和客人谈天数唱的一首歌,后来爸曾教了我,妈还说爸爸教我这种歌真是没大没小呢!
\par “那你唱,那你唱。”妞儿推着我,我却又不好意思唱了,她一定要我唱,我只好结结巴巴地用客家话念唱起来:
\par “你听着——想来么事想心肝,紧想心肝紧不安!我想心肝心肝想,正是心肝想心肝……”
\par 我还没数完呢,妞儿已经笑得挤出了眼泪,我也笑起来了,那几句词儿真拗嘴。
\par “谁教你的?什么心肝想心肝,心想心肝想的,哈哈哈!这是哪国的歌儿呀!”
\par 我们俩搂在一堆笑,一边瞎说着心肝心肝的,也闹不清是什么意思。
\par 我们真快乐,胡说,胡唱,胡玩,西厢房是我们的快乐窝,我连做梦都想着它。妞儿每次也是玩得够不够的才看看窗外,忽然叫道:“可得回去了!”说完她就跑,急得连“再见”都来不及说。
\par 忽然一连几天,横胡同里接不到妞儿了,我是多么的失望,站在那里等了又等。我慢慢走向井窝子去,希望碰见她,可是没有用。下午的井窝子没那么热闹了,因为送水的车子都是上午来,这时只有附近人家自己推了装着铅桶的小车子来买水。
\par 我看见长班老王也推了小车子来,他一趟一趟来好几趟了,见我一直站在那里,奇怪地问我:“小英子,你在这儿发什么傻?”
\par 我没有说什么,我自己心里的事,自己知道。我说:
\par “秀贞呢?”我想如果等不到妞儿,就去找秀贞,跨院里收拾得好干净了。但是老王没理我,他装满了两桶水,就推走了。
\par 我正在犹豫着怎么办的时候,忽然从西草厂口上,转过来一个熟悉的影子,那正是妞儿,我多高兴!我跑着迎上去,喊道:“妞儿!妞儿!”她竟不理我,就像不认识我,也像没听见有人叫她。我很奇怪,跟在她身边走,但她用手轻轻赶开我,皱着眉头眨眼,意思叫我走开。我不知道是怎么回事,但是她身后几步远有一个高大的男人,穿着蓝布大褂,手提着一个脏了的长布口袋,袋口上露出来我看见是胡琴。
\par 我想这一定是妞儿的爸爸。妞儿常说“我怕我爹打”、“我怕我爹骂”的话,我现在看那样子就知道我不能跟妞儿再说话了,便转身走回家,心里好难受。我口袋里有一块化石,可以在砖上写出白字来,我掏出来,就不由得顺着人家的墙上一直画下去,画到我家的墙上。心里想着如果没有妞儿一起玩,是多么没有意思呢!
\par 我刚要叫门,忽然听见横胡同里咚咚咚有人的跑步声,原来是妞儿气喘着跑来了,她匆匆忙忙神色不安地说:“我明儿再来找你。”没等我回答,她就又跑回横胡同了。
\par 第二天早晨,妞儿来找我,我们在西厢房里,蹲下来看小油鸡。掀开藤箱盖子,我们俩都把手伸进去摸小鸡的羽毛,这样摸着摸着,谁也没说话。我本是要说话的,但是没有出声,只是心里在问她:“妞儿,为什么好多天没来找我?”“妞儿,是你爸爸很厉害不许你来吗?”“妞儿,昨天为什么不许我跟你说话?”“妞儿,你一定有什么难受的事吧?”真奇怪,这些话都是我心里想的,并没有说出口,可是她怎么知道的,竟用眼泪来回答我?她不说话,也不用袖子去抹眼,就让眼泪滴答滴答落在藤箱里,都被小油鸡和着小米吃下去了!
\par 我不知怎么办好了,从侧面正看见她的耳朵,耳垂上扎了洞用一根红线穿过去,妞儿的耳朵没有洗干净,边沿上有一道黑泥。我再顺着她的肩膀向下看,手腕上有一条青色的伤痕,我伸手去撩起她的袖口看,她这才惊醒了,吓得一躲闪,随着就转过头来向我难过地笑笑。早晨的太阳,正照到西厢房里,照到她的不太干净的脸上,又湿又长的睫毛,一闪动,眼泪就流过泪坑淌到嘴边了。
\par 忽然,她站起来,撩开袖口,撩起裤角,轻轻地说:
\par “看我爸爸打的!”
\par 我是蹲着的,伸出手正好摸到她的腿上那一条条肿起的伤痕。我轻轻地摸,倒惹得她哭出声音来了。她因为不敢放声,嘤嘤地小声哭,真是可怜。我说:
\par “你爸爸干吗打你?”
\par 她当时说不出话来,哭了好一会儿才说:
\par “他不许我出来玩。”
\par “是因为在我家待太久了?”
\par 妞儿点点头。
\par 因为在我家玩久了,害得她挨打,我又难过,又害怕,想到那个高大的男人,我不由得说:
\par “那么你快回去吧!”她站着不动,说:
\par “他一早出去还没回来。”
\par “那么你妈呢?”
\par “我妈也拧我,她倒不管我出来的事。爸爸也打她。打了她,她就拧我,说是我害的。”
\par 妞儿哭了一阵子好些了,又跟我说这说那的,我说我从来没见过她的妈妈,妞儿说她的妈妈有点跛,一天到晚就是坐在炕头上给人缝补衣服赚钱。
\par 我告诉妞儿,我们从前不住在北京,是从一个很远的岛上来的,她也说:
\par “我们从前也不住在这儿,我们住在齐化门那边。”
\par “齐化门?”我点点头说,“我知道那地方。”
\par “你怎么会也知道齐化门呢?”妞儿奇怪地问我。
\par 我想不出我是怎么知道的,但我的确知道,好像有什么人大清早曾带我去过那里,而且我也像看见了那里的样子似的,不,不,不是,我所看见的很模糊,也许那是一个梦吧?因此我就回答妞儿说:
\par “我梦见过那个地方,有没有城墙?有一天,有一个女人抱着一个包袱,大清早上,偷偷地向城墙走去……”
\par “你是讲故事吧?”
\par “也许是故事,”我斜着头又深深地想了想,“反正我知道齐化门就是了。”
\par 妞儿笑了笑,手伸过来搂着我的脖子,我的手也伸过去搂住她的。但当我捏住她的肩头,她轻轻喊了一声“痛!痛!”
\par 我的手连忙松开,她又皱着眉说:“连这儿都给我抽肿了!”
\par “什么抽的?”
\par “掸子。”停了一下她又说,“我爸,还有我妈,他们……”但她顿住不说了。
\par “他们怎么样?”
\par “不说了,下回再跟你说。”
\par “我知道,你爸爸教你唱戏,要你赚钱给他们花。”这是我听宋妈跟妈妈讲过的,所以一下子就给说出来了。“要你赚钱还打你,凭什么!”我说到后来气愤起来了。
\par “喝喝,你瞧你什么都知道,我不是要跟你说唱戏的事,你哪儿知道我要跟你说什么呀!”
\par “到底要说什么呢?说嘛!”
\par “你这么着急,我就不说了。你要是跟我好,我有好些话要跟你说,就是不许你跟别人说,也别告诉你妈。”
\par “我不会,我们小声地说。”
\par 妞儿犹豫了一会儿,伏在我的耳旁小声而急快地说:
\par “我不是我妈生的,我爸爸也不是亲的。”
\par 她说得那样快,好像一个闪电过去那么快,跟着就像一声雷打进了我的心,使我的心跳了一大跳。她说完后,把附在我耳旁的手挪开,睁着大眼睛看我,好像在等着看我听了她的话,会怎么个样子。我呢,也只是和她对瞪着眼,一句话也说不出。
\par 我虽然答应妞儿不讲出她的秘密,可是妞儿走了以后,我心里一直在想着这件事,我越想越不放心,忽然跑到妈妈面前,愣愣地问:
\par “妈,我是不是你生的?”
\par “什么?”妈奇怪地看了我一眼,“怎么想起问这话?”
\par “你说是不是就好了。”
\par “是呀,怎么会不是呢?”停一下妈又说,“要不是亲生的,我能这么疼你吗?像你这样闹,早打扁了你了。”
\par 我点点头,妈妈的话的确很对,想想妞儿吧!“那么你怎么生的我?”这件事,我早就想问的。
\par “怎么生的呀,嗯——”妈想了想笑了,胳膊抬起来,指着胳肢窝说:
\par “从这里掉出来的。”
\par 说完,她就和宋妈大笑起来。

\subsubsection*{三}

\par 我手里拿着一个空瓶子和一根竹筷子,轻轻走进惠安馆,推开跨院的门,院里那棵槐树,果然又垂着许多绿虫子,秀贞说是吊死鬼,像秀贞的那几条蚕一样,嘴里吐着一条丝,从树上吊下来。我把吊死鬼一条条弄进我的空瓶里,回家去喂鸡吃,每天可以弄一瓶。那些吊死鬼装在小瓶里,咕囊咕囊地动,真是肉麻,我拿着装了吊死鬼的瓶子,胳膊常常觉得痒麻麻的,好像吊死鬼从瓶里爬到我的手上了,其实并没有。
\par 我在把吊死鬼往瓶里装的时候,忽然想到了妞儿,心里很不安。她昨天又挨揍了,拿了两件衣服偷偷地找我,进门就说:
\par “我要找我亲爹亲妈去!”她的脸有一边被打得红肿了。
\par “他们在哪儿呢?”
\par “我不知道,到齐化门,再慢慢地找。”
\par “齐化门在哪儿呢?”
\par “你不是说你也知道那地方吗?”
\par “我是说我好像做梦梦见过那地方的。”
\par 妞儿把两件衣服塞在西厢房的空箱子里,很有主意地抹干了眼泪,恨恨地说:
\par “我非找着我亲爹不可。”
\par “你知道他长得什么样子吗?”我真佩服她,但觉得这是一件太大太大的事。
\par “我一天一天地找,就会找到我亲爹跟我亲娘。他们的样子我心里知道。”
\par “那么——”我也不知道要说什么,因为我一点主意也没有。
\par 妞儿临走的时候说,她不定哪天就要偷偷地走了,但一定会先来这里跟我说一声,并且带走存在这里的两件衣服。
\par 我昨天一直在想妞儿的事,心里很不舒服,晚上就吃不下饭了,妈妈摸摸我的头说:
\par “好像有点热,不吃也好,早点去睡。”
\par 我上了床,心里还是不舒服,又说不出,就哭起来了,妈妈很奇怪,她说:
\par “哭什么?哪儿不舒服?”我不知怎么一来竟哭着说:
\par “妞儿她爸爸啊……”
\par “妞儿她爸爸?怎么啦?她爸爸怎么着你啦?”宋妈也过来了,她说:
\par “那个不是东西的,准是骂了我们英子了,还是打了你啦?”
\par “不是!”我忽然觉出我说了什么糊涂话,便撒赖地哭喊:“我要找我爸爸!”
\par “是要找你爸爸呀!唉!吓人!”宋妈和妈妈都笑了。妈妈说:
\par “你爸爸今天去看你叔叔,回来得晚点,你先睡吧!”她又对宋妈说:“英子一生下来,就给她爸爸惯的,一不舒服,爸爸抱着睡。”
\par “羞不羞?”宋妈用一个手指头划我的脸,我不理她,转过脸冲着墙闭上眼睛。
\par 今天我早晨起来就好得多了,不像昨天那样不安心。但是现在又想起妞儿,手里不由得停止了捉虫子的工作,呆呆地想,不知道什么时候,妞儿就会离开我。
\par 我把瓶子扔在树下,站起来走到窗下向里看。秀贞正在里屋床前的一把杌凳上坐着,面向着床,我只看到她那小平板儿似的背影,辫子也没梳好。她比手划脚,又扬手轰苍蝇,其实哪里有苍蝇?我轻轻地走进屋里,在外屋桌旁靠着,傻看她在干什么,只听她说:
\par “我准知道你昨儿晚上没吃饭就睡觉了,是不是?那怎么行!”
\par 咦!真奇怪,秀贞怎么知道我昨晚没吃饭就睡觉了呢?我倚在里屋的门框说:
\par “谁告诉你的?”
\par “啊?”她回过头来看见我愁眉不展的样子,很正经地对我说:
\par “还用人告诉我吗?这碗粥一动也没动呀!”说完指着床旁茶几上的一个碗和一双筷子。
\par 我这才知道秀贞说的不是我。自从天气暖和了,打开一向深闭的跨院门以后,秀贞就一天到晚在这两间屋里出出进进,说着那我又懂、又不懂的话。最先我以为是秀贞跟我玩“过家家儿”,后来才又觉得并不是假装的事情,它太像真事了!
\par 秀贞又向着那空床发呆看了一会儿,转过头来,轻手轻脚地拉着我走到屋外来,小声地说:
\par “睡着了,让他睡去吧!这一场病也真亏他,没亲没故的!”
\par 外屋书桌上摆着那缸春天买的金鱼,已经死了几条,可是秀贞还是天天勤着换水,玻璃缸里还加了几根水草,红色的鱼在绿色的水草中钻来钻去,非常好玩。我怎么知道鱼是红的草是绿的呢?妈妈教过我,她说快考小学了,老师要问颜色,要问住在哪儿,要问家里有几个人。秀贞还养了一盒蚕,她对我说过:
\par “你要上学,我们小桂子也该上学了,我养点蚕,吐了丝,好给小桂子装墨盒用。”
\par 有几条蚕已经在吐丝了,秀贞另外把它们放在一个蒙了纸的茶杯上,就让它们在那纸上吐丝。真有趣,那些蚕很乖,就不会爬到茶杯下面来。另外的许多蚕还在吃桑叶。
\par 秀贞在打扫蚕屎,她把一粒粒的蚕屎装进一个铁罐里,她已经留了许多,预备装成一个小枕头,给思康三叔用。因为他每天看书眼睛得保养,蚕屎是明眼的。
\par 我在旁边静静地看着鱼缸,看着吐丝。院子里的树,正靠在窗下,这屋里荫凉得很,我们俩都不敢大声说话,就像真的屋里躺着一个要休息的病人。
\par 秀贞忽然问我:
\par “英子,我跟你说的事记住没有?”
\par 我一时想不起是什么事,因为她对我说过的事,真真假假的太多了。她说过将来要我跟小桂子一块去上学,小桂子也要考厂甸小学。她又告诉我从厂甸小学回家,顺着琉璃厂直到厂西门,看见鹿犄角胡同雷万春的玻璃窗里那对大鹿犄角,一拐进椿树胡同就到家了。可是她又说过,她要带小桂子去找思康三叔,做了许多衣服和鞋子,行李都打点好了。
\par 我最记得秀贞说过的话,还是她讲的生小桂子的那回事。有一天,我早早溜到这里找秀贞,她看见我连辫子都没梳,就端出梳头匣子来,从里面拿出牛角梳子,骨头针和大红头绳,然后把我的头发散开来,慢慢地梳。她是坐在椅子上的,我就坐在小板凳上,夹在她的两腿中间,我的两只胳膊正好架在她的两腿上,两只手摸着她的两膝盖,两块骨头都成了尖石头,她瘦极了。我背着她,她问我:
\par “英子,你几月生的?”
\par “我呀?青草长起来,绿叶发出来,妈妈说,我生在那个不冷不热的春天。小桂子呢?”秀贞总把我的事情和小桂子的事情连在一起,所以我也就一下子想起小桂子。
\par “小桂子呀,”秀贞说,“青草要黄了,绿叶快掉了,她是生在那不冷不热的秋天。那个时光,桂花倒是香的,闻见没有?就像我给你擦的这个桂花油这么香。”她说着,把手掌送到我的鼻前来晃一晃。
\par “小——桂——子。”我吸了吸鼻子,闻着那油味,不由得一字字地念出来,我好像懂得点那意思了。
\par 秀贞很高兴地说:
\par “对了,小桂子,就是这么起的名儿。”
\par “我怎么没看见桂花树?这里哪棵树是桂花?”我问。
\par “又不是在这屋子里生的!”秀贞已经在编我的辫子了,编得那么紧,拉得我的头发根怪痛的,我说:
\par “为什么用这么大的力气呀?”
\par “我当时要是有这么大力气倒好了,我生了小桂子,浑身都没劲儿,就昏昏沉沉地睡,睡醒了,小桂子不在我身边了。我睡觉时还听见她哭,怎么醒了就没了呢?我问,孩子呢?我妈要说什么,我婶儿接过去了,她瞥了我妈一眼,跟我和和气气地说:你的身子弱,孩子哭,在你身边吵,我抱到我屋去了。我说,噢。我又睡着了。”秀贞说到这儿停住了,我的辫子已经扎好,她又接着说:
\par “仿佛我听我妈对我婶说:不能让她知道。真让人纳闷儿,到底是怎么档子事儿?我怎么到这儿就接不下去了呢?是她们把孩子给——?还是扔——绝不能够!绝不能够!”
\par 我已经站起来,脸冲着秀贞看,她皱着眉头,正呆呆地想。她说话常常都会忽然停住了,然后就低声地说“真让人纳闷儿,到底是怎么档子事儿?”的话。她收梳头匣子的时候,我看见我送小桂子的手表在匣子里,她拿起手表,放在掌心里,又说:
\par “小桂子她爹也有个大怀表死了当了,当了那个表,他才回的家,这份穷,就别提了!我当时就没告诉他我有了。反正他去个把月就回来,他跟我妈说,放心,他回家卖了山底下的白薯地,就到北京来娶我。千山万水,去一趟也不容易,我要是告诉他我有了,不也让他惦记着!你不知道他那情意多深!我也没告诉我妈我有了,就不出口,反正人归了他了,等嫁了再说也不迟……”
\par “有了什么了?”我不明白。
\par “有了小桂子呀!”
\par “你不是刚说什么没有了吗?”我更不明白。
\par “有了,没了,有了,没了,小英子,你怎么跟我乱扰?你听我给你算。”她把我给小桂子的表收起来,然后用手指捏着算给我听:
\par “他是春天走的。他走的那天,天儿多好,他提着那口箱子,都没敢多看我,他的同乡同学,有几个送他到门口儿的,所以他就没好再跟我说什么。好在头天晚上我给他收拾箱子的时候,我们俩也说得差不多了。他说,惠安的日子很苦,有办法的都到海外谋生去了,那儿的地不肥,不能种什么,白薯倒是种了不少。他们家,常年吃白薯,白薯饭,白薯粥,白薯干,白薯条,白薯片,能叫外头去的人吃出眼泪来。所以,他就舍不得让我这个北边人去吃那个苦头儿。我说可不是,我妈就生我独一个儿,跟了你去吃白薯,她怎么舍得我!他说,你是个孝女,我也是个孝子,万一我母亲扣住了我,不许我再到北京来了呢?我说,那我就追你去。
\par “送他到门口,看他上了洋车,抬头看看天,一块白云彩,像条船,慢慢地往天边儿上挪动,我仿佛上了船,心是飘的,就跟没了主儿似的。
\par “我送他出去,回到屋里来,恶心要吐,头也昏,有点儿后悔没告诉他这件事,想追出去,也来不及了。
\par “日子一天天地捱,他就始终没回来,我肚子大了,瞒不住我妈,她急得盘问我,让我说不出道不出的,可是我也顾不得害臊了,就都告诉了我妈。我说,他总有一天会回来,他不回来,我去!我妈听了拿手堵住我的嘴,直说:姑娘,可别这么说了,这份丢人呀!他真要是不回来,咱们可不能嚷嚷出去,就这么,把我送回了海甸。
\par “小桂子生下来,真不容易,我一点劲儿都没有,就闻着窗户外头那棵桂花树吹进来的一阵阵香气,我心说,生个女的就叫小桂子。接生的老娘婆叫我咬住了辫子,使劲,使劲,总算落了地,呱呱哭声好大呀!”
\par 秀贞说到这儿,喘了一大口气,她的脸色变青了,故事接不下去,就随便说了,她说:
\par “小英子,你不心疼你三婶吗?”
\par “谁是三婶?”
\par “我呀!你管思康叫三叔,我就是你三婶,你还算不过这账来。叫我一声。”
\par “嗯——”我笑了,有些难为情,但还是叫了她,“三婶。秀贞。”
\par “你要是看见小桂子就带她回来。”
\par “我怎么知道小桂子什么样儿?”
\par “她呀,”秀贞闭上眼睛想着说,“粉嘟嘟的一个小肉团子,生下来我看见一眼了,我睡昏过去那阵儿,听我妈跟老娘婆说,瞧!这真是造孽,脖子后头正中间儿一块青记,不该来,非要来,让阎王爷一生气用指头给戳到世上来的!小英子,脖子后头中间有指头大一块青记,那就是我们小桂子,记住没有?”
\par “记住了。”我糊里糊涂地回答。
\par 那么,她现在问我说的事记住没有,就是这件事吗?我回答她说:“记住了,不是小桂子那块青记的事吗?”
\par 秀贞点点头。
\par 秀贞把桌上的蚕盒收拾好,又对我说:
\par “趁着他睡觉,咱们染指甲吧。”她拉我到院子里。墙根底下有几盆花,秀贞指给我看,“这是薄荷叶,这是指甲草。”她摘下来了几朵指甲草上的红花,放在一个小瓷碟里,我们就到房门口儿台阶上坐下来。她用一块冰糖在轻轻地捣那红花。我问她:
\par “这是要吃的吗?还加冰糖?”
\par 秀贞笑得咯咯的,说:
\par “傻丫头,你就知道吃。这是白矾,哪儿来的冰糖呀!你就看着吧。”
\par 她把红花朵捣烂了,要我伸出手来,又从头上拿下一根卡子,挑起那烂玩意儿,堆在我的指甲上,一个个堆了后,叫我张着手不要碰掉,她说等它们干了,我的手指甲就变红了,像她的一样,她伸出手来给我看。
\par 我的手,张开了一会儿,已经不耐烦了,我说:
\par “我要回家去了。”
\par “你回家非弄坏了不可,别走,听我给你讲故事儿。”她说。
\par “我要听三叔的故事。”
\par “小声点儿,”她向我摆手,轻轻地说,“让我先看看他醒过来没有,他要不要喝水。”她进去了一下,又出来了,坐下后,手支撑在大腿上托着下巴颏儿,忽然向着槐树发起呆来。
\par “说呀!你。”我说。
\par 她惊了一下,“嗯?”好像没听见我的问话,但跟着眼泪掉下来了,“还说呢,人都没影儿了,都没影儿了!老的!小的!”
\par 我一声不响,她自己抽抽噎噎地哭了一会儿,才又大喘了一口气,望着我笑了,那泪坑!我就觉得在什么地儿看见过秀贞这个人,这个脸。
\par 秀贞用手指抹抹泪,拉过我的手托在她的手上,这样,我就轻松点,不觉得张开染指甲的手很累了。她又侧起身子看着跨院门,好像在张望什么人。她自言自语地说:
\par “就是这时节他来的,一卷铺盖,一口皮箱,搬进了这小屋里。他身穿一件灰大褂,大襟上别着一支笔。我正在屋里没打扫完呢!爹领他进来的,对他说,‘会馆里正院房子都住满了,陈家二老爷让给您腾出这两间小屋来。’他说:‘好,好,这样就很好。’爹给他打开行李,把那床又薄又旧的棉被摊开,我心想,他怎么过这北京的大冬天?小英子,住在会馆念书的学生,有几个有钱的?有钱的就住公寓去了。我爹常说,想当年,陈家二老爷上京来考举,还带着个小碎催伺候笔墨呢!二老爷中了举,在北京做官,就把这间会馆大翻修了一回,到如今,穷学生上京来念书,都是找着二老爷说话。二老爷说,思康是他们乡里的苦学生,能念出书来,要我们把堆煤的这两间小屋收拾了给他住。
\par “我还在赶着擦玻璃呢,没正眼看他。我爹对他说,这床被呀!过不了冬。爹真爱管人家的事,他准是不好意思了,就乱嗯嗯啊啊的没说出什么来。爹又问他在哪家学堂,他说在北京大学,喝!我爹又说了,这道不近,沙滩儿去了!可是个好学堂呀!
\par “爹帮着他收拾那几件破行李,就出去了,临走看见我还在擦玻璃,他说,行啦,姑娘。我跟出来了,回头看了他一眼,谁知道他也正抬眼看我呢!我心里一跳,迈门槛儿时差点摔出去!看他那模样儿,两只眼儿到底有多深!你还没看清楚他,他就把你看穿了。回到屋里来,我吃饭睡觉,眼前都摆着他的两只那么样看人的眼睛。这就是缘分,会馆一年到头,来来往往的大学生多了,怎么我就——我就……咳!”
\par 秀贞的脸微微地红涨,抬起我的手,看我染的指甲干了没有,她轻轻地吹着我的指甲,眼皮垂下来,睫毛像一排小帘子,她问我:
\par “小英子,你明白了吗?缘分?”她并不一定要我回答她,我也没打算回答她,只是心里想着,这样的长睫毛,有一个人也有的,我想到西厢房我那位爱哭的朋友了。秀贞又接着唠叨:
\par “我天天给他送开水去,这件事本该是我爹做的。早晚两趟,我们烧了大壶开水,送到各屋里给先生们洗脸,泡茶。爹走惯了正院,总是把跨院给忘了。有时候思康就自己到我们窗根底下来要。‘长班,’他就是这么轻轻地叫一声,‘有滚水吗?’爹这才想起来,赶紧给人家补送去。有时爹倒是没等叫就想起来了,可是他懒得再走,就支使我去。一来二去,这件差事——到跨院送开水,仿佛就该是我做的了。
\par “我送水,一句话也没跟他说过,我进了屋,他在书桌前坐着,就着灯看书呢,写字呢,我就绷着脸儿,打开那茶壶盖儿,刷——的,就听见开水灌进壶的声儿。他胆子小着呢,连眼都不敢斜过来,就那么搭着眼皮坐着。有一天,我也好新鲜,往前挪了一步,微探着身子看他写什么,谁知他也扭过头来了,说:‘认得字吗?’我摇了摇头。打这儿起,我们俩就说话了。”
\par “那时小桂子在哪儿呢?”我忽然想起这个跟秀贞有关系的人。
\par “她呀!”秀贞笑了,“还没影儿呢!对了,小桂子到底哪儿去了?你给找着没有?那是我们俩的命根子呀!我还没跟你说完呢,他有一天拉起我的手,就像我这么拉你的手,说:‘跟了我吧!’他喝了点儿酒,我也迷糊了,他喝酒是为的取暖,两间屋子,生一个小火,还时有时无的。那天风挺大,吹得门框直响,我爹跟我娘回海甸取地租去了,让舅妈来陪我,她睡了,我就溜到这跨院里来。他的脸滚烫,贴着我的脸,他说了好多话,酒气喷着我,我闻也闻醉了。
\par “他常爱喝点儿酒,驱驱寒意,我就偷偷地买了半空儿花生,送到他的屋里来,给他下酒喝。北风打着窗户纸,响得吹笛儿似的。我握着他的手,暖乎乎的,两个人就不冷了。
\par “他病了,我一趟一趟地跑,可瞒不住我妈了。那天我端着粥,要送给他吃,妈说:‘避点儿嫌疑,姑娘,懂得不懂得?’我一声也没言语。”
\par 我从秀贞的眼里,仿佛看见了躺在里屋床上的思康三叔了;他蓬着头发,喝水也没力气,吃饭也没力气,就哼哼着。
\par “后来呢?好了没有?”我不由得问。
\par “不好怎么走的?我可直要倒下了!原来是小桂子来了!”
\par “在哪里?”我转回头去看跨院门,并没有人影儿。在我的幻想中,跨院门边,应当站着一个女孩子:红花的衫裤,一条像狗尾巴似的黄毛辫子,大大的眼睛,一排小帘子似的长睫毛,一闪一闪的,在向我招手呢!我头有点昏,好像要倒下来,闭了一下眼睛,再睁开,门那边,果然有个影子,越走越近了,那么大的一个东西,原来--原来是秀贞的妈正向我招手,她说:
\par “秀贞,怎么让小英子在老爷儿里晒着?”
\par “刚才这地方没太阳。”秀贞说。
\par “快挪开,这边儿不是有荫凉吗?”老王妈过来拉起我。
\par 那幻影在我眼中消失了,我忽然又想起秀贞还没讲完的故事。我说:
\par “妞儿,不,小桂子在哪儿呢?我刚说的?”
\par 秀贞扑哧笑了,指着她的肚子:
\par “在这儿呢,还没生呢!”
\par 秀贞的妈是来这院里晾衣服。一根绳子从树枝上牵到墙那边,王妈正一件件地往上晾。
\par 秀贞看了说:
\par “妈,裤子晾在靠墙边去吧,思康出来进去的不合适。”
\par 王妈骂说:
\par “去你的!”
\par 秀贞被她妈妈骂一句,并不生气,又对我说:
\par “我妈倒是也疼思康,她跟我爹说,咱们没儿子,你这老东西又没念过书,有个读书识字的人在咱们家也是好事儿。我爹这才答应了。我刚才说到哪儿啦!噢,他好了我不是病了吗?他就说都是他害的我,他不是说要娶我,教我念书吗?就在这时候,他家里来了电报,他妈病了,叫他赶快回去……”
\par “小英子,”王妈忽然截住秀贞的话,对我说,“你怎么那么爱听她那颠三倒四的废话?也真怪,小孩子都怕她,躲着她,就是你不。”
\par “妈,您别搅,我这儿还没说完呢!我还有事托小英子呢!”
\par 老王妈不理她,只顾对我说:
\par “小英子,该回去了,刚才我听见宋妈在胡同里叫你,我不敢说你在这儿。”
\par 老王妈说完拿着空盆走了。秀贞看见她妈妈走出了跨院门,才又说:“思康这一去,有……”她扳着手指头算,“有一个多月了,有六年多了,不,还有一个多月就回来,不,还有一个月我就生小桂子了。”
\par 不管是六年,还是一个多月,秀贞跟我一样的算不清楚。她这时把我的手拿起来看看,便把指甲上的干烂花剔开,哟,我的指甲都是红的了!我高兴极了,直笑直笑,摆弄着我的手。
\par “小英子,”她又低声说,“我有件事托你,看见小桂子就叫她来,一块儿找她爹去,我们要是找到她爹,我病就好了。”
\par “什么病?”我看着秀贞的脸。
\par “英子,人家都说我得了疯病,你说我是不是疯子?人家疯子都满地捡东西吃,乱打人,我怎么会是疯子,你看我疯不疯?”
\par “不,”我摇摇头,真的,我只觉得秀贞那么可爱,那么可怜,她只是要找她的思康跟妞儿——不,跟小桂子。
\par “他们怎么都走了不回来了呢?”我又问。
\par “思康准是让他妈给扣住了。小桂子呢,我也纳闷是怎么档子事儿,没在海甸,没在我婶儿屋里。我一问,妈急了,说:‘扔啦!留那么一个南蛮子种儿干吗?反正他也不回来了,坑人!’我一听,登时就昏倒了,醒了,他们就说我是疯子。小英子,我千托万托你,看见小桂子就带她来,我什么都预备好了,回去吧。”
\par 我听得愣了,脑子里好像有一幅画,慢慢越张越大,我的头也有点不舒服似的,我一边答应:“好好,好好。”一边跑出跨院,跑出惠安馆,一路踢着小石块,看着我手上的红指甲,回到了家。

\subsubsection*{四}

\par “看你脸晒得那么红!快来吃饭。”妈妈看见我满头大汗地回来,并没有太责备我。
\par 但是我只想喝水,不想吃饭,我灌了几杯凉开水下去,坐到饭桌上,喘着气,拿起筷子,可是看我自己的指甲玩。
\par “谁给你染的?”妈问。
\par “小妖精,小孩子染指甲,做唔得!”爸爸也半生气地说。
\par “谁给你染的?”妈又问。
\par “嗯——”我想了一下,“思康三婶。”我不敢,也不肯说秀贞是疯子。
\par “跑到外面去认什么阿叔阿婶!”妈给我挟了一碟子菜,又对我说,“你叔叔说,还有一个月就要考小学了,你到底会数到什么数了?算算看,不会数就考不上的。”
\par “一,二,三……十八,十九,二十,二十六……”我的脑筋实在有些糊涂,只想扔下筷子去床上躺一会儿,但是我不肯这样做,因为他们会说我有病了,不许我出去。
\par “乱数!”妈妈瞪了我一眼,“听我给你算,二俗,二俗录一,二俗录二,二俗录三,二俗录素,二俗录五……”
\par 在旁边伺候盛饭的宋妈首先忍不住笑了,跟着我和爸爸都哈哈大笑起来,我乘此扔下筷子,说:
\par “妈,听你的北京话,我饭都吃不下了,二十,不是二俗;二十一,不是二俗录一;二十二,不是二俗录二……”
\par 妈也笑了,说:
\par “好啦好啦,不要学我了。”
\par 我没有吃饭,爸妈都没注意。大概刚才喝了凉开水,人好些了,我的头已经不晕了。爸妈去睡午觉,我走到院子里,在树下的小板凳上坐着,看那一群被放出来的小油鸡。小油鸡长得很大了,正满地啄米吃,树上蝉声“知了知了”地叫,四下很安静。我捡起一根树枝子在地上画,看见一只油鸡在啄虫吃,忽然想起在惠安馆捉的那瓶吊死鬼忘记带回来。
\par 我虽这样想着,但是竟懒得站起身来,好像要困了,不由得闭上了眼睛,随着俯下身子来,两手抱住头,深深地埋在大腿上。
\par 在这像睡不睡的梦中,我的眼前一片迷乱:在跨院的树下捉蚕,吊死鬼在玻璃瓶里蠕动着,一会儿又变成了秀贞屋里桌上的蚕,仰着头在吐丝,好像秀贞把蚕放在我的胳膊上爬,一发痒,猛睁开眼抬起头来看,原来是两只苍蝇在我的胳膊上飞绕。我扬扬手哄开苍蝇,又埋头睡下了。这回是一盆凉水,顺着我的脊背浇下来,凉飕飕的,我抱紧了头,不行,又是一盆凉水从脖子上灌下来,又凉又湿,我说冷啊!旁边有人咯咯地笑,我挣扎着站起来,猛下子醒了,睁开眼,闹不清这是什么时候了!因为天好像一下子暗了,记得我坐这里的时候是有阳光的呀!站在我面前的是妞儿,她在笑,我还觉得背脊是湿的冷的,用手背向后面去摸,却又不是湿的。但身上还是有些凉意,不禁打了一个哆嗦,随着又打了两个喷嚏,妞儿笑容收敛了,说:
\par “你怎么啦?傻呵呵的睡觉直说梦话。”
\par 我好像还没醒来,要站不住,便赶快又坐下来。这时雷声响了,从远处隆隆地响过来。对面的天色也像泼了墨一样的黑上来,浓云跟着大雷,就像一队黑色的恶鬼大踏步从天边压下来。起了微微的风,怪不得我身上觉得凉。我不由得问妞儿:
\par “你冷不冷?我怎么这么冷。”
\par 妞儿摇摇头,惊疑地看着我,问:
\par “你现在的样子真特别,好像吓着了,还是挨打了?”
\par “没有,没有,”我说,“爸爸只打我手心,从来不会像你爸爸打你那么凶。”
\par “那你是怎么了呢?”她又指指我的脸,“好难看啊!”
\par “我一定是饿的,中午没吃饭。”
\par 这时雷声更大了,好大的雨点滴落下来,宋妈到院子来收衣服,把小鸡赶到西厢房里。我和妞儿也跟着进来。宋妈把小鸡扣好在鸡笼里,就又跑出去,嘴里还说着:
\par “要下大雨了,妞儿回不去。”
\par 宋妈出去了以后,可不是,雨立刻下大了。我和妞儿倚着屋门看下雨。雨声那样大,噼噼啪啪地打落在砖地上,地上的雨水越来越多了,院角虽然有一个沟眼,但是也挤不过那么多的雨水。院子的水涨高了,漫过了较低的台阶,水溅到屋门来,溅到我们的裤脚上了,我和妞儿看这凶狠的雨水看呆了,眼睛注视着地上,一句话也不讲。忽然妈妈在北屋里窗内向我说话又扬手,话我听不见,扬手的意思是叫我们不要站在门口被雨溅湿了。我和妞儿便依着妈妈的手势进屋来,关上了门,跑到窗前向玻璃外面看。
\par “不知道要下多久?”妞儿问。
\par “你可回不去了。”我说完,连着又打了两个喷嚏。
\par 我望着屋里,想找个地方倒下来,最好有一床被让我卧在里面。屋里虽然有旧床铺,但床上堆了箱子和花盆,并且满是灰尘。我受不住了,不由得走向床那边去,靠在箱子上。忽然想起妞儿存在空箱里的两件衣服,便打开拿了出来。
\par 妞儿也过来了,她问:
\par “你要干吗?”
\par “帮我穿上,我冷了。”我说。
\par 妞儿笑笑说:
\par “你好娇啊!下一点雨,就又打喷嚏,又要穿衣服的。”
\par 她帮我穿上一件,另一件我裹在腿上。我们坐在一块洗衣板上,挤在墙角,这样我好像舒服一些。但是妞儿却心疼被我裹在腿上的衣服,说:
\par “我就这两件衣服,别给我拉扯坏了呀!”
\par “小气鬼,你妈给你做了好多衣服呢!借我一件都舍不得!”也许我的头又发晕,不知怎么,嘴里说妞儿的妈,心里可想到秀贞屋里炕桌上一包小桂子的衣服。
\par 妞儿瞪大了眼,指着她自己的鼻子说:
\par “我妈?给我做好多衣服?你睡醒了没有?”
\par “不是,不是,我说错了,”我仰起头,靠在墙上,闭上眼,想了一下才说:
\par “我是说秀贞。”
\par “秀贞?”
\par “我三婶。”
\par “你三婶,那还差不多,她给你做了好多衣服,多美呀!”
\par “不是给我做,是给小桂子做的。”我转过头,对着妞儿的脸看,她的一个脸,被我看成两个脸,两个脸又合成一个脸。是妞儿,还是小桂子,我分不清了,我心里想的,有时不是我嘴里说的,我的心好像管不住我的嘴了。
\par “干吗这么瞪我?”妞儿惊奇地把头略微闪躲了我一下。
\par “我在想一个人,对了,妞儿,讲讲你爸跟你妈的故事吧!”
\par “他们有什么可讲的!”妞儿撇了一下嘴,“我爸爸在前清家有皇上的时候,不用做事,一天到晚吃喝玩乐,后来前清家没有了,他就穷了,又不会做事,把钱全花光了,就靠拉胡琴赚钱,他教我唱戏,恨不得我一下子就唱得跟碧云霞那么好,那么赚钱。——嘿!小英子,我现在上天桥唱戏去了,围一圈子人听,唱完了我就捧着个小箩筐跟人要钱,一要钱人都溜了,回来我爸爸就揍我!他说,给钱的都是你爷爷,你得摆个笑脸儿,瞧你这份儿丧!说着他就拿棍子抡我。”
\par “你说的那个碧云霞也在天桥唱呀?”
\par “哪儿呀!人家在戏园子里唱,城南游艺园,离天桥也不远,听碧云霞的才都是大爷哪!可是我爸爸常说,在戏园子唱的,有好些是打天桥唱出来的。他就逼着我学,逼着我唱。”
\par “你不是也很爱唱吗?怎么说是他逼的。”
\par “我爱随我自己,愿意唱就唱,愿意给谁听就给谁听,那才有意思。就比如咱们俩在这屋里,我唱给你听。”
\par 是的,我想起刚认识妞儿的那天,油盐店的伙计要她唱,她眼睛含着泪的那样子。
\par “可是你还得唱呀!你不唱赚不了钱怎么办!”
\par “我呀,哼!”妞儿狠狠地哼了一声,“我还是要找我亲爹亲妈去!”
\par “那么你怎么原来不跟你亲爹亲妈在一起呢?”这是我始终不明白的一件事。
\par “谁知道!”妞儿犹豫着,要说不说的样子。外面的雨还是那么大,天像要塌下来,又像天上有一个大海的水都倒到地上来。
\par “有一天,我睡觉了,听我爸跟我妈吵架。我爸说:‘这孩子也够拗的,嗓门儿其实挺好,可是她说不玩就不玩,可有什么办法呢!’我那瘸子妈说:‘你越揍她,越不管事儿。’我爸说:‘不揍她,我怎么能出这口气!捡来的时候还没冬瓜大,我捧着抱着带回家,而今长得比桌子高了,可是不由人管了。’我妈说:‘你当初把她捡回来就错了主意,跟亲生亲养的到底不一样,说老实话,你也没按亲生那么疼她,她也不能拿你当亲爹那么孝顺。’我爸叹了口气,又说:‘一晃儿五六年了!我那天也真邪行,走到齐化门,屎到屁门了。’我妈说:‘是呀,你说一大早儿捡点煤核来烧,省得让人看见怪寒碜的,每天你不都是起来先出恭才漱口洗脸吗?那天你忙得没上茅房,饶着煤没捡回来,倒捡了个不知谁家的私生的小崽子来。’我爸又说:‘我想着找城根底下蹲蹲吧,谁知道就看见个小包袱了呢!我先还以为我要发邪财了,打开一看,敢情是她,活玩意儿,小眼还骨碌骨碌直转哪!’我妈妈说:‘哼!你如今打算在她身上发财,赶明儿唱得跟碧云霞那么红,可不易。'……”
\par 我又闭上眼睛,仰头靠着墙在听妞儿絮絮叨叨地说,我好像听过这故事,是谁讲的呢?还说大清早就把那孩子裹包裹包扔到齐化门城根去?也许我是做梦,我现在常常做梦,宋妈说我白天玩疯了晚饭又吃撑了,才又咬牙又撒癔症的。是吗?我就闭着眼问妞儿:
\par “妞儿,你跟我说了好几遍这故事啦!”
\par “胡说,我跟谁也没说过。我今儿头一回跟你说。你有时候糊里糊涂的,还说要上学呢!我瞧你考不上。”
\par “可是,我真是知道的呀!你生的那时候,正是青草要黄了,绿叶快掉了,那不冷不热的秋天,可是窗户外头倒是飘进来一阵子桂花的香气……”
\par 妞儿推推我,我睁开眼,她奇怪地问:
\par “你在说什么?是不是又睡着了撒癔症?”
\par “我刚才说了什么?”我有些忘了,刚才也许是在梦中。
\par 妞儿摸摸我的头,我的胳膊,她说:“你好烫啊!衣服穿多了吧!把我的衣服脱下来吧!”
\par “哪里热,我心里好冷啊!冷得我直想打哆嗦!”我说着,看自己的两条腿,果然抖起来。
\par 妞儿看着窗外说:
\par “雨停了,我该回去了。”
\par 她要站起来,我又拉住她,搂住她的脖子说:
\par “我要看你后脖子上的那块青记,小桂子,你妈说你后脖子有块青记,让我找找……”
\par 妞儿略微地挣开我,说:“你怎么今天总说小桂子小桂子的?你现在这样儿,就像我爸爸喝醉了说胡话一样!”
\par “是呀!你爸爸就爱喝口酒,冬天为的驱驱寒意,那天风挺大,你妈给他打了点酒,又买了半空儿花生。……”
\par 我糊里糊涂地说着,拉开妞儿那条狗尾巴小辫儿,可不是,可不是,恍恍惚惚地,我看见在那杂乱的黄头发根里面,中间是有一块指头大的青记。我浑身都抖起来了。
\par 妞儿把她的脸贴在我的脸上,惊奇地说:
\par “你怎么啦?你的脸好热啊!都红了,是不是病了?”
\par “没有,我没病,”我这时精神起来了,但是妞儿把我搂在她的怀里,我正好看到妞儿尖尖的下巴。她低下头来,一对大眼睛里,忽然含满了泪。我也好像有什么委屈,实在我是觉得头发重,支持不住了。妞儿这么搂着我,抚摸着我,一种亲爱的感觉,使我流出泪来了。妞儿说:
\par “英子,好可怜,身上这么烫!”
\par 我也说:
\par “你也好可怜,你的亲爹、亲妈——啊,妞儿,我带你找你的亲妈去,你们再一块儿去找你亲爹。”
\par “上哪儿找去?你睡觉吧,我怕你,你别瞎说了。”说着,她又搂紧我,拍哄我。但是我听了她的话,立刻从她怀里挣扎起来,喊着说:
\par “我不是瞎说!我是知道你亲妈在哪儿,就在不远,”我又搂着她的脖子附在她耳旁小声说:“我一定要带你去,你亲妈说的,教我看见你就带你去,就是,不错,脖子后面有块青记的嘛!”
\par 她又奇怪地望着我,好一会儿才说:
\par “你的嘴好臭,一定是吃多了上火。可是,真有这回事吗?……你说我亲妈?”
\par 我看着她那惊奇的眼睛,点点头。她的长睫毛是湿的,我一说,她微笑了,眼泪流到泪坑上!我觉得难过,又闭上眼,眼前冒着金星,再睁开眼,她变成秀贞的脸了,我抹去了眼泪再仔细看,还是妞儿的。我这时又管不住我的嘴了,我说:
\par “妞儿,晚上你吃完饭来找我,咱们在横胡同口见面,我就带你上秀贞那儿去,衣服你也不用带,她给你做了一大包袱,我还送了你一只手表,给你看时候。我也要送秀贞一点东西。”
\par 这时我听见妈在叫我。原来雨停了,天还是阴的。妞儿说:
\par “你妈叫你呢!咱们先别说了,那就晚上见吧!”说着她就站起身,匆匆地推门出去了。
\par 我很高兴,所以有一股力气站起来了,脱下妞儿的衣服,扔在鸡笼上。我推门出去,院子里一阵凉风吹着我,地上满是水,妈妈叫我顺着廊檐走,可是我已经蹚水过来了。妈妈拉起我的手,刚想骂我吧,忽然她又两手在我手上,身上,头上乱按,惊慌地说:
\par “怎么浑身这样烧,病了,看是不是?中午从太阳底下晒回来,脸通红,刚才又淋了雨,现在又蹚水。水,总是要玩水!去躺下吧!”
\par 我也觉得浑身没有力气了,随着妈妈拖我到小床来。她给我脱了湿的鞋,换了干的衣服,把我安置在床上躺下来,裹在软绵绵的被里,我的确很舒服,不由得闭上眼睛就睡着了。
\par 醒来的时候,觉得热了,踢开了被。这时屋里漆黑,隔着布帘子空隙,可以看见外屋已经点了灯。我忽然想起一件要紧的事,大声叫:
\par “妈,你们是不是在吃饭?”
\par “这样混,她居然要吃饭呢!”是爸爸的声音。跟着,妈妈进来了,端进来煤油灯放在桌上。我看见她的嘴还动着,嘴唇上有油,是吃了“回肉”吗?
\par 妈妈到床前来,吓唬着我说:“爸爸要打你了,玩病了还要吃。”
\par 我急了,说:
\par “我不是要吃饭,我今天根本一天没吃饭呀!就是问问你们吃饭了没有?我还有事呢!”
\par “鬼事!”妈妈把我又按着躺下,说:“身上还这样热,不知你烧到多少度了,吃完饭我去给你买药。”
\par “我不吃药,你给我药吃,我就跑走,你可别怪我!”
\par “瞎说!等一会儿宋妈吃完饭,叫她给你煮稀粥。”
\par 妈不理会我的话,她说完就又回外屋去吃饭了。我躺在床上,心里着急,想着和妞儿约会好吃完饭在横胡同口见面,不知她来了没有?细听外面又有淅淅沥沥的雨声,虽然不像白天那样大,可是横胡同里并没有可躲雨的地方,因为整条胡同都是人家的后墙。我急得胸口发痛,揉搓着,咳嗽了,一咳嗽,胸口就像许多针扎着那么痛。
\par 妈妈这时已经吃完饭,她和爸爸进来了。我的手按着嘴唇,是想用力压着别再咳嗽出来,但是手竟在嘴上发抖;我发抖,不是因为怕爸爸,我今天从下午起一直在抖;腿在抖,手也抖,心也抖,牙也抖。妈妈这时看见我发抖的样子,拿起我放在嘴唇上的手,说:
\par “烧得发抖了,我看还是你去请趟山本大夫吧!”
\par “不要!不要那个小日本儿!”
\par 爸爸这时也说:
\par “明天早晨再说吧,先用冰毛巾给她冰冰头管事的。我现在还要给老家写信,赶着明早发出去呢!”
\par 宋妈也进来看我了。她向妈妈出主意说:
\par “到菜市口西鹤年堂家买点小药,万应锭什么的,吃了睡个觉就好。”
\par 妈妈很听话,她向来就听爸爸的话,也听宋妈的话,所以她说:
\par “那好吆,宋妈,我们俩上街去买一趟。英子,乖乖地躺着,吃了药赶快好了好上学。等着,我还顺便到佛照楼给你带你爱吃的八珍梅回来。”
\par 现在,八珍梅并不能打动我了,我听妈和宋妈撑了伞走了,爸爸也到书房去了,我满心想着和妞儿的约会。她等急了吗?她会失望地回去了吗?
\par 我从被里爬出来,轻手轻脚地下了地,头很重,又咳嗽了,但是因为太紧张,这回并没有觉到胸口痛。我走到五屉橱的前面站住了,犹豫了一会儿,终于大胆地拉开了妈妈放衣服的那个抽屉,在最里面,最下面,是妈妈的首饰匣。妈妈开首饰匣只挑爸爸不在家的时候,她并不瞒我和宋妈的。
\par 首饰匣果然在衣服底下压着,我拿了出来打开,妈妈新打的那只金镯在里面!我心有点儿跳,要拿的时候,不免向窗外看了一眼,玻璃窗外黑漆漆的,没有人张望,但我可以照到自己的影子,我看见我怎样拿出金镯子,又怎样把首饰匣放回衣服底下,推合了抽屉,我的手是抖的。我要给秀贞她们做盘缠,妈妈说,二两金子值好多好多钱,可以到天津,到上海,到日本玩一趟,那么不是更可以够秀贞和妞儿到惠安去找思康三叔吗?这么一想,我觉得很有理,便很放心地把金镯子套在我的胳膊上面了。
\par 我再转过头,忽然看玻璃窗上,我的影子清楚了,不!吓了我一跳,原来是妞儿!她在向我招手,我赶快跑了出去,妞儿头发湿了,手上也有水,她小声对我说:
\par “我怕你真在横胡同等我,我吃完饭就偷偷跑出来了。我等了你一会儿,想着你不来了,我刚要回去,听见你妈跟宋妈过去了,好像说给谁买药去,我不放心你,来看看,你们家的大门倒是没闩上,我就进来了。”
\par “那咱们就去吧!”
\par “上哪儿去?就是你白天说的什么秀贞呀?”
\par 我笑着向她点了头。
\par “瞧你笑得怕人劲儿!你病糊涂了吧!”
\par “哪里!”我挺起胸脯来,立刻咳嗽了,赶快又弯下身子来才好些,我把手搭在她的肩上说:“你一去就知道了,她多惦记你啊!比着我的身子给你做了好些衣服。对了,妞儿,你心里想着你亲妈是什么样儿?”
\par “她呀,我心里常常想,她要思念我,也得像我这么瘦,脸是白白净净的……”
\par “是的,是的,你说得一点儿都没错儿。”我俩一边说着,一边向门外去,门洞黑乎乎的,我摸着开了门,有一阵风夹着雨吹进来,吹开了我的短褂子,肚皮上又凉又湿,我仍是对她说:
\par “你妈妈,她薄薄的嘴唇,一笑,眼底下就有两个泪坑,一哭,那眼睫毛又湿又长,她说:‘小英子,我千托万托你……'”
\par “嗯。”
\par “她说,小桂子可是我们俩的命根子呀!……”
\par “嗯。”
\par “她第一天见着我,就跟我说,见着小桂子,就叫她回来,饭不吃,衣服也不穿,就往外跑,急着找她爹去……”
\par “嗯。”
\par “她说,叫她回来,我们娘儿俩一块儿去,就说我不骂她……”
\par “嗯。”
\par 我们已经走到惠安馆门口了,妞儿听我说,一边“嗯,嗯”地答着,一边她就抽答着哭了,我搂着她,又说:
\par “她就是……”我想说疯子,停住了,因为我早就不肯称呼她是疯子了,我转了话口说:“人家都说她想你想疯啦!妞儿,你别哭,我们进去。”
\par 妞儿这时好像什么都不顾了,都要我给她做主意,她只是一边走,一边靠在我的肩头哭,她并没有注意这是什么地方。
\par 上了惠安馆的台阶,我轻轻地一推,那大门就开了。秀贞说,惠安馆的门,前半夜都不闩上,因为有的学生回来得很晚,一扇门用杠子顶住,那一半就虚关着。我轻声对妞儿说:
\par “别出声。”
\par 我们轻轻地,轻轻地走进去,经过门房的窗下,碰到了房檐下的水缸盖子,有了响,里面是秀贞的妈,问:
\par “谁呀?”
\par “我,小英子!”
\par “这孩子!黑了还要找秀贞,在跨院里呢!可别玩太晚了,听见没有?”
\par “嗯。”我答应着,搂着妞儿向跨院走去。
\par 我从没有黑天以后来这里,推开跨院的门,吱扭扭的一声响,像用一根针划过我的心,怎么那么不舒服!雨地里,我和妞儿迈步,我的脚碰着一个东西,我低头看是我早晨捉的那瓶吊死鬼,我拾起来,走到门边的时候,顺手把它放在窗台上。
\par 里屋点着灯,但不亮。我开开门,和妞儿进去,就站在通里屋的门边。我拉着妞儿的手,她的手也直抖。
\par 秀贞没理会我们进来,她又在床前整理那口箱子,背向着我们,她头也没回地说:
\par “妈,您不用催我,我就回屋睡去,我得先把思康的衣服收拾好呀!”
\par 秀贞以为进来的是她的妈妈,我听了也没答话,我不知道怎么办好了,我想说话,但抽了口气,话竟说不出口,只愣愣地看着秀贞的后背,辫子甩到前面去了,她常常喜欢这样,说是思康三叔喜欢她这样打扮,喜欢她用手指绕着辫梢玩的样子,也喜欢她用嘴咬辫梢想心事的样子。
\par 大概因为没有听我的答话吧?秀贞猛地回转身来“哟!”地喊了一声,“是你,英子,这一身水!”她跑过来,妞儿一下子躲到我身后去了。
\par 秀贞蹲下来,看见我身后的影子,她瞪大了眼睛,慢慢地,慢慢地,侧着头向我身后看,我的脖子后面吹过来一口一口的热气,是妞儿紧挨在我背后的缘故,她的热气一口比一口急,终于哇的一声哭出来,秀贞这时也哑着嗓子喊叫了一声:
\par “小桂子!是我苦命的小桂子!”
\par 秀贞把妞儿从我身后拉过去,搂起她,一下就坐在地上,搂着,亲着,摸着妞儿。妞儿傻了,哭着回头看我,我退后两步倚着门框,想要倒下去。
\par 秀贞好一会儿才松开妞儿,又急急地站起来,拉着妞儿到床前去,急急地说道:
\par “这一身湿,换衣服,咱们连夜地赶,准赶得上,听!”是静静的雨夜里传过来一声火车的汽笛声,尖得怕人。秀贞仰头听着想了一下又接着说:“八点五十有一趟车上天津,咱们再赶天津的大轮船,快快快!”
\par 秀贞从床上拿出包袱,打开来,里面全是妞儿,不,小桂子,不,妞儿的衣服。秀贞一件一件一件给妞儿穿上了好多件。秀贞做事那样快,那样急,我还是第一回看见。她又忙忙叨叨地从梳头匣子里取出了我送给小桂子的手表,上了上弦给妞儿戴上。妞儿随秀贞摆弄,但眼直望着秀贞的脸,一声也不响,好像变呆了。我的身子朝后一靠,胳膊碰着墙,才想起那只金镯子。我撩起袖子,从胳膊上把金镯子取下来,走到床前递给秀贞说:
\par “给你做盘缠。”
\par 秀贞毫不客气地接过去,立刻套在她的手腕上,也没说声谢谢,妈妈说人家给东西都要说谢谢的。
\par 秀贞忙了好一阵子,乱七八糟的东西塞了一箱子,然后提起箱子,拉着妞儿的手,忽然又放下来,对妞儿说道:“你还没叫我呢,叫我一声妈。”秀贞蹲下来,搂着妞儿,又扳过妞儿的头,撩开妞儿的小辫子看她的脖子后头,笑道:“可不是我那小桂子,叫呀!叫妈呀!”
\par 妞儿从进来还没说过一句话,她这时被秀贞搂着,问着,竟也伸出了两手,绕着秀贞的脖子,把脸贴在秀贞的脸上,轻轻而难为情地叫:
\par “妈!”
\par 我看见她们两个人的脸,变成一个脸,又分成两个脸,觉得眼花,立刻闭住眼扶住床栏,才站住了。我的脑筋糊涂了一会儿,没听见她们俩又说了什么,睁开眼,秀贞已经提起箱子了,她拉起妞儿的手,说:“走吧!”妞儿还有点认生,她总是看着我的行动,并伸出手来要我,我便和她也拉了手。
\par 我们轻手轻脚地走出去,外面的雨小些了,我最后一个出来,顺手又把窗台上的那瓶吊死鬼拿在手里。
\par 出了跨院门,顺着门房的廊檐下走,这么轻,脚底下也还是噗吱噗吱的有些声音。屋里秀贞的妈妈又说话了:
\par “是英子呀?还是回家去吧!赶明儿再来玩。”
\par “嗳。”我答应了。
\par 走出惠安馆的大门,街上漆黑一片,秀贞虽提着箱子拉着妞儿,但是她们竟走得那样快,秀贞还直说:
\par “快走,快走,赶不上火车了。”
\par 出了椿树胡同,我追不上她们了,手扶着墙,轻轻地喊:
\par “秀贞!秀贞!妞儿!妞儿!”
\par 远远的有一辆洋车过来了,车旁暗黄的小灯照着秀贞和妞儿的影子,她俩不顾我还在往前跑。秀贞听我喊,回过头来说:“英子,回家吧,我们到了就给你来信,回家吧!回家吧……”
\par 声音越细越小越远了,洋车过去,那一大一小的影儿又蒙在黑夜里。我扒着墙,支持着不让自己倒下去,雨水从人家房檐直落到我头上、脸上、身上,我还哑着嗓子喊:
\par “妞儿!妞儿!”
\par 我又冷,又怕,又舍不得,我哭了。
\par 这时洋车从我的身旁过去,我听车篷里有人在喊:
\par “英子,是咱们的英子,英子……”
\par 啊!是妈妈的声音!我哭喊着:
\par “妈啊!妈啊!”
\par 我一点力气没有了,我倒下去,倒下去,就什么都不知道了。



\subsubsection*{五}


\par 远远地,远远地,我听见一群家雀在叫,吱吱喳喳、吱吱喳喳。那声音越来越近了……不是家雀儿,是一个人,那声音就在我耳边。她说:
\par “……太太,您别着急了,自己的身子骨也要紧,大夫不是说了准保能醒过来吗?”
\par “可是她昏昏迷迷的有十天了!我怎么不着急!”
\par 我听出来了,这是宋妈和妈妈在说话。我想叫妈妈,但是嘴张不开,眼睛也睁不开,我的手,我的脚,我的身子,在什么地方哪?我怎么一动也不能动,也看不见自己一点点?
\par “这在俺们乡下,就叫中了邪气了。我刚又去前门关帝庙给烧了股香,您瞧,这包香灰,我带回来了,回头给她灌下去,好了您再上关帝庙给烧香还个愿去。”
\par 妈妈还在哭,宋妈又说:
\par “可也真怪事,她怎么一拐能拐了俩孩子走?咱们要是晚回来一步,咱们英子就追上去了,唉!越想越怕人,乖乖巧巧的妞儿!唉!那火车,俩人一块儿,唉!我就说妞儿长得俊倒是俊,就是有点薄相……”
\par “别说了,宋妈,我听一回,心惊一回。妞儿的衣服呢?”
\par “鸡笼子上扔的那两件吗?我给烧了。”
\par “在哪儿烧的?”
\par “我就在铁道旁边烧的。唉!挺俊的小姑娘!唉!”
\par “唉!”
\par 两个人唉声叹气的,停了一会儿没说话。
\par 等再听见茶匙搅着茶杯在响,宋妈又说话了:
\par “这就灌吧?”
\par “停一会儿,现在睡得挺好,等她翻身动弹时再说。——家里都收拾好了?”妈问。
\par “收拾好了,新房子真大,电灯今天也装好了,这回可方便喽!”
\par “搬了家比什么都强。”
\par “我说您都不听嘛!我说惠安馆房高墙高,咱们得在门口挂一个八卦镜照着它,你们都不信。”
\par “好了,不必谈了,反正现在已经离开那倒霉的地方就是了。等英子好了,什么也别跟她说,回到家,换了新地方,让她把过去的事儿全忘了才好,她要问什么,都装不知道,听见了没有?宋妈。”
\par “这您不用嘱咐,我也知道。”
\par 她们说的是什么,我全不明白,我在想,这是怎么回事儿?有什么事情不对了吗?我想着想着觉得自己在渐渐地升高,升高,我是躺在这里,高、高、高,鼻子要碰到屋顶了,“呀!”我浑身跳了一下,又从上面掉下来,一惊疑就睁开了眼睛。只听宋妈说:
\par “好了,醒了!”
\par 妈妈的眼睛又红又肿,宋妈也含着眼泪。但是我仍说不出话,不知怎么样才可以张开嘴。这时妈妈把我搂抱起来,捏住我的鼻子,我一张嘴,一匙水就一下给我灌了下去,我来不及反抗,就咽下了,然后我才喊:
\par “我不吃药!”
\par 宋妈对妈说:
\par “我说灵不是?我说关帝老爷灵验不是?喝下去立刻就会说话。”
\par 妈给我抹去嘴边的水,又把我弄躺下来。我这时才奇怪起来,看看白色的屋顶,白色的墙壁,白色的门窗和桌椅,这是什么地方?我记得我是在一个……我问妈妈说:
\par “妈,外面在下雨吗?”
\par “哪儿来的雨,是个大太阳天呀!”妈说。
\par 我还是愣愣地想,我要想出一件事情来。
\par 这时宋妈挨到我身边来,她很小心地问我:
\par “认得我吗?英子!”
\par 我点点头:“宋妈。”
\par 宋妈对妈笑笑。妈又说:
\par “你发烧病了十天了,爸爸和妈妈给你送到医院来住,等你好了,我们就回到新的家去,新的家还装了电灯呢!”
\par “新的家?”我很奇怪地问。
\par “新的家,是呀!我们的新家在新帘子胡同,记着,老师考你的时候,问你家住在哪儿?你就说,新——帘——子胡同。”
\par “那么……”有些事情我实在想不起来了,所以要说什么,也不能接下去,我就闭上眼睛。妈说:
\par “再睡会儿也好,你刚好还觉得累,是不是?”妈妈说着就摩抚我的嘴巴,我的眼皮,我的头发,忽然一个东西一下碰了我的头,疼了一下,我睁开眼看,是妈妈手上套的那只——那只金镯子!我不由得惊喊了一声:“镯子!”妈没说什么,把金镯子又推到手腕上去。我的眼睛直望着妈妈的金镯子,心想着,这只金镯子不是——不就是我给一个人的那只吗?那个人叫什么来着?我糊涂了,但不敢问,因为我现在不能把那件事情记得很清楚。我怎么就生病,就住到这医院里来了呢?我是一点儿也不清楚。
\par 妈妈拍拍我说:
\par “别发呆了,看你发烧睡大觉的时候,多少人给你送吃的、玩的东西来!”
\par 妈妈从床头的小桌上拿起来一个很好看的匣子,放在枕边,一边打开来,一边说:
\par “匣子是刘婆婆给你买的,留着装东西用,里面,喏,你看,这珠链子是张家三姨送你的。喏,这只自动铅笔是叔叔给你的。你自己玩吧!”她便转头跟宋妈说话去了。
\par 我随着妈妈的说明,一件件从匣里拿出来看,我再摸出来的是一只手表,上面镶了几颗钻,啊!这是我自己的东西!但是——我手举着表,一动也不动地看着,想着,它怎么会在这只匣子里?它不是,也被我送给人了吗?
\par “妈!”我不禁叫了一声,想问问。妈回过头看见,连忙接过表去,笑着说道:
\par “看,这只表我给你修理好了,你听!”
\par 妈把表挨近我的耳朵,果然发出小小滴答滴答的声音。然而这时我想起了一些事情,我想起了一个人,又一个人。她们的影子,在我眼前晃。
\par “妈!”我再叫一声还想问问。
\par 妈妈慌忙又从匣子里拿出别的玩意来哄我:
\par “喏,再看这个,是……”
\par 我忽然想起好些事情来了,我跟一个人,还有一个人的事情,但是妈妈为什么那样慌慌忙忙地不许人问?现在我是多么的思念她们!我心里太难受,真想哭,我忽然翻身伏在枕头上,就忍不住大声地哭起来。嘴里喊:“爸爸!爸爸!”
\par 妈妈和宋妈赶着来哄我,妈妈说:
\par “英子想爸爸了,爸爸知道多高兴,他下班就会来看你!”
\par 宋妈说:
\par “孩子委屈喽,孩子这回受大委屈喽!”
\par 妈妈把我抱起来搂着我,宋妈拍着我,她们全不懂得我!我是在想那两个人啊!我做了什么不对的事吗?我很怕!爸爸,爸爸,你是男人,你应当帮助我啊!我是为了这个才叫爸爸的。
\par 我哭了一阵子很累了,闭上眼睛偎在妈妈的怀里。妈妈轻轻摇着我,低声唱她的歌:
\par “天乌乌,要落雨,老公仔举锄头顺水路,顺着鲫仔鱼要娶某,龟举灯,鳖打鼓……”
\par 她又唱:
\par “饲阉鸡,阉鸡饲大只,刣给英子吃,英子吃不够,去后尾门仔眯眯哭!”那轻轻的摇动使我舒服多了,听到这里,我不由得睁开眼笑了。妈妈很高兴地亲着我的脸说:
\par “笑了,笑了,英子笑了。宋妈已经把家里的油鸡杀了给你煮汤喝呢!”
\par 宋妈从桌底下拿出一只小锅,打开来还冒着热气,她盛了一碗黄黄的汤还有几块肉,递到我面前,要我喝下去。我别过脸去不要看,不要吃。碗里是西厢房的小油鸡吗?我曾经摸着它们的黄黄软软的羽毛,曾经捉来绿色的吊死鬼喂它们,曾经有一个长长睫毛大眼睛里的泪滴落在它们的身上……我不说什么,把头钻进妈妈的胸怀里。妈妈说:
\par “她不想吃,再说吧,刚醒过来,是还没有胃口。”
\par 我在医院住了十几天,刚可以起床伏在楼窗口向下面看望,爸爸就雇来一辆马车,把我接回家。
\par 马车是敞篷的,一边是爸,一边是妈,我坐在中间,好神气。前面坐了两个赶马车的人,爸爸催他们快一点,皮鞭子抽在马身上,马蹄子嘚嘚嘚嘚,嘚嘚嘚嘚,一路跑下去。马车所经过的路,我全不认识。这条大街长又长,好像前面没尽没了。
\par 我觉得很新鲜,转身脸向着车后,跪在座位上,向街上呆呆地看。两边的树一棵棵地落在车后面,是车在走呢?是树在走呢?
\par 我仰起头来,望见了青蓝的天空,上面浮着一块白云彩,不,一条船。我记得她说:“那条船,慢慢儿地往天边上挪动,我仿佛上了船,心是飘的。”她现在在船上吗?往天边儿上去了吗?
\par 一阵小风吹散开我的前刘海,经过一棵树,忽然闻见了一阵香气,我回头看妈妈,心里想问:“妈,这是桂花香吗?”我没说出口,但是妈妈竟也嗅了嗅鼻子对爸说:
\par “这叫做马缨花,清香清香的!”她看我在看她,便又对我说:“小英子,还是坐下来吧,你这样跪着腿会疼,脸向后风也大。”
\par 我重新坐正,只好看赶马车的人狠心地抽打他的马。皮鞭子下去,那马身上会起一条条的青色的伤痕吗?像我在西厢房里,撩起一个人的袖子,看见她胳膊上的那样的伤痕吗?早晨的太阳,照到西厢房里,照到她那不太干净的脸上,那又湿又长的睫毛一闪动,眼泪就流过泪坑淌到嘴边了!我不要看那赶车人的皮鞭子!我闭上眼,用手蒙住了脸,只听那嘚嘚的马蹄声。
\par 太阳照在我身上,热得很,我快要睡着了,爸爸忽然用手指逗逗我的下巴说:
\par “那么爱说话的英子,怎么现在变得一句话都没有了呢?告诉爸,你在想什么?”
\par 这句话很伤了我的心吗?怎么一听爸说,我的眼皮就眨了两下,碰着我蒙在脸上的手掌,湿了,我更不敢放开我的手。
\par 妈妈这时一定在对爸爸使眼色吧?因为她说:
\par “我们小英子在想她将来的事呢!……”
\par “什么是将来的事?”从上了马车到现在,我这才说第一句话。
\par “将来的事就如英子要有新的家呀,新的朋友呀,新的学校呀……”
\par “从前的呢?”
\par “从前的事都过去了,没有意思了,英子都会慢慢忘记的。”
\par 我没有再答话,不由得在想西厢房的小油鸡,井窝子边闪过来的小红袄,笑时的泪坑,廊檐下的缸盖,跨院里的小屋,炕桌上的金鱼缸,墙上的胖娃娃,雨水中的奔跑……一切都算过去了吗?我将来会忘记吗?
\par “到了!到了!英子,新帘子胡同的新的家到了!快看!”
\par 新的家?妈妈刚说这是“将来”的事,怎么这样快就到眼前了?
\par 那么我就要放开蒙在脸上的手了。













\subsection{我们看海去}




\subsubsection*{一}

\par 妈妈说的,新帘子胡同像一把汤匙,我们家就住在靠近汤匙的底儿上,正是舀汤喝时碰到嘴唇的地方。于是爸爸就教训我,他绷着脸,瞪着眼说:
\par “讲唔听!喝汤不要出声,窣窣窣的,最不是女孩儿家相。舀汤时,汤匙也不要把碗碰得当当当地响……”
\par 我小心地拿着汤匙,轻轻慢慢地探进汤碗里,爸又发脾气了:
\par “小人家要等大人先舀过了再舀,不能上一个菜,你就先下手。”他又转过脸向妈妈:
\par “你平常对孩子全没教习也是不行的……”
\par 我心急得很,只想赶快吃了饭去到门口看方德成和刘平踢球玩,所以我就喝汤出了声,舀汤碰了碗,菜来先下手。我已经吃饱了,只好还坐在饭桌旁,等着给爸爸盛第二碗饭。爸爸说,不能什么都让佣人做,他这么大的人,在老家时,也还是吃完了饭仍站在一旁,听着爷爷的教训。
\par 我乘着给爸爸盛好饭,就溜开了饭桌,走向靠着窗前的书桌去,只听妈妈悄悄对爸爸说:
\par “也别把她管得这么严吧,孩子才多大?去年惠安馆的疯子把她吓得那么一大场病,到现在还有胆小的毛病,听见你大声骂她,她就一声不言语,她原来不是这样的孩子呀!现在搬到这里来,换了一个地方,忘记以前的事,又上学了,好容易脸上长胖些……”
\par 妈妈啊!你为什么又提起那件奇怪的事呢?你们又常常说,哪个是疯子,哪个是傻子,哪个是骗子,哪个是贼子,我分也分不清。就像我现在抬头看见窗外蓝色的天空上,飘着白色的云朵,就要想到国文书上第二十六课的那篇《我们看海去》:
\refdocument{
    \par 我们看海去!
    \par 我们看海去!
    \par 蓝色的大海上,
    \par 扬着白色的帆。
    \par 金红的太阳,
    \par 从海上升起来,
    \par 照到海面照到船头。
    \par 我们看海去!
    \par 我们看海去!
} 
\par 我就分不清天空和大海。金红的太阳,是从蓝色的大海升上来的呢?还是从蓝色的天空升上来的呢?但是我很喜欢念这课书,我一遍一遍地念,好像躺在船上,又像睡在云上。我现在已经能够背下来了,妈妈常对爸爸、对宋妈夸我用功,书念得好。我喜欢念的,当然就念得好,像上学期的“人手足刀尺狗牛羊一身二手……”那几课,我希望赶快忘掉它们!
\par 爸爸去睡午觉了,一家人都不许吵他,家里一点儿声音都没有,但是我听到街墙传来“嘭!嘭!”的声音,那准是方德成他们的皮球踢到墙上了。我在想,出去怎样跟他们说话,跟他们一起玩呢?在学校,我们女生是不跟男生说话的,理也不理他们,专门瞪他们,但是我现在很想踢球。
\par 好妈妈,她过来了:
\par “出去跟那两个野孩子说,不要在咱们家门口踢球,你爸爸睡觉呢!”
\par 有了这句话就好了,我飞快地向外跑,辫子又钩在门框的钉子上了,拔起我的头发根,痛死啦!这只钉子为什么不取掉?对了,是爸爸钉的,上面挂了一把鞋掸子,爸爸临出门和回家来,都先掸一掸鞋。他教我也要这样做,但是我觉得我鞋上的土,还是用跺脚的法子,跺得更干净些。
\par 宋妈在门道喂妹妹吃粥,她头上的簪子插着薄荷叶,太阳穴贴着小红萝卜皮,因为她在闹头痛的毛病。开街门的时候,宋妈问我:
\par “又哪儿疯去?”
\par “妈叫我出去的。”我理由充足地回答她。
\par 门外一块圆场地,全被太阳照着,就像盛得满满的一匙汤。我了不起地站到方德成的面前说:
\par “不许往我们家墙上踢球,我爸爸睡觉呢!”
\par 方德成从地上捡起皮球,傻呵呵地看着我。
\par 在我们家的斜对面,是一所空房子,里面没有人家住,只有一个看房的聋老头子,也还常常倒锁了街门到他的女儿家去住。宋妈不知从哪儿听来的,说这所房子总租不出去,是因为闹鬼。妈妈听了就跟爸爸说:“北京城怎么这么多闹鬼的房子?”
\par 在闹鬼房和另一所房的中间,有一块像一间房子那么大的空地,长满了草,前面也有看来我都能迈过去的矮破砖墙,里面的草长得比墙高。这块空地听说原来是闹鬼房子的马号,早就塌了,没有人修,就成一块空草地。
\par 我看着那片密密高高的草地,它旁边正接着一段闹鬼房子的墙,便对傻方德成他们说:
\par “不会上那边踢去,那房里没住人。”
\par 他们俩一听,转身就往对面跑去。球儿一脚一脚地踢到墙上又打回来,是多么的快活。
\par 这是条死胡同,做买卖的从汤匙的把儿进来,绕着汤匙底儿走一圈,就还得从原路出去。这时剃头挑子过来了,那两片铁夹子“唤头”弹得嗡嗡地响,也没人出来剃头。打糖锣的也来了,他的挑子上有酸枣面儿,有印花人儿,有山楂片,还有珠串子,是我最喜欢的,但是妈妈不给钱,又有什么办法!打糖锣的老头子看我站在他的挑子前,便轻轻对我说:
\par “去,去,回家要钱去!”
\par 教人要钱,这老头子真坏!我心里想着,便走开了。我不由得走向对面去,站在空草地的破砖墙前面,看方德成和刘平他们俩会不会叫我也参加踢球。球滚到我脚边来了,我赶快捡起来扔给他们。又滚到更远一点儿的墙边去了,我也跑过去替他们捡起来。这一次刘平一脚把球踢得老高老高的,他自己还夸嘴说:“瞧老子踢得多棒!”但是这回球从高处落到那片高草地里了。
\par “英子,你不是爱捡球吗?现在去给我们捡吧!”刘平一头汗地说。
\par 有什么不可以?我立刻就转身迈进破砖墙,脚踏在比我还高的草堆里。我用两手拨开草才想起,球掉到哪里了呢?怎么能一下就找到?不由得回头看他们,他们俩已经跑到打糖锣的挑子前,仰着脖子在喝那三大枚一瓶的汽水。
\par 我探身向草堆走了两步,是刘平的声音喊我:“留神脚底下狗屎,英子!”
\par 我听了吓得立刻停住了,向脚底下看看,还好,什么都没有。我拨开左面的草,右面的草,都找不到球。再向里走,快到最里面的墙角了,我脚下碰着一个东西,捡起来看,是把钳子,没有用,我把它往面前一丢,当的一声响了,我赶快又拨开面前的草,这才发现,钳子是落在一个铜盘子上面,盘子是反扣着的。真奇怪!我不由得蹲下来,掀开铜盘子,底下竟是叠得整整齐齐的一条很漂亮的带穗子的桌毯,和一件很讲究的绸衣服。我赶紧用铜盘子又盖住,心突突地跳,慌得很,好像我做了什么不对的事被人发现了,抬头看看,并没有人影,草被风吹得向前倒,打着我的头,我只看见草上面远远的那块蓝色的海,不,蓝色的天。
\par 我站起身来往出口的路走,心在想,要不要告诉刘平他们?我走出来,只见他们俩已经又在地上弹玻璃球了,打糖锣的老头子也走了。刘平头也没抬地问我:
\par “找着没有?”
\par “没有。”
\par “找不着算了,那里头也太脏,狗也进去拉屎,人也进去撒尿。”
\par 我离开他们回家去。宋妈正在院子里收衣服,她看见我便皱起眉头(小红萝卜皮立刻从太阳穴上掉下来了!)说:
\par “瞧裹得这身这脸的土!就跟那两个野小子踢球踢成这模样儿?”
\par “我没有踢球!”我的确没有踢球。
\par “骗谁!”宋妈撇嘴说着,又提起我的辫子,“你妈梳头是有名的手紧,瞧!还能让你玩散了呢!你说你够多淘!头绳儿哪?”
\par “是刚才那门上的钉子钩掉的。”我指着屋门那只挂鞋掸子的钉子争辩说。这时我低头看见我的鞋上也全是土,于是我在砖地上用力跺上几跺,土落下去不少。一抬头,看见妈妈隔着玻璃窗在屋里指点着我,我歪着头,皱起鼻子,向妈妈眯眯地笑了笑。她看见我这样笑,会原谅我的。



\subsubsection*{二}

\par 第二天,第三天,好几天过去了,方德成他们不再提起那个球,但是我可惦记着,我惦记的不是那个球,是那草地,草地里的那堆东西。我真想告诉妈或者宋妈,但是话到嘴边又收回去了。
\par 今天我的功课很快地就做完了,两位数的加法真难算,又要进位,又要加点,我只有十个手指头,加得忙不过来。算术算得太苦了,我就要背一遍“我们看海去”,我想,躺在那海中的白帆船上,会被太阳照得睁不开眼,船儿在水上摇呀摇的,我一定会睡着了。“我们看海去,我们看海去”,我收拾铅笔盒的时候,这样念着;我把书包挂在床栏上,这样念着;我跳出了屋门槛儿,这样念着。
\par 爸和妈正在院子里,妈妈抱着小妹妹,爸爸在剪花草;他说夹竹桃叶子太多了,花就开得少,该去掉一些叶子;又用细绳儿把枝子捆扎一下,那几棵夹竹桃,就不那么散散落落的了。他又给墙边的喇叭花牵上一条条的细绳子,钉在墙高处,早晨的太阳照在这堵墙上,喇叭花红紫黄蓝的全开开了,但现在不是早晨,几朵喇叭花已经萎了。
\par 妈妈对爸爸说:
\par “带把锁回来吧,贼闹得厉害,连新华街大街上还闹贼呢!”
\par 爸爸在专心剪裁花草,鼻孔一张一张的,他漫不经心地说:“新华街,离咱们这里还远呢!”然后抬头看见我,“是不是?英子!”
\par 我点点头,那空草地在我眼前闪了一下。
\par 小妹妹这时从妈妈的身上挣脱下来,她刚会走路,就喜欢我领她。我用跳舞的步子带着她走,小妹妹高兴死啦!咯咯地笑,我嘴里又念着“我们看海去”,念一句,跳一步舞,这样跳到门口。宋妈刚吃过饭,用她那银耳挖子在剔牙,每剔一下,就啧啧地吸着气,要剔好大的工夫,仿佛她的牙很重要!小妹妹抱住她的腿,她才把耳挖子在身上抹了抹,插到她的髻儿上去。
\par 宋妈抱起小妹妹走出街门了,她对妹妹说:“俺们逛街去喽!俺们逛街街去喽!”宋妈逛大街的瘾头很大,回来后就有许多新鲜事儿告诉妈妈:神妖贼怪,骡马驴牛。
\par 宋妈走远去了,小妹妹还在向我招手,天还没有黑,但是太阳不见了,只有对面空房子的墙角上,还有一丝丝光。再看过去,旁边的空草地上,也还有一片太阳闪着亮,草被风吹得轻轻地动,我看愣了,不由得向它走过去。我家隔壁的门前,停了一个收买破烂货的挑子,却不见人,大概是到谁家收买破烂去了吧!这时门前的空地上,一个人也没有。
\par 我走向空草地,一边迈过破墙,一边心里想,如果被宋妈或者什么人看见我到这里来的话,我就说,我要找那个皮球的,本来嘛!
\par 我没有专心找球,但也希望能看到它,我的脚步是走向那个神秘的墙角。我屏住气,拨动着高草,轻轻地向前探着脚步,我是怕又踩到什么东西。
\par 那些东西,能够还在这地方吗?我那天怎么不敢多看一看,立刻就返身退出来了呢?现在这些东西如果还在这地方的话,我又怎么办呢?当然没有办法,我只是想看一看,因为我喜欢奇怪的事。
\par 但是当我拨开那一丛草的时候,使我倒抽了一口气,惊奇地喊了一声:
\par “哦!”
\par 有一个人蹲在草地上!他也惊吓地回过头来“哦”了一声。瞪着眼望了我一阵,随后他笑了:
\par “小姑娘,你也上这儿来干吗?”
\par “我呀,”我竟答不出话来,愣了一下,终于想出来了,“我来找球。”
\par “球?是不是这个?”他说着,从身后的一堆东西里拿出一个皮球,果然是刘平他们丢的那个。我点点头,接过球来便转身退出去,但是他把我叫住了:
\par “嗯——小姑娘,你停停,咱们谈谈。”
\par 他是穿着一身短打裤褂,秃着头,浓浓的眉毛,他的厚嘴唇使我想起了会看相的李伯伯说过的话:“嘴唇厚厚敦敦的,是个老实人相。”我本来有点怕,想起这句话就好多了。他说话的声音仿佛有点发抖,人也不肯站起来,但是我知道他身后有一堆东西,不知道是不是那天的铜茶盘什么的。他说:
\par “小姑娘,你几岁啦?念书了没有?”
\par “七岁,在厂甸附小一年级。”常常有人问我同样的话,所以我能一下子就回答出来。
\par “喝!那是好学堂。谁接你送你上学呀?”
\par “我自己。”回答了以后,想起爸爸,所以我又说:“爸爸说,小孩子要早早养成自立的本事,现在,你知道不知道,新华街城墙打通了,叫做兴华门,我就不用绕顺治门啦!”
\par “小姑娘会说话,家教好,”他不住地点头。“你爸爸说得对,小孩子要早早地就学着自个儿,嗯——自个儿管自个儿的本事,唉——!”他忽然低头长长地叹一口气,又抬头望着我,笑笑问道,“你猜我是来干吗?”
\par “你呀我猜不出,”我摇摇头,但又忽然想起来了,“你是不是来这里拉屎?”
\par “拉屎?”他睁大了眼睛,“对啦,对啦,我是来出恭的啦!”
\par “不讲卫生!”
\par “我们这路人,没有卫生。”
\par 我又低头斜着眼望了一下他的背后,他好像在想什么,愣了一会儿,从短褂口袋里掏出了一把玻璃球,都是又圆又亮的汽水球:
\par “哪,这些个给你。”
\par “我不要!”这种事一点儿也不能坏我的心眼儿。爸爸说过,不许随便拿人家的东西。
\par “是我给你的呀!”他还是要塞到我手里,但是我的手掌努力张开着,并不拳起来,球没法落在我手里,就都掉在草地上了。我又说:
\par “人家给的也不能随便要。”
\par “这孩子!”他也很没有办法的样子,随后他又问我,“你们家知道你上这儿来吗?”
\par 我摇摇头。
\par “你回去要告诉你们家里的人看见我了吗?”
\par 我还是摇头。
\par “那好,可千万别跟人说看见我了呀!我也是好人。”
\par 谁又说他是坏人了呢?他的样子使我很奇怪!我猜想他不是来拉屎的,那堆东西,跟他有关系。
\par “回去吧!快黑了!”他指指天,乌鸦飞过去了。
\par “那你呢?”我问他。
\par “我也走呀,你先走。”他掸掸身上落下的碎草,好像要站起来,接着又说:“可别说出去呀,小姑娘,你还小,不懂事,等赶明儿,我跟你慢慢地谈,故事多着呢!”
\par “讲故事?”
\par “是呀!我常常来,我看你这小姑娘是好心肠,咱们交个道义朋友,我跟你讲我弟弟的故事儿呀,我的故事儿呀。”
\par “什么时候?”说到讲故事,我最喜欢。
\par “遇见了,咱们就聊聊,我一个人儿,也闷得慌。”
\par 他说的话,我不太懂,但是我觉得这样一个大朋友,可以交一交,我不知道他是好人,还是坏人,我分不清这些,就像我分不清海跟天一样,但是他的嘴唇是厚厚敦敦的。
\par 我转身向外拨动高草,又回过头来问他:
\par “明天你要来吗?”
\par “明天?不一定。”
\par 他正拿一个包袱摊开来包些东西,草下面很暗了,看不清,但是可以听见当当的声音,准是那个铜盘子碰着掉在地上的汽水球了。那些是他的东西吗?
\par 我走出了破砖墙,眼前这块地方还是没有人,但远远的我看见宋妈领着小妹妹回来了,我赶快向家里跑,路过隔壁的人家,看见那收破烂的挑子还摆在那里。
\par 我和宋妈同时到了家门口,便牵了小妹妹的手走进家门去,这时院子里的电灯亮了,电灯旁边的墙上爬着好几条蝎虎子,电灯上也飞绕着许多小虫儿。茶几已经摆在花池子旁边了,上面准是一壶香片茶,一包粉包烟,爸爸要在藤椅上躺好久好久,跟妈妈谈这谈那,李伯伯也许会来。
\par 我把皮球放在茶几上,随手便把粉包烟拿起来打开,抽出里面的洋画儿,爸爸笑笑问我:
\par “封神榜的洋画儿存完全了没有?”
\par “哪里会!那张姜子牙永远不会有。三只眼的杨戬我倒有三张啦!”
\par 爸爸摸摸我的头笑着对妈妈说:
\par “这孩子,也知道什么姜子牙啦,杨戬啦!”
\par 我也不知道是怎么个心气儿,忽然问爸爸:
\par “爸,什么叫做贼!”
\par “贼?”爸爸奇怪地望着我,“偷人东西的就叫贼。”
\par “贼是什么样子?”
\par “人的样子呀!一个鼻子俩眼睛。”妈回答着,她也奇怪地望着我:
\par “怎么问起这个来了?”
\par “随便问问!”
\par 我说着拿了小板凳来放在妈妈的脚下,还没坐下来呢,李伯伯也进来了,于是妈妈就赶我:
\par “去,屋里跟小妹妹玩去,不要在这里打岔。”




\subsubsection*{三}

\par 我洗脸的时候,把皮球也放在脸盆里用胰子洗了一遍,皮球是雪白的了,盆里的水可黑了。我把皮球收进书包里,这时宋妈走进来换洗脸水,她“哟”了一声,指着脸盆说:
\par “这是你的脸?多干净呀!”
\par “比你的臭小脚干净!”我说完扑哧笑了。我也不知为什么会想到宋妈的脚,大概是因为她的脚裹得太严紧了。妈妈说过,那里面是臭的。
\par 宋妈也笑了,她说:
\par “你嘴厉害不是?咬不动烧饼可别哭呀!”
\par 咬不动烧饼,实在是我每天早晨吃早点的一件痛苦的事。我的大牙都被虫蛀了,前面的又掉了两个,新的还没长出来,所以我就没法把烧饼麻花痛痛快快地吃下去。为了慢慢地吃早点,我迟到了;为了吃时碰到虫牙我痛得哭了。那么我就宁可什么也不吃,饿着肚子上学去。
\par 我把书包背挂在肩膀上,自己上学去。出了新帘子胡同照直向城门走去,兴华门虽然打通了,但是还没有做好,城门里外堆了一层层的砖土,车子不通行,只有人可以走过。早晨的太阳照在土坡上,我走上土坡,太阳就照满我的全身,我虽然没吃早点,但很舒服,就在土坡上站了一会儿,看着来来往往的行人。手扶着书包正碰着鼓起来的皮球,不由得想到了空草地里的情景,那个厚厚嘴唇的男人,他到底是干吗的?
\par 我呆想了一会儿,便走下土坡来,出了兴华门,马上就到学校了。
\par 五年级的童子军把着校门,他们的样子多凶啊!但是多让人羡慕啊!我几时能当上童子军呢?
\par “书包里是什么?”童子军指着我的书包问。
\par 我吓了一跳。
\par “是皮球,还给刘平的。”我说话都有点哆嗦了,我真怕他们。
\par 童子军对我很好,他没有检查,手一挥,放我进去了。我可看见他从别的同学的裤袋里查出蚕豆来,查出山楂糖来,全给没收了。不许带吃的。
\par 进了教室,我掏出皮球来给刘平,他愣着,大概忘了,我说:
\par “是你们那天丢的皮球呀!”
\par 他这才想起来,很高兴地接过去,也不说声谢谢。
\par 有一些同学们在吵吵闹闹,他们说,欢送毕业同学全校要开个游艺会,在大礼堂,每一班都要担任游艺会的一项表演节目,吵的就是我们这班会表演什么呢?我真奇怪,他们的消息是从哪里得来的?我怎么就不知道这些事情?
\par 在上课的时候,果然老师告诉我们,一二年级的同学不会表演整出的话剧什么的,只好唱唱歌,跳跳舞。教跳舞唱歌的韩老师要从一、二、三年级的同学里,挑出几个人来,合着演唱《麻雀与小孩》。啊!那是多么好听好看的一出歌舞啊!老师会选谁呢?会选我吗?我心跳了,因为我喜欢韩老师!她是我们附小韩主任的女儿。她冬天穿着一件藕荷色的旗袍,周身镶了白兔皮的边,在大礼堂里教我们跳舞,拉圈儿的时候,她刚好拉着我的手。她的手又热又软,我是多么喜欢她,她喜欢我吗?……
\par “……还有林英子,当小麻雀。”
\par 啊!我还在做梦呢,什么也没听见,什么?真的是在叫我的名字吗?
\par “林英子,从明天起,下了课要晚一点儿回家,每天都由韩老师教你们,到三甲的教室去,听明白了没有?记住,要告诉家里一声。”
\par 我只觉得脸热,真高兴死了,同学们会多么羡慕我啊!去跟三年级的大同学一起跳舞,虽然我当的是小小麻雀,只管飞来飞去,并不要唱什么。
\par 我觉得时间过得真慢,因为我要赶快回家告诉妈妈,不要告诉臭小脚宋妈,她一定会抱妹妹来看游艺会,我才不要她来!下课的时候,同学都围着我,问我跳舞那天穿什么衣裳?害怕不害怕?女同学都跑过来搂着我,好像我是她们每一个人的好朋友。
\par 好容易放学该回家吃午饭了,我加快了脚步,抢在同学的前面走出来。进了兴华门,过了高高低低的土坡,再走一小段路,就进新帘子胡同了。胡同里的第三家,是所大房子,平常大门关得严严的,今天却难得地敞开了,门口围着许多人,巡警也来了,不知道是什么事。但我下午还要上学,不能挤进人堆里去看,赶快跑回家来。
\par 宋妈正在气喘呼呼地跟妈讲什么,妈惊奇地瞪着眼听,又摇头,又啧啧。
\par “这回可大发了,偷了有三十件,八成是昨天天好拿出来晒衣服,让贼给瞄上了。”
\par “从外面怎么能看得见呢?不是黑大门的那家吗?我路过也难得看见他们打开门,总是阴森森的。”
\par “今天大门一敞开,咱们才看见,真是天棚石榴金鱼缸,院子可豁亮啦!”
\par “现在怎么样了呢?”
\par “巡警在那儿查呢!走,珠珠,咱们再看去,”宋妈领着小妹妹,回头看见了我,“小英子,你去不去看热闹?”
\par “热闹?人家丢了那么多东西,多着急呀,你还说是热闹呢!”我撇了她一嘴。
\par “好心没好报!”宋妈终于又抱着妹妹走了。
\par 我在饭桌上告诉妈妈,我参加表演《麻雀与小孩》的事,妈妈很高兴,她说要给我缝一件最漂亮的跳舞衣。
\par 我说:“缝好了就锁在箱子里,不要被贼偷走啊!”
\par “不会的,别说这丧话!”妈说。
\par 我忍不住又问妈:
\par “妈,贼偷了东西,他放在哪里去呢?”
\par “把那些东西卖给专收贼赃的人。”
\par “收贼赃的人什么样儿?”
\par “人都是一个样儿,谁脑门子上也没刻着哪个是贼,哪个又不是。”
\par “所以我不明白!”我心里正在纳闷儿一件事。
\par “你不明白的事情多着呢!上学去吧,我的洒丫头!”
\par 妈的北京话说得这么流利了,但是,我笑了:
\par “妈,是傻丫头,傻,‘ㄕㄚ’傻,不是‘ㄙㄚ’洒。我的洒妈妈!”说完我赶快跑走了。



\subsubsection*{四}

\par 因为放学后要练习跳舞,今天回来得晚一点儿。在兴华门的土坡上,我还是习惯地站了一会儿。城墙上面的那片天,是淡红的颜色了,海在这时也会变成红色的吗?我又默默地背起“我们看海去!我们看海去!……金红的太阳,从海上升起来……”那么现在不可以说是“金红的太阳,从天上落下去”吗?对的,我将来要写一本书,我要把天和海分清楚,我要把好人和坏人分清楚,我要把疯子和贼子分清楚,但是我现在却是什么也分不清。
\par 我从土坡上下来,边走边想,走到家门口,就在门墩儿上坐下来,愣愣地没有伸手去拍门,因为我看见收买破烂货的挑子又停在隔壁人家门口了。挑挑子的人呢?我不由得举起脚步走向空草地那边去。这时门前的空地上,只见远远地有一个男人蹲在大槐树底下,他没有注意我。我迈进破砖墙,拨开高草,一步步向里走。
\par 还是那个老地方,我看见了他!
\par “是你!”他也蹲在那里,嘴里咬着一根青草。他又向我身后张望了一下。招手叫我也蹲下来。我一蹲下来,书包就落在地上了。他小声地说:
\par “放学啦?”
\par “嗯。”
\par “怎么不回家?”
\par “我猜你在这里。”
\par “你怎么就能猜出来呢?”他斜起头看我,我看他的脸,很眼熟。
\par “我呀!”我笑笑。我只是心里觉得这样,就来了,我并不真的会猜什么事,“你该来了!”
\par “我该来了?你这话是什么意思?”他惊奇地问。
\par “没有什么意思呀!”我也惊奇地回答,“你还有故事没跟我讲哪!不是吗?”
\par “对对对,咱们得讲信用。”他点点头笑了。他靠坐在墙角,身旁有一大包东西,用油布包着,他就倚着这大包袱,好像宋妈坐在她的炕头上靠着被褥垛那样。
\par “你要听什么故事儿?”
\par “你弟弟的,你的。”
\par “好,可是我先问你,我还不知道你叫什么名儿呢?”
\par “英子。”
\par “英子,英子,”他轻轻地念着,“名儿好听。在学堂考第几?”
\par “第十二名。”
\par “这么聪明的学生才考十二名?应当考第一呀!准是贪玩分了你的心。”
\par 我笑了,他怎么知道我贪玩?我怎么能够不玩呢!
\par 他又接着说:
\par “我就是小时候贪玩,书也没念成,后悔也来不及了。我兄弟,那可是个好学生,年年考第一,有志气。他说,他长大毕了业,还要漂洋过海去念书。我的天老爷,就凭我这没出息的哥哥,什么能耐也没有,哪儿供得起呀!奔窝头,我们娘儿仨,还常常吃了上顿没下顿呢!唉!”他叹了口气,“走到这一步上,也是事非得已。小妹妹,明白我的话吗?”
\par 我似懂,又不懂,只是直着眼看他。他的眼角有一堆眼屎,眼睛红红的,好像昨天没睡觉,又像哭过似的。
\par “我那瞎老娘是为了我没出息哭瞎的,她现在就知道我把家当花光了,改邪归正做小买卖,她不知道我别的。我那一心啃书本的弟弟,更拿我当个好哥哥。可不是,我供弟弟念书,一心要供到让他漂洋过海去念书,我不是个好人吗?小英子,你说我是好人?坏人?嗯?”
\par 好人,坏人,这是我最没有办法分清楚的事,怎么他也来问我呢?我摇摇头。
\par “不是好人?”他瞪起眼,指着自己的鼻子。
\par 我还是摇摇头。
\par “不是坏人?”他笑了,眼泪从眼屎后面流出来。
\par “我不懂什么好人,坏人,人太多了,很难分。”我抬头看看天,忽然想起来了,“你分得清海跟天吗?我们有一课书,我念给你听。”
\par 我就背起“我们看海去”那课书,我一句一句慢慢地念,他斜着头仔细地听。我念一句,他点头“嗯”一声。念完了我说:
\par “金红的太阳是从蓝色的大海升上来的吗?可是它也从蓝色的天空升上来呀?我分不出海跟天,我分不出好人跟坏人。”
\par “对,”他点点头很赞成我,“小妹妹,你的头脑好,将来总有一天你分得清这些。将来,等我那兄弟要坐大轮船去外国念书的时候,咱们给他送行去,就可以看见大海了,看它跟天有什么不一样。”
\par “我们看海去!我们看海去!”我高兴得又念起来。
\par “对,我们看海去,我们看海去,蓝色的大海上,扬着白色的帆……还有什么太阳来着?”
\par “金红的太阳,从海上升起来……”
\par 我一句句教他念,他也很喜欢这课书了,他说:
\par “小妹妹,我一定忘不了你,我的心事跟别人没说过,就连我兄弟算上。”
\par 什么是他的心事呢?刚才他所说的话,都叫做心事吗?但是我并不完全懂,也懒得问。只是他的弟弟不知要好久才会坐轮船到外国去?不管怎么样,我们总算订了约会,订了“我们看海去”的约会。





\subsubsection*{五}


\par 妈妈那淡青色的头纱,借给我跳舞用。她在纱的四角各缀上一个小小铃儿,我把纱披在身上,再系在小拇指上,当做麻雀的翅膀。我的手一舞,铃儿就随着呤呤地响,好听极了。
\par 举行毕业典礼那天,同时也开欢送毕业同学会,爸妈都来了,坐在来宾席上,毕业同学坐在最前面,我们演员坐在他们后面。童子军维持秩序,神气死了,他们把童子军棍拦在礼堂的几个出入门口,不许这个进来,不许那个出去。典礼先开始了,韩主任发毕业证书,由考第一的同学代表去领取,那位同学上台领了以后,向韩主任鞠躬,转过身来又向台下大家一鞠躬,大家不住地鼓掌。我看这位领毕业文凭的同学很面熟,好像在哪里见过。唉!我真“洒”!每天在同一个学校里,当然我总会见过他的呀!
\par 我们唱欢送毕业同学的离别歌:“长亭外,古道边,芳草碧连天,问君此去几时来,来时莫徘徊……”我还不懂这歌词的意思,但是我唱时很想哭的,我不喜欢离别,虽然六年级的毕业同学我一个都不认识。
\par 轮到我们的“麻雀与小孩”上场了,我心里又高兴,又害怕,是我第一次登台,一场舞跳完,就像做梦一样,台下是什么样子,我一眼也不敢看,只听见嗡嗡嗡的还夹着鼓掌声。
\par 我下了台,来到爸妈的来宾席。妈妈给我买了大沙果,玉泉山的汽水和面包,我随便吃啦喝啦,童子军管不了喽!我并不愿意老老实实地坐在爸妈身边,便站起来,左看右看的,也为的让人家看看我就是刚才在台上的小麻雀。忽然,一晃眼,我看见一个熟悉的脸影,是坐在前边右面来宾席上的。他是?他侧过头来了,果然是他!我不知怎么,竟一下子蹲了下去,让前面的座位遮住我,我的脸好发烧,好像发生了什么事情。
\par 我低下头想,他怎么也来了?是不是来看我?在那青草丛里,我对他讲过学校要开游艺会和我要表演的事了吗?如果他不是来看我,又是来看谁的呢?
\par 我蹲在妈妈的脚旁太久,妈妈轻轻地踢了我一脚说:
\par “起来呀!你在找什么?”
\par 我从座位下站起身,挨着妈妈坐下来,低头轻轻地吃沙果,眼睛竟不敢向右前方看去。妈妈笑笑说:
\par “你不是说今天是特别日子,童子军不管同学吃零食的事吗?为什么还这么害怕?”
\par “谁说怕!”我把身子扭正过来。
\par 这个大沙果是很难吃完的,因为我的牙!我吃着沙果,一边看台上,一边想事。我想起来了,被我想起来了,他的弟弟!一定是他的考第一的弟弟在我们学校,就是考第一领毕业证书的那个!我差点儿喊出来,幸亏沙果堵在嘴上,我只能从鼻子里“哼——”了一声。
\par 游艺会仿佛很快地就闭幕了,我们都很舍不得地离开学校回家。回家来,我还直讲游艺会的事情,说了又说,说了又说,好像这一天的快乐,我永远永远都忘不了。爸爸很高兴,他说我这次期考竟进到十名以内了,要买点儿东西鼓励我,爸说:
\par “要继续努力啊!一年年地进步上去,到毕业的时候,要像今天那个考第一的代表同学那样领毕业证书。想一想,那位同学的爸爸坐在来宾席上,该是多么高兴呀!”
\par “他没有爸爸!”我突然这样喊出来,自己也惊奇了,他准是我所认为的那个人的弟弟吗?幸亏爸爸没有再问下去。但是这时却引起我要到一个地方去的念头。晚饭吃过了,天还不太晚,我溜出了家门。
\par 在门外乘凉的人很多,他们东一堆、西一堆地在说话,不会有人注意我。我假装不在意地走向空草地去。草长得更高,更茂盛了,拨开它,要用点力气呢!草里很暗,我不知道为什么要到这里来,也不知道他在不在,我只是一股子说不出的劲儿,就来了。
\par 他没有在这里,但是墙角可还有一个油布包袱,上面还压了两块石头。我很想把石头挪开,打开包袱看看,里面到底是些什么东西,但是我没有敢这么去做。我愣愣地看了一会儿,想了一会儿,眼睛竟湿了。我是想,夏天过去,秋天、冬天就会来了,他还会常常来这里吗?天气冷了怎么办?如果有一天,他的弟弟到外国去读书,那时他呢?还要到草地来吗?我蹲下来,让眼泪滴在草地上,我不知道为什么会这么伤心。我曾经有过一个朋友,人家说她是疯子,我却是喜欢她。现在这个人,人家又会管他叫什么呢?我很怕离别,将来会像那次离别疯子那样地和他离别吗?
\par 地上有一个东西闪着亮,我捡起来看,是一个小铜佛,我随便地把它拿在手里,就转身走出草地了。
\par 经过大槐树底下的时候,一个戴着草帽穿着对襟短褂的男人向我笑眯眯地走来,他说:
\par “小姑娘,你手里拿的是什么玩意儿呀?我看看行吗?”
\par 有什么不行呢,我立刻递给他。
\par “这是哪儿来的?你们家的吗?”
\par “不是,”我忽然想起这不是我家的东西,我怎么能随便拿在手里呢!于是我就指着空草地说:
\par “喏,那里捡来的。”
\par 他听了点点头,又笑眯眯地还给我,但是我不打算要了,因为回家去爸爸知道在外面捡东西也会骂的,我便用手一推,说:
\par “送给你吧!”
\par “谢谢你哟!”他真是和气,一定是个好人啦!


\subsubsection*{六}

\par 天气闷热,晚上蚊子咬得厉害,谁知半夜就下了一场大雨,一直下到大天亮。我们开完游艺会放三天假,三天以后再到学校去取作业题目,暑假就开始。今天不用上学了。
\par 雨把院子刷洗了一次,好干净!墙边的喇叭花被早晨的太阳一照,开得特别美。走到墙角,我忽然想起了另一个墙角。那个油布包袱,被雨冲坏了吗?还有他呢?
\par 我想到这儿,就忍不住跑出去,也不管会不会被别人看见。青草还是湿的,一拨开,水星全打到我的身上来,脸上来。
\par 他果然在里面!但他不是在游艺会上的样子了,昨天他端端正正地坐在礼堂里,腰板儿是直的,脖子是挺的。现在哪!他手上是水和泥,秃头上也是水珠子。他坐在什么东西上,两手支撑着下巴,厚厚的上嘴唇咬着厚厚的下嘴唇,看见我去了,也没有笑,他一定是在想他的心事,没有理会我。
\par 好一会儿,他才问我:
\par “小英子,我问你,你昨天有没有动过这包袱?”
\par 我摇摇头。斜头看那包袱,上面压着的石头没有了,包袱也不像昨天那样整齐。
\par “我想着也不是你,”他低下头自言自语地,“可是,要是你倒好了。”
\par “不是我!”我要起誓:“我搬不动那上面的石头。”我停了一下终于大胆地说道:“而且,昨天学校开游艺会,你也知道。”
\par “不错,我看见你了。”
\par 我笑笑,希望他夸我小麻雀演得好,但是他好像顾不得这些了,他拉过我的手,很难过地说道:
\par “这地方我不能久待了,你明白不?”
\par 我不明白,所以我直着眼望他,不点头,也不摇头。他又说:
\par “不要再到这儿找我了,咱们以后哪儿都能见着面,是不是?小妹妹,我忘不了你,又聪明,又伶俐,又厚道。咱们也是好朋友一场哪!这个给你,这回你可得收下了。”
\par 他从口袋掏出一串珠子,但是我不肯接过来。
\par “你放心,这是我自个儿的,奶奶给我的玩意儿多啦!全让我给败光了,就剩下这么一串小象牙佛珠,不知怎么,挂在镜框上,就始终没动过,今天本想着拿来送给你的,这是咱们有缘。小英子,记住,我可不是坏人呀!”
\par 他的话是诚实的,很动听,我就接过来了,绕两绕,套在我的手腕上。
\par 我还有许多话要跟他说呢,比如他的弟弟,昨天的游艺会,但是他扶着我的肩膀:
\par “回去吧,小英子,让我自个儿再仔细想想。这两天别再来了,外面风声仿佛——唉,仿佛不好呢!”
\par 我只好退出来了,我迈出破砖墙,不由得把珠串子推到胳膊上去,用袖子遮盖住,我是怕又碰见那个不认识的男人来要了去。

\subsubsection*{七}


\par 一天过去,两天过去,到了我到学校取暑假作业题目的日子了。
\par 美丽的韩老师正在操场上学骑车,那是一种时髦的事情呀!只有韩老师才这么赶时髦。她骑到我的面前停下了,笑笑对我说:
\par “来拿作业呀?”
\par 我点点头。
\par “暑假要快乐地过,下学期很快就开学了,那时候,你作业做好了,你的新牙也长出来了,兴华门也可以通车子了!”
\par 她的话多么好听,我笑了。但是想起牙,连忙捂住嘴,可是太好笑了,我的新牙虽然没有长出来,可也要笑,我就哈哈地大笑起来,韩老师也扶着车把大笑了。
\par 我和几个同路的同学一路回家,向兴华门走,土坡儿已经移开了许多,韩老师说得不错,下学期开学,一定可以有许多车辆打这里通过,韩老师当然也每天骑了车来上课啦。她骑在车上像仙女一样,我在路上见了她,一定向她招手说:“韩老师,早!”
\par 走进新帘子胡同,觉得今天特别热闹似的,人们来来往往的,好像在忙一件什么事。也有几个巡警向胡同里面走去。又是谁家丢了东西吗?我的心跳了,忽然觉得有什么不幸。
\par 越到胡同里面,人越多了。“走,看去!”“走,看去!”人们都这么说,到底是看什么呢?
\par 我也加紧了脚步,走到家门口时,看见家家的门都打开了,人们都站在门口张望,又好像在等什么,有的人就往空草地那面走去,大槐树底下也站满了人。
\par 我家门墩上被刘平和方德成站上去了。宋妈抱着珠珠也站在门口,妈妈可躲在大门里看,她这叫规矩。
\par “怎么啦,宋妈?”我扯扯宋妈的衣襟问。
\par “贼!逮住贼啦!”宋妈没看我,只管伸着脖子向前探望着。
\par “贼?”我的心一动,“在哪儿?”
\par “就出来,就出来,你看着呀!”
\par 人们嗡嗡地谈着,探着头。
\par “来啦!来啦!出来啦!”
\par 我的眼前被人群挡住了,只看见许多头在钻动。人们从草地那边拥着过来了。
\par “就是他呀!这不是收买破铜烂铁的那小子吗?”
\par 前面一个巡警手里捧着一个大包袱,啊!是那个油布包袱!那么这一定是逮住他了,我拉紧了宋妈的衣角。
\par “好嘛!”有人说话了。“他妈的,这倒方便,就在草堆里窝赃呀!”
\par “小子不是做贼的模样儿呀!人心大变啦!好人坏人看不出来啦!”
\par 一群人过来了,我很害怕,怕看见他,但是到底看见了,他的头低着,眼睛望着地下,手被白绳子捆上了,一个巡警牵着。我的手满是汗。
\par 在他的另一边,我又看见一个人,就是那个在槐树下向我要铜佛像的男人!他手里好像还拿着两个铜佛像。
\par “就是那个便衣儿破的案,他在这儿憋了好几天了。”有人说。
\par “哪个是便衣儿?”有人问。
\par “就是那戴草帽儿的呀!手里还拿着贼赃哪!说是一个小姑娘给点引的路才破了案……”
\par 我慢慢躲进大门里,依在妈妈的身边,很想哭。
\par 宋妈也抱着珠珠进来了,人们已经渐渐地散去,但还有的一直追下去看。妈妈说:
\par “小英子,看见这个坏人了没有?你不是喜欢做文章吗?将来你长大了,就把今天的事儿写一本书,说一说一个坏人怎么做了贼,又怎么落得这么个下场。”
\par “不!”我反抗妈妈这么教我!
\par 我将来长大了是要写一本书的,但绝不是像妈妈说的这么写。我要写的是:
\par “我们看海去”。













\subsection{兰姨娘}






\subsubsection*{一}


\par 从早上吃完点心起,我就和二妹分站在大门口左右两边的门墩儿上,等着看“出红差”的。这一阵子枪毙的人真多。除了土匪强盗以外,还有闹革命的男女学生。犯人还没出顺治门呢,这条大街上已挤满了等着看热闹的人。
\par 今天枪毙四个人,又是学生。学生和土匪同样是五花大绑在敞车上,但是他们的表情不同。要是土匪就热闹了,身上披着一道又一道从沿路绸缎庄要来的大红绸子,他们早喝醉了,嘴里喊着:
\par “过二十年又是一条好汉!”
\par “没关系,脑袋掉了碗大的疤瘌!”
\par “哥儿几个,给咱们来个好儿!”
\par 看热闹的人跟着就应一声:
\par “好!”
\par 是学生就不同了,他们总是低头不语,群众也起不了劲儿,只默默地拿怜悯的眼光看他们。我看今天又是枪毙学生,便想起这几天妈妈的忧愁,她前天才对爸爸说:
\par “这些日子,风声不好,你还留德先在家里住,他总是半夜从外面慌慌张张地跑来,怪吓人的。”
\par 爸爸不在乎,他伸长了脖子,用客家话反问了妈一句:
\par “惊么该?”
\par “别说咱们来往的客人多,就是自己家里的孩子、佣人也不少,总不太好吧?”
\par 爸爸还是满不在乎地说:
\par “你们女人懂什么?”
\par 我站在门墩儿上,看着一车又一车要送去枪毙的人,都是背了手不说话的大学生,不知怎么,便把爸妈所谈的德先叔联想起来了。
\par 德先叔是我们的同乡,在北京大学读书,住在沙滩附近的公寓里,去年开同乡会和爸认识的。爸很喜欢他,当做自己的弟弟一样。他能喝酒,爱说话,和爸很合得来,两个人只要一碟花生米,一盘羊头肉,四两烧刀子,就能谈到半夜。妈妈常在背地里用闽南话骂这个一坐下就不起身的客人:“长屁股!”
\par 半年以前的一天晚上,他慌慌张张地跑到我们家,跟爸爸用客家话谈着。总是为一件很要命的事吧,爸把他留在家里住下了。从此他就在我们家神出鬼没的,爸却说他是一个了不起的新青年。
\par 我是大姐,从我往下数,还有三个妹妹,一个弟弟,除了四妹还不会说话以外,我敢说我们几个人都不喜欢德先叔,因为他不理我们,这是第一个原因。还有就是他的脸太长,戴着大黑框眼镜,我们不喜欢这种脸。再就是,他来了,妈要倒霉,爸要妈添菜,还说妈烧不好客家菜,酿豆腐味儿淡啦!白斩鸡不够嫩啦!有一天妈高高兴兴烧了一道她自己的家乡菜,爸爸吃着明明是好,却对德先叔说:
\par “他们福佬人就知道烧五柳鱼!”
\par 凭了这些,我们也要站在妈妈这一头儿。德先叔每次来,我们对他都冷冷的,故意做出看不起他的样子,其实他并不注意。
\par 虽然这样,看着过“出红差”的,心里竟不安起来,仿佛这些要枪毙的学生,跟德先叔有什么关系似的,还没等过完,我便跑回家里问妈:
\par “妈!德先叔这几天怎么没来?”
\par “谁知道他死到哪儿去了!”妈很轻松地回答。停一下,她又奇怪地问我:“你问他干吗?不来不更好吗?”
\par “随便问问。”说完我就跑了,我仍跑回门外大街上去,刚才街上的景象全没有了,恢复了这条街每天上午的样子。卖切糕的,满身轻快地推着他的独轮车,上面是一块已经冷了的剩切糕,孤零零地插在一根竹签上。我八岁,两个门牙刚掉,卖切糕的问我买不买那块剩切糕,我摇摇头,他开玩笑说:
\par “对了,大小姐,你吃切糕不给钱,门牙都让人摘了去啦!”
\par 我使劲闭着嘴瞪他。
\par 到了黄昏,虎坊桥大街另是一种样子啦。对街新开了一家洋货店,门口坐满了晚饭后乘凉的大人小孩,正围着一个装了大喇叭的话匣子。放的是“百代公司特请谭鑫培老板唱《洪羊洞》”,唱片发出沙沙的声音,针头该换了。二妹说:
\par “大姐,咱们过去等着听《洋大人笑》去。”我们俩刚携起手跑,我又看见从对街那边,正有一队光头的人,向马路这边走来,他们穿着月白竹布褂,黑布鞋,是富连成科班要到广和楼去上夜戏。我对二妹说:
\par “看,什么来了?咱们还是回来数烂眼边儿吧!”
\par 我和二妹回到自己家门口,各骑在一个门墩儿上,静等着,队伍过来了,打头领队的个子高大,后面就是由小到大排下去。对街《洋大人笑》开始了,在“哈哈哈”的伴奏中,我每看队伍里过一个红烂着眼睛的孩子,便喊一声:
\par “烂眼边儿!”
\par 二妹说:“一个!”
\par 我再说:“烂眼边儿!”
\par 二妹说:“两个!”
\par 烂眼边儿,三个!烂眼边儿,四个!……今天共得十一个。富连成那些学戏的小孩子,比我们大不了多少,我们喊烂眼边儿,他们连头也不敢斜一斜,默默地向前走,大褂的袖子,老长老长,走起路来,甩搭甩搭的,都像傻子。
\par 我们正数得高兴,忽然一个人走近我的面前来,“嘿”的一声,吓我一跳,原来是施家的小哥,他也穿着月白竹布大褂。他很了不起地问我:
\par “英子,你爸妈在家吗?”
\par 我点点头。
\par 他朝门里走,我们也跟进去,问他什么事,他理也不理我们,我准知道他找爸妈有要紧的事。一进卧室的门,爸妈正在谈什么,看见小哥进来,他们仿佛愣了一下。小哥上前鞠躬,然后像背书一样地说:
\par “我爸叫我来跟林阿叔林阿婶说,如果我家兰姨娘来了,不要留她,因为我爸把她赶出去了。”
\par 这时妈走到通澡房的门口,我听见里面有哗啦哗啦的水声。爸爸点头说:
\par “好,好,回去告诉你爸爸,放心就是了。”
\par 小哥又一深鞠躬告退了,还是那么正正经经,看也不看我们一眼。小哥走后,爸爸窣窣窣地喝着香片茶,妈在点蚊香,两人都没说话。澡房的门打开了,呀!热气腾腾中,走出来的正是施家的兰姨娘!她是什么时候来的?她穿着一身外国麻纱的裤褂,走出来就平平衣襟,向后拢拢头发,笑眯眯地说:
\par “把在他们施家的一身晦气,都洗刷净啦!好痛快!”
\par 妈说:
\par “小哥刚才来了,你知道吧?”
\par “怎么不知道!”兰姨娘眉毛一挑,冷笑说:“说什么?他爸把我赶出来?怪不错的!我要走,大少奶奶还直说瞧她面子算了呢!这会儿又成了他赶我的喽!啧啧啧!”她的嘴直撇,然后又说:“别人留我不留,他也管得了?拦得住?走,秀子,跟我到前院去,叫你们家宋妈给我煮碗面吃。”说着她就拉着二妹的手走出去了。爸爸一直微笑地看着兰姨娘,伸长了脖子,脚下还打着拍子。
\par 妈脸上一点笑容都没有,兰姨娘出去了,她才站在桌子前,冲着爸的后背说:
\par “施大哥还特意打发小哥来说话,怎么办呢?”
\par “惊么该?”爸的脑袋挺着。
\par “怕什么?你总是招些惹事的人来!好容易这几天神出鬼没的德先没来,你又把人家下堂的姨奶奶留下了,施大哥知道了怎么说呢?”
\par “你平常跟她也不错,你好意思拒绝她吗?而且小哥迟来了一步,是她先进门的呀!”
\par 这时兰姨娘进来了,爸妈停止了争论,妈没好气地叫我:
\par “英子,到对门药铺给我买包豆蔻来,钱在抽屉里。”
\par “林太太,你怎么,又胃疼啦?林先生,准又是你给气的吧?”兰姨娘说完笑嘻嘻的。
\par 我从抽屉里拿了三大枚,心里想着:豆蔻嚼起来凉苏苏的,很有意思。兰姨娘在家里住下多么好!她可以常常带我到城南游艺园去,大戏场里是雪艳琴的《梅玉配》,文明戏场里是张笑影的《锯碗丁》,大鼓书场里是梳辫子的女人唱大鼓,还要吃小有天的冬菜包子。我一边跑出去,一边想,满眼都是那锣鼓喧天的欢乐场面。


\subsubsection*{二}

\par 兰姨娘在我们家住了一个礼拜了,家里到处都是她的语声笑影。爸上班去了,妈到广安市场买菜去了,她跟宋妈也有说有笑的。她把施家老伯伯骂个够,先从施伯伯的老模样儿说起,再说他的吝啬,他的刻薄,他的不通人情,然后又小声和宋妈说些什么,她们笑得吱吱喳喳的,奶妈高兴得眼泪都挤出来了。
\par 兰姨娘圆圆扁扁的脸儿,一排整整齐齐的白牙,我最喜欢她左边那颗镶金的牙,笑时左嘴角向上一斜,金牙便很合适地露出来。左嘴巴还有一处酒窝,随着笑声打漩儿。
\par 她的麻花髻梳得比妈的元宝髻俏皮多了,看她把头发拧成两股,一来二去就盘成一个髻,一排茉莉花总是清幽幽,半弯身地卧在那髻旁。她一身轻俏,掖在右襟上的麻纱手绢,一朵白菊花似的贴在那里。跟兰姨娘坐在一辆洋车上很舒服,她搂着我,连说:“往里靠,往里靠。”不像妈,黑花丝葛的裙子里,年年都装着一个大肚子。跟妈坐一辆洋车,她的大肚子把我顶得不好受,她还直说:“别挤我行不行!”现在妈又大肚子要生第六个孩子了。
\par 有了兰姨娘,妈做家事倒也不寂寞,她跟妈有诉说不尽的心事,奶妈,张妈,都喜欢靠拢来听,我也“小鱼上大串儿”地挤在大人堆里,仰头望着兰姨娘那张有表情的脸。她问妈说:
\par “林太太,你生英子十几岁?”
\par “才十六岁。”妈说。
\par 兰姨娘笑了:
\par “我开怀也只十六岁。”
\par “什么开怀?”我急着问。
\par “小孩子别乱插嘴!”妈叱责我,又向兰姨娘说,“当着孩子说话要小心,英子鬼着呢,会出去乱说。”
\par 兰姨娘叹了口气:
\par “我十四岁从苏州被人带进了北京,十六岁那什么,四年见识了不少人,二十岁到底还是跟了施大这个老鬼……”
\par “施大哥今年到底高寿了?”妈打岔问。
\par “管他多大!六十,七十,八十,反正老了,老得很!”
\par “我记得他是六十——六十几来着?”妈还是追问。
\par “他呀,”兰姨娘扑哧笑了,看看我,“跟英子一般大,减去一周甲子,才八岁!”
\par “你倒也跟了他五年了,你今年不是二十五岁了么?”
\par “别看他六十八岁了,硬朗着呢!再过下去,我熬不过他,他们一家人对付我一个人,我还有几个五年好活!我不愿把年轻的日子埋在他们家。可是,四海茫茫,我出来了,又该怎么样呢?我又没有亲人,苏州城里倒有一个三岁就把我卖了的亲娘,她住在哪条街上,我也记不得了呀!就记得那屋里有一盏油灯,照着躺在床上的哥哥,他病了,我娘坐在床边哭,应该就是为了这病哥哥才把我卖的吧!想起来梦似的,也不知道是我乱想的,还是真的……”
\par 兰姨娘说着,眼里闪着泪光,是她不愿意哭出来吧,嘴上还勉强笑着。
\par 妈不会说话,笨嘴拙舌的,也不劝劝兰姨娘。我想到去年七月半在北海看烧法船的时候,在人群里跟妈妈撒开了手,还急得大哭呢,一个人怎么能没有妈?三岁就没了妈,我也要哭了,我说:
\par “兰姨娘,就在我们家住下,我爸爸就爱留人住下,空房好几间呢!”
\par “乖孩子,好心肠,明天书念好了当女校长去,别嫁人,天底下男人没好的!要是你爸妈愿意,我就跟你们家住一辈子,让我拜你妈当姐姐,问她愿意不愿意?”兰姨娘笑着说。
\par “妈愿意吧?”我真的问了。
\par “愿——意呀!”妈的声音好像在醋里泡过,怎么这么酸!
\par 我可是很开心,如果兰姨娘能够好久好久地停留在我们家的话。她怎么也说我要当女校长呢?有一次,我站在对街的测字摊旁看热闹,测字的先生忽然从他的后领里抽出一把折扇,指着我对那些要算命的人说:“看见没有?这个小姑娘赶明儿能当女校长,她的鼻子又高又直,主意大着呢!有男人气。”兰姨娘的话,测字先生的话,让人听了都舒服得很,使我觉得自己很了不起。
\par 爸对兰姨娘也不错,那天我跟着爸妈到瑞蚨祥去买衣料,妈高高兴兴地为我和弟弟妹妹们挑选了一些衣料之后,爸忽然对我说:
\par “英子,你再挑一件给你兰姨娘,你知道她喜欢什么颜色的吗?”
\par “知道知道,”我兴奋得很,“她喜欢一件蛋青色的印度绸,镶上一道黑边儿,再压一道白芽儿……”我比手画脚说得高兴,一回头看见坐在玻璃柜旁的妈,妈正皱着眉头在瞪我。伙计早把深深浅浅的绸子捧来好几匹,爸挑了一色最浅的,低声下气地递到妈面前说:
\par “你看看这料子还好吗?是真丝的吗?”
\par 妈绷住脸,抓起那匹布的一端,大把地一攥,拳头紧紧的,像要把谁攥死。手松开来,那团绸子也慢慢散开,满是皱痕,妈说:
\par “你看好就买吧,我不懂!”
\par 我也真不懂妈为什么忽然跟爸生气,直到有一天,在那云烟缭绕的鸦片烟香中,我才也闻出那味道的不对。
\par 那个做九六公债的胡伯伯,常来我家打牌,他有一套烟具摆在我们家,爸爸有时也躺在那里陪胡伯伯玩两口。
\par 兰姨娘很会烧烟,因为施伯伯也是抽大烟的。是要吃晚饭的时候了,爸和兰姨娘横躺在床上,面对面,枕着荷叶边的绣花枕头,上面是妈绣的拉锁牡丹花,中间那份烟具我很喜欢,像爸给我从日本带回来的一盒玩具。白铜烟盘里摆着小巧的烟灯,冒着青黄的火苗,兰姨娘用一只银签子从一个洋钱形的银盒里挑出一撮烟膏,在烟灯上烧得嗞嗞地响,然后把烟泡在她那红红的掌心上滚滚,就这么来回烧着滚着,烧好了插在烟枪上,把银签子抽出来,中间正是个小洞口。烟枪递给爸,爸嘬着嘴,对着灯火窣窣地抽着。我坐在小板凳上看兰姨娘的手看愣了,那烧烟的手法,真是熟巧。忽然,在喷云吐雾里,兰姨娘的手,被爸一把捉住了,爸说:
\par “你这是朱砂手,可有福气呢!”
\par 兰姨娘用另一只手把爸的手甩打了一下,抽回手去,笑瞪着爸爸:
\par “别胡闹!没看见孩子?”
\par 爸也许真的忘记我在屋里了,他侧抬起头,冲我不自然地一笑,爸的那副嘴脸!我打了一个冷战,不知怎么,立刻想到妈。我站起来,掀起布帘子,走出卧室,往外院的厨房跑去。我不知道为什么要在这时候找母亲。跑到厨房,我喊了一声:“妈!”背手倚着门框。
\par 妈站在大炉灶前,头上满是汗,脸通红,她的肚子太大了,向外挺着,挺得像要把肚子送给人!锅里油热了,冒着烟,她把菜倒在锅里,才回过头来不耐烦地问我:“干吗?”我回答不出,直着眼看妈的脸。她急了,又催我:“说话呀!”
\par 我被逼得找话说,看她呱呱呱地用铲子敲着锅底,把炒熟的菜装在盘子里,那手法也是熟巧的,我只好说:
\par “我饿了,妈。”
\par 妈完全不知道刚才的那一幕使我多么同情她,她只是骂我:
\par “你急什么?吃了要去赴死吗?”她扬起锅铲赶我,“去去去,热得很,别在我这儿捣乱!”
\par 在我的泪眼中,妈妈的形象模糊了,我终于“哇”的一声哭了出来。宋妈把我一把拉出厨房,她说什么?“一点都不知道心疼你妈,看这么热天,这么大肚子!”
\par 我听了跳起脚尖哭。
\par 兰姨娘也从里院跑出来,她说:
\par “刚才不是还好好的吗?这会儿工夫怎么又捣乱捣到厨房来啦!”
\par 妈说:
\par “去叫她爸爸来揍她!”
\par 天快黑了,我被围在家中女人们的中间,她们越叫我吃饭,我越伤心;她们越说我不懂事,我越哭得厉害。
\par 在杂乱中,我忽然看见一个白色的影子从我身旁擦过,是——是多日不见的德先叔,他连看都不看我一眼,直往里院走。看着他那轻飘飘白绸子长衫的背影,我咬起牙,恨一切在我眼前的人,包括德先叔在内。

\subsubsection*{三}

\par 第二天早晨,我是全家最迟起来的人,醒来我还闭着眼睛想,早点是不是应当继续绝食下去?昨天抽大烟闹朱砂手的事,给我的不安还没有解开,她使我想到几件事:我记得妈跟别人说过,爸爸在日本吃花酒,一家挨一家,吃一整条街,从天黑吃到天亮。妈就在家里守到天亮,等着一个醉了的丈夫回来。我又记得我们住在城里时,每次到城南游艺园听夜戏回来,车子从胭脂胡同韩家潭穿过时,宋妈总会把我从睡梦中推醒:“醒醒,醒醒,大小姐!看,多亮!”我睁开眼,原来正经过辉煌光亮的胡同,各家门前挂着围了小电灯扎彩的镜框,上面写着什么“弟弟”“黛玉”“绿琴”等等字样,奶妈跟我说过,兰姨娘没到施伯伯家,也是在这种地方住。她们是刮男人的钱、毁男人的家的坏东西!因为这样,所以一看到爸和兰姨娘那样的事,觉得使妈受了委屈,使我们都受了委屈。把原来喜欢兰姨娘的心,打了大大的折扣,我又恨,又怕。
\par 我起床了,要到前院去,经过厢房时,一晃眼看见兰姨娘正在墙前的桌上摸骨牌,玩她的过五关斩六将,我装着没看见,直走过去,因为心中还恨恨的。
\par “英子!”兰姨娘隔着窗子在叫我。
\par 我不得不进屋了,兰姨娘推开桌上的骨牌,站起来拉着我的手,温柔地说:
\par “看你这孩子,昨天一晚上把眼睛都哭肿了,饭也没吃。”她抚摩着我的头发,我绷着劲儿,一点笑容都没有。她又说:
\par “别难过,后天就是七月十五了,你要提什么样的莲花灯,兰姨娘给你买。”
\par 我摇摇头,她又自管自地接着说:
\par “你不是说要特别花样的吗?我帮你做个西瓜灯,好\UncommonChar{𠲎}?要把瓜吃空了,皮削脱,剩薄薄格一层瓤子,里面点上灯,透明格,蛮有趣。”
\par 兰姨娘话说多了,就不由得带了她家乡的口音,轻轻软软,多么好听!我被她说得回心转意了,点点头。
\par 她见我答应了也很高兴,忽然又闲话问我:
\par “昨天跟你爸瞎三话四,讲到半夜的那只四眼狗是什么人?”
\par “四眼狗?”我不懂。
\par 兰姨娘淘气地笑了,她用手掌从脸上向下一抹,手指弯成两个圈,往眼上一比:
\par “喏!就是这个人呀!”
\par “啊——那是我德先叔。”
\par 这时,不知是什么心情,忽然使我站在德先叔这一边了,我有意把德先叔叫得亲热些,并且说:
\par “他是很有学问的,所以要戴眼镜。他在北京大学念书,爸说,他是顶、顶、顶新的新青年,很了不起!”我挑着大拇指说,很有把兰姨娘卑贱的身份硬压下去的意思。
\par “原来是大学生呀!”兰姨娘倒也缓和了,“那么就是你妈说过,常住在你们家躲风声的那个大学生喽?”
\par “是。”
\par “好,”兰姨娘点点头笑说,“你爸爸的心蛮好的,三六九等的人都留下了。”
\par 我从兰姨娘的屋里出来,就不由得往前院德先叔住的南屋走去。我有权利去,因为南屋书桌抽屉里放着我的功课,我的小布人儿,我的《儿童世界》,德先叔正占用那书桌,我走进去就不客气地拉开书桌抽屉,翻这翻那,毫无目的。他被我在他身旁闹得低下头来看。我说:“我的小刀呢?剪子呢?兰姨娘要给我做西瓜灯哪!”
\par “那个兰姨娘是你家什么人?我以前怎么没见过?”我多么高兴兰姨娘引起他的注意了。
\par “德先叔,你说那个兰姨娘好看不好看?”
\par “我不知道,我没看清楚。”
\par “她可看清楚你了,她说,你的眼睛很神气,戴着眼镜很有学问。”我想到“四眼狗”,简直不敢正眼朝他脸上看,只听见他说:
\par “哦?——哦?”
\par 吃午饭的时候,德先叔的话更多了,他不那样旁若无人地总对爸一个人说话了,也不时转过头向兰姨娘表示征求意见的样子,但是兰姨娘只顾给我夹菜,根本不留神他。
\par 下午,我又溜到兰姨娘的屋里。我找个机会对兰姨娘说:
\par “德先叔夸你哩!”
\par “夸我?夸我什么呀?”
\par “我早上到书房去找剪刀,他跟我说:‘你那个兰姨娘,很不错呀!'”
\par “哟!”兰姨娘抿着嘴笑了,“他还说什么?”
\par “他说——他说,他说你像他的一个女同学。”我瞎说。
\par “那——人家是大学堂的,我怎么比得了!”
\par 晚饭桌上,兰姨娘就笑眯眯的了,跟德先叔也搭搭话。爸更高兴,他说:
\par “我这人就是喜欢帮助落难的朋友,别人不敢答应的事,我不怕!”说着,他就拍拍胸脯。爸酒喝得够多,眼睛都红了,笑嘻嘻乜斜着眼看兰姨娘。妈的脸色好难看,站起来去倒茶,我的心又冷又怕,好像我和妈妈要被丢在荒野里。
\par 我整日守着兰姨娘,不让她有一点机会跟爸单独在一起。德先叔这次住在我们家倒是少出去,整日待在屋里发愣,要不就在院子里晃来晃去的。
\par 七月十五日的下午,兰姨娘的西瓜灯完成了。一吃过晚饭,天还没有黑,我就催着兰姨娘、宋妈,还有二妹,点上自己的灯到街上去,也逛别人的灯。临走的时候,我跑到德先叔的屋里,我说:
\par “我和兰姨娘去逛莲花灯,您去不去?我们在京华印书馆大楼底下等您!”说完我就跑了。
\par 行人道上挤满了提灯和逛灯的人,我的西瓜灯很新鲜,很引人注意。但是不久我们就和宋妈、二妹她们走散了,我牵着兰姨娘的手,一直往西去,到了京华印书馆的楼前停下了,我假装找失散的宋妈她们,其实是在盼望德先叔。我在附近东张西望一阵没看见,便失望地回到楼前来,谁知德先叔已经来了,他正笑眯眯地跟兰姨娘点头,兰姨娘有点不好意思,也点头微笑着。德先叔说:
\par “密斯黄,对于民间风俗很有兴趣。”
\par 兰姨娘仿佛很吃惊,不自然地说:
\par “哪里,哄哄孩子!您,您怎么知道我姓黄?”
\par 我想兰姨娘从来没有被人叫过“密斯黄”吧,我知道,人家没结过婚的女学生才叫“密斯”,兰姨娘倒也配!我不禁撇了一下嘴,心里真不服气,虽然我一心想把兰姨娘跟德先叔拉在一起。
\par “我听林太太讲起过,说密斯黄是一位很有志气的,敢向恶劣环境反抗的女性!”德先叔这么说就是了,我不信妈这样说过,妈根本不会说这样的话。
\par 这一晚上,我提着灯,兰姨娘一手紧紧地按在我的肩头上,倒像是我在领着一个瞎子走夜路。我们一路慢慢走着,德先叔和兰姨娘中间隔着一个我,他们在低低地谈着,兰姨娘一笑就用小手绢捂着嘴。
\par 第二天我再到德先叔屋里去,他跟我有的是话说了,他问我:
\par “你兰姨娘都看些什么书,你知道吗?”
\par “她正在看《二度梅》,你看过没有?”
\par 德先叔难得向我笑笑,摇摇头,他从书堆里翻出一本书递给我说:“拿去给她看吧。”
\par 我接过来一看,书面上印着:《易卜生戏剧集:傀儡家庭》。
\par 第三天,我给他们传递了一次纸条。第四天我们三个人去看了一次电影,我看不懂,但是兰姨娘看了当时就哭得欷欷的,德先叔递给她手绢擦,那电影是李丽吉舒主演的《二孤女》。第五天我们走得更远,到了三贝子花园。
\par 从三贝子花园回来,我兴奋得不得了,恨不得飞回家,飞到妈的身边告诉她,我在三贝子花园畅观楼里照哈哈镜玩时,怎样一回头看见兰姨娘和德先叔手拉手,那副肉麻相!而且我还要把全部告诉妈!但是回到家里,卧室的门关了,宋妈不许我进去,她说:
\par “你妈给你又生了小妹妹!”
\par 直到第二天,我才溜进去看,小妹妹瘦得很,白苍苍的小手,像鸡爪子,可是那接生的产婆山田太太直夸赞,她来给妹妹洗澡,一打开小被包,露出妹妹的鸡爪子,她就用日本话拉长了声说:“可爱亻礻!可爱亻礻!(可爱呀!可爱呀!)”
\par 妈端着一碗香喷喷的鸡酒煮挂面,望着澡盆里的小肉体微笑着。她没注意我正在床前的小茶几旁打转。我很喜欢妈生小孩子,因为可以跟着揩油吃些什么,小几上总有鸡酒啦,奶粉啦,黑糖水啦,我无所不好。但是我今天更兴奋的是,心里搁着一件事,简直是非告诉她不可啦!
\par 妈一眼看见我了:
\par “我好像好几天没看见你了,你在忙什么呢?这么热的天,野跑到哪儿去了?”
\par “我一直在家里,您不信问兰姨娘好了。”
\par “昨天呢?”
\par “昨天——”我也学会了鬼鬼祟祟,挤到妈床前,小声说:“兰姨娘没告诉您吗?我们到三贝子花园去了。妈,收票的大高人,好像更高了,我们三个人还跟他合照了一张相呢,我只到那人这里……”
\par “三个人?还有一个是谁?”
\par “您猜。”
\par “左不是你爸爸!”
\par “您猜错了,”看妈的一副苦相,我想笑,我不慌不忙地学着兰姨娘,用手掌从脸上向下一抹,然后用手指弯成两个圈往眼上一比,我说:
\par “喏!就是这个人呀!”
\par 妈皱起眉头在猜:
\par “这是谁?难道?难道是?——”
\par “是德先叔。”我得意地摇晃着身体,并且拍拍我的新妹妹的小被包。
\par “真的?”妈的苦相没了,又换了一副急相,“到底是怎么回事?你说,你从头说。”
\par 我从四眼狗讲到哈哈镜,妈出神地听我说着,她怀中的瘦鸡妹妹早就睡着了,她还在摇着。
\par “都是你一个人捣的鬼!”妈好像责备我,可是她笑得那么好看。
\par “妈,”我有好大的委屈,“您那天还要叫爸揍我呢!”
\par “对了,这些事你爸知道不?”
\par “要告诉他吗?”
\par “这样也好,”妈没理我,她低头呆想什么,微笑着自言自语地说。然后她又好像想起了什么,抬起头来对我说:
\par “你那天说要买什么来着?”
\par “一副滚铁环,一双皮鞋,现在我还要加上订一整年的《儿童世界》。”我毫不迟疑地说。

\subsubsection*{四}


\par 爸正在院子里浇花,这是他每天的功课,下班回家后,他换了衣服,总要到花池子花盆前摆弄好一阵子。那几盆石榴,春天爸给施了肥,满院子麻渣臭味,到五月,火红的花朵开了,现在中秋了,肥硕的大石榴都咧开了嘴向爸笑!但是今天爸并不高兴,他站在花前发呆。我看爸瘦瘦高高,穿着白纺绸裤褂的身子,晃晃荡荡的,显得格外的寂寞,他从来没有这样过。
\par 宋妈正在开饭,她一趟趟地往饭厅里运碗运盘,今天的菜很丰富,是给德先叔和兰姨娘送行。
\par 我正在屋里写最后的大字。今年暑假过得很快乐,很新奇,可是暑假作业全丢下没有做,这个暑假没有人管我了。兰姨娘最初还催我写九宫格,后来她只顾得看《傀儡家庭》了,就懒得理我的功课。九宫格里填满了我的潦草的墨迹,一张又一张的,我不像是写字,比鬼画符还难看。我从窗子正看到爸的白色的背影,不由得停下了笔,不知怎么,心里觉得很对不起爸。
\par 我很纳闷儿,德先叔和兰姨娘是怎么跟爸提起他们要一起走的事呢?我昨天晚上要睡觉时一进屋,只听到爸对妈说:
\par “……我怎么一点儿都不知道?”
\par 我不知道爸说的是什么事,所以起初没注意,一边换衣服一边想我自己的事:还有两天就开学了,明天可该把大字补写出来了,可是一张九个字,十张九十个字,四十张三百六十个字,让我怎么赶呀!还是求求兰姨娘给帮忙吧。这时又听见妈说:
\par “这种事怎么能叫你知道了去!哼!”妈冷笑了一下。
\par “那么你知道?”
\par “我?我也不知道呀,德先是怎么跟你提起的?”
\par “他先是说,这些日子风声又紧了,他必得离开北京,他打算先到天津看看,再坐船到上海去。随后他又说:‘我有一件事要告诉大哥的,密斯黄预备和我一齐走。'……”我这时才明白是讲的什么事,好奇地仔细听下去。
\par “哼!你听德先讲了还不吃一惊!”妈说。
\par “惊么该!”爸不服气,“不过出乎意料就是了,你真一点都不知道,一点都没看出来?”
\par “我从哪儿知道呢?”妈简直瞎说!停了一下妈又说:“平常倒也仿佛看出有那么点儿意思。”
\par “那为什么不跟我说?”
\par “哟!跟你说,难道你还能拦住人家不成,我看他们这样很不错。”
\par “好固然好,可是我对于德先这种偷偷摸摸的行为不赞成。”
\par 妈听了从鼻子里笑了一声,一回头看见了我,就骂我:
\par “小孩子听什么!还不睡去!”
\par 爸坐在那儿,两腿交叠着,不住地摇,我真想上前告诉他,在三贝子花园门口合照的相,德先叔还在上面题了字:“相逢何必曾相识”,兰姨娘给我讲了好几遍呢!可是我怕说出来爸会骂我,打我。我默默地爬上床,躺下去,又听妈说:
\par “他们决定明天就走吗?那总得烧几样菜送送他们吧?”
\par “随便你吧!”
\par 我再没听到什么了,心里只觉得舍不得兰姨娘,眼睛勉强睁开又闭上了。梦里还在写大字,兰姨娘按着我的右肩头,又仿佛是在逛灯的那晚上,我想举笔写字,她按得紧,抬不起手,怎么也写不成……
\par 可是现在我正一张又一张地写,终于在晚饭前写完了,我带着一嘴的黑胡子和黑手印上了饭桌,兰姨娘先笑了:
\par “你的大字倒刷好了?”
\par 我今天挨着兰姨娘坐,心中只觉依依不舍,妈直让酒,向兰姨娘和德先叔说:
\par “你们俩一路顺风!”
\par 爸不用人让,把自己灌得脸红红的,头上的青筋一条条像蚯蚓一样地暴露着,他举着酒杯伸出头,一直到兰姨娘的脸前,兰姨娘直朝后躲闪,嘴里说:
\par “林先生,你别再喝了,可喝不少了。”
\par 爸忽然又直起身子来,做出老大哥的神气,醉言醉语地说:
\par “我这个人最肯帮朋友的忙,最喜欢成全朋友,是不是?德先,你可得好好待她哟!她就像我自家的妹子一样哟!”爸又转过头来向兰姨娘说:“要是他待你不好,你尽管回到我这里来。”兰姨娘娇羞地笑着,就仿佛她是十八岁的大姑娘刚出嫁。
\par 宋妈在旁边侍候,也笑眯着,用很新鲜的眼光看兰姨娘。同时还把洒了双妹花露水的毛巾,一回又一回地送给爸爸擦脸。
\par 马车早就叫来停在大门口了。我们是全家大小在门口送行的,连刚满月的小妹妹都抱出大门口见风了。
\par 黄昏的虎坊桥大街很热闹,来来往往的,眼前都是人,也有邻居围在马车前等着看新鲜,宋妈早就告诉人家了吧!
\par 兰姨娘换了一个人,她的油光刷亮的麻花髻没有了,现在头发剪的是华伦王子式!就跟我故事书里画的一样:一排头发齐齐的齐着眉毛,两边垂到耳朵边。身上穿的正是那件蛋青绸子旗袍,做成长身坎肩另接两只袖子样式的,脖子上围一条白纱,斜斜地系成一个大蝴蝶结,就跟在女高师念书的张家三姨打扮得一样样!
\par 她跟爸妈说了多少感谢的话,然后低下身来摸着我的脸说:
\par “英子,好好地念书,可别像上回那么招你妈生气了,上三年级可是大姑娘啦!”
\par 我想哭,也想笑,不知什么滋味,看兰姨娘跟德先叔同进了马车,隔着窗子还跟我们招手。
\par 那马车越走越远越快了,扬起一阵滚滚灰尘,就什么也看不清了。我仰头看爸爸,他用手摸着胸口,像妈每次生了气犯胃病那样,我心里只觉得有些对不起爸,更是同情。我轻轻推爸爸的大腿,问他:
\par “爸,你要吃豆蔻吗?我去给你买。”
\par 他并没有听见,但冲那远远的烟尘摇摇头。







\subsection{驴打滚儿}


\par 换绿盆儿的,用他的蓝布掸子的把儿,使劲敲着那个两面的大绿盆说:
\par “听听!您听听!什么声儿!哪找这绿盆去,赛江西瓷!您再添吧!”
\par 妈妈用一堆报纸,三只旧皮鞋,两个破铁锅要换他的四只小板凳,一块洗衣板;宋妈还要饶一个小小绿盆儿,留着拌黄瓜用。
\par 我呢,抱着一个小板凳不放手。换绿盆儿的嚷着要妈妈再添东西。一件旧棉袄,两叠破书都加进去了,他还说:
\par “添吧,您。”
\par 妈说:“不换了!”叫宋妈把东西搬进去。我着急买卖不能成交,凳子要交还他,谁知换绿盆儿的大声一喊:
\par “拿去吧!换啦!”他挥着手垂头丧气地说,“唉!谁让今儿个没开张哪!”
\par 四只小板凳就摆在对门的大树阴底下,宋妈带着我们四个人——我,珠珠,弟弟,燕燕——坐在新板凳上讲故事。燕燕小,挤在宋妈的身边,半坐半靠着,吃她的手指头玩。
\par “你家小栓子多大了?”我问。
\par “跟你一般儿大,九岁喽!”
\par 小栓子是宋妈的儿子。她这两天正给我们讲她老家的故事:地里的麦穗长啦,山坡的青草高啦,小栓子摘了狗尾巴花扎在牛犄角上啦。她手里还拿着一只厚厚的鞋底,用粗麻绳纳得密密的,正是给小栓子做的。
\par “那么他也上三年级啦?”我问。
\par “乡下人有你这好命儿?他成年价给人看牛哪!”她说着停了手里的活儿,举起锥子在头发里划几下,自言自语地说:“今年个,可得回家看看了,心里老不顺序。”她说完愣愣的,不知在想什么。
\par “那么你家丫头子呢?”
\par 其实丫头子的故事我早已经知道了,宋妈讲过好几遍。宋妈的丫头子和弟弟一样,今年也四岁了。她生了丫头子,才到城里来当奶妈,一下就到我们家,做了弟弟的奶妈。她的奶水好,弟弟吃得又白又胖。她的丫头子呢,就在她来我家试妥了工以后,被她的丈夫抱回去给人家奶去了。我问一次,她讲一次,我也听不腻就是了。
\par “丫头子呀,她花钱给人家奶去啦!”宋妈说。
\par “将来还归不归你?”
\par “我的姑娘不归我?你归不归你妈?”她反问我。
\par “那你为什么不自己给奶?为什么到我家当奶妈?为什么你挣的钱又给人家去?”
\par “为什么?为的是——说了你也不懂,俺们乡下人命苦呀!小栓子他爸没出息,动不动就打我,我一狠心就出来当奶妈自己挣钱!”
\par 我还记得她刚来的那一天,是个冬天,她穿着大红棉袄,里子是白布的,油亮亮的很脏了。她把奶头塞到弟弟的嘴里,弟弟就咕嘟咕嘟地吸呀吸呀,吃了一大顿奶,立刻睡着了,过了很久才醒来,也不哭了。就这样留下她当奶妈的。
\par 过了三天,她的丈夫来了,拉着一匹驴,拴在门前的树干上。他有一张大长脸,黄板儿牙,怎么这么难看!妈妈下工钱了,折子上写着:一个月四块钱,两副银首饰,四季衣裳,一床新铺盖,过了一年零四个月才许回家去。
\par 穿着红棉袄的宋妈,把她的小孩子包裹在一条旧花棉被里,交给她的丈夫。她送她的丈夫和孩子出来时,哭了,背转身去掀起衣襟在擦眼泪,半天抬不起头来。媒人店的老张劝宋妈说:
\par “别哭了,小心把奶憋回去。”
\par 宋妈这才止住哭,她把钱算给老张,剩下的全给了她丈夫。她又嘱咐她丈夫许多话,她的丈夫说:
\par “你放心吧。”
\par 他就抱着孩子牵着驴,走远了。
\par 到了一年四个月,黄板儿牙又来了,他要接宋妈回去,但是宋妈舍不得弟弟,妈妈又要生小孩子,就又把她留下了。宋妈的大洋钱,数了一大垛交给她丈夫,他把钱放进蓝布袋子里,叮叮当当的,牵着驴又走了。
\par 以后他就每年来两回,小叫驴拴在院子里墙犄角,弄得满地的驴粪球,好在就一天,他准走。随着驴背滚下来的是一个大麻袋,里面不是大花生,就是大醉枣,是他送给老爷和太太——我爸爸和妈妈的。乡下有的是。
\par 我简直想不出宋妈要是真的回她老家去,我们家会成了什么样儿?老早起来谁给我梳辫子上学去?谁喂燕燕吃饭?弟弟挨爸爸打的时候谁来护着?珠珠拉了屎谁给擦?我们都离不开她呀!
\par 可是她常常要提回家去的话,她近来就问我们好几次:“我回俺们老家去好不好?”
\par “不许啦!”除了不会说话的燕燕以外,我们齐声反对。春天弟弟出麻疹闹得很凶,他紧闭着嘴不肯喝那芦根汤,我们围着鼻子眼睛起满了红疹的弟弟看。妈说:
\par “好,不吃药,就叫你奶妈回去!回去吧!宋妈!把衣服、玩意儿,都送给你们小栓子、小丫头子去!”
\par 宋妈假装一边往外走一边说:
\par “走喽!回家喽!回家找俺们小栓子、小丫头子去哟!”
\par “我喝!我喝!不要走!”弟弟可怜兮兮地张开手要过妈妈手里的那碗芦根汤,一口气喝下了大半碗。宋妈心疼得什么似的,立刻搂抱起弟弟,把头靠着弟弟滚烫的烂花脸儿说:
\par “不走!我不会走!我还是要俺们弟弟,不要小栓子,不要小丫头子!”跟着,她的眼圈可红了,弟弟在她的拍哄中渐渐睡着了。
\par 前几天,一个管宋妈叫大婶儿的小伙子来了,他来住两天,想找活儿做。他会用铁丝给大门的电灯编灯罩儿,免得灯泡被贼偷走。宋妈问他说:
\par “你上京来的时候,看见我们小栓子好吧?”
\par “嗯?”他好像吃了一惊,瞪着眼珠,“我倒没看见,我是打刘村我舅舅那儿来的!”
\par “噢,”宋妈怀着心思地呆了一下,又问:“你打你舅那儿来的,那,俺们丫头给刘村的金子他妈奶着,你可听说孩子结实吗?”
\par “哦?”他又是一惊,“没——没听说。准没错儿,放心吧!”
\par 停了一下他可又说:
\par “大婶儿,您要能回趟家看看也好,三四年没回去啦!”
\par 等到这个小伙子走了,宋妈跟妈妈说,她听了她侄子的话,吞吞吐吐的,很不放心。
\par 妈妈安慰她说:
\par “我看你这侄儿不正经,你听,他一会儿打你们家来,一会儿打他舅舅家来。他自己的话都对不上,怎么能知道你家孩子的事呢!”
\par 宋妈还是不放心,她说:
\par “我打今年个一开年心里就老不顺序,做了好几回梦啦!”
\par 她叫了算命的来给解梦。礼拜那天又叫我替她写信。她老家的地名我已经背下了:顺义县牛栏山冯村妥交冯大明吾夫平安家信。
\par “念书多好,看你九岁就会写信,出门丢不了啦!”
\par “信上说什么?”我拿着笔,铺一张信纸,逞起能来。
\par “你就写呀,家里大小可平安?小栓子到野地里放牛要小心,别尽顾得下水里玩。我给做好了两双鞋一套裤褂。丫头子那儿别忘了到时候送钱去!给人家多道道乏。拿回去的钱前后快二百块了,后坡的二分地该赎就赎回来,省得老种人家的地。还有,我这儿倒是平安,就是惦记着孩子,赶下个月要来的时候,把栓子带来我瞅瞅也安心。还有……”
\par “这封信太长了!”我拦住她没完没了的话,“还是让爸爸写吧!”
\par 爸爸给她写的信寄出去了,宋妈这几天很高兴。现在,她问弟弟说:
\par “要是小栓子来,你的新板凳给不给他坐?”
\par “给呀!”弟弟说着立刻就站起来。
\par “我也给。”珠珠说。
\par “等小栓子来,跟我一块儿上附小念书好不好?”我说。
\par “那敢情好,只要你妈答应让他在这儿住着。”
\par “我去说!我妈妈很听我的话。”
\par “小栓子来了,你们可别笑他呀,英子,你可是顶能笑话人!他是乡下人,可土着呢!”宋妈说的仿佛小栓子等会儿就到似的。她又看看我说:
\par “英子,他准比你高,四年了,可得长多老高呀!”
\par 宋妈高兴得抱起燕燕,放在她的膝盖上。膝盖头颠呀颠的,她唱起她的歌:
\par “鸡蛋鸡蛋壳壳儿,里头坐个哥哥儿,哥哥出来卖菜,里头坐个奶奶,奶奶出来烧香,里头坐个姑娘,姑娘出来点灯,烧了鼻子眼睛!”
\par 她唱着,用手扳住燕燕的小手指,指着鼻子和眼睛,燕燕笑得咯咯的。
\par 宋妈又唱那快板儿:
\par “槐树槐,槐树槐,槐树底下搭戏台,人家姑娘都来到,就差我的姑娘还没来;说着说着就来了,骑着驴,打着伞,光着屁股挽着髻……”
\par 太阳斜过来了,金黄的光从树叶缝里透过来,正照着我的眼,我随着宋妈的歌声,斜头躲过晃眼的太阳,忽然看见远远的胡同口外,一团黑在动着。我举起手遮住阳光仔细看,真是一匹小驴,嘚、嘚、嘚地走过来了。赶驴的人,蓝布的半截褂子上,蒙了一层黄土。哟!那不是黄板儿牙吗?我喊宋妈:
\par “你看,真有人骑驴来了!”
\par 宋妈停止了歌声,转过头去呆呆地看。
\par 黄板儿牙一声:“窝——哦!”小驴停在我们的面前。
\par 宋妈不说话,也不站起来,刚才的笑容没有了,绷着脸,眼直直瞅着她的丈夫,仿佛等什么。
\par 黄板儿牙也没说话,扑扑地掸他的衣服,黄土都飞起来了。我看不起他!拿手捂着鼻子。他又摘下了草帽扇着,不知道跟谁说:
\par “好热呀!”
\par 宋妈这才好像忍不住了,问说:
\par “孩子呢?”
\par “上——上他大妈家去了。”他又抬起脚来掸鞋,没看宋妈。他的白布袜子都变黄了,那也是宋妈给做的。他的袜子像鞋一样,底子好几层,细针密线儿纳的。
\par 我看着驴背上的大麻袋,不知里面这回装的是什么。黄板儿牙把口袋拿下来解开了,从里面掏出一大捧烤得倍儿干的挂落枣给我,咬起来是脆的,味儿是辣的,香的。
\par “英子,你带珠珠上小红她们家玩去,挂落枣儿多拿点儿去,分给人家吃。”宋妈说。
\par 我带着珠珠走了,回过头看,宋妈一手收拾起四个新板凳,一手抱燕燕,弟弟拉着她的衣角,他们正向家里走。黄板儿牙牵起小叫驴,走进我家门,他准又要住一夜。他的驴满地打滚儿,爸爸种的花草,又要被糟践了。
\par 等我们从小红家回来,天都快黑了,挂落枣没吃几个,小红用细绳穿好全给我挂在脖子上了。
\par 进门来,宋妈和她丈夫正在门道里。黄板儿牙坐在我们的新板凳上发呆,宋妈蒙着脸哭,不敢出声儿。
\par 屋里已经摆上饭菜了。妈妈在喂燕燕吃饭,皱着眉,抿着嘴,又摇头叹着气,神气挺不对。
\par “妈,”我小声地叫,“宋妈哭呢!”
\par 妈妈向我轻轻地摆手,禁止我说话。什么事情这样重要?
\par “宋妈的小栓子已经死了,”妈妈沙着嗓子对我说,她又转向爸爸,“唉!已经死了一两年,到现在才说出来,怪不得宋妈这一阵子总是心不安,一定要叫她丈夫来问问。她侄子那次来,是话里有意思的。两件事一齐发作,叫人怎么受!”
\par 爸爸也摇头叹息着,没有话可说。
\par 我听了也很难过,但不知另外还有一件事是什么,又不敢问。
\par 妈妈叫我去喊宋妈来,我也感觉是件严重的事,到门道里,不敢像每次那样大声吆喝她,我轻轻地喊:
\par “宋妈,妈叫你呢!”
\par 宋妈很不容易地止住抽噎的哭声,到屋里来。妈对她说:
\par “你明天跟他回家去看看吧,你也好几年没回家了。”
\par “孩子都没了,我还回去干么?不回去了,死也不回去了!”宋妈红着眼狠狠地说;并且接过妈妈手中的汤匙喂燕燕,好像这样就表示她待定在我们家不走了。
\par “你家丫头子到底给了谁呢?能找回来吗?”
\par “好狠心呀!”宋妈恨得咬着牙,“那年抱回去,敢情还没出哈德门,他就把孩子给了人,他说没要人家钱,我就不信!”
\par “给了谁,有名有姓,就有地方找去。”
\par “说是给了一个赶马车的,公母俩四十岁了没儿没女的,谁知道是真话假话!”
\par “问清楚了找找也好。”
\par 原来是这么一回事儿,宋妈成年跟我们念叨的小栓子和丫头子,这一下都没有了。年年宋妈都给他们两个做那么多衣服和鞋子,她的丈夫都送给了谁?旧花棉被里裹着的那个小婴孩,到了谁家了?我想问小栓子是怎么死的,可是看着宋妈的红肿的眼睛,就不敢问了。
\par “我看你还是回去。”妈妈又劝她,但是宋妈摇摇头,不说什么,尽管流泪。她一匙一匙地喂燕燕,燕燕也一口一口地吃,但两眼却盯着宋妈看。因为宋妈从来没有这个样子过。
\par 宋妈照样地替我们四个人打水洗澡,每个人的脸上、脖子上扑上厚厚的痱子粉,照样把弟弟和燕燕送上了床。只是她今天没有心思再唱她的打火链儿的歌儿了,光用扇子扑呀扑呀扇着他们睡了觉。一切都照常,不过她今天没有吃晚饭,把她的丈夫扔在门道儿里不理他。他呢,正用打火石打亮了火,吧嗒吧嗒地抽着旱烟袋。小驴大概饿了,它在地上卧着,忽然仰起脖子一声高叫,多么难听!黄板儿牙过去打开了一袋子干草,它看见吃的,一翻滚,站起来,小蹄子把爸爸种在花池子边的玉簪花给踩倒了两三棵。驴子吃上干草子,鼻子一抽一抽的,大黄牙齿露着。怪不得,奶妈的丈夫像谁来着,原来是它!宋妈为什么嫁给黄板儿牙,这蠢驴!
\par 第二天早上我起来,朝窗外看去,驴没了,地上留了一堆粪球,宋妈在打扫。她一抬头看见了我,招手叫我出去。
\par 我跑出来,宋妈跟我说:
\par “英子,别乱跑,等会儿跟我出趟门,你识字,帮我找地方。”
\par “到哪儿去?”我很奇怪。
\par “到哈德门那一带去找找——”说着她又哭了,低下头去,把驴粪撮进簸箕里,眼泪掉在那上面,“找丫头子。”
\par “好的。”我答应着。
\par 宋妈和我偷偷出去的,妈妈哄着弟弟他们在房里玩。出了门走不久,宋妈就后悔了:
\par “应当把弟弟带着,他回头看不见我准得哭,他一时一刻也没离开过我呀!”
\par 就是为了这个,宋妈才一年年留在我家的,我这时仗着胆子问:
\par “小栓子怎么死的?宋妈。”
\par “我不是跟你说过,冯村的后坡下有条河吗……”
\par “是呀,你说,叫小栓子放牛的时候要小心,不要就顾得玩水。”
\par “他掉在水里死的时候,还不会放牛呢,原来正是你妈妈生燕燕那一年。”
\par “那时候黄板——嗯,你的丈夫做什么去了?”
\par “他说他是上地里去了,他要不是上后坡草棚里耍钱去才怪呢!准是小栓子饿了一天找他要吃的去,给他轰出来了。不是上草棚,走不到后坡的河里去。”
\par “还有,你的丈夫为什么要把小丫头子送给人?”
\par “送了人不是更松心吗?反正是个姑娘不值钱。要不是小栓子死了,丫头子,我不要也罢。现在我就不能不找回她来,要花钱就花吧。”宋妈说。
\par 我们从绒线胡同穿过兵部洼,中街,西交民巷,出东交民巷就是哈德门大街。我在路上忽然又想起一句话。
\par “宋妈,你到我们家来,丢了两个孩子不后悔吗?”
\par “我是后悔——后悔早该把俺们小栓子接进城来,跟你一块儿念书认字。”
\par “你要找到丫头子呢,回家吗?”
\par “嗯。”宋妈瞎答应着,她并没有听清我的话。
\par 我们走到西交民巷的中国银行门口,宋妈在石阶上歇下来,过路来了一个卖吃的也停在这儿。他支起木架子把一个方木盘子摆上去,然后掀开那块盖布,在用黄色的面粉做一种吃的。
\par “宋妈,他在做什么?”
\par “啊?”宋妈正看着砖地在发愣,她抬起头来看看说:
\par “那叫驴打滚儿。把黄米面蒸熟了,包黑糖,再在绿豆粉里滚一滚,挺香,你吃不吃?”
\par 吃的东西起名叫“驴打滚儿”,很有意思,我哪有不吃的道理!我咽咽唾沫点点头,宋妈掏出钱来给我买了两个吃。她又多买了几个,小心地包在手绢里,我说:“是买给丫头子的吗?”
\par 出了东交民巷,看见了热闹的哈德门大街了,但是往哪边走?我们站在美国同仁医院的门口。宋妈的背,汗湿透了,她提起竹布褂的两肩头抖落着,一边东看看,西看看。
\par “走那边吧。”她指指斜对面,那里有一排不是楼房的店铺。走过了几家,果然看见一家马车行,里面很黑暗,门口有人闲坐着。宋妈问那人说:
\par “跟您打听打听,有个赶马车的老大哥,跟前有一个姑娘的,在您这儿吧?”
\par 那人很奇怪地把宋妈和我上下看了看:
\par “你们是哪儿的?”
\par “有个老乡亲托我给他带个信儿。”
\par 那人指着旁边的小胡同说:
\par “在家哪,胡同底那家就是。”
\par 宋妈很兴奋,直向那人道谢,然后她拉着我的手向胡同里走去。这是一条死胡同,走到底,是个小黑门,门虽关着,一推就开了,院子里有两三个孩子在玩土。
\par “劳驾,找人哪!”宋妈喊道。
\par 其中一个小孩子便向着屋里高声喊了好几声:
\par “姥姥,有人找。”
\par 屋里出来了一位老太太,她耳朵聋,大概眼睛也快瞎了,竟没看见我们站在门口,孩子们说话她也听不见,直到他们用手指着我们,她才向门口走来。宋妈大声地喊:
\par “您这院里住几家子呀?”
\par “啊啊,就一家。”老太太用手罩着耳朵才听见。
\par “您可有个姑娘呀!”
\par “有呀,你要找孩子他妈呀!”她指着三个男孩子。
\par 宋妈摇摇头,知道完全不对头了,没等老太太说完,便说:
\par “找错人了!”
\par 我们从哈德门里走到哈德门外,一共看见了三家马车行,都问得人家直摇头。我们就只好照着原路又走回来,宋妈在路上一句话也不说,半天才想起什么来,说:“英子,你走累了吧?咱们坐车好不?”
\par 我摇摇头,仰头看宋妈,她用手使劲捏着两眉间的肉,闭上眼,有点站不稳,好像要昏倒的样子。她又问我:
\par “饿了吧?”说着就把手巾包打开,拿出一个刚才买的驴打滚儿来,上面的绿豆粉已经被黄米面湿溶了。我嘴里念了一声:“驴打滚儿!”接过来,放在嘴里。
\par 我对宋妈说:
\par “我知道为什么叫驴打滚儿了,你家的驴在地上打个滚起来,屁股底下总有这么一堆。”我提起一个给她看,“像驴粪球不?”
\par 我是想逗宋妈笑的,但是她不笑,只说:
\par “吃罢!”
\par 半个月过去,宋妈说,她跑遍了北京城的马车行,也没有一点点丫头子的影子。
\par 树阴底下听不见冯村后坡上小栓子放牛的故事了;看不见宋妈手里那一双双厚鞋底了;也不请爸爸给写平安家信了。她总是把手上的银镯子转来转去地呆看着,没有一句话。
\par 冬天又来了,黄板儿牙又来了。宋妈让他蹲在下房里一整天,也不跟他说话。这是下雪的晚上,我们吃过晚饭挤在窗前看院子。宋妈把院子的电灯捻开,灯光照在白雪上,又平又亮。天空还在不断地落着雪,一层层铺上去。宋妈喂燕燕吃冻柿子,我念着国文上的那课叫做《雪》的课文:
\refdocument{
    \par 一片一片又一片,
    \par 两片三片四五片,
    \par 六片七片八九片,
    \par 飞入芦花都不见。
}
\par 老师说,这是一个不会做诗的皇帝做的诗,最后一句还是他的臣子给接上去的。但是念起来很顺嘴,很好听。
\par 妈妈在灯下做燕燕的红缎子棉袄,棉花撕得小小的、薄薄的,一层层地铺上去。妈妈说:“把你当家的叫来,信是我叫老爷偷着写的,你跟他回去吧,明年生了儿子再回这儿来。是儿不死,是财不散,小栓子和丫头子,活该命里都不归你,有什么办法!你不能打这儿起就不生养了!”
\par 宋妈一声不言语,妈妈又说:
\par “你瞧怎么样?”
\par 宋妈这才说:
\par “也好,我回家跟他算账去!”
\par 爸爸和妈妈都笑了。
\par “这几个孩子呢?”宋妈说。
\par “你还怕我亏待了他们吗?”妈妈笑着说。
\par 宋妈看着我说:
\par “你念书大了,可别欺侮弟弟呀!别净跟你爸爸告他的状,他小。”
\par 弟弟已经倒在椅子上睡着了,他现在很淘气,常常爬到桌子上翻我的书包。
\par 宋妈把弟弟抱到床上去,她轻轻给弟弟脱鞋,怕惊醒了他。她叹口气说:“明天早上看不见我,不定怎么闹。”她又对妈妈说:“这孩子脾气犟,叫老爷别动不动就打他;燕燕这两天有点咳嗽,您还是拿鸭梨炖冰糖给她吃;英子的毛窝我带回去做,有人上京就给捎了来;珠珠的袜子都该补了。还有……我看我还是……唉!”宋妈的话没有说完,就不说了。
\par 妈妈把折子拿出来,叫爸爸念着,算了许多这钱那钱给她;她丝毫不在乎地接过钱,数也不数,笑得很惨:
\par “说走就走了!”
\par “早点睡觉吧,明天你还得起早。”妈妈说。
\par 宋妈打开门看看天说:
\par “那年个,上京来的那天也是下着鹅毛大雪,一晃儿,四年了!”
\par 她的那件红棉袄,也早就拆了;旧棉花换了榧子儿,泡了梳头用;面子和里子,给小栓子纳鞋底了。
\par “妈,宋妈回去还来不来了?”我躺在床上问妈妈。
\par 妈妈摆手叫我小声点儿,她怕我吵醒了弟弟,她轻声地对我说:
\par “英子,她现在回去,也许到明年的下雪天又来了,抱着一个新的娃娃。”
\par “那时候她还要给我们家当奶妈吧?那您也再生一个小妹妹。”
\par “小孩子胡说!”妈妈摆着正经脸骂我。
\par “明天早上谁给我梳辫子?”我的头发又黄又短,很难梳,每天早上总是跳脚催着宋妈,她就要骂我:“催惯了,赶明儿要上花轿也这么催,多寒碜!”
\par “明天早点儿起来,还可以赶着让宋妈给你梳了辫子再走。”妈妈说。
\par 天刚蒙蒙亮,我就醒了,听见窗外沙沙的声音,我忽然想起一件事,赶快起床下地跑到窗边向外看。雪停了,干树枝上挂着雪,小驴拴在树干上,它一动弹,树枝上的雪就被抖落下来,掉在驴背上。
\par 我轻轻地穿上衣服出去,到下房找宋妈,她看见我这样早起来,吓了一跳。我说:
\par “宋妈,给我梳辫子。”
\par 她今天特别的和气,不唠叨我了。
\par 小驴儿吃好了早点,黄板儿牙把它牵到大门口,被褥一条条地搭在驴背上,好像一张沙发椅那么厚,骑上去一定很舒服。
\par 宋妈打点好了,她用一条毛线大围巾包住头,再在脖子上绕两绕。她跟我说:
\par “我不叫你妈了,稀饭在火上炖着呢!英子,好好念书,你是大姐,要有个样儿。”说完她就盘腿坐在驴背上,那姿势真叫绝!
\par 黄板儿牙拍了一下驴屁股,小驴儿朝前走,在厚厚雪地上印下了一个个清楚的蹄印儿。黄板儿牙在后面跟着驴跑,嘴里喊着:“嘚、嘚、嘚、嘚。”
\par 驴脖子上套了一串小铃铛,在雪后的清新空气里,响得真好听。




\subsection{爸爸的花儿落了\ 我也不再是小孩子}


\par 新建的大礼堂里,坐满了人;我们毕业生坐在前八排,我又是坐在最前一排的中间位子上。我的衣襟上有一朵粉红色的夹竹桃,是临来时妈妈从院子里摘下来给我别上的。她说:
\par “夹竹桃是你爸爸种的,戴着它,就像爸爸看见你上台时一样!”
\par 爸爸病倒了,他住在医院里不能来。
\par 昨天我去看爸爸,他的喉咙肿胀着,声音是低哑的。我告诉爸,行毕业典礼的时候,我代表全体同学领毕业证书,并且致谢词。我问爸,能不能起来,参加我的毕业典礼?六年前他参加我们学校的那次欢送毕业同学同乐会时,曾经要我好好用功,六年后也代表同学领毕业证书和致谢词。今天,“六年后”到了,我真的被选做这件事。
\par 爸爸哑着嗓子,拉起我的手笑笑说:
\par “我怎么能够去?”
\par 但是我说:
\par “爸爸,你不去,我很害怕,你在台底下,我上台说话就不发慌了。”
\par 爸爸说:
\par “英子,不要怕,无论什么困难的事,只要硬着头皮去做,就闯过去了。”
\par “那么爸不也可以硬着头皮从床上起来到我们学校去吗?”
\par 爸爸看着我,摇摇头,不说话了。他把脸转向墙那边,举起他的手,看那上面的指甲。然后,他又转过脸来叮嘱我:
\par “明天要早起,收拾好就到学校去,这是你在小学的最后一天了,可不能迟到!”
\par “我知道,爸爸。”
\par “没有爸爸,你更要自己管自己,并且管弟弟和妹妹,你已经大了,是不是?”
\par “是。”我虽然这么答应了,但是觉得爸爸讲的话很使我不舒服,自从六年前的那一次,我何曾再迟到过?
\par 当我在一年级的时候,就有早晨赖在床上不起床的毛病。每天早晨醒来,看到阳光照到玻璃窗上,我的心里就是一阵愁:已经这么晚了,等起来,洗脸,扎辫子,换制服,再到学校去,准又是一进教室被罚站在门边。同学们的眼光,会一个个向你投过来。我虽然很懒惰,却也知道害羞呀!所以又愁又怕,每天都是怀着恐惧的心情,奔向学校去。最糟的是爸爸不许小孩子上学乘车的,他不管你晚不晚。
\par 有一天,下大雨,我醒来就知道不早了,因为爸爸已经在吃早点。我听着,望着大雨,心里愁得了不得。我上学不但要晚了,而且要被妈妈打扮得穿上肥大的夹袄(是在夏天!)和踢拖着不合脚的油鞋,举着一把大油纸伞,走向学校去!想到这么不舒服的上学,我竟有勇气赖在床上不起来了。
\par 等一下,妈妈进来了。她看我还没有起床,吓了一跳,催促着我,但是我皱紧了眉头,低声向妈哀求说:
\par 北平国立女子师范大学旧址的校门前
\par “妈,今天晚了,我就不去上学了吧?”
\par 妈妈就是做不了爸爸的主意,当她转身出去,爸爸就进来了。他瘦瘦高高的,站在床前来,瞪着我:
\par “怎么还不起来,快起!快起!”
\par “晚了!爸!”我硬着头皮说。
\par “晚了也得去,怎么可以逃学!起!”
\par 一个字的命令最可怕,但是我怎么啦!居然有勇气不挪窝。
\par 爸气极了,一把把我从床上拖起来,我的眼泪就流出来了。爸左看右看,结果从桌上抄起鸡毛掸子倒转来拿,藤鞭子在空中一抡,就发出咻咻的声音,我挨打了!
\par 爸把我从床头打到床角,从床上打到床下,外面的雨声混合着我的哭声。我哭号,躲避,最后还是冒着大雨上学去了。我是一只狼狈的小狗,被宋妈抱上了洋车——第一次花五大枚坐车去上学。
\par 我坐在放下雨篷的洋车里,一边抽抽搭搭地哭着,一边撩起裤脚来检查我的伤痕。那一条条鼓起的鞭痕,是红的,而且发着热。我把裤脚向下拉了拉,遮盖住最下面的一条伤痕,我最怕被同学耻笑。
\par 虽然迟到了,但是老师并没有罚我站,这是因为下雨天可以原谅的缘故。
\par 老师教我们先静默再读书。坐直身子,手背在身后,闭上眼睛,静静地想五分钟。老师说:想想看,你是不是听爸妈和老师的话?昨天的功课有没有做好?今天的功课全带来了吗?早晨跟爸妈有礼貌地告别了吗?……我听到这儿,鼻子抽搭了一下,幸好我的眼睛是闭着的,泪水不至于流出来。
\par 正在静默的当中,我的肩头被拍了一下,急忙地睁开了眼,原来是老师站在我的位子边。他用眼势告诉我,教我向教室的窗外看去,我猛一转头看,是爸爸那瘦高的影子!
\par 我刚安静下来的心又害怕起来了!爸为什么追到学校来?爸爸点头示意招我出去。我看看老师,征求他的同意,老师也微笑地点点头,表示答应我出去。
\par 我走出了教室,站在爸面前。爸没说什么,打开了手中的包袱,拿出来的是我的花夹袄。他递给我,看着我穿上,又拿出两个铜板来给我。
\par 后来怎么样了,我已经不记得,因为那是六年以前的事了。只记得,从那以后,到今天,每天早晨我都是等待着校工开大铁栅校门的学生之一。冬天的清晨站在校门前,戴着露出五个手指头的那种手套,举了一块热乎乎的烤白薯在吃着。夏天的早晨站在校门前,手里举着从花池里摘下的玉簪花,送给亲爱的韩老师,她教我跳舞。
\par 啊!这样的早晨,一年年都过去了,今天是我最后一天在这学校里啦!
\par 当当当,钟响了,毕业典礼就要开始。看外面的天,有点阴,我忽然想,爸爸会不会忽然从床上起来,给我送来花夹袄?我又想,爸爸的病几时才能好?妈妈今早的眼睛为什么红肿着?院里大盆的石榴和夹竹桃今年爸爸都没有给上麻渣,他为了叔叔给日本人害死,急得吐血了。到了五月节,石榴花没有开得那么红,那么大。如果秋天来了,爸还要买那样多的菊花,摆满在我们的院子里、廊檐下、客厅的花架上吗?
\par 爸是多么喜欢花。
\par 每天他下班回来,我们在门口等他,他把草帽推到头后面抱起弟弟,经过自来水龙头,拿起灌满了水的喷水壶,唱着歌儿走到后院来。他回家来的第一件事就是浇花。那时太阳快要下去了,院子里吹着凉爽的风,爸爸摘下一朵茉莉插到瘦鸡妹妹的头发上。陈家的伯伯对爸爸说:“老林,你这样喜欢花,所以你太太生了一堆女儿!”我有四个妹妹,只有两个弟弟。我才十二岁……
\par 我为什么总想到这些呢?韩主任已经上台了,他很正经地说:
\par “各位同学都毕业了,就要离开上了六年的小学到中学去读书,做了中学生就不是小孩子了,当你们回到小学来看老师的时候,我一定高兴看你们都长高了,长大了……”
\par 于是我唱了五年的骊歌,现在轮到同学们唱给我们送别:
\par “长亭外,古道边,芳草碧连天。问君此去几时来,来时莫徘徊!天之涯,地之角,知交半零落,人生难得是欢聚,惟有别离多……”
\par 我哭了,我们毕业生都哭了。我们是多么喜欢长高了变成大人,我们又是多么怕呢!当我们回到小学来的时候,无论长得多么高,多么大,老师!你们要永远拿我当个孩子呀!
\par 做大人,常常有人要我做大人。
\par 宋妈临回她的老家的时候说:
\par “英子,你大了,可不能跟弟弟再吵嘴!他还小。”
\par 兰姨娘跟着那个四眼狗上马车的时候说:
\par “英子,你大了,可不能招你妈妈生气了!”
\par 蹲在草地里的那个人说:
\par “等到你小学毕业了,长大了,我们看海去。”
\par 虽然,这些人都随着我的长大没有了影子了。是跟着我失去的童年一起失去了吗?爸爸也不拿我当孩子了,他说:
\par “英子,去把这些钱寄给在日本读书的陈叔叔。”
\par “爸爸!——”
\par “不要怕,英子,你要学做许多事,将来好帮着你妈妈。你最大。”
\par 于是他数了钱,告诉我怎样到东交民巷的正金银行去寄这笔钱——到最里面的台子上去要一张寄款单,填上“金柒拾圆也”,写上日本横滨的地址,交给柜台里的小日本儿!
\par 我虽然很害怕,但是也得硬着头皮去。——这是爸爸说的,无论什么困难的事,只要硬着头皮去做,就闯过去了。
\par “闯练,闯练,英子。”我临去时爸爸还这样叮嘱我。
\par 我心情紧张,手里捏紧一卷钞票到银行去。等到从高台阶的正金银行出来,看着东交民巷街道中的花圃种满了蒲公英,我高兴地想:闯过来了,快回家去,告诉爸爸,并且要他明天在花池里也种满蒲公英。
\par 快回家去!快回家去!拿着刚发下来的小学毕业文凭——红丝带子系着的白纸筒,催着自己,我好像怕赶不上什么事情似的,为什么呀?
\par 进了家门来,静悄悄的,四个妹妹和两个弟弟都坐在院子里的小板凳上,他们在玩沙土,旁边的夹竹桃不知什么时候垂下了好几个枝子,散散落落的很不像样,是因为爸爸今年没有收拾它们——修剪、捆扎和施肥。
\par 石榴树大盆底下也有几粒没有长成的小石榴,我很生气,问妹妹们:
\par “是谁把爸爸的石榴摘下来的?我要告诉爸爸去!”
\par 妹妹们惊奇地睁大了眼,她们摇摇头说:“是它们自己掉下来的。”
\par 我捡起小青石榴。缺了一根手指头的厨子老高从外面进来了,他说:
\par “大小姐,别说什么告诉你爸爸了,你妈妈刚从医院来了电话,叫你赶快去,你爸爸已经……”他为什么不说下去了?我忽然觉得着急起来,大声喊着说:
\par “你说什么?老高。”
\par “大小姐,到了医院,好好儿劝劝你妈,这里就数你大了!就数你大了!”
\par 瘦鸡妹妹还在抢燕燕的小玩意儿,弟弟把沙土灌进玻璃瓶里。是的,这里就数我大了,我是小小的大人。我对老高说:
\par “老高,我知道是什么事了,我就去医院。”我从来没有过这样的镇定,这样的安静。
\par 我把小学毕业文凭,放到书桌的抽屉里,再出来,老高已经替我雇好了到医院的车子。走过院子,看到那垂落的夹竹桃,我默念着:
\par 爸爸的花儿落了,
\par 我也不再是小孩子。






