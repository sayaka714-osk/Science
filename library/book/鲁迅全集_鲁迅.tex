


\section{鲁迅全集1}


\par 书名:鲁迅全集(全十八卷)
\par 作者:鲁迅
\par 出版社:人民文学出版社
\par 出版时间:2015-11
\par ISBN:9787020050338

\subsection{出版说明}


\par 《鲁迅全集》最早的版本,由鲁迅先生纪念委员会编辑,收入作者的著作、译文和部分辑录的古籍,共二十卷,于1938年印行。
\par 新中国成立后,我社重新编辑出版新的注释本《鲁迅全集》。这部《全集》只收作者自己的原创著作,包括创作、评论、文学史专著及部分书信,并加了必要的注释,共十卷,于1956年至1958年间印行。
\par 1981年,我社又在十卷本的基础上进行较大的增补和修订,增收了《集外集拾遗补编》、《古籍序跋集》、《译文序跋集》和日记,以及当时搜集到的全部书信,并对所收著作都加了注释;另加附集一卷,收作者著译年表、《全集》篇目索引和注释索引。全书共十六卷。
\par 本版《全集》以1981年版为基础进行修订,根据增补不足,修正错讹的原则,补入迄今搜集到并经确认的佚文佚信,收入《两地书》的鲁迅原信和《答增田涉问信件集录》。对原有注释作了增补和修改,所收著作又据作者生前审定(或写定)的文本作了校核。此外,作者翻译的外国作品和校辑的中国文史古籍,以及早期编著的《中国矿产志》(与顾琅合编)和生理课程讲义《人生象敩》等,分别编为《鲁迅译文集》(十卷)、《鲁迅辑录古籍丛编》(四卷)和《鲁迅自然科学论著》(一卷),另行出版。
\par 修订后的《鲁迅全集》共十八卷,吸纳了迄今鲁迅研究的新成果,是目前最为完备的《鲁迅全集》的新版本。
\par 《鲁迅全集》的编注工作,一直受到中央和国家的重视,得到众多高等院校、科研机构和鲁迅研究界的专家学者的协助,其中有的直接参加了编注定稿工作,也得到广大读者的关心和帮助。对他们为《全集》的出版所做的贡献,我们在此表示衷心的感谢。
\par 《鲁迅全集》注释涉及的范围十分广泛和繁杂,虽然作了努力,但疏漏还会难以避免,我们仍期待读者的指教。
\par \rightline{人民文学出版社}
\par \rightline{2005年9月}



\subsection{坟}

本书收作者1907年至1925年所作论文二十三篇。1927年3月由北京未名社初版,1929年3月第二次印刷时曾经作者校订。1930年4月第三次印刷改由上海北新书局出版。作者生前共印行四版次。


\subsubsection*{题记\footnote{本篇最初发表于1926年11月20日北京《语丝》周刊一〇六期,题为《〈坟〉的题记》。}}


\par 将这些体式上截然不同的东西,集合了做成一本书样子的缘由,说起来是很没有什么冠冕堂皇的。首先就因为偶尔看见了几篇将近二十年前所做的所谓文章。这是我做的么?我想。看下去,似乎也确是我做的。那是寄给《河南》\footnote{《河南》清末留日学生创办的杂志。1907年(清光绪三十三年)12月创刊于东京。初为月刊,后不定期出版。程克、孙竹丹等主编。发行人署名武人,总编辑刘炽等。1909年12月出至第九期被禁。1901年“辛丑条约”后至辛亥革命期间,中国留日学生所办杂志多以各省留日同乡会或各省留日同人的名义出版,内容偏重各省当时的政治、社会和文化问题,从事民族民主革命宣传和科学启蒙宣传,如《浙江潮》、《江苏》、《汉声》、《洞庭波》、《云南》、《四川》等。《河南》是其中的一种,作者在该刊发表的文章,有收入本书的《人之历史》等四篇,收入《集外集拾遗补编》的《破恶声论》和收入《鲁迅译文集》第十卷《译从补》的《裴彖飞诗论》(两篇都是未完稿)。}的稿子;因为那编辑先生有一种怪脾气,文章要长,愈长,稿费便愈多。所以如《摩罗诗力说》那样,简直是生凑。倘在这几年,大概不至于那么做了。又喜欢做怪句子和写古字,这是受了当时的《民报》\footnote{《民报》同盟会的机关杂志。1905年(清光绪三十一年)11月在日本东京创刊。初为月刊,后不定期出版,共出二十六期。初由胡汉民、张继等主编,自1906年8月第六号至十八号、二十三号至二十四号由章太炎主编。章太炎(1869—1936),名炳麟,号太炎,浙江余杭人,清末革命家、学者。他在《民报》发表的文章,喜用古字和生僻字句。这里说的受《民报》的影响,即指受章太炎的影响。}的影响;现在为排印的方便起见,改了一点,其余的便都由他。这样生涩的东西,倘是别人的,我恐怕不免要劝他“割爱”,但自己却总还想将这存留下来,而且也并不“行年五十而知四十九年非”\footnote{“行年五十而知四十九年非” 语出《淮南子·原道训》:“蘧伯玉年五十而有四十九年非。何者?先者难为知,而后者易为攻也。”},愈老就愈进步。其中所说的几个诗人,至今没有人再提起,也是使我不忍抛弃旧稿的一个小原因。他们的名,先前是怎样地使我激昂呵,民国告成以后,我便将他们忘却了,而不料现在他们竟又时时在我的眼前出现。
\par 其次,自然因为还有人要看,但尤其是因为又有人憎恶着我的文章。说话说到有人厌恶,比起毫无动静来,还是一种幸福。天下不舒服的人们多着,而有些人们却一心一意在造专给自己舒服的世界。这是不能如此便宜的,也给他们放一点可恶的东西在眼前,使他有时小不舒服,知道原来自己的世界也不容易十分美满。苍蝇的飞鸣,是不知道人们在憎恶他的;我却明知道,然而只要能飞鸣就偏要飞鸣。我的可恶有时自己也觉得,即如我的戒酒,吃鱼肝油,以望延长我的生命,倒不尽是为了我的爱人,大大半乃是为了我的敌人,——给他们说得体面一点,就是敌人罢——要在他的好世界上多留一些缺陷。君子之徒\footnote{这里的君子之徒和下文的所谓正人君子,指当时现代评论派文人。《现代评论》周刊是当时一部分留学英美的大学教授所办的同人杂志,1924年12月创刊于北京,1927年7月移至上海出版,至1928年12月停刊。主要刊登政论,同时也发表文艺创作、文艺评论。主要撰稿人有王世杰、高一涵、胡适、陈源(笔名西滢)、徐志摩、唐有壬等,也采用一些外来投稿。“正人君子”,是当时拥护北洋军阀政府的《大同晚报》在1925年8月7日的一篇报道中赞扬现代评论派的话;鲁迅在杂文中常引用来讽刺这一派文人。}曰:你何以不骂杀人不眨眼的军阀呢\footnote{这里说的不骂军阀和下文的“无枪阶级”,都见于《现代评论》第四卷第八十九期(1926年8月21日)署名涵庐(即高一涵)的一则《闲话》中,原文说:“我二十四分的希望一般文人彼此收起互骂的法宝,做我们应该做的和值得做的事业。万一骂溜了嘴,不能收束,正可以同那实在可骂而又实在不敢骂的人们,斗斗法宝,就是到天桥走走,似乎也还值得些!否则既不敢到天桥去,又不肯不骂人,所以专将法宝在无枪阶级的头上乱祭,那末,骂人诚然是骂人,却是高傲也难乎其为高傲罢。”按当时北京的刑场在天桥附近。}?斯亦卑怯也已!但我是不想上这些诱杀手段的当的。木皮道人\footnote{木皮道人 应作木皮散人,是明代遗民贾凫西的别号。贾凫西(约1590—约1676),名应宠,字思退,山东曲阜人,曾任刑部郎中。这里所引的话,见于他所著的《木皮散人鼓词》中关于周武王灭商纣王的一段:“多亏了散宜生定下胭粉计,献上个兴周灭商的女娇娃;……他爷们(按指周文王、武王父子等)昼夜商量行仁政,那纣王胡里胡涂在黑影爬;几年家软刀子割头不觉死,只等得太白旗悬才知道命有差。”}说得好,“几年家软刀子割头不觉死”,我就要专指斥那些自称“无枪阶级”而其实是拿着软刀子的妖魔。即如上面所引的君子之徒的话,也就是一把软刀子。假如遭了笔祸了,你以为他就尊你为烈士了么?不,那时另有一番风凉话。倘不信,可看他们怎样评论那死于三一八惨杀的青年\footnote{三一八惨案 1926年3月12日,冯玉祥所部国民军与奉系军阀作战,日本帝国主义出动军舰支持奉军,炮击国民军,并联合英美法意等国,于16日以最后通牒向北洋政府提出撤除大沽口国防设备等无理要求。3月18日,北京各界二万余人在天安门集会抗议,会后结队赴段祺瑞执政府请愿,要求拒绝八国通牒,段竟令卫队开枪射击,当场死四十七人,伤二百余人。惨案发生后,《现代评论》第三卷第六十八期(1926年3月27日)发表陈西滢评论此案的《闲话》,说爱国群众的被惨杀,是由于“居高位者的明令暗示”与“行凶”者的“惨苦残暴”,他们“都负有杀人的罪”;青年学生“还没有审判力”,做师长的“叫他们去参加种种他们还莫明其妙的运动”,“冒枪林弹雨的险,受践踏死伤的苦”,应当负“不加劝阻禁止”的“责任”;天安门抗议活动的主持者如果误听流言,宣布执政府的卫队已经解除了武装,“也未免太不负民众领袖的责任”,如果明知没有解除武装却故意那样说,“他的罪孽当然不下于开枪杀人者”;捏造和散布流言的人,“犯了故意引人去死地的嫌疑”,等等。参看《华盖集续编》中的《“死地”》、《空谈》等篇。}。
\par 此外,在我自己,还有一点小意义,就是这总算是生活的一部分的痕迹。所以虽然明知道过去已经过去,神魂是无法追蹑的,但总不能那么决绝,还想将糟粕收敛起来,造成一座小小的新坟,一面是埋藏,一面也是留恋。至于不远的踏成平地,那是不想管,也无从管了。
\par 我十分感谢我的几个朋友,替我搜集,抄写,校印,各费去许多追不回来的光阴。我的报答,却只能希望当这书印钉成工时,或者可以博得各人的真心愉快的一笑。别的奢望,并没有什么;至多,但愿这本书能够暂时躺在书摊上的书堆里,正如博厚的大地,不至于容不下一点小土块。再进一步,可就有些不安分了,那就是中国人的思想,趣味,目下幸而还未被所谓正人君子所统一,譬如有的专爱瞻仰皇陵,有的却喜欢凭吊荒冢,无论怎样,一时大概总还有不惜一顾的人罢。只要这样,我就非常满足了;那满足,盖不下于取得富家的千金云。
\par 一九二六年十月三十大风之夜,鲁迅记于厦门。






\section{鲁迅全集2}


\subsection{彷徨}

本书收作者1924年至1925年所作小说十一篇。1926年8月由北京北新书局初版,列为作者所编的《乌合丛书》之一。作者生前共印行十五版次。

\refdocument{
    \par 朝发轫于苍梧兮,夕余至乎县圃;欲少留此灵琐兮,日忽忽其将暮。
    \par 吾令羲和弭节兮,望崦嵫而勿迫;路漫漫其修远兮,吾将上下而求索。
    \par \rightline{屈原:《离骚》。\footnote{屈原(约前340—约前278)名平,字原,又字灵均,生于郢(在今湖北江陵)。战国后期楚国诗人。楚怀王时官左徒,由于他的政治主张不见容于贵族集团而屡遭迫害,后被顷襄王放逐到沅、湘流域,忿而作长诗《离骚》,以抒发其忿激心情和追求理想的决心。作者从《离骚》中引这几句诗作为本书的题辞,在1932年12月写的《〈自选集〉自序》(《南腔北调集》)中曾有过说明,可参看。诗中的苍梧,山名,又名九嶷山,在今湖南宁远境内,相传舜死后葬在这里;县圃,神话中昆仑山上神仙居住的地方;灵琐,神话中仙人宫阙的大门;羲和,神话中给太阳赶车的神;崦嵫,山名,神话中太阳住宿的地方。}}
}

\subsubsection*{祝福\footnote{本篇最初发表于1924年3月25日上海《东方杂志》半月刊第二十一卷第六号。}}

\par 旧历的年底毕竟最像年底,村镇上不必说,就在天空中也显出将到新年的气象来。灰白色的沉重的晚云中间时时发出闪光,接着一声钝响,是送灶\footnote{送灶 旧俗以夏历十二月二十四日为灶神升天的日子,在这一天或前一天祭送灶神,称为送灶。}的爆竹;近处燃放的可就更强烈了,震耳的大音还没有息,空气里已经散满了幽微的火药香。我是正在这一夜回到我的故乡鲁镇的。虽说故乡,然而已没有家,所以只得暂寓在鲁四老爷的宅子里。他是我的本家,比我长一辈,应该称之曰“四叔”,是一个讲理学的老监生\footnote{理学 又称道学,是宋代周敦颐、程颢、程颐、朱熹等人阐释儒家学说而形成的思想体系。它认为“理”是宇宙的本体,把“三纲五常”等封建伦理道德说成是“天理”,提出“存天理,灭人欲”的主张。监生,国子监生员的简称。国子监原是封建时代中央最高学府,清代乾隆以后可以通过援例捐资取得监生名义,不一定在监读书。}。他比先前并没有什么大改变,单是老了些,但也还未留胡子,一见面是寒暄,寒暄之后说我“胖了”,说我“胖了”之后即大骂其新党\footnote{新党 清末对主张或倾向维新的人的称呼;辛亥革命前后,也用来称呼革命党人及拥护革命的人。}。但我知道,这并非借题在骂我:因为他所骂的还是康有为\footnote{康有为(1858—1927)字广厦,号长素,广东南海人,清末维新运动领袖。他主张“变法维新”,改君主专制为君主立宪。1898年他与谭嗣同、梁启超等受光绪皇帝任用,参预政事,试行变法,因遭到以慈禧太后为首的封建顽固派的激烈反对而失败。康有为在变法失败后逃亡国外,组织保皇党,反对孙中山领导的民主革命运动;辛亥革命后又联络军阀张勋扶植清废帝溥仪复辟。}。但是,谈话是总不投机的了,于是不多久,我便一个人剩在书房里。
\par 第二天我起得很迟,午饭之后,出去看了几个本家和朋友;第三天也照样。他们也都没有什么大改变,单是老了些;家中却一律忙,都在准备着“祝福”\footnote{“祝福” 旧时江南一带每年年终的一种习俗。清代范寅《越谚·风俗》载:“祝福,岁暮谢年,谢神祖,名此。”}。这是鲁镇年终的大典,致敬尽礼,迎接福神,拜求来年一年中的好运气的。杀鸡,宰鹅,买猪肉,用心细细的洗,女人的臂膊都在水里浸得通红,有的还带着绞丝银镯子。煮熟之后,横七竖八的插些筷子在这类东西上,可就称为“福礼”了,五更天陈列起来,并且点上香烛,恭请福神们来享用;拜的却只限于男人,拜完自然仍然是放爆竹。年年如此,家家如此,——只要买得起福礼和爆竹之类的,——今年自然也如此。天色愈阴暗了,下午竟下起雪来,雪花大的有梅花那么大,满天飞舞,夹着烟霭和忙碌的气色,将鲁镇乱成一团糟。我回到四叔的书房里时,瓦楞上已经雪白,房里也映得较光明,极分明的显出壁上挂着的朱拓\footnote{朱拓 用银朱等红颜料从碑刻上拓下的文字或图形。}的大“夀”字,陈抟\footnote{陈抟(?—989)五代时亳州真源(今河南鹿邑)人。后唐长庆年间举进士不第,先后隐居武当山和华山修道。后人把他附会为“神仙”。}老祖写的;一边的对联已经脱落,松松的卷了放在长桌上,一边的还在,道是“事理通达心气和平”\footnote{“事理通达心气和平” 语出朱熹《论语集注》。朱熹在《季氏》篇中“不学诗无以言”和“不学礼无以立”语下分别注云:“事理通达而心气和平,故能言”;“品节详明而德性坚定,故能立”。}。我又无聊赖的到窗下的案头去一翻,只见一堆似乎未必完全的《康熙字典》,一部《近思录集注》和一部《四书衬》\footnote{《康熙字典》清代康熙年间张玉书、陈廷敬等奉旨编纂的一部大型字典,康熙五十五年(1716)刊行。《近思录》,是一部所谓理学入门书,宋代朱熹、吕祖谦选录周敦颐、程颢、程颐以及张载四人的文字编成,共十四卷。清初茅星来和江永分别为它作过集注。《四书衬》,清代骆培著,是一部解说“四书”(《论语》、《孟子》、《大学》、《中庸》)的书。}。无论如何,我明天决计要走了。
\par 况且,一想到昨天遇见祥林嫂的事,也就使我不能安住。那是下午,我到镇的东头访过一个朋友,走出来,就在河边遇见她;而且见她瞪着的眼睛的视线,就知道明明是向我走来的。我这回在鲁镇所见的人们中,改变之大,可以说无过于她的了:五年前的花白的头发,即今已经全白,全不像四十上下的人;脸上瘦削不堪,黄中带黑,而且消尽了先前悲哀的神色,仿佛是木刻似的;只有那眼珠间或一轮,还可以表示她是一个活物。她一手提着竹篮,内中一个破碗,空的;一手拄着一支比她更长的竹竿,下端开了裂:她分明已经纯乎是一个乞丐了。
\par 我就站住,豫备她来讨钱。
\par “你回来了?”她先这样问。
\par “是的。”
\par “这正好。你是识字的,又是出门人,见识得多。我正要问你一件事——”她那没有精采的眼睛忽然发光了。
\par 我万料不到她却说出这样的话来,诧异的站着。
\par “就是——”她走近两步,放低了声音,极秘密似的切切的说,“一个人死了之后,究竟有没有魂灵的?”
\par 我很悚然,一见她的眼钉着我的,背上也就遭了芒刺一般,比在学校里遇到不及豫防的临时考,教师又偏是站在身旁的时候,惶急得多了。对于魂灵的有无,我自己是向来毫不介意的;但在此刻,怎样回答她好呢?我在极短期的踌蹰中,想,这里的人照例相信鬼,然而她,却疑惑了,——或者不如说希望:希望其有,又希望其无……。人何必增添末路的人的苦恼,为她起见,不如说有罢。
\par “也许有罢,——我想。”我于是吞吞吐吐的说。
\par “那么,也就有地狱了?”
\par “阿!地狱?”我很吃惊,只得支梧着,“地狱?——论理,就该也有。——然而也未必,……谁来管这等事……。”
\par “那么,死掉的一家的人,都能见面的?”
\par “唉唉,见面不见面呢?……”这时我已知道自己也还是完全一个愚人,什么踌蹰,什么计画,都挡不住三句问。我即刻胆怯起来了,便想全翻过先前的话来,“那是,……实在,我说不清……。其实,究竟有没有魂灵,我也说不清。”
\par 我乘她不再紧接的问,迈开步便走,匆匆的逃回四叔的家中,心里很觉得不安逸。自己想,我这答话怕于她有些危险。她大约因为在别人的祝福时候,感到自身的寂寞了,然而会不会含有别的什么意思的呢?——或者是有了什么豫感了?倘有别的意思,又因此发生别的事,则我的答话委实该负若干的责任……。但随后也就自笑,觉得偶尔的事,本没有什么深意义,而我偏要细细推敲,正无怪教育家要说是生着神经病;而况明明说过“说不清”,已经推翻了答话的全局,即使发生什么事,于我也毫无关系了。
\par “说不清”是一句极有用的话。不更事的勇敢的少年,往往敢于给人解决疑问,选定医生,万一结果不佳,大抵反成了怨府,然而一用这说不清来作结束,便事事逍遥自在了。我在这时,更感到这一句话的必要,即使和讨饭的女人说话,也是万不可省的。
\par 但是我总觉得不安,过了一夜,也仍然时时记忆起来,仿佛怀着什么不祥的豫感;在阴沉的雪天里,在无聊的书房里,这不安愈加强烈了。不如走罢,明天进城去。福兴楼的清燉鱼翅,一元一大盘,价廉物美,现在不知增价了否?往日同游的朋友,虽然已经云散,然而鱼翅是不可不吃的,即使只有我一个……。无论如何,我明天决计要走了。
\par 我因为常见些但愿不如所料,以为未必竟如所料的事,却每每恰如所料的起来,所以很恐怕这事也一律。果然,特别的情形开始了。傍晚,我竟听到有些人聚在内室里谈话,仿佛议论什么事似的,但不一会,说话声也就止了,只有四叔且走而且高声的说:
\par “不早不迟,偏偏要在这时候,——这就可见是一个谬种!”
\par 我先是诧异,接着是很不安,似乎这话于我有关系。试望门外,谁也没有。好容易待到晚饭前他们的短工来冲茶,我才得了打听消息的机会。
\par “刚才,四老爷和谁生气呢?”我问。
\par “还不是和祥林嫂?”那短工简捷的说。
\par “祥林嫂?怎么了?”我又赶紧的问。
\par “老了。”
\par “死了?”我的心突然紧缩,几乎跳起来,脸上大约也变了色。但他始终没有抬头,所以全不觉。我也就镇定了自己,接着问:
\par “什么时候死的?”
\par “什么时候?——昨天夜里,或者就是今天罢。——我说不清。”
\par “怎么死的?”
\par “怎么死的?——还不是穷死的?”他淡然的回答,仍然没有抬头向我看,出去了。
\par 然而我的惊惶却不过暂时的事,随着就觉得要来的事,已经过去,并不必仰仗我自己的“说不清”和他之所谓“穷死的”的宽慰,心地已经渐渐轻松;不过偶然之间,还似乎有些负疚。晚饭摆出来了,四叔俨然的陪着。我也还想打听些关于祥林嫂的消息,但知道他虽然读过“鬼神者二气之良能也”\footnote{“鬼神者二气之良能也” 语出宋代张载的《张子全书·正蒙》,也见《近思录》。意思是:鬼神是阴阳二气自然变化而成的。},而忌讳仍然极多,当临近祝福时候,是万不可提起死亡疾病之类的话的;倘不得已,就该用一种替代的隐语,可惜我又不知道,因此屡次想问,而终于中止了。我从他俨然的脸色上,又忽而疑他正以为我不早不迟,偏要在这时候来打搅他,也是一个谬种,便立刻告诉他明天要离开鲁镇,进城去,趁早放宽了他的心。他也不很留。这样闷闷的吃完了一餐饭。
\par 冬季日短,又是雪天,夜色早已笼罩了全市镇。人们都在灯下匆忙,但窗外很寂静。雪花落在积得厚厚的雪褥上面,听去似乎瑟瑟有声,使人更加感得沉寂。我独坐在发出黄光的菜油灯下,想,这百无聊赖的祥林嫂,被人们弃在尘芥堆中的,看得厌倦了的陈旧的玩物,先前还将形骸露在尘芥里,从活得有趣的人们看来,恐怕要怪讶她何以还要存在,现在总算被无常\footnote{无常 佛家语,原指世间一切事物都在变异灭坏的过程中;后引申为死的意思,也用作迷信传说中“勾魂使者”的名称。}打扫得干干净净了。魂灵的有无,我不知道;然而在现世,则无聊生者不生,即使厌见者不见,为人为己,也还都不错。我静听着窗外似乎瑟瑟作响的雪花声,一面想,反而渐渐的舒畅起来。
\par 然而先前所见所闻的她的半生事迹的断片,至此也联成一片了。
\par 她不是鲁镇人。有一年的冬初,四叔家里要换女工,做中人的卫老婆子带她进来了,头上扎着白头绳,乌裙,蓝夹袄,月白背心,年纪大约二十六七,脸色青黄,但两颊却还是红的。卫老婆子叫她祥林嫂,说是自己母家的邻舍,死了当家人,所以出来做工了。四叔皱了皱眉,四婶已经知道了他的意思,是在讨厌她是一个寡妇。但看她模样还周正,手脚都壮大,又只是顺着眼,不开一句口,很像一个安分耐劳的人,便不管四叔的皱眉,将她留下了。试工期内,她整天的做,似乎闲着就无聊,又有力,简直抵得过一个男子,所以第三天就定局,每月工钱五百文。
\par 大家都叫她祥林嫂;没问她姓什么,但中人是卫家山人,既说是邻居,那大概也就姓卫了。她不很爱说话,别人问了才回答,答的也不多。直到十几天之后,这才陆续的知道她家里还有严厉的婆婆;一个小叔子,十多岁,能打柴了;她是春天没了丈夫的;他本来也打柴为生,比她小十岁:大家所知道的就只是这一点。
\par 日子很快的过去了,她的做工却毫没有懈,食物不论,力气是不惜的。人们都说鲁四老爷家里雇着了女工,实在比勤快的男人还勤快。到年底,扫尘,洗地,杀鸡,宰鹅,彻夜的煮福礼,全是一人担当,竟没有添短工。然而她反满足,口角边渐渐的有了笑影,脸上也白胖了。
\par 新年才过,她从河边淘米回来时,忽而失了色,说刚才远远地看见一个男人在对岸徘徊,很像夫家的堂伯,恐怕是正为寻她而来的。四婶很惊疑,打听底细,她又不说。四叔一知道,就皱一皱眉,道:
\par “这不好。恐怕她是逃出来的。”
\par 她诚然是逃出来的,不多久,这推想就证实了。
\par 此后大约十几天,大家正已渐渐忘却了先前的事,卫老婆子忽而带了一个三十多岁的女人进来了,说那是祥林嫂的婆婆。那女人虽是山里人模样,然而应酬很从容,说话也能干,寒暄之后,就赔罪,说她特来叫她的儿媳回家去,因为开春事务忙,而家中只有老的和小的,人手不够了。
\par “既是她的婆婆要她回去,那有什么话可说呢。”四叔说。
\par 于是算清了工钱,一共一千七百五十文,她全存在主人家,一文也还没有用,便都交给她的婆婆。那女人又取了衣服,道过谢,出去了。其时已经是正午。
\par “阿呀,米呢?祥林嫂不是去淘米的么?……”好一会,四婶这才惊叫起来。她大约有些饿,记得午饭了。
\par 于是大家分头寻淘箩。她先到厨下,次到堂前,后到卧房,全不见淘箩的影子。四叔踱出门外,也不见,直到河边,才见平平正正的放在岸上,旁边还有一株菜。
\par 看见的人报告说,河里面上午就泊了一只白篷船,篷是全盖起来的,不知道什么人在里面,但事前也没有人去理会他。待到祥林嫂出来淘米,刚刚要跪下去,那船里便突然跳出两个男人来,像是山里人,一个抱住她,一个帮着,拖进船去了。祥林嫂还哭喊了几声,此后便再没有什么声息,大约给用什么堵住了罢。接着就走上两个女人来,一个不认识,一个就是卫婆子。窥探舱里,不很分明,她像是捆了躺在船板上。
\par “可恶!然而……。”四叔说。
\par 这一天是四婶自己煮午饭;他们的儿子阿牛烧火。
\par 午饭之后,卫老婆子又来了。
\par “可恶!”四叔说。
\par “你是什么意思?亏你还会再来见我们。”四婶洗着碗,一见面就愤愤的说,“你自己荐她来,又合伙劫她去,闹得沸反盈天的,大家看了成个什么样子?你拿我们家里开玩笑么?”
\par “阿呀阿呀,我真上当。我这回,就是为此特地来说说清楚的。她来求我荐地方,我那里料得到是瞒着她的婆婆的呢。对不起,四老爷,四太太。总是我老发昏不小心,对不起主顾。幸而府上是向来宽洪大量,不肯和小人计较的。这回我一定荐一个好的来折罪……。”
\par “然而……。”四叔说。
\par 于是祥林嫂事件便告终结,不久也就忘却了。
\par 只有四婶,因为后来雇用的女工,大抵非懒即馋,或者馋而且懒,左右不如意,所以也还提起祥林嫂。每当这些时候,她往往自言自语的说,“她现在不知道怎么样了?”意思是希望她再来。但到第二年的新正,她也就绝了望。
\par 新正将尽,卫老婆子来拜年了,已经喝得醉醺醺的,自说因为回了一趟卫家山的娘家,住下几天,所以来得迟了。她们问答之间,自然就谈到祥林嫂。
\par “她么?”卫老婆子高兴的说,“现在是交了好运了。她婆婆来抓她回去的时候,是早已许给了贺家墺的贺老六的,所以回家之后不几天,也就装在花轿里抬去了。”
\par “阿呀,这样的婆婆!……”四婶惊奇的说。
\par “阿呀,我的太太!你真是大户人家的太太的话。我们山里人,小户人家,这算得什么?她有小叔子,也得娶老婆。不嫁了她,那有这一注钱来做聘礼?她的婆婆倒是精明强干的女人呵,很有打算,所以就将她嫁到里山去。倘许给本村人,财礼就不多;惟独肯嫁进深山野墺里去的女人少,所以她就到手了八十千\footnote{八十千 旧时以一千文钱为一贯或一吊,所以几千文钱也称为几贯或几吊,但也有些地方直称为多少千。八十千即八十吊。}。现在第二个儿子的媳妇也娶进了,财礼只花了五十,除去办喜事的费用,还剩十多千。吓,你看,这多么好打算?……”
\par “祥林嫂竟肯依?……”
\par “这有什么依不依。——闹是谁也总要闹一闹的;只要用绳子一捆,塞在花轿里,抬到男家,捺上花冠,拜堂,关上房门,就完事了。可是祥林嫂真出格,听说那时实在闹得利害,大家还都说大约因为在念书人家做过事,所以与众不同呢。太太,我们见得多了:回头人出嫁,哭喊的也有,说要寻死觅活的也有,抬到男家闹得拜不成天地的也有,连花烛都砸了的也有。祥林嫂可是异乎寻常,他们说她一路只是嚎,骂,抬到贺家墺,喉咙已经全哑了。拉出轿来,两个男人和她的小叔子使劲的擒住她也还拜不成天地。他们一不小心,一松手,阿呀,阿弥陀佛,她就一头撞在香案角上,头上碰了一个大窟窿,鲜血直流,用了两把香灰,包上两块红布还止不住血呢。直到七手八脚的将她和男人反关在新房里,还是骂,阿呀呀,这真是……。”她摇一摇头,顺下眼睛,不说了。
\par “后来怎么样呢?”四婶还问。
\par “听说第二天也没有起来。”她抬起眼来说。
\par “后来呢?”
\par “后来?——起来了。她到年底就生了一个孩子,男的,新年就两岁了。我在娘家这几天,就有人到贺家墺去,回来说看见他们娘儿俩,母亲也胖,儿子也胖;上头又没有婆婆;男人所有的是力气,会做活;房子是自家的。——唉唉,她真是交了好运了。”
\par 从此之后,四婶也就不再提起祥林嫂。
\par 但有一年的秋季,大约是得到祥林嫂好运的消息之后的又过了两个新年,她竟又站在四叔家的堂前了。桌上放着一个荸荠式的圆篮,檐下一个小铺盖。她仍然头上扎着白头绳,乌裙,蓝夹袄,月白背心,脸色青黄,只是两颊上已经消失了血色,顺着眼,眼角上带些泪痕,眼光也没有先前那样精神了。而且仍然是卫老婆子领着,显出慈悲模样,絮絮的对四婶说:
\par “……这实在是叫作‘天有不测风云’,她的男人是坚实人,谁知道年纪青青,就会断送在伤寒上?本来已经好了的,吃了一碗冷饭,复发了。幸亏有儿子;她又能做,打柴摘茶养蚕都来得,本来还可以守着,谁知道那孩子又会给狼衔去的呢?春天快完了,村上倒反来了狼,谁料到?现在她只剩了一个光身了。大伯来收屋,又赶她。她真是走投无路了,只好来求老主人。好在她现在已经再没有什么牵挂,太太家里又凑巧要换人,所以我就领她来。——我想,熟门熟路,比生手实在好得多……。”
\par “我真傻,真的,”祥林嫂抬起她没有神采的眼睛来,接着说。“我单知道下雪的时候野兽在山墺里没有食吃,会到村里来;我不知道春天也会有。我一清早起来就开了门,拿小篮盛了一篮豆,叫我们的阿毛坐在门槛上剥豆去。他是很听话的,我的话句句听;他出去了。我就在屋后劈柴,淘米,米下了锅,要蒸豆。我叫阿毛,没有应,出去一看,只见豆撒得一地,没有我们的阿毛了。他是不到别家去玩的;各处去一问,果然没有。我急了,央人出去寻。直到下半天,寻来寻去寻到山墺里,看见刺柴上挂着一只他的小鞋。大家都说,糟了,怕是遭了狼了。再进去;他果然躺在草窠里,肚里的五脏已经都给吃空了,手上还紧紧的捏着那只小篮呢。……”她接着但是呜咽,说不出成句的话来。
\par 四婶起初还踌蹰,待到听完她自己的话,眼圈就有些红了。她想了一想,便教拿圆篮和铺盖到下房去。卫老婆子仿佛卸了一肩重担似的嘘一口气;祥林嫂比初来时候神气舒畅些,不待指引,自己驯熟的安放了铺盖。她从此又在鲁镇做女工了。
\par 大家仍然叫她祥林嫂。
\par 然而这一回,她的境遇却改变得非常大。上工之后的两三天,主人们就觉得她手脚已没有先前一样灵活,记性也坏得多,死尸似的脸上又整日没有笑影,四婶的口气上,已颇有些不满了。当她初到的时候,四叔虽然照例皱过眉,但鉴于向来雇用女工之难,也就并不大反对,只是暗暗地告诫四婶说,这种人虽然似乎很可怜,但是败坏风俗的,用她帮忙还可以,祭祀时候可用不着她沾手,一切饭菜,只好自己做,否则,不干不净,祖宗是不吃的。
\par 四叔家里最重大的事件是祭祀,祥林嫂先前最忙的时候也就是祭祀,这回她却清闲了。桌子放在堂中央,系上桌帏,她还记得照旧的去分配酒杯和筷子。
\par “祥林嫂,你放着罢!我来摆。”四婶慌忙的说。
\par 她讪讪的缩了手,又去取烛台。
\par “祥林嫂,你放着罢!我来拿。”四婶又慌忙的说。
\par 她转了几个圆圈,终于没有事情做,只得疑惑的走开。她在这一天可做的事是不过坐在灶下烧火。
\par 镇上的人们也仍然叫她祥林嫂,但音调和先前很不同;也还和她讲话,但笑容却冷冷的了。她全不理会那些事,只是直着眼睛,和大家讲她自己日夜不忘的故事:
\par “我真傻,真的,”她说。“我单知道雪天是野兽在深山里没有食吃,会到村里来;我不知道春天也会有。我一大早起来就开了门,拿小篮盛了一篮豆,叫我们的阿毛坐在门槛上剥豆去。他是很听话的孩子,我的话句句听;他就出去了。我就在屋后劈柴,淘米,米下了锅,打算蒸豆。我叫,‘阿毛!’没有应。出去一看,只见豆撒得满地,没有我们的阿毛了。各处去一问,都没有。我急了,央人去寻去。直到下半天,几个人寻到山墺里,看见刺柴上挂着一只他的小鞋。大家都说,完了,怕是遭了狼了。再进去;果然,他躺在草窠里,肚里的五脏已经都给吃空了,可怜他手里还紧紧的捏着那只小篮呢。……”她于是淌下眼泪来,声音也呜咽了。
\par 这故事倒颇有效,男人听到这里,往往敛起笑容,没趣的走了开去;女人们却不独宽恕了她似的,脸上立刻改换了鄙薄的神气,还要陪出许多眼泪来。有些老女人没有在街头听到她的话,便特意寻来,要听她这一段悲惨的故事。直到她说到呜咽,她们也就一齐流下那停在眼角上的眼泪,叹息一番,满足的去了,一面还纷纷的评论着。
\par 她就只是反复的向人说她悲惨的故事,常常引住了三五个人来听她。但不久,大家也都听得纯熟了,便是最慈悲的念佛的老太太们,眼里也再不见有一点泪的痕迹。后来全镇的人们几乎都能背诵她的话,一听到就烦厌得头痛。
\par “我真傻,真的,”她开首说。
\par “是的,你是单知道雪天野兽在深山里没有食吃,才会到村里来的。”他们立即打断她的话,走开去了。
\par 她张着口怔怔的站着,直着眼睛看他们,接着也就走了,似乎自己也觉得没趣。但她还妄想,希图从别的事,如小篮,豆,别人的孩子上,引出她的阿毛的故事来。倘一看见两三岁的小孩子,她就说:
\par “唉唉,我们的阿毛如果还在,也就有这么大了。……”
\par 孩子看见她的眼光就吃惊,牵着母亲的衣襟催她走。于是又只剩下她一个,终于没趣的也走了。后来大家又都知道了她的脾气,只要有孩子在眼前,便似笑非笑的先问她,道:
\par “祥林嫂,你们的阿毛如果还在,不是也就有这么大了么?”
\par 她未必知道她的悲哀经大家咀嚼赏鉴了许多天,早已成为渣滓,只值得烦厌和唾弃;但从人们的笑影上,也仿佛觉得这又冷又尖,自己再没有开口的必要了。她单是一瞥他们,并不回答一句话。
\par 鲁镇永远是过新年,腊月二十以后就忙起来了。四叔家里这回须雇男短工,还是忙不过来,另叫柳妈做帮手,杀鸡,宰鹅;然而柳妈是善女人\footnote{善女人 佛家语,指信佛的女人。},吃素,不杀生的,只肯洗器皿。祥林嫂除烧火之外,没有别的事,却闲着了,坐着只看柳妈洗器皿。微雪点点的下来了。
\par “唉唉,我真傻,”祥林嫂看了天空,叹息着,独语似的说。
\par “祥林嫂,你又来了。”柳妈不耐烦的看着她的脸,说。“我问你:你额角上的伤疤,不就是那时撞坏的么?”
\par “唔唔。”她含胡的回答。
\par “我问你:你那时怎么后来竟依了呢?”
\par “我么?……”
\par “你呀。我想:这总是你自己愿意了,不然……。”
\par “阿阿,你不知道他力气多么大呀。”
\par “我不信。我不信你这么大的力气,真会拗他不过。你后来一定是自己肯了,倒推说他力气大。”
\par “阿阿,你……你倒自己试试看。”她笑了。
\par 柳妈的打皱的脸也笑起来,使她蹙缩得像一个核桃;干枯的小眼睛一看祥林嫂的额角,又钉住她的眼。祥林嫂似乎很局促了,立刻敛了笑容,旋转眼光,自去看雪花。
\par “祥林嫂,你实在不合算。”柳妈诡秘的说。“再一强,或者索性撞一个死,就好了。现在呢,你和你的第二个男人过活不到两年,倒落了一件大罪名。你想,你将来到阴司去,那两个死鬼的男人还要争,你给了谁好呢?阎罗大王只好把你锯开来,分给他们。我想,这真是……。”
\par 她脸上就显出恐怖的神色来,这是在山村里所未曾知道的。
\par “我想,你不如及早抵当。你到土地庙里去捐一条门槛,当作你的替身,给千人踏,万人跨,赎了这一世的罪名,免得死了去受苦。”
\par 她当时并不回答什么话,但大约非常苦闷了,第二天早上起来的时候,两眼上便都围着大黑圈。早饭之后,她便到镇的西头的土地庙里去求捐门槛。庙祝\footnote{庙祝 旧时庙宇中管理香火的人。}起初执意不允许,直到她急得流泪,才勉强答应了。价目是大钱十二千。
\par 她久已不和人们交口,因为阿毛的故事是早被大家厌弃了的;但自从和柳妈谈了天,似乎又即传扬开去,许多人都发生了新趣味,又来逗她说话了。至于题目,那自然是换了一个新样,专在她额上的伤疤。
\par “祥林嫂,我问你:你那时怎么竟肯了?”一个说。
\par “唉,可惜,白撞了这一下。”一个看着她的疤,应和道。
\par 她大约从他们的笑容和声调上,也知道是在嘲笑她,所以总是瞪着眼睛,不说一句话,后来连头也不回了。她整日紧闭了嘴唇,头上带着大家以为耻辱的记号的那伤痕,默默的跑街,扫地,洗菜,淘米。快够一年,她才从四婶手里支取了历来积存的工钱,换算了十二元鹰洋\footnote{鹰洋 指墨西哥银元,币面铸有鹰的图案。鸦片战争后曾大量流入我国。},请假到镇的西头去。但不到一顿饭时候,她便回来,神气很舒畅,眼光也分外有神,高兴似的对四婶说,自己已经在土地庙捐了门槛了。
\par 冬至的祭祖时节,她做得更出力,看四婶装好祭品,和阿牛将桌子抬到堂屋中央,她便坦然的去拿酒杯和筷子。
\par “你放着罢,祥林嫂!”四婶慌忙大声说。
\par 她像是受了炮烙\footnote{炮烙 亦作炮格,相传为殷纣王时的一种酷刑。据《史记·殷本纪》裴骃集解引《列女传》:“膏铜柱,下加之炭,令有罪者行焉,辄堕炭中,妲己笑,名曰炮格之刑。”}似的缩手,脸色同时变作灰黑,也不再去取烛台,只是失神的站着。直到四叔上香的时候,教她走开,她才走开。这一回她的变化非常大,第二天,不但眼睛窈陷下去,连精神也更不济了。而且很胆怯,不独怕暗夜,怕黑影,即使看见人,虽是自己的主人,也总惴惴的,有如在白天出穴游行的小鼠;否则呆坐着,直是一个木偶人。不半年,头发也花白起来了,记性尤其坏,甚而至于常常忘却了去淘米。
\par “祥林嫂怎么这样了?倒不如那时不留她。”四婶有时当面就这样说,似乎是警告她。
\par 然而她总如此,全不见有怜悧起来的希望。他们于是想打发她走了,教她回到卫老婆子那里去。但当我还在鲁镇的时候,不过单是这样说;看现在的情状,可见后来终于实行了。然而她是从四叔家出去就成了乞丐的呢,还是先到卫老婆子家然后再成乞丐的呢?那我可不知道。
\par 我给那些因为在近旁而极响的爆竹声惊醒,看见豆一般大的黄色的灯火光,接着又听得毕毕剥剥的鞭炮,是四叔家正在“祝福”了;知道已是五更将近时候。我在蒙胧中,又隐约听到远处的爆竹声联绵不断,似乎合成一天音响的浓云,夹着团团飞舞的雪花,拥抱了全市镇。我在这繁响的拥抱中,也懒散而且舒适,从白天以至初夜的疑虑,全给祝福的空气一扫而空了,只觉得天地圣众歆享了牲醴和香烟,都醉醺醺的在空中蹒跚,豫备给鲁镇的人们以无限的幸福。
\par \rightline{一九二四年二月七日。}






