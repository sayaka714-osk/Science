



\section{亲爱的三毛}


\par 书名:亲爱的三毛
\par 作者:三毛
\par 出版社:北京出版集团北京十月文艺出版社
\par 出版时间:2017-03
\par ISBN:9787530214794



\refdocument{
    \par 在这个日渐快速的时代里,我张望街头,每每看见一张张冷漠麻木、没有表情的面容匆匆行过。我总是警惕自己,不要因为长时间生活在这般的大环境里,不知不觉也变成了那其中的一个。他们使我黯然到不太敢照影子。
    \par 也许,透过书信呼应的方式,加上声音,我们人和人之间,所竖立起来的高墙,能够成为透明的。或说,不必那么晶莹剔透,或而有些光线照亮一霎间幽暗的心灵,带来一丝欣慰,然后再不打扰,各自安静存活。
    \par \rightline{三毛}
}







\subsection{谈心}


\refdocument{
    \par 借着读者朋友的来信,
    \par 看见了本身的不足和缺点,
    \par 这些信件,
    \par 是一面又一面明镜,
    \par 擦拂了我朦胧的内心。
    \par 这份收获,
    \par 是读者给予的,
    \par 谢谢来信共勉。
}




\subsubsection{三毛信箱}

\par 两年多前,我刚从远地做了一场长长的旅行回来。为着说说远方的故事,去了台中。
\par 也就是在台中那一场公开谈话结束之后,“明道文艺社”的社长,老友陈宪仁兄邀我次日清晨去一趟设在台中县乌日乡的明道高级中学,说校长汪广平先生很喜欢我去参加学校的升旗典礼,如果能够去一趟,是十分欢迎的。汪校长自然是早已认识的长辈。
\par 当时,立即就答应了,可是为着早起这桩事情,担了一夜的心,生怕睡了就醒不来,所以没有敢睡,一直等着天亮。
\par 生平怕的事情不多,可是最怕学校和老师。这和我当年是个逃学生当然有着不可分割的心理因素。
\par 明道中学是台湾中部著名的好学校,去了更心虚。升“国旗”、唱“国歌”,面对着那大操场上的师长和同学,我都站得正正的,动都不敢动。就是身上那条蓝布裤子看上去不合校规,弄得十分不自在,而那次去台中,没有带裙子。
\par 升完了旗,汪校长笑眯眯地突然点到我的名字,说请上台去讲十分钟的话。当时,我没法逃掉,吓得很厉害,因为校长怎么上千百人都不点名,光就点了我——而且笑笑的。
\par 只有一步一步上去了,心里一直想古时的曹植、曹植,走了七步路出来了一首诗,那么我走了几步可以上台去讲十分钟的话?那么多精明的老师都在看着我,笑笑的。
\par 就说了,说五分钟话送给女生,另外五分钟给男生。十分钟整,下台鞠躬。
\par 说完校长请同学们乖乖回教室去上课——好孩子的一天开始了。又说,要同学跟三毛姐姐道个早安加再见吧!
\par 才说呢,一霎间,男生的帽子哗一下丢上了天空,朝阳下蓝天里,就看见一群飞鸽似的帽子漫天翻舞,夹着女生的尖叫——就在校长和老师们的面前。
\par 当时,嗳!我笑湿了眼眶——为着这不同的一个时代和少年。在我的时代里,哪有这种师生的场面?
\par 以后,想起乌日乡,总看见听见晴空里那些帽子在尖叫。
\par 后来,宪仁兄问我给不给明道的弟弟妹妹们写些东西?我猛点头,说:“写好了!当然写!”
\par 《明道文艺》是一份极好的刊物,这许多年来,坚守着明确的方向默默耕耘。它不只是一份最好的学校刊物,也是社会上一股难得的清流,校外订阅的人也是极多。
\par 就这么,“三毛信箱”,因为个人深喜《明道文艺》的风格,也就一期一期地写了下来。
\par 感谢宪仁兄的鼓励,使得一向最懒于回信的我,回出了一些比较具有建设性的读者来信。
\par 其实,回信之后,受善最多的人,可能还是我自己。借着读者朋友的来信,看见了本身的不足和缺点,这些信件,是一面又一面明镜,擦拂了我朦胧的内心。这份收获,是读者给予的,谢谢来信共勉。
\par (本篇原为台湾皇冠出版社三毛全集《谈心》后记,本书中改为此名。)


\subsubsection{自爱而不自怜}

\refdocument{
    \par \leftline{三毛大姐您好:}
    \par 前些日子在城区部参加了您的座谈,一直有股冲动想写信给您,虽然料必此种来信您定看得不胜其烦,但相信您定能深切了解一个不快乐者的心情,因此很抱歉又给您增添麻烦,只希望能借您的指点,给我精神上的鼓舞。
    \par 我是淡江夜间部的学生。基于那种对自我的期许,我参加了大学联考。现在我正积极地准备托福,由于英文程度不挺好,因而让自己搞得好累,有不胜负荷之感。出国留学的真正目的为何?我真的不知道,可能就只为了逞强吧!由于自小好胜心强,再加上感情的挫折,让我一直有股“向上爬”的意愿,三毛姐,别劝我放弃出国,因为这是不可能的。
    \par 在座谈会中,您提到“我真的很不快乐”。我好感动,您知道吗?因为我也觉得自己好孤单,好寂寞。三毛姐,您能否告诉我,是什么力量支持您孤独地浪迹天涯?您精神上的寄托为何?既然您不快乐,难道不曾想过以死作为解脱(很抱歉我直言)?
    \par 三毛姐,原谅我的用词不当和辞不达意。我心里一直很苦闷,但是没人能指点我,再下去,准上松山精神病院。
    \par 三毛姐,不管您有多忙,请您务必给我回信好吗?但,请您不要劝我放弃出国的念头,我现在所需要的是您的鼓励,我也想去尝尝那种独在异乡为异客的感觉。再次声明,绝非意气用事。
    \par 附上相片一张,看看我该是何种人物?当然最重要的,为了寄回相片您就得给我回信的,不是吗?先谢了,三毛姐!
    \par \rightline{陈惠凤}
}

\par \leftline{陈小姐:}
\par 你的照片寄回,请查收。
\par 为了讨回这张照片而强迫一个人回信,是勉强他人的行为。可是看了内容之后,仍然感谢你对我的信任,不由得想写几句话给你。
\par 你的来信很不快乐,个性看似倔强,又没有执著的目标和对象,对前途一片茫然,却又在积极预备托福考试。
\par 照片中的你,看上去清秀又哀愁。没有直直地站着,靠在一棵树上。姿势是靠着,感觉却不能放松,不只是因为面对镜头,而是根本不能放松。两手握着书本,不是扎扎实实地握,而是像一件道具似的在做样子。
\par 要我由照片中看看你是什么样的人,这实在不很容易,可是您的身体语言,毕竟也说明了一些藏着的东西。眼神很弱,里面没有确定的自信和追求。这一点,观察十分主观,请原谅。(我猜,这是一张你自己较满意的照片。)
\par 事实上,没有一个人是禁得起分析的,能够试着了解,已是不容易了。
\par 来信中,两度提起:“别劝我放弃出国,这是不可能的。”事实上我并不认识你,也没有任何权利劝导别人的选择。而你,潜意识里,可能对出国之事仍有迷茫,便肯定那一份否决会在我的回信中出现,因此自己便先问了,又替我回答了。(其实是你自己在挣扎。)
\par 你说:“出国留学的真正目的可能就只是为了逞强。”我看了心里十分惊讶。又说:“一直有股向上爬的意愿。”而结论是,出国就是向上爬,又使我十二分地诧异。
\par 在我的人生观里,向上爬,逞强,都不是以出不出国为准则的。我以为,不断地自我突破,自我调整,自我修正,才是一生中向上爬的力量。
\par 如果,一个人,在台湾不能快乐,不能有自信,那么到了国外,便能因为出过国,而有所改变,有所肯定吗?或者,是不是我们少数人,有着不能解释的民族自卑,而觉得到国外去,便是一种自我价值的再肯定呢?很抱歉我的直言,因为你恰好问到了我。
\par 从另一个角度来看,能到国外去体验一下不同的风俗人情,也是可贵的。至于“也想尝尝异乡为客的感觉”,这个“也”字,其实并不可能每一个人都相同。再说,国外居,大不易,除了捕捉一份感觉之外,自己的语文条件、能力、健康,甚而谋生的本事,都是很现实而不那么浪漫的事情,请先有些心理准备和认识才去。
\par 是的,在座谈会上,我曾经说过,我的日子不是每天都快乐,而且有时因为压力大,非常不快乐。许多时候,我的不快乐,并不是因为寂寞,而是太多的“不得已”没法冲破,太多的兴趣和追求,因为时间不够用,而不得不割舍。事实上,我十分安然于一本好书、一个长夜和一杯热茶的宁静生活。对于人生,这已是很大的福分,因为我们没有生活在战乱和极权统治的国家里,这份自由,是我十分感激而珍爱的。不敢再多求什么了,只求时间的安排上,能够稍稍宽裕一点就好了。
\par 是什么支持我浪迹天涯?是求知欲,是自信,更是“万物静观皆自得”的对大地万物的那份欣赏。
\par 你又问我,不快乐的时候,有没有想到过以死为解脱?我很诚实地答复你:有过,有过两次。可是当时年纪小,不懂得——死,并不是解脱,而是逃避。
\par 我也反问,一个叫我三毛姐姐的大学生:如果你,有死的勇气,难道没有活的勇气吗?
\par 请你,担负起对自己的责任来,不但是活着就算了,更要活得热烈而起劲,不要懦弱,更不要别人太多的指引。每一天,活得踏实,将分内的工作,做得尽自己能力之内的完美,就无愧于天地。
\par 请不要怪责我这种回信的方法,孩子,你太没有自信,也太要听别人的话了,有些自怜,更有些作茧自缚。请放开眼去望一望,这个世界上,有多少事物和人,是值得我们去真诚地付出,也值得真诚地去投入——这里面,也包括你自己。请不要小看了自己,试着自爱,而不是自怜,去试试看,好不好?
\par 松山精神病院不必再去想它,这又是自我逃避的一个地方。国外是,松山又是,却不知,逃来逃去,逃不出自己的心魔。
\par 天下本无事,庸人自扰之。以这句话,与你共同勉励,因为我自己,也有想不开的时候,也有挣不脱的枷。我们一同海阔天空地做做人,试一试,请你,也是请我自己。
\par 最后,我很想说的是:一个人,有他本身的物质基础和基因。如果我们身体好一点,强壮些,许多烦恼和神经质的反应,都会比较容易对付,这便必需一个健康的身体来支持我们。
\par 你做不做运动?散不散步?有没有每天大笑三次?有没有深呼吸?吃得够不够营养?以上都是快乐的泉源之一二,请一定试试看。请试半个月,看看有没有改变好吗?
\par 照片上的你,十分孱弱,再胖些或再精神些,心情必然有些转变的。
\par 这封信回得很长,因为太多此类的来信,多多少少都是想要求鼓励与指引。
\par 我的看法是,我们活着,要求他人的帮助是很自然的事情,但是无论如何,他人告诉你一件事情或由你自己去了解一件事情,在本质上是不相同的。了解自己是由内而来的,当你了解了,不必别人来指引,也便能明白。除了你自己之外,没有人能替你找出生命之路。
\par 谢谢你!祝
\par 健康快乐
\par \rightline{三毛上}
\par 又及:如果你观察了自己几个月,发觉情绪的低潮是周期性的,那么可能是生理上的情形。医生可以帮助我们解决许多病状,心理的和生理的,请你再想想好吗?

\subsubsection{}




\subsubsection{}




\subsubsection{}




\subsubsection{}




\subsubsection{}




\subsubsection{}




\subsubsection{}




\subsubsection{}




\subsubsection{}




\subsubsection{}




\subsubsection{}




\subsubsection{}




\subsubsection{}




\subsubsection{}




\subsubsection{}




\subsubsection{}




\subsubsection{}




\subsubsection{}




\subsubsection{}




\subsubsection{}




\subsubsection{}




\subsubsection{}




\subsubsection{}




\subsubsection{}




\subsubsection{}




\subsubsection{}




\subsubsection{}




\subsubsection{}




\subsubsection{}




\subsubsection{}




\subsubsection{}




\subsubsection{}




\subsubsection{}




















\subsection{随想}

\refdocument{
    \par 你快乐吗?
    \par 你快乐吗?你快乐吗?
    \par 试试看,每天吃一颗糖,
    \par 然后告诉自己——
    \par 今天的日子,果然又是甜的。
}


\subsubsection{孩子}


\par 没有孩子的女人是特别受祝福的。
\par 养一个小人,没有问题。
\par 为这份爱,担一生一世的心,
\par 担不起。
\par 
\par 遇到不能解决的事情,去问孩子,
\par 孩子脱口而出的意见,
\par 往往就是最精确而实际的答案。
\par 
\par 成年人最幼稚的想法就是——小孩子又懂得什么?
\par 其实,大半的孩子都不很享受作为一个孩子的滋味。
\par 这种情形,在中国偏又多些。
\par 
\par 适度地责骂孩子,可能使孩子的心灵更有安全感。
\par 
\par 中国夫妇,
\par 对于不圆满的婚姻,大半采取瓦全。
\par 理由是——为了孩子。
\par 欧美父母,
\par 处理不愉快的结合,常常宁愿玉碎。
\par 理由也是——为了孩子。
\par 
\par 孩子并不以为自己小,是大人一再灌输大小的观念,
\par 才造成孩子的“承认事实”。
\par 
\par 童年,只有在回忆中显现时,
\par 才成就了那份完美。


\subsubsection{快乐}

\par 比较快乐的人生看法,
\par 在于起床时,对于将临的一日,
\par 没有那么深沉的算计。
\par 
\par 完全没有缺乏的人,也不可能再有更多的快乐了。
\par 
\par 快乐是一种等待的过程。
\par 突然而来的所谓“惊喜”,事实上叫人手足无措。
\par 
\par 一般性的快乐往往可以言传。
\par 真正深刻的快乐,没有可能使得他人意会。
\par 快乐和悲伤都是寂寞。
\par 
\par 快乐是不堪闻问的鬼东西,
\par 如果不相信,请问自己三遍——
\par 我快乐吗?
\par 
\par 快乐是另外一件国王的新衣。
\par 这一回,如果国王穿着它出来游街,大家都笑死了——
\par 笑一个国王怎么不穿衣服出来乱跑呀!
\par 
\par 你快乐吗?
\par 你快乐吗?你快乐吗?
\par 
\par 试试看,每天吃一颗糖,
\par 然后告诉自己——
\par 今天的日子,果然又是甜的。


\subsubsection{}




