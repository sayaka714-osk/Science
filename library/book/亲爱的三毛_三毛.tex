



\section{亲爱的三毛}


\par 书名:亲爱的三毛
\par 作者:三毛
\par 出版社:北京出版集团北京十月文艺出版社
\par 出版时间:2017-03
\par ISBN:9787530214794



\refdocument{
    \par 在这个日渐快速的时代里,我张望街头,每每看见一张张冷漠麻木、没有表情的面容匆匆行过。我总是警惕自己,不要因为长时间生活在这般的大环境里,不知不觉也变成了那其中的一个。他们使我黯然到不太敢照影子。
    \par 也许,透过书信呼应的方式,加上声音,我们人和人之间,所竖立起来的高墙,能够成为透明的。或说,不必那么晶莹剔透,或而有些光线照亮一霎间幽暗的心灵,带来一丝欣慰,然后再不打扰,各自安静存活。
    \par \rightline{三毛}
}







\subsection{谈心}


\refdocument{
    \par 借着读者朋友的来信,
    \par 看见了本身的不足和缺点,
    \par 这些信件,
    \par 是一面又一面明镜,
    \par 擦拂了我朦胧的内心。
    \par 这份收获,
    \par 是读者给予的,
    \par 谢谢来信共勉。
}




\subsubsection{三毛信箱}

\par 两年多前,我刚从远地做了一场长长的旅行回来。为着说说远方的故事,去了台中。
\par 也就是在台中那一场公开谈话结束之后,“明道文艺社”的社长,老友陈宪仁兄邀我次日清晨去一趟设在台中县乌日乡的明道高级中学,说校长汪广平先生很喜欢我去参加学校的升旗典礼,如果能够去一趟,是十分欢迎的。汪校长自然是早已认识的长辈。
\par 当时,立即就答应了,可是为着早起这桩事情,担了一夜的心,生怕睡了就醒不来,所以没有敢睡,一直等着天亮。
\par 生平怕的事情不多,可是最怕学校和老师。这和我当年是个逃学生当然有着不可分割的心理因素。
\par 明道中学是台湾中部著名的好学校,去了更心虚。升“国旗”、唱“国歌”,面对着那大操场上的师长和同学,我都站得正正的,动都不敢动。就是身上那条蓝布裤子看上去不合校规,弄得十分不自在,而那次去台中,没有带裙子。
\par 升完了旗,汪校长笑眯眯地突然点到我的名字,说请上台去讲十分钟的话。当时,我没法逃掉,吓得很厉害,因为校长怎么上千百人都不点名,光就点了我——而且笑笑的。
\par 只有一步一步上去了,心里一直想古时的曹植、曹植,走了七步路出来了一首诗,那么我走了几步可以上台去讲十分钟的话?那么多精明的老师都在看着我,笑笑的。
\par 就说了,说五分钟话送给女生,另外五分钟给男生。十分钟整,下台鞠躬。
\par 说完校长请同学们乖乖回教室去上课——好孩子的一天开始了。又说,要同学跟三毛姐姐道个早安加再见吧!
\par 才说呢,一霎间,男生的帽子哗一下丢上了天空,朝阳下蓝天里,就看见一群飞鸽似的帽子漫天翻舞,夹着女生的尖叫——就在校长和老师们的面前。
\par 当时,嗳!我笑湿了眼眶——为着这不同的一个时代和少年。在我的时代里,哪有这种师生的场面?
\par 以后,想起乌日乡,总看见听见晴空里那些帽子在尖叫。
\par 后来,宪仁兄问我给不给明道的弟弟妹妹们写些东西?我猛点头,说:“写好了!当然写!”
\par 《明道文艺》是一份极好的刊物,这许多年来,坚守着明确的方向默默耕耘。它不只是一份最好的学校刊物,也是社会上一股难得的清流,校外订阅的人也是极多。
\par 就这么,“三毛信箱”,因为个人深喜《明道文艺》的风格,也就一期一期地写了下来。
\par 感谢宪仁兄的鼓励,使得一向最懒于回信的我,回出了一些比较具有建设性的读者来信。
\par 其实,回信之后,受善最多的人,可能还是我自己。借着读者朋友的来信,看见了本身的不足和缺点,这些信件,是一面又一面明镜,擦拂了我朦胧的内心。这份收获,是读者给予的,谢谢来信共勉。
\par (本篇原为台湾皇冠出版社三毛全集《谈心》后记,本书中改为此名。)


\subsubsection{自爱而不自怜}

\refdocument{
    \par \leftline{三毛大姐您好:}
    \par 前些日子在城区部参加了您的座谈,一直有股冲动想写信给您,虽然料必此种来信您定看得不胜其烦,但相信您定能深切了解一个不快乐者的心情,因此很抱歉又给您增添麻烦,只希望能借您的指点,给我精神上的鼓舞。
    \par 我是淡江夜间部的学生。基于那种对自我的期许,我参加了大学联考。现在我正积极地准备托福,由于英文程度不挺好,因而让自己搞得好累,有不胜负荷之感。出国留学的真正目的为何?我真的不知道,可能就只为了逞强吧!由于自小好胜心强,再加上感情的挫折,让我一直有股“向上爬”的意愿,三毛姐,别劝我放弃出国,因为这是不可能的。
    \par 在座谈会中,您提到“我真的很不快乐”。我好感动,您知道吗?因为我也觉得自己好孤单,好寂寞。三毛姐,您能否告诉我,是什么力量支持您孤独地浪迹天涯?您精神上的寄托为何?既然您不快乐,难道不曾想过以死作为解脱(很抱歉我直言)?
    \par 三毛姐,原谅我的用词不当和辞不达意。我心里一直很苦闷,但是没人能指点我,再下去,准上松山精神病院。
    \par 三毛姐,不管您有多忙,请您务必给我回信好吗?但,请您不要劝我放弃出国的念头,我现在所需要的是您的鼓励,我也想去尝尝那种独在异乡为异客的感觉。再次声明,绝非意气用事。
    \par 附上相片一张,看看我该是何种人物?当然最重要的,为了寄回相片您就得给我回信的,不是吗?先谢了,三毛姐!
    \par \rightline{陈惠凤}
}

\par \leftline{陈小姐:}
\par 你的照片寄回,请查收。
\par 为了讨回这张照片而强迫一个人回信,是勉强他人的行为。可是看了内容之后,仍然感谢你对我的信任,不由得想写几句话给你。
\par 你的来信很不快乐,个性看似倔强,又没有执著的目标和对象,对前途一片茫然,却又在积极预备托福考试。
\par 照片中的你,看上去清秀又哀愁。没有直直地站着,靠在一棵树上。姿势是靠着,感觉却不能放松,不只是因为面对镜头,而是根本不能放松。两手握着书本,不是扎扎实实地握,而是像一件道具似的在做样子。
\par 要我由照片中看看你是什么样的人,这实在不很容易,可是您的身体语言,毕竟也说明了一些藏着的东西。眼神很弱,里面没有确定的自信和追求。这一点,观察十分主观,请原谅。(我猜,这是一张你自己较满意的照片。)
\par 事实上,没有一个人是禁得起分析的,能够试着了解,已是不容易了。
\par 来信中,两度提起:“别劝我放弃出国,这是不可能的。”事实上我并不认识你,也没有任何权利劝导别人的选择。而你,潜意识里,可能对出国之事仍有迷茫,便肯定那一份否决会在我的回信中出现,因此自己便先问了,又替我回答了。(其实是你自己在挣扎。)
\par 你说:“出国留学的真正目的可能就只是为了逞强。”我看了心里十分惊讶。又说:“一直有股向上爬的意愿。”而结论是,出国就是向上爬,又使我十二分地诧异。
\par 在我的人生观里,向上爬,逞强,都不是以出不出国为准则的。我以为,不断地自我突破,自我调整,自我修正,才是一生中向上爬的力量。
\par 如果,一个人,在台湾不能快乐,不能有自信,那么到了国外,便能因为出过国,而有所改变,有所肯定吗?或者,是不是我们少数人,有着不能解释的民族自卑,而觉得到国外去,便是一种自我价值的再肯定呢?很抱歉我的直言,因为你恰好问到了我。
\par 从另一个角度来看,能到国外去体验一下不同的风俗人情,也是可贵的。至于“也想尝尝异乡为客的感觉”,这个“也”字,其实并不可能每一个人都相同。再说,国外居,大不易,除了捕捉一份感觉之外,自己的语文条件、能力、健康,甚而谋生的本事,都是很现实而不那么浪漫的事情,请先有些心理准备和认识才去。
\par 是的,在座谈会上,我曾经说过,我的日子不是每天都快乐,而且有时因为压力大,非常不快乐。许多时候,我的不快乐,并不是因为寂寞,而是太多的“不得已”没法冲破,太多的兴趣和追求,因为时间不够用,而不得不割舍。事实上,我十分安然于一本好书、一个长夜和一杯热茶的宁静生活。对于人生,这已是很大的福分,因为我们没有生活在战乱和极权统治的国家里,这份自由,是我十分感激而珍爱的。不敢再多求什么了,只求时间的安排上,能够稍稍宽裕一点就好了。
\par 是什么支持我浪迹天涯?是求知欲,是自信,更是“万物静观皆自得”的对大地万物的那份欣赏。
\par 你又问我,不快乐的时候,有没有想到过以死为解脱?我很诚实地答复你:有过,有过两次。可是当时年纪小,不懂得——死,并不是解脱,而是逃避。
\par 我也反问,一个叫我三毛姐姐的大学生:如果你,有死的勇气,难道没有活的勇气吗?
\par 请你,担负起对自己的责任来,不但是活着就算了,更要活得热烈而起劲,不要懦弱,更不要别人太多的指引。每一天,活得踏实,将分内的工作,做得尽自己能力之内的完美,就无愧于天地。
\par 请不要怪责我这种回信的方法,孩子,你太没有自信,也太要听别人的话了,有些自怜,更有些作茧自缚。请放开眼去望一望,这个世界上,有多少事物和人,是值得我们去真诚地付出,也值得真诚地去投入——这里面,也包括你自己。请不要小看了自己,试着自爱,而不是自怜,去试试看,好不好?
\par 松山精神病院不必再去想它,这又是自我逃避的一个地方。国外是,松山又是,却不知,逃来逃去,逃不出自己的心魔。
\par 天下本无事,庸人自扰之。以这句话,与你共同勉励,因为我自己,也有想不开的时候,也有挣不脱的枷。我们一同海阔天空地做做人,试一试,请你,也是请我自己。
\par 最后,我很想说的是:一个人,有他本身的物质基础和基因。如果我们身体好一点,强壮些,许多烦恼和神经质的反应,都会比较容易对付,这便必需一个健康的身体来支持我们。
\par 你做不做运动?散不散步?有没有每天大笑三次?有没有深呼吸?吃得够不够营养?以上都是快乐的泉源之一二,请一定试试看。请试半个月,看看有没有改变好吗?
\par 照片上的你,十分孱弱,再胖些或再精神些,心情必然有些转变的。
\par 这封信回得很长,因为太多此类的来信,多多少少都是想要求鼓励与指引。
\par 我的看法是,我们活着,要求他人的帮助是很自然的事情,但是无论如何,他人告诉你一件事情或由你自己去了解一件事情,在本质上是不相同的。了解自己是由内而来的,当你了解了,不必别人来指引,也便能明白。除了你自己之外,没有人能替你找出生命之路。
\par 谢谢你!祝
\par 健康快乐
\par \rightline{三毛上}
\par 又及:如果你观察了自己几个月,发觉情绪的低潮是周期性的,那么可能是生理上的情形。医生可以帮助我们解决许多病状,心理的和生理的,请你再想想好吗?

\subsubsection{祝福中国}

\par \leftline{金门居住的先生:}
\par 您没有留下名字,信封上,只有一个邮箱号码。
\par 牛皮纸做的信,红丝线装订出来的边,一个大红盘花扣,左面一个春字,是信的外观。
\par 打开来,七个毛笔字,就只写了这两句话:“祝福中国,祝福您。”
\par 壬戌岁末的上面,一个浅红色的印章,也看不出是什么字。淡淡的红色;您故意盖淡的,那么谦虚的情怀,在一颗章里显得明明白白。
\par 受不起这么盛重的一针一线,当不起这三个字的祝福。您,没有留下名字的朋友,您的名字和颜色——就叫中国。
\par 这份宝贝,是收信中一件极品。双手捧着它,不知如何地珍爱,正如不知如何地爱中国,才叫合了一个人的心愿。
\par 我要好好地看守自己,对待自己,活得像一个唐人女子,来报答我们共同的父母。他们的名字,也叫中国,正如你我。
\par 另外,也照着没有姓名的地址回了一张信给您。一张白纸,上面没有黑字,盖的只是印章,也只是一颗我爱之如狂的章。笨笨拙拙地,刻了四个字,那便算是我的回信。您想来也收到了。
\par 再不必说什么,有心的人,我们各自在自己的岗位上去努力,就算彼此的鼓励。
\par 您懂,我也懂了。
\par 也祝福中国,祝福您。
\par \rightline{三毛敬上}


\subsubsection{人生何处不相逢}

\refdocument{
    \par \leftline{三毛小姐:}
    \par 很抱歉打扰你的时间,从来我就是很欣赏你的文章和你的个性,很早以前就想写信给你,但又怕你没时间给我回信,我今天是抱着即使得不到回信也算了的心情,来写这封信。
    \par 对了,我想请问你,嗯!一月十五日下午两点左右,你是否开车要寄放在中山堂的地下停车场?那天我和朋友刚好要过马路,这时有一部车突然停在我们旁边,不晓得为什么,一股力量吸引着我往车内看去,忽然间,我像是遇到了老朋友似的,不由自主地叫出了“三毛”,而后却站在那里不动,最后还是那位驾驶小姐挥着手要让我们先过,她温和又满脸笑容,我不晓得她是不是真的是你——“三毛”,过了马路,我仍是发呆地站在那边,我想我应该不会看错才对,照片中的你,风尘仆仆的,而车上的那位小姐,正是如此,而且更和蔼可亲,哎!我形容得不晓得是对还是错,但“三毛”在我的感觉一直是如此。三毛小姐,如果那天遇到真的是你的话,给我回封信好吗?因为我已期盼很久了,再则,那日回宿舍后,看到“联副”上有你的文章,我想你一定是回来了,最后我仍是希望你能抽空为我回封信,好让我清楚那段“奇遇”。谢谢!祝
    \par 心怡
    \par \rightline{邱兰芬 敬上}
}

\par \leftline{兰芬:}
\par 是我!
\par 再见!
\par \rightline{三毛上}


\subsubsection{隔离与沟通}

\par \leftline{亲爱的弟弟妹妹们:}
\par 一次两小时的聚会,得到了你们的友谊和雪片一般飞来的书信。在这里,我要向你们道谢这份爱护,更使我感动的是信中对我付出的那份全然的信任。
\par 以半生的生活体验来说,爱和欣赏,在我往往是容易些的,而信任一个人,却并不那么一厢情愿。起码从自己待人接物的态度上来说,我不是一个轻信的人。
\par 由此推想,各位在信中对我全心全意的信赖,也是不容易的。这使我非常难以轻易下笔回信,担心自己偶尔在一句话上的疏忽,而影响了许多年幼的心灵。
\par 归纳起大部分的来信,其中最明显的烦恼和苦闷都在于各位对家庭生活和关系的不满。我知道这些诚恳的信都是出自各位的肺腑之言,从某一个角度看来,完全是对的,一点也没有错。
\par 可是,世上的事情,并不是只有从一个角度上去观察,就能够说它是唯一的真理。如果我们去做一次家庭访问,听听父母们如何讲孩子,很可能,父母也有一大篇合理的抱怨,也会说,孩子们不了解做父母本身的种种困难和对孩子在教育方式上的挫折,也更可能,父母除了孩子之外,尚有本身的苦难与折磨要去应付。
\par 公平地说,做父母的比做孩子的,在担当人生责任上,重了许多。亲爱的孩子,试着也去分析父母和他们本身的问题,也试着去了解,你的那份学费和衣食是父母的血汗钱换来的,这么一想,养育之恩,我们都不能回报,又何忍对他们要求太多呢?
\par 往往,大部分中国的父母,将孩子当做命根,将孩子视为自己生命的延伸与继续,期望自己一生没能完成的理想和光荣,都能在孩子的身上实现。更以为,自己人生的经验,百分之百,都可以转移到教育下一代的身上去,又以为孩子是必须无条件听命于父母而不可反抗的,压力便由是产生了。
\par 这种观念,造成了父子之间的悲剧和冲突,也造成了成年人与青少年孩子之间的深沟。本来,天伦之乐是人间最可贵的一种情操和欣慰,很可惜的是,每一个家庭中,或多或少,父母子女的观念与行事为人不能完全一致,不愉快的心情也随之而来了。
\par 父母子女之间心灵上的隔离,是爱的方式不很有技巧而造成的。成年人与年轻人的未能沟通,在我个人看来,也是出于同一个字,那就是深刻的爱。
\par 我相信,天下的父母和子女,没有一个人故意存心去破坏家庭的和谐,这是不可能的。如果问题产生了,也不是刻意的行为,而是根深柢固的社会观念,因为有了时代的变迁,双方不知适时调整而造成的结果。
\par 一个问题的出现,解决的方法,不该是怨天尤人地去怪罪对方,甚而自责,而是冷静地去理出问题症结的所在,尽可能在个性上、思想上、行为及语言上,慢慢地改进,取得彼此的谅解。
\par 这件事情,不能急切,不能以火爆似的争吵去解决,更不能以离家出走,甚而激烈地试图以毁灭自己的念头去反抗,这实在是一种愚昧而无用的方式。而做孩子的,包括我自己在内,都往往选了这种笨法子,伤人害己,于任何祥和的人生都是背道而驰。
\par 耐心、韧性、谅解、宽容、包涵,都是爱的代名词。亲爱的孩子们,在你们的来信里,我或多或少是看见了这些字。可是,在来信中,也不可避免地看见了一些不讲理的父母,动手痛打孩子,不给孩子任何解释的余地,冷淡孩子,甚而父母之间大打出手,以夫妇之间的不和,怪责孩子生命的拖累……
\par 当我一次又一次拆阅来信,看见不知有多少信中写着“陈姐姐,我但愿不要回家,永远不要回那个没有温暖的家……”这样的句子时,我的心里,充满了欲哭无泪的重压。
\par 孩子,你有一个妈妈,她打你,骂你,羞辱你,也许是她情感上不平衡,也许是她不知你在身边是她的福分,这,都不能改变她仍是你妈妈的事实。
\par 试着用自己的智慧去改变父母,不要伤心。中国人的忍字是如何写的,我们都知道。学着取悦父母,念书上要出人头地,家务上尽可能帮忙,一旦如此,父母仍是特别不喜欢你,那么,爱你自己吧!好好地储备自己的知识,将来自食其力之后,父母也年老了,那时候,回家去孝顺他们,他们不可能不感谢你的孝心的。
\par 沟通,有待双方的努力——父母和子女的。而我的书信,只有做孩子的看得见,又有多大的效果呢?
\par 至于另外一些来信,父母都是爱你的,而爱的方式中少了一份对子女的信任与尊重,这个问题便比一些破碎家庭的孩子来得简单些。请相信有一日你不再是初中生或高中生,你会成长,会成熟,会有自己的人生方向。如果在一场人生的战役上打得漂亮,做得有声有色,到那时候,父母不但不会再管你,而且会以你是他们的孩子为骄傲。这种家庭问题,由另一个角度去看,便不严重了。好孩子们,父母大半的管教都是出于一片爱心,我们又何忍在方式上去怪责他们呢?
\par 讲了这种话,各位写信来的弟弟妹妹们也许会感觉到,陈姐姐是站在父母那一边的。事实上,父母的年纪已经比较大了,要改变一个成年人的观念总是困难的,而青少年的一代,都仍有极大的可塑性,在许多地方,便必须请青少年包涵父母,谅解父母,更重要的是,将来一旦本身完成学业,成家之后,也有了子女时,再不犯同样的错误,做一个开明而得子女信赖的人。
\par 我总认为,孩子可以教育,某些父母也是可以再教育的。问题是,好似双方都是坚持自己的看法,自以为是,这就难了。
\par 记得在我小学六年级毕业的那一年,因为将一本别班男同学的纪念册偷带回家,写上了几句送别的话,而被母亲搜了出来,母亲为了这一件小事情,将我关在房间里审问,弄得我因为羞愧而痛哭,并且答应悔改。
\par 后来我渐渐长大了,有了与异性的交往,也因为来往的朋友都是正正派派的好青年,我自己也主动将这些朋友带回家去请父母过目,当年害怕我变坏的父母,在无形中有了观念上的改变,再也不会一如当初般地将男女的性别看成太严重了。
\par 这只是一个小小的例子而已。其他许多事情的价值观、判断法、自主权和人生的看法,在经过了多年的沟通之后,与母亲父亲都能取得程度上的了解。所以说,我认为,教育子女是父母的责任,可是子女在家庭中被不被误解,与个人的表现也是有关的。当然,我有一对开明的好父母,这是个人极大的福分,而他们的开明之中,亦有我多年的努力。凡事禀报父母,凡事开诚布公,若有不能一致的想法而我又自认为对得起良知时,甚而勇于在良好的态度和口气下向父母辩论、讲解,请求认同。我不隐瞒、不欺骗,不将自己的想法藏在心里,都有助于父母和自己之间的认识。
\par 又有一封来信,写出了处身一个大家庭中做孩子的悲哀。看见嫂嫂对母亲的猖狂,看见哥哥的纵容妻子,看见母亲的忍辱和委屈……这封来信,写得生动而感人,是一个有着表达笔力的好孩子痛苦不平的心声,也是一篇成功的散文。
\par 大家庭的和睦与否,关联着太多人为而复杂的因素,写信来的这位孩子,因为心痛受欺压的母亲,进而对生命的公平产生了怀疑。很难过的是,这位孝顺的孩子,我不能帮助你,只有鼓励你,用功读书,出人头地,有一日进入社会时,赚钱反哺受苦的母亲,将她接出来与你同住,好好对待她,给母亲一个幸福平静的晚年。孩子,你的孝心感人,责任也重大,一个有责任的人,是可贵的,这表示她有能力担起这份责任。目前学业尚未完成,在经济上可能无力承担母亲,可是尽力去爱妈妈,下课回家去时,尽可能表现你对她的爱和看重,这对做母亲的来说,比什么都要欣慰,你目前的能力和责任便是这个。
\par 又有的信中,家庭不看重女孩子,不愿再供给念大学的学费,做女儿的来信中伤心沮丧,几乎没有了方向。好孩子,中国人有一句谚语:“行行出状元。”我个人也认为,进大学不是唯一的人生之路,请看社会上多少成功人物的学历都不显赫,可是他们成功的例子比比皆是。再说,自我教育是很重要的,如果自己不肯教育自己,一张大学文凭又能够代表什么呢?
\par 在我所知的文化大学和东海大学,工读生都在每一个角落做事,半工半读,养活自己,同时进学,这种情况也是很多的,只是在体力上要劳累些。甚而,我有两个学生,她们是高中毕业之后,先去做两年女工,然后存足了两学年的学费,再来大学进修,也是另一条可行的路。做事,是一种磨练,对任何人只有好处而无坏处。只问你吃不吃这份有代价的劳苦。
\par 孩子们,在近乎一沓书本样厚的来信里,很多人都不够快乐,不够开朗,不懂得如何从无可奈何的情况里去求取生存之道,这也是无可厚非的,因为毕竟年纪还小,生命也仍孱弱。就算我自己吧,活到半生,又能够说我了解了人生的真谛和全然地活得完美吗?
\par 既然大家都喊我陈姐姐,我便欣然答应,在这里,与各位再共同勉励一次,我们要做聪明人,做有智慧、有慈爱又肯诚实对人对己的勇者,就算天大的事情来了,也不逃避它,心平气和地为自己争取最合理的解决之道。不可以做一个弱者,凡是一不顺心便跌倒的人,是要被社会所淘汰的,做一个有弹性的人,当是我们一生追寻的目标。
\par 很抱歉不能一一回信给各位,因为从各处转来的信实在是太多了,请原谅我时间实在不够,而那份关爱各位的心怀意念,却是强烈而真诚的。再见了!祝
\par 做一个智者仁者勇者
\par \rightline{三毛上}
\par \rightline{一九八三年三月廿七日}


\subsubsection{不满、不满、不满}
\refdocument{
    \par 陈姐姐你好:
\par 我是个高中女生,心中有很多不满,好几次想去了断自己(自杀),但每次反过来想,我有去死的勇气,那何不好好地活下去,如果就这么死去,人生不是白走一遭吗?所以想通以后,“死”离我便是很遥远了。过去我曾经投书到“学生辅导中心”及“张老师信箱”,但我发觉他们都无法帮我解决困难。为什么我说我有很多不满?不是没根据的,就拿家庭来说吧!母亲是个很迷信且重男轻女的家庭主妇,她要我回家后帮做家事,这虽是应该做的,但她不为我想一想,我是个高中学生,功课越来越重,回家时的自习时间都被占了,我以后怎么上考场?我时常同她谈起,但她无法和我沟通,她根本不了解现在的孩子,我无法充分地念书,我的前途不能就这么地送掉,所以我不满。
\par 朋友方面,以前我有很多要好的朋友,现在可是一个也没有,孔子说得很对:“唯小人与女子难养也。”认识愈深就愈失望,我觉得对她们越好,相反地她们也越看不起我,以前大家总是说说笑笑的,可是现在连见了面均不打招呼,而且还被同学耍了好几次,现在我对朋友完全失去信心,虽然心有不甘,我又能如何呢?
\par 当然这只是我不满的之一二罢了,虽说家丑不可外扬,但我执著一意念,我要好好地活下去,所以我将它说出来。
}
\par \leftline{孩子:}
\par 在你的来信中,我好似看见自己过去的影子,心里感触很深。
\par 我也曾经有过这样的少年时期,觉得全世界的人都不了解我,包括父母手足在内都不能沟通,至于朋友,那根本是不存在的。
\par 许多年过去了,回想自己一生的悲喜剧,大半是个性所造成的,怨不得天,尤不得人。
\par 很多事情,只因我固执于只从“以自己为本位”的角度去观察,以为那是唯一的真理和途径,结果不但活得不好,对他人也没有什么真正的付出。
\par 孩子,你目前看见的只是不公平,看见的只是朋友们不理睬你,看见的,很坦白地说——只是你自己,眼中并没有别人的任何理由。
\par 在你目前的年龄,这是被允许的,只要你不太钻牛角尖,更不可以有自杀的念头。
\par 可是,如果在以后成长的岁月里,你的眼光仍是如此,那么我肯定你将会得到一个并不快乐也没有太多意义的人生,而且不很容易在社会上与人和谐而友爱地相处——这都是你的个性造成的。当然,这和体质也有关联,你身体健康吗?
\par 我以为,母亲要求你做家事,也是应该的,因为你也是家中的一分子。甚而,她不要求你,都当态度和悦地主动替她分担。母亲不是虐待你,只因她不了解,在升学的竞争和压力下,一个学生念书的时间非常紧凑,如果分担了家事而丧失了读书的分分秒秒,对一个求好心切的好孩子来说,也是苦痛的。
\par 这种事情,想来你已与母亲之间交换过意见,而没有结果,才会写信给我。
\par 我的看法是,如果家务不是太重太重,你可以想出一种快速处理的方法。手脚快,做事有条理,有安排,两件家事一同做。(例如烧开水的同时,便去洗衣,洗衣的同时,浸泡其他的衣物,晒衣服时,一方面煮饭。只要警觉性高些,不要做了这、忘了那,家事时间可以利用技术管理而发挥快速的效果。)
\par 母亲的教育程度和你不同,在价值观上自然也有距离,可是父母供你念到高中,就是他们的伟大。我看到你所说的母亲,心中很受感动,她不懂念书有什么用,她仍给你念,你有没有想过这一点?
\par 你说母亲不为你想一想,对不起,请问你为她又想过了多少?
\par 你的前途不会因为做家事分占了念书而送掉的。学问之道,是人格的建立、生命的领悟、凡事广涵的体认——而不是做一架“念书机器”。如果你以为,你死啃书本,考上大学,就是前途的代名词,那仍是虚空而幼稚的,因为你没能了解,书本只是工具而已,念了一大堆书,仍不懂做人,那个书,就是白读了。
\par 写到这儿,再看你的来信,你的信中,“不满”都有理由,“不甘心”也很有理由地写出来。
\par 三张信纸,出现了三次“不满”,而且说——这只是不满之一二而已。我真不知,人生这么多的不满又是为了什么?这么多的“不甘心”,又是为了什么?孩子,你很自私,对不起,恕我直言。不要难过这句话。小时候的我,也是这样的。
\par 在这种心态下,你求教于“辅导中心”“张老师”,现在来找我,其实都不是诚心地要求我们帮助你,而是将我们当做发泄的对象而已。
\par 你不合作,不改变自己的观念,不肯看见他人的优点,我们又怎能解决你的困难?
\par 你的朋友,在你眼中,全是一批对不起你的家伙,我绝不赞成你说的话:你对她们越好,她们越看不起你。
\par 人,都是以心换心的,起码百分之七十是如此。请你对人类要有信心,不要因为一些小事,而不肯原谅他人。试试看,再试一次,试着不要太计算,试着以德报怨,好不好?
\par 你的来信中,最可贵的一句话,就是——我要好好地活下去。
\par 好好地活下去,快乐是第一要素,胸襟是基础,体谅他人,是有学问的另一种解释。如果培养这种观念,人生是可以好好过下去的。
\par 孩子,也许,你看了这封信,心里不但失望而且气愤,也可能对我,更有不满。可是我的良知不允许我写下同意你观点的话——那叫迎合。迎合你,可以使你视我为天下唯一的知己,而对你的人生,我却没有尽到劝告和开解的作用,那就不对了。我不能欺骗自己,更不能欺骗你。
\par 这封回信,你可能看了就撕掉,如果你不接受。但是起码你必须看完一遍才会撕掉,必有一些东西留在你心里,撕也撕不掉,对不对?
\par 好孩子,在你没有改变的时候,请不要再来信,当你有了一点点不同的观人观事的态度,我们才再通信好吗?
\par 谢谢你这么信任我,对我写下了真诚的话,我很感谢你,真的。祝你好好地活下去。
\par 每天,看一下天空,看看那广大的天空好吗?
\par \rightline{三毛上}


\subsubsection{真聪明的好孩子}

\refdocument{
    \par \leftline{三毛您好:}
    \par 小时候我自卑极了,小小心灵就知道什么是势利了,寂寞的童年与书为伍居多,本来就不好的脑子,塞进一些乱七八糟的故事后,更是退化了。
    \par 我看过坊间您的每一本书,因为没有人比您更直截了当,更坦白地写出自己,让我知道世界有人可以活得这样自然,这样的亲切。
    \par 三月二十五日在彰化听您演讲,您说得真好,讲到快乐时,我的心也跟着快乐,说到悲伤事,您的语调令人心酸。
    \par 我最爱您的是:您也爱人,爱一些平凡的人。您并不算是漂亮的人,但是接近您,不由得迷上您那股特殊的气质,那种气质是教养、是修养、是发自内心的,这种气质能使人真正的“高贵”,能使人无论十八岁或八十岁都觉得美丽。
    \par 我知道您好忙好忙,可是忍不住还要写信给您,告诉您一些话,想说的太多,拿起笔反而不知道怎么下笔。愿您
    \par 保重自己
    \par \rightline{梁美华敬上}
}


\par \leftline{美华:}
\par 看了你的来信,最使我感动的一点,是你掌握了一场讲演会中的特质。
\par 我们听一个人讲话,胜于去看一个人长得是不是好看。我们听一个人演说,不只光是去看热闹,而是由他人的观点中,汲取自认为对生命有帮助的东西,这才应该是去参加的目的。
\par 社会上,每一个人,每一种职业,在我,只有人格的高尚与否,而没有工作的贵贱。每一个人在这个世界上,都有不同的功能,并不只有知识分子才是高贵的。这一点,想来我们都有了一样的认同和理解,这真是很好的事情。
\par 你的智慧高,心情平和,观事温和。童年的际遇并没有使你走上极端之路,反而更为宽厚,是十分难得的。这么一来,苦难对我们,就成了一种功课,一种教育,你好好地利用了这份苦难,就是聪明。
\par 好孩子,我真喜欢看见这样美丽的信,不因你赞美我,而是你那颗平和的心。
\par 谢谢来信。祝
\par \leftline{平安}
\par \rightline{三毛上}


\subsubsection{没有找呀!}

\refdocument{
    \par \leftline{三毛姐姐:}
    \par 您好,看到您的信箱,很高兴。
    \par 想问您,找到另一个荷西了没有?
    \par 你愿不愿意再另外找一个伴呢?(我是指丈夫。)
    \par 告诉我您的近况好吗?比如说有什么新书出版的。
    \par 等您的回信。谢谢。
    \par \rightline{小弟弟上}
}
\par \leftline{亲爱的小弟弟:}
\par 你问我:“找到另外一个荷西没有?”
\par 很坦白地跟你说——我根本没有找。
\par 世上没有两个相同的人,包括双胞胎在内,都不可能完全相同。所以我并没有在找另一个荷西。因为再没有了另一个。
\par 荷西的躯体的确是由这个世上消失了,可是他的灵魂,仍是存在的。我不必找他,因为他没有消失。
\par 至于我愿不愿再找一个伴侣的问题,你在此句中又用了一个“找”字。
\par 好孩子,刻意去找的东西,往往是找不到的。天下万物的来和去,都有它的时间。
\par 你听过一首英文歌吗?歌词中说:“是你的,就是你的,不是你的,就不是你的。”我很喜欢这句话中的含意,尤其是用在情感和金钱的观念上,特别喜欢。
\par 我认为,人有权利追求幸福,一个肯于认清这个事实的人,是有智慧而且进取的。
\par 问题是,每一个人对于幸福的定义并不尽相同。一个伴侣,固然是一种幸福,可是人生还有其他值得我们去付出和追求的东西。所以,以我目前的情况来说,并不特别想有一个伴侣。
\par 也许再过两三个月,会有新书。
\par 谢谢你的关心。祝
\par \leftline{快乐健康}
\par \rightline{三毛上}


\subsubsection{教书不是塔}

\refdocument{
    \par \leftline{三毛:}
    \par 我是你忠实的读者,看了你最近在报章上的文章,觉得和过去有很大的不同,生活味道似乎减少了,好像回到了象牙塔中,不知你自己是否有自觉,你的看法又是怎样呢?
    \par \rightline{陈明发}
}

\par \leftline{明发:}
\par 谢谢你如此真诚而坦白地告诉我对我文章的感想,非常感激你的真诚。
\par 我是一个以本身生活为基础的非小说文字工作者。要求自己的,便是如何以朴实而简单的文字,记下生命中的某些历程。
\par 最近的文字,的确和以前有了很大的不同,原因是:生活在变,生命在延续,观念有改变,这都是无可奈何的人生之旅所造成的。于是,我也对自己的笔诚实,写下现在的自己,这也是我所坚持的写作方向。
\par 但是,我不同意我生活在象牙塔里的看法。如果说,生活中起伏变化大,因而在文字上,记载出来的比较显明而活泼,那是可能的,那是一种灿烂的生命。
\par 目前,我在教书,这不只是一个职业,同时也是因为环境变化而不得不做的角色调整。人生的角色变了,笔下出来的东西,便也不同于以往,因此我绝对不为写作而去创造生活。
\par 现在的文章,的确在风格上慢慢趋于宁静祥和而且更平淡。我很珍惜这份守淡的心情,它不是象牙塔。事实上,现在更加入世,已不是当年与世隔绝的那个沙漠女子了。
\par 很感谢你,谢谢!祝好
\par \rightline{三毛上}


\subsubsection{最重要的是被爱吗?}

\refdocument{
    \par \leftline{三毛你好:}
    \par 三月廿五日那天,你到彰化演讲,曾说,有天你在三更半夜,实在过不下去,想打电话找朋友聊聊的机会都没有,实在太苦了!
    \par 不过你比我还好,因你有很多很多朋友,然而不幸地,我连一个朋友都没有。
    \par 人,活在这世界上,最重要的是被爱,生活在没有爱的日子里,又怎能认识人生?敬祝
    \par \leftline{愉快}
    \par \rightline{淑芬}
}

\par \leftline{淑芬:}
\par 事实上,在三更半夜,以教养来说,没有一个朋友可以去打扰——除非是生命线。
\par 如果又不是生死大事,只是内心寂寞,便不当找哪个单位,快快强迫自己入睡的好。
\par 我不是孤独寂寞的人,那是偶尔有一年,有过这样的心情,现在不是没有,只是化解了很多。
\par 我认识的人很多,朋友并不多。
\par 西洋有一句名言:“一个朋友很好,两个朋友就多了一点,三个朋友未免太多了。”
\par 我很赞成这句话。知音,能有一个已经很好了,不必太多。如果实在一个也没有,还有自己,好好对待自己,跟自己相处,也是一个朋友。
\par 人活在世界上,最重要的是有爱人的能力,而不是被爱。我们不懂得爱人又如何能被人所爱?
\par 不要自怜,不要怨叹——你信中说自己“不幸”,不幸当是生命极大的苦难来时,才能用的字。你没有知音就算不幸,万一别的不顺心来了,要叫它什么呢?
\par 你不认识人生,是没有认识去爱人的快乐。
\par 试一试,好吗?谢谢你。祝
\par \leftline{幸运}
\par \rightline{三毛上}


\subsubsection{为什么、为什么?}

\refdocument{
    \par \leftline{三毛:}
    \par 心血来潮,给你一封信。是第一封,当然也是最后一封。(彼此仅有一纸信笺之缘。)
    \par 不要问我从哪里来,索性忘了我是谁吧。因为我是个不折不扣的“流浪汉”,一个在阳光底下拖着慈悲的影子,默默地一步一步趋向救苦救难的平凡的人。
    \par 三毛!“神爱世人”,请你告诉我,还有多少个不幸的人,悄悄地躲在黑暗的角落里受苦受难?
    \par 你、我是人。他们也是人,为什么?为什么忍心让他们残缺?折腾?
    \par 对不起,三毛,我找错对象啦,毕竟你不是神,教你怎么回答呢?
    \par (不祝福你。容许我将祝福转给天底下不幸的人,祝他们快乐!阿门!)
    \par \rightline{若尘}
}

\par \leftline{若尘:}
\par 世界上,的确有许许多多生活在苦难里的人,而绝大群的苦难者又往往是一个大时代的变乱之下,可怜的牺牲者,他们本身是没有罪的。
\par 你说,神爱世人,为何要给人这么多的苦难?这种问题,亦是我过去一次又一次仰问上苍的。
\par 在老子《道德经》里,有一句话:“天地不仁,以万物为刍狗。”这句话,初听时,很可能对它有误解——“如果我们观宇宙天地是以人为本位的话。”
\par 刍狗之草,本是祭祀所用,燎帛之具也。天地的化育,及于万物,自然也及于刍狗,它虽然在人眼中视为至贱,也是万物中的一物。一体同视,一般化育。天地以无心为心,不刻意有仁,正是仁的至高处。所以说,天地不仁,以万物为刍狗。
\par 我常常想到老子这句话,深以为是。
\par 我是一个自然主义者,对于自然界——这当然包括我们人类,所发生的任何事情,已不再拿个人的得失、喜怒、生死,去做一个苦难与否的判断和评价,因为我们也不过是一如枯荣的小草一般渺小而已。
\par 你的问题,可以说,在我,已有了答案,在你,是没有给你答案,十分抱歉。
\par 可是,我个人,在生命的可能里,并不因为有以上的想法,而忘却了自己的责任。在当做的时候,在可以为他人付出时,仍是真诚而慈爱地去做。这是出于内心的一种自然行为,而不是刻意为行善或为了使命而做的。
\par 你是一个高尚的人,看了你的来信,十二分地敬爱你。我,也有与你同样的胸怀和意念。大家为了人类,做一支小火柴,照亮幽暗的世界。如果世上每一个人,都做一支火柴,那么这一点点火花,也是不可忽视的。
\par 谢谢你的共勉。敬祝
\par \leftline{安康}
\par \rightline{三毛敬上}


\subsubsection{}




\subsubsection{}




\subsubsection{}




\subsubsection{}




\subsubsection{}




\subsubsection{}




\subsubsection{}




\subsubsection{}




\subsubsection{}




\subsubsection{}




\subsubsection{}




\subsubsection{}




\subsubsection{}




\subsubsection{}




\subsubsection{}




\subsubsection{}




\subsubsection{}




\subsubsection{}




\subsubsection{}




\subsubsection{}




\subsubsection{}




\subsubsection{}




\subsubsection{}




\subsubsection{}




















\subsection{随想}

\refdocument{
    \par 你快乐吗?
    \par 你快乐吗?你快乐吗?
    \par 试试看,每天吃一颗糖,
    \par 然后告诉自己——
    \par 今天的日子,果然又是甜的。
}


\subsubsection{孩子}


\par 没有孩子的女人是特别受祝福的。
\par 养一个小人,没有问题。
\par 为这份爱,担一生一世的心,
\par 担不起。
\par 
\par 遇到不能解决的事情,去问孩子,
\par 孩子脱口而出的意见,
\par 往往就是最精确而实际的答案。
\par 
\par 成年人最幼稚的想法就是——小孩子又懂得什么?
\par 其实,大半的孩子都不很享受作为一个孩子的滋味。
\par 这种情形,在中国偏又多些。
\par 
\par 适度地责骂孩子,可能使孩子的心灵更有安全感。
\par 
\par 中国夫妇,
\par 对于不圆满的婚姻,大半采取瓦全。
\par 理由是——为了孩子。
\par 欧美父母,
\par 处理不愉快的结合,常常宁愿玉碎。
\par 理由也是——为了孩子。
\par 
\par 孩子并不以为自己小,是大人一再灌输大小的观念,
\par 才造成孩子的“承认事实”。
\par 
\par 童年,只有在回忆中显现时,
\par 才成就了那份完美。


\subsubsection{快乐}

\par 比较快乐的人生看法,
\par 在于起床时,对于将临的一日,
\par 没有那么深沉的算计。
\par 
\par 完全没有缺乏的人,也不可能再有更多的快乐了。
\par 
\par 快乐是一种等待的过程。
\par 突然而来的所谓“惊喜”,事实上叫人手足无措。
\par 
\par 一般性的快乐往往可以言传。
\par 真正深刻的快乐,没有可能使得他人意会。
\par 快乐和悲伤都是寂寞。
\par 
\par 快乐是不堪闻问的鬼东西,
\par 如果不相信,请问自己三遍——
\par 我快乐吗?
\par 
\par 快乐是另外一件国王的新衣。
\par 这一回,如果国王穿着它出来游街,大家都笑死了——
\par 笑一个国王怎么不穿衣服出来乱跑呀!
\par 
\par 你快乐吗?
\par 你快乐吗?你快乐吗?
\par 
\par 试试看,每天吃一颗糖,
\par 然后告诉自己——
\par 今天的日子,果然又是甜的。


\subsubsection{}




