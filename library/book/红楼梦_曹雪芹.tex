

\section{红楼梦}


\par 红楼梦:全三册
\par 作者:(清)曹雪芹著;(清)无名氏续;(清)程伟元,(清)高鹗整理;中国艺术研究院红楼梦研究所校注.—3版
\par 出版社:人民文学出版社
\par 版本:1982年3月北京第1版,2008年7月北京第3版;2020年6月第2次印刷
\par 书号:978-7-02-016117-1

\subsection*{《红楼梦》校注本三版序言}

\par 本书初版于一九八二年,至今忽忽已历二十五周年,发行量已逾三百五十万套。一九九四年,当此书面世十二年的时候,我们曾修订过一次,改正了初版中的一些疏漏讹误,也吸收了红学研究上的新成果。现在距离上一次的修订,又已过了十三个年头。红学是一门最具群众性的学问,它拥有的研究队伍和读者,可能远比其他学科的人数要多得多。这十三年的过程,在红学的研究上,自然又有很多的收获,因此,我们决定再次进行修订。
\par 记得一九七五年校订开始之初,我们曾为选用底本,进行过热烈的争论,最后决定采用乾隆二十五年的庚辰本(指底本的年代)为底本,现在看来,当时的这个选择是正确的。广大读者和研究者接受和认可这个本子就是最好的证明。同时,对庚辰本的研究不断深入,而且一九九四年齐鲁书社又出版了同样以庚辰本为底本而又汇集脂评的校订本,到二〇〇六年,作家出版社又出版了一种庚辰本的校订本,这说明庚辰本的真正价值,日益为学术界所认识了。我们作为首次大胆采用庚辰本为底本来校订《红楼梦》的学人,当然是欢迎的。《诗经·小雅·伐木》说:“嘤其鸣矣,求其友声。”这种学术上的求同之心,是大家可以理解的。
\par 我们注意到,新出的以庚辰本为底本的校本,尤其是二〇〇六年的作家本,大量采用了我们的校订成果,这是值得欢迎的。当时我们遵国务院古籍整理组组长李一氓先生之嘱,校记要精,只有重要的改动才作校记。这样做,一方面是为了方便读者的阅读,避免繁琐;另方面,也是为了降低书的定价,有利于读者购买。所以我们大量校改的文字并未出校记。遗憾的是,作家本的校者,并不说明他的校本上的校文,基本上是用了前人的成果,他把这些校文用黑体字排出,还在《校勘说明》里明确说:“补改文字,一律用黑体,使之和原抄文字相区别,便于读者区分与比较。”这段话分明就是告诉读者,这些用黑体字排的文字,全是他新校出来的。而实际上这些用黑体字排的校文,有百分之九十以上是我们早就校出来的。这只要用人民文学出版社出版、中国艺术研究院红楼梦研究所的校订本一对就明白了。
\par 我们的校订本,距今已二十五年了,当时用了七年时间才完成了这项任务。现在有的同志同样采用庚辰本作底本,大量采用我们的校文,这足以说明当时对底本的选择和校订文字的斟酌去取,是经得起时间的考验的,也为后来的校订者起了铺路的作用。
\par 学无止境,学问是与时推移,日新月异的,红学也是一样。所以我们这次的校订,参阅了近十多年来的多种新校本和红学论著,自觉收获较大。这些收获,当然不是个人的,而是反映了红学研究的成果,应该看作是红学界的共同成果。
\par 这次校订,计正文修订共五百馀条,校记修订共一百馀条,注释修订共三百馀条,其中增加条目二百馀条;修改条目一百馀条;凡例修订共三条。
\par 以上是这次修订的总情况。
\par 这次校订,校和注两方面都有相当的进展,这些都已包含在书里,不再一一列举。
\par 这次参加校订工作的人手较少,主要是冯其庸和胡文彬、吕启祥、林冠夫四个人。冯其庸同志负责正文的校订,吕启祥同志负责注释的修订和增补,胡文彬同志正文和注释两方面的工作都参加,并且由他来承担校和注两方面的合成工作,林冠夫同志,考虑到他的身体,主要是请他参加讨论和商量去取。胡文彬同志合成后,最后由冯其庸同志统一审阅和修改定稿。由于第一道工序校和注都做得很认真,所以校注两方面的修改面和难度虽然较大,但质量却比以往有所提高。胡文彬同志的合成工作,负担很重,文字量也大,但做得非常认真细致。尽管恰值酷暑,我们还是尽心尽力尽快地完成了预期的工作任务。
\par 当然,在这项工作启动以前,原校订组的副组长李希凡同志和我们四人,还有人民文学出版社的有关领导曾一起开会商量确定这项工程,之后还分别取得了散处在各地的原校订组成员的同意。这也是促使我们四个人加紧努力的因素。
\par 这里特别要谢谢陈熙中教授,他应我们的邀请,为我们写了几十条修订意见,都十分可贵。还有老友黄能馥老先生,重新为我们审定修改了有关服饰方面的注释,安徽的老友周中明教授,应我们的邀请,花了整整两个月的时间,为我们检核全书,写出了不少关于通假字、同音字厘定的意见和正文校补的意见。由于以上几位同志的帮助,使本书的校订,较前更有所提高。
\par 在整个校订过程中,任晓辉同志协助我们做了许多诸如查阅资料,复印稿件,递送信息等等的工作,使得这项工作得以快速有效的运转。
\par 本书自初版以来,不断收到各地热心的红学朋友的来信来稿,有的是热情鼓励,有的是指出错误,对我们都有很大的帮助。最近,我们又收到河南新安县冯东先生的来信,他为我们细心地查出了错字、注码误差等等问题。还有河北的一位红友萧凤芝同志,他来信告诉我们《红楼梦》第四十七回庚辰本作“十月一”是对的。这是北方为已故亲人送寒衣的民俗节日,不能改作“十月初一”。我们请教了周围的老北京人和北方的朋友,都说至今仍有“十月一,送寒衣”的民俗,所以我们仍依庚辰原本作“十月一”。在此我们敬向以往所有在报刊上发表文章指谬商榷和来信来电的读者朋友表示衷心的感谢!
\par 凡此,都说明,《红楼梦》的研究和校订,既离不开红学研究者,也离不开广大读者。《红楼梦》的修订工作,不会到此结束。我们希望今后能继续走专家和群众结合的路线,实事求是地将这部名著整理得更为完善。
\par \rightline{红楼梦校注组}
\par \rightline{二〇〇七年八月十三日}
\par 本书自二〇〇八年第三版至今,忽忽又将五载。在此期间,承广大读者和学界同仁关切,我们亦时时自省自检,发现仍存在若干疏漏和个别修改失当之处,包括印制过程中的失校。为了对读者负责,今年初,在冯其庸先生建议和主导下,对全书再做修订,具体操作悉委胡文彬、吕启祥二位,仍由胡文彬汇总,计改动正文及标点四十六处、注释十六处。馀者技术性的改动均归责任编辑徐文凯同志。
\par 此次小的改动乃属第三版即同一版次中的修葺。我们深知校书如扫落叶、注释如爬高坡,无有止境。在力所能及的范围内使本书减少讹误、趋向完善,是我们的真诚愿望。
\par 是为记。
\par \rightline{红楼梦校注组}
\par \rightline{二〇一三年四月}


\subsection*{《红楼梦》校注本再版序}
\begin{center}
    \par 冯其庸
\end{center}
\par 本书初版于一九八二年三月,距今转瞬已十二年。本书初版以来,受到广大读者和专家们的欢迎,也得到了不少指正。
\par 这十二年的岁月,使我们进一步认识到,我们当时确定的几个原则是正确的:一是我们所选择的底本——庚辰本,确是一个学术价值很高、接近曹雪芹原稿的珍贵本子,我们以此为底本,就使这个校本有了很好的基础;二是我们确定的校勘原则(详见《校注凡例》)也是正确的,这样就使我们的校勘工作做到了审慎和准确,不至于随意改动底本文字,从而较好地保持了原本的历史面貌;三是我们确定的注释原则(见《校注凡例》)也是切合实际的,对象适中,繁简得宜,因而使得本书避免臃肿烦琐之病。
\par 但是,学问是无止境的,“红学”更是日新月异,这十多年来“红学”研究已有了长足的进展,不少重要的专著相继出版了,不少重要的论文陆续发表了,还连续发现了有关曹雪芹家世的文献和实物。在《红楼梦》的名物考订上,也有不少的进展。对照我们的校本,就感到有了历史的差距。为了对读者负责,对我们的“红学”事业负责,我们深深感到有对本书作一次全面修订的必要。于是趁此书再版之机,我们就着手这一新的繁重工作。
\par 此书初版是由《红楼梦》校订组先作出初稿,然后由校订组和注释组反复修改定稿的,其详细情况见初版的《前言》。现在全面重新整理、重新校注时,不可能把原有的人都邀请回来了,幸好初版定稿时的三位同志:冯其庸、林冠夫、吕启祥都还在原工作岗位上,因此本次的校注,即由他们三位负责:由冯其庸总负其责,林、吕二位分别作校、注的具体修订工作。
\par 关于注释的修改和增补,计新增注释八十七条,补充和修改原注一百六十五条,其馀所有注释,共二千四百零一条,从内容到文字也重新作了一次审核认定;在校勘和标点、分段方面,重校的文字为数亦不在少。特别是此次重校过程中,仍用各脂本仔细复核,所以费力较多;至于标点和分段,则改动更多,不能一一列举。最后,由我作校、注两方面的审核工作,修改核定校、注的条目和文字,以及有关此次重新校注的其他事宜。以上种种,无法一一详述,读者翻阅此再版校注本,当可了然。
\par 尽管此书又作了较为详慎的重校和重注,但初版的创始之功自不可没,故初版的《前言》、《校注凡例》等文字仍置于卷首,以示不忘,也正说明前后校注的原则是一贯的。
\par 这里特别要说明的是,一九七四年至一九七五年间,倡议对《红楼梦》作校注整理,是由袁水拍同志向上级提出的。一九七四年秋天,袁水拍同志到我住处看望我,并提到整理古籍的问题。当时我提及《红楼梦》的校注问题,水拍同志极为重视,不久就要我草拟一个报告。此事后经国务院有关部门正式批准,由水拍同志任校注组的组长,由我和李希凡任副组长。初期校勘的稿子(即内部用的大字本),水拍同志还认真看过,凡是他不理解的地方,他还都提出来认真地询问过。后来,他因事因病,不再过问此事,但在重病中,仍希望能看到此书的出版。所以《红楼梦》的新校注工作得以正式立项并由政府拨款,调集一批专家和研究人员来工作,水拍同志是起了倡导推动作用的。现在水拍同志已经作古十年,当着此书修订再版的时候,我有责任将此经过叙明,亦以慰逝者于地下。
\par 时间虽然只过了十二年,但参加此书工作的同志,却已经作古了好多位,其中顾问有:叶圣陶、吴世昌、吴恩裕、吴组缃先生。参加校注工作的有:沈彭年、陶建基、徐贻庭、朱彤、祝肇年、江辛眉、杨廷福诸先生。国务院古籍整理组组长李一氓先生,一直关心《红楼梦》的研究工作,此书出版后,他还亲自写过评论文章,热情地肯定了这个新校注本,但李一氓先生也已经不幸逝世了。对以上十二位已故的先生和朋友,我们只能寄以哀思,以志永怀!
\par 此次修订再版,虽然由我们三人负责,但前人之功不可没,而“红学”方兴未艾,且无止境,故以后的校注工作,亦无有止境。故我们本次的修订,只是万里长途中的一站,瞻望前途,曷其有极!惟愿再奋馀勇,更求寸进,并望“红学”同人和专家读者进而教之,则《红》书幸甚!“红学”幸甚!
\par 是为序。
\par \rightline{一九九四年七月六日于京华宽堂}


\subsection*{前言}


\par 曹雪芹,是中国文学史上最伟大也是最复杂的作家,《红楼梦》也是中国文学史上最伟大而又最复杂的作品。
\par 关于曹雪芹,目前还存在着不少有争论的问题,不仅他的生卒年一直存在着争议,甚至连他的“字”、“号”也不能十分确定,按照曹雪芹的好友张宜泉的说法,应该是“姓曹名霑,字梦阮,号芹溪居士”,但有的研究者认为他的“字”是“芹圃”,号“雪芹”。
\par 他的生卒年问题,已经争论了几十年。他的生年,现在主要的有两种看法,一种认为他生于公元一七一五年,即康熙五十四年乙未;另一种说法认为他生于公元一七二四年,即雍正二年甲辰。他的卒年,主要有三种看法,一种认为他卒于公元一七六三年,即乾隆二十七年壬午除夕;另一种说法认为他卒于公元一七六四年,即乾隆二十八年癸未除夕;还有一种说法认为他卒于公元一七六四年初春,即乾隆二十九年甲申岁首\footnote{按乾隆二十八年除夕,已经是西历一七六四年二月一日,故乾隆二十九年甲申仍为一七六四年。}。现在大都倾向于第一种看法。
\par 曹雪芹的父亲,现在也有两种看法。一种认为是曹颙,曹雪芹是他的遗腹子;另一种看法,则认为是曹頫。
\par 曹雪芹的上世的籍贯,据近三十年来发现的大量可靠史料,证明他的祖籍是辽阳,后迁沈阳,他的上祖曹振彦原是明代驻守辽东的下级军官,大约于天命六年后金攻下辽阳时归附,以后随清兵入关。\footnote{周汝昌、杨向奎先生认为曹雪芹祖籍是河北丰润,但这是没有任何根据的臆想,是不可信的。详见冯其庸著《曹雪芹家世新考》(上海古籍出版社1980年版)、《再论曹雪芹的家世、祖籍和〈红楼梦〉的著作权》(《红楼梦学刊》1995年第1期)。}
\par 曹振彦归附后金以后,先是属佟养性管辖,后来又归了多尔衮属下的满洲正白旗,当了佐领。旋即跟随清兵入关。曹振彦在入关前的明、金战争中以及入关后的平姜瓖之叛的战争中是立过功的,他历任过山西吉州知州、阳和府知府、浙江盐法道等官职。曹家的发迹,实是从曹振彦开始的。此后,曹振彦之媳,即曹玺之妻孙氏当了康熙的保母。康熙二年,曹玺首任江宁织造之职,专差久任,至二十三年在江宁织造任上病故,康熙旋即命其子曹寅任苏州织造,后又继任江宁织造、两淮巡盐御史等职,并命其纂刻《全唐诗》、《佩文韵府》等书于扬州。曹寅很得康熙的信任和赏识,康熙南巡时曾主持过四次接驾大典。康熙五十一年曹寅在扬州任上病危,康熙特命快马送药抢救,曹寅病故后,又特命其子曹颙继任江宁织造。康熙五十三年曹颙病故,康熙又特命曹寅的胞弟曹荃(宣)之子曹頫过继给曹寅并继任织造之职,直至雍正五年十二月二十四日曹頫被抄家败落,曹家在江南祖孙三代先后共历六十馀年。
\par 《红楼梦》的作者伟大作家曹雪芹就是出生在南京的。直到雍正六年曹家抄没后才全家迁回北京。当时,曹雪芹尚年幼,按生于乙未说是虚岁十四岁,按生于甲辰说是虚岁才五岁。曹家回北京以后的情况,文献绝少记载,曹頫曾经在给康熙的奏折里说到“惟京中住房二所,外城鲜鱼口空房一所;通州典地六百亩,张家湾当铺一所,本银七千两”\footnote{《关于江宁织造曹家档案史料》132页。}等等。在曹家被抄以后,隋赫德的报告里也说到:“曹頫家属,蒙恩谕少留房屋,以资养赡,今其家属不久回京,奴才应将在京房屋人口,酌量拨给。”\footnote{《关于江宁织造曹家档案史料》188页。}据近年发现的雍正六年六月二十一日《曹頫骚扰驿站获罪结案题本》及雍正七年七月《刑部移会》,得知曹頫抄家前,尚有骚扰驿站案,并于雍正六年结案,曹頫被枷号催追赔款。雍正七年七月,曹頫尚在枷号中。又据《刑部移会》得知曹家尚有“京城崇文门外蒜市口地方房十七间半,家仆三对,给与曹寅之妻孀妇度命”。但以后情况究竟如何?究竟拨给了哪些房子?曹雪芹究竟住在何处?他的青年时期是如何度过的?这些问题,统因文献无征,不能确指。据红学家们的考证,认为他与敦诚、敦敏成为亲密朋友,是在右翼宗学里开始结识的,后来落魄住到了西郊,他的不朽的巨著《石头记》就是在西郊的山村里写成的。
\par 曹雪芹晚年的生活穷愁潦倒而又嗜酒狂放,朋友们常把他比作晋朝的阮籍。他甚至穷困到“举家食粥”的地步,常常要靠卖画来换酒喝。他的画很为当时的朋友们所推重。敦敏《题芹圃画石》诗说:“傲骨如君世已奇,嶙峋更见此支离;醉馀奋扫如椽笔,写出胸中磈礧时!”可见曹雪芹的胸襟和画风。可惜他的遗作至今尚未发现。
\par 伟大作家曹雪芹,终于在穷愁困顿中于公元一七六三年即乾隆二十七年壬午除夕去世。他的不朽巨著《石头记》的前八十回,早在他去世前十年左右就已经传抄问世;书的后半部分据专家们研究,认为基本上已经完成,只是由于某种原因未能传抄行世,后来终于迷失,这是不可弥补的损失。
\par 《红楼梦》是一部具有高度思想性和高度艺术性的伟大作品,从本书反映的思想倾向来看,作者具有初步的民主主义思想,他对现实社会包括宫廷及官场的黑暗,封建贵族阶级及其家庭的腐朽,封建的科举制度、婚姻制度、奴婢制度、等级制度,以及与此相适应的社会统治思想即孔孟之道和程朱理学、社会道德观念等等,都进行了深刻的批判并且提出了朦胧的带有初步民主主义性质的理想和主张。这些理想和主张正是当时正在滋长的资本主义经济萌芽因素的曲折反映。
\par 《红楼梦》塑造了众多的人物形象,他们各自具有自己独特而鲜明的个性特征,成为不朽的艺术典型,在中国文学史和世界文学史上永远放射着奇光异彩。
\par 《红楼梦》的情节结构,在以往传统小说的基础上,也有了新的重大的突破。它改变了以往如《水浒传》、《西游记》等一类长篇小说情节和人物单线发展的特点,创造了一个宏大完整而又自然的艺术结构,使众多的人物活动于同一空间和时间,并且使情节的推移也具有整体性,表现出作者卓越的艺术才思。
\par 《红楼梦》的语言艺术成就,更是代表了我国古典小说语言艺术的高峰。作者往往只需用三言两语,就可以勾画出一个活生生的具有鲜明的个性特征的形象;作者笔下每一个典型形象的语言,都具有自己独特的个性,从而使读者仅仅凭借这些语言就可以判别人物。作者的叙述语言,也具有高度的艺术表现力,包括小说里的诗词曲赋,不仅能与小说的叙事融成一体,而且这些诗词的创作也能为塑造典型性格服务,做到了“诗如其人”——切合小说中人物的身份口气。
\par 由于以上各方面的卓越的成就,因而使《红楼梦》无论是在思想内容上或是艺术技巧上都具有自己崭新的面貌,具有永久的艺术魅力,使它足以卓立于世界文学之林而毫无逊色。
\par 现存《红楼梦》的后四十回,是程伟元和高鹗在公元一七九一年即乾隆五十六年辛亥和公元一七九二年即乾隆五十七年壬子先后以木活字排印行世的,其所据底本旧说以为是高鹗的续作,据近年来的研究,高续之说尚有可疑,要之非雪芹原著,而续作者为谁,则尚待探究。续书无论思想或艺术较之原著,已大相悬殊,然与同时或后起的续书相比,则自有其存在之价值,故至今仍能附原著以传。
\par 《红楼梦》在乾隆中叶以后,带脂砚斋评的八十回抄本日多,乾隆末叶即可公开在庙市中抄卖,并且价昂至数十金一部。今传乾隆时期的《石头记》抄本,尚有十一种之多,计有:己卯本、庚辰本、甲戌本、《红楼梦稿》本、蒙古王府本、戚蓼生序本、南京图书馆藏本、梦觉主人序本、舒元炜序本、郑振铎藏本、苏联列宁格勒亚洲图书馆藏本等\footnote{原苏联列宁格勒亚洲图书馆藏本,以往均称列藏本。今因该图书馆已更名为圣彼得堡俄罗斯科学院东方古籍文献研究所,故此书统称俄藏本。特予说明。}。另有南京靖应鹍藏本,今已遗失,又程甲本的前八十回底本,原也是抄本,如果一并计入,则可以说现知的抄本已有十三种之多。当然上面所说的己卯本、庚辰本、甲戌本等名称,其干支年代,都不能代表现有这些本子的抄定年代,都只能表明它们的底本的年代,这一点早已为红学家们指出了。
\par 在以上这些抄本中,己卯、庚辰、甲戌的底本是比较早的。其中己卯本已确知为怡亲王府抄本,其抄成年代约在公元一七六〇年即乾隆二十五年庚辰以后,现存庚辰本抄定的年代,大约是在公元一七六一年即乾隆二十六年以后,甲戌本底本的年代应是公元一七五四年,即乾隆十九年甲戌,但现在所传甲戌本的抄成年代,则是比较晚的。在上述这些抄本中,庚辰本是抄得较早而又比较完整的唯一的一种,它虽然存在着少量的残缺,但却保存了原稿的面貌,未经后人修饰增补(其六十四、六十七两回的残缺,各本皆然,现存各本的这两回或是据程本,或是经后人增补过的),因此本书在校勘过程中决定采用庚辰本为底本,以其他各种脂评抄本为主要参校本,以程本及其他早期刻本为参考本。凡底本文字可通而主要参校本虽有异文但并不见长者,仍依底本;凡底本明显错误而主要参校本不误者,即依主要参校本;凡底本脱漏之文字,有主要参校本可资校补者,即依主要参校本补齐。
\par 本书的注释,凡一应典章制度名物典故以及难解之语词,一般均尽可能作注释,但由于我们的能力有限,而《红楼梦》的注释又极为繁难,因此我们的注不仅可能挂一漏万,而且也可能注释得不尽恰当;我们的校订也同样是如此。有关校订和注释方面的具体情况,均见本书《校注凡例》,这里不再一一详述。
\par 本书校注工作开始于一九七五年,其间参加工作的人员陆续有所更替,工作时间亦长短不一,难以一一表明,现以参加时间先后和姓氏笔划为序,计参加本书校注工作的有:冯其庸、李希凡、刘梦溪、吕启祥、孙逊、沈天佑、沈彭年、应必诚、周雷、林冠夫、胡文彬、曾扬华、顾平旦、陶建基、徐贻庭、朱彤、张锦池、蔡义江、祝肇年、丁维忠。
\par 参加本书最后修改定稿的,校勘方面有:冯其庸、林冠夫、徐贻庭。由冯其庸负责。注释方面有:陶建基、吕启祥、朱彤、张锦池、丁维忠。由陶建基负责。
\par 全书的校注工作由冯其庸同志总负责。
\par 吴世昌、吴恩裕、吴组缃、周汝昌、启功等几位老红学家担任本书校注工作的顾问。
\par 叶圣陶老先生和叶至善同志对本书的校点和注释提了不少宝贵的意见,本书的前半部分,叶圣老还亲自标点、修改过不少地方。为本书的校注提过不少修改意见和撰写过许多条目及注文的还有王雪苔、江辛眉、朱家溍、巫君玉、杨廷福、杨乃济等同志。此外还有不少同志对本书的校注提过宝贵意见或帮助修改过注文,这里限于篇幅,无法一一列出。实际上本书的校注工作是在全国广大群众的热情支持下,是在他们作出的丰硕成果的基础上进行的。
\par 我所的行政工作人员和资料室的同志,也为本书的校注做了不少工作。
\par 南京图书馆对本书的注释和校订曾多次提出书面的修改意见,其他如中国人民大学、北京大学、北京师范大学、北京师范学院、复旦大学、上海师范学院、中山大学、安徽师范大学、哈尔滨师范大学、杭州大学、中央戏剧学院等单位,都给予了热情的支持,我院戏曲研究所和美术研究所、音乐研究所在涉及有关专业方面的问题上,也给予了我们不少指导和帮助。
\par 北京图书馆、北京大学图书馆、中国社会科学院文学研究所图书馆、中国科学院图书馆、南京图书馆、人民文学出版社资料室,为我们提供了许多重要版本和资料。我院的图书馆,则为我们提供了全部的基本参考图书并给予了种种方便和支持。
\par 本书的责任编辑、人民文学出版社古典文学编辑室的王思宇同志对本书的校和注,都提供了许多宝贵的修改意见,付出了不少精力。
\par 本书的校注工作,自始至终,一直是在中国艺术研究院党委和院领导的热情支持下进行的。本院其他行政部门也给我们以多方面的协助,使这项工作得以顺利进行。
\par 对于以上给予本书的校注以大力支持的同志和单位,我们表示衷心的感谢。
\par 本书的校和注,一定还存在着许多缺点,我们衷心期望得到国内外的读者和专家的指正,以便不断修订。
\par \rightline{中国艺术研究院红楼梦研究所}
\par \rightline{一九八一年五月二十日}
\par 一九九四年七月改定关于曹雪芹的祖籍、家世和卒年部分。 冯其庸。 一九九四年七月七日雨窗。


\subsection*{校注凡例}


\par 关于校勘方面:
\par 一、本书以《脂砚斋重评石头记(庚辰〔一七六〇年〕秋月定本)》(简称庚辰本)为底本。底本若干处缺文均依其他脂本或程本补齐,第六十四、六十七回缺文,则采用程甲本补配。
\par 二、以下列各脂评本、抄本及程甲、乙本为参校本:
\par (一)清乾隆甲戌(一七五四年)脂砚斋重评本(简称甲戌本)。
\par (二)乾隆己卯(一七五九年)冬月脂砚斋四阅评本(简称己卯本)。
\par (三)蒙古王府本(简称蒙府本)。
\par (四)戚蓼生序有正书局石印本(简称戚序本)。
\par (五)戚蓼生序南京图书馆藏本(简称戚宁本。此本与戚序本为同一祖本,唯个别处有异。校记中凡未特别列出戚宁异文者,即与戚序本同)。
\par (六)乾隆甲辰(一七八四年)梦觉主人序本(简称甲辰本)。
\par (七)乾隆己酉(一七八九年)舒元炜序本(简称舒序本)。
\par (八)郑振铎藏本(简称郑藏本)。
\par (九)《红楼梦》稿本(简称梦稿本)。
\par (十)圣彼得堡俄罗斯科学院东方古籍文献研究所藏《石头记》(简称俄藏本)。
\par (十一)《卞藏脂本红楼梦》(简称卞藏本)。
\par (十二)乾隆辛亥(一七九一年)程伟元初排活字本(简称程甲本)。
\par (十三)乾隆壬子(一七九二年)程伟元第二次排活字本(简称程乙本)。
\par 三、底本与各参校本之异文,凡属底本明显的衍夺讹舛者,据参校本增删改乙,凡校改底本之处,择要作出校记;凡底本文字可通者,悉仍其旧。
\par 四、其他脂本之回前、回后题诗,一律录入校记,以备查阅。
\par 五、底本旁添旁改文字,大致有以下几类情况:一是原抄时抄漏抄误后随即旁添旁改的,这类文字,可用己卯本校核;二是后来又用其他脂本校对后旁添旁改的,这类文字也可用其他脂本校核,以上两类旁添旁改文字,实际上是正文错抄或抄漏,应与正文同样处理;三是底本明显错误不通,旁改后文句通顺,但却无其他脂本可据,这类改文,以文理之确否而定其去取;四是后人妄改的文字,这类文字一律不取。
\par 六、底本系抄本,所用异体字、俗字甚多,凡属异体字、俗字,一般均以目前通行的规范字予以统一,不再另作校记。但遇有特殊情况,则不作统一,仍按底本文字。
\par 七、底本文字明显错误,各脂本亦沿袭其误,此类情况,即径改之,并作校记说明。
\par 八、后四十回虽系无名氏所续,为适应读者阅读需要,仍接于八十回后,不另标出“附录”字样。但因续作者已无可考,故只署程伟元、高鹗整理刊行者的名字,以与曹雪芹原著区别。
\par 九、后四十回以萃文书屋辛亥排印本(即程甲本)为底本,校以藤花榭本、本衙藏版本、王雪香评本、程乙本等,改字原则一如前八十回。
\par 关于注释方面:
\par 一、本书注释大体上以具有中等文化水平的读者为对象。
\par 二、本书注释条目选取的范围大体上包括书中涉及的典章故实、职官名称、服饰陈设、古代建筑、琴棋书画、释道信仰、医药占卜、方言俗语以及较生僻的字、词等。
\par 三、本书的诗、词、曲、赋、偈语、灯谜、酒令等均加注释。为了便于读者阅读,除注明其中字、词、典故外,必要时对某些句、联以至通首大意亦略加解释。
\par 四、注释内容力求简明,但必要时亦注明出处或径引所据原文,以资查证。
\par 五、相同条目原则上只注一次,但有时该条目在正文的不同回次中含义各有侧重,则重出另注,俾可前后参照,以利阅读。
\par 六、注文中凡涉及学术界尚有争论的问题,或者阙疑,或者介绍其中某一种、两种说法,以供读者参考。
\par 七、本书注释,曾参阅已出之注释或研究、考证文章,为避免繁琐,不再一一标出,非敢掠美。
\par \rightline{中国艺术研究院红楼梦研究所}
\par \rightline{一九八一年五月三十日}
\par \rightline{二〇〇七年八月八日修订}


\subsection*{第一回\ 甄士隐梦幻识通灵\ 贾雨村风尘怀闺秀}


\par 此开卷第一回也。作者自云:因曾历过一番梦幻之后,故将真事隐去,而借“通灵”之说,撰此《石头记》一书也。故曰“甄士隐”云云。但书中所记何事何人?自又云:“今风尘碌碌,一事无成,忽念及当日所有之女子,一一细考较去,觉其行止见识,皆出于我之上。何我堂堂须眉\footnote{须眉——代指男子。},诚不若彼裙钗\footnote{裙钗——代指女子。}哉?实愧则有馀,悔又无益之大无可如何之日也!当此,则自欲将已往所赖天恩祖德,锦衣纨袴\footnote{锦衣纨(wán丸)袴——富贵者的穿着,引申为富家子弟的代称。锦:色彩华美的丝织物。纨:细绢。}之时,饫甘餍肥\footnote{饫(yù玉)甘餍(yàn厌)肥——犹言饱食香甜肥美的食品。饫、餍,吃饱吃腻的意思。}之日,背父兄教育之恩,负师友规训之德,以至今日一技无成、半生潦倒之罪,编述一集,以告天下人:我之罪固不免,然闺阁中本自历历有人,万不可因我之不肖,自护己短,一并使其泯灭也。虽今日之茅椽蓬牖\footnote{茅椽(chuán传)蓬牖(yǒu友)——代指草房陋室,贫者所居。茅、蓬都是野草。椽,房椽子;牖,窗户。},瓦灶绳床\footnote{瓦灶绳床——瓦灶为土坯烧成的简陋的灶,俗称行灶。绳床亦名胡床、交床,为一种简易的坐具。《演繁露》:“今之交床,本自虏来,始名胡床……唐穆宗时又名绳床。”},其晨夕风露,阶柳庭花,亦未有妨我之襟怀笔墨者。虽我未学,下笔无文,又何妨用假语村言,敷演\footnote{敷演——叙述生发。}出一段故事来,亦可使闺阁昭传,复可悦世之目,破人愁闷,不亦宜乎?”故曰“贾雨村”云云。
\par 此回中凡用“梦”用“幻”等字,是提醒阅者眼目,亦是此书立意本旨。\footnote{“此开卷第一回也”以下一大段文字,唯甲戌本在第一回回目之前,作为全书“凡例”的第五条,文字与各本少异,并另有回前诗。底本和其馀各本,都在回目之后,作为正文的开头。陈毓罴最早提出:这是脂批。正文应是从“列位看官”开始。从这段文字的内容和行文的特点看,这个结论是可信的。但考虑到其内容主要是“作者自云”,而在各本中又起着相当于楔子的作用,故仍作特殊处理,放在卷首,并在排字时低二格,以示区别。}
\par 列位看官:你道此书从何而来?说起根由虽近荒唐,细按则深有趣味。待在下将此来历注明,方使阅者了然不惑。
\par 原来女娲氏炼石补天\footnote{女娲(wā洼)氏炼石补天——古代神话传说。女娲氏:传说中的上古“三皇”之一,又称娲皇。《淮南子·览冥训》:“往古之时,四极废,九州裂,天不兼覆,地不周载,……于是女娲炼五色石以补苍天,断鳌足以立四极。”}之时,于大荒山无稽崖\footnote{大荒山无稽崖——大荒山:《山海经·大荒西经》:“大荒之中有山名曰大荒之山。”这里寓“荒唐”。无稽崖和后文“青埂峰”,均属作者虚拟,分别寓“无稽”、“情根”之意。《红楼梦》一书用人名地名谐音寓意,如后文由脂砚斋注明的有:甄士隐(真事隐),贾雨村(假语存),甄英莲(真应怜),霍启(祸起),封肃(风俗),娇杏(侥幸),冯渊(逢冤),元、迎、探、惜(原应叹息)等等,不再一一作注。}炼成高经十二丈、方经二十四丈顽石三万六千五百零一块。娲皇氏只用了三万六千五百块,只单单剩了一块未用,便弃在此山青埂峰下。谁知此石自经煅炼之后,灵性已通,因见众石俱得补天,独自己无材不堪入选,遂自怨自叹,日夜悲号惭愧。
\par 一日,正当嗟悼之际,俄见一僧一道远远而来,生得骨格不凡,丰神迥异,说说笑笑来至峰下,坐于石边高谈快论。先是说些云山雾海神仙玄幻之事,后便说到红尘中荣华富贵。此石听了,不觉打动凡心,也想要到人间去享一享这荣华富贵;但自恨粗蠢,不得已,便口吐人言,向那僧道说道:“大师,弟子蠢物,不能见礼了。适闻二位谈那人世间荣耀繁华,心切慕之。弟子质虽粗蠢,性却稍通;况见二师仙形道体,定非凡品,必有补天济世之材,利物济人之德。如蒙发一点慈心,携带弟子得入红尘,在那富贵场中、温柔乡里受享几年,自当永佩洪恩,万劫不忘也。”二仙师听毕,齐憨笑道:“善哉,善哉!那红尘中有却有些乐事,但不能永远依恃;况又有‘美中不足,好事多魔’八个字紧相连属,瞬息间则又乐极悲生,人非物换,究竟是到头一梦,万境归空,倒不如不去的好。”
\par 这石凡心已炽,那里听得进这话去,乃复苦求再四。二仙知不可强制,乃叹道:“此亦静极思动,无中生有之数也。既如此,我们便携你去受享受享,只是到不得意时,切莫后悔。”石道:“自然,自然。”那僧又道:“若说你性灵,却又如此质蠢,并更无奇贵之处。如此也只好踮脚\footnote{踮脚——犹言“垫脚”。}而已。也罢,我如今大施佛法助你助,待劫终之日,复还本质,以了此案。你道好否?”石头听了,感谢不尽。那僧便念咒书符,大展幻术,将一块大石登时变成\footnote{ “说说笑笑”至“登时变成”共四百二十九字,原作“来至石下,席地而坐长谈,见”十一字,各本同。从甲戌本增。}一块鲜明莹洁的美玉,且又缩成扇坠\footnote{扇坠——悬于扇柄的饰物,多用玉、石等制成。}大小的可佩可拿。那僧托于掌上,笑道:“形体倒也是个宝物了!还只没有实在的好处,须得再镌上数字,使人一见便知是奇物方妙。然后携你到那昌明隆盛之邦,诗礼簪缨之族\footnote{诗礼簪(zān)缨之族——指书香门第,官宦家族。诗礼:读诗书,讲礼仪。簪缨:贵者的冠饰,这里代指作官。簪:一种横插髻上或连接冠与髻的长针。缨:帽带。},花柳繁华地,温柔富贵乡去安身乐业。”石头听了,喜不能禁,乃问:“不知赐了弟子那几件奇处,又不知携了弟子到何地方?望乞明示,使弟子不惑。”那僧笑道:“你且莫问,日后自然明白的。”说着,便袖了这石,同那道人飘然而去,竟不知投奔何方何舍。
\par 后来,又不知过了几世几劫\footnote{劫——佛家用语。梵文音译“劫波”之略,意为“远大时节”。佛教认为,世界有周期性的生灭过程,它经历若干万年后,就要毁灭一次,重新开始,此一周期称为一“劫”。},因有个空空道人访道求仙,忽从这大荒山无稽崖青埂峰下经过,忽见一大块石上字迹分明,编述历历。空空道人乃从头一看,原来就是无材补天,幻形入世,蒙茫茫大士、渺渺真人携入红尘,历尽离合悲欢炎凉世态的一段故事。后面又有一首偈\footnote{偈(jì记)——梵文音译“偈陀”或“伽陀”之略,意译为颂。一般为四句之韵文。}云:
\refdocument{
    \par 无材可去补苍天,枉入红尘若许年。
    \par 此系身前身后事,倩谁\footnote{倩谁——倩:一读qìnɡ音庆,作动词,意为请。又读qiàn音欠,如倩影。倩谁,即请谁。}记去作奇传?
}
\par 诗后便是此石坠落之乡,投胎之处,亲自经历的一段陈迹故事。其中家庭闺阁琐事,以及闲情诗词倒还全备,或可适趣解闷;然朝代年纪,地舆邦国却反失落无考。
\par 空空道人遂向石头说道:“石兄,你这一段故事,据你自己说有些趣味,故编写在此,意欲问世传奇。据我看来,第一件,无朝代年纪可考;第二件,并无大贤大忠理朝廷治风俗的善政,其中只不过几个异样女子,或情或痴,或小才微善,亦无班姑、蔡女之德能\footnote{班姑、蔡女之德能——班姑:即班昭,东汉史学家班固之妹,博学,曾参与续《汉书》。和帝时担任过宫廷教师,号称“大家(ɡū)”,故称“班姑”。编有《女诫》七篇,历来奉为妇德的典范。见《后汉书·曹世叔妻传》。蔡女:指蔡文姬,名琰,东汉文学家蔡邕之女,博学多才,精通音律,是历史上有名的“才女”。见《后汉书·董祀妻传》。}。我纵抄去,恐世人不爱看呢。”石头笑答道:“我师何太痴耶!若云无朝代可考,今我师竟假借汉唐等年纪添缀,又有何难?但我想,历来野史\footnote{野史——一般是指与官修正史相对而言的私家编撰的史类著作。“野史”之名始见于《新唐书·艺文志》,后渐与小说家言的“稗官”连用,称“稗官野史”。这里即指小说。},皆蹈一辙,莫如我这不借此套者,反倒新奇别致,不过只取其事体情理罢了,又何必拘拘于朝代年纪哉!再者,市井俗人喜看理治之书\footnote{理治之书——泛指古代“理朝廷治风俗”的书籍。}者甚少,爱适趣闲文者特多。历来野史,或讪谤君相,或贬人妻女,奸淫凶恶,不可胜数。更有一种风月笔墨\footnote{风月笔墨——原指描写风花雪月、儿女私情的文字。这里专指着意渲染色情的作品。},其淫秽污臭,屠毒笔墨,坏人子弟,又不可胜数\footnote{“更有一种”至“又不可胜数”二十六字,原无。梦稿、甲戌、蒙府、戚序、俄藏、卞藏、甲辰本均存,文字小异。从梦稿、甲戌本补。}。至若佳人才子等书,则又千部共出一套,且其中终不能不涉于淫滥,以致满纸潘安、子建、西子、文君\footnote{潘安、子建、西子、文君——这里代指才子佳人。潘安:即潘安仁,晋代文人,著名美男子。子建:曹植的字,三国时文学家,以才高著称。西子:即西施,春秋时越国美女。文君:汉代卓王孙的女儿,新寡后“私奔”文学家司马相如,结为夫妇。},不过作者要写出自己的那两首情诗艳赋来,故假拟出男女二人名姓,又必旁出一小人其间拨乱,亦如剧中之小丑然。且鬟婢开口即者也之乎,非文即理。故逐一看去,悉皆自相矛盾、大不近情理之话,竟不如我半世亲睹亲闻的这几个女子,虽不敢说强似前代书中所有之人,但事迹原委,亦可以消愁破闷;也有几首歪诗熟话,可以喷饭供酒。至若离合悲欢,兴衰际遇,则又追踪蹑迹,不敢稍加穿凿,徒为供人之目而反失其真传者。今之人,贫者日为衣食所累,富者又怀不足之心,纵一时稍闲,又有贪淫恋色、好货寻愁之事,那里去有工夫看那理治之书?所以我这一段故事,也不愿世人称奇道妙,也不定要世人喜悦检读,只愿他们当那醉淫饱卧\footnote{“醉淫饱卧”,底本、梦稿、俄藏、卞藏本同。蒙府、戚序本作“醉饱淫卧”,甲戌本作“醉馀饱卧”,甲辰本作“醉心饱卧”,舒序本作“醉酒饱卧”。}之时,或避事\footnote{“避事”,梦稿、甲辰、舒序、俄藏、卞藏本同。甲戌、蒙府、戚序本作“避世”。}去愁之际,把此一玩,岂不省了些寿命筋力?就比那谋虚逐妄,却也省了口舌是非之害,腿脚奔忙之苦。再者,亦令世人换新眼目,不比那些胡牵乱扯忽离忽遇,满纸才人淑女、子建文君红娘小玉\footnote{ 红娘、小玉——红娘:唐代元稹《会真记》(至元代王实甫衍为杂剧《西厢记》)中崔莺莺的丫鬟。小玉:唐代蒋防《霍小玉传》中的女主人公。}等通共熟套之旧稿。我师意为何如?”
\par 空空道人听如此说,思忖半晌,将《石头记》再检阅一遍,因见上面虽有些指奸责佞贬恶诛邪之语,亦非伤时骂世之旨;及至君仁臣良父慈子孝,凡伦常\footnote{伦常——即封建伦理道德。伦:人伦,封建社会指人与人之间关系及行为的准则。封建社会以君臣、父子、夫妇、兄弟、朋友为五伦,认为是不可改变的常道,亦称五常。}所关之处,皆是称功颂德,眷眷无穷,实非别书之可比。虽其中大旨谈情,亦不过实录其事,又非假拟妄称,一味淫邀艳约、私订偷盟之可比。因毫不干涉时世,方从头至尾抄录回来,问世传奇。从此空空道人\footnote{“从此空空道人”,原无,各脂本均同。从程甲本补。}因空\footnote{空——“空”与下文的“色”、“情”,均佛教用语。佛教认为“空”乃天地万物的本体,一切终属空虚。“色”乃万物本体(空)的瞬息生灭的假象;“情”乃对此等假象(色)所产生的种种感情,如爱、憎等等。这里是借用,已注入了作家的人生体验。}见色,由色生情,传情入色,自色悟空,遂易名为情僧,改《石头记》为《情僧录》。东鲁孔梅溪则题曰《风月宝鉴》\footnote{《风月宝鉴》——甲戌本眉批云:“雪芹旧有《风月宝鉴》之书,乃其弟棠村序也。”甲戌本“凡例”云:《红楼梦》“又曰《风月宝鉴》,是戒妄动风月之情”。风月:指男女之情。宝鉴:宝镜。}。后因曹雪芹于悼红轩中披阅十载,增删五次,纂成目录,分出章回,则题曰《金陵十二钗》\footnote{金陵十二钗——金陵,古邑名,楚威王七年(公元前333年)置,在今南京市。后即为南京市的别称。钗:本为妇女的头饰。旧称女子为“裙钗”或“金钗”。十二钗,语本《古乐府》:“头上金钗十二行”,原言髻高插钗之多。又作十二女子代称。此书又“题曰《金陵十二钗》”,通常认为是由第五回“册子”上所写的十二个女子得名。}。并题一绝云:
\refdocument{
    \par 满纸荒唐言,一把辛酸泪。
    \par 都云作者痴,谁解其中味!
}
\par 出则既明,且看石上是何故事。按那石上书云:
\par 当日地陷东南\footnote{地陷东南——东南大地塌陷下沉。古代神话:共工与颛顼(zhuān xū专须)争帝,怒而触不周山,折天柱,绝地维,天倾西北,地不满东南。见《淮南子·天文训》。},这东南一隅有处曰姑苏,有城曰阊门\footnote{姑苏、阊(chānɡ昌)门——姑苏:苏州的别称,因其西南有姑苏山而得名。这里是指旧苏州府辖境。阊门:苏州城的西北门,又名破楚门。这里代指苏州城。}者,最是红尘中一二等富贵风流之地。这阊门外有个十里街,街内有个仁清巷\footnote{十里街、仁清巷——据脂批,谐音“势利街”、“人情巷”。},巷内有个古庙,因地方窄狭,人皆呼作葫芦庙。庙旁住着一家乡宦,姓甄,名费,字士隐。嫡妻封氏,情性贤淑,深明礼义。家中虽不甚富贵,然本地便也推他为望族了。因这甄士隐禀性恬淡,不以功名为念,每日只以观花修竹、酌酒吟诗为乐,倒是神仙一流人品。只是一件不足:如今年已半百,膝下无儿,只有一女,乳名唤作英莲\footnote{“英莲”,原作“英菊”,己卯本同。从梦稿、甲戌、蒙府、戚序、甲辰、舒序、俄藏、卞藏本改。下此名重出时,各本情况大体相同,不再作校记。},年方三岁。
\par 一日,炎夏永昼,士隐于书房闲坐,至手倦抛书\footnote{手倦抛书——见北宋人蔡确《夏日登车盖亭》诗(收入《千家诗》)其前二句:“纸屏石枕竹方床,手倦抛书午梦长。”},伏几少憩,不觉朦胧睡去。梦至一处,不辨是何地方。忽见那厢来了一僧一道,且行且谈。
\par 只听道人问道:“你携了这蠢物,意欲何往?”那僧笑道:“你放心,如今现有一段风流公案正该了结,这一干风流冤家\footnote{风流冤家——“冤家”,原为佛教用语。《五灯会元》:“佛教慈悲,冤亲平等。”后既作“仇人”、“对头”解,也用作对所爱之人的昵称,即爱极的反语。“风流冤家”指极相爱恋之男女。},尚未投胎入世。趁此机会,就将此蠢物夹带于中,使他去经历经历。”那道人道:“原来近日风流冤孽又将造劫历世去不成?但不知落于何方何处?”那僧笑道:“此事说来好笑,竟是千古未闻的罕事。只因西方灵河岸上三生石\footnote{西方灵河岸上三生石——西方灵河岸上:作者假想的神仙境界。西方:原指佛教的发源地天竺(古代印度)。灵河:原指恒河,今印度人犹称之为“圣水”。三生:指前生、今生和来生,这是佛教转世投胎的说法。三生石:传说唐代李源与和尚圆观交情很好,后有“三生石上旧精魂”、“此身虽异性常存”之句。见唐代袁郊《甘泽谣·圆观》。后以“三生石”喻因缘前定。}畔,有绛珠草一株,时有赤瑕宫神瑛侍者,日以甘露灌溉,这绛珠草始得久延岁月。后来既受天地精华,复得雨露滋养,遂得脱却草胎木质,得换人形,仅修成个女体,终日游于离恨天外,饥则食蜜青果为膳,渴则饮灌愁海\footnote{离恨天、蜜青果、灌愁海——离恨天:俗传“三十三天,离恨天最高;四百四十病,相思病最苦”。蜜青谐“秘情”。灌愁海:喻愁深。皆寓男女之情及其怨恨愁苦。}水为汤。只因尚未酬报灌溉之德,故其五内\footnote{五内——五脏,即心、肝、脾、肺、肾。亦泛言内心深处。}便郁结着一段缠绵不尽之意。恰近日这神瑛侍者凡心偶炽,乘此昌明太平朝世,意欲下凡造历幻缘,已在警幻仙子案前挂了号。警幻亦曾问及,灌溉之情未偿,趁此倒可了结的。那绛珠仙子道:‘他是甘露之惠,我并无此水可还。他既下世为人,我也去下世为人,但把我一生所有的眼泪还他,也偿还得过他了。’因此一事,就勾出多少风流冤家来,陪他们去了结此案。”
\par 那道人道:“果是罕闻。实未闻有还泪之说。想来这一段故事,比历来风月事故更加琐碎细腻了。”那僧道:“历来几个风流人物,不过传其大概以及诗词篇章而已;至家庭闺阁中一饮一食,总未述记。再者,大半风月故事,不过偷香窃玉、暗约私奔而已,并不曾将儿女之真情发泄一二。想这一干人入世,其情痴色鬼、贤愚不肖\footnote{不肖——旧时称不能继承父业之子曰不肖。肖:像。}者,悉与前人传述不同矣。”那道人道:“趁此何不你我也去下世度脱\footnote{度脱——佛家用语。超度解脱。}几个,岂不是一场功德?”那僧道:“正合吾意。你且同我到警幻仙子宫中,将蠢物交割清楚,待这一干风流孽鬼下世已完,你我再去。如今虽已有一半落尘,然犹未全集。”道人道:“既如此,便随你去来。”
\par 却说甄士隐俱听得明白,但不知所云“蠢物”系何东西。遂不禁上前施礼,笑问道:“二仙师请了。”那僧道也忙答礼相问。士隐因说道:“适闻仙师所谈因果,实人世罕闻者。但弟子愚浊,不能洞悉明白,若蒙大开痴顽,备细一闻,弟子则洗耳谛听,稍能警省,亦可免沉沦\footnote{警省(xǐnɡ醒)、沉沦——均佛家用语。警省:警觉省悟。沉沦:指在生死轮回中永远不得解脱。}之苦。”二仙笑道:“此乃玄机\footnote{玄机——道家用语。谓玄奥微妙的道理。这里义同天机。}不可预泄者。到那时不要忘我二人,便可跳出火坑\footnote{火坑——佛家用语,指苦难的人世。}矣。”士隐听了,不便再问。因笑道:“玄机不可预泄,但适云‘蠢物’,不知为何,或可一见否?”那僧道:“若问此物,倒有一面之缘。”说着,取出递与士隐。
\par 士隐接了看时,原来是块鲜明美玉,上面字迹分明,镌着“通灵宝玉”四字,后面还有几行小字。正欲细看时,那僧便说已到幻境,便强从手中夺了去,与道人竟过一大石牌坊,上书四个大字,乃是“太虚幻境”\footnote{太虚幻境——作者虚拟的仙境。太虚:空幻虚无的意思。}。两边又有一副对联,道是:
\refdocument{
    \par 假作真时真亦假,无为有处有还无\footnote{对联,原作“假作真时真作假,无为有处有为无”。己卯、梦稿本上联同底本。舒本略为特殊,作“色色空空地,真真假假天”。馀各本均作“假作真时真亦假,无为有处有还无”。第五回此联重出时,底本及其他各本,均作“真亦假”、“有还无”。从改。}。
}
\par 士隐意欲也跟了过去,方举步时,忽听一声霹雳,有若山崩地陷。士隐大叫一声,定睛一看,只见烈日炎炎,芭蕉冉冉,所梦之事便忘了大半。又见奶母正抱了英莲走来。士隐见女儿越发生得粉妆玉琢,乖觉可喜,便伸手接来,抱在怀内,逗他顽耍一回,又带至街前,看那过会\footnote{过会——旧时遇节庆,随地聚演百戏杂耍、笙乐鼓吹之类,观者如潮。}的热闹。
\par 方欲进来时,只见从那边来了一僧一道:那僧则癞头跣脚,那道则跛足蓬头,疯疯癫癫,挥霍\footnote{挥霍——亦作“挥攉”。《韵会》:“摇手曰挥,反手曰攉。”本谓动作轻捷,这里是挥洒自如的意思。}谈笑而至。及至到了他门前,看见士隐抱着英莲,那僧便大哭起来,又向士隐道:“施主,你把这有命无运\footnote{有命无运——旧时“算命”,用人出生的年、月、日、时所属的干支和金、木、水、火、土五行的生克来推断人的吉凶祸福;称一生的境遇好坏为“命”,一段时间的遭际为“运”。有命无运,这里意谓平生“行运”乖逆,遭际堪悲。}、累及爹娘之物,抱在怀内作甚?”士隐听了,知是疯话,也不去睬他。那僧还说:“舍我罢,舍我罢!”士隐不耐烦,便抱女儿撤身要进去,那僧乃指着他大笑,口内念了四句言词道:
\refdocument{
    \par 惯养娇生笑你痴,菱花空对雪澌澌\footnote{“菱花”句——隐喻英莲被呆霸王薛蟠强占作妾的不幸遭遇。菱花:指后来英莲改名香菱。雪:谐音“薛”,指薛蟠。菱在夏日开花而竟遇冰雪,喻英莲“生不逢时,遇又非偶”(脂评),定然遭到摧残。澌(sī司)澌:形容雪声。}。
    \par 好防佳节元宵后,便是烟消火灭时。
}
\par 士隐听得明白,心下犹豫,意欲问他们来历。只听道人说道:“你我不必同行,就此分手,各干营生去罢。三劫后,我在北邙山\footnote{北邙(mánɡ芒)山——也作“北芒山”,即邙山。在今河南省洛阳市北。东汉及北魏的王侯公卿多葬于此。后常被用来泛指墓地。}等你,会齐了同往太虚幻境销号。”那僧道:“最妙,最妙!”说毕,二人一去,再不见个踪影了。士隐心中此时自忖:这两个人必有来历,该试一问,如今悔却晚也。
\par 这士隐正痴想,忽见隔壁葫芦庙内寄居的一个穷儒——姓贾名化、字表时飞、别号雨村者走了出来。这贾雨村原系湖州\footnote{“湖州”,底本、甲戌、己卯本第二回、卞藏本作“胡州”,梦稿本第二回作“湖北”。从己卯(第一回)、蒙府、戚序、甲辰、舒序、俄藏本改。}\footnote{湖州——地名。今浙江省湖州市。脂评:谐音“胡诌也”。}人氏,也是诗书仕宦之族,因他生于末世,父母祖宗根基已尽,人口衰丧,只剩得他一身一口,在家乡无益,因进京求取功名,再整基业。自前岁来此,又淹蹇\footnote{淹蹇(yān jiǎn烟简)——即偃蹇。原指境遇困顿、不得意,这里是耽搁、阻滞的意思。}住了,暂寄庙中安身,每日卖字作文为生,故士隐常与他交接。
\par 当下雨村见了士隐,忙施礼陪笑道:“老先生倚门伫望,敢\footnote{敢——意谓莫非、恐怕、或许,这里作“莫非”解。}街市上有甚新闻否?”士隐笑道:“非也。适因小女啼哭,引他出来作耍,正是无聊之甚,兄来得正妙,请入小斋一谈,彼此皆可消此永昼。”说着,便令人送女儿进去,自与雨村携手来至书房中。小童献茶。方谈得三五句话,忽家人飞报:“严老爷来拜。”士隐慌的忙起身谢罪道:“恕诳驾\footnote{诳(kuánɡ狂)驾——邀来客人后,因故不能陪待,向客人道歉之词,犹言“失陪”。诳或作“诓”,欺骗的意思。驾:对客人的尊称。}之罪,略坐,弟即来陪。”雨村忙起身亦让道:“老先生请便。晚生乃常造之客,稍候何妨。”说着,士隐已出前厅去了。
\par 这里雨村且翻弄书籍解闷。忽听得窗外有女子嗽声,雨村遂起身往窗外一看,原来是一个丫鬟,在那里撷\footnote{撷(xié协)——采摘、捋取。唐代王维《相思》:“愿君多采撷,此物(红豆)最相思。”}花,生得仪容不俗,眉目清明,虽无十分姿色,却亦有动人之处。雨村不觉看的呆了。
\par 那甄家丫鬟撷了花,方欲走时,猛抬头见窗内有人,敝巾旧服,虽是贫窘,然生得腰圆背厚,面阔口方,更兼剑眉星眼,直鼻权腮\footnote{权腮——俗称颧骨腮,指人颧骨长得很高,相法认为是一种贵相。沈括《梦溪笔谈·人事》:“公满面权骨,不十年必总枢柄。”}。这丫鬟忙转身回避,心下乃想:“这人生的这样雄壮,却又这样褴褛,想他定是我家主人常说的什么贾雨村了,每有意帮助周济,只是没甚机会。我家并无这样贫窘亲友,想定是此人无疑了。怪道又说他必非久困之人。”如此想来,不免又回头两次。
\par 雨村见他回了头,便自为这女子心中有意于他,便狂喜不尽,自为此女子必是个巨眼英雄\footnote{巨眼英雄——有远见,能识鉴人才的人。},风尘\footnote{风尘——这里指扰攘的尘世,又有旅居在外,备尝艰辛之意。}中之知己也。一时小童进来,雨村打听得前面留饭,不可久待,遂从夹道中自便出门去了。士隐待客既散,知雨村自便,也不去再邀。
\par 一日,早又中秋佳节。士隐家宴已毕,乃又另具一席于书房,却自己步月至庙中来邀雨村。原来雨村自那日见了甄家之婢曾回顾他两次,自为是个知己,便时刻放在心上。今又正值中秋,不免对月有怀,因而口占五言一律\footnote{口占五言一律——口占:随口吟成,与下文“口号”义同。五言一律:每句五个字的律诗一首。}云:
\refdocument{
    \par 未卜三生愿\footnote{ “未卜”句——未卜:不能预知。全句意为同娇杏结姻缘的愿望不知能否实现。},频添一段愁\footnote{“频添”句——意即把这段愁绪时刻挂在心上。频:屡屡;时时。}。
    \par 闷来时敛额\footnote{敛额——皱眉头。},行去几回头。
    \par 自顾风前影,谁堪月下俦?\footnote{“自顾”二句——意谓风前自顾身影,有谁能赏识自己,成为我的终身伴侣呢?自顾风前影:由“顾影自怜”化出。堪:能,配得上。月下俦(chóu愁):成婚配的意思。传说唐代韦固在宋城遇一老人在月下检天下婚姻之书,囊中并有赤绳,一系男女之足,则必成夫妇。见李复言《续玄怪录》。后因称管婚姻之神为“月下老人”或“月老”,也用来代称媒人。俦:伴侣。}
    \par 蟾光如有意,先上玉人楼。\footnote{“蟾光”二句——蟾光:指月光。原意是:月光如真有情意,希望先照玉人的妆楼。暗含若得科举及第,定先到玉人楼上求婚之意。“蟾光”句,亦寓“蟾宫折桂”(即科举及第)之意。玉人楼:美人居住的地方,玉人,指娇杏。}
}
\par 雨村吟罢,因又思及平生抱负,苦未逢时,乃又搔首对天长叹,复高吟一联曰:
\refdocument{
    \par 玉在\UncommonCharB{\symbol{30165}}中求善价,钗于奁内待时飞。\footnote{“玉在”一联——这里贾雨村自比玉、钗,企图得到赏识,以求飞黄腾达。上句意谓美玉藏在匣子里希望卖得好价钱。\UncommonCharB{\symbol{30165}}(dú读):即“椟”。木匣;木柜。下句意谓玉钗放在镜盒中,等待时机而飞腾。传说汉武帝元鼎元年,有神女留一玉钗,昭帝时,有人偷开匣子,不见玉钗,只见一只白燕从中飞出,升天而去。见郭宪《洞冥记》。}
}
\par 恰值士隐走来听见,笑道:“雨村兄真抱负不浅也!”雨村忙笑道:“不过偶吟前人之句,何敢狂诞至此。”因问:“老先生何兴至此?”士隐笑道:“今夜中秋,俗谓‘团圆之节’,想尊兄旅寄僧房,不无寂寥之感,故特具小酌,邀兄到敝斋一饮,不知可纳芹意\footnote{芹意——古时有人认为芹菜的味道很美,就向乡豪称赞,乡豪尝后,却觉得很难吃。见《列子·杨朱篇》。后人常用“献芹”、“芹意”等作为送礼或请客的谦词。}否?”雨村听了,并不推辞,便笑道:“既蒙厚爱,何敢拂此盛情。”说着,便同士隐复过这边书院中来。
\par 须臾茶毕,早已设下杯盘,那美酒佳肴自不必说。二人归坐,先是款斟漫饮,次渐谈至兴浓,不觉飞觥限斝\footnote{飞觥(ɡōnɡ工)限斝(jiǎ甲)——觥筹交错、饮宴尽欢的情景。觥、斝:两种古代酒器,前者为角形,后者圆口平底。飞觥:挥杯;限斝:行酒令时限定饮酒数量。}起来。当时街坊上家家箫管,户户弦歌,当头一轮明月,飞彩凝辉,二人愈添豪兴,酒到杯干。雨村此时已有七八分酒意,狂兴不禁,乃对月寓怀,口号\footnote{口号——犹言“口占”,不借笔墨、随口吟成。《宋史·乐志》:“乐工致辞,继以诗一章,谓之口号。”}一绝云:
\refdocument{
    \par 时逢三五\footnote{三五——十五,指阴历十五日。}便团圆,满把晴光护玉栏\footnote{ “满把”句——满把:满握。满把晴光:极言月光皎洁充盈。护玉栏:玉石栏杆沉浸在皎洁的月光里。}。
    \par 天上一轮才捧出,人间万姓仰头看。\footnote{“天上”二句——据说赵匡胤未登极时,曾拿《咏月》诗给徐铉看,徐铉读到“未离海底千山黑,才到中天万国明”这两句时,认为帝王之兆已显。见宋代陈师道《后山诗话》。贾诗后两句所抒胸臆类此,故甄士隐说他“飞腾之兆已见”。}
}
\par 士隐听了,大叫:“妙哉!吾每谓兄必非久居人下者,今所吟之句,飞腾之兆已见,不日可接履于云霓之上\footnote{接履于云霓之上——犹言平步青云。接履:一步紧接一步。云霓:喻高位。}矣。可贺,可贺!”乃亲斟一斗为贺。雨村因干过,叹道:“非晚生酒后狂言,若论时尚之学\footnote{时尚之学——时人所崇尚的学问。这里指明清科举考试用的“八股文”和“试帖诗”等。},晚生也或可去充数沽名,只是目今行囊路费一概无措,神京\footnote{神京——与下文“神都”,均指京城。}路远,非赖卖字撰文即能到者。”士隐不待说完,便道:“兄何不早言。愚每有此心,但每遇兄时,兄并未谈及,愚故未敢唐突。今既及此,愚虽不才,‘义利’二字\footnote{“义利”二字——《论语·里仁》:“君子喻于义,小人喻于利。”义:道义。利:功利,这里指钱财。}却还识得。且喜明岁正当大比,兄宜作速入都,春闱\footnote{大比、春闱(wéi围)——明清科举制,考试分为三级。第一级是院试,考府县的童生,考取的为“生员”(秀才);第二级是乡试,考一省的生员,考取的为“举人”;第三级是会试,考全国的举人,考取的为“贡士”(再经殿试赐进士出身)。乡试、会试均三年一科,也称“大比”。乡试在秋天,称为“秋闱”;会试在春天,称为“春闱”。闱:指考场。这里的“大比”是指会试。}一战,方不负兄之所学也。其盘费馀事,弟自代为处置,亦不枉兄之谬识矣!”当下即命小童进去,速封五十两白银,并两套冬衣。又云:“十九日乃黄道之期,兄可即买舟西上,待雄飞高举,明冬再晤,岂非大快之事耶!”雨村收了银衣,不过略谢一语,并不介意,仍是吃酒谈笑。那天已交了三更,二人方散。
\par 士隐送雨村去后,回房一觉,直至红日三竿方醒。因思昨夜之事,意欲再写两封荐书与雨村带至神都,使雨村投谒个仕宦之家为寄足之地。因使人过去请时,那家人去了回来说:“和尚说,贾爷今日五鼓已进京去了,也曾留下话与和尚转达老爷,说‘读书人不在黄道黑道\footnote{黄道黑道——为我国古代天文学的专名,黄道指日,黑道指月。《汉书·天文志》:“日有中道”,“中道者黄道,一曰光道。”又云:“月有九行者,黑道二。”后星占者将每日的干支阴阳分为“黄道”和“黑道”,黄道主吉,黑道主凶。},总以事理为要,不及面辞了。’”士隐听了,也只得罢了。
\par 真是闲处光阴易过,倏忽又是元宵佳节矣。士隐命家人霍启抱了英莲去看社火花灯\footnote{社火花灯——这里指元宵节灯火。社:社日。祭祀土神之日,分春秋两祭,立春后第五个戊日为春社,立秋后第五个戊日为秋社。社火:社日扮演的各种杂戏。花灯:正月十五元宵节有放花灯的习俗。},半夜中,霍启因要小解,便将英莲放在一家门槛上坐着。待他小解完了来抱时,那有英莲的踪影?急得霍启直寻了半夜,至天明不见,那霍启也就不敢回来见主人,便逃往他乡去了。那士隐夫妇,见女儿一夜不归,便知有些不妥,再使几人去寻找,回来皆云连音响皆无。夫妻二人,半世只生此女,一旦失落,岂不思想,因此昼夜啼哭,几乎不曾寻死。看看的一月,士隐先就得了一病;当时封氏孺人\footnote{孺人——《礼记·曲礼下》:“天子之妃曰后,诸侯曰夫人,大夫曰孺人,士曰妇人,庶人曰妻。”孺人在明清为七品官之母或妻的封号。后通用为妇人的尊称。}也因思女构疾,日日请医疗治。
\par 不想这日三月十五,葫芦庙中炸供\footnote{炸供——油炸供神用的食品。},那些和尚不加小心,致使油锅火逸,便烧着窗纸。此方人家多用竹篱木壁者,大抵也因劫数,于是接二连三,牵五挂四,将一条街烧得如火焰山一般。彼时虽有军民来救,那火已成了势,如何救得下?直烧了一夜,方渐渐的熄去,也不知烧了几家。只可怜甄家在隔壁,早已烧成一片瓦砾场了。只有他夫妇并几个家人的性命不曾伤了。急得士隐惟跌足长叹而已。只得与妻子商议,且到田庄上去安身。偏值近年水旱不收,鼠盗蜂起,无非抢田夺地,鼠窃狗偷,民不安生,因此官兵剿捕,难以安身。士隐只得将田庄都折变了,便携了妻子与两个丫鬟投他岳丈家去。
\par 他岳丈名唤封肃,本贯大如州人氏,虽是务农,家中都还殷实。今见女婿这等狼狈而来,心中便有些不乐。幸而士隐还有折变田地的银子未曾用完,拿出来托他随分就价薄置些须房地,为后日衣食之计。那封肃便半哄半赚,些须与他些薄田朽屋。士隐乃读书之人,不惯生理稼穑等事,勉强支持了一二年,越觉穷了下去。封肃每见面时,便说些现成话,且人前人后又怨他们不善过活,只一味好吃懒作等语。士隐知投人不着,心中未免悔恨,再兼上年惊唬,急忿怨痛,已有积伤,暮年之人,贫病交攻,竟渐渐的露出那下世的光景来\footnote{“下世”句——下世:此指死亡。全句是指快要死亡、不久于世的意思。}。
\par 可巧这日拄了拐杖挣挫到街前散散心时,忽见那边来了一个跛足道人,疯癫落脱\footnote{落脱——即“落拓”、“落托”。这里是行为狂放的意思。},麻屣鹑衣\footnote{麻屣(xǐ洗)鹑(chún纯)衣——麻屣:麻鞋。鹑:鹌鹑,鸟名。其尾短秃,如补绽百结,故称破烂衣服为鹑衣。},口内念着几句言词,道是:
\refdocument{
    \par 世人都晓神仙好,惟有功名忘不了!
    \par 古今将相在何方?荒冢一堆草没了。
    \par 世人都晓神仙好,只有金银忘不了!
    \par 终朝只恨聚无多,及到多时眼闭了。
    \par 世人都晓神仙好,只有姣妻忘不了!
    \par 君生日日说恩情,君死又随人去了。
    \par 世人都晓神仙好,只有儿孙忘不了!
    \par 痴心父母古来多,孝顺儿孙谁见了?
}
\par 士隐听了,便迎上来道:“你满口说些什么?只听见些‘好’‘了’‘好’‘了’。”那道人笑道:“你若果听见‘好’‘了’二字,还算你明白。可知世上万般,好便是了,了便是好。若不了,便不好;若要好,须是了。我这歌儿,便名《好了歌》。”士隐本是有宿慧\footnote{宿慧——佛家用语。指超越常人的智慧,认为这种智慧是宿世(即前世)带来的。}的,一闻此言,心中早已彻悟\footnote{彻悟——即佛教所说的大彻大悟,看破红尘。}。因笑道:“且住!待我将你这《好了歌》解注出来何如?”道人笑道:“你解,你解。”士隐乃说道:
\par 陋室空堂,当年笏满床\footnote{笏(hù户)满床——形容家中做大官的人很多。笏:一名“手板”。封建时代臣僚上朝时手中所拿的狭长板子,用象牙或木、竹片制成,可作临时记事之用。};衰草枯杨,曾为歌舞场。蛛丝儿结满雕梁,绿纱今又糊在蓬窗上。说什么脂正浓、粉正香,如何两鬓又成霜?昨日黄土陇头\footnote{黄土陇(lǒnɡ拢)头——指坟墓。陇:通“垄”,田中高地;坟墓。}送白骨,今宵红灯帐底卧鸳鸯。金满箱,银满箱,展眼乞丐人皆谤。正叹他人命不长,那知自己归来丧!训有方,保不定日后作强梁\footnote{强梁——横暴;蛮不讲理。《庄子·山木》:“从其强梁。”唐代陆德明《释文》:“强梁,多力也。”这里指强盗。}。择膏粱\footnote{择膏粱——意谓挑选富贵人家子弟作婿。膏:脂肪;油。粱:精米。膏粱:本指精美的饭菜,这里用作“膏粱子弟”的省称。},谁承望流落在烟花巷\footnote{烟花巷——旧时妓院聚集的地方。烟花:歌女;娼妓。}!因嫌纱帽小,致使锁枷扛;昨怜破袄寒,今嫌紫蟒\footnote{紫蟒——紫色的蟒袍。紫:古代按官阶等级穿着不同颜色的公服;唐制,亲王及三品服用紫色。}长:乱烘烘你方唱罢我登场,反认他乡是故乡\footnote{反认他乡是故乡——这里把现实人生比作暂时寄居的他乡,而把超脱尘世的虚幻世界当作人生本源的故乡;因而说那些为功名利禄、姣妻美妾、儿女后事奔忙而忘掉人生本源的人是错将他乡当作故乡。}。甚荒唐,到头来都是为他人作嫁衣裳\footnote{为他人作嫁衣裳——喻白白替他人奔忙,死后一切皆空。唐代秦韬玉《贫女》诗:“苦恨年年压金线,为他人作嫁衣裳。”}!
\par 那疯跛道人听了,拍掌笑道:“解得切,解得切!”士隐便说一声“走罢!”将道人肩上褡裢\footnote{褡裢——一种中间开口而两端装钱物的长方口袋,小的可以挂在腰带上,大的可以搭在肩膀上。}抢了过来背着,竟不回家,同了疯道人飘飘而去。当下烘动街坊,众人当作一件新闻传说。封氏闻得此信,哭个死去活来,只得与父亲商议,遣人各处访寻,那讨音信?无奈何,少不得依靠着他父母度日。幸而身边还有两个旧日的丫鬟服侍,主仆三人,日夜作些针线发卖,帮着父亲用度。那封肃虽然日日抱怨,也无可奈何了。
\par 这日,那甄家大丫鬟在门前买线,忽听街上喝道之声,众人都说新太爷到任。丫鬟于是隐在门内看时,只见军牢快手\footnote{军牢快手——封建官吏手下执行缉捕、防卫和行刑的隶卒。官僚出巡,常由他们前呼后拥,以示威势。},一对一对的过去,俄而大轿抬着一个乌帽猩袍的官府过去。丫鬟倒发了个怔,自思这官好面善,倒像在那里见过的。于是进入房中,也就丢过不在心上。至晚间,正待歇息之时,忽听一片声打的门响,许多人乱嚷,说:“本府太爷差人来传人问话。”封肃听了,唬得目瞪口呆,不知有何祸事,且听下回分解。\footnote{“且听下回分解”——底本无,俄藏本作“下回便晓”。据卞藏本补。}




\subsection*{第二回\ 贾夫人仙逝扬州城\ 冷子兴演说\footnote{演说——铺演陈说。《北史·熊安生传》:“问所疑,安生皆一一演说,咸就其根本。”}荣国府\footnote{本回回目之后,原有一大段解题性文字和回前诗,今迻录于下:“此回亦非正文。本旨只在冷子兴一人,即俗〈语所〉谓(〈\ 〉中系底本旁添或旁改文字,下同)冷中出热、无中生有也。其演说荣府一篇者,盖因族大人多,若从作者笔下一一叙出,〔尽〕(〔\ 〕中系原文点去者,下同)一二回不能〈说〉明,〔则〕成何文字。故借用冷〔字〕〈子兴〉一人略出其文,〈好〉使阅者心中已有一荣府隐隐在心。然后用黛玉宝钗等两三次皴染,〔则〕〈必〉耀然于心中眼中矣。此即画家三染法也。未写荣府正人,先写外戚,是由远及近、由小及大也。若使先叙出荣府,然后一一叙及外戚,又一一至朋友,至奴仆,其死反(疑系“板”之抄误)拮据之笔,岂作十二钗人手中之物也。今先写外戚者,正是写荣国一府也。故又怕闲文[插图]瘰,开笔即写贾夫人已死,是特使黛玉入荣府之速也。通灵宝玉于士隐梦中一出,今又于子兴口中一出,阅者已洞然矣。然后于黛玉宝钗二人目中极精极细一描,则是文章〈关〉锁何(此字疑衍)处,盖不肯一笔直下,有若放闸之水,然信之爆,使其精华一泄而无馀也。究竟此玉原应出自钗黛目中,方有照应。今预从子兴口中说出,实虽写而却未写。观其后文,可知此一回则是虚敲傍击之文,笔则是反逆隐曲之笔。诗云:
一局输赢料不真,香销茶尽尚逡巡。
欲知目下兴衰兆,须问旁观冷眼人。}}
\par 却说封肃因听见公差传唤,忙出来陪笑启问。那些人只嚷:“快请出甄爷来!”封肃忙陪笑道:“小人姓封,并不姓甄。只有当日小婿姓甄,今已出家一二年了,不知可是问他?”那些公人道:“我们也不知什么‘真’‘假’,因奉太爷之命来问,他既是你女婿,便带了你去亲见太爷面禀,省得乱跑。”说着,不容封肃多言,大家推拥他去了。封家人个个都惊慌,不知何兆。
\par 那天约二更时,只见封肃\footnote{“封肃”,原作“封素”,从各本改。}方回来,欢天喜地。众人忙问端的。他乃说道:“原来本府新升的太爷姓贾名化,本贯湖州人氏,曾与女婿旧日相交。方才在咱门前过去,因见娇杏那丫头买线,所以他只当女婿移住于此。我一一将原故回明,那太爷倒伤感叹息了一回;又问外孙女儿,我说看灯丢了。太爷说:‘不妨,我自使番役\footnote{番役——又称“番子”。原为明代厂卫司中任缉捕的差役。清代则“专司缉捕盗贼,访拿逃亡及娼赌凶棍”。见福格《听雨丛谈》。这里泛称官衙中负责稽查、捕盗的差役。}务必探访回来。’说了一回话,临走倒送了我二两银子。”甄家娘子听了,不免心中伤感。一宿无话。
\par 至次日,早有雨村遣人送了两封银子、四匹锦缎,答谢甄家娘子;又寄一封密书与封肃,转托问甄家娘子要那娇杏\footnote{“娇杏”,原作“姣杏”,俄藏、卞藏本同底本。本回“娇杏”之名凡四见,底本及己卯、蒙府本均“娇”、“姣”通用。今从甲戌、戚序本之一。}作二房。封肃喜的屁滚尿流,巴不得去奉承,便在女儿前一力撺掇\footnote{撺掇(cuān duó汆夺)——怂恿。}成了,乘夜只用一乘小轿,便把娇杏送进去了。雨村欢喜,自不必说,乃封百金赠封肃,外谢甄家娘子许多物事,令其好生养赡,以待寻访女儿下落。封肃回家无话。
\par 却说娇杏这丫鬟,便是那年回顾雨村者。因偶然一顾,便弄出这段事来,亦是自己意料不到之奇缘。谁想他命运两济,不承望自到雨村身边,只一年便生了一子;又半载,雨村嫡妻忽染疾下世,雨村便将他扶侧作正室\footnote{扶侧作正室——谓嫡妻死后将妾作妻。旧时称妻为正室,妾为侧室或曰偏房。《儒林外史》第五回:“王氏道:‘何不向你爷说,明日我若死了,就把你扶正做个填房?’”《啼笑姻缘》第十九回:“你若是跟着我,也许就把你扶正。”}夫人了。正是:
\refdocument{
    \par 偶因一着错,便为人上人。\footnote{“偶因”二句——一着:下棋术语,一步棋谓之一着。这里比喻人的一个行动。女子私顾外人,是封建礼法所不允许的,故云“一着错”;但娇杏却因此由奴婢变为主子,成了“人上人”。}
}
\par 原来,雨村因那年士隐赠银之后,他于十六日便起身入都,至大比之期,不料他十分得意,已会了进士,选入外班\footnote{会了进士,选入外班——指会试考中进士,分发外省任官。进士分为三甲(三等),除一甲三名外,其馀进士再经“朝考”,录取的称庶吉士;没有录取的,经过候选的程序,分发各部或外省听候委用。“班”是指官员补缺的班次。},今已升了本府知府。虽才干优长,未免有些贪酷之弊;且又恃才侮上,那些官员皆侧目而视。不上一年,便被上司寻了个空隙,作成一本,参\footnote{参——参究、稽考,引伸为控告、弹劾。弹劾所用的文书,称“详参”。}他“生情狡猾,擅纂礼仪\footnote{擅纂礼仪——擅纂:擅自纂集。封建时代的礼制仪式,例由礼部掌管,官员擅自纂集,要受惩处。},且沽清正之名,而暗结虎狼之属,致使地方多事,民命不堪”等语。龙颜大怒,即批革职。该部文书一到,本府官员无不喜悦。那雨村心中虽十分惭恨,却面上全无一点怨色,仍是嘻笑自若;交代过公事,将历年做官积的些资本并家小人属送至原籍,安排妥协\footnote{妥协——意谓妥当、妥贴、停当。},却是自己担风袖月,游览天下胜迹。
\par 那日,偶又游至维扬\footnote{维扬——即扬州,今江苏省扬州市。《尚书·禹贡》:“淮海惟扬州。”“惟”,通“维”,后称“维扬”,本此。}地面,因闻得今岁鹾政\footnote{鹾(cuō搓)政——这里指朝廷派到地方管理盐务的官员,带原衔品级。鹾:盐。}点的是林如海。这林如海姓林名海,字表如海,乃是前科的探花\footnote{探花——明、清科举制度,殿试取为第三名者称探花。},今已升至兰台寺大夫\footnote{“兰台寺大夫”,底本、俄藏、卞藏本作“蓝台寺大人”,从己卯、梦稿、甲戌、蒙府、戚序、甲辰本改。}\footnote{兰台寺大夫——作者沿古虚拟的官名。兰台是汉朝宫内藏书的地方,由御史中丞主管,兼任纠察。后因称主管弹劾的御史台为兰台,御史府也叫兰台寺,设官曰兰台史令。},本贯姑苏人氏,今钦点出为巡盐御史,到任方一月有馀。原来这林如海之祖,曾袭过列侯,今到如海,业经五世。起初时,只封袭三世,因当今隆恩盛德,远迈前代,额外加恩,至如海之父,又袭了一代;至如海,便从科第出身。虽系钟鼎之家\footnote{钟鼎之家——“钟鸣鼎食之家”的简称。钟:乐器。鼎:一种三足两耳的金属器皿,这里是指盛菜肴的食具。贵族家庭宴享祭祀时,鸣钟列鼎。后常用“钟鼎之家”代指贵族豪门。},却亦是书香之族。只可惜这林家支庶不盛,子孙有限,虽有几门,却与如海俱是堂族而已,没甚亲支嫡派的。今如海年已四十,只有一个三岁之子,偏又于去岁死了。虽有几房姬妾,奈他命中无子,亦无可如何之事。今只有嫡妻贾氏生得一女,乳名黛玉,年方五岁。夫妻无子,故爱如珍宝,且又见他聪明清秀,便也欲使他读书识得几个字,不过假充养子之意,聊解膝下荒凉\footnote{膝下荒凉——指没有子嗣。膝下:指幼儿环绕于父母的膝下,后为子女代称。见《孝经·圣治》。}之叹。
\par 雨村正值偶感风寒,病在旅店,将一月光景方渐愈。一因身体劳倦,二因盘费不继,也正欲寻个合式之处,暂且歇下。幸有两个旧友,亦在此境居住,因闻得鹾政欲聘一西宾\footnote{西宾——亦称西席。古代以西为尊,宾客或教师的座位,居西面东。见梁章钜《称谓录》。故以西宾或西席为家庭教师或官僚幕客的代称。},雨村便相托友力,谋了进去,且作安身之计。妙在只一个女学生,并两个伴读丫鬟,这女学生年又小,身体又极怯弱,工课不限多寡,故十分省力。
\par 堪堪\footnote{堪堪——即“看看”,“看”本多音字,这里读如kān。估量时间之辞,义近转眼。}又是一载的光阴,谁知女学生之母贾氏夫人一疾而终。女学生侍汤奉药,守丧尽哀,遂又将辞馆别图。林如海意欲令女守制\footnote{守制——古人父母或祖父母死后,嫡长子或承重孙(长房嫡长孙)要守孝三年,须闭门读书,谢绝世务,称为“守制”。}读书,故又将他留下。近因女学生哀痛过伤,本自怯弱多病的,触犯旧症,遂连日不曾上学。雨村闲居无聊,每当风日晴和,饭后便出来闲步。
\par 这日,偶至郭外\footnote{郭外——指城外郊区。《孟子·公孙丑下》:“三里之城,七里之郭。”},意欲赏鉴那村野风光。忽信步至一山环水旋、茂林深竹之处,隐隐的有座庙宇,门巷倾颓,墙垣朽败,门前有额,题着“智通寺”三字,门旁又有一副旧破的对联,曰:
\par 身后有馀忘缩手,眼前无路想回头。
\par 雨村看了,因想到:“这两句话,文虽浅近,其意则深。我也曾游过些名山大刹,倒不曾见过这话头,其中想必有个翻过筋斗来的\footnote{翻过筋斗来的——比喻饱经世事动荡或遭受重大挫折后“看破世情”的人。筋斗:一作“觔斗”,通作“跟头”。}亦未可知,何不进去试试。”想着走入,看时只有一个龙钟老僧在那里煮粥。雨村见了,便不在意。及至问他两句话,那老僧既聋且昏,齿落舌钝,所答非所问。
\par 雨村不耐烦,便仍出来,意欲到那村肆\footnote{村肆——这里指乡村酒店。}中沽饮三杯,以助野趣,于是款步行来。将入肆门,只见座上吃酒之客有一人起身大笑,接了出来,口内说:“奇遇,奇遇。”雨村忙看时,此人是都中在古董行中贸易的号冷子兴者,旧日在都相识。雨村最赞这冷子兴是个有作为大本领的人,这子兴又借雨村斯文之名,故二人说话投机,最相契合。
\par 雨村忙笑问道:“老兄何日到此?弟竟不知。今日偶遇,真奇缘也。”子兴道:“去年岁底到家,今因还要入都,从此顺路找个敝友说一句话,承他之情,留我多住两日。我也无紧事,且盘桓两日,待月半时也就起身了。今日敝友有事,我因闲步至此,且歇歇脚,不期这样巧遇!”一面说,一面让雨村同席坐了,另整上酒肴来。二人闲谈漫饮,叙些别后之事。
\par 雨村因问:“近日都中可有新闻没有?”子兴道:“倒没有什么新闻,倒是老先生你贵同宗\footnote{同宗——按古代宗法制度,本指同出于一个远祖者为“同宗”,后用以泛称同族或同姓。}家,出了一件小小的异事。”雨村笑道:“弟族中无人在都,何谈及此?”子兴笑道:“你们同姓,岂非同宗一族?”雨村问是谁家。子兴道:“荣国府贾府中,可也玷辱了先生的门楣么?”雨村笑道:“原来是他家。若论起来,寒族人丁却不少,自东汉贾复\footnote{贾复——东汉南阳冠军(今属河南邓县)人,曾任执金吾、左将军,封胶东侯。见《后汉书·贾复传》。}以来,支派繁盛,各省皆有,谁逐细考查得来?若论荣国一支,却是同谱。但他那等荣耀,我们不便去攀扯,至今故越发生疏难认了。”
\par 子兴叹道:“老先生休如此说。如今的这宁荣两门,也都萧疏了,不比先时的光景。”雨村道:“当日宁荣两宅的人口也极多,如何就萧疏了?”冷子兴道:“正是,说来也话长。”雨村道:“去岁我到金陵地界,因欲游览六朝遗迹,那日进了石头城\footnote{石头城——故址在今南京市。三国时孙权所建。后用以代指金陵或南京。上文讲的六朝(吴、东晋、南朝的宋、齐、梁、陈),皆建都金陵。},从他老宅门前经过。街东是宁国府,街西是荣国府,二宅相连,竟将大半条街占了。大门前虽冷落无人,隔着围墙一望,里面厅殿楼阁,也还都峥嵘轩峻;就是后一带花园子里面树木山石,也还都有蓊蔚洇润\footnote{蓊(wēnɡ翁)蔚洇(yīn因)润——茂盛润泽的样子。}之气,那里像个衰败之家?”冷子兴笑道:“亏你是进士出身,原来不通!古人有云:‘百足之虫,死而不僵。’\footnote{百足之虫,死而不僵——比喻大贵族官僚家庭,虽已衰败,但表面仍能维持某种繁荣的假象。语见三国魏曹冏(jǒnɡ窘)《六代论》。百足之虫:指马陆、蜈蚣一类节肢动物。僵:仆倒。}如今虽说不及先年那样兴盛,较之平常仕宦之家,到底气象不同。如今生齿\footnote{生齿——人口,古代把长出乳齿的男女登入户籍。}日繁,事务日盛,主仆上下,安富尊荣者尽多,运筹谋画者无一;其日用排场费用,又不能将就省俭,如今外面的架子虽未甚倒,内囊却也尽上来了。这还是小事。更有一件大事:谁知这样钟鸣鼎食之家,翰墨诗书之族,如今的儿孙,竟一代不如一代了!”雨村听说,也纳罕道:“这样诗礼之家,岂有不善教育之理?别门不知,只说这宁、荣二宅,是最教子有方的。”
\par 子兴叹道:“正说的是这两门呢。待我告诉你:当日宁国公与荣国公\footnote{“与荣国公”,原无。卞藏本作“宁国府与荣国府”。从己卯、梦稿、甲戌、蒙府、戚序本补。}是一母同胞弟兄两个。宁公居长,生了四个儿子。宁公死后,贾代化袭了官,也养了两个儿子:长名贾敷,至八九岁上便死了,只剩了次子贾敬袭了官,如今一味好道,只爱烧丹炼汞\footnote{烧丹炼汞——道教以朱砂(丹)、水银(汞)等烧炼“仙药”,即所谓“金丹”的一种方术,以此妄求飞升成仙,长生不死。},馀者一概不在心上。幸而早年留下一子,名唤贾珍,因他父亲一心想作神仙,把官倒让他袭了。他父亲又不肯回原籍来,只在都中城外和道士们胡羼\footnote{胡羼(chàn忏)——犹言“鬼混”。《说文》:“羼,羊相厕也。”引申为搀杂。}。这位珍爷倒生了一个儿子,今年才十六岁,名叫贾蓉。如今敬老爹一概不管。这珍爷那里肯读书,只一味高乐\footnote{高乐——恣意寻欢作乐。}不了,把宁国府竟翻了过来,也没有人敢来管他。再说荣府你听,方才所说异事,就出在这里。自荣公死后,长子贾代善袭了官,娶的也是金陵世勋史侯家的小姐为妻,生了两个儿子:长子贾赦,次子贾政。如今代善早已去世,太夫人尚在,长子贾赦袭着官;次子贾政,自幼酷喜读书,祖、父最疼,原欲以科甲出身的,不料代善临终时遗本一上,皇上因恤先臣,即时令长子袭官外,问还有几子,立刻引见,遂额外赐了这政老爹一个主事\footnote{主事——清代六部之下设司,司的主管官是郎中,其副手是员外郎,再下就是主事。下文的入部,指的是工部,主管建筑、水利诸事。}之衔,令其入部习学,如今现已升了员外郎了。这政老爹的夫人王氏,头胎生的公子,名唤贾珠,十四岁进学,不到二十岁就娶了妻生了子,一病死了。第二胎生了一位小姐,生在大年初一,这就奇了;不想次年又生了一位公子,说来更奇,一落胎胞,嘴里便衔下一块五彩晶莹的玉来,上面还有许多字迹,就取名叫作宝玉。你道是新奇异事不是?”
\par 雨村笑道:“果然奇异。只怕这人来历不小。”子兴冷笑道:“万人皆如此说,因而乃祖母便先爱如珍宝。那年周岁时,政老爹便要试他将来的志向,便将那世上所有之物摆了无数,与他抓取\footnote{抓取——即“抓周”,俗名“试儿”。婴儿满一周岁,家人陈列各种物品、用具,任其抓取,以预测他未来的志向和前途。见《颜氏家训·风操》。}。谁知他一概不取,伸手只把些脂粉钗环抓来。政老爹便大怒了,说:‘将来酒色之徒耳!’因此便大不喜悦。独那史老太君还是命根一样。说来又奇,如今长了七八岁,虽然淘气异常,但其聪明乖觉处,百个不及他一个。说起孩子话来也奇怪,他说:‘女儿是水作的骨肉\footnote{底本“水”涂改为“木”,俄藏、卞藏本作“木”。甲戌、蒙府、戚序、甲辰、舒序本均作“水”,从原文。},男人是泥作的骨肉。我见了女儿,我便清爽;见了男子,便觉浊臭逼人。’你道好笑不好笑?将来色鬼无疑了!”雨村罕然厉色忙止道:“非也!可惜你们不知道这人来历。大约政老前辈也错以淫魔色鬼看待了。若非多读书识事,加以致知格物之功,悟道参玄\footnote{致知格物、悟道参玄——致知格物:语出《大学》:“致知在格物,物格而后知至。”致:推导。格:推究。悟道参玄:宗教用语。领会和推究宗教中玄妙的道理。}之力,不能知也。”
\par 子兴见他说得这样重大,忙请教其端。雨村道:“天地生人,除大仁大恶两种,馀者皆无大异。若大仁者,则应运而生,大恶者,则应劫而生\footnote{应运而生、应劫而生——运:宋邵雍《皇极经世书》中,以三十年为一世,十二世为一运,三十运为一会,十二会为一元。这里指气数,吉祥和顺的时代气运。劫:这里指时代的灾难、厄运。}。运生世治,劫生世危。尧、舜、禹、汤、文、武、周、召、孔、孟、董、韩、周、程、张、朱,皆应运而生者。蚩尤、共工、桀、纣、始皇、王莽、曹操、桓温、安禄山、秦桧\footnote{尧、舜……秦桧——尧、舜:即唐尧虞舜,传说中原始社会的两个部落联盟领袖。禹:夏禹,夏代开国君主。汤:成汤,商代开国君主。文、武:周文王、周武王。姬姓,周朝开国的两个君主。周、召:即周公旦、召(shào邵)公奭(shì试)。周武王的两个弟弟,也是辅佐他开国的两个大臣。孔、孟:即孔丘、孟轲,两个儒家代表人物。董:西汉经学家董仲舒。韩:唐代文学家韩愈。周:北宋理学家周敦颐。程:北宋理学家程颢(hào号)、程颐兄弟。张:北宋思想家张载。朱:南宋理学家朱熹。蚩尤:传说中的上古部族首领,曾与黄帝战于“涿鹿之野”。共工:传说中的“四凶”之一。桀:夏朝末代君主。纣:商朝末代君主。始皇:即秦始皇。王莽:西汉末年的大官僚贵族,篡位称帝,改国号为新。曹操:即魏武帝,三国时政治家、军事家。桓温:东晋时大司马,专擅朝政。安禄山:胡人,唐玄宗时节度使,曾与史思明发起“安史之乱”。秦桧:南宋高宗时宰相,历史上有名的奸臣。}等,皆应劫而生者。大仁者,修治天下;大恶者,挠乱天下。清明灵秀,天地之正气,仁者之所秉也;残忍乖僻,天地之邪气,恶者之所秉也。今当运隆祚永\footnote{运隆祚永——国运兴隆,皇位传世久远。}之朝,太平无为之世,清明灵秀之气所秉者,上至朝廷,下及草野,比比皆是。所馀之秀气,漫无所归,遂为甘露,为和风,洽然\footnote{洽(qià恰)然——协和滋润的样子。}溉及四海。彼残忍乖僻之邪气,不能荡溢于光天化日之中,遂凝结充塞于深沟大壑之内,偶因风荡,或被云摧,略有摇动感发之意,一丝半缕误而泄出者,偶值灵秀之气适过,正不容邪,邪复妒正,两不相下,亦如风水雷电,地中既遇,既不能消,又不能让,必至搏击掀发后始尽。故其气亦必赋人,发泄一尽始散。使男女偶秉此气而生者,在上则不能成仁人君子,下亦不能为大凶大恶。置之于万万人中,其聪俊灵秀之气,则在万万人之上;其乖僻邪谬不近人情之态,又在万万人之下。若生于公侯富贵之家,则为情痴情种;若生于诗书清贫之族,则为逸士高人;纵再偶生于薄祚寒门,断不能为走卒健仆,甘遭庸人驱制驾驭,必为奇优名倡。如前代之许由、陶潜、阮籍、嵇康、刘伶、王谢二族、顾虎头、陈后主、唐明皇、宋徽宗、刘庭芝、温飞卿、米南宫、石曼卿、柳耆卿、秦少游,近日之倪云林、唐伯虎、祝枝山,再如李龟年、黄幡绰、敬新磨、卓文君、红拂、薛涛、崔莺、朝云\footnote{许由……朝云——许由:传说中的上古高士。陶潜:东晋诗人。阮籍:魏晋之际诗人。嵇康:魏晋之际文学家。刘伶:与阮、嵇皆为“竹林七贤”之一。王谢二族:指东晋王导、谢安两家族。顾虎头:顾恺之,字虎头。东晋名画家。陈后主:南朝陈末代皇帝陈叔宝。唐明皇:即唐玄宗李隆基。宋徽宗:北宋末代皇帝赵佶(jí急)。刘庭芝:刘希夷,字庭芝,唐代诗人。温飞卿:温庭筠,字飞卿,晚唐诗人。米南宫:即米芾,北宋书画家。石曼卿:石延年,字曼卿,北宋文学家。柳耆卿:柳永,字耆卿,北宋词人。秦少游:秦观,字少游,北宋词人。倪云林:倪瓒(zàn赞),号云林子,元代山水画家。唐伯虎:唐寅,字伯虎,明代画家、文学家。祝枝山:祝允明,号枝山,明代书法家、文学家。李龟年:唐玄宗时宫廷乐师。黄幡绰(fān chuò番辍):唐玄宗时艺人。敬新磨:五代后唐庄宗时宫廷艺人。卓文君:见第5页注⑤。红拂:隋代越国公杨素的侍女,私奔李靖。薛涛:唐代名妓。崔莺:即《会真记》中的莺莺。朝云:宋代钱塘名妓。}之流,此皆易地则同之人也。”
\par 子兴道:“依你说,‘成则王侯败则贼’了。”雨村道:“正是这意。你还不知,我自革职以来,这两年遍游各省,也曾遇见两个异样孩子。所以,方才你一说这宝玉,我就猜着了八九亦是这一派人物。不用远说,只金陵城内,钦差金陵省体仁院总裁\footnote{钦差金陵省体仁院总裁——钦差:由皇帝指派出外办理重大事情的官员,其中由皇帝特命并授予关防者,权力更大,称“钦差大臣”。体仁院总裁:作者虚拟的官衔。}甄家,你可知么?”子兴道:“谁人不知!这甄府和贾府就是老亲,又系世交。两家来往,极其亲热的。便在下也和他家来往非止一日了。”
\par 雨村笑道:“去岁我在金陵,也曾有人荐我到甄府处馆。我进去看其光景,谁知他家那等显贵,却是个富而好礼之家,倒是个难得之馆。但这一个学生,虽是启蒙,却比一个举业\footnote{启蒙、举业——启蒙:启发蒙昧。这里指旧时儿童开始上学读书,读物有《三字经》、《百家姓》之类。举业:指旧时科举应试,其读物有《四书》、《五经》之类。}的还劳神。说起来更可笑,他说:‘必得两个女儿伴着我读书,我方能认得字,心里也明白;不然我自己心里糊涂。’又常对跟他的小厮们说:‘这女儿两个字,极尊贵、极清净的,比那阿弥陀佛\footnote{阿弥陀佛——梵文音译,习称“弥陀”。意译为“无量寿”、“无量光”,为大乘佛教的佛名。佛经说他是“极乐世界”的教主,净土宗宣称诵此名号,即可往西方“极乐世界”。}、元始天尊\footnote{元始天尊——道教的尊神。道经说他“生于太元之先”,故称“元始”。他居于天界最高的“玉清”仙境,为“三清”之首。}的这两个宝号还更尊荣无对的呢!你们这浊口臭舌,万不可唐突了这两个字要紧。但凡要说时,必须先用清水香茶漱了口才可;设若失错,便要凿牙穿腮等事。’其暴虐浮躁,顽劣憨痴,种种异常。只一放了学,进去见了那些女儿们,其温厚和平,聪敏文雅,竟又变了一个人了。因此,他令尊也曾下死笞楚\footnote{笞(chī痴)楚——即鞭打;抽打。笞:竹板。楚:荆条。都是打人的工具。这里作动词用。}过几次,无奈竟不能改。每打的吃疼不过时,他便‘姐姐’‘妹妹’乱叫起来。后来听得里面女儿们拿他取笑:‘因何打急了只管叫姐妹做甚?莫不是求姐妹去说情讨饶?你岂不愧些!’他回答的最妙。他说:‘急疼之时,只叫“姐姐”“妹妹”字样,或可解疼也未可知,因叫了一声,便果觉不疼了,遂得了秘法:每疼痛之极,便连叫姐妹起来了。’你说可笑不可笑?也因祖母溺爱不明,每因孙辱师责子,因此我就辞了馆出来。如今在这巡盐御史林家做馆了。你看,这等子弟,必不能守祖父之根基,从师长之规谏的。只可惜他家几个姊妹都是少有的。”
\par 子兴道:“便是贾府中,现有的三个也不错。政老爹的长女,名元春,现因贤孝才德,选入宫中作女史\footnote{女史——古代宫中女官名。掌管王后的礼职。见《周礼·天官·女史》。后也成为尊贵、文雅女子的泛称。}去了。二小姐乃赦老爹之妾所出\footnote{“赦老爹之妾所出”,原作“政老爹前妻所出”。诸本各异:己卯、梦稿本作“赦老爷之女,政老爷养为己女”;甲辰本作“赦老爷姨娘所出”;舒序本作“赦老爷前妻所出”;甲戌本作“赦老爹前妻所出”。俄藏、卞藏本作“赦老爷之妻所生”。蒙府、戚序本作“赦老爷之妾所出”,从之,并据上下文改“爷”为“爹”。},名迎春;三小姐乃政老爹之庶出,名探春;四小姐乃宁府珍爷之胞妹,名唤惜春。因史老夫人极爱孙女,都跟在祖母这边一处读书,听得个个不错。”雨村道:“更妙在甄家的风俗,女儿之名,亦皆从男子之名命字,不似别家另外用这些‘春’、‘红’、‘香’、‘玉’等艳字的。何得贾府亦落此俗套?”子兴道:“不然。只因现今大小姐是正月初一日所生,故名元春,馀者方从了‘春’字。上一辈的,却也是从弟兄而来的。现有对证:目今你贵东家林公之夫人,即荣府中赦、政二公之胞妹,在家时名唤贾敏。不信时,你回去细访可知。”雨村拍案笑道:“怪道这女学生读至凡书中有‘敏’字,皆念作‘密’\footnote{“敏”念“密”——古代有避讳之制,对君亲的名字,不能直读其音,直书其字。必须改字、改音或省笔,以示敬避之意。}字,每每如是;写字遇着‘敏’字,又减一二笔,我心中就有些疑惑。今听你说,的是\footnote{的是——的确是。}为此无疑矣。怪道我这女学生言语举止另是一样,不与近日女子相同,度其母必不凡,方得其女,今知为荣府外孙,又不足罕矣,可伤上月竟亡故了。”子兴叹道:“老姊妹四个,这一个是极小的,又没了。长一辈的姊妹,一个也没了。只看这小一辈的,将来之东床\footnote{东床——指女婿。晋代太尉郗(xī希)鉴派人到丞相王导家选女婿,王家的子弟都很矜持,惟独王羲之不以为意,坦腹躺在东床上吃东西。郗鉴欣赏他这种“名士”风度,就选中了他。见《世说新语·雅量》。}如何呢?”
\par 雨村道:“正是。方才说这政公,已有衔玉之儿,又有长子所遗一个弱孙。这赦老竟无一个不成?”子兴道:“政公既有玉儿之后,其妾又生了一个,倒不知其好歹。只眼前现有二子一孙,却不知将来如何。若问那赦公,也有二子,长名贾琏,今已二十来往了,亲上作亲,娶的就是政老爹夫人王氏之内侄女,今已娶了二年。这位琏爷身上现捐的是个同知,也是不肯读书,于世路上好机变,言谈去的,所以如今只在乃叔政老爷家住着,帮着料理些家务。谁知自娶了他令夫人之后,倒上下无一人不称颂他夫人的,琏爷倒退了一射之地\footnote{一射之地——约当一百二十至一百五十步。亦称“一箭道”。见宋代法云编《翻译名义集·数量》。}:说模样又极标致,言谈又爽利,心机又极深细,竟是个男人万不及一的。”
\par 雨村听了,笑道:“可知我前言不谬。你我方才所说的这几个人,都只怕是那正邪两赋而来一路之人,未可知也。”子兴道:“邪也罢,正也罢,只顾算别人家的帐,你也吃一杯酒才好。”雨村道:“正是,只顾说话,竟多吃了几杯。”子兴笑道:“说着别人家的闲话,正好下酒,即多吃几杯何妨。”雨村向窗外看道:“天也晚了,仔细关了城。我们慢慢的进城再谈,未为不可。”于是,二人起身,算还酒帐。方欲走时,又听得后面有人叫道:“雨村兄,恭喜了!特来报个喜信的。”雨村忙回头看时——



\subsection*{第三回\ 贾雨村夤缘\footnote{夤(yín寅)缘——攀附权要以求升进。}复旧职\ 林黛玉抛父进京都}
\par 却说雨村忙回头看时,不是别人,乃是当日同僚一案参革的号张如圭者。他本系此地人,革后家居,今打听得都中奏准起复\footnote{起复——旧时官吏因事降革者,恢复原官、原衔叫“开复”;因父母之丧离职,守孝期满而复用者叫“起复”。“起复”和“开复”在习惯用法上没有严格区别。}旧员之信,他便四下里寻情找门路,忽遇见雨村,故忙道喜。二人见了礼,张如圭便将此信告诉雨村,雨村自是欢喜,忙忙的叙了两句,遂作别各自回家。冷子兴听得此言,便忙献计,令雨村央烦林如海,转向都中去央烦贾政。雨村领其意,作别回至馆中,忙寻邸报\footnote{邸(dǐ底)报——邸:本为来朝诸侯王或上京办事官僚的居处,后用以泛称王侯和大官僚的府第。邸报:又名“邸钞”、“宫门钞”,是邸中给诸藩官僚的书面报导,内容包括传钞的诏令、奏章和其他新闻记事等,是我国最早的一种报纸。起于汉代。后世亦称政府官报为邸报。}看真确了。
\par 次日,面谋之如海。如海道:“天缘凑巧,因贱荆\footnote{贱荆——荆:指“荆钗布裙”。语出《列女传》。旧时谦称自己的妻子为贱荆、拙荆、山荆等。}去世,都中家岳母念及小女无人依傍教育,前已遣了男女\footnote{男女——指仆人。}船只来接,因小女未曾大痊,故未及行。此刻正思向蒙训教之恩未经酬报,遇此机会,岂有不尽心图报之理。但请放心。弟已预为筹画至此,已修下荐书一封,转托内兄务为周全协佐,方可稍尽弟之鄙诚,即有所费用之例,弟于内兄信中已注明白,亦不劳尊兄多虑矣。”雨村一面打恭,谢不释口,一面又问:“不知令亲大人现居何职?只怕晚生草率,不敢骤然入都干渎\footnote{干渎(dú读)——又作“干黩”。冒犯的意思。}。”如海笑道:“若论舍亲,与尊兄犹系同谱,乃荣公之孙:大内兄现袭一等将军,名赦,字恩侯;二内兄名政,字存周,现任工部员外郎,其为人谦恭厚道,大有祖父遗风,非膏粱轻薄仕宦之流,故弟方致书烦托。否则不但有污尊兄之清操,即弟亦不屑为矣。”雨村听了,心下方信了昨日子兴之言,于是又谢了林如海。如海乃说:“已择了出月初二日小女入都,尊兄即同路而往,岂不两便?”雨村唯唯听命,心中十分得意。如海遂打点礼物并饯行之事,雨村一一领了。
\par 那女学生黛玉,身体方愈,原不忍弃父而往;无奈他外祖母致意务去,且兼如海说:“汝父年将半百,再无续室之意;且汝多病,年又极小,上无亲母教养,下无姊妹兄弟扶持,今依傍外祖母及舅氏姊妹去,正好减我顾盼之忧,何反云不往?”黛玉听了,方洒泪拜别,随了奶娘及荣府几个老妇人登舟而去。雨村另有一只船,带两个小童,依附黛玉而行。
\par 有日到了都中,进入神京\footnote{都中、神京——这里说的“都中”包括京畿,即京城周围地区。神京:即京城。},雨村先整了衣冠,带了小童,拿着宗侄的名帖\footnote{名帖——即名片。旧时在纸片上书写自己的姓名、籍贯、官职、爵位,拜访时,投以通名。始行汉代,最早用削平的木条写上姓名里居。两汉时叫“谒”,汉末叫“刺”,后代虽用纸制,亦相沿称“名刺”。},至荣府的门前投了。彼时贾政已看了妹丈之书,即忙请入相会。见雨村相貌魁伟,言语不俗,且这贾政最喜读书人,礼贤下士,济弱扶危,大有祖风;况又系妹丈致意,因此优待雨村,更又不同,便竭力内中协助。题奏之日,轻轻谋了一个复职候缺,不上两个月,金陵应天府缺出,便谋补了此缺,拜辞了贾政,择日上任去了。不在话下。
\par 且说黛玉自那日弃舟登岸时,便有荣国府打发了轿子并拉行李的车辆久候了。这林黛玉常听得母亲说过,他外祖母家与别家不同。他近日所见的这几个三等仆妇,吃穿用度,已是不凡了,何况今至其家。因此步步留心,时时在意,不肯轻易多说一句话,多行一步路,惟恐被人耻笑了他去。
\par 自上了轿,进入城中,从纱窗向外瞧了一瞧,其街市之繁华,人烟之阜盛,自与别处不同。又行了半日,忽见街北蹲着两个大石狮子,三间兽头大门,门前列坐着十来个华冠丽服之人。正门却不开,只有东西两角门有人出入。正门之上有一匾,匾上大书“敕造\footnote{敕(chì斥)造——奉皇帝之命建造。敕:本为自上命下之词,南北朝以前,通用于长官对下属,长辈对晚辈,之后,则为皇帝发布诏令的专称。}宁国府”五个大字。黛玉想道:“这必是外祖之长房了。”想着,又往西行,不多远,照样也是三间大门,方是荣国府了。却不进正门,只进了西边角门。那轿夫抬进去,走了一射之地,将转弯时,便歇下退出去了。后面的婆子们已都下了轿,赶上前来。另换了三四个衣帽周全十七八岁的小厮上来,复抬起轿子。众婆子步下围随至一垂花门\footnote{垂花门——旧家宅院,进入大门之后,内院院门例有雕刻的垂花,倒悬于门额两侧,门上边盖有宫殿式的小屋顶,称垂花门。}前落下。众小厮退出,众婆子上来打起轿帘,扶黛玉下轿。林黛玉扶着婆子的手,进了垂花门,两边是抄手游廊\footnote{抄手游廊——院门内两侧环抱的走廊。},当中是穿堂\footnote{穿堂——座落在前后两个院落之间可以穿行的厅堂。},当地放着一个紫檀架子大理石的大插屏\footnote{大插屏——放在穿堂中的大屏风,除作装饰外,还可以遮蔽视线,以免进入穿堂,直见正房。}。转过插屏\footnote{“转过插屏”,原无,从己卯、甲戌、蒙府、戚序、俄藏、卞藏本补。},小小的三间厅,厅后就是后面的正房大院。正面五间上房,皆雕梁画栋,两边穿山游廊\footnote{穿山游廊——山:指山墙,房子两侧的墙。“穿山游廊”是从山墙开门接起的游廊。}厢房,挂着各色鹦鹉、画眉等鸟雀。台矶之上,坐着几个穿红着绿的丫头,一见他们来了,便忙都笑迎上来,说:“刚才老太太还念呢,可巧就来了。”于是三四人争着打起帘笼,一面听得人回话:“林姑娘到了。”
\par 黛玉方进入房时,只见两个人搀着一位鬓发如银的老母迎上来,黛玉便知是他外祖母。方欲拜见时,早被他外祖母一把搂入怀中,心肝儿肉叫着大哭起来。当下地下侍立之人,无不掩面涕泣,黛玉也哭个不住。一时众人慢慢解劝住了,黛玉方拜见了外祖母。——此即冷子兴所云之史氏太君,贾赦贾政之母也。当下贾母一一指与黛玉:“这是你大舅母;这是你二舅母;这是你先珠大哥的媳妇珠大嫂子。”黛玉一一拜见过。贾母又说:“请姑娘们来。今日远客才来,可以不必上学去了。”众人答应了一声,便去了两个。
\par 不一时,只见三个奶嬷嬷并五六个丫鬟,簇拥着三个姊妹来了。第一个肌肤微丰,合中身材,腮凝新荔,鼻腻鹅脂,温柔沉默,观之可亲。第二个削肩细腰,长挑身材,鸭蛋脸面,俊眼修眉,顾盼神飞,文彩精华,见之忘俗。第三个身量未足,形容尚小。其钗环裙袄,三人皆是一样的妆饰。黛玉忙起身迎上来见礼,互相厮认过,大家归了坐。丫鬟们斟上茶来。不过说些黛玉之母如何得病,如何请医服药,如何送死发丧。不免贾母又伤感起来,因说:“我这些儿女,所疼者独有你母,今日一旦先舍我而去,连面也不能一见,今见了你,我怎不伤心!”说着,搂了黛玉在怀,又呜咽起来。众人忙都宽慰解释\footnote{解释——劝解消释,去烦除恼。},方略略止住。
\par 众人见黛玉年貌虽小,其举止言谈不俗,身体面庞虽怯弱不胜,却有一段自然的风流态度,便知他有不足之症\footnote{不足之症——中医病症名。由身体虚弱引起。如脾胃虚弱,叫中气不足;气血虚弱,叫正气不足。}。因问:“常服何药,如何不急为疗治?”黛玉道:“我自来是如此,从会吃饮食时便吃药,到今日未断,请了多少名医修方配药,皆不见效。那一年我三岁时,听得说来了一个癞头和尚,说要化我去出家,我父母固是不从。他又说:‘既舍不得他,只怕他的病一生也不能好的了。若要好时,除非从此以后总不许见哭声;除父母之外,凡有外姓亲友之人,一概不见,方可平安了此一世。’疯疯癫癫,说了这些不经之谈,也没人理他。如今还是吃人参养荣丸。”贾母道:“正好,我这里正配丸药呢。叫他们多配一料就是了。”
\par 一语未了,只听后院中有人笑声,说:“我来迟了,不曾迎接远客!”黛玉纳罕道:“这些人个个皆敛声屏气,恭肃严整如此,这来者系谁,这样放诞无礼?”心下想时,只见一群媳妇丫鬟围拥着一个人从后房门进来。这个人打扮与众姑娘不同:彩绣辉煌,恍若神妃仙子。头上戴着金丝八宝攒珠髻\footnote{金丝八宝攒(cuán)珠髻——用金丝穿绕珍珠和镶嵌八宝(玛瑙、碧玉之类)制成的珠花的发髻。攒:凑聚。用金丝或银丝把珍珠穿扭成各种花样叫“攒珠花”。},绾着朝阳五凤挂珠钗\footnote{朝阳五凤挂珠钗——一种长钗,样子是一支钗上分出五股,每股一支凤凰,口衔一串珍珠。};项上带着赤金盘螭璎珞圈\footnote{赤金盘螭(chī吃)璎珞圈——螭:古代传说中的无角龙。璎珞:联缀起来的珠玉。圈:项圈。};裙边系着豆绿宫绦双衡比目玫瑰珮\footnote{“双衡比目玫瑰珮”,“比目”原作“皆”,从己卯、梦稿、甲戌、俄藏、卞藏、蒙府本改。}\footnote{双衡比目玫瑰珮——珮:玉珮。古代贵族佩带的玉器,常雕琢成各种形状。比目:鱼名,传说这种鱼成双而行。“比目玫瑰珮”是玫瑰色的玉片雕琢成双鱼形的玉珮。衡:亦作“珩”(hén横),珮玉上部的小横杠,用以系饰物。};身上穿着缕金百蝶穿花大红洋缎\footnote{“洋缎”,底本、俄藏、卞藏本作“萍缎”,己卯、梦稿、甲辰、舒序本均同。从甲戌、蒙府、戚序本改。}窄裉袄\footnote{缕金百蝶穿花大红洋缎窄裉(kèn垦去声)袄——指在大红洋缎的衣面上用金线绣成百蝶穿花图案的紧身袄。裉:上衣前后两幅在腋下合缝的部分。},外罩五彩刻丝石青银鼠褂\footnote{五彩刻丝石青银鼠褂——石青色的衣面上有各种彩色刻丝、衣里是银鼠皮的褂子。刻丝:在丝织品上用丝平织成的图案,与凸出的绣花不同。石青:淡灰青色。褂:对襟外衣。};下着翡翠撒花洋绉裙\footnote{翡翠撒花洋绉裙——翡翠:翠绿色。撒花:在绸缎上用散点式的小花点组成的纹饰图案。绉:用拈丝作经,两种不同拈向的强拈丝作纬,平纹织成的丝结物。}。一双丹凤三角眼,两弯柳叶吊梢眉\footnote{“吊梢眉”,底本、俄藏、卞藏本作“掉稍眉”。“梢”从戚序、舒序本改;“掉”、“吊”底本通用,此处俄藏、卞藏“吊”是。}\footnote{丹凤三角眼、柳叶吊梢眉——眼角向上微翘,俗称“丹凤眼”。柳叶吊梢眉:形容眉梢斜飞入鬓的样子。}],身量苗条,体格风骚。粉面含春威不露,丹唇未启笑先闻。黛玉连忙起身接见。贾母笑道:“你不认得他,他是我们这里有名的一个泼皮破落户儿\footnote{泼皮破落户儿——原指没有正当生活来源的无赖。这里形容凤姐泼辣,是戏谑的称谓。},南省俗谓作‘辣子’,你只叫他‘凤辣子’就是了。”
\par 黛玉正不知以何称呼,只见众姊妹都忙告诉他道:“这是琏嫂子。”黛玉虽不识,也曾听见母亲说过,大舅贾赦之子贾琏,娶的就是二舅母王氏之内侄女,自幼假充男儿教养的,学名王熙凤。黛玉忙陪笑见礼,以“嫂”呼之。
\par 这熙凤携着黛玉的手,上下细细打谅了一回,仍送至贾母身边坐下,因笑道:“天下真有这样标致的人物,我今儿才算见了!况且这通身的气派,竟不像老祖宗的外孙女儿,竟是个嫡亲的孙女,怨不得老祖宗天天口头心头一时不忘。只可怜我这妹妹这样命苦,怎么姑妈偏就去世了!”说着,便用帕拭泪。贾母笑道:“我才好了,你倒来招我。你妹妹远路才来,身子又弱,也才劝住了,快再休提前话。”这熙凤听了,忙转悲为喜道:“正是呢!我一见了妹妹,一心都在他身上了,又是喜欢,又是伤心,竟忘记了老祖宗。该打,该打!”又忙携黛玉之手,问:“妹妹几岁了?可也上过学?现吃什么药?在这里不要想家,想要什么吃的、什么玩的,只管告诉我;丫头老婆们不好了,也只管告诉我。”一面又问婆子们:“林姑娘的行李东西可搬进来了?带了几个人来?你们赶早打扫两间下房,让他们去歇歇。”
\par 说话时,已摆了茶果上来。熙凤亲为捧茶捧果。又见二舅母问他:“月钱\footnote{月钱——每月按身份等级发给家中上下人等供零用的钱。}放过了不曾?”熙凤道:“月钱已放完了。才刚带着人到后楼上找缎子,找了这半日,也并没有见昨日太太说的那样的,想是太太记错了?”王夫人道:“有没有,什么要紧。”因又说道:“该随手拿出两个来给你这妹妹去裁衣裳的,等晚上想着叫人再去拿罢,可别忘了。”熙凤道:“这倒是我先料着了,知道妹妹不过这两日到的,我已预备下了,等太太回去过了目好送来。”王夫人一笑,点头不语。
\par 当下茶果已撤,贾母命两个老嬷嬷带了黛玉去见两个母舅。时贾赦之妻邢氏忙亦起身,笑回道:“我带了外甥女过去,倒也便宜\footnote{便(biàn)宜——这里是方便之意。}。”贾母笑道\footnote{“笑回道”至“贾母”共十七字,原无。各本均存,唯文字少异。此从己卯、蒙府、戚序、舒序、俄藏、卞藏本补。}:“正是呢,你也去罢,不必过来了。”邢夫人答应了一声“是”字,遂带了黛玉与王夫人作辞。大家送至穿堂前。
\par 出了垂花门,早有众小厮们拉过一辆翠幄青䌷车\footnote{翠幄(wò握)青䌷车——翠幄:指用粗厚的绿色绸类做的轿车车帐。青䌷(䌷即绸字):这里指用青色绸做的车帘。},邢夫人携了黛玉,坐在上面,众婆子们放下车帘,方命小厮们抬起,拉至宽处,方驾上驯骡,亦出了西角门,往东过荣府正门,便入一黑油大门中,至仪门\footnote{仪门——旧时官衙、府第的大门之内的门,取有仪可象之意,又具装饰作用。一说,旁门也可称仪门,系由“謻门”(即官署的旁门)讹转而来。见《在阁新知录》。}前方下来。众小厮退出,方打起车帘,邢夫人搀着黛玉的手,进入院中。黛玉度其房屋院宇,必是荣府中花园隔断过来的。进入三层仪门,果见正房厢庑游廊,悉皆小巧别致,不似方才那边轩峻壮丽;且院中随处之树木山石皆在。一时进入正室,早有许多盛妆丽服之姬妾丫鬟迎着,邢夫人让黛玉坐了,一面命人到外面书房去请贾赦。一时人来回话说:“老爷说了:‘连日身上不好,见了姑娘彼此倒伤心,暂且不忍相见。劝姑娘不要伤心想家,跟着老太太和舅母,即同家里一样。姊妹们虽拙,大家一处伴着,亦可以解些烦闷。或有委屈之处,只管说得,不要外道才是。’”黛玉忙站起来,一一听了。再坐一刻,便告辞。
\par 邢夫人苦留吃过晚饭去,黛玉笑回道:“舅母爱惜赐饭,原不应辞,只是还要过去拜见二舅舅,恐领了赐迟去不恭,异日再领,未为不可。望舅母容谅。”邢夫人听说,笑道:“这倒是了。”遂令两三个嬷嬷用方才的车好生送了姑娘过去。于是黛玉告辞。邢夫人送至仪门前,又嘱咐了众人几句,眼看着车去了方回来。
\par 一时黛玉进了荣府,下了车。众嬷嬷引着,便往东转弯,穿过一个东西的穿堂,向南大厅之后,仪门内大院落,上面五间大正房,两边厢房鹿顶耳房钻山\footnote{两边厢房鹿顶耳房钻山——两边的厢房用钻山的方式与鹿顶的耳房相连接。厢房:指四合院中东西两边的房子。鹿顶:一作盝顶,此语首见于宋《营造法式》,单独用时指平屋顶。耳房:连接在正房两侧的小房子。钻山:指山墙上开门或开洞,与相邻的房子或游廊相接。},四通八达,轩昂壮丽,比贾母处不同。黛玉便知这方是正经正内室,一条大甬路\footnote{甬路——庭院中间的通道,多用砖石铺砌而成。},直接出大门的。进入堂屋中,抬头迎面先看见一个赤金九龙青地大匾,匾上写着斗大的三个大字,是“荣禧堂”,后有一行小字:“某年月日,书赐荣国公贾源”,又有“万几宸翰之宝”\footnote{万几宸(chén辰)翰之宝——这是皇帝印章上的文字。几:同机。“万几”即万事,形容皇帝政务繁多,“日理万几”的意思。宸:北宸,即北极星。皇帝坐北朝南,故以北宸代指皇帝。翰:墨迹、书法。宸翰:皇帝的笔迹。宝:皇帝的印玺。}。大紫檀雕螭案上,设着三尺来高青绿古铜鼎,悬着待漏随朝墨龙大画\footnote{待漏随朝墨龙大画——待漏:封建时代大臣要在五更前到朝房里等待上朝的时刻。漏:指“铜壶滴漏”,古代计时器,代指时间。随朝:按照大臣的班列朝见皇帝。墨龙大画:巨龙在云雾海潮中隐现的大幅水墨画。因旧时以龙象征帝王,又画中之“潮”与朝见之“朝”谐音。隐寓上朝陛见君王之意。贵族家中悬挂此画以示身份地位之荣耀。},一边是金蜼彝\footnote{金蜼彝(wěiyí伟夷)——原为有蜼形图案的青铜祭器,后作贵重陈设品。蜼:一种长尾猿。彝:古代青铜器中礼器的通称。},一边是玻璃\UncommonChar{𥁐}\footnote{\UncommonChar{𥁐}(hǎi海)——盛酒器。}。地下两溜十六张楠木交椅,又有一副对联,乃乌木联牌,镶着錾银\footnote{錾(zàn赞)银——一种银雕工艺。錾:雕刻。}的字迹,道是:
\refdocument{
    \par 座上珠玑昭日月,堂前黼黻焕烟霞。\footnote{“座上”一联——珠玑:珍珠,兼喻诗文之美。黼黻:古代官僚贵族礼服上绣的花纹。黼(fǔ府):半黑半白的斧形图案。黻(fú服):“[插图]”形图案。两句形容座中人和堂上客的衣饰华贵:佩带的珠玉如日月般光彩照人,衣服的图饰如烟霞般绚丽夺目。}
}
\par 下面一行小字,道是:“同乡世教弟\footnote{世教弟——世交,两代以上的交谊;教弟,同辈年龄较大者对较小者的谦称。}勋袭东安郡王穆莳拜手书”。
\par 原来王夫人时常居坐宴息,亦不在这正室,只在这正室东边的三间耳房内。于是老嬷嬷引黛玉进东房门来。临窗大炕上铺着\footnote{“上铺着”,底本、俄藏、卞藏本无,从蒙府、甲辰本补。}猩红洋罽\footnote{罽(jì计)——毛织的毯子。},正面设着大红金钱蟒靠背,石青金钱蟒引枕\footnote{引枕——坐时搭扶胳膊的一种圆墩形的倚枕。},秋香色\footnote{秋香色——淡黄绿色。}金钱蟒大条褥。两边设一对梅花式洋漆小几。左边几上文王鼎匙箸香盒\footnote{文王鼎匙箸香盒——文王鼎:指周代的传国国鼎,此处说的是小型仿古香炉,内烧粉状檀香之类的香料。匙箸:拨弄香灰的用具。香盒:盛香料的盒子。};右边几上汝窑美人觚\footnote{汝窑美人觚(ɡū孤)——宋代河南汝州窑烧制的一种仿古瓷器。觚:古代盛酒器,长身细腰,形如美人,故称。}——觚内插着时鲜花卉,并茗碗痰盒等物。地下面西一溜四张椅上,都搭着银红撒花椅搭\footnote{椅搭——搭在椅上的一种长方形的绣花呢缎饰物。},底下四副脚踏。椅之两边,也有一对高几,几上茗碗瓶花俱备。其馀陈设,自不必细说。
\par 老嬷嬷们让黛玉炕上坐,炕沿上却有两个锦褥对设,黛玉度其位次,便不上炕,只向东边椅子上坐了。本房内的丫鬟忙捧上茶来。黛玉一面吃茶,一面打谅这些丫鬟们,妆饰衣裙,举止行动,果亦与别家不同。茶未吃了,只见一个穿红绫袄青缎掐牙\footnote{掐牙——锦缎双叠成细条,嵌在衣服或背心的夹边上,仅露少许,作为装饰,叫掐牙。}背心的丫鬟走来笑说道:“太太说,请林姑娘到那边坐罢。”老嬷嬷听了,于是又引黛玉出来,到了东廊三间小正房内。
\par 正面炕上横设一张炕桌,桌上磊\footnote{磊——此处读luò(音同洛),叠放。}着书籍茶具,靠东壁面西设着半旧的青缎靠背引枕。王夫人却坐在西边下首,亦是半旧的青缎靠背坐褥。见黛玉来了,便往东让。黛玉心中料定这是贾政之位。因见挨炕一溜三张椅子上,也搭着半旧的弹墨椅袱\footnote{弹墨椅袱——以纸剪镂空图案覆于织品上,用墨色或其他颜色弹或喷成各种图案花样,叫弹墨。椅袱:用棉、缎之类做成的椅套。},黛玉便向椅上坐了。王夫人再四携他上炕,他方挨王夫人坐了。王夫人因说:“你舅舅今日斋戒\footnote{斋戒——古人在祭祀、礼佛或举行隆重大典前,沐浴、吃素、静养一至三日,摒除杂念,以示诚敬,叫斋戒。}去了,再见罢。只是有一句话嘱咐你:你三个姊妹倒都极好,以后一处念书认字学针线,或是偶一顽笑,都有尽让的。但我不放心的最是一件:我有一个孽根祸胎,是家里的‘混世魔王’,今日因庙里还愿去了,尚未回来,晚间你看见便知了。你只以后不要睬他,你这些姊妹都不敢沾惹他的。”
\par 黛玉亦常听得母亲说过,二舅母生的有个表兄,乃衔玉而诞,顽劣异常,极恶读书,最喜在内帏\footnote{内帏——即内室,女子的居处。帏:幕帐。}厮混;外祖母又极溺爱,无人敢管。今见王夫人如此说,便知说的是这表兄了。因陪笑道:“舅母说的,可是衔玉所生的这位哥哥?在家时亦曾听见母亲常说,这位哥哥比我大一岁,小名就唤宝玉,虽极憨顽,说在姊妹情中极好的。况我来了,自然只和姊妹同处,兄弟们自是别院另室的,岂得去沾惹之理?”王夫人笑道:“你不知道原故:他与别人不同,自幼因老太太疼爱,原系同姊妹们一处娇养惯了的。若姊妹们有日不理他,他倒还安静些,纵然他没趣,不过出了二门,背地里拿着他两个小幺儿\footnote{小幺(yāo妖)儿——身边使唤的小仆人。幺:幼小。}出气\footnote{“拿着他两个小幺儿出气”,“小幺儿”原作“小优儿”,又点改为“小子”;“拿着他”旁添为“拿着跟他的”。诸本各异。从原文,并据己卯、甲戌、舒序本改“优”为“幺”。},咕唧一会子就完了。若这一日姊妹们和他多说一句话,他心里一乐,便生出多少事来。所以嘱咐你别睬他。他嘴里一时甜言蜜语,一时有天无日,一时又疯疯傻傻,只休信他。”
\par 黛玉一一的都答应着。只见一个丫鬟来回:“老太太那里传晚饭了。”王夫人忙携黛玉从后房门由后廊往西,出了角门,是一条南北宽夹道。南边是倒座三间小小的抱厦厅\footnote{倒座、抱厦厅——“倒座”是与正房相对、朝向相反的房子。抱厦厅:回绕堂屋后面的侧室。},北边立着一个粉油大影壁\footnote{影壁——俗称照墙。于门内或门外用作屏障或装饰。},后有一半大门,小小一所房室。王夫人笑指向黛玉道:“这是你凤姐姐的屋子,回来你好往这里找他来,少什么东西,你只管和他说就是了。”这院门上也有四五个才总角\footnote{总角——儿童向上分开的两个发髻。代指儿童时代。《诗经·齐风·甫田》疏:“总聚其发,以为两角。”}的小厮,都垂手侍立。王夫人遂携黛玉穿过一个东西穿堂,便是贾母的后院了。
\par 于是,进入后房门,已有多人在此伺候,见王夫人来了,方安设桌椅。贾珠之妻李氏捧饭,熙凤安箸,王夫人进羹。贾母正面榻上独坐,两边四张空椅,熙凤忙拉了黛玉在左边第一张椅上坐了,黛玉十分推让。贾母笑道:“你舅母你嫂子们不在这里吃饭。你是客原应如此坐的。”黛玉方告了座,坐了。贾母命王夫人坐了。迎春姊妹三个告了座方上来。迎春便坐右手第一,探春坐左第二\footnote{“左第二”,原作“坐第二”,从各本改(己卯、梦稿本作“左边第二”)。},惜春坐右第二\footnote{“惜春右第二”,原作“惜春又在右第二”,点改为“惜春就坐右手第二”。己卯、梦稿、俄藏、卞藏本作“惜春右边第二”。馀各本均作“惜春右第二”,从改。}。旁边丫鬟执着拂尘\footnote{拂尘——形如马尾,后有持柄,用以拂拭尘土,或驱赶蝇蚊,俗称“蝇甩子”。古时多用麈兽之尾制成,故又称麈尾。}、漱盂、巾帕。李、凤二人立于案旁布让\footnote{布让——宴席间向客人敬菜、劝餐叫布让。}。外间伺候之媳妇丫鬟虽多,却连一声咳嗽不闻。
\par 寂然饭毕,各有丫鬟用小茶盘捧上茶来。当日林如海教女以惜福养身,云饭后务待饭粒咽尽,过一时再吃茶,方不伤脾胃。今黛玉见了这里许多事情不合家中之式,不得不随的,少不得一一改过来,因而接了茶。早见人又捧过漱盂来,黛玉也照样漱了口。盥手毕,又捧上茶来,这方是吃的茶。贾母便说:“你们去罢,让我们自在说话儿。”王夫人听了,忙起身,又说了两句闲话,方引凤、李二人去了。贾母因问黛玉念何书。黛玉道:“只刚念了《四书》\footnote{《四书》——《大学》、《中庸》、《论语》、《孟子》合称为《四书》,宋代朱熹把它们编在一起,作《四书章句集注》,故有此称。是元、明、清三代科举考试的必读之书。}。”黛玉又问姊妹们读何书。贾母道:“读的是什么书,不过是认得两个字,不是睁眼的瞎子罢了!”
\par 一语未了,只听外面一阵脚步响,丫鬟进来笑道:“宝玉来了!”黛玉心中正疑惑着:“这个宝玉,不知是怎生个惫\UncommonChar{𪬯}\footnote{惫\UncommonChar{𪬯}(bèi lài备赖)——涎皮赖脸的意思。}人物,懵懂顽童?——倒不见那蠢物也罢了。”心中想着,忽见丫鬟话未报完,已进来了一位年轻的公子:
\par 头上戴着束发嵌宝紫金冠,齐眉勒着二龙抢珠金抹额\footnote{嵌宝紫金冠、二龙抢珠金抹额——紫金冠:把头发束扎在顶部的一种髻冠,上面插戴各种饰物或镶嵌珠玉。抹额:围扎在额前,用以压发、束额。二龙抢珠:是抹额上的装饰图案。};穿一件二色金百蝶穿花大红箭袖\footnote{二色金百蝶穿花大红箭袖——用两色金线绣成的百蝶穿花图案的大红窄袖衣服。箭袖:原为便于射箭穿的窄袖衣服,这里指男子穿的一种服式。},束着五彩丝攒花结长穗宫绦\footnote{五彩丝攒花结长穗宫绦(tāo滔)——长穗宫绦:指系在腰间的绦带。长穗:是绦带端部下垂的穗子。五彩丝攒花结:用五彩丝攒聚成花朵的结子,指绦带上的装饰花样。},外罩石青起花八团倭缎排穗褂\footnote{石青起花八团倭缎排穗褂——团:圆形起绒毛的团花。因其凸出,故云“起花”。倭缎:福建漳州、泉州等地仿日本织法制成的缎面起绒花的缎子。排穗:排缀在衣服下面边缘的彩穗。};登着青缎粉底小朝靴\footnote{青缎、朝靴——青缎:近黑的深青色缎子。朝靴:古代百官穿的“乌皮履”。这里指黑色缎面、白色厚底、半高筒的靴子。}。面若中秋之月,色如春晓之花,鬓若刀裁,眉如墨画,面如桃瓣,目若秋波。虽怒时而若笑,即瞋视而有情。项上金螭璎珞,又有一根五色丝绦,系着一块美玉。
\par 黛玉一见,便吃一大惊,心下想道:“好生奇怪\footnote{“好生奇怪”,底本、俄藏、卞藏本作“这生奇怪”,似经两次点改:第一次改为“这么奇怪”,第二次改为“奇怪呀”。从各本改。},倒像在那里见过一般,何等眼熟到如此!”只见这宝玉向贾母请了安\footnote{请安——即问安。清代的请安礼节是,男子打千,即右膝半跪,较隆重时双膝跪下,女子双手扶左膝,右腿微屈,往下蹲身,口称“请某人安”。},贾母便命:“去见你娘来。”宝玉即转身去了。一时回来,再看,已换了冠带:头上周围一转的短发,都结成小辫,红丝结束,共攒至顶中胎发,总编一根大辫,黑亮如漆,从顶至梢,一串四颗大珠,用金八宝坠角\footnote{坠角——用于朝珠、床帐等下端起下垂作用的小装饰品,这里是指辫子梢部所坠的饰物。};身上穿着银红撒花半旧大袄,仍旧带着项圈、宝玉、寄名锁、护身符\footnote{寄名锁、护身符——旧时怕幼儿夭亡,给寺院或道观一定财物,让幼儿当“寄名”弟子,并在幼儿的项下系一小金锁,名“寄名锁”。护身符:是从道观领来的一种符箓,带在身上,避祸免灾。}等物;下面半露松花撒花绫裤腿,锦边弹墨袜,厚底大红鞋。越显得面如敷粉,唇若施脂;转盼多情,语言常笑。天然一段风骚,全在眉梢;平生\footnote{ “平生”,原作“半生”,从各本改。}万种情思,悉堆眼角。看其外貌最是极好,却难知其底细。后人有《西江月》二词,批宝玉极恰,其词曰:
\par 无故寻愁觅恨,有时似傻如狂。纵然生得好皮囊,腹内原来草莽。 潦倒不通世务,愚顽怕读文章。行为偏僻性乖张,那管世人诽谤!
\par 富贵不知乐业,贫穷难耐凄凉。可怜辜负好韶光,于国于家无望。 天下无能第一,古今不肖无双。寄言纨袴与膏粱:莫效此儿形状!\footnote{《西江月》二词——这两首词用似贬实褒、寓褒于贬的手法揭示了贾宝玉的性格。皮囊:一作“皮袋”,指人的躯壳。草莽:丛生的杂草,喻不学无术。文章:此指四书五经及时文八股之类。乐业:这里是满意、安于富贵的意思。纨袴(同裤):代指富家子弟。纨:素色细绢。}
\par 贾母因笑道:“外客未见,就脱了衣裳,还不去见你妹妹!”宝玉早已看见多了一个姊妹,便料定是林姑妈之女,忙来作揖。厮见毕归坐,细看形容,与众各别:
\par 两弯似蹙非罥烟眉\footnote{罥(juàn绢)烟眉——形容眉毛像一抹轻烟。罥:挂的意思。},一双似泣非泣含露目\footnote{“两弯似蹙非罥烟眉,一双似泣非泣含露目”,原作“两湾半蹙鹅眉,一对多情杏眼”。各本均异,且大多残阙。甲辰本作“两湾似蹙非蹙笼烟眉,一双似喜非喜含情目”。“笼烟”己卯本、卞本作“罥烟”(蒙府、戚序本作“罩烟”,梦稿、甲戌本(原文)作“冒烟”,最初似亦为“罥烟”)。下句卞本作“似飘非飘”,与各本皆异。据俄藏本改。}。态生两靥之愁,娇袭一身之病\footnote{态生两靥(yè夜)之愁,娇袭一身之病——意思是妩媚的风韵生于含愁的面容,娇怯的情态出于孱弱的病体。态:情态,风韵。靥:面颊上的酒涡。袭:承继,由……而来。}。泪光点点,娇喘微微。闲静时如姣花照水,行动处似弱柳扶风。心较比干多一窍,病如西子胜三分\footnote{心较比干多一窍,病如西子胜三分——比干:商(殷)代纣王的叔父。《史记·殷本纪》载:纣王淫乱,“比干曰:‘为人臣者,不得不以死争。’廼(乃)强谏纣。纣怒曰:‘吾闻圣人心有七窍。’剖比干,观其心。”古人认为心窍越多越有智慧。上句极言林黛玉聪明颖悟。西子:即西施,相传“西施病心而\UncommonChar{𪾸}(通“颦”,皱眉)”,益增妩媚。见《庄子·天运》。下句是说黛玉病弱娇美胜过西施。}。
\par 宝玉看罢,因笑道:“这个妹妹我曾见过的。”贾母笑道:“可又是胡说,你又何曾见过他?”宝玉笑道:“虽然未曾见过他,然我看着面善,心里就算是旧相识,今日只作远别重逢,亦未为不可。”贾母笑道:“更好,更好,若如此,更相和睦了。”宝玉便走近黛玉身边\footnote{“更好”至“坐下”几句,原仅作“更好好的坐下”,又旁添为“更胡说了,你好好的坐下罢。宝玉坐下”。从俄藏、卞藏本补。}坐下,又细细打量一番,因问:“妹妹可曾读书?”黛玉道:“不曾读,只上了一年学,些须认得几个字。”宝玉又道:“妹妹尊名是那两个字?”黛玉便说了名。宝玉又问表字。黛玉道:“无字。”宝玉笑道:“我送妹妹一妙字,莫若‘颦颦’二字极妙。”探春便问何出。宝玉道:“《古今人物通考》\footnote{《古今人物通考》——未详。从下文来看,可能是宝玉的杜撰。}上说:‘西方有石名黛,可代画眉之墨。’况这林妹妹眉尖若蹙,用取这两个字,岂不两妙!”探春笑道:“只恐又是你的杜撰。”宝玉笑道:“除《四书》外,杜撰的太多,偏只我是杜撰不成?”又问黛玉:“可也有玉没有?”众人不解其语,黛玉便忖度着因他有玉,故问我有也无,因答道:“我没有那个。想来那玉是一件罕物,岂能人人有的。”
\par 宝玉听了,登时发作起痴狂病来,摘下那玉,就狠命摔去,骂道:“什么罕物,连人之高低不择,还说‘通灵’不‘通灵’呢!我也不要这劳什子\footnote{劳什子——如同说“东西”、“玩意”,含有厌恶之意。}了!”吓的众人一拥争去拾玉。贾母急的搂了宝玉道:“孽障!你生气,要打骂人容易,何苦摔那命根子!”宝玉满面泪痕泣道:“家里姐姐妹妹都没有,单我有,我说没趣;如今来了这们一个神仙似的妹妹也没有,可知这不是个好东西。”贾母忙哄他道:“你这妹妹原有这个来的,因你姑妈去世时,舍不得你妹妹,无法处,遂将他的玉带了去了:一则全殉葬\footnote{殉葬——古代把活人或器物随同死者埋在墓中叫“殉葬”。}之礼,尽你妹妹之孝心;二则你姑妈之灵,亦可权作见了女儿之意。因此他只说没有这个,不便自己夸张之意。你如今怎比得他?还不好生慎重带上,仔细你娘知道了。”说着,便向丫鬟手中接来,亲与他带上。宝玉听如此说,想一想大有情理,也就不生别论了。
\par 当下,奶娘来请问黛玉之房舍。贾母说:“今将宝玉挪出来,同我在套间暖阁儿\footnote{套间、暖阁儿——套间:与正房相连的两侧房间。暖阁:是指在套间内再隔断成为小房间,内设炕褥,两边安有隔扇,上边有一横眉,形成床帐的样子,称“暖阁”。}里,把你林姑娘暂安置碧纱橱\footnote{碧纱橱——是清代建筑内檐装修中隔断的一种,亦称隔扇门、格门。清代《装修作则例》中写作“隔扇碧纱橱”。用以隔断开间,中间两扇可以开关。格心多灯笼框式样,灯笼心上常糊以纸,纸上画花或题字;宫殿或富贵人家常在隔心处安装玻璃或糊各色纱,故叫“碧纱橱”,俗称“格扇”。参见刘致平《中国建筑类型及结构》。这里的“碧纱橱里”,是指以碧纱橱隔开的里间。}里。等过了残冬,春天再与他们收拾房屋,另作一番安置罢。”宝玉道:“好祖宗,我就在碧纱橱外的床上很妥当,何必又出来闹的老祖宗不得安静。”贾母想了一想说:“也罢哩。”每人一个奶娘并一个丫头照管,馀者在外间上夜听唤。一面早有熙凤命人送了一顶藕合色花帐,并几件锦被缎褥之类。
\par 黛玉只带了两个人来:一个是自幼奶娘王嬷嬷,一个是十岁的小丫头,亦是自幼随身的,名唤作雪雁。贾母见雪雁甚小,一团孩气,王嬷嬷又极老,料黛玉皆不遂心省力的,便将自己身边的一个二等丫头,名唤鹦哥者与了黛玉。外亦如迎春等例,每人除自幼乳母外,另有四个教引嬷嬷\footnote{教引嬷嬷——清代皇子一落生,即有保母、乳母各八人;断乳后,增“谙达”,“凡饮食、言语、行步、礼节皆教之。”(见《清稗类钞》)世家大族家庭的“教引嬷嬷”,其职务与皇宫的“谙达”近似。},除贴身掌管钗钏盥沐两个丫鬟外,另有五六个洒扫房屋来往使役的小丫鬟。当下,王嬷嬷与鹦哥陪侍黛玉在碧纱橱内。宝玉之乳母李嬷嬷,并大丫鬟名唤袭人者,陪侍在外面大床上。
\par 原来这袭人亦是贾母之婢,本名珍珠。贾母因溺爱宝玉,生恐宝玉之婢无竭力尽忠之人,素喜袭人心地纯良,克尽职任,遂与了宝玉。宝玉因知他本姓花,又曾见旧人诗句上有“花气袭人”之句\footnote{“花气袭人”句——全句为“花气袭人知骤暖”,见宋代陆游诗《村居书喜》。意思是花香扑人,知道天气骤然和暖了。},遂回明贾母,更名袭人。这袭人亦有些痴处:服侍贾母时,心中眼中只有一个贾母;如今服侍宝玉,心中眼中又只有一个宝玉。只因宝玉性情乖僻,每每规谏宝玉不听,心中着实忧郁。
\par 是晚,宝玉李嬷嬷已睡了,他见里面黛玉和鹦哥犹未安息,他自卸了妆,悄悄进来,笑问:“姑娘怎么还不安息?”黛玉忙让:“姐姐请坐。”袭人在床沿上坐了。鹦哥笑道:“林姑娘正在这里伤心,自己淌眼抹泪的说:‘今儿才来,就惹出你家哥儿的狂病,倘或摔坏了那玉,岂不是因我之过!’因此便伤心,我好容易劝好了。”袭人道:“姑娘快休如此,将来只怕比这个更奇怪的笑话儿还有呢!若为他这种行止,你多心伤感,只怕你伤感不了呢。快别多心!”黛玉道:“姐姐们说的,我记着就是了。究竟那玉不知是怎么个来历?上面还有字迹?”袭人道:“连一家子也不知来历,上头还有现成的眼儿,听得说,落草\footnote{ 落草——“妇人分娩曰坐草”。见清吴翌《灯窗丛录》引《魏志》。引申其义,小儿落生叫“落草”。}时是从他口里掏出来的。等我拿来你看便知。”黛玉忙止道:“罢了,此刻夜深,明日再看也不迟。”大家又叙了一回,方才安歇。
\par 次日起来,省\footnote{省(xǐng醒)——家庭日常礼节。子女对父母早上问安叫“省”,晚上服侍就寝叫“定”。见《礼记·曲礼上》。}过贾母,因往王夫人处来,正值王夫人与熙凤在一处拆金陵来的书信看,又有王夫人之兄嫂处遣了两个媳妇来说话的。黛玉虽不知原委,探春等却都晓得是议论金陵城中所居的薛家姨母之子姨表兄薛蟠,倚财仗势,打死人命,现在应天府案下审理。如今母舅王子腾得了信息,故遣他家内的人来告诉这边,意欲唤取进京之意。




































