

\section{红楼梦}


\par 红楼梦:全三册
\par 作者:(清)曹雪芹著;(清)无名氏续;(清)程伟元,(清)高鹗整理;中国艺术研究院红楼梦研究所校注.—3版
\par 出版社:人民文学出版社
\par 版本:1982年3月北京第1版,2008年7月北京第3版;2020年6月第2次印刷
\par 书号:978-7-02-016117-1

\subsection*{《红楼梦》校注本三版序言}

\par 本书初版于一九八二年,至今忽忽已历二十五周年,发行量已逾三百五十万套。一九九四年,当此书面世十二年的时候,我们曾修订过一次,改正了初版中的一些疏漏讹误,也吸收了红学研究上的新成果。现在距离上一次的修订,又已过了十三个年头。红学是一门最具群众性的学问,它拥有的研究队伍和读者,可能远比其他学科的人数要多得多。这十三年的过程,在红学的研究上,自然又有很多的收获,因此,我们决定再次进行修订。
\par 记得一九七五年校订开始之初,我们曾为选用底本,进行过热烈的争论,最后决定采用乾隆二十五年的庚辰本(指底本的年代)为底本,现在看来,当时的这个选择是正确的。广大读者和研究者接受和认可这个本子就是最好的证明。同时,对庚辰本的研究不断深入,而且一九九四年齐鲁书社又出版了同样以庚辰本为底本而又汇集脂评的校订本,到二〇〇六年,作家出版社又出版了一种庚辰本的校订本,这说明庚辰本的真正价值,日益为学术界所认识了。我们作为首次大胆采用庚辰本为底本来校订《红楼梦》的学人,当然是欢迎的。《诗经·小雅·伐木》说:“嘤其鸣矣,求其友声。”这种学术上的求同之心,是大家可以理解的。
\par 我们注意到,新出的以庚辰本为底本的校本,尤其是二〇〇六年的作家本,大量采用了我们的校订成果,这是值得欢迎的。当时我们遵国务院古籍整理组组长李一氓先生之嘱,校记要精,只有重要的改动才作校记。这样做,一方面是为了方便读者的阅读,避免繁琐;另方面,也是为了降低书的定价,有利于读者购买。所以我们大量校改的文字并未出校记。遗憾的是,作家本的校者,并不说明他的校本上的校文,基本上是用了前人的成果,他把这些校文用黑体字排出,还在《校勘说明》里明确说:“补改文字,一律用黑体,使之和原抄文字相区别,便于读者区分与比较。”这段话分明就是告诉读者,这些用黑体字排的文字,全是他新校出来的。而实际上这些用黑体字排的校文,有百分之九十以上是我们早就校出来的。这只要用人民文学出版社出版、中国艺术研究院红楼梦研究所的校订本一对就明白了。
\par 我们的校订本,距今已二十五年了,当时用了七年时间才完成了这项任务。现在有的同志同样采用庚辰本作底本,大量采用我们的校文,这足以说明当时对底本的选择和校订文字的斟酌去取,是经得起时间的考验的,也为后来的校订者起了铺路的作用。
\par 学无止境,学问是与时推移,日新月异的,红学也是一样。所以我们这次的校订,参阅了近十多年来的多种新校本和红学论著,自觉收获较大。这些收获,当然不是个人的,而是反映了红学研究的成果,应该看作是红学界的共同成果。
\par 这次校订,计正文修订共五百馀条,校记修订共一百馀条,注释修订共三百馀条,其中增加条目二百馀条;修改条目一百馀条;凡例修订共三条。
\par 以上是这次修订的总情况。
\par 这次校订,校和注两方面都有相当的进展,这些都已包含在书里,不再一一列举。
\par 这次参加校订工作的人手较少,主要是冯其庸和胡文彬、吕启祥、林冠夫四个人。冯其庸同志负责正文的校订,吕启祥同志负责注释的修订和增补,胡文彬同志正文和注释两方面的工作都参加,并且由他来承担校和注两方面的合成工作,林冠夫同志,考虑到他的身体,主要是请他参加讨论和商量去取。胡文彬同志合成后,最后由冯其庸同志统一审阅和修改定稿。由于第一道工序校和注都做得很认真,所以校注两方面的修改面和难度虽然较大,但质量却比以往有所提高。胡文彬同志的合成工作,负担很重,文字量也大,但做得非常认真细致。尽管恰值酷暑,我们还是尽心尽力尽快地完成了预期的工作任务。
\par 当然,在这项工作启动以前,原校订组的副组长李希凡同志和我们四人,还有人民文学出版社的有关领导曾一起开会商量确定这项工程,之后还分别取得了散处在各地的原校订组成员的同意。这也是促使我们四个人加紧努力的因素。
\par 这里特别要谢谢陈熙中教授,他应我们的邀请,为我们写了几十条修订意见,都十分可贵。还有老友黄能馥老先生,重新为我们审定修改了有关服饰方面的注释,安徽的老友周中明教授,应我们的邀请,花了整整两个月的时间,为我们检核全书,写出了不少关于通假字、同音字厘定的意见和正文校补的意见。由于以上几位同志的帮助,使本书的校订,较前更有所提高。
\par 在整个校订过程中,任晓辉同志协助我们做了许多诸如查阅资料,复印稿件,递送信息等等的工作,使得这项工作得以快速有效的运转。
\par 本书自初版以来,不断收到各地热心的红学朋友的来信来稿,有的是热情鼓励,有的是指出错误,对我们都有很大的帮助。最近,我们又收到河南新安县冯东先生的来信,他为我们细心地查出了错字、注码误差等等问题。还有河北的一位红友萧凤芝同志,他来信告诉我们《红楼梦》第四十七回庚辰本作“十月一”是对的。这是北方为已故亲人送寒衣的民俗节日,不能改作“十月初一”。我们请教了周围的老北京人和北方的朋友,都说至今仍有“十月一,送寒衣”的民俗,所以我们仍依庚辰原本作“十月一”。在此我们敬向以往所有在报刊上发表文章指谬商榷和来信来电的读者朋友表示衷心的感谢!
\par 凡此,都说明,《红楼梦》的研究和校订,既离不开红学研究者,也离不开广大读者。《红楼梦》的修订工作,不会到此结束。我们希望今后能继续走专家和群众结合的路线,实事求是地将这部名著整理得更为完善。
\par \rightline{红楼梦校注组}
\par \rightline{二〇〇七年八月十三日}
\par 本书自二〇〇八年第三版至今,忽忽又将五载。在此期间,承广大读者和学界同仁关切,我们亦时时自省自检,发现仍存在若干疏漏和个别修改失当之处,包括印制过程中的失校。为了对读者负责,今年初,在冯其庸先生建议和主导下,对全书再做修订,具体操作悉委胡文彬、吕启祥二位,仍由胡文彬汇总,计改动正文及标点四十六处、注释十六处。馀者技术性的改动均归责任编辑徐文凯同志。
\par 此次小的改动乃属第三版即同一版次中的修葺。我们深知校书如扫落叶、注释如爬高坡,无有止境。在力所能及的范围内使本书减少讹误、趋向完善,是我们的真诚愿望。
\par 是为记。
\par \rightline{红楼梦校注组}
\par \rightline{二〇一三年四月}


\subsection*{《红楼梦》校注本再版序}
\begin{center}
    \par 冯其庸
\end{center}
\par 本书初版于一九八二年三月,距今转瞬已十二年。本书初版以来,受到广大读者和专家们的欢迎,也得到了不少指正。
\par 这十二年的岁月,使我们进一步认识到,我们当时确定的几个原则是正确的:一是我们所选择的底本——庚辰本,确是一个学术价值很高、接近曹雪芹原稿的珍贵本子,我们以此为底本,就使这个校本有了很好的基础;二是我们确定的校勘原则(详见《校注凡例》)也是正确的,这样就使我们的校勘工作做到了审慎和准确,不至于随意改动底本文字,从而较好地保持了原本的历史面貌;三是我们确定的注释原则(见《校注凡例》)也是切合实际的,对象适中,繁简得宜,因而使得本书避免臃肿烦琐之病。
\par 但是,学问是无止境的,“红学”更是日新月异,这十多年来“红学”研究已有了长足的进展,不少重要的专著相继出版了,不少重要的论文陆续发表了,还连续发现了有关曹雪芹家世的文献和实物。在《红楼梦》的名物考订上,也有不少的进展。对照我们的校本,就感到有了历史的差距。为了对读者负责,对我们的“红学”事业负责,我们深深感到有对本书作一次全面修订的必要。于是趁此书再版之机,我们就着手这一新的繁重工作。
\par 此书初版是由《红楼梦》校订组先作出初稿,然后由校订组和注释组反复修改定稿的,其详细情况见初版的《前言》。现在全面重新整理、重新校注时,不可能把原有的人都邀请回来了,幸好初版定稿时的三位同志:冯其庸、林冠夫、吕启祥都还在原工作岗位上,因此本次的校注,即由他们三位负责:由冯其庸总负其责,林、吕二位分别作校、注的具体修订工作。
\par 关于注释的修改和增补,计新增注释八十七条,补充和修改原注一百六十五条,其馀所有注释,共二千四百零一条,从内容到文字也重新作了一次审核认定;在校勘和标点、分段方面,重校的文字为数亦不在少。特别是此次重校过程中,仍用各脂本仔细复核,所以费力较多;至于标点和分段,则改动更多,不能一一列举。最后,由我作校、注两方面的审核工作,修改核定校、注的条目和文字,以及有关此次重新校注的其他事宜。以上种种,无法一一详述,读者翻阅此再版校注本,当可了然。
\par 尽管此书又作了较为详慎的重校和重注,但初版的创始之功自不可没,故初版的《前言》、《校注凡例》等文字仍置于卷首,以示不忘,也正说明前后校注的原则是一贯的。
\par 这里特别要说明的是,一九七四年至一九七五年间,倡议对《红楼梦》作校注整理,是由袁水拍同志向上级提出的。一九七四年秋天,袁水拍同志到我住处看望我,并提到整理古籍的问题。当时我提及《红楼梦》的校注问题,水拍同志极为重视,不久就要我草拟一个报告。此事后经国务院有关部门正式批准,由水拍同志任校注组的组长,由我和李希凡任副组长。初期校勘的稿子(即内部用的大字本),水拍同志还认真看过,凡是他不理解的地方,他还都提出来认真地询问过。后来,他因事因病,不再过问此事,但在重病中,仍希望能看到此书的出版。所以《红楼梦》的新校注工作得以正式立项并由政府拨款,调集一批专家和研究人员来工作,水拍同志是起了倡导推动作用的。现在水拍同志已经作古十年,当着此书修订再版的时候,我有责任将此经过叙明,亦以慰逝者于地下。
\par 时间虽然只过了十二年,但参加此书工作的同志,却已经作古了好多位,其中顾问有:叶圣陶、吴世昌、吴恩裕、吴组缃先生。参加校注工作的有:沈彭年、陶建基、徐贻庭、朱彤、祝肇年、江辛眉、杨廷福诸先生。国务院古籍整理组组长李一氓先生,一直关心《红楼梦》的研究工作,此书出版后,他还亲自写过评论文章,热情地肯定了这个新校注本,但李一氓先生也已经不幸逝世了。对以上十二位已故的先生和朋友,我们只能寄以哀思,以志永怀!
\par 此次修订再版,虽然由我们三人负责,但前人之功不可没,而“红学”方兴未艾,且无止境,故以后的校注工作,亦无有止境。故我们本次的修订,只是万里长途中的一站,瞻望前途,曷其有极!惟愿再奋馀勇,更求寸进,并望“红学”同人和专家读者进而教之,则《红》书幸甚!“红学”幸甚!
\par 是为序。
\par \rightline{一九九四年七月六日于京华宽堂}


\subsection*{前言}


\par 曹雪芹,是中国文学史上最伟大也是最复杂的作家,《红楼梦》也是中国文学史上最伟大而又最复杂的作品。
\par 关于曹雪芹,目前还存在着不少有争论的问题,不仅他的生卒年一直存在着争议,甚至连他的“字”、“号”也不能十分确定,按照曹雪芹的好友张宜泉的说法,应该是“姓曹名霑,字梦阮,号芹溪居士”,但有的研究者认为他的“字”是“芹圃”,号“雪芹”。
\par 他的生卒年问题,已经争论了几十年。他的生年,现在主要的有两种看法,一种认为他生于公元一七一五年,即康熙五十四年乙未;另一种说法认为他生于公元一七二四年,即雍正二年甲辰。他的卒年,主要有三种看法,一种认为他卒于公元一七六三年,即乾隆二十七年壬午除夕;另一种说法认为他卒于公元一七六四年,即乾隆二十八年癸未除夕;还有一种说法认为他卒于公元一七六四年初春,即乾隆二十九年甲申岁首\footnote{按乾隆二十八年除夕,已经是西历一七六四年二月一日,故乾隆二十九年甲申仍为一七六四年。}。现在大都倾向于第一种看法。
\par 曹雪芹的父亲,现在也有两种看法。一种认为是曹颙,曹雪芹是他的遗腹子;另一种看法,则认为是曹頫。
\par 曹雪芹的上世的籍贯,据近三十年来发现的大量可靠史料,证明他的祖籍是辽阳,后迁沈阳,他的上祖曹振彦原是明代驻守辽东的下级军官,大约于天命六年后金攻下辽阳时归附,以后随清兵入关。\footnote{周汝昌、杨向奎先生认为曹雪芹祖籍是河北丰润,但这是没有任何根据的臆想,是不可信的。详见冯其庸著《曹雪芹家世新考》(上海古籍出版社1980年版)、《再论曹雪芹的家世、祖籍和〈红楼梦〉的著作权》(《红楼梦学刊》1995年第1期)。}
\par 曹振彦归附后金以后,先是属佟养性管辖,后来又归了多尔衮属下的满洲正白旗,当了佐领。旋即跟随清兵入关。曹振彦在入关前的明、金战争中以及入关后的平姜瓖之叛的战争中是立过功的,他历任过山西吉州知州、阳和府知府、浙江盐法道等官职。曹家的发迹,实是从曹振彦开始的。此后,曹振彦之媳,即曹玺之妻孙氏当了康熙的保母。康熙二年,曹玺首任江宁织造之职,专差久任,至二十三年在江宁织造任上病故,康熙旋即命其子曹寅任苏州织造,后又继任江宁织造、两淮巡盐御史等职,并命其纂刻《全唐诗》、《佩文韵府》等书于扬州。曹寅很得康熙的信任和赏识,康熙南巡时曾主持过四次接驾大典。康熙五十一年曹寅在扬州任上病危,康熙特命快马送药抢救,曹寅病故后,又特命其子曹颙继任江宁织造。康熙五十三年曹颙病故,康熙又特命曹寅的胞弟曹荃(宣)之子曹頫过继给曹寅并继任织造之职,直至雍正五年十二月二十四日曹頫被抄家败落,曹家在江南祖孙三代先后共历六十馀年。
\par 《红楼梦》的作者伟大作家曹雪芹就是出生在南京的。直到雍正六年曹家抄没后才全家迁回北京。当时,曹雪芹尚年幼,按生于乙未说是虚岁十四岁,按生于甲辰说是虚岁才五岁。曹家回北京以后的情况,文献绝少记载,曹頫曾经在给康熙的奏折里说到“惟京中住房二所,外城鲜鱼口空房一所;通州典地六百亩,张家湾当铺一所,本银七千两”\footnote{《关于江宁织造曹家档案史料》132页。}等等。在曹家被抄以后,隋赫德的报告里也说到:“曹頫家属,蒙恩谕少留房屋,以资养赡,今其家属不久回京,奴才应将在京房屋人口,酌量拨给。”\footnote{《关于江宁织造曹家档案史料》188页。}据近年发现的雍正六年六月二十一日《曹頫骚扰驿站获罪结案题本》及雍正七年七月《刑部移会》,得知曹頫抄家前,尚有骚扰驿站案,并于雍正六年结案,曹頫被枷号催追赔款。雍正七年七月,曹頫尚在枷号中。又据《刑部移会》得知曹家尚有“京城崇文门外蒜市口地方房十七间半,家仆三对,给与曹寅之妻孀妇度命”。但以后情况究竟如何?究竟拨给了哪些房子?曹雪芹究竟住在何处?他的青年时期是如何度过的?这些问题,统因文献无征,不能确指。据红学家们的考证,认为他与敦诚、敦敏成为亲密朋友,是在右翼宗学里开始结识的,后来落魄住到了西郊,他的不朽的巨著《石头记》就是在西郊的山村里写成的。
\par 曹雪芹晚年的生活穷愁潦倒而又嗜酒狂放,朋友们常把他比作晋朝的阮籍。他甚至穷困到“举家食粥”的地步,常常要靠卖画来换酒喝。他的画很为当时的朋友们所推重。敦敏《题芹圃画石》诗说:“傲骨如君世已奇,嶙峋更见此支离;醉馀奋扫如椽笔,写出胸中磈礧时!”可见曹雪芹的胸襟和画风。可惜他的遗作至今尚未发现。
\par 伟大作家曹雪芹,终于在穷愁困顿中于公元一七六三年即乾隆二十七年壬午除夕去世。他的不朽巨著《石头记》的前八十回,早在他去世前十年左右就已经传抄问世;书的后半部分据专家们研究,认为基本上已经完成,只是由于某种原因未能传抄行世,后来终于迷失,这是不可弥补的损失。
\par 《红楼梦》是一部具有高度思想性和高度艺术性的伟大作品,从本书反映的思想倾向来看,作者具有初步的民主主义思想,他对现实社会包括宫廷及官场的黑暗,封建贵族阶级及其家庭的腐朽,封建的科举制度、婚姻制度、奴婢制度、等级制度,以及与此相适应的社会统治思想即孔孟之道和程朱理学、社会道德观念等等,都进行了深刻的批判并且提出了朦胧的带有初步民主主义性质的理想和主张。这些理想和主张正是当时正在滋长的资本主义经济萌芽因素的曲折反映。
\par 《红楼梦》塑造了众多的人物形象,他们各自具有自己独特而鲜明的个性特征,成为不朽的艺术典型,在中国文学史和世界文学史上永远放射着奇光异彩。
\par 《红楼梦》的情节结构,在以往传统小说的基础上,也有了新的重大的突破。它改变了以往如《水浒传》、《西游记》等一类长篇小说情节和人物单线发展的特点,创造了一个宏大完整而又自然的艺术结构,使众多的人物活动于同一空间和时间,并且使情节的推移也具有整体性,表现出作者卓越的艺术才思。
\par 《红楼梦》的语言艺术成就,更是代表了我国古典小说语言艺术的高峰。作者往往只需用三言两语,就可以勾画出一个活生生的具有鲜明的个性特征的形象;作者笔下每一个典型形象的语言,都具有自己独特的个性,从而使读者仅仅凭借这些语言就可以判别人物。作者的叙述语言,也具有高度的艺术表现力,包括小说里的诗词曲赋,不仅能与小说的叙事融成一体,而且这些诗词的创作也能为塑造典型性格服务,做到了“诗如其人”——切合小说中人物的身份口气。
\par 由于以上各方面的卓越的成就,因而使《红楼梦》无论是在思想内容上或是艺术技巧上都具有自己崭新的面貌,具有永久的艺术魅力,使它足以卓立于世界文学之林而毫无逊色。
\par 现存《红楼梦》的后四十回,是程伟元和高鹗在公元一七九一年即乾隆五十六年辛亥和公元一七九二年即乾隆五十七年壬子先后以木活字排印行世的,其所据底本旧说以为是高鹗的续作,据近年来的研究,高续之说尚有可疑,要之非雪芹原著,而续作者为谁,则尚待探究。续书无论思想或艺术较之原著,已大相悬殊,然与同时或后起的续书相比,则自有其存在之价值,故至今仍能附原著以传。
\par 《红楼梦》在乾隆中叶以后,带脂砚斋评的八十回抄本日多,乾隆末叶即可公开在庙市中抄卖,并且价昂至数十金一部。今传乾隆时期的《石头记》抄本,尚有十一种之多,计有:己卯本、庚辰本、甲戌本、《红楼梦稿》本、蒙古王府本、戚蓼生序本、南京图书馆藏本、梦觉主人序本、舒元炜序本、郑振铎藏本、苏联列宁格勒亚洲图书馆藏本等\footnote{原苏联列宁格勒亚洲图书馆藏本,以往均称列藏本。今因该图书馆已更名为圣彼得堡俄罗斯科学院东方古籍文献研究所,故此书统称俄藏本。特予说明。}。另有南京靖应鹍藏本,今已遗失,又程甲本的前八十回底本,原也是抄本,如果一并计入,则可以说现知的抄本已有十三种之多。当然上面所说的己卯本、庚辰本、甲戌本等名称,其干支年代,都不能代表现有这些本子的抄定年代,都只能表明它们的底本的年代,这一点早已为红学家们指出了。
\par 在以上这些抄本中,己卯、庚辰、甲戌的底本是比较早的。其中己卯本已确知为怡亲王府抄本,其抄成年代约在公元一七六〇年即乾隆二十五年庚辰以后,现存庚辰本抄定的年代,大约是在公元一七六一年即乾隆二十六年以后,甲戌本底本的年代应是公元一七五四年,即乾隆十九年甲戌,但现在所传甲戌本的抄成年代,则是比较晚的。在上述这些抄本中,庚辰本是抄得较早而又比较完整的唯一的一种,它虽然存在着少量的残缺,但却保存了原稿的面貌,未经后人修饰增补(其六十四、六十七两回的残缺,各本皆然,现存各本的这两回或是据程本,或是经后人增补过的),因此本书在校勘过程中决定采用庚辰本为底本,以其他各种脂评抄本为主要参校本,以程本及其他早期刻本为参考本。凡底本文字可通而主要参校本虽有异文但并不见长者,仍依底本;凡底本明显错误而主要参校本不误者,即依主要参校本;凡底本脱漏之文字,有主要参校本可资校补者,即依主要参校本补齐。
\par 本书的注释,凡一应典章制度名物典故以及难解之语词,一般均尽可能作注释,但由于我们的能力有限,而《红楼梦》的注释又极为繁难,因此我们的注不仅可能挂一漏万,而且也可能注释得不尽恰当;我们的校订也同样是如此。有关校订和注释方面的具体情况,均见本书《校注凡例》,这里不再一一详述。
\par 本书校注工作开始于一九七五年,其间参加工作的人员陆续有所更替,工作时间亦长短不一,难以一一表明,现以参加时间先后和姓氏笔划为序,计参加本书校注工作的有:冯其庸、李希凡、刘梦溪、吕启祥、孙逊、沈天佑、沈彭年、应必诚、周雷、林冠夫、胡文彬、曾扬华、顾平旦、陶建基、徐贻庭、朱彤、张锦池、蔡义江、祝肇年、丁维忠。
\par 参加本书最后修改定稿的,校勘方面有:冯其庸、林冠夫、徐贻庭。由冯其庸负责。注释方面有:陶建基、吕启祥、朱彤、张锦池、丁维忠。由陶建基负责。
\par 全书的校注工作由冯其庸同志总负责。
\par 吴世昌、吴恩裕、吴组缃、周汝昌、启功等几位老红学家担任本书校注工作的顾问。
\par 叶圣陶老先生和叶至善同志对本书的校点和注释提了不少宝贵的意见,本书的前半部分,叶圣老还亲自标点、修改过不少地方。为本书的校注提过不少修改意见和撰写过许多条目及注文的还有王雪苔、江辛眉、朱家溍、巫君玉、杨廷福、杨乃济等同志。此外还有不少同志对本书的校注提过宝贵意见或帮助修改过注文,这里限于篇幅,无法一一列出。实际上本书的校注工作是在全国广大群众的热情支持下,是在他们作出的丰硕成果的基础上进行的。
\par 我所的行政工作人员和资料室的同志,也为本书的校注做了不少工作。
\par 南京图书馆对本书的注释和校订曾多次提出书面的修改意见,其他如中国人民大学、北京大学、北京师范大学、北京师范学院、复旦大学、上海师范学院、中山大学、安徽师范大学、哈尔滨师范大学、杭州大学、中央戏剧学院等单位,都给予了热情的支持,我院戏曲研究所和美术研究所、音乐研究所在涉及有关专业方面的问题上,也给予了我们不少指导和帮助。
\par 北京图书馆、北京大学图书馆、中国社会科学院文学研究所图书馆、中国科学院图书馆、南京图书馆、人民文学出版社资料室,为我们提供了许多重要版本和资料。我院的图书馆,则为我们提供了全部的基本参考图书并给予了种种方便和支持。
\par 本书的责任编辑、人民文学出版社古典文学编辑室的王思宇同志对本书的校和注,都提供了许多宝贵的修改意见,付出了不少精力。
\par 本书的校注工作,自始至终,一直是在中国艺术研究院党委和院领导的热情支持下进行的。本院其他行政部门也给我们以多方面的协助,使这项工作得以顺利进行。
\par 对于以上给予本书的校注以大力支持的同志和单位,我们表示衷心的感谢。
\par 本书的校和注,一定还存在着许多缺点,我们衷心期望得到国内外的读者和专家的指正,以便不断修订。
\par \rightline{中国艺术研究院红楼梦研究所}
\par \rightline{一九八一年五月二十日}
\par 一九九四年七月改定关于曹雪芹的祖籍、家世和卒年部分。 冯其庸。 一九九四年七月七日雨窗。


\subsection*{校注凡例}


\par 关于校勘方面:
\par 一、本书以《脂砚斋重评石头记(庚辰〔一七六〇年〕秋月定本)》(简称庚辰本)为底本。底本若干处缺文均依其他脂本或程本补齐,第六十四、六十七回缺文,则采用程甲本补配。
\par 二、以下列各脂评本、抄本及程甲、乙本为参校本:
\par (一)清乾隆甲戌(一七五四年)脂砚斋重评本(简称甲戌本)。
\par (二)乾隆己卯(一七五九年)冬月脂砚斋四阅评本(简称己卯本)。
\par (三)蒙古王府本(简称蒙府本)。
\par (四)戚蓼生序有正书局石印本(简称戚序本)。
\par (五)戚蓼生序南京图书馆藏本(简称戚宁本。此本与戚序本为同一祖本,唯个别处有异。校记中凡未特别列出戚宁异文者,即与戚序本同)。
\par (六)乾隆甲辰(一七八四年)梦觉主人序本(简称甲辰本)。
\par (七)乾隆己酉(一七八九年)舒元炜序本(简称舒序本)。
\par (八)郑振铎藏本(简称郑藏本)。
\par (九)《红楼梦》稿本(简称梦稿本)。
\par (十)圣彼得堡俄罗斯科学院东方古籍文献研究所藏《石头记》(简称俄藏本)。
\par (十一)《卞藏脂本红楼梦》(简称卞藏本)。
\par (十二)乾隆辛亥(一七九一年)程伟元初排活字本(简称程甲本)。
\par (十三)乾隆壬子(一七九二年)程伟元第二次排活字本(简称程乙本)。
\par 三、底本与各参校本之异文,凡属底本明显的衍夺讹舛者,据参校本增删改乙,凡校改底本之处,择要作出校记;凡底本文字可通者,悉仍其旧。
\par 四、其他脂本之回前、回后题诗,一律录入校记,以备查阅。
\par 五、底本旁添旁改文字,大致有以下几类情况:一是原抄时抄漏抄误后随即旁添旁改的,这类文字,可用己卯本校核;二是后来又用其他脂本校对后旁添旁改的,这类文字也可用其他脂本校核,以上两类旁添旁改文字,实际上是正文错抄或抄漏,应与正文同样处理;三是底本明显错误不通,旁改后文句通顺,但却无其他脂本可据,这类改文,以文理之确否而定其去取;四是后人妄改的文字,这类文字一律不取。
\par 六、底本系抄本,所用异体字、俗字甚多,凡属异体字、俗字,一般均以目前通行的规范字予以统一,不再另作校记。但遇有特殊情况,则不作统一,仍按底本文字。
\par 七、底本文字明显错误,各脂本亦沿袭其误,此类情况,即径改之,并作校记说明。
\par 八、后四十回虽系无名氏所续,为适应读者阅读需要,仍接于八十回后,不另标出“附录”字样。但因续作者已无可考,故只署程伟元、高鹗整理刊行者的名字,以与曹雪芹原著区别。
\par 九、后四十回以萃文书屋辛亥排印本(即程甲本)为底本,校以藤花榭本、本衙藏版本、王雪香评本、程乙本等,改字原则一如前八十回。
\par 关于注释方面:
\par 一、本书注释大体上以具有中等文化水平的读者为对象。
\par 二、本书注释条目选取的范围大体上包括书中涉及的典章故实、职官名称、服饰陈设、古代建筑、琴棋书画、释道信仰、医药占卜、方言俗语以及较生僻的字、词等。
\par 三、本书的诗、词、曲、赋、偈语、灯谜、酒令等均加注释。为了便于读者阅读,除注明其中字、词、典故外,必要时对某些句、联以至通首大意亦略加解释。
\par 四、注释内容力求简明,但必要时亦注明出处或径引所据原文,以资查证。
\par 五、相同条目原则上只注一次,但有时该条目在正文的不同回次中含义各有侧重,则重出另注,俾可前后参照,以利阅读。
\par 六、注文中凡涉及学术界尚有争论的问题,或者阙疑,或者介绍其中某一种、两种说法,以供读者参考。
\par 七、本书注释,曾参阅已出之注释或研究、考证文章,为避免繁琐,不再一一标出,非敢掠美。
\par \rightline{中国艺术研究院红楼梦研究所}
\par \rightline{一九八一年五月三十日}
\par \rightline{二〇〇七年八月八日修订}


\subsection*{第一回\ 甄士隐梦幻识通灵\ 贾雨村风尘怀闺秀}


\par 此开卷第一回也。作者自云:因曾历过一番梦幻之后,故将真事隐去,而借“通灵”之说,撰此《石头记》一书也。故曰“甄士隐”云云。但书中所记何事何人?自又云:“今风尘碌碌,一事无成,忽念及当日所有之女子,一一细考较去,觉其行止见识,皆出于我之上。何我堂堂须眉\footnote{须眉——代指男子。},诚不若彼裙钗\footnote{裙钗——代指女子。}哉?实愧则有馀,悔又无益之大无可如何之日也!当此,则自欲将已往所赖天恩祖德,锦衣纨袴\footnote{锦衣纨(wán丸)袴——富贵者的穿着,引申为富家子弟的代称。锦:色彩华美的丝织物。纨:细绢。}之时,饫甘餍肥\footnote{饫(yù玉)甘餍(yàn厌)肥——犹言饱食香甜肥美的食品。饫、餍,吃饱吃腻的意思。}之日,背父兄教育之恩,负师友规训之德,以至今日一技无成、半生潦倒之罪,编述一集,以告天下人:我之罪固不免,然闺阁中本自历历有人,万不可因我之不肖,自护己短,一并使其泯灭也。虽今日之茅椽蓬牖\footnote{茅椽(chuán传)蓬牖(yǒu友)——代指草房陋室,贫者所居。茅、蓬都是野草。椽,房椽子;牖,窗户。},瓦灶绳床\footnote{瓦灶绳床——瓦灶为土坯烧成的简陋的灶,俗称行灶。绳床亦名胡床、交床,为一种简易的坐具。《演繁露》:“今之交床,本自虏来,始名胡床……唐穆宗时又名绳床。”},其晨夕风露,阶柳庭花,亦未有妨我之襟怀笔墨者。虽我未学,下笔无文,又何妨用假语村言,敷演\footnote{敷演——叙述生发。}出一段故事来,亦可使闺阁昭传,复可悦世之目,破人愁闷,不亦宜乎?”故曰“贾雨村”云云。
\par 此回中凡用“梦”用“幻”等字,是提醒阅者眼目,亦是此书立意本旨。\footnote{“此开卷第一回也”以下一大段文字,唯甲戌本在第一回回目之前,作为全书“凡例”的第五条,文字与各本少异,并另有回前诗。底本和其馀各本,都在回目之后,作为正文的开头。陈毓罴最早提出:这是脂批。正文应是从“列位看官”开始。从这段文字的内容和行文的特点看,这个结论是可信的。但考虑到其内容主要是“作者自云”,而在各本中又起着相当于楔子的作用,故仍作特殊处理,放在卷首,并在排字时低二格,以示区别。}
\par 列位看官:你道此书从何而来?说起根由虽近荒唐,细按则深有趣味。待在下将此来历注明,方使阅者了然不惑。
\par 原来女娲氏炼石补天\footnote{女娲(wā洼)氏炼石补天——古代神话传说。女娲氏:传说中的上古“三皇”之一,又称娲皇。《淮南子·览冥训》:“往古之时,四极废,九州裂,天不兼覆,地不周载,……于是女娲炼五色石以补苍天,断鳌足以立四极。”}之时,于大荒山无稽崖\footnote{大荒山无稽崖——大荒山:《山海经·大荒西经》:“大荒之中有山名曰大荒之山。”这里寓“荒唐”。无稽崖和后文“青埂峰”,均属作者虚拟,分别寓“无稽”、“情根”之意。《红楼梦》一书用人名地名谐音寓意,如后文由脂砚斋注明的有:甄士隐(真事隐),贾雨村(假语存),甄英莲(真应怜),霍启(祸起),封肃(风俗),娇杏(侥幸),冯渊(逢冤),元、迎、探、惜(原应叹息)等等,不再一一作注。}炼成高经十二丈、方经二十四丈顽石三万六千五百零一块。娲皇氏只用了三万六千五百块,只单单剩了一块未用,便弃在此山青埂峰下。谁知此石自经煅炼之后,灵性已通,因见众石俱得补天,独自己无材不堪入选,遂自怨自叹,日夜悲号惭愧。
\par 一日,正当嗟悼之际,俄见一僧一道远远而来,生得骨格不凡,丰神迥异,说说笑笑来至峰下,坐于石边高谈快论。先是说些云山雾海神仙玄幻之事,后便说到红尘中荣华富贵。此石听了,不觉打动凡心,也想要到人间去享一享这荣华富贵;但自恨粗蠢,不得已,便口吐人言,向那僧道说道:“大师,弟子蠢物,不能见礼了。适闻二位谈那人世间荣耀繁华,心切慕之。弟子质虽粗蠢,性却稍通;况见二师仙形道体,定非凡品,必有补天济世之材,利物济人之德。如蒙发一点慈心,携带弟子得入红尘,在那富贵场中、温柔乡里受享几年,自当永佩洪恩,万劫不忘也。”二仙师听毕,齐憨笑道:“善哉,善哉!那红尘中有却有些乐事,但不能永远依恃;况又有‘美中不足,好事多魔’八个字紧相连属,瞬息间则又乐极悲生,人非物换,究竟是到头一梦,万境归空,倒不如不去的好。”
\par 这石凡心已炽,那里听得进这话去,乃复苦求再四。二仙知不可强制,乃叹道:“此亦静极思动,无中生有之数也。既如此,我们便携你去受享受享,只是到不得意时,切莫后悔。”石道:“自然,自然。”那僧又道:“若说你性灵,却又如此质蠢,并更无奇贵之处。如此也只好踮脚\footnote{踮脚——犹言“垫脚”。}而已。也罢,我如今大施佛法助你助,待劫终之日,复还本质,以了此案。你道好否?”石头听了,感谢不尽。那僧便念咒书符,大展幻术,将一块大石登时变成\footnote{ “说说笑笑”至“登时变成”共四百二十九字,原作“来至石下,席地而坐长谈,见”十一字,各本同。从甲戌本增。}一块鲜明莹洁的美玉,且又缩成扇坠\footnote{扇坠——悬于扇柄的饰物,多用玉、石等制成。}大小的可佩可拿。那僧托于掌上,笑道:“形体倒也是个宝物了!还只没有实在的好处,须得再镌上数字,使人一见便知是奇物方妙。然后携你到那昌明隆盛之邦,诗礼簪缨之族\footnote{诗礼簪(zān)缨之族——指书香门第,官宦家族。诗礼:读诗书,讲礼仪。簪缨:贵者的冠饰,这里代指作官。簪:一种横插髻上或连接冠与髻的长针。缨:帽带。},花柳繁华地,温柔富贵乡去安身乐业。”石头听了,喜不能禁,乃问:“不知赐了弟子那几件奇处,又不知携了弟子到何地方?望乞明示,使弟子不惑。”那僧笑道:“你且莫问,日后自然明白的。”说着,便袖了这石,同那道人飘然而去,竟不知投奔何方何舍。
\par 后来,又不知过了几世几劫\footnote{劫——佛家用语。梵文音译“劫波”之略,意为“远大时节”。佛教认为,世界有周期性的生灭过程,它经历若干万年后,就要毁灭一次,重新开始,此一周期称为一“劫”。},因有个空空道人访道求仙,忽从这大荒山无稽崖青埂峰下经过,忽见一大块石上字迹分明,编述历历。空空道人乃从头一看,原来就是无材补天,幻形入世,蒙茫茫大士、渺渺真人携入红尘,历尽离合悲欢炎凉世态的一段故事。后面又有一首偈\footnote{偈(jì记)——梵文音译“偈陀”或“伽陀”之略,意译为颂。一般为四句之韵文。}云:
\refdocument{
    \par 无材可去补苍天,枉入红尘若许年。
    \par 此系身前身后事,倩谁\footnote{倩谁——倩:一读qìnɡ音庆,作动词,意为请。又读qiàn音欠,如倩影。倩谁,即请谁。}记去作奇传?
}
\par 诗后便是此石坠落之乡,投胎之处,亲自经历的一段陈迹故事。其中家庭闺阁琐事,以及闲情诗词倒还全备,或可适趣解闷;然朝代年纪,地舆邦国却反失落无考。
\par 空空道人遂向石头说道:“石兄,你这一段故事,据你自己说有些趣味,故编写在此,意欲问世传奇。据我看来,第一件,无朝代年纪可考;第二件,并无大贤大忠理朝廷治风俗的善政,其中只不过几个异样女子,或情或痴,或小才微善,亦无班姑、蔡女之德能\footnote{班姑、蔡女之德能——班姑:即班昭,东汉史学家班固之妹,博学,曾参与续《汉书》。和帝时担任过宫廷教师,号称“大家(ɡū)”,故称“班姑”。编有《女诫》七篇,历来奉为妇德的典范。见《后汉书·曹世叔妻传》。蔡女:指蔡文姬,名琰,东汉文学家蔡邕之女,博学多才,精通音律,是历史上有名的“才女”。见《后汉书·董祀妻传》。}。我纵抄去,恐世人不爱看呢。”石头笑答道:“我师何太痴耶!若云无朝代可考,今我师竟假借汉唐等年纪添缀,又有何难?但我想,历来野史\footnote{野史——一般是指与官修正史相对而言的私家编撰的史类著作。“野史”之名始见于《新唐书·艺文志》,后渐与小说家言的“稗官”连用,称“稗官野史”。这里即指小说。},皆蹈一辙,莫如我这不借此套者,反倒新奇别致,不过只取其事体情理罢了,又何必拘拘于朝代年纪哉!再者,市井俗人喜看理治之书\footnote{理治之书——泛指古代“理朝廷治风俗”的书籍。}者甚少,爱适趣闲文者特多。历来野史,或讪谤君相,或贬人妻女,奸淫凶恶,不可胜数。更有一种风月笔墨\footnote{风月笔墨——原指描写风花雪月、儿女私情的文字。这里专指着意渲染色情的作品。},其淫秽污臭,屠毒笔墨,坏人子弟,又不可胜数\footnote{“更有一种”至“又不可胜数”二十六字,原无。梦稿、甲戌、蒙府、戚序、俄藏、卞藏、甲辰本均存,文字小异。从梦稿、甲戌本补。}。至若佳人才子等书,则又千部共出一套,且其中终不能不涉于淫滥,以致满纸潘安、子建、西子、文君\footnote{潘安、子建、西子、文君——这里代指才子佳人。潘安:即潘安仁,晋代文人,著名美男子。子建:曹植的字,三国时文学家,以才高著称。西子:即西施,春秋时越国美女。文君:汉代卓王孙的女儿,新寡后“私奔”文学家司马相如,结为夫妇。},不过作者要写出自己的那两首情诗艳赋来,故假拟出男女二人名姓,又必旁出一小人其间拨乱,亦如剧中之小丑然。且鬟婢开口即者也之乎,非文即理。故逐一看去,悉皆自相矛盾、大不近情理之话,竟不如我半世亲睹亲闻的这几个女子,虽不敢说强似前代书中所有之人,但事迹原委,亦可以消愁破闷;也有几首歪诗熟话,可以喷饭供酒。至若离合悲欢,兴衰际遇,则又追踪蹑迹,不敢稍加穿凿,徒为供人之目而反失其真传者。今之人,贫者日为衣食所累,富者又怀不足之心,纵一时稍闲,又有贪淫恋色、好货寻愁之事,那里去有工夫看那理治之书?所以我这一段故事,也不愿世人称奇道妙,也不定要世人喜悦检读,只愿他们当那醉淫饱卧\footnote{“醉淫饱卧”,底本、梦稿、俄藏、卞藏本同。蒙府、戚序本作“醉饱淫卧”,甲戌本作“醉馀饱卧”,甲辰本作“醉心饱卧”,舒序本作“醉酒饱卧”。}之时,或避事\footnote{“避事”,梦稿、甲辰、舒序、俄藏、卞藏本同。甲戌、蒙府、戚序本作“避世”。}去愁之际,把此一玩,岂不省了些寿命筋力?就比那谋虚逐妄,却也省了口舌是非之害,腿脚奔忙之苦。再者,亦令世人换新眼目,不比那些胡牵乱扯忽离忽遇,满纸才人淑女、子建文君红娘小玉\footnote{ 红娘、小玉——红娘:唐代元稹《会真记》(至元代王实甫衍为杂剧《西厢记》)中崔莺莺的丫鬟。小玉:唐代蒋防《霍小玉传》中的女主人公。}等通共熟套之旧稿。我师意为何如?”
\par 空空道人听如此说,思忖半晌,将《石头记》再检阅一遍,因见上面虽有些指奸责佞贬恶诛邪之语,亦非伤时骂世之旨;及至君仁臣良父慈子孝,凡伦常\footnote{伦常——即封建伦理道德。伦:人伦,封建社会指人与人之间关系及行为的准则。封建社会以君臣、父子、夫妇、兄弟、朋友为五伦,认为是不可改变的常道,亦称五常。}所关之处,皆是称功颂德,眷眷无穷,实非别书之可比。虽其中大旨谈情,亦不过实录其事,又非假拟妄称,一味淫邀艳约、私订偷盟之可比。因毫不干涉时世,方从头至尾抄录回来,问世传奇。从此空空道人\footnote{“从此空空道人”,原无,各脂本均同。从程甲本补。}因空\footnote{空——“空”与下文的“色”、“情”,均佛教用语。佛教认为“空”乃天地万物的本体,一切终属空虚。“色”乃万物本体(空)的瞬息生灭的假象;“情”乃对此等假象(色)所产生的种种感情,如爱、憎等等。这里是借用,已注入了作家的人生体验。}见色,由色生情,传情入色,自色悟空,遂易名为情僧,改《石头记》为《情僧录》。东鲁孔梅溪则题曰《风月宝鉴》\footnote{《风月宝鉴》——甲戌本眉批云:“雪芹旧有《风月宝鉴》之书,乃其弟棠村序也。”甲戌本“凡例”云:《红楼梦》“又曰《风月宝鉴》,是戒妄动风月之情”。风月:指男女之情。宝鉴:宝镜。}。后因曹雪芹于悼红轩中披阅十载,增删五次,纂成目录,分出章回,则题曰《金陵十二钗》\footnote{金陵十二钗——金陵,古邑名,楚威王七年(公元前333年)置,在今南京市。后即为南京市的别称。钗:本为妇女的头饰。旧称女子为“裙钗”或“金钗”。十二钗,语本《古乐府》:“头上金钗十二行”,原言髻高插钗之多。又作十二女子代称。此书又“题曰《金陵十二钗》”,通常认为是由第五回“册子”上所写的十二个女子得名。}。并题一绝云:
\refdocument{
    \par 满纸荒唐言,一把辛酸泪。
    \par 都云作者痴,谁解其中味!
}
\par 出则既明,且看石上是何故事。按那石上书云:
\par 当日地陷东南\footnote{地陷东南——东南大地塌陷下沉。古代神话:共工与颛顼(zhuān xū专须)争帝,怒而触不周山,折天柱,绝地维,天倾西北,地不满东南。见《淮南子·天文训》。},这东南一隅有处曰姑苏,有城曰阊门\footnote{姑苏、阊(chānɡ昌)门——姑苏:苏州的别称,因其西南有姑苏山而得名。这里是指旧苏州府辖境。阊门:苏州城的西北门,又名破楚门。这里代指苏州城。}者,最是红尘中一二等富贵风流之地。这阊门外有个十里街,街内有个仁清巷\footnote{十里街、仁清巷——据脂批,谐音“势利街”、“人情巷”。},巷内有个古庙,因地方窄狭,人皆呼作葫芦庙。庙旁住着一家乡宦,姓甄,名费,字士隐。嫡妻封氏,情性贤淑,深明礼义。家中虽不甚富贵,然本地便也推他为望族了。因这甄士隐禀性恬淡,不以功名为念,每日只以观花修竹、酌酒吟诗为乐,倒是神仙一流人品。只是一件不足:如今年已半百,膝下无儿,只有一女,乳名唤作英莲\footnote{“英莲”,原作“英菊”,己卯本同。从梦稿、甲戌、蒙府、戚序、甲辰、舒序、俄藏、卞藏本改。下此名重出时,各本情况大体相同,不再作校记。},年方三岁。
\par 一日,炎夏永昼,士隐于书房闲坐,至手倦抛书\footnote{手倦抛书——见北宋人蔡确《夏日登车盖亭》诗(收入《千家诗》)其前二句:“纸屏石枕竹方床,手倦抛书午梦长。”},伏几少憩,不觉朦胧睡去。梦至一处,不辨是何地方。忽见那厢来了一僧一道,且行且谈。
\par 只听道人问道:“你携了这蠢物,意欲何往?”那僧笑道:“你放心,如今现有一段风流公案正该了结,这一干风流冤家\footnote{风流冤家——“冤家”,原为佛教用语。《五灯会元》:“佛教慈悲,冤亲平等。”后既作“仇人”、“对头”解,也用作对所爱之人的昵称,即爱极的反语。“风流冤家”指极相爱恋之男女。},尚未投胎入世。趁此机会,就将此蠢物夹带于中,使他去经历经历。”那道人道:“原来近日风流冤孽又将造劫历世去不成?但不知落于何方何处?”那僧笑道:“此事说来好笑,竟是千古未闻的罕事。只因西方灵河岸上三生石\footnote{西方灵河岸上三生石——西方灵河岸上:作者假想的神仙境界。西方:原指佛教的发源地天竺(古代印度)。灵河:原指恒河,今印度人犹称之为“圣水”。三生:指前生、今生和来生,这是佛教转世投胎的说法。三生石:传说唐代李源与和尚圆观交情很好,后有“三生石上旧精魂”、“此身虽异性常存”之句。见唐代袁郊《甘泽谣·圆观》。后以“三生石”喻因缘前定。}畔,有绛珠草一株,时有赤瑕宫神瑛侍者,日以甘露灌溉,这绛珠草始得久延岁月。后来既受天地精华,复得雨露滋养,遂得脱却草胎木质,得换人形,仅修成个女体,终日游于离恨天外,饥则食蜜青果为膳,渴则饮灌愁海\footnote{离恨天、蜜青果、灌愁海——离恨天:俗传“三十三天,离恨天最高;四百四十病,相思病最苦”。蜜青谐“秘情”。灌愁海:喻愁深。皆寓男女之情及其怨恨愁苦。}水为汤。只因尚未酬报灌溉之德,故其五内\footnote{五内——五脏,即心、肝、脾、肺、肾。亦泛言内心深处。}便郁结着一段缠绵不尽之意。恰近日这神瑛侍者凡心偶炽,乘此昌明太平朝世,意欲下凡造历幻缘,已在警幻仙子案前挂了号。警幻亦曾问及,灌溉之情未偿,趁此倒可了结的。那绛珠仙子道:‘他是甘露之惠,我并无此水可还。他既下世为人,我也去下世为人,但把我一生所有的眼泪还他,也偿还得过他了。’因此一事,就勾出多少风流冤家来,陪他们去了结此案。”
\par 那道人道:“果是罕闻。实未闻有还泪之说。想来这一段故事,比历来风月事故更加琐碎细腻了。”那僧道:“历来几个风流人物,不过传其大概以及诗词篇章而已;至家庭闺阁中一饮一食,总未述记。再者,大半风月故事,不过偷香窃玉、暗约私奔而已,并不曾将儿女之真情发泄一二。想这一干人入世,其情痴色鬼、贤愚不肖\footnote{不肖——旧时称不能继承父业之子曰不肖。肖:像。}者,悉与前人传述不同矣。”那道人道:“趁此何不你我也去下世度脱\footnote{度脱——佛家用语。超度解脱。}几个,岂不是一场功德?”那僧道:“正合吾意。你且同我到警幻仙子宫中,将蠢物交割清楚,待这一干风流孽鬼下世已完,你我再去。如今虽已有一半落尘,然犹未全集。”道人道:“既如此,便随你去来。”
\par 却说甄士隐俱听得明白,但不知所云“蠢物”系何东西。遂不禁上前施礼,笑问道:“二仙师请了。”那僧道也忙答礼相问。士隐因说道:“适闻仙师所谈因果,实人世罕闻者。但弟子愚浊,不能洞悉明白,若蒙大开痴顽,备细一闻,弟子则洗耳谛听,稍能警省,亦可免沉沦\footnote{警省(xǐnɡ醒)、沉沦——均佛家用语。警省:警觉省悟。沉沦:指在生死轮回中永远不得解脱。}之苦。”二仙笑道:“此乃玄机\footnote{玄机——道家用语。谓玄奥微妙的道理。这里义同天机。}不可预泄者。到那时不要忘我二人,便可跳出火坑\footnote{火坑——佛家用语,指苦难的人世。}矣。”士隐听了,不便再问。因笑道:“玄机不可预泄,但适云‘蠢物’,不知为何,或可一见否?”那僧道:“若问此物,倒有一面之缘。”说着,取出递与士隐。
\par 士隐接了看时,原来是块鲜明美玉,上面字迹分明,镌着“通灵宝玉”四字,后面还有几行小字。正欲细看时,那僧便说已到幻境,便强从手中夺了去,与道人竟过一大石牌坊,上书四个大字,乃是“太虚幻境”\footnote{太虚幻境——作者虚拟的仙境。太虚:空幻虚无的意思。}。两边又有一副对联,道是:
\refdocument{
    \par 假作真时真亦假,无为有处有还无\footnote{对联,原作“假作真时真作假,无为有处有为无”。己卯、梦稿本上联同底本。舒本略为特殊,作“色色空空地,真真假假天”。馀各本均作“假作真时真亦假,无为有处有还无”。第五回此联重出时,底本及其他各本,均作“真亦假”、“有还无”。从改。}。
}
\par 士隐意欲也跟了过去,方举步时,忽听一声霹雳,有若山崩地陷。士隐大叫一声,定睛一看,只见烈日炎炎,芭蕉冉冉,所梦之事便忘了大半。又见奶母正抱了英莲走来。士隐见女儿越发生得粉妆玉琢,乖觉可喜,便伸手接来,抱在怀内,逗他顽耍一回,又带至街前,看那过会\footnote{过会——旧时遇节庆,随地聚演百戏杂耍、笙乐鼓吹之类,观者如潮。}的热闹。
\par 方欲进来时,只见从那边来了一僧一道:那僧则癞头跣脚,那道则跛足蓬头,疯疯癫癫,挥霍\footnote{挥霍——亦作“挥攉”。《韵会》:“摇手曰挥,反手曰攉。”本谓动作轻捷,这里是挥洒自如的意思。}谈笑而至。及至到了他门前,看见士隐抱着英莲,那僧便大哭起来,又向士隐道:“施主,你把这有命无运\footnote{有命无运——旧时“算命”,用人出生的年、月、日、时所属的干支和金、木、水、火、土五行的生克来推断人的吉凶祸福;称一生的境遇好坏为“命”,一段时间的遭际为“运”。有命无运,这里意谓平生“行运”乖逆,遭际堪悲。}、累及爹娘之物,抱在怀内作甚?”士隐听了,知是疯话,也不去睬他。那僧还说:“舍我罢,舍我罢!”士隐不耐烦,便抱女儿撤身要进去,那僧乃指着他大笑,口内念了四句言词道:
\refdocument{
    \par 惯养娇生笑你痴,菱花空对雪澌澌\footnote{“菱花”句——隐喻英莲被呆霸王薛蟠强占作妾的不幸遭遇。菱花:指后来英莲改名香菱。雪:谐音“薛”,指薛蟠。菱在夏日开花而竟遇冰雪,喻英莲“生不逢时,遇又非偶”(脂评),定然遭到摧残。澌(sī司)澌:形容雪声。}。
    \par 好防佳节元宵后,便是烟消火灭时。
}
\par 士隐听得明白,心下犹豫,意欲问他们来历。只听道人说道:“你我不必同行,就此分手,各干营生去罢。三劫后,我在北邙山\footnote{北邙(mánɡ芒)山——也作“北芒山”,即邙山。在今河南省洛阳市北。东汉及北魏的王侯公卿多葬于此。后常被用来泛指墓地。}等你,会齐了同往太虚幻境销号。”那僧道:“最妙,最妙!”说毕,二人一去,再不见个踪影了。士隐心中此时自忖:这两个人必有来历,该试一问,如今悔却晚也。
\par 这士隐正痴想,忽见隔壁葫芦庙内寄居的一个穷儒——姓贾名化、字表时飞、别号雨村者走了出来。这贾雨村原系湖州\footnote{“湖州”,底本、甲戌、己卯本第二回、卞藏本作“胡州”,梦稿本第二回作“湖北”。从己卯(第一回)、蒙府、戚序、甲辰、舒序、俄藏本改。}\footnote{湖州——地名。今浙江省湖州市。脂评:谐音“胡诌也”。}人氏,也是诗书仕宦之族,因他生于末世,父母祖宗根基已尽,人口衰丧,只剩得他一身一口,在家乡无益,因进京求取功名,再整基业。自前岁来此,又淹蹇\footnote{淹蹇(yān jiǎn烟简)——即偃蹇。原指境遇困顿、不得意,这里是耽搁、阻滞的意思。}住了,暂寄庙中安身,每日卖字作文为生,故士隐常与他交接。
\par 当下雨村见了士隐,忙施礼陪笑道:“老先生倚门伫望,敢\footnote{敢——意谓莫非、恐怕、或许,这里作“莫非”解。}街市上有甚新闻否?”士隐笑道:“非也。适因小女啼哭,引他出来作耍,正是无聊之甚,兄来得正妙,请入小斋一谈,彼此皆可消此永昼。”说着,便令人送女儿进去,自与雨村携手来至书房中。小童献茶。方谈得三五句话,忽家人飞报:“严老爷来拜。”士隐慌的忙起身谢罪道:“恕诳驾\footnote{诳(kuánɡ狂)驾——邀来客人后,因故不能陪待,向客人道歉之词,犹言“失陪”。诳或作“诓”,欺骗的意思。驾:对客人的尊称。}之罪,略坐,弟即来陪。”雨村忙起身亦让道:“老先生请便。晚生乃常造之客,稍候何妨。”说着,士隐已出前厅去了。
\par 这里雨村且翻弄书籍解闷。忽听得窗外有女子嗽声,雨村遂起身往窗外一看,原来是一个丫鬟,在那里撷\footnote{撷(xié协)——采摘、捋取。唐代王维《相思》:“愿君多采撷,此物(红豆)最相思。”}花,生得仪容不俗,眉目清明,虽无十分姿色,却亦有动人之处。雨村不觉看的呆了。
\par 那甄家丫鬟撷了花,方欲走时,猛抬头见窗内有人,敝巾旧服,虽是贫窘,然生得腰圆背厚,面阔口方,更兼剑眉星眼,直鼻权腮\footnote{权腮——俗称颧骨腮,指人颧骨长得很高,相法认为是一种贵相。沈括《梦溪笔谈·人事》:“公满面权骨,不十年必总枢柄。”}。这丫鬟忙转身回避,心下乃想:“这人生的这样雄壮,却又这样褴褛,想他定是我家主人常说的什么贾雨村了,每有意帮助周济,只是没甚机会。我家并无这样贫窘亲友,想定是此人无疑了。怪道又说他必非久困之人。”如此想来,不免又回头两次。
\par 雨村见他回了头,便自为这女子心中有意于他,便狂喜不尽,自为此女子必是个巨眼英雄\footnote{巨眼英雄——有远见,能识鉴人才的人。},风尘\footnote{风尘——这里指扰攘的尘世,又有旅居在外,备尝艰辛之意。}中之知己也。一时小童进来,雨村打听得前面留饭,不可久待,遂从夹道中自便出门去了。士隐待客既散,知雨村自便,也不去再邀。
\par 一日,早又中秋佳节。士隐家宴已毕,乃又另具一席于书房,却自己步月至庙中来邀雨村。原来雨村自那日见了甄家之婢曾回顾他两次,自为是个知己,便时刻放在心上。今又正值中秋,不免对月有怀,因而口占五言一律\footnote{口占五言一律——口占:随口吟成,与下文“口号”义同。五言一律:每句五个字的律诗一首。}云:
\refdocument{
    \par 未卜三生愿\footnote{ “未卜”句——未卜:不能预知。全句意为同娇杏结姻缘的愿望不知能否实现。},频添一段愁\footnote{“频添”句——意即把这段愁绪时刻挂在心上。频:屡屡;时时。}。
    \par 闷来时敛额\footnote{敛额——皱眉头。},行去几回头。
    \par 自顾风前影,谁堪月下俦?\footnote{“自顾”二句——意谓风前自顾身影,有谁能赏识自己,成为我的终身伴侣呢?自顾风前影:由“顾影自怜”化出。堪:能,配得上。月下俦(chóu愁):成婚配的意思。传说唐代韦固在宋城遇一老人在月下检天下婚姻之书,囊中并有赤绳,一系男女之足,则必成夫妇。见李复言《续玄怪录》。后因称管婚姻之神为“月下老人”或“月老”,也用来代称媒人。俦:伴侣。}
    \par 蟾光如有意,先上玉人楼。\footnote{“蟾光”二句——蟾光:指月光。原意是:月光如真有情意,希望先照玉人的妆楼。暗含若得科举及第,定先到玉人楼上求婚之意。“蟾光”句,亦寓“蟾宫折桂”(即科举及第)之意。玉人楼:美人居住的地方,玉人,指娇杏。}
}
\par 雨村吟罢,因又思及平生抱负,苦未逢时,乃又搔首对天长叹,复高吟一联曰:
\refdocument{
    \par 玉在\UncommonCharB{\symbol{30165}}中求善价,钗于奁内待时飞。\footnote{“玉在”一联——这里贾雨村自比玉、钗,企图得到赏识,以求飞黄腾达。上句意谓美玉藏在匣子里希望卖得好价钱。\UncommonCharB{\symbol{30165}}(dú读):即“椟”。木匣;木柜。下句意谓玉钗放在镜盒中,等待时机而飞腾。传说汉武帝元鼎元年,有神女留一玉钗,昭帝时,有人偷开匣子,不见玉钗,只见一只白燕从中飞出,升天而去。见郭宪《洞冥记》。}
}
\par 恰值士隐走来听见,笑道:“雨村兄真抱负不浅也!”雨村忙笑道:“不过偶吟前人之句,何敢狂诞至此。”因问:“老先生何兴至此?”士隐笑道:“今夜中秋,俗谓‘团圆之节’,想尊兄旅寄僧房,不无寂寥之感,故特具小酌,邀兄到敝斋一饮,不知可纳芹意\footnote{芹意——古时有人认为芹菜的味道很美,就向乡豪称赞,乡豪尝后,却觉得很难吃。见《列子·杨朱篇》。后人常用“献芹”、“芹意”等作为送礼或请客的谦词。}否?”雨村听了,并不推辞,便笑道:“既蒙厚爱,何敢拂此盛情。”说着,便同士隐复过这边书院中来。
\par 须臾茶毕,早已设下杯盘,那美酒佳肴自不必说。二人归坐,先是款斟漫饮,次渐谈至兴浓,不觉飞觥限斝\footnote{飞觥(ɡōnɡ工)限斝(jiǎ甲)——觥筹交错、饮宴尽欢的情景。觥、斝:两种古代酒器,前者为角形,后者圆口平底。飞觥:挥杯;限斝:行酒令时限定饮酒数量。}起来。当时街坊上家家箫管,户户弦歌,当头一轮明月,飞彩凝辉,二人愈添豪兴,酒到杯干。雨村此时已有七八分酒意,狂兴不禁,乃对月寓怀,口号\footnote{口号——犹言“口占”,不借笔墨、随口吟成。《宋史·乐志》:“乐工致辞,继以诗一章,谓之口号。”}一绝云:
\refdocument{
    \par 时逢三五\footnote{三五——十五,指阴历十五日。}便团圆,满把晴光护玉栏\footnote{ “满把”句——满把:满握。满把晴光:极言月光皎洁充盈。护玉栏:玉石栏杆沉浸在皎洁的月光里。}。
    \par 天上一轮才捧出,人间万姓仰头看。\footnote{“天上”二句——据说赵匡胤未登极时,曾拿《咏月》诗给徐铉看,徐铉读到“未离海底千山黑,才到中天万国明”这两句时,认为帝王之兆已显。见宋代陈师道《后山诗话》。贾诗后两句所抒胸臆类此,故甄士隐说他“飞腾之兆已见”。}
}
\par 士隐听了,大叫:“妙哉!吾每谓兄必非久居人下者,今所吟之句,飞腾之兆已见,不日可接履于云霓之上\footnote{接履于云霓之上——犹言平步青云。接履:一步紧接一步。云霓:喻高位。}矣。可贺,可贺!”乃亲斟一斗为贺。雨村因干过,叹道:“非晚生酒后狂言,若论时尚之学\footnote{时尚之学——时人所崇尚的学问。这里指明清科举考试用的“八股文”和“试帖诗”等。},晚生也或可去充数沽名,只是目今行囊路费一概无措,神京\footnote{神京——与下文“神都”,均指京城。}路远,非赖卖字撰文即能到者。”士隐不待说完,便道:“兄何不早言。愚每有此心,但每遇兄时,兄并未谈及,愚故未敢唐突。今既及此,愚虽不才,‘义利’二字\footnote{“义利”二字——《论语·里仁》:“君子喻于义,小人喻于利。”义:道义。利:功利,这里指钱财。}却还识得。且喜明岁正当大比,兄宜作速入都,春闱\footnote{大比、春闱(wéi围)——明清科举制,考试分为三级。第一级是院试,考府县的童生,考取的为“生员”(秀才);第二级是乡试,考一省的生员,考取的为“举人”;第三级是会试,考全国的举人,考取的为“贡士”(再经殿试赐进士出身)。乡试、会试均三年一科,也称“大比”。乡试在秋天,称为“秋闱”;会试在春天,称为“春闱”。闱:指考场。这里的“大比”是指会试。}一战,方不负兄之所学也。其盘费馀事,弟自代为处置,亦不枉兄之谬识矣!”当下即命小童进去,速封五十两白银,并两套冬衣。又云:“十九日乃黄道之期,兄可即买舟西上,待雄飞高举,明冬再晤,岂非大快之事耶!”雨村收了银衣,不过略谢一语,并不介意,仍是吃酒谈笑。那天已交了三更,二人方散。
\par 士隐送雨村去后,回房一觉,直至红日三竿方醒。因思昨夜之事,意欲再写两封荐书与雨村带至神都,使雨村投谒个仕宦之家为寄足之地。因使人过去请时,那家人去了回来说:“和尚说,贾爷今日五鼓已进京去了,也曾留下话与和尚转达老爷,说‘读书人不在黄道黑道\footnote{黄道黑道——为我国古代天文学的专名,黄道指日,黑道指月。《汉书·天文志》:“日有中道”,“中道者黄道,一曰光道。”又云:“月有九行者,黑道二。”后星占者将每日的干支阴阳分为“黄道”和“黑道”,黄道主吉,黑道主凶。},总以事理为要,不及面辞了。’”士隐听了,也只得罢了。
\par 真是闲处光阴易过,倏忽又是元宵佳节矣。士隐命家人霍启抱了英莲去看社火花灯\footnote{社火花灯——这里指元宵节灯火。社:社日。祭祀土神之日,分春秋两祭,立春后第五个戊日为春社,立秋后第五个戊日为秋社。社火:社日扮演的各种杂戏。花灯:正月十五元宵节有放花灯的习俗。},半夜中,霍启因要小解,便将英莲放在一家门槛上坐着。待他小解完了来抱时,那有英莲的踪影?急得霍启直寻了半夜,至天明不见,那霍启也就不敢回来见主人,便逃往他乡去了。那士隐夫妇,见女儿一夜不归,便知有些不妥,再使几人去寻找,回来皆云连音响皆无。夫妻二人,半世只生此女,一旦失落,岂不思想,因此昼夜啼哭,几乎不曾寻死。看看的一月,士隐先就得了一病;当时封氏孺人\footnote{孺人——《礼记·曲礼下》:“天子之妃曰后,诸侯曰夫人,大夫曰孺人,士曰妇人,庶人曰妻。”孺人在明清为七品官之母或妻的封号。后通用为妇人的尊称。}也因思女构疾,日日请医疗治。
\par 不想这日三月十五,葫芦庙中炸供\footnote{炸供——油炸供神用的食品。},那些和尚不加小心,致使油锅火逸,便烧着窗纸。此方人家多用竹篱木壁者,大抵也因劫数,于是接二连三,牵五挂四,将一条街烧得如火焰山一般。彼时虽有军民来救,那火已成了势,如何救得下?直烧了一夜,方渐渐的熄去,也不知烧了几家。只可怜甄家在隔壁,早已烧成一片瓦砾场了。只有他夫妇并几个家人的性命不曾伤了。急得士隐惟跌足长叹而已。只得与妻子商议,且到田庄上去安身。偏值近年水旱不收,鼠盗蜂起,无非抢田夺地,鼠窃狗偷,民不安生,因此官兵剿捕,难以安身。士隐只得将田庄都折变了,便携了妻子与两个丫鬟投他岳丈家去。
\par 他岳丈名唤封肃,本贯大如州人氏,虽是务农,家中都还殷实。今见女婿这等狼狈而来,心中便有些不乐。幸而士隐还有折变田地的银子未曾用完,拿出来托他随分就价薄置些须房地,为后日衣食之计。那封肃便半哄半赚,些须与他些薄田朽屋。士隐乃读书之人,不惯生理稼穑等事,勉强支持了一二年,越觉穷了下去。封肃每见面时,便说些现成话,且人前人后又怨他们不善过活,只一味好吃懒作等语。士隐知投人不着,心中未免悔恨,再兼上年惊唬,急忿怨痛,已有积伤,暮年之人,贫病交攻,竟渐渐的露出那下世的光景来\footnote{“下世”句——下世:此指死亡。全句是指快要死亡、不久于世的意思。}。
\par 可巧这日拄了拐杖挣挫到街前散散心时,忽见那边来了一个跛足道人,疯癫落脱\footnote{落脱——即“落拓”、“落托”。这里是行为狂放的意思。},麻屣鹑衣\footnote{麻屣(xǐ洗)鹑(chún纯)衣——麻屣:麻鞋。鹑:鹌鹑,鸟名。其尾短秃,如补绽百结,故称破烂衣服为鹑衣。},口内念着几句言词,道是:
\refdocument{
    \par 世人都晓神仙好,惟有功名忘不了!
    \par 古今将相在何方?荒冢一堆草没了。
    \par 世人都晓神仙好,只有金银忘不了!
    \par 终朝只恨聚无多,及到多时眼闭了。
    \par 世人都晓神仙好,只有姣妻忘不了!
    \par 君生日日说恩情,君死又随人去了。
    \par 世人都晓神仙好,只有儿孙忘不了!
    \par 痴心父母古来多,孝顺儿孙谁见了?
}
\par 士隐听了,便迎上来道:“你满口说些什么?只听见些‘好’‘了’‘好’‘了’。”那道人笑道:“你若果听见‘好’‘了’二字,还算你明白。可知世上万般,好便是了,了便是好。若不了,便不好;若要好,须是了。我这歌儿,便名《好了歌》。”士隐本是有宿慧\footnote{宿慧——佛家用语。指超越常人的智慧,认为这种智慧是宿世(即前世)带来的。}的,一闻此言,心中早已彻悟\footnote{彻悟——即佛教所说的大彻大悟,看破红尘。}。因笑道:“且住!待我将你这《好了歌》解注出来何如?”道人笑道:“你解,你解。”士隐乃说道:
\par 陋室空堂,当年笏满床\footnote{笏(hù户)满床——形容家中做大官的人很多。笏:一名“手板”。封建时代臣僚上朝时手中所拿的狭长板子,用象牙或木、竹片制成,可作临时记事之用。};衰草枯杨,曾为歌舞场。蛛丝儿结满雕梁,绿纱今又糊在蓬窗上。说什么脂正浓、粉正香,如何两鬓又成霜?昨日黄土陇头\footnote{黄土陇(lǒnɡ拢)头——指坟墓。陇:通“垄”,田中高地;坟墓。}送白骨,今宵红灯帐底卧鸳鸯。金满箱,银满箱,展眼乞丐人皆谤。正叹他人命不长,那知自己归来丧!训有方,保不定日后作强梁\footnote{强梁——横暴;蛮不讲理。《庄子·山木》:“从其强梁。”唐代陆德明《释文》:“强梁,多力也。”这里指强盗。}。择膏粱\footnote{择膏粱——意谓挑选富贵人家子弟作婿。膏:脂肪;油。粱:精米。膏粱:本指精美的饭菜,这里用作“膏粱子弟”的省称。},谁承望流落在烟花巷\footnote{烟花巷——旧时妓院聚集的地方。烟花:歌女;娼妓。}!因嫌纱帽小,致使锁枷扛;昨怜破袄寒,今嫌紫蟒\footnote{紫蟒——紫色的蟒袍。紫:古代按官阶等级穿着不同颜色的公服;唐制,亲王及三品服用紫色。}长:乱烘烘你方唱罢我登场,反认他乡是故乡\footnote{反认他乡是故乡——这里把现实人生比作暂时寄居的他乡,而把超脱尘世的虚幻世界当作人生本源的故乡;因而说那些为功名利禄、姣妻美妾、儿女后事奔忙而忘掉人生本源的人是错将他乡当作故乡。}。甚荒唐,到头来都是为他人作嫁衣裳\footnote{为他人作嫁衣裳——喻白白替他人奔忙,死后一切皆空。唐代秦韬玉《贫女》诗:“苦恨年年压金线,为他人作嫁衣裳。”}!
\par 那疯跛道人听了,拍掌笑道:“解得切,解得切!”士隐便说一声“走罢!”将道人肩上褡裢\footnote{褡裢——一种中间开口而两端装钱物的长方口袋,小的可以挂在腰带上,大的可以搭在肩膀上。}抢了过来背着,竟不回家,同了疯道人飘飘而去。当下烘动街坊,众人当作一件新闻传说。封氏闻得此信,哭个死去活来,只得与父亲商议,遣人各处访寻,那讨音信?无奈何,少不得依靠着他父母度日。幸而身边还有两个旧日的丫鬟服侍,主仆三人,日夜作些针线发卖,帮着父亲用度。那封肃虽然日日抱怨,也无可奈何了。
\par 这日,那甄家大丫鬟在门前买线,忽听街上喝道之声,众人都说新太爷到任。丫鬟于是隐在门内看时,只见军牢快手\footnote{军牢快手——封建官吏手下执行缉捕、防卫和行刑的隶卒。官僚出巡,常由他们前呼后拥,以示威势。},一对一对的过去,俄而大轿抬着一个乌帽猩袍的官府过去。丫鬟倒发了个怔,自思这官好面善,倒像在那里见过的。于是进入房中,也就丢过不在心上。至晚间,正待歇息之时,忽听一片声打的门响,许多人乱嚷,说:“本府太爷差人来传人问话。”封肃听了,唬得目瞪口呆,不知有何祸事,且听下回分解。\footnote{“且听下回分解”——底本无,俄藏本作“下回便晓”。据卞藏本补。}










