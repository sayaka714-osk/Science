
\section{流星雨}


\subsection{演讲}


\subsubsection{一个男孩子的爱情}


\par 今天要说的只是一个爱的故事,是一个有关三十岁就过世的一个男孩子,十三年来爱情的经过,那个人就是我的先生。他的西班牙名字是Jose,我给他取了一个中文名字叫荷西,取荷西这个名字实在是为了容易写,可是如果各位认识他的话,应该会同意他该改叫和曦,和祥的“和”,晨曦的“曦”,因为他就是这样的一个人。可是他说,那个“曦”字实在太难写了,他学不会,所以我就教他写这个我顺口喊出来的“荷西”了。
\paragraph{这么英俊的男孩!}
\par 认识荷西的时候,他不到十八岁,在一个耶诞节的晚上,我在朋友家里,他刚好也来向我的一些中国朋友祝贺耶诞节。西班牙有一个风俗,耶诞夜十二点一过的时候,邻居们就要向左邻右舍楼上、楼下一家家地恭贺,并说:“平安。”有一点像我们国人拜年的风俗。那时荷西刚好从楼上跑下来,我第一眼看见他时,触电了一般,心想,世界上怎么会有这么英俊的男孩子?如果有一天可以作为他的妻子,在虚荣心上,也该是一种满足了,那是我对他的第一次印象。过了不久,我常常去这个朋友家玩,荷西就住在附近,在这栋公寓的后面有一个很大的院子,我们就常常在那里打棒球,或在下雪的日子里打雪仗,有时也一齐去逛旧货市场。口袋里没什么钱,常常从早上九点逛到下午四点,可能只买了一支鸟羽毛,那时荷西高三,我大学三年级。
\paragraph{表弟来啰!}
\par 有一天我在书院宿舍里读书,我的西班牙朋友跑来告诉我:“Echo,楼下你的表弟来找你了。”“表弟”在西班牙文里带有嘲弄的意思,她们不断地叫着“表弟来啰!表弟来啰!”我觉得很奇怪,我并没有表弟,哪来的表弟在西班牙呢?于是我跑到阳台上去看,看到荷西那个孩子,手臂里抱了几本书,手中捏着一顶他常戴的法国帽,紧张得好像要捏出水来。
\par 因为他的年纪很小,不敢进会客室,所以站在书院外的一棵大树下等我,我看是他,匆匆忙忙地跑下去,到了他面前还有点生气,推了他一把说:“你怎么来了?”他不说话,我紧接着问:“你的课不是还没有上完吗?”他答道:“最后两节不想上了。”我又问:“你来做什么?”因为我总觉得自己比他大了很多,所以总是以一个姊姊的口气在教训他。他在口袋里掏出了十四块西币来(相当于当时的七块台币),然后说:“我有十四块钱,正好够买两个人的入场券,我们一起去看电影好吗?但是要走路去,因为已经没有车钱了。”我看了他一眼。我是一个很敏感的人,觉得这个小孩子有一点不对劲了,但是我还是答应了他,并且建议看附近电影院的电影,这样就不需要车钱。第二天他又逃课来了,第三天、第四天……于是树下那个手里总是捏着一顶法国帽而不戴上去的小男孩,变成了我们宿舍里的一个笑话,她们总是喊:“表弟又来啰!”我每次跑下楼去,总要推荷西一把或打他一下,对他说:“以后不要来了,这样逃课是不行的!”因为最后两节课他总是不上,可是他仍是常常来找我。因为两个人都没钱,就只有在街上走走,有时就到皇宫去看看,捡捡人家垃圾场里的废物,还会惊讶地说:“你看看这支铁钉好漂亮哟!哇!你看看这个……”渐渐地我觉得这个交往不能再发展下去了,因为这个男孩子认真了,而他对我是无能为力的,因为他大学还没有念,但老实说我心里实在是蛮喜欢他的。
\paragraph{你再等我六年!}
\par 有一日,天已经很冷了,我们没有地方去,把横在街上的板凳,搬到地下车的出风口,当地下车经过的时候一阵热风吹出来,就是我们的暖气。两个人就冻在那个板凳上像乞丐一样。这时我对荷西说:“你从今天起不要来找我了。”我为什么会跟他说这种话呢?因为他坐在我的旁边很认真地跟我说:“再等我六年,让我四年念大学,二年服兵役,六年以后我们可以结婚了,我一生的想望就是有一个很小的公寓,里面有一个像你这样的太太,然后我去赚钱养活你,这是我一生最幸福的梦想。”他又说:“在我自己的家里得不到家庭的温暖。”我听到他这个梦想的时候,突然有一股要流泪的冲动,我跟他说:“荷西,你才十八岁,我比你大很多,希望你不要再做这个梦了,从今天起,不要再来找我,如果你又站在那个树下的话,我也不会再出来了,因为六年的时间实在太长了,我不知道我会去哪里,我也不会等你六年。你要听我的话,不可以来缠我,你来缠的话,我是会怕的。”他愣了一下,问:“这阵子来,我是不是做错了什么?”我说:“你没有做错什么,我跟你讲这些话,是因为你实在太好了,我不愿意再跟你交往下去。”接着,我站起来,他也跟着站起来,一齐走到马德里皇宫的一个公园里,园里有个小坡,我跟他说:“我站在这里看你走,这是最后一次看你,你永远不要再回来了。”他说:“我站这里看你走好了。”我说:“不!不!不!我站在这里看你走,而且你要听我的话哟,永远不可以再回来了。”那时候我很怕他再来缠我,我就说:“你也不要来缠我,从现在开始,我要跟我班上的男同学出去,不能再跟你出去了。”这么一讲,自己又紧张起来,因为我害怕伤害到一个初恋的年轻人,通常初恋的人感情总是脆弱的。他就说:“好吧!我不会再来缠你,你也不要把我当做一个小孩子,因为我们这几个星期来的交往,你始终把我当做一个孩子,你说‘你不要再来缠我了’,我心里也想过,除非你自己愿意,我永远不会来缠你。”
\paragraph{Echo再见!}
\par 讲完那段话,天已经很晚了,他开始慢慢地跑起来,一面跑一面回头,一面回头,脸上还挂着笑,口中喊着:“Echo再见!Echo再见!”我站在那里看他,马德里是很少下雪的,但就在那个夜里,天下起了雪来。荷西在那片大草坡上跑着,一手挥着法国帽,仍然频频地回头,我站在那里看荷西渐渐地消失在黑茫茫的夜色与皑皑的雪花里,那时我几乎忍不住喊叫起来:“荷西!你回来吧!”可是我没有说。以后每当我看《红楼梦》宝玉出家的那一幕,总会想到荷西十八岁那年在那空旷的雪地里,怎么样跑着、叫着我的名字:“Echo再见!Echo再见!”
\par 他跑了以后,果然没有再来找过我,也没有来缠过我。我跟别的同学出去的时候,在街上常会碰见他,他看见我总是用西班牙的礼节握住我的双手,亲吻我的脸,然后说:“你好!”我也说:“荷西!你好,这是我的男朋友××人。”他就会跟别人握握手。
\paragraph{他留了胡子,长大了!}
\par 这样一别,别了六年,我学业告了一个段落,离开西班牙,回到了台湾。在台湾时,来了一位西班牙的朋友,他说:“你还记不记得那个Jose呀!”我说:“记得呀!”他说:“噢!他现在不同了,留了胡子,也长大了。”“真的!”他又说:“我这里有一封他写给你的信还有一张照片,你想不想看?”我惊讶地说:“好呀!”因为我心里仍在挂念着他,但那位朋友说:“他说如果你已经把他给忘了,就不要看这封信了。”我答道:“天晓得,我没有忘记过这个人,只是我觉得他年纪比我小,既然他认真了,就不要伤害他。”我从那个朋友手中接过那封信,一张照片从中掉落出来,照片上是一个留了大胡子穿着一条泳裤在海里抓鱼的年轻人,我立刻就说:“这是希腊神话里的海神嘛!”打开了信,信上写着:“过了这么多年,也许你已经忘记了西班牙文,可是我要告诉你一个秘密,在我十八岁那个下雪的晚上,你告诉我,你不再见我了,你知道那个少年伏枕流了一夜的泪,想要自杀?这么多年来,你还记得我吗?我和你约的期限是六年。”就是这样的一封信,我没有给他回信,把那封信放在一边,跟那个朋友说:“你告诉他我收到了这封信,请代我谢谢他。”半年以后,我在感情上遇到了一些波折,离开台湾,又回到了西班牙。
\paragraph{荷西,我回来了!}
\par 当时荷西在服最后的一个月兵役,荷西的妹妹老是要我写信给荷西,我说:“我已经不会西班牙文了,怎么写呢?”然后她强迫将信封写好,声明只要我填里面的字,于是我写了一封英文的信到营区去,说:“荷西!我回来了,我是Echo,我在××地址。”结果那封信传遍了营里,却没有一个人懂英文,急得荷西来信说,不知道我说些什么,所以不能回信给我,他剪了很多潜水者的漫画寄给我,并且指出其中一个说:“这就是我。”我没有回信,结果荷西就从南部打长途电话来了:“我二十三日要回马德里,你等我噢!”到了二十三日我完全忘了这件事,与另一个同学跑到一个小城去玩,当我回家时,同室的女友告诉我有个男孩打了十几个电话找我,我想来想去,怎么样也想不起会是哪个男孩找我。正在那时我接到我的女友——一位太太的电话,说是有件很要紧的事与我商量,要我坐计程车去她那儿。我赶忙乘计程车赶到她家,她把我接进客厅,要我闭上眼睛,我不知她要玩什么把戏忙将拳头握紧,把手摆在背后,生怕她在我手上放小动物吓我。当我闭上眼睛,听到有一个脚步声向我走来,接着就听到那位太太说她要出去了,但要我仍闭着眼睛。突然,背后一双手臂将我拥抱了起来,我打了个寒颤,眼睛一张开就看到荷西站在我眼前,我兴奋得尖叫起来,那天我正巧穿着一条曳地长裙,他穿的是一件枣红色的套头毛衣。他揽着我兜圈子,长裙飞了起来,我尖叫着不停地捶打着他,又忍不住捧住他的脸亲他。站在客厅外的人,都开怀地大笑着,因为大家都知道,我和荷西虽不是男女朋友,感情却好得很。
\par 在我说要与荷西永别后的第六年,命运又将我带回了他的身旁。
\paragraph{你是不是还想结婚?}
\par 在马德里的一个下午,荷西邀请我到他的家去。到了他的房间,正是黄昏的时候,他说:“你看墙上!”我抬头一看,整面墙上都贴满了我发了黄的放大黑白照片,照片上,剪短发的我正印在百叶窗透过来的一道道的光纹下。看了那一张张照片,我沉默了很久,问荷西:“我从来没有寄照片给你,这些照片是哪里来的?”他说:“在徐伯伯的家里。你常常寄照片来,他们看过了就把它摆在纸盒里,我去他们家玩的时候,就把他们的照片偷来,拿到相馆去再做底片放大,然后再把原来的照片偷偷地放回盒子里。”我问:“你们家里的人出出进进怎么说?”“他们就说我发神经病了,那个人已经不见了,还贴着她的照片发痴。”我又问:“这些照片怎么都黄了?”他说:“是嘛!太阳要晒它,我也没办法,我就把百叶窗放下,可是百叶窗有条纹,还是会晒到。”说的时候,一副歉疚的表情,我顺手将墙上一张照片取下来,墙上一块白色的印子。我转身问荷西:“你是不是还想结婚?”这时轮到他呆住了,仿佛我是个幽灵似的。他呆望着我,望了很久,我说:“你不是说六年吗?我现在站在你的面前了。”我突然忍不住哭了起来,又说:“还是不要好了,不要了。”他忙问:“为什么?怎么不要?”那时我的新愁旧恨突然都涌了出来,我对他说:“你那时为什么不要我?如果那时候你坚持要我的话,我还是一个好好的人,今天回来,心已经碎了。”他说:“碎的心,可以用胶水把它粘起来。”我说:“粘过后,还是有缝的。”他就把我的手拉向他的胸口说:“这边还有一颗,是黄金做的,把你那颗拿过来,我们交换一下吧!”
\par 七个月后我们结婚了。
\par 我只是感觉冥冥中都有安排,感谢上帝,给了我六年这么美满的生活,我曾经在书上说过:“在结婚以前我没有疯狂地恋爱过,但在我结婚的时候,我却有这么大的信心,把我的手交在他的手里,后来我发觉我的决定是对的。”如果他继续活下去,仍要说我对这个婚姻永远不后悔。所以我认为年龄、经济、国籍,甚至于学识都不是择偶的条件,固然对一般人来说这些条件当然都是重要的,但是我认为最重要的,还是彼此的品格和心灵,这才是我们所要讲求的所谓“门当户对”的东西。
\paragraph{你不死、你不死……}
\par 荷西死的时候是三十岁。我常常问他:“你要怎么死?”他也问我:“你要怎么死?”我总是说:“我不死。”有一次《爱书人》杂志向我邀一篇“假如你只有三个月可活,你要怎么办”的稿子,我把邀稿信拿给荷西看,并随口说:“鬼晓得,人要死的时候要做什么!”他就说:“这个题目真奇怪呀!”我仍然继续地揉面,荷西就问我:“这个稿子你写不写?你到底死前三个月要做什么,你到底要怎么写嘛?”我仍继续地揉面,说:“你先让我把面揉完嘛!“”你到底写不写啊?”他直问,我就转过头来,看着荷西,用我满是面糊的手摸摸他的头发,对他说:“傻子啊!我不肯写,因为我还要替你做饺子。”讲完这话,我又继续地揉面,荷西突然将他的手绕着我的腰,一直不肯放开,我说:“你神经啦!”因为当时没有擀面棍,我要去拿茶杯权充一下,但他紧搂着我不动,我就说:“走开嘛!”我死劲地想走开,他还是不肯放手,“你这个人怎么这么讨厌……”话正说了一半,我猛一回头,看到他整个眼睛充满了泪水,我呆住了,他突然说:“你不死,你不死,你不死……”然后又说:“这个《爱书人》杂志我们不要理他,因为我们都不死。“”那么我们怎么样才死?”我问。“要到你很老我也很老,两个人都走不动也扶不动了,穿上干干净净的衣服,一齐躺在床上,闭上眼睛说:好吧!一齐去吧!”所以一直到现在,我还是没有为《爱书人》写那篇稿子《,爱书人》最近也问我,你为什么没有写呢?我告诉他们因为我有一个丈夫,我要做饺子,所以没能写。
\paragraph{你要叫他爸爸}
\par 我的父母要到加纳利群岛以前,先到西班牙,荷西就问我看到了我爸爸,该怎么称呼?是不是该叫他陈先生?我说:“你如果叫他陈先生,他一下飞机就会马上乘原机回台北,我不是叫你父亲做爸爸吗?”他说:“可是我们全家都觉得你很肉麻呀!”原来在西班牙不叫自己的公公婆婆做父亲、母亲,而叫××先生,××太太。但我是一个中国人,我拒绝称呼他们为先生、太太,我的婆婆叫马利亚,我就称她马利亚母亲,叫公公做西撒父亲。荷西就说:“我叫爸爸陈先生好了!”我说:“你不能叫他陈先生,你要叫他爸爸。”结果我陪我的父母在西班牙过了十六天,回到加纳利群岛,荷西请了假在机场等我们。我曾对他说:“我的生命里有三个人,一个是爸爸,一个是妈妈,还有就是你,再者就是我自己,可惜没有孩子,否则这个生命的环会再大一点,今天我的父母能够跟你在一起,我最深的愿望好像都达成了,我知道你的心地是很好的,但你的语气和脾气却不一定好,我求求你在我父母来的时候,一次脾气也不可发,因为老人家,有的时候难免会有一点噜苏。”他说:“我怎么会发脾气?我快乐还来不及呢!”为了要见我的父母,他每天要念好几小时的英文,他的英文还是三年以前在尼日利亚学的。当他看到我们从机场走出来时,他一只手抱着妈妈,另一只手抱着爸爸,当他发现没有手可以抱我时就对我说:“你过来。”然后他把我们四个人都环在一起,因为他已经十六天没有看到我了。然后又放开手紧紧地抱抱妈妈、爸爸,然后再抱我。他第一眼看到爸爸时很紧张,突然用中国话喊:“爸爸!”然后看看妈妈,说:“妈妈!”接着,好像不知道该说些什么,低下头拼命去提箱子,提了箱子又拼命往车子里乱塞,车子发动时我催他:“荷西,说说话嘛!你的英文可以用,不会太差的。”他就用西班牙文说:“我实在太紧张了,我已经几个晚上没睡觉了,我怕得不得了。”那时我才明白,也许一个中国人喊岳父、岳母为爸爸妈妈很顺口,但一个外国人你叫他喊从未见过面的人为爸、妈,除非他对自己的妻子有太多的亲情,否则是不容易的。回到家里,我们将房间让给父母住,我和荷西就住进更小的一间。有一天在餐桌上,我与父母聊得愉快,荷西突然对我说,该轮到他说话了,然后用生硬的英语说:“爹爹,你跟Echo说我买摩托车好不好?”荷西很早就想买一辆摩托车,但要通过我的批准,听了他这句话,我站起来走到洗手间去,拿起毛巾捂住眼睛,就出不来了。从荷西叫出“爹爹”这个字眼时(爹爹原本是三毛对爸爸的称呼),我相信他与我父母之间又跨进了一大步。
\par 我的父母本来是要去欧洲玩的,父亲推掉了所有的业务,打了无数的电话、电报,终于见到了他们的女婿,他们相处整整有一个月的时间。我和荷西曾约定只要我俩在一起小孩子还是别出世吧,如果是个女的我会把她打死,因为我会吃醋,若是个男孩,荷西要把他倒吊在阳台上,因为我会太爱那孩子。事后,我也讶异这样孩子气及自私的话竟会从一对夫妻的口中说出。当我的父母来了一个月后,荷西突然问:“你觉不觉得我们该有一个孩子?”我说:“是的,我觉得。”他又说:“自从爸妈来了以后,家里增添了很多家庭气氛,我以前的家就没有这样的气氛。”
\paragraph{永远的挥别}
\par 在我要陪父母到伦敦以及欧洲旅游时,荷西到机场来送行,他抱着我的妈妈说:“妈妈,我可不喜欢看见你流泪哟!明年一月你就要在台北的机场接我了,千万不要难过,Echo陪你去玩。”我们坐的是一架小型的螺旋桨飞机,因为我们要住的那个小岛,喷射机是不能到的。上飞机前,我站在机肚那里看荷西,就在那时,荷西正跳过一个花丛,希望能从那里,再看到我们,上了飞机,我又不停地向他招手,他也不停地向我招手,直到服务小姐示意我该坐下。坐下后,旁边有位太太就问我:“那个人是你的丈夫吗?”我说:“是的。”她又问荷西来做什么,我就将我父母来度假他来送行的事简单地告诉她,她就告诉我:“我是来看我儿子的。”然后就递给我一张名片,西班牙有一个风俗,如果你是守寡的女人,名片上你就要在自己的名字后面,加上一句“某某人的未亡人”,而那名片上正有那几个字,使我感到很刺眼,很不舒服,不知道要跟她再说些什么,只好说声:“谢谢!”没想到就在收到那张名片的两天后,我自己也成了那样的身份……
\par (说到这里,三毛的声音哽咽,她在台上站了很久,再说不出一句话来,演讲中断……)


\subsubsection{我的写作生活}
















\subsection{采访}












