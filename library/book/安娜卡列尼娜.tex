
\section{安娜·卡列尼娜}


\par 书名:安娜·卡列尼娜(译文名著精选)
\par 作者:[俄]列夫·托尔斯泰
\par 译者:高惠群,傅石球,于国畔,顾生根
\par 出版社:上海译文出版社
\par 出版时间:2010-08
\par ISBN:9787532751204


\subsection*{译本序}

\par 《安娜·卡列尼娜》(1873—1878)是俄国大文豪列夫·托尔斯泰的第二部长篇巨著。起初,托翁只打算把它写成一部“一个不忠实的妻子以及由此而发生的全部悲剧”(贝奇科夫语),仅用了五十天他便粗略地完成了全书。五年多以后,在前后用过《年轻太太》、《两段婚姻》、《两对夫妻》等书名后,它以《安娜·卡列尼娜》的名字问世了。
\par 这部小说的主要意义应该包括三方面,即安娜的个人悲剧;1860年代的俄国社会——沙龙、军官俱乐部、舞会、戏院、赛马……以及自传的性质。
\par 《安娜·卡列尼娜》开篇第一句话,对于中国读者,甚至没有读过此书的中国人来说,都不陌生:“幸福的家庭无不相似,不幸的家庭各有不幸。”安娜是一位穿着黑衣的最迷人的少妇,她善良、聪慧、生命力旺盛,男人和女人都为她着迷。她身上迸发出的爱情“含有一种暴烈的、肉感的、专横的性格”(罗曼·罗兰语)。其实,作家对婚姻、家庭问题的思考可以追溯到动笔撰写这部小说前的五年,即1868年,这一年,他在题为《论婚姻和妇女的天职》一文中说:“男人的天职是做人类社会蜂房的工蜂,那是无限多样化的;而母亲的天职呢,没有她们便不可能繁衍后代,这是唯一确定无疑的。”托尔斯泰借莱温和基季的恋爱婚姻表达出这一妇女观、家庭观。紧随这段话托翁又说:“虽然如此,妇女还是常常看不到这一使命,而选择虚假的,即其他的使命……这一使命的重要性和无限性,以及它只能在一夫一妻的形式(即过去和现在生活着的人称之为家庭的形式)下才能实现……因而一个妇女为了献身于母亲的天职而抛弃个人的追求越多,她就越完美。”由此不难理解,托尔斯泰为何将安娜命运的结局安排为卧轨自杀——在小说接近尾声的第七部第三十章,安娜还在想着“只要办完离婚手续,阿列克谢·亚力山德罗维奇把谢廖扎还给我,我就与弗龙斯基结婚”。既然还不牺牲个人的追求,在托翁看来,这样的女子就完美不起来,那就让她毁灭吧!可小说并没有因为安娜的死亡而结束。整个第八部的十九个章节的内容,就如同《战争与和平》长长的“尾声”,如果以西欧小说式的结局为标准,这已不像是“尾声”。
\par 可见,《安娜·卡列尼娜》不只是关注安娜的死,安娜的悲剧一直扩展到所有家庭的幸与不幸。在对安娜形象的塑造上,托尔斯泰倾注了他对人的肉体本能因素、人的伦理因素、人的“灵魂”因素、人的社会因素等的思考与体悟。在此部小说之后的《忤悔录》(1879—1882)中,托翁还在进行着与上述内容相关的精神探索。
\par \rightline{查晓燕}
\par \rightline{2006.3.22}


\clearpage

\begin{center}
    伸冤在我,我必报应。\footnote{语出《新约·罗马书》第12章第19节,全句为:“亲爱的弟兄,不要自己伸冤,宁可让步,听凭主怒,因为经上记着:‘主说,伸冤在我,我必报应。’”}
\end{center}

\subsection*{第一部}



\subsubsection*{一}

\par 幸福的家庭无不相似,不幸的家庭各有不幸。
\par 奥布隆斯基家里全乱了套。妻子得知丈夫和过去的法国女家庭教师有染,就对丈夫声称,不可能和他同住在一个家里。这种局面僵持到第三天,夫妻双方及全体家人都有痛切感受。大家觉得住在一起实在无聊,随便哪家客店里偶然相逢的人也会比他们,奥布隆斯基家的人关系更好些。妻子不出房门,丈夫三日不归。孩子们满屋乱跑,无人照料。英国女家庭教师跟女管家吵了架,写信请朋友重新找份工作。厨师昨天就离开了家,在午餐时走的。打下手的厨娘,还有马车夫也都要求辞工。
\par 吵闹的第三天,斯捷潘·阿尔卡季奇·奥布隆斯基公爵(在社交场合他叫斯季瓦)在通常时间、即上午八时醒来,但不在妻子的卧室,而在书房里的山羊皮长沙发上。他在弹簧上翻了一下保养得很好的丰满身体,紧紧搂住枕头,把脸埋进去,似乎还想好好睡一觉,可是他突然一骨碌爬起来,坐在沙发上,睁开了眼睛。
\par “哦,哦,梦见什么了?”他想起做过一个梦。“哦,梦见什么了?对了!阿拉宾在达姆施塔特\footnote{德国西部城市。}举行宴会;不,不在达姆施塔特,而是美国的什么地方。对,那个达姆施塔特在美国。对,阿拉宾在玻璃餐桌上设宴,对的,大家都唱Il mio tesoro\footnote{意大利语:我的宝贝。},不是Il mio tesoro,比这更好听的,还有那些细颈小玻璃瓶,原来都是一个个女人,”他回忆着。
\par 斯捷潘·阿尔卡季奇眼睛里闪出快乐的光,微笑着沉思起来。“哦,是个好梦,非常之好。梦里还有许多美妙的东西,难以言传,醒了连什么情景也说不清楚了。”他看见一道亮光从呢绒窗幔的边缘射进来,高兴地把腿伸到沙发下面,用脚探到妻子为他绣上花的那双金黄色羊皮便鞋(去年的生日礼物),按照九年来的老习惯,并不起身,把手伸向他在卧室里挂睡衣的老地方。这当儿他才猛然想起,他怎么和为什么没有睡在妻子的卧室而睡在书房里。脸上的笑容不见了,他蹙起了额头。
\par “唉,唉!……”他咕咕哝哝地说,回忆起事情的全部经过。脑海中又出现了同妻子口角的所有细节,想起他那进退维谷的处境,还有他犯下的最使人痛苦的过错。
\par “是啊!她不肯宽恕我,不可能宽恕我。最糟糕的是一切皆由我而起,而又不能怪我。这是整个悲剧所在,”他这样想。“唉,唉!”他悲观失望,又想起了这场口角中最令他痛苦的那些情景。
\par 最难堪的是起初的那一刻,当时他刚看完戏回家,高高兴兴,心满意足,手里还拿着一只大梨子准备送给妻子,可是在客厅里没见到她;奇怪的是,她也不在书房,最后在卧室里找到她,她手里正拿着那封使丑事败露的倒霉的信。
\par 多莉是个操劳不停、他认为不大聪明的女人。这时她手里拿着那封信,一动不动地坐在那里,带着恐惧、绝望而愤怒的神情注视着他。
\par “这是什么?这?”她指着信问道。
\par 每次回想到这里,斯捷潘·阿尔卡季奇总是很苦恼,倒不是为了那件事本身,主要是他对妻子的质问竟然作出了那样的回答。
\par 当时他的处境,正像那些干了十分丢脸的事突然被揭发出来的人一样。妻子揭了他的丑,而他却不能神色镇定地应付他面临的局面。他本可以表示委屈,可以否认、辩解、求饶,甚至哪怕是满不在乎也好,可是他却干了什么啊!在他的脸上,居然不由自主地(那是“大脑反射”——爱好生理学的斯捷潘·阿尔卡季奇这样认为)露出了他平时那副憨厚的,而现在却是愚蠢的微笑。
\par 他不能自宥这愚蠢的一笑。多莉看到这副笑容,仿佛肉体疼痛似地颤栗了一下,接着就狠狠地发作起来,以她特有的急躁劲,滔滔不绝地喷吐了一通尖酸刻薄的话,然后奔出房间。打那以后,她再也不见丈夫的面了。
\par “都怪这愚蠢的一笑,”斯捷潘·阿尔卡季奇想。
\par “可是怎么办呢?怎么办呢?”他在绝望地自言自语,找不到答案。

\subsubsection*{二}

\par 斯捷潘·阿尔卡季奇对自己倒也实事求是。他并不自我欺骗,相信自己对做过的事追悔莫及。他是一个三十四岁、漂亮而多情的男子。妻子只比他小一岁,是现有的五个孩子、还有两个已夭折的孩子的母亲。他并不爱她,对此他至今倒也无悔。他所后悔的,只是没有把那件事更好地瞒住她。不过,他仍然感到处境困难,妻子、孩子,还有自己都很可怜。要是他早先料到这个消息对妻子打击如此之大,也许会对她紧紧掩盖住自己的罪过。对于这个问题,他从来没有认真考虑过,他只是模糊感到,妻子早已觉察到他的不忠,只不过眼开眼闭罢了。他甚至觉得,她身体虚弱,人老珠黄,姿色平常,毫无出众之处,仅仅是一位家庭慈母,平心而论,她应该是宽大为怀的。结果事情却闹得适得其反。
\par “唉,可怕!哎呀呀,真可怕!”斯捷潘·阿尔卡季奇不断嘀咕着,却想不出办法。“这以前一切多么美好,我们生活得多么和睦!她有孩子们在身边,感到满足和幸福,我也从不干涉她,让她忙孩子忙家务,遂了她的心意。说实在的,糟糕的就是她来当了我们的家庭教师。勾搭家庭教师确实有些庸俗下流。可她是个多么漂亮的家庭教师啊!(他真切地回忆起Mlle Roland\footnote{法语:罗朗小姐。}那双调皮的黑眼睛和她的微笑。)她在我家时,我丝毫也不曾放肆。最糟的是她现在已经……偏偏这就像故意作对似的!哎呀呀!这可怎么办,怎么办呢?”
\par 答案没有找到。只有生活能给他提供一个普通的解答,可以用它来应付所有无法解决的难题。这个解答是:去过日常生活,把烦恼丢在脑后。他想回到梦中去,这要等到夜晚才行。梦中的音乐,那些玻璃瓶女人的歌唱此刻不可能重温。看来,他只能在糊里糊涂的日子里去忘忧解愁了。
\par “以后再说吧,”斯捷潘·阿尔卡季奇自语道,穿上那件浅蓝丝绸衬里的灰色睡衣,系好绦带,往宽阔的胸腔里足足地吸了口气,迈开他丰满身躯下面那双轻快的外八字脚,像平时一样精神抖擞地走到窗前,拉开窗帘,使劲地按了按铃。应声进来的是他的老仆马特维,手里拿着衣服、靴子和一封电报。随后走进来的是带着刮脸用具的理发匠。
\par “机关里有公文来吗?”斯捷潘·阿尔卡季奇问,接过电报,在镜子前坐下来。
\par “放在桌上了,”马特维答道,带着询问和关切的神情瞥了主人一眼,停了一会,又狡黠地笑笑说:“车夫主人那边有人来过。”
\par 斯捷潘·阿尔卡季奇没有答腔,只是从镜子里瞥了马特维一眼;两人目光在镜中相遇,可以看出,他们是心照不宣的。斯捷潘·阿尔卡季奇的眼色仿佛在问:“你干吗说这个?难道你不知道吗?”
\par 马特维把双手插进外衣口袋,挪了挪腿,脸上带着笑意,默默地、和善地看了主人一眼。
\par “我叫他们礼拜天来,何必来早了麻烦您又自找麻烦,”马特维的这句话显然是事先考虑好的。
\par 斯捷潘·阿尔卡季奇明白,马特维是想说句笑话逗引别人的注意。他拆开电报看了一遍,猜懂了译电中常见的几个错别字,顿时喜形于色。
\par “马特维,我妹妹安娜·阿尔卡季耶夫娜明天就到,”他说。这时理发匠正在修剪他那又长又鬈的络腮胡子,使淡红色的皮肤显露出来,他示意那只溜光的胖手暂停一下。
\par “谢天谢地,”马特维说,表示他和主人同样明白这次来访的意义,也就是说,斯捷潘·阿尔卡季奇这位亲爱的胞妹安娜·阿尔卡季耶夫娜,她可能促成兄嫂重新和好。
\par “一个人来,还是同她先生一道来?”马特维问。
\par 斯捷潘·阿尔卡季奇不能说话,理发匠正在剃他的上唇的胡子,他竖起一根手指,马特维在镜子里点点头。
\par “一个人。要在楼上收拾房间吗?”
\par “禀报达里娅·亚历山德罗夫娜,她自会吩咐的。”
\par “达里娅·亚历山德罗夫娜?”马特维有些怀疑地问。
\par “对,禀报她。把电报也拿去,然后告诉我她有何吩咐。”
\par “您想试探一下,”马特维心里明白,嘴上却说:
\par “遵命。”
\par 斯捷潘·阿尔卡季奇梳洗完毕,正准备穿戴,这时马特维手里拿着那份电报,慢吞吞地,把靴子踩得吱吱作响地回到房里来。理发匠已经走了。
\par “达里娅·亚历山德罗夫娜叫我禀报您,她就要走了。说随便他,也就是您,爱怎么办就怎么办吧,”马特维说,眼睛里含着笑意,把双手插进衣袋,侧着脑袋,凝视着主人。
\par 斯捷潘·阿尔卡季奇沉默了一会,漂亮的脸上露出宽厚而又可怜的笑容。
\par “啊?马特维?”他摇摇头说。
\par “没关系,老爷,会顺利解决的,”马特维说。
\par “会顺利解决?”
\par “是的,老爷。”
\par “你这样认为吗?那是谁呀?”斯捷潘·阿尔卡季奇问,他听见门外有女人衣裙的窸窣声。
\par “是我。”一个稳重悦耳的女人声音说,接着,奶妈马特廖娜·菲利莫诺夫娜那张严肃的麻脸从门外伸了进来。
\par “什么事,马特廖莎?”斯捷潘·阿尔卡季奇朝门口迎去。
\par 虽然斯捷潘·阿尔卡季奇应该对妻子负全部罪责,他自己也觉得是这样,但是家中几乎所有的人,包括达里娅·亚历山德罗夫娜的心腹奶妈在内,全都站在他一边。
\par “什么事呀?”他闷闷不乐地问。
\par “您去一趟吧,老爷,去认个错。也许上帝会帮助您。她痛苦极了,看着多可怜,家里都闹翻天了。老爷,可怜可怜孩子们吧。认个错,老爷。没有办法呀!想图快活也得要……”
\par “她不肯见我……”
\par “您只管去认错。上帝会发慈悲的,您祷告上帝,老爷,祷告上帝吧。”
\par “那好,你去吧,”斯捷潘·阿尔卡季奇说,突然涨红了脸。“喂,现在穿衣服吧,”他对马特维说,动作利落地脱下了睡衣。
\par 马特维把衬衫张开伺候着,就像举着一个马轭,轻轻吹去上面的纤尘,带着明显满意的神情把它套在主人娇贵的身体上。

\subsubsection*{三}

\par 斯捷潘·阿尔卡季奇穿好衣服,往身上喷些香水,整理好衬衫的袖子,以习惯动作将香烟、皮夹、火柴和双链条带坠子的怀表分别放进几个口袋里,然后抖了抖手帕。虽然他遇上了倒霉事,但觉得自己还是那么清洁、芳香,身体健康而有朝气。他微微颠着腿走进餐厅,那儿已经摆好了咖啡,旁边是信件和机关里来的公文。
\par 他先看了信件。其中一个商人的来信很扫他的兴。此人想买妻子田庄上那片森林。森林固然该卖,只是眼下没有跟妻子和好前万万不可谈这件事。尤其令他不快的是,这种事情很可能使他面临的夫妻和解问题牵扯到金钱上的利害关系。难道他谋求与妻子和好就是出于这种利害关系,为了能卖掉那片森林吗?想到这里他感到受了侮辱。
\par 看罢来信,斯捷潘·阿尔卡季奇把公文挪过来,匆匆翻阅了两个案卷,用粗大的铅笔做了些记号,然后推开公文,端起咖啡,打开油墨未干的晨报,边喝咖啡边看起报来。
\par 斯捷潘·阿尔卡季奇订的是一份自由主义报纸,不是极端自由主义的,而是多数人赞成的那种自由主义。尽管他其实对科学、艺术和政治都不感兴趣,但他坚决拥护多数人和他订的报纸对这三类问题所持的观点,并且随着多数人观点的改变而改变,或者毋宁说,他并不改变观点,而是观点本身在他头脑中不知不觉地变化着。
\par 斯捷潘·阿尔卡季奇并不选择派别和观点,倒是这些派别和观点向他不招自来,就像他并不挑选礼帽或常礼服的样式,别人穿戴什么他就跟着买什么。对于生活在上流社会的他,对于一个成年人通常要开展某些精神活动而言,持有一种观点,就像戴一顶礼帽那样必需。如果说,他更有理由喜欢自由派,而不像他圈子里的许多人士那样赞成保守派,那倒并不是他认为自由派更有道理些,而是因为自由主义更适合他的生活方式。自由党常把俄国说得一无是处,说的倒不假,斯捷潘·阿尔卡季奇就是债台高筑,正缺钱花。自由党说婚姻制度过时,必须加以改革,不错,家庭生活对斯捷潘·阿尔卡季奇甚少乐趣,还迫使他违心地撒谎和装模作样。自由党说,或者毋宁说是暗示,宗教不过是给野蛮人套上的笼头,确实,斯捷潘·阿尔卡季奇只做一会儿祈祷两腿就疼得要命;再说他也不明白,现世的生活本可以过得很快活,为什么还要用恐怖夸张的语言谈论来世呢。斯捷潘·阿尔卡季奇也爱开个玩笑,捉弄一下老实人,例如他说,既然要炫耀家族门第,就不该只算到留里克\footnote{留里克王朝(869—1598)的奠基者,俄国王族及某些贵族被认为是其后裔。}为止,还应该承认最早的祖先——猿猴。就这样,斯捷潘·阿尔卡季奇对自由主义已习以为常,他喜欢看自己订的报纸,犹如饭后抽一支雪茄烟,使他头脑中产生轻雾似的朦胧感。他看到社论里说,有人叫嚷什么激进主义要吞噬一切保守分子,政府必须采取措施阻挡革命祸水,这种叫嚷在当代实在大可不必,相反,“据我们看来,危险并不在于什么假想的革命祸水,而在于传统势力之顽固不化,阻碍进步”云云。他又看到另一篇文章谈到财政问题,其中提到边沁\footnote{边沁(1748—1832),英国哲学家,功利主义哲学创始人。}和米勒\footnote{米勒(1806—1873),英国唯心主义哲学家、经济学家。},并对财政部语涉讥诮。凭着他特有的敏捷思路,他懂得各种讥诮的含义:谁讥诮谁以及因为何事而发;这种揣测常使他感受到一种乐趣。但是今天,想起了马特廖娜·菲利莫诺夫娜出的主意,想到家中诸事不遂,乐趣就变成了扫兴。报上还说,据闻,贝斯特伯爵已经到了威斯巴登。报上还有那些染头发、卖马车、征婚之类的广告,这些消息都不能像往常那样使他觉得滑稽有趣了。
\par 看过报纸,喝完第二杯咖啡,吃了一块黄油白面包,他站起身,抖去西装背心上的面包屑,舒展一下宽阔的胸膛,愉快地笑了——倒不是他的心情特别愉快,而是因为他的消化功能良好。
\par 不过,这愉快的一笑立刻勾起了全部往事,他又陷入了沉思。
\par 门外传来两个孩子的说话声(斯捷潘·阿尔卡季奇听出来是小儿子格里沙和大女儿塔尼娅)。他俩在搬运什么东西,弄翻在地上了。
\par “我说过,不能让旅客坐在车顶上,”小姑娘用英语嚷道,“去捡起来呀!”
\par “全都乱了套,”斯捷潘·阿尔卡季奇心想,“让孩子们自己到处乱跑。”他走到门口叫住了他们。姐弟俩扔下当作火车玩的小匣子,朝父亲走来。
\par 小姑娘是父亲的宝贝,她大胆地跑了进来,搂住父亲,笑着吊在他脖子上,像平时那样喜欢闻他络腮胡子上熟悉的香水气味。最后,小姑娘吻了吻父亲因为弯下身体而涨红了的那张慈爱的脸,松开双手,待要跑出去,父亲却拉住了她。
\par “妈妈怎么样?”他问道,一边抚摩着女儿柔嫩光滑的脖子。“你好,”他又朝向他问好的男孩子微笑说。
\par 他意识到自己不太喜欢儿子,所以总是努力做得公平些;儿子感到了这一点,对父亲冷淡的微笑并不报以笑容。
\par “妈妈?她刚起床,”小姑娘说。
\par 斯捷潘·阿尔卡季奇叹了口气。“这么说,她又是彻夜未眠,”他想。
\par “她高兴吗?”
\par 小姑娘知道父母亲吵过嘴,母亲不可能高兴,这一点父亲该是知道的,现在他这么随便地问,就是在装模作样。女儿为父亲脸红了。父亲立刻觉察到这一点,也脸红了。
\par “不知道,”她说。“她没叫我们读书,叫我们跟古莉小姐到外祖母家去玩。”
\par “哦,去吧,我的坦丘罗奇卡\footnote{塔尼娅的昵称。}。哦,等一下,”他说,仍然拉住女儿不放,抚摩着她柔嫩的小手。
\par 他从壁炉上取下昨天放在那里的一小盒糖果,挑了两块女儿爱吃的巧克力和水果软糖,递给她。
\par “这一块给格里沙吗?”小姑娘指着巧克力糖说。
\par “好的,好的。”他又抚摩了一下女儿的肩膀,在她发根上和脖子上亲了一下,才放她走。
\par “马车备好了,”马特维说。“可是有个女人求见,”他又补充道。
\par “等了很久吗?”斯捷潘·阿尔卡季奇问。
\par “有半个小时了。”
\par “对你说过多少次了,这种事情要立即禀报!”
\par “总得让您把咖啡喝完呀,”马特维以一种粗率友好的口气说,使人听了也不好生气。
\par “那就快请吧,”奥布隆斯基扫兴地皱起眉头说。
\par 求见者是一位上尉的妻子,叫加里宁娜。虽然她提出的请求无法满足,而且讲得前言不对后语,斯捷潘·阿尔卡季奇还是照例请她坐下来,毫不打断地倾听她的陈述,然后仔细替她出主意,叫她如何如何去找某某人,他甚至用他那清晰、漂亮、又长又粗的字体,工整而流畅地写下一封便函,让她拿去见那个能够周济她的人。
\par 打发走上尉的妻子,斯捷潘·阿尔卡季奇拿起礼帽,但他欲行又止,寻思是否忘记了什么事。看来,除了他想忘却的妻子之外,他并没有忘记什么。
\par “哎呀!”他垂下了头,漂亮的脸上露出忧愁的表情。“去还是不去呢?”他自言自语,但内心却在说,不必去了,除了虚情假意不会有别的,他俩的关系已经不可修复,因为既不能使她重新具有魅力而激发爱情,也不能把他变成失去恋爱能力的老人。现在除了虚伪和谎言,不可能有别的结果,而虚伪和撒谎却是有违他的本性的。
\par “可是迟早总得去,不能就这样不了了之,”他想,尽量使自己鼓起勇气。他挺起胸膛,掏出一支香烟,点燃后吸了两口,把它扔在珍珠贝做的烟灰缸里。他快步穿过光线阴暗的客厅,推开了另一扇门,那是通向妻子卧室的门。

\subsubsection*{四}

\par 达里娅·亚历山德罗夫娜的房间里,到处是散乱的衣物,她站在杂物当中,从她面前一个打开的小柜子里挑拣什么东西。她穿着短衫,往日一头浓密的秀发已经变得稀疏,编成辫子盘在后脑勺上。她容颜憔悴,两只大眼睛从消瘦的脸上凸显出来,露出惊恐的神色。听到丈夫的脚步声,她停住手,转眼望着门口,想在脸上装出严厉和鄙夷的表情,却怎么也装不像。她觉得自己害怕他,害怕眼下的会面。刚才她要做的事,这三天内已经尝试过多次:收拾自己的和孩子们的东西,送到娘家去。但她还是下不了决心。和前几次一样,这一次她也对自己说,不能这样就算完,一定得想办法惩罚和羞辱他,用他带给她的痛苦,哪怕只是一小部分,来报复他,让他也尝尝痛苦的滋味。她反复说要离开他,可是又觉得这不可能,因为她已经习惯了把他作为自己的丈夫并且爱他。另外她还觉得,在自己家照料五个儿女都快要忙不过来,带到外婆家,他们的情况将会更糟。何况这三天里,小儿子吃了不干净的肉汤已经生病,其余的孩子昨天几乎就没吃饭。她意识到走是不可能的,但为了骗骗自己,仍然拾掇东西,装成要走的样子。
\par 看见丈夫进来,她把手伸到小柜子的抽屉里,像是在寻找什么,丈夫走到她跟前,她才回过头望望他。她原想装出一副严厉而坚决的面孔,可是却流露出慌乱和痛苦的神情。
\par “多莉!”他畏怯地小声说,缩起脑袋,想装出可怜而顺从的样子,但还是显得那么喜气洋洋和气色健康。
\par 她很快地从头到脚打量一眼他那红光满面的健康身体。“是啊,瞧他多么称心如意!”她想,“而我?……他这副和气嘴脸真让人讨厌。大家因此喜欢他,夸他,我就恨他这副样子,”她想道,紧紧抿起了嘴,她那容易抽搐的苍白的脸上,右颊的肌肉开始颤抖。
\par “您要干什么?”她用急促的、气得变了腔调的低沉声音问道。
\par “多莉!”他又说,声音在打颤,“安娜今天要来了。”
\par “关我什么事?我不接待她!”她叫喊道。
\par “可是,也应该,多莉……”
\par “走开,走开,走开!”她望也不望他,又喊道,像是肉体受了痛苦发出的叫喊。
\par 当斯捷潘·阿尔卡季奇独自想到妻子的时候,他还能够保持镇定,指望事情像马特维所说的那样,会顺利解决,所以他能从容不迫地看报纸、喝咖啡。可是现在,当他目睹妻子这疲惫不堪的痛苦的面容,听见她听天由命、充满绝望的声音时,他突然感到呼吸困难,喉头哽咽,眼睛里也闪起了泪花。
\par “天哪,我干了什么啊!多莉!看在上帝的份上!……要知道……”他说不下去了,一阵呜咽堵住了他的喉咙。
\par 她啪的一声关上柜门,瞪了他一眼。
\par “多莉,我能说什么呢?只有一句话:饶恕我,饶恕我吧……你回想一下,难道九年的生活不能抵偿一时的,一时的……”
\par 她垂下眼睛在听他说,等他把话说完,仿佛在哀求他,希望他能够说服她。
\par “一时的忘情……”他终于说出口来,正想接着说下去,只见她又抿紧了嘴唇,像在忍受肉体的痛楚,右颊上的肌肉又抽搐起来。
\par “走开,从这儿走开!”她叫起来,声音更尖,“别对我说您的忘情,您的肮脏行为!”
\par 她想要走出去,身子晃了一下,连忙扶住椅背。他鼓胀着脸,嘴唇噘起,眼里含满了泪水。
\par “多莉!”他呜呜咽咽地说,“看在上帝份上,想想孩子们吧,他们是无辜的。全是我的错,你惩罚我,让我来赎罪吧。只要能办到,我什么都愿意做!我罪过,真是罪过啊!可是多莉,你饶恕我吧!”
\par 她坐了下来。他听着她沉重的大声喘息,说不出对她有多么的可怜。她几次想说话却开不了口。他等着她。
\par “你想到孩子,只是为了逗他们玩,而我想到他们,知道他们现在都给毁了,”她说出了显然是这三天来心中反复说过的一句话。
\par 她称他为“你”,他感激地看了她一眼,并想走过去拉她的手,她厌恶地避开了。
\par “我惦记着孩子们,为了救孩子我什么都愿意做,可是我不知道怎样救他们:让他们离开父亲,还是留在伤风败俗的父亲,是的,伤风败俗的父亲身边……您倒说说,发生了那种……事情之后,难道我们还能在一起生活吗?难道这可能吗?您说呀,难道这可能吗?”她重复说,声音越来越高。“我的丈夫,孩子的父亲,同自己孩子的女家庭教师发生了这种关系之后……”
\par “可是怎么办呢?怎么办呢?”他可怜巴巴地说,自己也不知道在说些什么,把头垂得越来越低了。
\par “您让我恶心,讨厌!”她喊叫起来,火气越来越大。“您的眼泪像水一样不值钱!您从来就不爱我。您既没有心肝也不光明正大!您叫我厌恶、恶心,您是陌生人,完全是陌生人!”她痛苦地、恶狠狠地说出了她感到可怕的这个字眼——陌生人。
\par 他望望她,她脸上的怒气使他既害怕又吃惊。他不明白,他的怜悯反而激怒了她。她看出来,他对她只是可怜而不是爱。“不,她恨我。她不会宽恕我,”他想。
\par “这太可怕了!太可怕了!”他说。
\par 这时,隔壁房间里有个小孩哭叫起来,大概是跌倒了。达里娅·亚历山德罗夫娜侧耳细听,脸色立刻缓和下来。
\par 她定了定神,似乎不知道自己身在何处,该做什么,随后她一下子站起来,向门口走去。
\par “瞧,她爱我的小孩,”他看见她听到孩子哭叫时脸色的变化,这样想,“她爱我的小孩,又怎么会恨我呢?”
\par “多莉,再听我说一句,”他跟在她身后说。
\par “您要跟着我,我就叫人来,叫孩子们来!让大家都知道您是个卑鄙的人!我今天就走,让您跟您的情妇住在这里吧!”
\par 她砰的一声带上门,走了。
\par 斯捷潘·阿尔卡季奇叹了口气,擦了擦脸,轻手轻脚地往外走。“马特维说会顺利解决。结果怎么样呢?我看简直没有可能。唉,唉,太可怕了!她那样叫喊真是俗气,”他自语道,回想起她的喊声和她的用词:卑鄙的人和情妇。“也许女仆们都听到了!真是俗不可耐。”斯捷潘·阿尔卡季奇独自站了一会,揩揩眼睛,叹息一声,然后挺起胸脯,走出了房间。
\par 今天是礼拜五。德国钟表匠正在餐厅里给钟上发条。斯捷潘·阿尔卡季奇想起他曾拿这个干活认真的秃头钟表匠开过玩笑,说德国人“为了给钟表上发条,自己一生上足了发条”。想到这里不禁莞尔一笑。斯捷潘·阿尔卡季奇喜欢俏皮的笑话。“说不定真的会顺利解决!这话真有趣:会顺利解决,”他想,“要讲讲此话的来历。”
\par “马特维!”他喊道,“你和玛丽亚把休息室收拾一下,迎接安娜·阿尔卡季耶夫娜,”他向走过来的马特维说。
\par “遵命。”
\par 斯捷潘·阿尔卡季奇穿上毛皮大衣,走到台阶上。
\par “您不回来吃饭吗?”送他出来的马特维说。
\par “看情况吧。这个拿去开销,”他说,从钱夹里掏出十卢布交给马特维。“够吗?”
\par “够不够都得应付过去,”马特维说,砰地关上车门,退回到台阶上。
\par 这时,达里娅·亚历山德罗夫娜已经哄好了孩子,听马车声知道丈夫已走,就又回到自己的卧室。这里是她唯一的避风港,可以躲一躲家务琐事的烦扰。只要她一出房门,那些琐事就缠得她不可开交。刚才就是这样,她到儿童室只去了不大一会儿工夫,英国女家庭教师和马特廖娜·菲利莫诺夫娜就向她提了好些个问题,而且都是迫不及待、唯有她才能答复的问题,诸如:孩子们穿什么衣服去散步?是否给他们喝牛奶?要不要派人另找一名厨师?等等。
\par “唉,别烦我,别烦我了!”她说。回到卧室后,她又坐到同丈夫说话的那个位置,她紧握双手,戒指从瘦削的手指上滑落下来,她开始回味整个谈话的经过。“他走了!他和她结果怎样了?”她心里想道。“莫非还要去见她?我干吗不问问他?不,不,和解是不可能的。即使我俩留在一个家里,也只是陌生人,永远是陌生人!”她意味深长地重复这个令她害怕的字眼。“可是我原先多么爱他,天哪,多么爱他!……我多么爱他啊!即使是现在,难道我就不爱他?难道不比从前更加爱他吗?最可怕的是……”她有了一个想法,但是没来得及想完,因为这时候马特廖娜·菲利莫诺夫娜从门外探进头来。
\par “您派人去找我兄弟来吧,”她说,“他好歹会做个饭,要不又像昨天那样,孩子们到六点钟也吃不上饭。”
\par “好吧,我马上出来安排。新鲜牛奶叫人去拿了吗?”
\par 于是,达里娅·亚历山德罗夫娜重又投身于日常的琐事中,并借此暂时排解一下她心里的悲伤。


\subsubsection*{五}

\par 斯捷潘·阿尔卡季奇凭着他良好的天赋,在学校时成绩不错,可是他疏懒顽皮,结果落到了最后几名。虽然他一向生活放纵,既无显赫头衔,也非年高德劭,他却能在莫斯科政府机关里占据一个相当体面而又薪水丰厚的官职。这个职位是通过他妹妹安娜的丈夫阿列克谢·亚历山德罗维奇·卡列宁谋得的。卡列宁在这个机关所属的部里担任要职。不过,即使卡列宁不派内兄出任这个职位,斯季瓦·奥布隆斯基也会通过上百个别的人,包括兄弟、姐妹、嫡亲、表亲、叔伯和七大姑八大姨的关系弄到这样的或类似这样的位置,年薪可以拿到六千卢布,这笔钱是他所亟需的,因为,虽然他妻子有大宗财产,他自己的事业却弄得很糟。
\par 莫斯科和彼得堡几乎有一半人是斯捷潘·阿尔卡季奇的亲戚朋友。他所出生的那个环境中,所有的人或曾经是,或后来成了达官显贵。三分之一是老一辈国家栋梁,是他的父执,从他孩提时代就认识他。另外三分之一是他的至交。还有三分之一是老熟人。因此,那些以授职、租赁、租让等形式分配世间福祉的人皆是他的朋友,是决不会漏掉他这位同道的。奥布隆斯基无需花大力气就能弄到一个肥缺,只要他不拒绝,不忌妒,不争吵,不抱怨就行,而他为人素称随和,是从来不会那样做的。假如有人对他说,他得不到他所需要的那种肥缺,他会觉得好笑,何况他的要求并不过分,他只想得到他的同龄人都能得到的东西,至于任职能力,他是不会比任何人逊色的。
\par 所有认识斯捷潘·阿尔卡季奇的人,都喜欢他善良快乐的性格和无庸置疑的诚实,而且,他那漂亮、开朗的外表,炯炯有神的眼睛,乌黑的眉毛、头发,还有白里透红的脸庞,都会对遇到他的人从生理上产生某种亲切而愉快的感染力。“啊哈!斯季瓦!奥布隆斯基!可不是他!”碰到他的人几乎总是高兴地笑着说。虽然有时跟他谈谈话也算不得什么赏心乐事,但是过一两天后再见到他时,大伙还是照样高兴。
\par 斯捷潘·阿尔卡季奇在莫斯科某机关任长官已有三年。他的同僚、下级、上司及所有跟他打过交道的人无不喜欢他,而且尊重他。斯捷潘·阿尔卡季奇能博得同事如此普遍的尊重,主要靠他的三大品质:第一,他知道自己的短处,故待人亦异常宽容;第二,他是彻底的自由主义,不是从报纸上看来的那种,而是浸透在他血液里的自由主义,他以这种态度一视同仁地对待所有人而不论其头衔大小、地位高低;第三,也是最主要的一点,他对职务上的事兴趣不大,从不过分热心,也就从不犯错误。
\par 斯捷潘·阿尔卡季奇抵达职所后,由门房恭敬地陪着,手提公文包走进他自己的小办公室,穿上制服后,再来到机关办公室。录事和职员们全都站起来,高兴而恭敬地向他鞠躬。斯捷潘·阿尔卡季奇像平时那样匆匆走向自己的座位,跟委员们一一握手,坐了下来。他很得体地说了两句笑话,就开始办公。谁都不及斯捷潘·阿尔卡季奇那样善于掌握随便、简单和公事公办之间的分寸,这种分寸是保持办公愉快氛围所需要的。一位秘书拿着公文,像机关里所有的人那样高兴而恭敬地走过来,用斯捷潘·阿尔卡季奇所提倡的自由主义亲昵语调说:
\par “我们搞到一份奔萨省府的报告。您是否要……”
\par “终于拿到了?”斯捷潘·阿尔卡季奇用手指按住一份公文说,“那么,先生们……”于是办公开始了。
\par “他们可知道,”他想,一面郑重其事地低下头听着报告,“半小时前他们的主任就像做了错事的小孩子!”别人念报告时,他的眼睛始终是笑眯眯的。办公一直持续到下午两点,然后是休息和午餐。
\par 两点钟不到,议事厅的玻璃门忽然打开,有个人走进来。委员们很高兴有了轻松一下的机会,纷纷从沙皇肖像和守法镜\footnote{顶部有双头鹰的三棱镜,为帝俄时官厅中陈设物。}下面朝门口转过头去,但是门边的守卫立刻把那人赶了出去,随后又把玻璃门关上了。
\par 公文念完之后,斯捷潘·阿尔卡季奇站起来,伸了个懒腰,为了顺应自由主义时尚,他在机关办公室里拿起一根香烟,然后向他的主任室走去。他的两位同事,老官吏尼基京和低级侍从官格里涅维奇,也随他一起走出来。
\par “我们午饭后还来得及办完,”斯捷潘·阿尔卡季奇说。
\par “当然来得及!”尼基京说。
\par “那个福明真是大滑头,”格里涅维奇提到他们所审查案件的一位当事人说。
\par 斯捷潘·阿尔卡季奇听见此话皱了皱眉,示意不应该过早下判断,但是没有回答格里涅维奇。
\par “刚才进来的那人是谁?”他问门卫。
\par “大人,一个人趁我转身的工夫溜进来,说是要见您。我告诉他,等委员们出来的时候……”
\par “他在哪儿?”
\par “大概到门厅去了,刚才一直在这儿走来走去。瞧,就是他,”门卫指着一个身板壮实、宽肩膀、鬈胡须的人说。只见那人还戴着一顶羊皮帽,正以轻快的步子踏着磨损的石阶跑上来。下台阶的人中有个提公事包的瘦官员停了下来,不以为然地望望跑上台阶的那人的一双脚,又询问似地瞥了奥布隆斯基一眼。
\par 斯捷潘·阿尔卡季奇站在台阶上面。他的脸衬着制服的绣金领子显得和蔼而有精神,当他认出闯进门的那个人是谁时,他更加容光焕发了。
\par “果然不错!莱温,你到底来了!”他带着友好而嘲弄的微笑打量迎面走来的莱温。“怎么屈驾到这穷窝里来找我呀?”斯捷潘·阿尔卡季奇说,嫌握手还不够,又吻了吻朋友。“来好久了吗?”
\par “刚到,我很想见你,”莱温答道,腼腆而又有些气恼不安地望望四周。
\par “走,上我办公室去,”斯捷潘·阿尔卡季奇说,他了解他朋友的腼腆是由于自尊心强和容易激怒,便拉住莱温的手,仿佛领他通过危险区,把他带走了。
\par 斯捷潘·阿尔卡季奇几乎对所有的熟人都以“你”相称,无论是六旬老翁、二十岁的青年、演员、部长、商人还是侍从将官都一视同仁,这样一来,在社会的最高层和最底层都有许多跟他相称尔汝的朋友,这些人一旦得知是奥布隆斯基使他们也有某种共同之处,一定会惊讶莫名。凡是跟他喝过香槟的人,他都称“你”,而他又是跟什么人都可以一起喝香槟的,所以,万一要当着下级的面会晤他那些厚脸皮的“你”们(他这样戏称他的许多朋友),凭着他特有的机灵,他懂得怎样淡化在下级心目中留下的不快印象。莱温不是厚脸皮的“你”,但是奥布隆斯基也机灵地感到,莱温一定认为他当着下级的面不愿流露他俩的亲密关系,所以连忙把他带到他的小办公室来了。
\par 莱温和奥布隆斯基年龄相若,但不是只跟他喝香槟酒的那种“你”。莱温是他少年时代的伙伴和朋友。他俩性格、爱好虽然不同,却像一对从小就要好的朋友那样互相喜爱。不过,尽管这样,他们也像选择了不同行业的人所常有的那样,彼此谈论起来固然也肯定对方的职业,其实他们心里是互相瞧不起的。他们各自觉得,唯有自己的生活才是真正的生活,而对方却在想入非非。奥布隆斯基见到莱温时,禁不住露出嘲弄的微笑。他曾多次见莱温从乡下来到莫斯科,莱温在乡下做事,但究竟何事,斯捷潘·阿尔卡季奇向来不甚了了,也不感兴趣。莱温每次来莫斯科都是情绪激动,行色匆匆,还有点不好意思,他为这不好意思感到恼火,而且大抵还要带来某种出人意料的崭新的观点。斯捷潘·阿尔卡季奇既嘲笑他也喜欢他这一点。同样,莱温打心眼里鄙视朋友的都市生活方式,还有他那些鸡毛蒜皮的公务,并讥笑这一切。所不同者,奥布隆斯基在做一般人都做的事情,所以他嘲笑人时显得平心静气而有自信,而莱温的讥笑则显得自信心不足,有时还是气呼呼的。
\par “我们早就盼望你来了,”斯捷潘·阿尔卡季奇说,走进自己的办公室后,松开了莱温的手,仿佛表示在这里危险已经过去。“见到你非常、非常高兴,”他接着说,“你怎么样?好吗?什么时候到的?”
\par 莱温没有回答,他不时望望奥布隆斯基两位同事的陌生脸孔,尤其是温文尔雅的格里涅维奇的那只手,手指又白又长,黄色的长指甲尖端朝里弯曲,还有衬衫上那些闪闪发光的大钮扣,而那双手似乎已吸引了莱温全副的注意力,弄得他不能自由地思想了。奥布隆斯基马上觉察到这一点,笑了笑。
\par “噢,让我给你们介绍一下,”他说,“我的同事:菲利普·伊万内奇·尼基京,米哈伊尔·斯坦尼斯拉维奇·格里涅维奇,”然后转向莱温:“地方自治局代表,新派地方自治人士,一手能举五普特\footnote{普特,俄国重量单位,1普特合16.38公斤。}重的体操运动员,畜牧专家,猎手,我的朋友,康斯坦丁·德米特里奇·莱温,谢尔盖·伊万内奇·科兹内舍夫的兄弟。”
\par “幸会,”那个小老头说。
\par “我有幸认识令兄谢尔盖·伊万内奇,”格里涅维奇说,伸出他那留着长指甲的纤细的手。
\par 莱温皱起眉头,冷淡地握握他的手,马上向奥布隆斯基转过身去。虽然他很敬重他的同母异父兄长,那位全俄知名的作家,但是现在,当别人只把他看成是著名的科兹内舍夫的兄弟,而不是康斯坦丁·莱温时,他简直不能忍受。
\par “不,我已经不是地方自治局代表,我跟他们吵翻了,再也不去参加地方自治局代表会议了,”他对奥布隆斯基说。
\par “这么快?”奥布隆斯基微笑说。“是怎么回事?为什么?”
\par “说来话长。以后我再告诉你,”莱温说,可是他马上开始讲起来。“简单地说,我确信没有任何地方自治活动,也不可能有,”他开始说话的样子,就像刚才有人欺侮了他,“一方面,那是个玩具,他们玩弄议会那一套,而我既不算小也不够老,不想耍弄这些玩具。另,另一方面(他口吃了一下),这是县里的coterie\footnote{法语:一帮,一伙人。}捞取钱财的工具。过去有监护机构、法院,现在有地方自治局,它们不是以受贿的形式,而是通过白拿薪水来捞钱,”他说得激昂慷慨,好像在座的人有谁会对他的意见提出异议。
\par “嘿!我看你又跨入了新阶段,保守主义的新阶段,”斯捷潘·阿尔卡季奇说。“不过这个以后再谈吧。”
\par “好吧,以后再谈。不过我有事找你,”莱温说,憎恶地盯着格里涅维奇的那只手。
\par 斯捷潘·阿尔卡季奇难以觉察地微微一笑。
\par “你不是说过,再也不穿西装了吗?”他说,一面打量着莱温那身显然是法国裁缝做的新衣服。“原来如此!我看这也是新阶段。”
\par 莱温刷地涨红了脸,不是像成年人那样微微地、不自觉地脸红,而是像小男孩那样,觉得自己腼腆得可笑,结果越加害臊和脸红,简直要哭出来了。看着这张聪明而刚毅的脸变得如此孩子气,真有些奇怪,所以奥布隆斯基不再朝他看了。
\par “我们在哪儿见面呢?我非常、非常需要跟你谈谈,”莱温说。
\par 奥布隆斯基像是考虑了一下,说:
\par “这样吧,我们上古林去吃午饭,就在那里谈谈。三点钟以前我有空。”
\par “不必了,”莱温想了想说,“我还得到别处去一趟。”
\par “也好,那就一起吃晚饭吧。”
\par “吃晚饭?其实我也没有什么特别的事,只要三言两语问一下,以后再细谈。”
\par “那么现在先说说三言两语,晚饭的时候再详谈。”
\par “三言两语是这样的,”莱温说,“不过,也没有什么特别的事。”
\par 莱温在努力克服他的腼腆,所以脸上忽然又出现了恼火的表情。
\par “谢尔巴茨基一家都在做什么?一切还照旧吗?”他说。
\par 斯捷潘·阿尔卡季奇早就知道莱温爱上了他的小姨子基季,他微微一笑,眼睛里露出愉快的神色。
\par “你说了三言两语,可是我无法用三言两语答复你,因为……对不起,稍等一下……”
\par 秘书走了进来,一副亲昵而恭敬的样子,他像所有的秘书一样,谦逊地意识到自己在办公务方面比首长懂行,拿着文件走到奥布隆斯基跟前,装作请示的样子,开始解释某个棘手的问题。斯捷潘·阿尔卡季奇没等他说完,就温和地把手按在他的袖口上。
\par “不,您就照我说过的办,”他说,一面用微笑缓和一下他的语气,随后简短地表明了他的看法,就把文件推开了:“请您照此办理,就这样吧,扎哈尔·尼基季奇。”
\par 秘书很尴尬地走了。莱温在秘书说事的时候已经完全克服了腼腆。他站在那里,把胳膊肘撑在椅背上,脸上带着专注的、讥讽的表情。
\par “我真不懂,不懂,”他说。
\par “你不懂什么?”奥布隆斯基说,仍然愉快地笑着,拿起一支香烟,等莱温说出什么乖谬的话来。
\par “我不懂你在干什么,”莱温耸耸肩膀说。“你怎么能一本正经地干这个?”
\par “为什么不能?”
\par “因为无事可做。”
\par “这是你的想法,我们可忙得不可开交呢。”
\par “埋头案牍。是呀,你有这方面的才干,”莱温说。
\par “也就是说,你认为我还缺点什么?”
\par “也许是的,”莱温说,“不过我还是欣赏你的气派,为我的朋友是如此伟大的人物而感到骄傲。不过你还没有回答我的问题,”他说完后,竭力直视着奥布隆斯基的眼睛。
\par “哦,好了,好了。等着瞧吧,你也会到这一步的。你在卡拉津县有三千俄亩\footnote{1俄亩等于1.09公顷。}土地,这该多好。你这么肌肉发达,你容光焕发得像个十二岁的小姑娘,可是你也肯定会落到我们这一步的。至于你打听的事情,告诉你:情况没有变化,只可惜你好久都没来了。”
\par “怎么了?”莱温惊恐地问道。
\par “没什么,”奥布隆斯基说。“这事我们再谈吧。你这次来究竟为了什么事?”
\par “唉,这个也以后再谈吧,”莱温说,他的脸又红到了耳根。
\par “那好吧。我明白了,”斯捷潘·阿尔卡季奇说。“你瞧,本来我想叫你上我家去,可是妻子身体不大好。我看这样吧,你若是想见他们,这会儿他们大概正在动物园,从四点待到五点。基季在溜冰。你先坐车去那儿,回头我也去,带你一道找个地方吃晚饭。”
\par “好极了,那就再见。”
\par “你可当心,我了解你,你会把说好的事情忘了,要不就突然跑回乡下去!”斯捷潘·阿尔卡季奇笑着大声说。
\par “决不会的。”
\par 莱温走出办公室,到了门口才想起来,他忘了向奥布隆斯基的两位同事道别。
\par “看样子,这位先生精力很充沛,”莱温出去后,格里涅维奇说。
\par “是啊,老兄,”斯捷潘·阿尔卡季奇摇着头说,“他真是个幸运儿!在卡拉津县有三千俄亩土地,前程远大,而且多么有朝气!可不像我们这班人。”
\par “怎么您也抱怨起来了,斯捷潘·阿尔卡季奇?”
\par “糟糕啊,糟透了,”斯捷潘·阿尔卡季奇重重地叹了口气说。

\subsubsection*{六}

\par 奥布隆斯基问莱温,他究竟为何事而来,他脸红了,并且为脸红生自己的气。他不好意思对奥布隆斯基说:“我是来向你小姨子求婚的,”虽然他正是专为此事而来的。
\par 莱温和谢尔巴茨基两个家族都是莫斯科古老的贵族世家,两家关系一向亲密。在莱温上大学时这种交谊更加深厚了。莱温曾与年轻的谢尔巴茨基公爵、即多莉和基季的哥哥,一起温习功课并同时考进大学。他经常出入谢尔巴茨基家并爱上了这一家人。看似奇怪,可是康斯坦丁·莱温确实爱上了这个家庭,爱上了这一家子,特别是这一家的女人们。莱温已不记得自己的母亲,他只有一个年长的姐姐,所以,他是在谢尔巴茨基家中,第一次领略到他因父母双亡而失去的那种有教养而待人诚挚的贵族世家的生活氛围。这一家所有的人,尤其是女人们,都让他觉得是笼罩在一片神秘的诗意帷幕里。他不但看不到她们有任何缺点,而且在这诗意帷幕的笼罩下,设想她们具有最崇高的情感和绝对完美的品质。为什么这三位小姐要今天说法语,明天说英语?为什么她们在一定的时间轮流弹钢琴,琴声传到楼上她们哥哥的房间时,有两位大学生正在这里做功课?为什么这些教师要到家里来上法国文学、音乐、绘画和舞蹈课?为什么三位小姐要和Mlle Linon\footnote{法语:林农小姐。}在一定的时间坐马车去特维尔林荫道,还要穿上绸缎面子的外套,多莉穿长的,娜塔莉穿半长的,基季的外套很短,把她那双匀称的、给红色长袜绷得紧紧的小腿都露在外边?为什么她们要在一个有金色帽徽的仆人的伴随下,在特维尔林荫道上来回漫步?所有这些,还有在她们的神秘世界中发生的许多别的事情,他都弄不明白,但是他知道这一切都是美好的,他就沉醉在这种神秘感之中。
\par 莱温上大学时差一点爱上大姐多莉,但不久多莉就嫁给了奥布隆斯基。后来他开始钟情于二姐。他似乎觉得,他应该爱上三姊妹中的一个,只是他弄不清楚究竟该爱哪一个。娜塔莉也是刚进入社交界不久,就嫁给了外交官利沃夫。莱温大学毕业时基季还是个孩子。年轻的谢尔巴茨基投身海军后,在波罗的海淹死了。虽然莱温和奥布隆斯基是朋友,但他与谢尔巴茨基家的关系就此变得疏远了。今年初冬,莱温在乡下住了一年之后,再次来到莫斯科。他见到了谢尔巴茨基一家人,这时他终于明白了,三姊妹中他注定要爱的究竟是哪一个。
\par 他这位出身名门、算得上富有的三十二岁男子,如果开口向谢尔巴茨基公爵小姐求婚,这事看起来最容易不过了,他很可能立即被认为是门当户对的人选。然而,堕入情网的莱温觉得,基季是个十全十美的女子,超过了一切世间之人,而自己则是个卑微的凡夫俗子,简直难以想象,别人和她本人会认为他配得上她。
\par 为了见到基季,他开始出入交际场合,几乎每天在那里和她照面,他就这样神魂颠倒地在莫斯科过了两个月,忽然断定此事决无成功的可能,就回到乡下去了。
\par 莱温确信这件事不可能的理由是,在女方亲属的心目中他配不上迷人的基季,这样她太不上算,而基季本人也是不会爱他的。
\par 在亲属们看来,他在上流社会没有熟练、固定的职业及地位;他今年三十二岁,而他的同学们现在有的是上校、侍从武官,有的是教授,有的担任银行、铁路的经理,有的像奥布隆斯基那样当上了政府机关的头头。而他(他很清楚别人对他的印象)则不过是个地主,只会养养母牛,打打鸻鸟,搞搞建筑,是个没有出息的庸才,按照社会上的看法,他干的尽是些无能之辈所干的事。
\par 神秘而迷人的基季不会爱上这种其貌不扬(他自认为如此),特别是这种普普通通、毫不出众的人。此外,他觉得,过去他是基季哥哥的朋友,对她就像大人对待孩子一样,这种关系会成为爱情的又一个障碍。他认为像他这样其貌不扬的老好人,只可以作为朋友来爱,要获得他爱基季那样的爱情,则必须是个美男子,尤其必须是个不同凡响的人。
\par 他听说女人常常会爱上不漂亮的普通男人,他不相信,因为从他自己来判断,他只爱漂亮、神秘而特别的女人。
\par 他独自在乡下待了两个月后,确信这一次的爱情与少年时代的那两次不同。这一次的感情不给他片刻安宁。一天不解决她是否做他的妻子这个问题,他就无法生活下去。他的绝望只是他自己的想象所致,他没有任何根据断定他的求婚将遭到拒绝。所以,他此次到莫斯科来是下定了决心来求婚的,要是对方答应,他就结婚。要是……他简直不敢设想,如果他遭到拒绝,他将会怎么样。


\subsubsection*{七}

\par 莱温乘上午的火车到达莫斯科,在他异父同母的哥哥科兹内舍夫家里落脚。他换好衣服就到书房来见哥哥,想立即告知他此行的目的并征求他的意见,但是书房里有客。哥哥的客人是一位著名的哲学教授,他专程从哈尔科夫赶来,为的是解释一下他俩在某个重大哲学问题上发生的误会。教授在和唯物论者进行激烈的辩论,谢尔盖·科兹内舍夫很关注这场论战,他看了教授最近发表的一篇论文后,写信给他表示不同意见,指责他对唯物论者作了太多的让步。教授立即赶来和他统一看法。他们谈的是一个时髦问题:在人类活动中有没有心理现象和生理现象之间的界线,如果有,它在哪里?
\par 谢尔盖·伊万诺维奇见到弟弟,就像对所有人那样,脸上露出亲切然而冷淡的微笑。他把莱温和教授作了介绍,然后继续谈话。脸色蜡黄、额头狭窄、身材矮小、戴着副眼镜的教授略为停了停,向莱温道了声好,又接着说下去,不再理会他。莱温坐在一旁等教授走,可是不久他也对谈话内容发生了兴趣。
\par 莱温曾在杂志上见到过他们所谈的那些论文,而且饶有兴趣地阅读过。作为一名自然学科大学生,他想从中了解他所熟悉的自然科学原理的发展状况。但是,他从来没有把作为动物的人类的起源以及反射作用、生物学和社会学方面的科学结论与生和死的意义问题联系起来,而这些问题近来越来越经常地引起他的思考。
\par 从哥哥和教授的谈话中他注意到,他们常常把科学问题与精神的问题联系在一起。好几次他们几乎要谈到精神问题,可是每当他们接近他认为是最重要的内容时,他们便匆匆避开,重新陷入到繁琐分类、补充说明、暗示和引经据典中去,所以他很难弄懂他们究竟在讨论什么。
\par “我不能容许,”谢尔盖·伊万诺维奇以他素有的明确的表达方式和悠扬悦耳的口音说,“我决不能同意凯斯的意见,也认为我对外部世界的概念完全来自印象。我并非通过感觉获得关于存在的最基本概念,因为没有专门的器官来传达这个概念。”
\par “不错,可是武尔斯特、克瑙斯特,还有普里帕索夫,他们能回答您,说您的存在意识是由全部感觉的总和而来,这种存在意识就是感觉的结果。武尔斯特甚至直截了当地说,倘若没有感觉,也就没有存在的概念。”
\par “我的意见相反,”谢尔盖·伊万诺维奇又开始说。这时莱温再次感到,他们在接触到核心问题时又开始离题了,于是决定向教授提一个问题。
\par “如此说来,倘若我的感觉消灭了,倘若我的躯体死亡了,就不可能有存在了吗?”
\par 教授被人打断了话头似乎有些恼火,扫兴地望望这位奇怪的提问者,见他不像是搞哲学的,倒更像是个纤夫,便转眼望着谢尔盖·伊万诺维奇,仿佛在问他:叫我可怎么说呢?谢尔盖·伊万诺维奇决不像教授那样偏激,而是有分寸地既回答教授的话,也考虑莱温提问中那朴实自然的想法。他笑笑说:
\par “这个问题我们还无权回答……”
\par “我们没有资料,”教授证实道,接着又阐述他上面提到的论据,“不,我要指出一点,既然如普里帕索夫直接说的,印象是感觉的基础,那么我们就要严格区分这两个概念。”
\par 莱温不再听下去,只等教授赶快走。


\subsubsection*{八}

\par 教授走后,谢尔盖·伊万诺维奇对弟弟说:
\par “你来了我很高兴。能待久吗?农场怎么样?”
\par 莱温知道哥哥对农场兴趣不大,只不过顺便问一句,所以他也只说了说卖麦子和钱财上的事。
\par 莱温原想把结婚的打算告诉哥哥,并听听他的意见,他甚至下定了决心就这么做。可是当他见到哥哥,听了他同教授的谈话,又听到他不自觉地用保护人的口气询问农场的情况(母亲留给他俩的地产没有分家,都由莱温掌管)时,不知为什么,他觉得结婚的事难以向哥哥启齿。他感到哥哥对此事的看法不会符合自己的初衷。
\par “你们的地方自治局怎么样,情况如何?”谢尔盖·伊万诺维奇问道,他对地方自治局很感兴趣,认为它意义重大。
\par “噢,其实,我也不清楚……”
\par “怎么?你不是执行委员吗?”
\par “不,现在不是了,我退出了,”莱温答道,“再也不去开会了。”
\par “多可惜!”谢尔盖·伊万诺维奇皱起眉头说。
\par 为了替自己辩解,莱温开始讲述县地方自治代表会议上发生的情况。
\par “事情总是这样!”谢尔盖·伊万诺维奇打断他说。“我们俄国人永远是这样。或许这是我们的优点——能看到自己的缺点,但是往往做过了头,专以冷嘲热讽为乐,张口就是挖苦。我只告诉你,要是把我们这样的地方自治权赋予别的欧洲民族,譬如德国人和英国人,他们就能从中培养出自由来,而我们却只会挖苦嘲笑。”
\par “可是有什么办法呢?”莱温歉疚地说。“这是我最后的尝试。我真心实意地试过了。我不行。我没有能力。”
\par “不是你没有能力,”谢尔盖·伊万诺维奇说,“而是你对事情的看法不对头。”
\par “也许吧,”莱温沮丧地说。
\par “噢,你可知道,尼古拉弟弟又到这儿来了。”
\par 尼古拉是康斯坦丁·莱温的亲哥哥,谢尔盖·伊万诺维奇的又一个异父同母的弟弟。此人已经完全堕落,他把自己的财产挥霍殆尽,成天在荒唐污秽的圈子里鬼混,跟两个兄弟都吵翻了。
\par “你说什么?”莱温惊惧地叫道。“你是怎么知道的?”
\par “普罗科菲在街上见到过他。”
\par “在这里,在莫斯科吗?现在他在哪儿?你知道吗?”莱温从桌边站起来,好像马上就要走。
\par “我不该把这事告诉你,”谢尔盖·伊万诺维奇说,看见小弟弟不安的样子,他不住地摇头。“我派人打听到他的住处,把借据送还给他,向那个特鲁宾如数付了钱。这是他给我的回信。”
\par 谢尔盖·伊万诺维奇从吸墨器下面抽出一张字条,递给弟弟。
\par 莱温看见纸条上几行怪异而亲切的笔迹这样写道:“恳请别来打扰我。这是我对两位贤兄弟的唯一要求。尼古拉·莱温。”
\par 莱温看了这几行字,仍然低着头,两手拿着纸条站在谢尔盖·伊万诺维奇的面前。他在作思想斗争,他想现在就忘掉这个不幸的哥哥,又意识到这样做很不好。
\par “他显然是想侮辱我,”谢尔盖·伊万诺维奇说,“但是他侮辱不了我。本来我很想帮助他,现在我知道这是办不到的。”
\par “是的,是的,”莱温连连说。“我理解你也佩服你这样对待他,不过我还得去找他。”
\par “你实在想去就去一趟,但我还是劝你别去,”谢尔盖·伊万诺维奇说。“至于涉及到我,我倒不用担心,他不可能挑唆你和我吵架。为你着想,劝你还是不去为好。他这个忙是帮不上的。但话又说回来,你想干什么就干什么吧。”
\par “也许是帮不了他,但我觉得,特别是眼下这种时候,哦,这是另外一码事……总之我觉得,我于心不安。”
\par “这我就不懂了,”谢尔盖·伊万诺维奇说。“不过我明白一点,”他又补充道,“要学会克制自己。自从尼古拉弟弟变成现在这个样子,我开始用不同的眼光,比较宽容地看待所谓卑鄙了……你知道他都干了些什么……”
\par “唉,这真可怕,可怕!”莱温连连说。
\par 莱温向谢尔盖·伊万诺维奇的仆人打听到尼古拉的地址,准备马上去见他,但考虑了一下,决定等到下午再去。为了使心情平静下来,首先要解决他为之来到莫斯科的那件事。莱温从哥哥家来到奥布隆斯基的机关,在这里打听到谢尔巴茨基一家的情况,然后根据别人的指点,坐马车前往他可能见到基季的地方。


\subsubsection*{九}

\par 下午四点,莱温在动物园门口下了马车,感到心跳得厉害。他顺着小路向小山边的溜冰场走去,知道在那里准能找到她,因为他看见入口处停着谢尔巴茨基家的马车。
\par 天气晴冷。溜冰场门口停放着一排排轿式马车、雪橇、载客马车,还站着不少宪兵。在进口处,在打扫干净的小道上,在雕花屋脊的俄式小木屋之间,处处都是穿戴整洁的人群,礼帽在明丽的阳光下闪闪发亮。花园里密匝匝的老白桦树被雪压弯了枝条,仿佛穿着节日的新装。
\par 他顺着甬道走向溜冰场,一面自言自语道:“不能激动,要镇静。”“你说什么?你怎么了?住嘴,蠢东西!”他甚至这样对自己的心灵说话。他越是想镇静,越是紧张得透不过气来。一位熟人迎面而过,叫了他一声,他也没认出来是谁。快走到小山时,传来了升降小雪橇的铁链的铿锵声,雪橇滑行的刷刷声和一阵阵的欢声笑语。他又走了几步,溜冰场就展现在面前,他立刻在溜冰的人群中认出了她。
\par 他看见她就在这里,感到既高兴又害怕。她站在溜冰场的那一头,在跟一位太太说话。她的衣着和姿态并无特别显眼之处,但莱温在人群中一眼就认出了她,好像在荨麻丛中看见一朵玫瑰花似的。一切都因她而大放光彩。她是给四周充满欢愉的微笑。“难道我能走到冰上去,到她跟前去吗?”他想。她站立的那个地方,他觉得是无法到达的圣地,他害怕起来,差一点回头走掉。他得控制自己,冷静地想一想,既然她身边什么样的人都有,他当然也可以到那边去溜冰。他走上了溜冰场,却不敢老盯着她,就像不能长时间地望着太阳,但即使不望她也能像看见阳光一样看到她。
\par 每星期的这一天,在这个时间聚集到溜冰场来的,都是同一个社交圈子里的人,大家彼此相识。其中有喜欢当众出出风头的溜冰好手,有胆怯而笨拙地扶着椅子练习的初学者,有小孩子,也有为健身而溜冰的老人。莱温觉得这些人都是难得的幸运儿,能在这里待在她的左右。这些溜冰者显然都很随便地和她相互追逐,甚至跟她说说话,全然不受她的影响而尽情地自得其乐,享受这美妙的冰面和晴朗的天气。
\par 基季的堂弟尼古拉·谢尔巴茨基坐在长凳上。他穿着短上衣和紧身裤,脚上穿着溜冰鞋,看见莱温便朝他喊起来:
\par “喂,俄国最棒的溜冰好手!来了好久了吗?冰场挺不错的,快穿上冰鞋吧!”
\par “我连冰鞋也没有,”莱温回答,在她面前竟这么大胆放肆地说话,连他本人都感到惊讶。他眼睛不望她,却丝毫没有让她离开过视线。他感到太阳在向他接近。她正滑到转弯处,穿着高统冰鞋的瘦小的双脚踩在冰面上,小心翼翼地朝他这边滑过来。一个穿俄式衣衫的男孩子,把身体俯向冰面,拼命摆动着双臂要超过她。她滑得不太平稳,把手从吊在带子上的皮手筒里抽了出来,随时防备摔倒。这时她看见了莱温,认出了他,向他微笑,也笑自己这么害怕跌跤。她完成了转弯动作,富有弹性的小腿在冰面上一蹬,照直滑到堂弟面前,一把抓住了他,笑着向莱温点点头。她的美丽超过了他的想象。每当他想念她时,他能生动地想象出她的全貌,尤其是那长着淡黄秀发的可爱的脑袋,在匀称的少女肩上左顾右盼,让人感到一种孩子般的清纯。天真的脸部表情加上优美的体态,使她具有特殊的魅力,这一点他清楚地记得。而每每令人意外惊喜的,是她那温柔、安详而诚实的眼神。特别是她的微笑,总是把莱温带进奇幻的世界,令他如醉如痴,仿佛回到了童年时代难得的美妙时光。
\par “您来这儿好久了吗?”她向他伸出手说。“多谢,”她又说,这时他捡起了从她手筒里掉下来的手帕。
\par “我吗?不久前,我是昨天……噢,是今天……才到的,”莱温答道,由于激动他没有马上听懂她的问话。“我本想上您家里去,”说到这里他立即想起了来找她的意图,心里一慌,脸就红了。“我不知道您爱溜冰,还溜得这么好。”
\par 她凝视了他一会,似乎想弄明白他为何发窘。
\par “难得您的夸奖。这里一直传说您是位溜冰好手,”她说,一面用戴着黑手套的小手拂去手筒上的霜花。
\par “是的,我过去对溜冰很热衷,想达到尽善尽美的水平。”
\par “好像您做什么事情都很热衷,”她微笑着说。“我很想看看您溜冰。穿上冰鞋,我们一起溜吧。”
\par “跟她一起溜冰?难道这可能吗?”莱温望着她心里想。
\par “我这就去穿,”他说。
\par 莱温去穿溜冰鞋。
\par “先生,您好久没上我们这儿来了,”溜冰场工人对莱温说,一面托着他的腿,帮他旋紧冰鞋的后跟。“您走了以后再也没有好手了。看这样行吗?”工人抽紧冰鞋上的皮带,问道。
\par “行,行,请您快一点,”莱温说,勉强克制着脸上憋不住的幸福笑容。他想:“是啊,这就是生活,这就是幸福!她说了:让我们一起溜冰吧。现在就对她说吗?我害怕,因为现在我是幸福的,哪怕只是怀着幸福的希望也好……那么以后呢?……应该说!应该,应该!胆怯什么!”
\par 莱温站起来,脱去外套,在小屋旁边粗糙的冰面上助跑了一段,就驰到光滑的冰场上。他溜起来毫不费力,随心所欲地忽快忽慢和改变方向。他溜到她跟前时心里还在发虚,但她的微笑又使他放心了。
\par 她向他伸出一只手,他们并肩溜了起来,随着速度的加快,她把他的手握得越来越紧。
\par “和您一起我能学得快些,不知怎的,我很信赖您,”她对他说。
\par “您信赖我,也使我对自己有了信心,”这话一出口他就吓坏了,脸也涨得通红。果然,他刚说完这句话,她脸上的亲切表情顿时完全消失,仿佛乌云遮住了太阳。这时莱温看到了他所熟悉的脸部变化:当她思考问题时,光洁的额头上就出现一道皱纹。
\par “您没有不高兴的事吧?不过,我也无权过问,”他急忙说。
\par “为什么?……不,我没有任何不高兴的事,”她冷淡地答道,马上又说:“您没有去看林农小姐吗?”
\par “还没有。”
\par “去见见她吧,她那么喜欢您。”
\par “这是怎么了?我得罪她了。上帝啊,帮助我!”莱温这样想,就向坐在长凳上的一位鬈发灰白的法国老妇溜过去。而她像遇见老朋友那样向他微笑,露出了一口假牙。
\par “是啊,我们有的长大了,”她用眼睛示意基季,对他说,“有的变老了。tiny bear\footnote{英语:小熊。}长成大熊了!”法国女人笑着继续说,向莱温提起了他曾把三位小姐比作英国童话里三只熊的那个笑话。“还记得吗,这是您说的?”
\par 他全然不记得这件事了,而她十年来却常常为了这个笑话发笑,觉得很有意思。
\par “好了,去吧,去溜冰吧。我们的基季已经溜得挺好了,不是吗?”
\par 莱温又回到基季身边时,她的脸色已不再严肃,目光又变得诚挚可亲,但是莱温觉得,在她的亲切态度中有一种故作镇静的特别的神情。他感到有些怅然。
\par 基季谈了谈她年老的女家庭教师的种种怪癖,然后询问莱温的生活情况。
\par “冬天您待在乡下难道不寂寞吗?”她问。
\par “不,不寂寞。我很忙,”他说。他感到她在迫使他适应这种平静的调子,而他又像初冬时那样,无法从这种调子中挣脱出来。
\par “您这次来能多住些日子吗?”基季问他。
\par “我不知道,”他心不在焉地回答。这时他有了一个想法:如果他屈服于她这种平静友好的调子,他又将一事无成地回到乡下去。他愤然地下定了决心。
\par “您怎么会不知道呢?”
\par “不知道。这要取决于您,”他这样说,立刻被自己的话吓了一跳。
\par 不知是没有听见他的话,还是不愿意听,她像要绊倒似的,用脚在冰上磕了两下,连忙从他身边溜开了。她溜到林农小姐那里,对她说了几句话就到女士们脱冰鞋的小屋里去了。
\par “天哪,我干的好事!我的上帝!帮帮我,教教我吧!”莱温说,他祈祷着,这时他忽然想做一个猛烈的动作,就在冰场上疾驰起来,划出了一道道的圆圈。
\par 这时候,一个溜冰娴熟的年轻人,嘴里叼着香烟从咖啡室走出来。他穿着冰鞋直接跑向台阶,从上面急滑而下,冰鞋在阶梯上不住地颠跳,碰出咚咚的响音。他飞驰而下时甚至没有改变双臂的自由姿势,接着就在冰场上溜起来。
\par “嗬,这可是新玩艺儿!”莱温立即跑上台阶去试试这个新玩艺儿。
\par “当心摔着,您没练习过!”尼古拉·谢尔巴茨基向他喊道。
\par 莱温站到台阶上,从那里猛跑几步疾驰而下,双臂不习惯地尽力保持着平衡。滑到最后一级时他绊了一下,一只手几乎触到了冰面,他马上使劲稳住身体,笑着向前滑去。
\par “好样的,亲爱的,”基季心想,这时她和林农小姐从小屋里走出来,带着平静的亲昵的微笑望着他,就像望着一位可爱的兄长。“莫非我有什么过错?我做了什么不好的事吗?别人说我卖弄风情。我知道他不是我的所爱,可是跟他在一起我仍然很愉快,他人又这样好。但是,他为什么说那种话呢?……”她在想。
\par 莱温由于急速的运动脸涨得通红,这时他看见基季要走,她母亲正到台阶上来接她,就停止了溜冰,沉吟起来。他脱下了冰鞋,在动物园门口赶上了母女俩。
\par “很高兴见到您,”公爵夫人说。“和往常一样,我们每逢星期四接待客人。”
\par “也就是今天?”
\par “很高兴在舍下见到您,”公爵夫人冷冷地说。
\par 这种语气令基季感到不快,她忍不住想缓和一下母亲的冷淡态度,就转过头来微笑着说:
\par “再见吧。”
\par 这时,斯捷潘·阿尔卡季奇斜戴着礼帽,目光炯炯、容光焕发地走进动物园来,就像个喜气洋洋的胜利者。他走到岳母跟前,脸上露出忧愁和负疚的神情,回答了她关于多莉健康状况的几个问题。他沮丧地和岳母低语了一会,就挺起胸膛,挽住莱温的胳膊。
\par “那么,我们是不是就走?”他问。“我一直在想你的事,真高兴你来了,”他意味深长地望着他的眼睛说。
\par “我们走,我们走,”莱温说。他感到幸福,耳边一直回响着那声“再见”,眼前浮现出说这句话时的笑容。
\par “上‘英吉利’饭店,还是‘埃尔米塔日’饭店?”
\par “我无所谓。”
\par “那就到‘英吉利’吧,”斯捷潘·阿尔卡季奇说。他选择“英吉利”是因为他在那边比在“埃尔米塔日”欠的账多些,觉得不去不大好。“你有马车吗?太好了,我的车打发回去了。”
\par 两个朋友一路无话。莱温在捉摸基季脸上表情的变化是什么意思。他忽而觉得大有希望,忽而又灰心丧气,明白自己的指望是不理智的,但他又感到自己完全变了样,跟看见那嫣然一笑、听见那声“再见”之前判若两人。
\par 斯捷潘·阿尔卡季奇一路上在考虑晚餐的菜谱。
\par “你不是爱吃比目鱼吗?”车到饭店时,他对莱温说。
\par “什么?”莱温反问道。“比目鱼?对,比目鱼我喜欢极了。”

\subsubsection*{十}

\par 莱温跟着奥布隆斯基走进饭店,注意到他的脸上乃至全身都有一种特别的神气,像是喜滋滋按捺不住的样子。奥布隆斯基脱下外套,歪戴着礼帽走进餐厅,向几个身穿燕尾服、手拿餐巾围着他打转的鞑靼人吩咐些什么。像在别处一样,这里也有熟人欢迎他。他不住地左右点头,走到小吃柜台边,就着鱼肉喝了杯伏特加酒。柜台后面坐着个满头鬈发、浓妆艳抹的法国女人,衣服上扎着许多带子,镶着许多花边,他对她说了几句什么话,逗得她开心地笑了。这个法国女人使莱温恼火。看样子,她整个儿是用别人的头发加上poudre de riz和vinaigre de toilette\footnote{法文:米粉调的香粉和化妆用的醋。}做成的。就为这个,他连伏特加也没喝,像避开脏地方那样,连忙从她那里走开了。他的整个心灵都充满了对基季的回忆,眼睛里流露出得意和幸福的微笑。
\par “大人,您这边请,这里没有人打扰,大人,”花白头发的老鞑靼人特别殷勤地说。由于盆骨宽大,燕尾服的后襟在他臀部上面就分岔了。“请吧,大人,”他对莱温说。为了表示对斯捷潘·阿尔卡季奇的尊敬,他也殷勤招待他的客人。
\par 一转眼工夫,鞑靼人在青铜烛吊架下面那张已经铺有台布的圆桌上又加了一块台布。他推过来几张丝绒面子的椅子,拿着餐巾和菜单站在斯捷潘·阿尔卡季奇面前,等候他吩咐。
\par “大人,您想要单间的话,马上就有空,戈利岑公爵和一位夫人已经用完。新鲜的牡蛎到货了。”
\par “啊,牡蛎。”
\par 斯捷潘·阿尔卡季奇在考虑。
\par “要不要改变计划,莱温?”他的手指停在菜单上,脸上露出煞有介事的犹豫神色。
\par “牡蛎是新鲜的吗?你可得仔细了!”
\par “是弗伦斯堡\footnote{德国城市。}的牡蛎,大人,没有奥斯坦德\footnote{比利时城市。}的。”
\par “弗伦斯堡的也罢,新鲜吗?”
\par “昨天刚到的货。”
\par “那么,就先来牡蛎,干脆把整个计划都改了吧?你看呢?”
\par “我随便。最好给我来点菜汤和粥,可是这里没有。”
\par “您想要俄式粥吗?”鞑靼人像保姆俯在小孩身上那样问莱温。
\par “不,说正经的,你点的就好。我刚溜过冰,有点饿了。”他发现奥布隆斯基脸色有些不快,就补充道:“别以为我不喜欢你点的菜。我很乐意好好吃一顿。”
\par “可不是!不管怎么说,这也是一种生活乐趣,”斯捷潘·阿尔卡季奇说,“那么,伙计,你就上二十个,不够,上三十个牡蛎,再来个蔬菜汤吧……”
\par “新鲜菜,”鞑靼人跟着用法语说。可是斯捷潘·阿尔卡季奇显然不想让他卖弄法语菜名。
\par “蔬菜汤,你知道吗?然后就上浓汁比目鱼,然后是……干炸牛里脊,注意,要好的。再来个阉鸡怎么样?还要些罐头水果。”
\par 鞑靼人想起了斯捷潘·阿尔卡季奇不按法国菜单点菜的习惯,就不再跟着他一一核对菜名,而是把全部点好的菜最后用法语照单再念一遍。接着,他像从弹簧上蹦起来似的,飞快地把这份菜单放下,又抓过一张酒单,呈到斯捷潘·阿尔卡季奇面前。
\par “喝什么酒呢?”
\par “随你的便,我只能喝一点儿,就来香槟吧,”莱温说。
\par “怎么?开始就喝香槟?不过,也许你是对的。你喜欢白商标的吗?”
\par “白标,”鞑靼人跟着说。
\par “先上这种酒和牡蛎,其余的再说。”
\par “遵命。葡萄酒您要哪一种?”
\par “上点纽伊葡萄酒。不,最好还是老牌沙勃利白葡萄酒。”
\par “遵命。再来点您爱吃的干酪?”
\par “好的,帕尔马干酪。你喜欢另外一种吗?”
\par “不,我都行,”莱温说。他脸上忍不住又露出了微笑。
\par 鞑靼人摆动着燕尾服后襟飞快地走了。五分钟后他又奔了进来,托着一盘贝壳张开的牡蛎,手指间夹着一瓶酒。
\par 斯捷潘·阿尔卡季奇把浆过的餐巾揉揉软,巾角塞在背心里,把手摆得舒服些,就开始吃牡蛎。
\par “味道不错,”他用银餐叉把滑腻的牡蛎肉从贝壳里挖出来,一个接一个吞下去。“味道不错,”他又说,那双湿润发亮的眼睛看看莱温,又看看鞑靼人。
\par 莱温也吃牡蛎,不过他更喜欢白面包夹干酪。他在欣赏奥布隆斯基的吃相。就连那个鞑靼人也一面开瓶塞,把冒着泡的香槟酒倒进细长的高脚杯,一面带着得意的微笑理理他的白领结,不时望一眼斯捷潘·阿尔卡季奇。
\par “你不大爱吃牡蛎吧?”斯捷潘·阿尔卡季奇边喝酒边说,“要不,你就有什么顾虑,啊?”
\par 他想让莱温高兴些。莱温也不是不高兴,只是感到不自在。以他此时的心情,坐在这家饭店里,前后都有人陪女士在包间吃喝,周围一片嘈杂和忙乱,使他觉得又难受又尴尬。这个尽是青铜器皿、镜子、汽灯和鞑靼人的环境使他十分恼火。他唯恐洋溢在他心头的那一团情愫被玷污了。
\par “我吗?是的,我有顾虑,而且这里的一切都让我感到不自在,”他说。“你想象不出,我这个乡下人对这些东西多么不习惯,就像看到你机关里那位先生的手指甲……”
\par “是的,我看见可怜的格里涅维奇的指甲使你很感兴趣,”斯捷潘·阿尔卡季奇笑着说。
\par “我受不了,”莱温说。“你尽量设身处地从我的角度,一个乡下人的立场想一想。我们在乡下总是尽量让双手干活方便些,所以把指甲剪短,有时还捋起袖子。可是这里的人故意留指甲,尽量留得长长的,袖子上的钮扣也大得像个小碟子,结果一双手什么事也不能做。”
\par 斯捷潘·阿尔卡季奇愉快地微笑着。
\par “对,这就是标志,表示此人不需要干粗活。他是劳心的人……”
\par “也许吧。不过我还是不习惯,就像现在吧,我们乡下人要快些吃饱肚子,好去干自己的活,可是我们俩却在尽量磨蹭着不让肚子饱,所以就来吃牡蛎……”
\par “理当如此,”斯捷潘·阿尔卡季奇接着说。“这正是文明的目的所在:从各种事情中获得乐趣。”
\par “如果是这样的目的,我宁可做个野蛮人。”
\par “你够野蛮的了。你们莱温家的人都野蛮。”
\par 莱温叹了口气。他想起了尼古拉哥哥,感到羞愧和痛苦,不禁皱起了眉头。但是奥布隆斯基又谈起别的话题,打断了他的思路。
\par “今晚你上我们那边去,也就是上谢尔巴茨基家去,怎么样?”他说,一面推开吃光了肉的粗糙的空贝壳,拿过干酪,闪闪的目光意味深长。
\par “好,我一定去,”莱温回答。“虽然我觉得,公爵夫人叫我去有些勉强。”
\par “瞧你说的!胡扯!她那是搭架子……喂,伙计,上汤吧!……这是她grande dame\footnote{法语:贵妇人。}的架子,”斯捷潘·阿尔卡季奇说。“我也去,不过我得上巴宁伯爵夫人家去排练合唱。你还不算野蛮吗?你忽然从莫斯科消失了,又作何解释呢?谢尔巴茨基家人不断向我打听你的消息,好像我一定知道似的。而我只知道一点:你总是做别人都不做的事。”
\par “是啊,”莱温拖长声音激动地说。“你说得对,我野蛮。不过,我的野蛮不在于上次我走了,而在于这次我来了。现在我来……”
\par “哦,你真是幸运儿!”斯捷潘·阿尔卡季奇望着莱温的眼睛,接过来说。
\par “怎么样?”
\par “看烙印知道哪一匹是烈马,看眼睛知道小伙子爱上她,”斯捷潘·阿尔卡季奇背诵了两句诗。“你前程远大啊。”
\par “难道你日暮途穷了?”
\par “不,虽说不至于日暮途穷,可是未来是属于你的,而我却只拥有现在,还弄得乱糟糟的。”
\par “是怎么回事呀?”
\par “情况不妙。我不想谈自己的事,有些事情也解释不清,”斯捷潘·阿尔卡季奇说。“你究竟为了什么事到莫斯科来?……喂,收拾一下!”他叫鞑靼人。
\par “你猜到一点了?”莱温说道,他那目光深邃的眼睛一直盯着斯捷潘·阿尔卡季奇。
\par “猜到一点,但这件事不能由我先开口。就凭这一点你能看出我猜得对不对,”斯捷潘·阿尔卡季奇带着含蓄的微笑望着莱温说。
\par “你到底要告诉我什么?”莱温声音发颤地说,他感到脸上的肌肉都在哆嗦。“你对这事有什么看法?”
\par 斯捷潘·阿尔卡季奇慢慢喝干杯子里的沙勃利葡萄酒,眼睛一直望着莱温。
\par “我吗?”他说,“这符合我最好的愿望,最好的。倘能如此,再好不过。”
\par “你不会搞错吧?你知道我们在谈什么事吗?”莱温两眼死死盯住对方说。“你认为这有可能吗?”
\par “我想有可能。为什么不可能呢?”
\par “不,你仔细想想,这是可能的吗?不,你把你的想法全都告诉我!如果,如果我遭到拒绝呢?……我甚至确信……”
\par “为什么你要这样想?”见他如此激动,斯捷潘·阿尔卡季奇微笑着说。
\par “有时我有这种感觉。这对我对她都太可怕了。”
\par “至少对姑娘来说,这没有什么可怕。求婚会使任何一位姑娘感到骄傲。”
\par “是的,任何一位姑娘,但不是她。”
\par 斯捷潘·阿尔卡季奇微微一笑。他很了解莱温这种感情。现在对他来说,天下的姑娘分为两类,一类是除她以外的所有姑娘,她们具有一切人类弱点,是极其平凡的姑娘;另一类只有她一个人,没有任何弱点,胜过人间一切。
\par “等一下,你加点酱油,”他止住莱温推开酱油瓶的手说。
\par 莱温顺从地倒了点酱油,但他不让斯捷潘·阿尔卡季奇吃菜。
\par “不,你等等,等等,”他说。“你要明白,这对我是生死攸关的问题。我从未和任何人谈过这件事。除了你,我谁也不能说。虽然你我在各方面是完全不同的人,爱好、见解等等一切,毫无共同之处,但是我知道你喜欢我也理解我,所以我也非常喜欢你。看在上帝份上,你就对我开诚布公吧。”
\par “我是在和你谈我的想法,”斯捷潘·阿尔卡季奇微笑着说。“我还要告诉你:我妻子是个非凡的女人……”斯捷潘·阿尔卡季奇想起了和妻子的关系,叹了口气。他沉默了一会,继续说:“她有预见的天赋。她能洞察人,这还不算,她能未卜先知,特别是在婚姻方面。例如,她预言沙霍夫斯卡娅会嫁给布连捷林,当时谁也不相信,结果应验了。她也是支持你的。”
\par “此话怎讲?”
\par “就是说,她不只是喜欢你,她还说基季一定会做你的妻子。”
\par 一听此言,莱温顿时眉开眼笑,几乎要流出感动的泪水。
\par “这是她说的!”莱温喊了起来。“我一向就说,你妻子她是大好人。这就够了,这件事不需要再说了,”他说着就站起来。
\par “好吧,可是你坐下。”
\par 莱温哪里坐得住。他迈着坚定的步子在斗室里走了两圈,眨了几下眼睛,以免别人看见他的泪水,然后又坐到桌边来。
\par “你明白吗,这不是一般的爱情。我曾经爱过,这回可不一样。不是内心的情感,而是某种外部力量控制了我。上次我走了,因为我肯定事情没有希望,你明白吗,这是世间不可能有的幸福。但是,经过了一番自我斗争我又看到,得不到这种幸福就无法活下去。所以必须下决心……”
\par “上次你为什么要离开呢?”
\par “嘿,等等!嘿,我有多少想法啊!有多少事要问你啊!你听我说。你无法想象,你刚才说的话帮了我多大忙。我太幸福了,甚至都变得卑劣了。我忘记了一切……我今天才知道尼古拉哥哥……你晓得吧,他也在这里……我连他都忘记了。现在我觉得连他也是幸福的。这简直是发疯。但有一点很可怕……你是结了婚的人,懂得这种感情……可怕的是,我们都已年岁老大,有过一段往事,不是爱情,而是罪过的往事……而现在我们忽然要亲近一个纯洁无瑕的人,这是可恶的行为,所以我觉得自己实在配不上她。”
\par “哦,你的罪过并不多。”
\par “唉,毕竟是有的,”莱温说,“毕竟是有的。‘我厌恶地审视我的一生,我颤栗,我诅咒,我痛苦地抱怨……’是啊。”
\par “有什么法子呢,世道如此,”斯捷潘·阿尔卡季奇说。
\par “我喜欢祷告中的这句话:别看我的功劳饶恕我,而凭你的仁慈宽恕我。这使我得到安慰。这样一来,她也只好宽恕我了。”


\subsubsection*{十一}

\par 莱温喝完了杯中的酒,两人都沉默了一会。
\par “还有一点我要告诉你。你认识弗龙斯基吗?”斯捷潘·阿尔卡季奇问莱温。
\par “不,不认识。你干吗问这个?”
\par “再来一瓶酒,”斯捷潘·阿尔卡季奇对正在斟酒的鞑靼人说。那个鞑靼侍者,客人不叫他的时候,他也在这里不停地转来转去。
\par “我为什么要认识弗龙斯基?”
\par “因为这个弗龙斯基是你的情敌之一,所以你必须认识他。”
\par “弗龙斯基是何许人?”莱温问。奥布隆斯基刚才还在欣赏的莱温那张孩子般高兴的脸,忽然变得凶狠难看了。
\par “弗龙斯基是基里尔·伊万诺维奇·弗龙斯基伯爵的儿子。他是彼得堡纨袴子弟的一个典范。我在特维尔服役时见过他,当时他去那儿招募新兵。他十分富有,人长得漂亮,交游也很广,虽然是个侍从武官,却很可爱,人很好。还不光人好,我在这里听说他既有教养又聪明。这可是个前程远大的人。”
\par 莱温眉头紧锁,一言不发。
\par “你走后不久,他就到这里来了。据我看,他对基季一往情深。你知道,做母亲的……”
\par “对不起,我一点也不明白,”莱温沉下脸来说。他立刻又想到了尼古拉哥哥,想到自己如此卑劣,竟把哥哥也忘了。
\par “你别急,别急,”斯捷潘·阿尔卡季奇笑着碰碰他的手说,“我把我了解的情况告诉了你。再说一遍,我看这件微妙的好事琢磨起来,希望还是在你一边。”
\par 莱温往椅背上一靠,脸色苍白。
\par “不过,我劝你尽快把事情定下来,”奥布隆斯基一边给他斟满酒一边说。
\par “不要了,谢谢你,我不能再喝了,”莱温推开酒杯说。“我要醉了……那么,你过得怎么样啊?”他问,显然是想换个话题。
\par “我再说一句:无论如何你要快点解决问题。今天不必说了,”斯捷潘·阿尔卡季奇说。“明天上午你坐车去,堂堂正正去求婚。愿上帝保佑你……”
\par “你不是总想上我那儿打猎吗?等开春你就来吧,”莱温说。
\par 现在他十分后悔自己和斯捷潘·阿尔卡季奇说这番话。谈什么彼得堡军官的情场竞争,还有斯捷潘·阿尔卡季奇的种种推测和劝告,这一切都玷污了他心中那一份特别的情感。
\par 斯捷潘·阿尔卡季奇微微一笑。他理解此时莱温心中的感受。
\par “我会去的,”他说。“唉,老弟,女人好比螺旋桨,把什么都带得团团转。我的情况不好,很不好。也都是因为女人。你坦率跟我讲,”他拿起一支雪茄,一手扶着酒杯,继续说,“你给我出个主意吧。”
\par “究竟怎么回事?”
\par “是这么回事。比方说,你结了婚,爱自己的妻子,可是你又迷上了另一个女人……”
\par “对不起,我怎么也弄不懂这码事,就像……我还是不明白,就好像我现在吃饱了饭,在走过面包房的时候还要偷一块面包。”
\par 斯捷潘·阿尔卡季奇的眼睛比平时更亮了。
\par “为什么不呢?面包有时候香气诱人,叫你克制不住。
\par  
\par Himmlisch ist's,wenn ich bezwungen
\par Meine irdische Begier;
\par Aber noch wenn's nicht gelungen,
\par Hatt'ich auch recht hübsch Plaisir!\footnote{德语:我若战胜世俗的情欲,这是多么壮美之举;我的努力一旦失败,也算尝到了人间乐趣!}”
\par  
\par 说这段话时,斯捷潘·阿尔卡季奇脸上露出微妙的笑容。莱温也不禁一笑。
\par “好吧,说正经的,”奥布隆斯基接着说。“你要明白,女人是可爱、温柔而多情的人,她那么可怜和孤独,做出了一切牺牲。现在木已成舟,你明白吧,难道现在可以抛弃她吗?假如为了不破坏家庭而分手,难道不应该怜悯她,抚慰她,减轻她的痛苦吗?”
\par “对不起,你知道,我认为所有的女人分为两类……不……确切些说,有一部分是女人,也有……我不曾见过堕落而又美好的女人,以后也不会见到的。像柜台边那个涂脂抹粉、满头鬈发的法国女人,我看她是败类,一切堕落女人都是这样的。”
\par “福音书上的那个女人\footnote{指《圣经》中改过自新的妓女抹大拉的马利亚。}呢?”
\par “咳,你别说了!基督要是知道他的话被如此滥用,就决不会说那些话了。有人只记住了整个福音书里这几句话。不过,我说的不是我的想法,而是我的感觉。我厌恶堕落女人。你害怕蜘蛛,而我害怕那些败类。你大概没有研究过蜘蛛,不了解它的习性。我对那种女人也是如此。”
\par “你乐得这样说。就像狄更斯小说里的那位先生,遇到难题就用左手把它扔到右肩膀后面去。但是,否认事实并不等于答案。应该怎么办,你告诉我,应该怎么办呢?妻子在一天天衰老,而你还充满着活力。转眼之间你就感到,无论你怎样尊重妻子,已经不可能爱她。在这种时候突然有了爱情的际遇,你就毁了,毁了!”斯捷潘·阿尔卡季奇沮丧地说。
\par 莱温冷笑一声。
\par “是啊,毁了,”奥布隆斯基接着说。“可是有什么办法呀?”
\par “不去偷面包呗。”
\par 斯捷潘·阿尔卡季奇大笑起来。
\par “哦,你这道德家!你可明白,有两个女人:一个坚持一定要得到自己的权利,这权利就是你不可能给她的爱情;另一个为你牺牲了一切却没有提出任何要求。你怎么办?如何行事?这是可怕的悲剧。”
\par “如果你想听听我对这种事的内心想法,我可以告诉你,我不相信这是什么悲剧。因为据我看,爱情……有两种爱,你记得吧,柏拉图在《会饮篇》里下过定义,这两种爱都是人们的试金石。有些人懂得这一种爱,另一些人懂得那一种爱。对于只懂得非柏拉图式爱的人,根本谈不上什么悲剧。这种爱决无悲剧可言。‘由衷感谢您带给我的享受,祝您好运’,这就是所谓悲剧的全部。对于柏拉图式的爱,更不可能存在悲剧,因为这种爱完全是纯洁无瑕的,因为……”
\par 此刻莱温想起了自己的过失和经历过的内心斗争,忽然又说:
\par “其实,也许你是对的。很有可能……不过我不知道,真的不知道。”
\par “你瞧,”斯捷潘·阿尔卡季奇说,“你是个纯正的人。这是你的美德也是你的缺点。你自己有纯正的品格,便希望全部生活都是由纯正的现象组成,而这是不可能的。你看不起社会服务活动,希望凡事始终要有目的性,这也是不可能的。你还要求个人的活动总是目标明确,爱情与家庭生活永远统一,这又是不可能的。生活的一切妩媚多姿,一切的美都是由阴暗面和光明面组成的。”
\par 莱温叹了口气,没有回答。他在想自己的事,不再听奥布隆斯基说了。
\par 两人忽然都感到,虽说他们是朋友,在一起吃饭喝酒,酒又是使人亲密的东西,但是他们却在各想各的心事,彼此毫不相干。奥布隆斯基多次经历过他俩在饭后意见不是趋于一致而是更加分歧的情况,他知道在这种情况下该怎么办。
\par “结账!”他叫了一声就走到隔壁大厅去了。在那里他马上遇到一个当副官的熟人,跟他聊起某某女演员和她的姘夫如何如何来。跟副官聊天顿使奥布隆斯基松了口气,他可以稍作休息,因为每次同莱温谈话,他在脑力上和精神上都绷得太紧了。
\par 鞑靼人拿来账单,共计二十六个多卢布,外加小费。莱温应付十四卢布,这个数目在平时会把他这乡下人吓一大跳,可是今天他毫不介意地付了钱。他决定马上回家,换换衣服就上谢尔巴茨基家去,在那里将要决定他的命运。


\subsubsection*{十二}

\par 基季·谢尔巴茨卡娅公爵小姐今年十八岁。这个冬天她首次进入社交界。她在上流社会的成功已经超过了两位姐姐,甚至出乎公爵夫人的始料。在莫斯科舞会上跳舞的青年人几乎都爱上了基季,这还不算,在第一个冬季就有了两名郑重其事的追求者。一个是莱温,另一个是他刚走后就出现的弗龙斯基伯爵。
\par 入冬时莱温的出现,他的频繁造访及对基季明显的爱慕之情,都促使基季的父母开始认真商量女儿的未来,也导致了公爵和公爵夫人为此发生争执。公爵垂意莱温,认为他是基季的最佳选择。公爵夫人采用女人转弯抹角的惯技,只说基季太年轻,莱温尚未表现出诚意,基季也没有属意于他。她还找了种种理由,就是不肯说出主要的一点:她要等着给女儿择一位佳婿,莱温并不称她的意,而且她也不了解他的为人。莱温突然离去后,公爵夫人颇为高兴,洋洋得意地对丈夫说:“看见了吧,我是对的。”后来弗龙斯基的出现使她更加高兴,确信自己的想法对头:基季要找的不是一般的好配偶,而是一位乘龙佳婿。
\par 在这位母亲的眼里,莱温和弗龙斯基无法相比。她不喜欢莱温那些古怪而激烈的议论,还有,她认为他在社交界可能是由于傲气而表现出来的古板态度。她认为他在乡下过着一种野蛮生活,成天和牲畜、农夫打交道。使她很不以为然的是,他已经爱上她的女儿,在一个半月里不断上她家来,但他似乎还在等待和观望,像是担心求婚会丢了他的面子;而且他竟然不懂得,上女方家来打算求婚,总应该自我表白一下。他没有表白,倒突然离去了。“幸亏他不讨人喜欢,基季没有爱上他,”做母亲的这样想。
\par 弗龙斯基能使这位母亲如愿以偿。他很富有,聪明,出身显贵,又是步步高升的宫廷武官,人也很有魅力。真是再好不过了。
\par 弗龙斯基在舞会上明显地追求基季,和她跳舞,然后又上家里来,他的郑重其事看来是无可怀疑的。但尽管如此,整整一个冬天,基季母亲的心一直忐忑不安。
\par 公爵夫人自己出嫁是在三十年前,由姑母做的媒。未婚夫的情况事先已经了解。他上门来见未婚妻,也让人家看看他。做大媒的姑母知道了见面的印象,将其转告双方。双方印象良好。于是择日向女方父母提亲,女方的应允已在意料之中。一切进行得毫不费力,非常简单。至少公爵夫人感觉如此。现在轮到了女儿这一辈,倒让她觉得女大当嫁这件看来寻常之事,其实做起来并非简单容易。为了两个大女儿——达里娅和娜塔莉出嫁,她是那样担惊受怕,绞尽脑汁,花了多少钱,和丈夫吵了多少嘴啊!现在小女儿又要出嫁了,她同样是担惊受怕,疑虑重重,比前两个女儿出嫁时跟丈夫吵得更凶了。老公爵像所有做父亲的一样,特别注重自家女儿的名誉和清白。他对女儿们,尤其对他最宠爱的基季的苛求,使他动不动就和公爵夫人吵闹,怪她损害了女儿的名声。公爵夫人在两个大女儿出嫁时就已对此习惯,不过她现在倒觉得公爵的挑剔更有些道理了。她看到近来社会风气多有变化,做母亲的责任更加重大。她发现像基季那样大的女孩子在组织什么社团,到什么地方去听课。她们和男人自由交往,乘车出入于大街小巷,许多姑娘见人不行屈膝礼,而最主要的是,她们都确信选择丈夫是自己的事,用不着父母操心。“如今嫁女儿可不比从前了,”不仅所有的年轻姑娘,就连老一辈的人也都这么想和这么说。但是今天究竟该怎样嫁女儿,公爵夫人又无从打听。按照法国风俗,女儿的命运得由父母决定,这种做法现已无人采用,还受到了谴责。按照英国风俗,姑娘可以完全自主,这一做法也没人实行,况且在俄国社会上也是行不通的。俄国自己的风俗则是嫁娶凭媒妁之言,这又被认为不成体统,遭到了包括公爵夫人自己在内的众人的嘲笑。那么到底应该怎样出嫁和嫁女,谁也不知道。凡是跟公爵夫人谈论过这事的人,都对她说同样的话:“算了吧,如今该丢掉老规矩了。其实是年轻人结婚,又不是他们的父母结婚。让年轻人自己去做主吧。”说得倒轻巧,这些人又没有女儿。公爵夫人心里明白,女儿接近男人就可能产生爱情。她会爱上一个不想娶她的男人,或者爱上一个不适合当她丈夫的人。不管别人怎样劝说如今要让青年人自己安排自己的命运,公爵夫人就是不相信这一点,好比不相信如今五岁儿童的最佳玩具是上了真子弹的手枪一样。所以,公爵夫人之担心基季,超过了对她的两个姐姐。
\par 现在她担心,弗龙斯基是不是只跟她女儿玩玩罢了。她看出女儿已经爱上他,就自我安慰地想,他是个正人君子,不会干出那种事来。但她也知道,现在的女孩子容易被社交自由冲昏头脑,而男人们对那种罪过却看得轻描淡写。上星期基季把她和弗龙斯基跳马祖尔卡舞时谈话的内容告诉了母亲。这次谈话使公爵夫人心中稍安,但也还不能完全放心。弗龙斯基对基季说,他和哥哥习惯于事事听从母亲,不跟母亲商量是不敢采取重大行动的。“母亲就要从彼得堡来了,现在我等她来,就是在等待一种特别的幸福,”他说。
\par 说这些话时基季丝毫没有在意,但母亲的理解却不同。她知道,这边在眼巴巴盼望老太太来,老太太会对儿子的选择感到高兴。但是公爵夫人也觉得奇怪,弗龙斯基怎么会怕惹母亲生气就不来求婚呢?她很想结成这门亲事,当然更想先消除掉心中的疑虑,愿意相信事情正如基季所说的那样。她看到长女多莉眼下遭了不幸而打算离开丈夫,固然感到伤心,但是小女儿的命运也到了决定关头,这更使她激动不安,把她整个的心思都占据了。今天莱温的出现又增加了她几分忧虑。她怕女儿因为曾经一度属意莱温(她有这种感觉),出于过分的诚实而一口回绝弗龙斯基。总之,她不希望莱温的到来把快要大功告成的事情扰乱了或耽误了。
\par “他怎么,来了好些日子了?”母女俩回到家后,公爵夫人问起莱温。
\par “是今天到的,妈妈。”
\par “我想说一件事,”公爵夫人开始说。基季从她脸上严肃而兴奋的神色就猜到她要说什么事。
\par “妈妈,”她羞红了脸,忙回过头来对母亲说,“请您,请您别说这个了。我知道,我都知道。”
\par 她的心愿和母亲是一样的,但是母亲的动机使她感到委屈。
\par “我只想告诉你,既然你已经让一个人抱有希望……”
\par “妈妈,亲爱的,看在上帝份上,您就别说了。真害怕说这件事。”
\par “不说了,不说了,”母亲看见女儿的眼泪都流出来了,“可是有一点,我的宝贝,你答应过对我不保密。你能做到吗?”
\par “永远不保密,决不,”基季涨红了脸,正视着母亲的脸说。“不过现在我没有什么可说。我……我……假如我想,我不知道说什么,该怎么说……我不知道……”
\par “是的,看这对眼睛她不会说谎的,”母亲见她这副激动和幸福的样子,微笑着思忖。公爵夫人还觉得好笑,这可怜的孩子竟把自己此刻心中产生的情感看得如此意义重大深远。


\subsubsection*{十三}

\par 从晚饭后直到家庭晚间聚会开始前,基季的心情就跟初上战场的小伙子的感受差不多。她心跳得厉害,思想怎么也集中不了。
\par 今晚他们两人第一次相遇,她觉得这将是决定她命运的一晚。她老是在想象他们两个人,时而逐个地想,时而合起来想。当她回忆往事时,她怀着快慰和温馨想起了和莱温相处的日子。回忆童年,回忆莱温和她已故哥哥的友谊,使她和莱温的关系带上了一种特别的诗意美。她确实知道莱温爱她,这种爱使她得意和欣喜。所以回忆莱温时她的心情是轻松的。弗龙斯基是个风度文雅、神态从容的人,基季想到他时感到有些局促不安,似乎在他俩的关系里掺进了某种做作的成分。这做作不是在他而在她自己,因为他是那样朴实可爱。相比之下,和莱温相处时她觉得自己非常单纯而开朗。然而,当她把自己的未来和弗龙斯基连在一起时,她眼前就浮现出幸福光明的前景。而和莱温连在一起时,她则感到前途迷茫。
\par 她上楼换衣服准备参加晚间聚会,对镜时,欣喜地发现自己精神焕发,朝气蓬勃。现在她就该是这样的。她觉得自己外表镇静,举止娴雅。
\par 七点半钟,她刚下楼来到客厅,就听见仆人禀报:“康斯坦丁·德米特里奇·莱温到。”这时公爵夫人还在自己房间里,公爵也没有出来。“果然是他,”基季想,顿时觉得全身血液一下子涌上心头。她朝镜子瞥了一眼,看见自己脸色苍白,吃了一惊。
\par 她很清楚,他提前来就是为了和她单独见面并向她求婚。此刻她才第一次看到了事情的另外一面,不同的一面。此刻她才明白,问题不仅关系到她一个人:她跟谁在一起幸福及她爱谁。问题是她马上就要使她喜爱的一个人受到屈辱。难堪的屈辱……因为什么?就因为这个可爱的人喜欢她,爱上了她。但是毫无办法,她必须这样做,必须如此。
\par “天哪,难道要我亲口对他说吗?”她想。“对他说什么呢?告诉他我不爱他?这不是真话。我对他说什么呢?说我爱的是别人?不,这不可能。我离开,我走开。”
\par 听到他的脚步声时她都快要走到门口了。“不能!这是不诚实。有什么可害怕呢?我没做过任何对不起他的事。听其自然吧!我要说真话。和他在一起不会感到别扭的。他来了,”她自语道。这时她看见了他那强壮而畏怯的身影和一双亮闪闪盯着她的眼睛。她直视着他的脸,似乎在请求他宽恕,把手伸给他。
\par “我没有按时,看样子来早了,”他环顾一下空荡荡的客厅说。他看到情况正如所料,这里没有什么东西会妨碍他向她开口,这时他的脸色阴沉下来。
\par “哦,等等,”基季说罢在桌边坐下来。
\par “我就是想单独见您,”他开口说,并不坐下,眼睛也不看她,害怕失去勇气。
\par “妈妈马上就下来。昨天她累坏了。昨天……”
\par 她说话时自己也不知道在说什么,她那哀求、温情的目光始终没有离开他。
\par 他望望她。她脸红了,缄口不语。
\par “我对您说过,我不知道这次来能不能久留……这要取决于您……”
\par 她的头垂得越来越低,她不知道怎样应付渐渐迫近的那件事。
\par “这要取决于您,”他重复说。“我是想说……我想说……我是为这件事来的……就是……要您做我的妻子!”他不知所云,说了这些话,但他知道最可怕的话已经说出了口,就住了嘴,朝她望望。
\par 她深深地喘息着,眼睛也不看他。她感到一阵欣喜,心中充满了幸福。她怎么也不曾料到,他的爱情表白竟会给她如此强烈的印象。但这种感觉瞬息即逝。她想起了弗龙斯基。她抬起那双明亮诚实的眼睛望着莱温,看见他脸上绝望的神色,就匆忙地答道:
\par “这是不可能的……请原谅我……”
\par 一分钟前她对他那样亲近,对他的生活那样重要!可是现在,她变得多么陌生,多么疏远啊!
\par “不会有别的结果,”他没有看她,说。
\par 他鞠了一躬,准备离去。


\subsubsection*{十四}

\par 就在这时,公爵夫人走了进来。看见他俩单独在一起,一脸无精打采的样子,公爵夫人脸上露出了惊恐的神色。莱温向她鞠躬,没有说话。基季默不作声,没有抬起眼睛。“谢天谢地,她拒绝了,”母亲想,脸上顿时漾起了平素每周四迎接客人时的微笑。她坐下来,问起莱温在乡下的生活情况。他只得又坐下来,打算等客人都到了再悄悄离开。
\par 五分钟后,基季的女友、去年冬天才出嫁的诺德斯顿伯爵夫人走了进来。
\par 这是一个身材消瘦、脸色发黄的女人,长着一对闪闪发亮的黑眼睛,面带病容而且神经质。她喜欢基季,就像大抵已婚的女子喜欢未婚的姑娘那样。她想按照自己的幸福理想替基季物色如意郎君,惟其如此,她希望她嫁给弗龙斯基。冬季开始时,她常在这里遇见莱温。她对莱温一向反感,每次看到他,她喜欢做的一件事就是揶揄他。
\par “我就喜欢他傲气十足地看待我。要么认为我愚蠢而不愿和我进行智慧的谈话,要么只好降尊纡贵迁就我。我喜欢他降尊纡贵的样子!很高兴他看见我就受不了,”她这样谈论莱温。
\par 她说的不错。莱温确实受不了她,鄙视她津津乐道、引以为荣的那些东西,例如她的神经质,她对一切粗朴平常的事物的露骨蔑视和漠不关心态度。
\par 像诺德斯顿伯爵夫人和莱温这样的关系在社交圈里并不鲜见。两个人表面上友好,内心却相互鄙视,以至于彼此不屑于认真交往,甚至没有办法使对方生气。
\par 诺德斯顿夫人马上向莱温发动攻势。
\par “啊!康斯坦丁·德米特里奇!您又到我们道德败坏的巴比伦来了,”她向他伸出皮肤发黄的小手,想起了他在冬初把莫斯科叫作巴比伦的那番话。“怎么样,是巴比伦变好了,还是您变坏了?”她加上一句,冷笑一声,望望基季。
\par “伯爵夫人,您如此牢记我的话,使我不胜荣幸,”莱温回答。他已经恢复平静,现在又按习惯对诺德斯顿夫人采取了那种半开玩笑的敌对态度。“那番话果然对您很起作用。”
\par “可不是嘛!我总是把您的话记录下来。基季,你又溜冰了吗?……”
\par 她和基季聊了起来。莱温觉得现在就走虽然不大方便,但总比整个晚上都待在这里看见基季要好受些。基季不时望望他,却避开他的目光。他刚想站起来,公爵夫人见他默默无言,就过来找他说话:
\par “您这次来莫斯科,能多住些日子吗?您好像在忙地方自治局的事吧,那就不能在莫斯科久留了。”
\par “不,公爵夫人,我不再管地方自治局的事了,”他说。“我这次来只住几天。”
\par “他有点不对头,”诺德斯顿伯爵夫人盯着他神情严肃的脸,心想,“他似乎不大想高谈阔论。我要逗他发表议论,极希望他在基季面前像个傻瓜,我定要逗逗他。”
\par “康斯坦丁·德米特里奇,”她对他说,“请您给我解释一下,您都知道的,这到底是怎么回事。我们卡卢加省乡下的庄稼汉和婆娘们把什么东西都拿去换酒喝了,现在他们一点租子也不缴了。这是怎么回事呀?您可是总在夸奖那班庄稼汉的。”
\par 这时又有一位太太走进客厅。莱温站起身。
\par “对不起,伯爵夫人,其实我对此一无所知,无可奉告,”说罢,他回头望了望跟着那位太太进来的一位军人。
\par “此人一定是弗龙斯基了,”莱温想,为了证实这一点,他朝基季望了望。基季瞥了弗龙斯基一眼,又回头睃了睃莱温。就凭这情不自禁喜形于色的一瞥,莱温明白了,基季所爱的正是此人,就好像她亲口告诉他一样。这是一个什么样的人物呢?
\par 现在莱温好歹是不能走了。他要弄弄清楚,她所爱的究竟是怎样的一个人。
\par 有些人不管在什么事情上碰到幸运的对手,马上就鄙弃对方的一切长处而光看他身上的短处。还有一些人则相反,他们特别想在幸运者身上发现他借以制胜的那些品质,并强忍住揪心的痛苦,特意去找对方的优点。莱温属于后一种人。他不费什么劲就发现了弗龙斯基身上的优点和吸引人的地方。这是一眼就能看出来的。弗龙斯基是个身材不高、体格结实的黑发男子,有着一副和蔼漂亮的面孔,显得十分安详而坚定。在他脸上和身上,从那剪得短短的黑发,刮得精光的下巴,到那身宽松的崭新军服,一切都显得那样素雅。弗龙斯基让那位太太先进去,然后走到公爵夫人面前,又向基季走过去。
\par 当他向她走去的时候,他那漂亮的眼睛显得炯炯有神,特别温柔,脸上带着难以觉察的谦逊而得意的幸福微笑(莱温有此感觉)。他恭恭敬敬地向基季低头行礼,把他那并不肥大然而宽厚的手伸给她。
\par 他跟在场的所有人打过招呼,寒暄数语,就坐了下来,并未向始终盯着他的莱温看过一眼。
\par “让我介绍你们认识一下,”公爵夫人指着莱温说。“这位是康斯坦丁·德米特里奇·莱温。这位是阿列克谢·基里洛维奇·弗龙斯基伯爵。”
\par 弗龙斯基站起身,友善地望着莱温的眼睛,握握他的手。
\par “我觉得,今年冬天是早该有机会和您一起吃饭的,”他说着,露出他那朴实开朗的笑容,“不想您忽然回到乡下去了。”
\par “康斯坦丁·德米特里奇鄙视和仇恨城市,还有我们这些城里人,”诺德斯顿伯爵夫人说。
\par “看样子,我说过的话对您作用很大,难怪您牢记不忘,”莱温说罢,想起他已经这样说过了,不觉脸上一红。
\par 弗龙斯基望望莱温,又望望诺德斯顿伯爵夫人,微微一笑。
\par “您总是待在乡下吗?”他问。“我想冬天会感到寂寞吧?”
\par “有事做就不寂寞,自己一个人也不会寂寞,”莱温语气生硬地回答。
\par “我喜欢乡村,”弗龙斯基说,觉察到莱温的口气,但佯作不知。
\par “伯爵,您不会永远住到乡下去吧,”诺德斯顿伯爵夫人说。
\par “不知道。我没有长久住过。我有一个奇怪感觉,”他接着说。“我和母亲在尼斯\footnote{法国南方城市。}过了一个冬天,打那以后我特别怀念乡村,有树皮鞋子和庄稼汉的俄国乡村。你们知道,尼斯是个枯燥乏味的地方。那不勒斯和索伦托\footnote{都是意大利南方城市。}的美妙也只有短暂时间。正是在那些地方你会特别亲切地想起俄国,想起俄国的乡村。那些地方就像……”
\par 他在对基季说,也在对莱温说,他那安详友好的目光时而望望她,时而又望望他。他显然是想到哪里聊到哪里。
\par 这时他见诺德斯顿伯爵夫人想开口,就打住话头,注意听她讲。
\par 谈话一刻不停地进行着。一旦谈话出现冷场,公爵夫人早已备好了两门“重炮”,一个话题是古典教育与现实教育,另一个是普遍兵役制。现在她无需推出“重炮”,而诺德斯顿夫人也没有机会逗莱温。
\par 莱温想加入大家的谈话,但是插不上嘴。他老在嘀咕:“现在就走。”可是他并没有走,像在等待着什么。
\par 谈话扯到了扶乩和鬼神。诺德斯顿夫人相信招魂术,开始讲她亲眼看到过的奇迹。
\par “啊,伯爵夫人,看在上帝份上,您务必带我见见那些神灵!我到处找神奇的东西,可是从来也没见过,”弗龙斯基笑着说。
\par “行,下礼拜六吧,”诺德斯顿伯爵夫人说。“您怎么样,康斯坦丁·德米特里奇,相信吗?”她问莱温。
\par “何必问我呢?您知道我会说什么。”
\par “可是我想听听您的看法。”
\par “我的看法是,”莱温答道,“那些转动的桌子证明了,所谓有教养的人们并不比庄稼汉高明些。庄稼汉相信毒眼、中邪,还有蛊术,而我们……”
\par “这么说,您不相信?”
\par “我不可能相信,伯爵夫人。”
\par “如果是我亲眼目睹的呢?”
\par “农妇们也讲,她们如何如何亲眼看见了家神。”
\par “那么您认为我是在说谎了?”
\par 她很不自然地笑起来。
\par “哦,不,玛莎,康斯坦丁·德米特里奇是说,他不可能相信,”基季说。她为莱温脸红了。莱温见状更加恼火,他正要回答,这时弗龙斯基带着开朗快乐的笑容连忙过来圆场,以免谈话变成不快。
\par “您完全不承认这种可能性吗?”他问。“为什么呢?我们不懂电,但是承认它的存在。为什么不可能有一种新的力,我们还不知道的力,它……”
\par “电被发现的时候,”莱温很快打断他的话,“当初只看到它的现象,并不知道它从何而来和有什么作用,经过好几个世纪之后人们才想到运用它。招魂者则相反,一开始就是什么桌子写字,神灵附体,然后再说这是一种未知的力。”
\par 弗龙斯基照样注意地听莱温说,显然对他的话发生了兴趣。
\par “可是招魂者说:我们现在还不知道这是一种什么力,但力是存在的,它就在这样的条件下发生作用。至于这种力的具体内容,还是让科学家们去揭示吧。我不明白,为什么就不能有一种新的力,既然它……”
\par “因为,”莱温打断他的话,“如果您做电的实验,只要用松香磨擦毛皮,就会产生大家知道的现象。可是招魂术并非每一次都灵验,可见这不是什么自然现象。”
\par 大概弗龙斯基觉得这种谈话对客厅气氛来说显得过于严肃,就不再争辩,而想改变一下话题,于是脸上露出快乐的笑容,来找女士们说话。
\par “伯爵夫人,我们现在就来试试吧,”他说,可是莱温还想证明一下自己的想法。
\par “我认为,”莱温接着说,“招魂者试图把他们的怪诞事情解释为某种新的力,这完全是徒劳的尝试。他们直言不讳地讲的是一种精神力量,却要拿它来进行物质试验。”
\par 大家都在等他说完,他也觉察到了这一点。
\par “我想,您会成为一个出色的扶乩人,”诺德斯顿伯爵夫人说,“您身上有一种狂热的东西。”
\par 莱温张口想说什么,可是他脸上一红,又不作声了。
\par “请吧,公爵小姐,我们用桌子来试验一下吧,”弗龙斯基说。“公爵夫人,您允许吗?”
\par 弗龙斯基站起来,用眼睛四处搜寻小桌子。
\par 基季走到小桌子边站住。她从莱温身旁经过时,两人的目光碰到一起。她出自内心地怜悯他,特别是,造成他不幸的原因正是她自己。“如果能原谅我,就请您原谅吧,”她的目光在说,“我现在很幸福。”
\par “我恨所有的人,恨您,也恨我自己,”他的目光在回答。他伸手去拿礼帽,然而合该他走不掉。大伙正忙着在小桌子边坐下来,而莱温正想离开时,老公爵恰好走进客厅。他向太太们问过好,就来和莱温说话。
\par “啊!”他乐呵呵地说。“来多日了吗?我不知道你在这里。见到您真高兴。”
\par 老公爵对莱温时而称“你”时而称“您”。他拥抱了莱温,只顾同他说话,并没有注意弗龙斯基站了起来,静静地等待他和自己打招呼。
\par 基季觉得,发生了那件事以后,父亲的殷勤一定会使莱温不舒服。她还看到,父亲终于冷冰冰地回答了弗龙斯基的鞠躬,而弗龙斯基带着友好却又莫名其妙的神情望了望她的父亲,似乎想弄明白而终于弄不明白,公爵为什么对他这样不客气。基季脸红了。
\par “公爵,您让康斯坦丁·德米特里奇到这边来吧,”诺德斯顿伯爵夫人说。“我们要做个试验。”
\par “什么试验?转动桌子吗?对不起,女士们先生们,我看还是玩套圈更有趣些,”老公爵看了一眼弗龙斯基说,猜想这是他出的主意。“套圈还有点意思。”
\par 弗龙斯基用那双神情坚定的眼睛惊奇地望望公爵,又微微一笑,马上和诺德斯顿伯爵夫人谈起下星期要举行的大型舞会来。
\par “我想您会参加吧?”他对基季说。
\par 莱温趁老公爵转身的工夫悄悄走出了客厅。这一晚留给他的最后印象,就是基季在回答弗龙斯基是否参加舞会时她那张幸福的笑脸。


\subsubsection*{十五}

\par 晚会结束后,基季把她和莱温的谈话告诉了母亲。尽管她很怜悯莱温,但想到有人曾向她求婚,心中就充满了喜悦。她毫不怀疑自己做对了。可是她上床后久久不能入睡。有一个印象紧紧追随着她。那就是莱温的那张脸,当时他站在那里听父亲说话,不时望望她和弗龙斯基,他眉头紧锁,一双和善的眼睛里充满了忧郁。她真是可怜他,不由得泪水盈眶。不过她马上又想到了舍他而换得的另一个人。她真切地回忆起那个人的刚毅坚定的面容、高雅镇定的风度和事事处处待人谦和的可贵品格。想起所爱之人对她的爱,她重又喜上心头。她带着幸福的微笑躺到枕上。“我怜悯他,怜悯他,又有什么办法?这不是我的错,”她自语道,但在内心深处并不是这样说。她不知道自己是否要后悔不该引起莱温的爱慕,或者不该拒绝他。她的幸福感被这些疑虑搅坏了。“上帝保佑,上帝保佑,上帝保佑!”她嘟哝着,终于睡着了。
\par 这时在楼下,在公爵的小书房里,父母亲经常为爱女而争吵的一幕又在重演。
\par “什么?原来如此!”公爵嚷道,使劲地挥动两臂,又掩了掩他那灰鼠皮的睡衣。“原来您这么不自尊,不自重,用这种下流愚蠢的做媒手段来羞辱女儿,坑害女儿!”
\par “哪有这种事!看在上帝份上,公爵,我做错了什么了?”公爵夫人带着哭腔说。
\par 公爵夫人同女儿谈过话后,怀着幸福满足的心情像平时一样来跟公爵道晚安。她不想把莱温求婚和基季拒绝的事告诉他,但她暗示丈夫,女儿和弗龙斯基的事看来完全没有问题,只等他母亲一到就可以定下来了。公爵一听此言就大发雷霆,嚷出些不体面的话来。
\par “您干了些什么?听着:首先,您引诱求婚者,全莫斯科的人都会这样说,他们有理由这样说。您要开晚会,那就把大家都叫来,而不光是叫您挑好的求婚者。把那班宝贝蛋(公爵这样称呼莫斯科的年轻人)通通叫来,再雇个钢琴手,让他们都跳舞,而不是像今天这样只叫求婚者来,把他们跟女儿撮合到一起。让我看着就讨厌,讨厌。您达到目的了,让小姑娘冲昏了头脑。莱温比他们要好一千倍。那个彼得堡花花公子,在机器上都能成批做出来,他们全都一个样,全是些废物。就算他是真正的王子,我女儿也不需要这样的人!”
\par “我究竟做了什么了?”
\par “做了……”公爵怒吼道。
\par “我知道,要是听你的话,”公爵夫人打断他说,“我们的女儿永远嫁不出去。照这样就得搬到乡下去。”
\par “到乡下去更好。”
\par “你听着。难道我巴结谁了?我谁也没有巴结。一个年轻人,他很不错,爱上了基季,她似乎也……”
\par “哼,似乎!若是她真的爱上他,他却像我似的不想结婚又怎么办?……咳!真不愿看到这种事!……啊,招魂术!啊,尼斯!啊,舞会!……”公爵故意模仿妻子的样子,说每句话时行一个屈膝礼。“到那时候,我们给卡坚卡\footnote{基季的小名。}造成了不幸,而她自己也真正明白过来……”
\par “为什么你要这样想?”
\par “我不是想,而是知道。这种事情我们看得很准,不像妇道人家。我看出来有个人他是真心实意的,那就是莱温。我还看到一个浮滑之徒,他像只鹌鹑,不过是想寻欢作乐罢了。”
\par “唉,你一定要这样认为的话……”
\par “等到日后回想起来就太晚了,就像达申卡\footnote{达里娅(多莉)的小名。}的事情那样。”
\par “好吧,好吧,我们别再说了,”公爵夫人想起了不幸的多莉,没让丈夫说下去。
\par “那好吧,再见!”
\par 夫妇俩互相画了十字,接过吻,仍然各持己见地走开了。
\par 一开始公爵夫人确信今晚已经决定了基季的命运,而弗龙斯基的意图也无庸置疑。但是丈夫的一番话使她倒没了主意。回到房间后,她也像基季一样对前途的未卜感到恐惧,在心里连连祝愿道:“上帝保佑,上帝保佑,上帝保佑!”


\subsubsection*{十六}




\subsubsection*{十七}




\subsubsection*{十八}




\subsubsection*{十九}




\subsubsection*{二十}




\subsubsection*{二十一}




\subsubsection*{二十二}




\subsubsection*{二十三}




\subsubsection*{二十四}




\subsubsection*{二十五}




\subsubsection*{二十六}




\subsubsection*{二十七}




\subsubsection*{二十八}




\subsubsection*{二十九}




\subsubsection*{三十}




\subsubsection*{三十一}




\subsubsection*{三十二}




\subsubsection*{三十三}




\subsubsection*{三十四}





























\subsection*{第二部}




\subsubsection*{一}
\subsubsection*{二}
\subsubsection*{三}
\subsubsection*{四}
\subsubsection*{五}
\subsubsection*{六}
\subsubsection*{七}
\subsubsection*{八}
\subsubsection*{九}
\subsubsection*{十}
\subsubsection*{十一}
\subsubsection*{十二}
\subsubsection*{十三}
\subsubsection*{十四}
\subsubsection*{十五}
\subsubsection*{十六}
\subsubsection*{十七}
\subsubsection*{十八}
\subsubsection*{十九}
\subsubsection*{二十}
\subsubsection*{二十一}
\subsubsection*{二十二}
\subsubsection*{二十三}
\subsubsection*{二十四}
\subsubsection*{二十五}
\subsubsection*{二十六}
\subsubsection*{二十七}
\subsubsection*{二十八}
\subsubsection*{二十九}
\subsubsection*{三十}
\subsubsection*{三十一}
\subsubsection*{三十二}
\subsubsection*{三十三}
\subsubsection*{三十四}
\subsubsection*{三十五}







\subsection*{第三部}



\subsubsection*{一}
\subsubsection*{二}
\subsubsection*{三}
\subsubsection*{四}
\subsubsection*{五}
\subsubsection*{六}
\subsubsection*{七}
\subsubsection*{八}
\subsubsection*{九}
\subsubsection*{十}
\subsubsection*{十一}
\subsubsection*{十二}
\subsubsection*{十三}
\subsubsection*{十四}
\subsubsection*{十五}
\subsubsection*{十六}
\subsubsection*{十七}
\subsubsection*{十八}
\subsubsection*{十九}
\subsubsection*{二十}
\subsubsection*{二十一}
\subsubsection*{二十二}
\subsubsection*{二十三}
\subsubsection*{二十四}
\subsubsection*{二十五}
\subsubsection*{二十六}
\subsubsection*{二十七}
\subsubsection*{二十八}
\subsubsection*{二十九}
\subsubsection*{三十}
\subsubsection*{三十一}
\subsubsection*{三十二}







\subsection*{第四部}



\subsubsection*{一}
\subsubsection*{二}
\subsubsection*{三}
\subsubsection*{四}
\subsubsection*{五}
\subsubsection*{六}
\subsubsection*{七}
\subsubsection*{八}
\subsubsection*{九}
\subsubsection*{十}
\subsubsection*{十一}
\subsubsection*{十二}
\subsubsection*{十三}
\subsubsection*{十四}
\subsubsection*{十五}
\subsubsection*{十六}
\subsubsection*{十七}
\subsubsection*{十八}
\subsubsection*{十九}
\subsubsection*{二十}
\subsubsection*{二十一}
\subsubsection*{二十二}
\subsubsection*{二十三}







\subsection*{第五部}




\subsubsection*{一}
\subsubsection*{二}
\subsubsection*{三}
\subsubsection*{四}
\subsubsection*{五}
\subsubsection*{六}
\subsubsection*{七}
\subsubsection*{八}
\subsubsection*{九}
\subsubsection*{十}
\subsubsection*{十一}
\subsubsection*{十二}
\subsubsection*{十三}
\subsubsection*{十四}
\subsubsection*{十五}
\subsubsection*{十六}
\subsubsection*{十七}
\subsubsection*{十八}
\subsubsection*{十九}
\subsubsection*{二十}
\subsubsection*{二十一}
\subsubsection*{二十二}
\subsubsection*{二十三}
\subsubsection*{二十四}
\subsubsection*{二十五}
\subsubsection*{二十六}
\subsubsection*{二十七}
\subsubsection*{二十八}
\subsubsection*{二十九}
\subsubsection*{三十}
\subsubsection*{三十一}
\subsubsection*{三十二}
\subsubsection*{三十三}






\subsection*{第六部}





\subsubsection*{一}
\subsubsection*{二}
\subsubsection*{三}
\subsubsection*{四}
\subsubsection*{五}
\subsubsection*{六}
\subsubsection*{七}
\subsubsection*{八}
\subsubsection*{九}
\subsubsection*{十}
\subsubsection*{十一}
\subsubsection*{十二}
\subsubsection*{十三}
\subsubsection*{十四}
\subsubsection*{十五}
\subsubsection*{十六}
\subsubsection*{十七}
\subsubsection*{十八}
\subsubsection*{十九}
\subsubsection*{二十}
\subsubsection*{二十一}
\subsubsection*{二十二}
\subsubsection*{二十三}
\subsubsection*{二十四}
\subsubsection*{二十五}
\subsubsection*{二十六}
\subsubsection*{二十七}
\subsubsection*{二十八}
\subsubsection*{二十九}
\subsubsection*{三十}
\subsubsection*{三十一}
\subsubsection*{三十二}






\subsection*{第七部}



\subsubsection*{一}
\subsubsection*{二}
\subsubsection*{三}
\subsubsection*{四}
\subsubsection*{五}
\subsubsection*{六}
\subsubsection*{七}
\subsubsection*{八}
\subsubsection*{九}
\subsubsection*{十}
\subsubsection*{十一}
\subsubsection*{十二}
\subsubsection*{十三}
\subsubsection*{十四}
\subsubsection*{十五}
\subsubsection*{十六}
\subsubsection*{十七}
\subsubsection*{十八}
\subsubsection*{十九}
\subsubsection*{二十}
\subsubsection*{二十一}
\subsubsection*{二十二}
\subsubsection*{二十三}
\subsubsection*{二十四}
\subsubsection*{二十五}
\subsubsection*{二十六}
\subsubsection*{二十七}
\subsubsection*{二十八}
\subsubsection*{二十九}
\subsubsection*{三十}
\subsubsection*{三十一}







\subsection*{第八部}





\subsubsection*{一}
\subsubsection*{二}
\subsubsection*{三}
\subsubsection*{四}
\subsubsection*{五}
\subsubsection*{六}
\subsubsection*{七}
\subsubsection*{八}
\subsubsection*{九}
\subsubsection*{十}
\subsubsection*{十一}
\subsubsection*{十二}
\subsubsection*{十三}
\subsubsection*{十四}
\subsubsection*{十五}
\subsubsection*{十六}
\subsubsection*{十七}
\subsubsection*{十八}
\subsubsection*{十九}













