
\section{安娜·卡列尼娜}


\par 书名:安娜·卡列尼娜(译文名著精选)
\par 作者:[俄]列夫·托尔斯泰
\par 译者:高惠群,傅石球,于国畔,顾生根
\par 出版社:上海译文出版社
\par 出版时间:2010-08
\par ISBN:9787532751204


\subsection*{译本序}

\par 《安娜·卡列尼娜》(1873—1878)是俄国大文豪列夫·托尔斯泰的第二部长篇巨著。起初,托翁只打算把它写成一部“一个不忠实的妻子以及由此而发生的全部悲剧”(贝奇科夫语),仅用了五十天他便粗略地完成了全书。五年多以后,在前后用过《年轻太太》、《两段婚姻》、《两对夫妻》等书名后,它以《安娜·卡列尼娜》的名字问世了。
\par 这部小说的主要意义应该包括三方面,即安娜的个人悲剧;1860年代的俄国社会——沙龙、军官俱乐部、舞会、戏院、赛马……以及自传的性质。
\par 《安娜·卡列尼娜》开篇第一句话,对于中国读者,甚至没有读过此书的中国人来说,都不陌生:“幸福的家庭无不相似,不幸的家庭各有不幸。”安娜是一位穿着黑衣的最迷人的少妇,她善良、聪慧、生命力旺盛,男人和女人都为她着迷。她身上迸发出的爱情“含有一种暴烈的、肉感的、专横的性格”(罗曼·罗兰语)。其实,作家对婚姻、家庭问题的思考可以追溯到动笔撰写这部小说前的五年,即1868年,这一年,他在题为《论婚姻和妇女的天职》一文中说:“男人的天职是做人类社会蜂房的工蜂,那是无限多样化的;而母亲的天职呢,没有她们便不可能繁衍后代,这是唯一确定无疑的。”托尔斯泰借莱温和基季的恋爱婚姻表达出这一妇女观、家庭观。紧随这段话托翁又说:“虽然如此,妇女还是常常看不到这一使命,而选择虚假的,即其他的使命……这一使命的重要性和无限性,以及它只能在一夫一妻的形式(即过去和现在生活着的人称之为家庭的形式)下才能实现……因而一个妇女为了献身于母亲的天职而抛弃个人的追求越多,她就越完美。”由此不难理解,托尔斯泰为何将安娜命运的结局安排为卧轨自杀——在小说接近尾声的第七部第三十章,安娜还在想着“只要办完离婚手续,阿列克谢·亚力山德罗维奇把谢廖扎还给我,我就与弗龙斯基结婚”。既然还不牺牲个人的追求,在托翁看来,这样的女子就完美不起来,那就让她毁灭吧!可小说并没有因为安娜的死亡而结束。整个第八部的十九个章节的内容,就如同《战争与和平》长长的“尾声”,如果以西欧小说式的结局为标准,这已不像是“尾声”。
\par 可见,《安娜·卡列尼娜》不只是关注安娜的死,安娜的悲剧一直扩展到所有家庭的幸与不幸。在对安娜形象的塑造上,托尔斯泰倾注了他对人的肉体本能因素、人的伦理因素、人的“灵魂”因素、人的社会因素等的思考与体悟。在此部小说之后的《忤悔录》(1879—1882)中,托翁还在进行着与上述内容相关的精神探索。
\par \rightline{查晓燕}
\par \rightline{2006.3.22}


\clearpage

\begin{center}
    伸冤在我,我必报应。\footnote{语出《新约·罗马书》第12章第19节,全句为:“亲爱的弟兄,不要自己伸冤,宁可让步,听凭主怒,因为经上记着:‘主说,伸冤在我,我必报应。’”}
\end{center}

\subsection*{第一部}



\subsubsection*{一}

\par 幸福的家庭无不相似,不幸的家庭各有不幸。
\par 奥布隆斯基家里全乱了套。妻子得知丈夫和过去的法国女家庭教师有染,就对丈夫声称,不可能和他同住在一个家里。这种局面僵持到第三天,夫妻双方及全体家人都有痛切感受。大家觉得住在一起实在无聊,随便哪家客店里偶然相逢的人也会比他们,奥布隆斯基家的人关系更好些。妻子不出房门,丈夫三日不归。孩子们满屋乱跑,无人照料。英国女家庭教师跟女管家吵了架,写信请朋友重新找份工作。厨师昨天就离开了家,在午餐时走的。打下手的厨娘,还有马车夫也都要求辞工。
\par 吵闹的第三天,斯捷潘·阿尔卡季奇·奥布隆斯基公爵(在社交场合他叫斯季瓦)在通常时间、即上午八时醒来,但不在妻子的卧室,而在书房里的山羊皮长沙发上。他在弹簧上翻了一下保养得很好的丰满身体,紧紧搂住枕头,把脸埋进去,似乎还想好好睡一觉,可是他突然一骨碌爬起来,坐在沙发上,睁开了眼睛。
\par “哦,哦,梦见什么了?”他想起做过一个梦。“哦,梦见什么了?对了!阿拉宾在达姆施塔特\footnote{德国西部城市。}举行宴会;不,不在达姆施塔特,而是美国的什么地方。对,那个达姆施塔特在美国。对,阿拉宾在玻璃餐桌上设宴,对的,大家都唱Il mio tesoro\footnote{意大利语:我的宝贝。},不是Il mio tesoro,比这更好听的,还有那些细颈小玻璃瓶,原来都是一个个女人,”他回忆着。
\par 斯捷潘·阿尔卡季奇眼睛里闪出快乐的光,微笑着沉思起来。“哦,是个好梦,非常之好。梦里还有许多美妙的东西,难以言传,醒了连什么情景也说不清楚了。”他看见一道亮光从呢绒窗幔的边缘射进来,高兴地把腿伸到沙发下面,用脚探到妻子为他绣上花的那双金黄色羊皮便鞋(去年的生日礼物),按照九年来的老习惯,并不起身,把手伸向他在卧室里挂睡衣的老地方。这当儿他才猛然想起,他怎么和为什么没有睡在妻子的卧室而睡在书房里。脸上的笑容不见了,他蹙起了额头。
\par “唉,唉!……”他咕咕哝哝地说,回忆起事情的全部经过。脑海中又出现了同妻子口角的所有细节,想起他那进退维谷的处境,还有他犯下的最使人痛苦的过错。
\par “是啊!她不肯宽恕我,不可能宽恕我。最糟糕的是一切皆由我而起,而又不能怪我。这是整个悲剧所在,”他这样想。“唉,唉!”他悲观失望,又想起了这场口角中最令他痛苦的那些情景。
\par 最难堪的是起初的那一刻,当时他刚看完戏回家,高高兴兴,心满意足,手里还拿着一只大梨子准备送给妻子,可是在客厅里没见到她;奇怪的是,她也不在书房,最后在卧室里找到她,她手里正拿着那封使丑事败露的倒霉的信。
\par 多莉是个操劳不停、他认为不大聪明的女人。这时她手里拿着那封信,一动不动地坐在那里,带着恐惧、绝望而愤怒的神情注视着他。
\par “这是什么?这?”她指着信问道。
\par 每次回想到这里,斯捷潘·阿尔卡季奇总是很苦恼,倒不是为了那件事本身,主要是他对妻子的质问竟然作出了那样的回答。
\par 当时他的处境,正像那些干了十分丢脸的事突然被揭发出来的人一样。妻子揭了他的丑,而他却不能神色镇定地应付他面临的局面。他本可以表示委屈,可以否认、辩解、求饶,甚至哪怕是满不在乎也好,可是他却干了什么啊!在他的脸上,居然不由自主地(那是“大脑反射”——爱好生理学的斯捷潘·阿尔卡季奇这样认为)露出了他平时那副憨厚的,而现在却是愚蠢的微笑。
\par 他不能自宥这愚蠢的一笑。多莉看到这副笑容,仿佛肉体疼痛似地颤栗了一下,接着就狠狠地发作起来,以她特有的急躁劲,滔滔不绝地喷吐了一通尖酸刻薄的话,然后奔出房间。打那以后,她再也不见丈夫的面了。
\par “都怪这愚蠢的一笑,”斯捷潘·阿尔卡季奇想。
\par “可是怎么办呢?怎么办呢?”他在绝望地自言自语,找不到答案。

\subsubsection*{二}

\par 斯捷潘·阿尔卡季奇对自己倒也实事求是。他并不自我欺骗,相信自己对做过的事追悔莫及。他是一个三十四岁、漂亮而多情的男子。妻子只比他小一岁,是现有的五个孩子、还有两个已夭折的孩子的母亲。他并不爱她,对此他至今倒也无悔。他所后悔的,只是没有把那件事更好地瞒住她。不过,他仍然感到处境困难,妻子、孩子,还有自己都很可怜。要是他早先料到这个消息对妻子打击如此之大,也许会对她紧紧掩盖住自己的罪过。对于这个问题,他从来没有认真考虑过,他只是模糊感到,妻子早已觉察到他的不忠,只不过眼开眼闭罢了。他甚至觉得,她身体虚弱,人老珠黄,姿色平常,毫无出众之处,仅仅是一位家庭慈母,平心而论,她应该是宽大为怀的。结果事情却闹得适得其反。
\par “唉,可怕!哎呀呀,真可怕!”斯捷潘·阿尔卡季奇不断嘀咕着,却想不出办法。“这以前一切多么美好,我们生活得多么和睦!她有孩子们在身边,感到满足和幸福,我也从不干涉她,让她忙孩子忙家务,遂了她的心意。说实在的,糟糕的就是她来当了我们的家庭教师。勾搭家庭教师确实有些庸俗下流。可她是个多么漂亮的家庭教师啊!(他真切地回忆起Mlle Roland\footnote{法语:罗朗小姐。}那双调皮的黑眼睛和她的微笑。)她在我家时,我丝毫也不曾放肆。最糟的是她现在已经……偏偏这就像故意作对似的!哎呀呀!这可怎么办,怎么办呢?”
\par 答案没有找到。只有生活能给他提供一个普通的解答,可以用它来应付所有无法解决的难题。这个解答是:去过日常生活,把烦恼丢在脑后。他想回到梦中去,这要等到夜晚才行。梦中的音乐,那些玻璃瓶女人的歌唱此刻不可能重温。看来,他只能在糊里糊涂的日子里去忘忧解愁了。
\par “以后再说吧,”斯捷潘·阿尔卡季奇自语道,穿上那件浅蓝丝绸衬里的灰色睡衣,系好绦带,往宽阔的胸腔里足足地吸了口气,迈开他丰满身躯下面那双轻快的外八字脚,像平时一样精神抖擞地走到窗前,拉开窗帘,使劲地按了按铃。应声进来的是他的老仆马特维,手里拿着衣服、靴子和一封电报。随后走进来的是带着刮脸用具的理发匠。
\par “机关里有公文来吗?”斯捷潘·阿尔卡季奇问,接过电报,在镜子前坐下来。
\par “放在桌上了,”马特维答道,带着询问和关切的神情瞥了主人一眼,停了一会,又狡黠地笑笑说:“车夫主人那边有人来过。”
\par 斯捷潘·阿尔卡季奇没有答腔,只是从镜子里瞥了马特维一眼;两人目光在镜中相遇,可以看出,他们是心照不宣的。斯捷潘·阿尔卡季奇的眼色仿佛在问:“你干吗说这个?难道你不知道吗?”
\par 马特维把双手插进外衣口袋,挪了挪腿,脸上带着笑意,默默地、和善地看了主人一眼。
\par “我叫他们礼拜天来,何必来早了麻烦您又自找麻烦,”马特维的这句话显然是事先考虑好的。
\par 斯捷潘·阿尔卡季奇明白,马特维是想说句笑话逗引别人的注意。他拆开电报看了一遍,猜懂了译电中常见的几个错别字,顿时喜形于色。
\par “马特维,我妹妹安娜·阿尔卡季耶夫娜明天就到,”他说。这时理发匠正在修剪他那又长又鬈的络腮胡子,使淡红色的皮肤显露出来,他示意那只溜光的胖手暂停一下。
\par “谢天谢地,”马特维说,表示他和主人同样明白这次来访的意义,也就是说,斯捷潘·阿尔卡季奇这位亲爱的胞妹安娜·阿尔卡季耶夫娜,她可能促成兄嫂重新和好。
\par “一个人来,还是同她先生一道来?”马特维问。
\par 斯捷潘·阿尔卡季奇不能说话,理发匠正在剃他的上唇的胡子,他竖起一根手指,马特维在镜子里点点头。
\par “一个人。要在楼上收拾房间吗?”
\par “禀报达里娅·亚历山德罗夫娜,她自会吩咐的。”
\par “达里娅·亚历山德罗夫娜?”马特维有些怀疑地问。
\par “对,禀报她。把电报也拿去,然后告诉我她有何吩咐。”
\par “您想试探一下,”马特维心里明白,嘴上却说:
\par “遵命。”
\par 斯捷潘·阿尔卡季奇梳洗完毕,正准备穿戴,这时马特维手里拿着那份电报,慢吞吞地,把靴子踩得吱吱作响地回到房里来。理发匠已经走了。
\par “达里娅·亚历山德罗夫娜叫我禀报您,她就要走了。说随便他,也就是您,爱怎么办就怎么办吧,”马特维说,眼睛里含着笑意,把双手插进衣袋,侧着脑袋,凝视着主人。
\par 斯捷潘·阿尔卡季奇沉默了一会,漂亮的脸上露出宽厚而又可怜的笑容。
\par “啊?马特维?”他摇摇头说。
\par “没关系,老爷,会顺利解决的,”马特维说。
\par “会顺利解决?”
\par “是的,老爷。”
\par “你这样认为吗?那是谁呀?”斯捷潘·阿尔卡季奇问,他听见门外有女人衣裙的窸窣声。
\par “是我。”一个稳重悦耳的女人声音说,接着,奶妈马特廖娜·菲利莫诺夫娜那张严肃的麻脸从门外伸了进来。
\par “什么事,马特廖莎?”斯捷潘·阿尔卡季奇朝门口迎去。
\par 虽然斯捷潘·阿尔卡季奇应该对妻子负全部罪责,他自己也觉得是这样,但是家中几乎所有的人,包括达里娅·亚历山德罗夫娜的心腹奶妈在内,全都站在他一边。
\par “什么事呀?”他闷闷不乐地问。
\par “您去一趟吧,老爷,去认个错。也许上帝会帮助您。她痛苦极了,看着多可怜,家里都闹翻天了。老爷,可怜可怜孩子们吧。认个错,老爷。没有办法呀!想图快活也得要……”
\par “她不肯见我……”
\par “您只管去认错。上帝会发慈悲的,您祷告上帝,老爷,祷告上帝吧。”
\par “那好,你去吧,”斯捷潘·阿尔卡季奇说,突然涨红了脸。“喂,现在穿衣服吧,”他对马特维说,动作利落地脱下了睡衣。
\par 马特维把衬衫张开伺候着,就像举着一个马轭,轻轻吹去上面的纤尘,带着明显满意的神情把它套在主人娇贵的身体上。

\subsubsection*{三}

\par 斯捷潘·阿尔卡季奇穿好衣服,往身上喷些香水,整理好衬衫的袖子,以习惯动作将香烟、皮夹、火柴和双链条带坠子的怀表分别放进几个口袋里,然后抖了抖手帕。虽然他遇上了倒霉事,但觉得自己还是那么清洁、芳香,身体健康而有朝气。他微微颠着腿走进餐厅,那儿已经摆好了咖啡,旁边是信件和机关里来的公文。
\par 他先看了信件。其中一个商人的来信很扫他的兴。此人想买妻子田庄上那片森林。森林固然该卖,只是眼下没有跟妻子和好前万万不可谈这件事。尤其令他不快的是,这种事情很可能使他面临的夫妻和解问题牵扯到金钱上的利害关系。难道他谋求与妻子和好就是出于这种利害关系,为了能卖掉那片森林吗?想到这里他感到受了侮辱。
\par 看罢来信,斯捷潘·阿尔卡季奇把公文挪过来,匆匆翻阅了两个案卷,用粗大的铅笔做了些记号,然后推开公文,端起咖啡,打开油墨未干的晨报,边喝咖啡边看起报来。
\par 斯捷潘·阿尔卡季奇订的是一份自由主义报纸,不是极端自由主义的,而是多数人赞成的那种自由主义。尽管他其实对科学、艺术和政治都不感兴趣,但他坚决拥护多数人和他订的报纸对这三类问题所持的观点,并且随着多数人观点的改变而改变,或者毋宁说,他并不改变观点,而是观点本身在他头脑中不知不觉地变化着。
\par 斯捷潘·阿尔卡季奇并不选择派别和观点,倒是这些派别和观点向他不招自来,就像他并不挑选礼帽或常礼服的样式,别人穿戴什么他就跟着买什么。对于生活在上流社会的他,对于一个成年人通常要开展某些精神活动而言,持有一种观点,就像戴一顶礼帽那样必需。如果说,他更有理由喜欢自由派,而不像他圈子里的许多人士那样赞成保守派,那倒并不是他认为自由派更有道理些,而是因为自由主义更适合他的生活方式。自由党常把俄国说得一无是处,说的倒不假,斯捷潘·阿尔卡季奇就是债台高筑,正缺钱花。自由党说婚姻制度过时,必须加以改革,不错,家庭生活对斯捷潘·阿尔卡季奇甚少乐趣,还迫使他违心地撒谎和装模作样。自由党说,或者毋宁说是暗示,宗教不过是给野蛮人套上的笼头,确实,斯捷潘·阿尔卡季奇只做一会儿祈祷两腿就疼得要命;再说他也不明白,现世的生活本可以过得很快活,为什么还要用恐怖夸张的语言谈论来世呢。斯捷潘·阿尔卡季奇也爱开个玩笑,捉弄一下老实人,例如他说,既然要炫耀家族门第,就不该只算到留里克\footnote{留里克王朝(869—1598)的奠基者,俄国王族及某些贵族被认为是其后裔。}为止,还应该承认最早的祖先——猿猴。就这样,斯捷潘·阿尔卡季奇对自由主义已习以为常,他喜欢看自己订的报纸,犹如饭后抽一支雪茄烟,使他头脑中产生轻雾似的朦胧感。他看到社论里说,有人叫嚷什么激进主义要吞噬一切保守分子,政府必须采取措施阻挡革命祸水,这种叫嚷在当代实在大可不必,相反,“据我们看来,危险并不在于什么假想的革命祸水,而在于传统势力之顽固不化,阻碍进步”云云。他又看到另一篇文章谈到财政问题,其中提到边沁\footnote{边沁(1748—1832),英国哲学家,功利主义哲学创始人。}和米勒\footnote{米勒(1806—1873),英国唯心主义哲学家、经济学家。},并对财政部语涉讥诮。凭着他特有的敏捷思路,他懂得各种讥诮的含义:谁讥诮谁以及因为何事而发;这种揣测常使他感受到一种乐趣。但是今天,想起了马特廖娜·菲利莫诺夫娜出的主意,想到家中诸事不遂,乐趣就变成了扫兴。报上还说,据闻,贝斯特伯爵已经到了威斯巴登。报上还有那些染头发、卖马车、征婚之类的广告,这些消息都不能像往常那样使他觉得滑稽有趣了。
\par 看过报纸,喝完第二杯咖啡,吃了一块黄油白面包,他站起身,抖去西装背心上的面包屑,舒展一下宽阔的胸膛,愉快地笑了——倒不是他的心情特别愉快,而是因为他的消化功能良好。
\par 不过,这愉快的一笑立刻勾起了全部往事,他又陷入了沉思。
\par 门外传来两个孩子的说话声(斯捷潘·阿尔卡季奇听出来是小儿子格里沙和大女儿塔尼娅)。他俩在搬运什么东西,弄翻在地上了。
\par “我说过,不能让旅客坐在车顶上,”小姑娘用英语嚷道,“去捡起来呀!”
\par “全都乱了套,”斯捷潘·阿尔卡季奇心想,“让孩子们自己到处乱跑。”他走到门口叫住了他们。姐弟俩扔下当作火车玩的小匣子,朝父亲走来。
\par 小姑娘是父亲的宝贝,她大胆地跑了进来,搂住父亲,笑着吊在他脖子上,像平时那样喜欢闻他络腮胡子上熟悉的香水气味。最后,小姑娘吻了吻父亲因为弯下身体而涨红了的那张慈爱的脸,松开双手,待要跑出去,父亲却拉住了她。
\par “妈妈怎么样?”他问道,一边抚摩着女儿柔嫩光滑的脖子。“你好,”他又朝向他问好的男孩子微笑说。
\par 他意识到自己不太喜欢儿子,所以总是努力做得公平些;儿子感到了这一点,对父亲冷淡的微笑并不报以笑容。
\par “妈妈?她刚起床,”小姑娘说。
\par 斯捷潘·阿尔卡季奇叹了口气。“这么说,她又是彻夜未眠,”他想。
\par “她高兴吗?”
\par 小姑娘知道父母亲吵过嘴,母亲不可能高兴,这一点父亲该是知道的,现在他这么随便地问,就是在装模作样。女儿为父亲脸红了。父亲立刻觉察到这一点,也脸红了。
\par “不知道,”她说。“她没叫我们读书,叫我们跟古莉小姐到外祖母家去玩。”
\par “哦,去吧,我的坦丘罗奇卡\footnote{塔尼娅的昵称。}。哦,等一下,”他说,仍然拉住女儿不放,抚摩着她柔嫩的小手。
\par 他从壁炉上取下昨天放在那里的一小盒糖果,挑了两块女儿爱吃的巧克力和水果软糖,递给她。
\par “这一块给格里沙吗?”小姑娘指着巧克力糖说。
\par “好的,好的。”他又抚摩了一下女儿的肩膀,在她发根上和脖子上亲了一下,才放她走。
\par “马车备好了,”马特维说。“可是有个女人求见,”他又补充道。
\par “等了很久吗?”斯捷潘·阿尔卡季奇问。
\par “有半个小时了。”
\par “对你说过多少次了,这种事情要立即禀报!”
\par “总得让您把咖啡喝完呀,”马特维以一种粗率友好的口气说,使人听了也不好生气。
\par “那就快请吧,”奥布隆斯基扫兴地皱起眉头说。
\par 求见者是一位上尉的妻子,叫加里宁娜。虽然她提出的请求无法满足,而且讲得前言不对后语,斯捷潘·阿尔卡季奇还是照例请她坐下来,毫不打断地倾听她的陈述,然后仔细替她出主意,叫她如何如何去找某某人,他甚至用他那清晰、漂亮、又长又粗的字体,工整而流畅地写下一封便函,让她拿去见那个能够周济她的人。
\par 打发走上尉的妻子,斯捷潘·阿尔卡季奇拿起礼帽,但他欲行又止,寻思是否忘记了什么事。看来,除了他想忘却的妻子之外,他并没有忘记什么。
\par “哎呀!”他垂下了头,漂亮的脸上露出忧愁的表情。“去还是不去呢?”他自言自语,但内心却在说,不必去了,除了虚情假意不会有别的,他俩的关系已经不可修复,因为既不能使她重新具有魅力而激发爱情,也不能把他变成失去恋爱能力的老人。现在除了虚伪和谎言,不可能有别的结果,而虚伪和撒谎却是有违他的本性的。
\par “可是迟早总得去,不能就这样不了了之,”他想,尽量使自己鼓起勇气。他挺起胸膛,掏出一支香烟,点燃后吸了两口,把它扔在珍珠贝做的烟灰缸里。他快步穿过光线阴暗的客厅,推开了另一扇门,那是通向妻子卧室的门。

\subsubsection*{四}

\par 达里娅·亚历山德罗夫娜的房间里,到处是散乱的衣物,她站在杂物当中,从她面前一个打开的小柜子里挑拣什么东西。她穿着短衫,往日一头浓密的秀发已经变得稀疏,编成辫子盘在后脑勺上。她容颜憔悴,两只大眼睛从消瘦的脸上凸显出来,露出惊恐的神色。听到丈夫的脚步声,她停住手,转眼望着门口,想在脸上装出严厉和鄙夷的表情,却怎么也装不像。她觉得自己害怕他,害怕眼下的会面。刚才她要做的事,这三天内已经尝试过多次:收拾自己的和孩子们的东西,送到娘家去。但她还是下不了决心。和前几次一样,这一次她也对自己说,不能这样就算完,一定得想办法惩罚和羞辱他,用他带给她的痛苦,哪怕只是一小部分,来报复他,让他也尝尝痛苦的滋味。她反复说要离开他,可是又觉得这不可能,因为她已经习惯了把他作为自己的丈夫并且爱他。另外她还觉得,在自己家照料五个儿女都快要忙不过来,带到外婆家,他们的情况将会更糟。何况这三天里,小儿子吃了不干净的肉汤已经生病,其余的孩子昨天几乎就没吃饭。她意识到走是不可能的,但为了骗骗自己,仍然拾掇东西,装成要走的样子。
\par 看见丈夫进来,她把手伸到小柜子的抽屉里,像是在寻找什么,丈夫走到她跟前,她才回过头望望他。她原想装出一副严厉而坚决的面孔,可是却流露出慌乱和痛苦的神情。
\par “多莉!”他畏怯地小声说,缩起脑袋,想装出可怜而顺从的样子,但还是显得那么喜气洋洋和气色健康。
\par 她很快地从头到脚打量一眼他那红光满面的健康身体。“是啊,瞧他多么称心如意!”她想,“而我?……他这副和气嘴脸真让人讨厌。大家因此喜欢他,夸他,我就恨他这副样子,”她想道,紧紧抿起了嘴,她那容易抽搐的苍白的脸上,右颊的肌肉开始颤抖。
\par “您要干什么?”她用急促的、气得变了腔调的低沉声音问道。
\par “多莉!”他又说,声音在打颤,“安娜今天要来了。”
\par “关我什么事?我不接待她!”她叫喊道。
\par “可是,也应该,多莉……”
\par “走开,走开,走开!”她望也不望他,又喊道,像是肉体受了痛苦发出的叫喊。
\par 当斯捷潘·阿尔卡季奇独自想到妻子的时候,他还能够保持镇定,指望事情像马特维所说的那样,会顺利解决,所以他能从容不迫地看报纸、喝咖啡。可是现在,当他目睹妻子这疲惫不堪的痛苦的面容,听见她听天由命、充满绝望的声音时,他突然感到呼吸困难,喉头哽咽,眼睛里也闪起了泪花。
\par “天哪,我干了什么啊!多莉!看在上帝的份上!……要知道……”他说不下去了,一阵呜咽堵住了他的喉咙。
\par 她啪的一声关上柜门,瞪了他一眼。
\par “多莉,我能说什么呢?只有一句话:饶恕我,饶恕我吧……你回想一下,难道九年的生活不能抵偿一时的,一时的……”
\par 她垂下眼睛在听他说,等他把话说完,仿佛在哀求他,希望他能够说服她。
\par “一时的忘情……”他终于说出口来,正想接着说下去,只见她又抿紧了嘴唇,像在忍受肉体的痛楚,右颊上的肌肉又抽搐起来。
\par “走开,从这儿走开!”她叫起来,声音更尖,“别对我说您的忘情,您的肮脏行为!”
\par 她想要走出去,身子晃了一下,连忙扶住椅背。他鼓胀着脸,嘴唇噘起,眼里含满了泪水。
\par “多莉!”他呜呜咽咽地说,“看在上帝份上,想想孩子们吧,他们是无辜的。全是我的错,你惩罚我,让我来赎罪吧。只要能办到,我什么都愿意做!我罪过,真是罪过啊!可是多莉,你饶恕我吧!”
\par 她坐了下来。他听着她沉重的大声喘息,说不出对她有多么的可怜。她几次想说话却开不了口。他等着她。
\par “你想到孩子,只是为了逗他们玩,而我想到他们,知道他们现在都给毁了,”她说出了显然是这三天来心中反复说过的一句话。
\par 她称他为“你”,他感激地看了她一眼,并想走过去拉她的手,她厌恶地避开了。
\par “我惦记着孩子们,为了救孩子我什么都愿意做,可是我不知道怎样救他们:让他们离开父亲,还是留在伤风败俗的父亲,是的,伤风败俗的父亲身边……您倒说说,发生了那种……事情之后,难道我们还能在一起生活吗?难道这可能吗?您说呀,难道这可能吗?”她重复说,声音越来越高。“我的丈夫,孩子的父亲,同自己孩子的女家庭教师发生了这种关系之后……”
\par “可是怎么办呢?怎么办呢?”他可怜巴巴地说,自己也不知道在说些什么,把头垂得越来越低了。
\par “您让我恶心,讨厌!”她喊叫起来,火气越来越大。“您的眼泪像水一样不值钱!您从来就不爱我。您既没有心肝也不光明正大!您叫我厌恶、恶心,您是陌生人,完全是陌生人!”她痛苦地、恶狠狠地说出了她感到可怕的这个字眼——陌生人。
\par 他望望她,她脸上的怒气使他既害怕又吃惊。他不明白,他的怜悯反而激怒了她。她看出来,他对她只是可怜而不是爱。“不,她恨我。她不会宽恕我,”他想。
\par “这太可怕了!太可怕了!”他说。
\par 这时,隔壁房间里有个小孩哭叫起来,大概是跌倒了。达里娅·亚历山德罗夫娜侧耳细听,脸色立刻缓和下来。
\par 她定了定神,似乎不知道自己身在何处,该做什么,随后她一下子站起来,向门口走去。
\par “瞧,她爱我的小孩,”他看见她听到孩子哭叫时脸色的变化,这样想,“她爱我的小孩,又怎么会恨我呢?”
\par “多莉,再听我说一句,”他跟在她身后说。
\par “您要跟着我,我就叫人来,叫孩子们来!让大家都知道您是个卑鄙的人!我今天就走,让您跟您的情妇住在这里吧!”
\par 她砰的一声带上门,走了。
\par 斯捷潘·阿尔卡季奇叹了口气,擦了擦脸,轻手轻脚地往外走。“马特维说会顺利解决。结果怎么样呢?我看简直没有可能。唉,唉,太可怕了!她那样叫喊真是俗气,”他自语道,回想起她的喊声和她的用词:卑鄙的人和情妇。“也许女仆们都听到了!真是俗不可耐。”斯捷潘·阿尔卡季奇独自站了一会,揩揩眼睛,叹息一声,然后挺起胸脯,走出了房间。
\par 今天是礼拜五。德国钟表匠正在餐厅里给钟上发条。斯捷潘·阿尔卡季奇想起他曾拿这个干活认真的秃头钟表匠开过玩笑,说德国人“为了给钟表上发条,自己一生上足了发条”。想到这里不禁莞尔一笑。斯捷潘·阿尔卡季奇喜欢俏皮的笑话。“说不定真的会顺利解决!这话真有趣:会顺利解决,”他想,“要讲讲此话的来历。”
\par “马特维!”他喊道,“你和玛丽亚把休息室收拾一下,迎接安娜·阿尔卡季耶夫娜,”他向走过来的马特维说。
\par “遵命。”
\par 斯捷潘·阿尔卡季奇穿上毛皮大衣,走到台阶上。
\par “您不回来吃饭吗?”送他出来的马特维说。
\par “看情况吧。这个拿去开销,”他说,从钱夹里掏出十卢布交给马特维。“够吗?”
\par “够不够都得应付过去,”马特维说,砰地关上车门,退回到台阶上。
\par 这时,达里娅·亚历山德罗夫娜已经哄好了孩子,听马车声知道丈夫已走,就又回到自己的卧室。这里是她唯一的避风港,可以躲一躲家务琐事的烦扰。只要她一出房门,那些琐事就缠得她不可开交。刚才就是这样,她到儿童室只去了不大一会儿工夫,英国女家庭教师和马特廖娜·菲利莫诺夫娜就向她提了好些个问题,而且都是迫不及待、唯有她才能答复的问题,诸如:孩子们穿什么衣服去散步?是否给他们喝牛奶?要不要派人另找一名厨师?等等。
\par “唉,别烦我,别烦我了!”她说。回到卧室后,她又坐到同丈夫说话的那个位置,她紧握双手,戒指从瘦削的手指上滑落下来,她开始回味整个谈话的经过。“他走了!他和她结果怎样了?”她心里想道。“莫非还要去见她?我干吗不问问他?不,不,和解是不可能的。即使我俩留在一个家里,也只是陌生人,永远是陌生人!”她意味深长地重复这个令她害怕的字眼。“可是我原先多么爱他,天哪,多么爱他!……我多么爱他啊!即使是现在,难道我就不爱他?难道不比从前更加爱他吗?最可怕的是……”她有了一个想法,但是没来得及想完,因为这时候马特廖娜·菲利莫诺夫娜从门外探进头来。
\par “您派人去找我兄弟来吧,”她说,“他好歹会做个饭,要不又像昨天那样,孩子们到六点钟也吃不上饭。”
\par “好吧,我马上出来安排。新鲜牛奶叫人去拿了吗?”
\par 于是,达里娅·亚历山德罗夫娜重又投身于日常的琐事中,并借此暂时排解一下她心里的悲伤。


\subsubsection*{五}

\par 斯捷潘·阿尔卡季奇凭着他良好的天赋,在学校时成绩不错,可是他疏懒顽皮,结果落到了最后几名。虽然他一向生活放纵,既无显赫头衔,也非年高德劭,他却能在莫斯科政府机关里占据一个相当体面而又薪水丰厚的官职。这个职位是通过他妹妹安娜的丈夫阿列克谢·亚历山德罗维奇·卡列宁谋得的。卡列宁在这个机关所属的部里担任要职。不过,即使卡列宁不派内兄出任这个职位,斯季瓦·奥布隆斯基也会通过上百个别的人,包括兄弟、姐妹、嫡亲、表亲、叔伯和七大姑八大姨的关系弄到这样的或类似这样的位置,年薪可以拿到六千卢布,这笔钱是他所亟需的,因为,虽然他妻子有大宗财产,他自己的事业却弄得很糟。
\par 莫斯科和彼得堡几乎有一半人是斯捷潘·阿尔卡季奇的亲戚朋友。他所出生的那个环境中,所有的人或曾经是,或后来成了达官显贵。三分之一是老一辈国家栋梁,是他的父执,从他孩提时代就认识他。另外三分之一是他的至交。还有三分之一是老熟人。因此,那些以授职、租赁、租让等形式分配世间福祉的人皆是他的朋友,是决不会漏掉他这位同道的。奥布隆斯基无需花大力气就能弄到一个肥缺,只要他不拒绝,不忌妒,不争吵,不抱怨就行,而他为人素称随和,是从来不会那样做的。假如有人对他说,他得不到他所需要的那种肥缺,他会觉得好笑,何况他的要求并不过分,他只想得到他的同龄人都能得到的东西,至于任职能力,他是不会比任何人逊色的。
\par 所有认识斯捷潘·阿尔卡季奇的人,都喜欢他善良快乐的性格和无庸置疑的诚实,而且,他那漂亮、开朗的外表,炯炯有神的眼睛,乌黑的眉毛、头发,还有白里透红的脸庞,都会对遇到他的人从生理上产生某种亲切而愉快的感染力。“啊哈!斯季瓦!奥布隆斯基!可不是他!”碰到他的人几乎总是高兴地笑着说。虽然有时跟他谈谈话也算不得什么赏心乐事,但是过一两天后再见到他时,大伙还是照样高兴。
\par 斯捷潘·阿尔卡季奇在莫斯科某机关任长官已有三年。他的同僚、下级、上司及所有跟他打过交道的人无不喜欢他,而且尊重他。斯捷潘·阿尔卡季奇能博得同事如此普遍的尊重,主要靠他的三大品质:第一,他知道自己的短处,故待人亦异常宽容;第二,他是彻底的自由主义,不是从报纸上看来的那种,而是浸透在他血液里的自由主义,他以这种态度一视同仁地对待所有人而不论其头衔大小、地位高低;第三,也是最主要的一点,他对职务上的事兴趣不大,从不过分热心,也就从不犯错误。
\par 斯捷潘·阿尔卡季奇抵达职所后,由门房恭敬地陪着,手提公文包走进他自己的小办公室,穿上制服后,再来到机关办公室。录事和职员们全都站起来,高兴而恭敬地向他鞠躬。斯捷潘·阿尔卡季奇像平时那样匆匆走向自己的座位,跟委员们一一握手,坐了下来。他很得体地说了两句笑话,就开始办公。谁都不及斯捷潘·阿尔卡季奇那样善于掌握随便、简单和公事公办之间的分寸,这种分寸是保持办公愉快氛围所需要的。一位秘书拿着公文,像机关里所有的人那样高兴而恭敬地走过来,用斯捷潘·阿尔卡季奇所提倡的自由主义亲昵语调说:
\par “我们搞到一份奔萨省府的报告。您是否要……”
\par “终于拿到了?”斯捷潘·阿尔卡季奇用手指按住一份公文说,“那么,先生们……”于是办公开始了。
\par “他们可知道,”他想,一面郑重其事地低下头听着报告,“半小时前他们的主任就像做了错事的小孩子!”别人念报告时,他的眼睛始终是笑眯眯的。办公一直持续到下午两点,然后是休息和午餐。
\par 两点钟不到,议事厅的玻璃门忽然打开,有个人走进来。委员们很高兴有了轻松一下的机会,纷纷从沙皇肖像和守法镜\footnote{顶部有双头鹰的三棱镜,为帝俄时官厅中陈设物。}下面朝门口转过头去,但是门边的守卫立刻把那人赶了出去,随后又把玻璃门关上了。
\par 公文念完之后,斯捷潘·阿尔卡季奇站起来,伸了个懒腰,为了顺应自由主义时尚,他在机关办公室里拿起一根香烟,然后向他的主任室走去。他的两位同事,老官吏尼基京和低级侍从官格里涅维奇,也随他一起走出来。
\par “我们午饭后还来得及办完,”斯捷潘·阿尔卡季奇说。
\par “当然来得及!”尼基京说。
\par “那个福明真是大滑头,”格里涅维奇提到他们所审查案件的一位当事人说。
\par 斯捷潘·阿尔卡季奇听见此话皱了皱眉,示意不应该过早下判断,但是没有回答格里涅维奇。
\par “刚才进来的那人是谁?”他问门卫。
\par “大人,一个人趁我转身的工夫溜进来,说是要见您。我告诉他,等委员们出来的时候……”
\par “他在哪儿?”
\par “大概到门厅去了,刚才一直在这儿走来走去。瞧,就是他,”门卫指着一个身板壮实、宽肩膀、鬈胡须的人说。只见那人还戴着一顶羊皮帽,正以轻快的步子踏着磨损的石阶跑上来。下台阶的人中有个提公事包的瘦官员停了下来,不以为然地望望跑上台阶的那人的一双脚,又询问似地瞥了奥布隆斯基一眼。
\par 斯捷潘·阿尔卡季奇站在台阶上面。他的脸衬着制服的绣金领子显得和蔼而有精神,当他认出闯进门的那个人是谁时,他更加容光焕发了。
\par “果然不错!莱温,你到底来了!”他带着友好而嘲弄的微笑打量迎面走来的莱温。“怎么屈驾到这穷窝里来找我呀?”斯捷潘·阿尔卡季奇说,嫌握手还不够,又吻了吻朋友。“来好久了吗?”
\par “刚到,我很想见你,”莱温答道,腼腆而又有些气恼不安地望望四周。
\par “走,上我办公室去,”斯捷潘·阿尔卡季奇说,他了解他朋友的腼腆是由于自尊心强和容易激怒,便拉住莱温的手,仿佛领他通过危险区,把他带走了。
\par 斯捷潘·阿尔卡季奇几乎对所有的熟人都以“你”相称,无论是六旬老翁、二十岁的青年、演员、部长、商人还是侍从将官都一视同仁,这样一来,在社会的最高层和最底层都有许多跟他相称尔汝的朋友,这些人一旦得知是奥布隆斯基使他们也有某种共同之处,一定会惊讶莫名。凡是跟他喝过香槟的人,他都称“你”,而他又是跟什么人都可以一起喝香槟的,所以,万一要当着下级的面会晤他那些厚脸皮的“你”们(他这样戏称他的许多朋友),凭着他特有的机灵,他懂得怎样淡化在下级心目中留下的不快印象。莱温不是厚脸皮的“你”,但是奥布隆斯基也机灵地感到,莱温一定认为他当着下级的面不愿流露他俩的亲密关系,所以连忙把他带到他的小办公室来了。
\par 莱温和奥布隆斯基年龄相若,但不是只跟他喝香槟酒的那种“你”。莱温是他少年时代的伙伴和朋友。他俩性格、爱好虽然不同,却像一对从小就要好的朋友那样互相喜爱。不过,尽管这样,他们也像选择了不同行业的人所常有的那样,彼此谈论起来固然也肯定对方的职业,其实他们心里是互相瞧不起的。他们各自觉得,唯有自己的生活才是真正的生活,而对方却在想入非非。奥布隆斯基见到莱温时,禁不住露出嘲弄的微笑。他曾多次见莱温从乡下来到莫斯科,莱温在乡下做事,但究竟何事,斯捷潘·阿尔卡季奇向来不甚了了,也不感兴趣。莱温每次来莫斯科都是情绪激动,行色匆匆,还有点不好意思,他为这不好意思感到恼火,而且大抵还要带来某种出人意料的崭新的观点。斯捷潘·阿尔卡季奇既嘲笑他也喜欢他这一点。同样,莱温打心眼里鄙视朋友的都市生活方式,还有他那些鸡毛蒜皮的公务,并讥笑这一切。所不同者,奥布隆斯基在做一般人都做的事情,所以他嘲笑人时显得平心静气而有自信,而莱温的讥笑则显得自信心不足,有时还是气呼呼的。
\par “我们早就盼望你来了,”斯捷潘·阿尔卡季奇说,走进自己的办公室后,松开了莱温的手,仿佛表示在这里危险已经过去。“见到你非常、非常高兴,”他接着说,“你怎么样?好吗?什么时候到的?”
\par 莱温没有回答,他不时望望奥布隆斯基两位同事的陌生脸孔,尤其是温文尔雅的格里涅维奇的那只手,手指又白又长,黄色的长指甲尖端朝里弯曲,还有衬衫上那些闪闪发光的大钮扣,而那双手似乎已吸引了莱温全副的注意力,弄得他不能自由地思想了。奥布隆斯基马上觉察到这一点,笑了笑。
\par “噢,让我给你们介绍一下,”他说,“我的同事:菲利普·伊万内奇·尼基京,米哈伊尔·斯坦尼斯拉维奇·格里涅维奇,”然后转向莱温:“地方自治局代表,新派地方自治人士,一手能举五普特\footnote{普特,俄国重量单位,1普特合16.38公斤。}重的体操运动员,畜牧专家,猎手,我的朋友,康斯坦丁·德米特里奇·莱温,谢尔盖·伊万内奇·科兹内舍夫的兄弟。”
\par “幸会,”那个小老头说。
\par “我有幸认识令兄谢尔盖·伊万内奇,”格里涅维奇说,伸出他那留着长指甲的纤细的手。
\par 莱温皱起眉头,冷淡地握握他的手,马上向奥布隆斯基转过身去。虽然他很敬重他的同母异父兄长,那位全俄知名的作家,但是现在,当别人只把他看成是著名的科兹内舍夫的兄弟,而不是康斯坦丁·莱温时,他简直不能忍受。
\par “不,我已经不是地方自治局代表,我跟他们吵翻了,再也不去参加地方自治局代表会议了,”他对奥布隆斯基说。
\par “这么快?”奥布隆斯基微笑说。“是怎么回事?为什么?”
\par “说来话长。以后我再告诉你,”莱温说,可是他马上开始讲起来。“简单地说,我确信没有任何地方自治活动,也不可能有,”他开始说话的样子,就像刚才有人欺侮了他,“一方面,那是个玩具,他们玩弄议会那一套,而我既不算小也不够老,不想耍弄这些玩具。另,另一方面(他口吃了一下),这是县里的coterie\footnote{法语:一帮,一伙人。}捞取钱财的工具。过去有监护机构、法院,现在有地方自治局,它们不是以受贿的形式,而是通过白拿薪水来捞钱,”他说得激昂慷慨,好像在座的人有谁会对他的意见提出异议。
\par “嘿!我看你又跨入了新阶段,保守主义的新阶段,”斯捷潘·阿尔卡季奇说。“不过这个以后再谈吧。”
\par “好吧,以后再谈。不过我有事找你,”莱温说,憎恶地盯着格里涅维奇的那只手。
\par 斯捷潘·阿尔卡季奇难以觉察地微微一笑。
\par “你不是说过,再也不穿西装了吗?”他说,一面打量着莱温那身显然是法国裁缝做的新衣服。“原来如此!我看这也是新阶段。”
\par 莱温刷地涨红了脸,不是像成年人那样微微地、不自觉地脸红,而是像小男孩那样,觉得自己腼腆得可笑,结果越加害臊和脸红,简直要哭出来了。看着这张聪明而刚毅的脸变得如此孩子气,真有些奇怪,所以奥布隆斯基不再朝他看了。
\par “我们在哪儿见面呢?我非常、非常需要跟你谈谈,”莱温说。
\par 奥布隆斯基像是考虑了一下,说:
\par “这样吧,我们上古林去吃午饭,就在那里谈谈。三点钟以前我有空。”
\par “不必了,”莱温想了想说,“我还得到别处去一趟。”
\par “也好,那就一起吃晚饭吧。”
\par “吃晚饭?其实我也没有什么特别的事,只要三言两语问一下,以后再细谈。”
\par “那么现在先说说三言两语,晚饭的时候再详谈。”
\par “三言两语是这样的,”莱温说,“不过,也没有什么特别的事。”
\par 莱温在努力克服他的腼腆,所以脸上忽然又出现了恼火的表情。
\par “谢尔巴茨基一家都在做什么?一切还照旧吗?”他说。
\par 斯捷潘·阿尔卡季奇早就知道莱温爱上了他的小姨子基季,他微微一笑,眼睛里露出愉快的神色。
\par “你说了三言两语,可是我无法用三言两语答复你,因为……对不起,稍等一下……”
\par 秘书走了进来,一副亲昵而恭敬的样子,他像所有的秘书一样,谦逊地意识到自己在办公务方面比首长懂行,拿着文件走到奥布隆斯基跟前,装作请示的样子,开始解释某个棘手的问题。斯捷潘·阿尔卡季奇没等他说完,就温和地把手按在他的袖口上。
\par “不,您就照我说过的办,”他说,一面用微笑缓和一下他的语气,随后简短地表明了他的看法,就把文件推开了:“请您照此办理,就这样吧,扎哈尔·尼基季奇。”
\par 秘书很尴尬地走了。莱温在秘书说事的时候已经完全克服了腼腆。他站在那里,把胳膊肘撑在椅背上,脸上带着专注的、讥讽的表情。
\par “我真不懂,不懂,”他说。
\par “你不懂什么?”奥布隆斯基说,仍然愉快地笑着,拿起一支香烟,等莱温说出什么乖谬的话来。
\par “我不懂你在干什么,”莱温耸耸肩膀说。“你怎么能一本正经地干这个?”
\par “为什么不能?”
\par “因为无事可做。”
\par “这是你的想法,我们可忙得不可开交呢。”
\par “埋头案牍。是呀,你有这方面的才干,”莱温说。
\par “也就是说,你认为我还缺点什么?”
\par “也许是的,”莱温说,“不过我还是欣赏你的气派,为我的朋友是如此伟大的人物而感到骄傲。不过你还没有回答我的问题,”他说完后,竭力直视着奥布隆斯基的眼睛。
\par “哦,好了,好了。等着瞧吧,你也会到这一步的。你在卡拉津县有三千俄亩\footnote{1俄亩等于1.09公顷。}土地,这该多好。你这么肌肉发达,你容光焕发得像个十二岁的小姑娘,可是你也肯定会落到我们这一步的。至于你打听的事情,告诉你:情况没有变化,只可惜你好久都没来了。”
\par “怎么了?”莱温惊恐地问道。
\par “没什么,”奥布隆斯基说。“这事我们再谈吧。你这次来究竟为了什么事?”
\par “唉,这个也以后再谈吧,”莱温说,他的脸又红到了耳根。
\par “那好吧。我明白了,”斯捷潘·阿尔卡季奇说。“你瞧,本来我想叫你上我家去,可是妻子身体不大好。我看这样吧,你若是想见他们,这会儿他们大概正在动物园,从四点待到五点。基季在溜冰。你先坐车去那儿,回头我也去,带你一道找个地方吃晚饭。”
\par “好极了,那就再见。”
\par “你可当心,我了解你,你会把说好的事情忘了,要不就突然跑回乡下去!”斯捷潘·阿尔卡季奇笑着大声说。
\par “决不会的。”
\par 莱温走出办公室,到了门口才想起来,他忘了向奥布隆斯基的两位同事道别。
\par “看样子,这位先生精力很充沛,”莱温出去后,格里涅维奇说。
\par “是啊,老兄,”斯捷潘·阿尔卡季奇摇着头说,“他真是个幸运儿!在卡拉津县有三千俄亩土地,前程远大,而且多么有朝气!可不像我们这班人。”
\par “怎么您也抱怨起来了,斯捷潘·阿尔卡季奇?”
\par “糟糕啊,糟透了,”斯捷潘·阿尔卡季奇重重地叹了口气说。

\subsubsection*{六}




\subsubsection*{七}




\subsubsection*{八}




\subsubsection*{九}




\subsubsection*{十}




\subsubsection*{十一}




\subsubsection*{十二}




\subsubsection*{十三}




\subsubsection*{十四}




\subsubsection*{十五}




\subsubsection*{十六}




\subsubsection*{十七}




\subsubsection*{十八}




\subsubsection*{十九}




\subsubsection*{二十}




\subsubsection*{二十一}




\subsubsection*{二十二}




\subsubsection*{二十三}




\subsubsection*{二十四}




\subsubsection*{二十五}




\subsubsection*{二十六}




\subsubsection*{二十七}




\subsubsection*{二十八}




\subsubsection*{二十九}




\subsubsection*{三十}




\subsubsection*{三十一}




\subsubsection*{三十二}




\subsubsection*{三十三}




\subsubsection*{三十四}





























\subsection*{第二部}




\subsubsection*{一}
\subsubsection*{二}
\subsubsection*{三}
\subsubsection*{四}
\subsubsection*{五}
\subsubsection*{六}
\subsubsection*{七}
\subsubsection*{八}
\subsubsection*{九}
\subsubsection*{十}
\subsubsection*{十一}
\subsubsection*{十二}
\subsubsection*{十三}
\subsubsection*{十四}
\subsubsection*{十五}
\subsubsection*{十六}
\subsubsection*{十七}
\subsubsection*{十八}
\subsubsection*{十九}
\subsubsection*{二十}
\subsubsection*{二十一}
\subsubsection*{二十二}
\subsubsection*{二十三}
\subsubsection*{二十四}
\subsubsection*{二十五}
\subsubsection*{二十六}
\subsubsection*{二十七}
\subsubsection*{二十八}
\subsubsection*{二十九}
\subsubsection*{三十}
\subsubsection*{三十一}
\subsubsection*{三十二}
\subsubsection*{三十三}
\subsubsection*{三十四}
\subsubsection*{三十五}







\subsection*{第三部}



\subsubsection*{一}
\subsubsection*{二}
\subsubsection*{三}
\subsubsection*{四}
\subsubsection*{五}
\subsubsection*{六}
\subsubsection*{七}
\subsubsection*{八}
\subsubsection*{九}
\subsubsection*{十}
\subsubsection*{十一}
\subsubsection*{十二}
\subsubsection*{十三}
\subsubsection*{十四}
\subsubsection*{十五}
\subsubsection*{十六}
\subsubsection*{十七}
\subsubsection*{十八}
\subsubsection*{十九}
\subsubsection*{二十}
\subsubsection*{二十一}
\subsubsection*{二十二}
\subsubsection*{二十三}
\subsubsection*{二十四}
\subsubsection*{二十五}
\subsubsection*{二十六}
\subsubsection*{二十七}
\subsubsection*{二十八}
\subsubsection*{二十九}
\subsubsection*{三十}
\subsubsection*{三十一}
\subsubsection*{三十二}







\subsection*{第四部}



\subsubsection*{一}
\subsubsection*{二}
\subsubsection*{三}
\subsubsection*{四}
\subsubsection*{五}
\subsubsection*{六}
\subsubsection*{七}
\subsubsection*{八}
\subsubsection*{九}
\subsubsection*{十}
\subsubsection*{十一}
\subsubsection*{十二}
\subsubsection*{十三}
\subsubsection*{十四}
\subsubsection*{十五}
\subsubsection*{十六}
\subsubsection*{十七}
\subsubsection*{十八}
\subsubsection*{十九}
\subsubsection*{二十}
\subsubsection*{二十一}
\subsubsection*{二十二}
\subsubsection*{二十三}







\subsection*{第五部}




\subsubsection*{一}
\subsubsection*{二}
\subsubsection*{三}
\subsubsection*{四}
\subsubsection*{五}
\subsubsection*{六}
\subsubsection*{七}
\subsubsection*{八}
\subsubsection*{九}
\subsubsection*{十}
\subsubsection*{十一}
\subsubsection*{十二}
\subsubsection*{十三}
\subsubsection*{十四}
\subsubsection*{十五}
\subsubsection*{十六}
\subsubsection*{十七}
\subsubsection*{十八}
\subsubsection*{十九}
\subsubsection*{二十}
\subsubsection*{二十一}
\subsubsection*{二十二}
\subsubsection*{二十三}
\subsubsection*{二十四}
\subsubsection*{二十五}
\subsubsection*{二十六}
\subsubsection*{二十七}
\subsubsection*{二十八}
\subsubsection*{二十九}
\subsubsection*{三十}
\subsubsection*{三十一}
\subsubsection*{三十二}
\subsubsection*{三十三}






\subsection*{第六部}





\subsubsection*{一}
\subsubsection*{二}
\subsubsection*{三}
\subsubsection*{四}
\subsubsection*{五}
\subsubsection*{六}
\subsubsection*{七}
\subsubsection*{八}
\subsubsection*{九}
\subsubsection*{十}
\subsubsection*{十一}
\subsubsection*{十二}
\subsubsection*{十三}
\subsubsection*{十四}
\subsubsection*{十五}
\subsubsection*{十六}
\subsubsection*{十七}
\subsubsection*{十八}
\subsubsection*{十九}
\subsubsection*{二十}
\subsubsection*{二十一}
\subsubsection*{二十二}
\subsubsection*{二十三}
\subsubsection*{二十四}
\subsubsection*{二十五}
\subsubsection*{二十六}
\subsubsection*{二十七}
\subsubsection*{二十八}
\subsubsection*{二十九}
\subsubsection*{三十}
\subsubsection*{三十一}
\subsubsection*{三十二}






\subsection*{第七部}



\subsubsection*{一}
\subsubsection*{二}
\subsubsection*{三}
\subsubsection*{四}
\subsubsection*{五}
\subsubsection*{六}
\subsubsection*{七}
\subsubsection*{八}
\subsubsection*{九}
\subsubsection*{十}
\subsubsection*{十一}
\subsubsection*{十二}
\subsubsection*{十三}
\subsubsection*{十四}
\subsubsection*{十五}
\subsubsection*{十六}
\subsubsection*{十七}
\subsubsection*{十八}
\subsubsection*{十九}
\subsubsection*{二十}
\subsubsection*{二十一}
\subsubsection*{二十二}
\subsubsection*{二十三}
\subsubsection*{二十四}
\subsubsection*{二十五}
\subsubsection*{二十六}
\subsubsection*{二十七}
\subsubsection*{二十八}
\subsubsection*{二十九}
\subsubsection*{三十}
\subsubsection*{三十一}







\subsection*{第八部}





\subsubsection*{一}
\subsubsection*{二}
\subsubsection*{三}
\subsubsection*{四}
\subsubsection*{五}
\subsubsection*{六}
\subsubsection*{七}
\subsubsection*{八}
\subsubsection*{九}
\subsubsection*{十}
\subsubsection*{十一}
\subsubsection*{十二}
\subsubsection*{十三}
\subsubsection*{十四}
\subsubsection*{十五}
\subsubsection*{十六}
\subsubsection*{十七}
\subsubsection*{十八}
\subsubsection*{十九}













