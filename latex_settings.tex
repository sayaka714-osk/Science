

%%%%%%%  章节  %%%%%%

% Options: Sonny, Lenny, Glenn, Conny, Rejne, Bjarne, Bjornstrup
\usepackage[Lenny]{fncychap}
\ChTitleVar{\Large}

% 标题格式
\usepackage{titlesec}
\titleformat{\part}{\centering\Huge\bfseries}{第\Roman{part}部分}{1em}{}
\titleformat{\chapter}{\centering\huge\bfseries}{第\arabic{chapter}章}{1em}{}
\titleformat{\section}{\LARGE\bfseries}{第\arabic{section}节}{1em}{}
\titleformat{\subsection}{\Large\bfseries}{\arabic{subsection}.}{1em}{}
\titleformat{\subsubsection}{\large\bfseries}{\arabic{subsection}.\arabic{subsubsection}.}{1em}{}
\titleformat{\paragraph}{\normalsize\bfseries}{\arabic{subsection}.\arabic{subsubsection}.\arabic{paragraph}.}{1em}{}
\titleformat{\subparagraph}{\normalsize\bfseries}{\arabic{subsection}.\arabic{subsubsection}.\arabic{paragraph}.\arabic{subparagraph}.}{1em}{}

% 索引
\usepackage[xindy]{imakeidx}
\makeindex[columns=2, program=truexindy, intoc=true, options=-M texindy -I xelatex -C utf8, title={Index}]




%%%%%%%  页面设置  %%%%%%
\usepackage{geometry}   % 页面设置
\geometry{a4paper,left=2.5cm,right=2.5cm,top=2.5cm,bottom=2.5cm}
\usepackage{pdflscape}   % 页面横置
\usepackage{ragged2e}    % 两端对齐
\usepackage{indentfirst} % 首行缩进
%\usepackage{setspace}  % 间距
\setcounter{tocdepth}{7}  % 标题深度
\setcounter{secnumdepth}{7}
%\setlength{\baselineskip}{20pt} % 行距
\setlength{\parindent}{2em} % 首行缩进

% 页眉页脚
\usepackage{fancyhdr}
\pagestyle{fancy} % 设置页眉  
\lhead{}
\chead{}
\rhead{}
\cfoot{\thepage}
\rfoot{}
\lfoot{}
\renewcommand{\headrulewidth}{0pt}  %页眉线宽,设为0可以去页眉线


%%%%%%%  字体  %%%%%%
\usepackage{polyglossia} % 多语言

\usepackage{fontspec} % 字体
\setmainfont{CMU Serif}

%%%%%%%%  表格  %%%%%%%%
\usepackage{supertabular}
\usepackage{tabularx}      % 表格自动换行
\usepackage{longtable}
\usepackage{tabu}
\usepackage{booktabs}      % 表格线条
\usepackage{makecell}
\usepackage{multirow}    % 单元格合并
\usepackage{caption}

% \columnwidth  当前分栏的宽度
% \linewidth    当前行的宽度
% \textwidth    整个页面版芯的宽度
% \paperwidth   整个页面纸张的宽度

\newcommand{\tablelist}[3]{
    \parbox{#2}{
        \begin{enumerate}[#1]
            #3
        \end{enumerate}
    }
}



%%%%%%%%%  编号  %%%%%%%%%
\usepackage{enumerate}   % 条目 
\usepackage{enumitem}
%\setitemsize[1]{itemsep=0pt,partopsep=0pt,parsep=\parskip,topsep=0pt}
\usepackage{amsmath}
\usepackage{paralist}
\let\itemize\compactitem
\let\enditemize\endcompactitem
\let\enumerate\compactenum
\let\endenumerate\endcompactenum
\let\description\compactdesc
\let\enddescription\endcompactdesc


%%%%%%%% 公式
\usepackage{bm} % 数学公式加粗
\usepackage{bbm}
\usepackage{amsfonts}
\usepackage{amssymb}
\usepackage{breqn}


%%%%%%%  绘图  %%%%%%
\usepackage{graphicx}    % 图

\usepackage{tikz}
\usetikzlibrary{trees,calc,quotes}
\usetikzlibrary{angles,patterns,datavisualization}
\usetikzlibrary{arrows,intersections}
\usetikzlibrary{graphs}
\newcommand{\treegraph}[1]{
\usetikzlibrary{trees}
\tikzstyle{every node}=[draw=black,thick,anchor=west]
\begin{tikzpicture}[
        grow via three points={one child at (0.5,-0.7) and
                two children at (0.5,-0.7) and (0.5,-1.4)},
        edge from parent path={(\tikzparentnode.south) |- (\tikzchildnode.west)}]
    #1
    ;
\end{tikzpicture}
}

\tikzset{eaxis/.style={->,>=stealth}}
\tikzset{elegant/.style={smooth,thick,samples=50,cyan}}

\usepackage{tikz-cd}
\usetikzlibrary{matrix,arrows,decorations.pathmorphing}

\usepackage{pgfplots}
\usepgfplotslibrary{groupplots}

%%%%%%%%  标记  %%%%%%%%%
\usepackage[breaklinks,colorlinks,linkcolor=black,citecolor=black,urlcolor=black]{hyperref}




%%%%%%%  颜色  %%%%%%

\usepackage{color}
\definecolor{codeKeyword}{RGB}{200,0,100}
\definecolor{codeString}{RGB}{200,100,40}
\definecolor{codeComment}{RGB}{0,100,0}
\definecolor{codeNumber}{RGB}{128,128,128}
\definecolor{codeBackground}{RGB}{242,242,242}

% Matlab highlight color settings
%\definecolor{mBasic}{RGB}{248,248,242}       % default
\definecolor{mKeyword}{RGB}{0,0,255}          % bule
\definecolor{mString}{RGB}{160,32,240}        % purple
\definecolor{mComment}{RGB}{34,139,34}        % green
\definecolor{mBackground}{RGB}{245,245,245}   % lightgrey
\definecolor{mNumber}{RGB}{128,128,128}       % gray

\definecolor{Numberbg}{RGB}{237,240,241}     % lightgrey

% Python highlight color settings
%\definecolor{pBasic}{RGB}{248, 248, 242}     % default
\definecolor{pKeyword}{RGB}{228,0,128}        % magenta
\definecolor{pString}{RGB}{148,0,209}         % purple
\definecolor{pComment}{RGB}{117,113,94}       % gray
\definecolor{pIdentifier}{RGB}{166, 226, 46}  %
\definecolor{pBackground}{RGB}{245,245,245}   % lightgrey
\definecolor{pNumber}{RGB}{128,128,128}       % gray


\usepackage{xcolor}

%%%%%%%%%%  模块  %%%%%%%%%%%%


\usepackage{tcolorbox}
\newcommand{\informationBox}[1]{
    \small
    \begin{tcolorbox}[colback=gray!10!white,colframe=gray!30!white]
        #1
    \end{tcolorbox}
}


\newcommand{\mindmap}[1]{
    \begin{cases}
        #1
    \end{cases}\\
}












%%%%%%%%%%%%%%%%%%%%%%%%%%%%%%%%%


\usepackage[UTF8]{ctex}
\usepackage{autobreak}
\usepackage[utf8]{inputenc} % Required for inputting international characters

\usepackage{adjustbox}   % 调整box大小

\usepackage{tipa}
\usepackage{CJKfntef}
\usepackage{pdfpages}

\usepackage{blindtext}
\usepackage{verbatim}
\usepackage{ascmac}
\usepackage{xpinyin} % 拼音


